% run this through LaTeX with the appropriate wrapper

\dotopic{APPLICATION SERVICES}


\begin{bwslide}
\part*	{OUTLINE}\bf

\begin{description}
\item[PART I:]		APPLICATION LAYER STRUCTURE

\item[PART II:]		APPLICATION SERVICE ELEMENTS
\end{description}
\end{bwslide}


\begin{bwslide}
\ctitle	{A BIG ACKNOWLEDGEMENT}

\begin{nrtc}
\item	THIS PART OF THE TALK HAS BEEN REWORKED SEVERAL TIMES

\item	IN AN EFFORT TO BALANCE UNDERSTANDABILITY AND COMPLETENESS

\item	IT IS BASED, IN PART, ON AN EARLIER PRESENTATION BY
	CHRISTOPHER W.~MOORE AND MARSHALL T.~ROSE
\end{nrtc}
\end{bwslide}


\begin{bwslide}
\part	{APPLICATION LAYER STRUCTURE}\bf

\begin{nrtc}
\item	UPPER LAYER INFRASTRUCTURE

\item	APPLICATION CONTEXTS

\item	APPLICATION ENTITIES
\end{nrtc}
\end{bwslide}


\begin{bwslide}
\part*	{UPPER LAYERS INFRASTRUCTURE}\bf

\begin{nrtc}
\item	PROVIDES A COMMON FRAMEWORK FOR APPLICATIONS \emph{DESIGNERS}

\item	TO MAKE BUILDING APPLICATIONS EASIER FOR \emph{IMPLEMENTORS}

\item	IN THEORY, THIS YIELDS BETTER APPLICATIONS FOR THE \emph{USERS}
\end{nrtc}
\end{bwslide}


\begin{bwslide}
\ctitle	{PAYOFF OF THE INFRASTRUCTURE}

\begin{nrtc}
\item	RAPID PROTOTYPING OF APPLICATIONS

\item	FOCUS DESIGN/DEVELOPMENT ON \emph{APPLICATION} ISSUES,
	RATHER THAN UNDERLYING ISSUES

\item	MINIMIZE LONG-TERM INVESTMENT OF SOFTWARE-RELATED INVESTMENT
    \begin{nrtc}
    \item	(AT EXPENSE OF START-UP COST)
    \end{nrtc}
\end{nrtc}
\end{bwslide}


\begin{bwslide}
\ctitle	{OSI UPPER-LAYER INFRASTRUCTURE}

%\vskip.5in
\diagram[p]{figureA-1}
\end{bwslide}


\begin{bwslide}
\ctitle	{UPPER LAYERS INTERFACE TO\\ TRANSPORT SERVICE}

\begin{nrtc}
\item	FULL-DUPLEX CIRCUITS

\item	PUMP BYTES RELIABLY ACROSS THE NETWORK

\item	ALL DISCUSSION HENCE FORTH WILL BE CO-MODE
    \begin{nrtc}
    \item	AND VERY BRIEF!
    \end{nrtc}
\end{nrtc}
\end{bwslide}


\begin{bwslide}
\ctitle	{SESSION SERVICE}

\begin{nrtc}
\item	ADD \emph{CONTROL} FACILITIES TO MANIPULATE DIALOGUES

\item	THAT ARE SPECIFIC TO A PARTICULAR APPLICATION TASK
    \begin{nrtc}
    \item	e.g., ACCESSING A FILE OR DELIVERYING A MAIL MESSAGE
    \end{nrtc}
\end{nrtc}
\end{bwslide}


\begin{bwslide}
\ctitle	{SESSION SERVICE (cont.)}

\begin{nrtc}
\item	IN ADDITION TO TRANSPORT FUNCTIONALITY:
    \begin{nrtc}
    \item	TOKEN MANAGEMENT

    \item	DIALOGUE CONTROL

    \item	ACTIVITY MANAGEMENT

    \item	EXCEPTION REPORTING
    \end{nrtc}
\end{nrtc}
\end{bwslide}



\begin{bwslide}
\ctitle {MOVING DATA AT SESSION}

\begin{nrtc}
\item	VERSION 1 OF SESSION RESTRICTS SSDU SIZE TO 512 OCTETS FOR MANY
	SERVICES (e.g., \verb"S-CONNECT")

\item	TOO RESTRICTIVE FOR MANY OSI APPLICATIONS
    \begin{nrtc}
    \item	e.g., PASSING OSI DIRECTORY NAMES ON CONNECTIONS
    \end{nrtc}

\item	VERSION 2 REMOVES THIS LIMIT
    \begin{nrtc}
    \item	PRACTICAL LIMIT OF 10K OCTETS USED
    \end{nrtc}

\item	SESSON PROVIDES FOR DOWN-NEGOTIATION
	\begin{nrtc}
	\item	BUT MUST BE CAREFUL SINCE NOT ALL IMPLEMENTATIONS SUPPORT IT

	\item	CHOICES
	    \begin{nrtc}
	    \item	APPLICATION CHOOSES ONE, ELSE

	    \item	IF CONNECT SSDU $>$ 512: USE VERSION 2
	    \end{nrtc}
	\end{nrtc}
\end{nrtc}
\end{bwslide}


\begin{bwslide}
\ctitle	{ABSTRACT SYNTAX}

\begin{nrtc}
\item	ADD \emph{STRUCTURE} FACILITIES TO MANIPULATE THE DATA

\item	THAT IS EXCHANGED BETWEEN APPLICATIONS
    \begin{nrtc}
    \item	e.g., RECORD STRUCTURE IN A FILE OR BODY PARTS IN A MAIL
		MESSAGE 
    \end{nrtc}
\end{nrtc}
\end{bwslide}


\begin{bwslide}
\ctitle	{ABSTRACT SYNTAX (cont.)}

\begin{nrtc}
\item	ABSTRACT SYNTAX NOTATION
    \begin{nrtc}
    \item	FORMAL LANGUAGE FOR DESCRIBING DATA STRUCTURES

    \item	ASN.1 PROVIDES DESCRIPTIVE FRAMEWORK FOR UPPER TWO LAYERS
    \end{nrtc}

\item	TRANSFER SYNTAX NOTATION
    \begin{nrtc}
    \item	RESPONSIBLE FOR MAPPING INSTANCES OF DATA STRUCTURES
		INTO A BIT STREAM (SERIALIZATION)

    \item	BASIC ENCODING RULES (BER) ARE USED FOR ASN.1
    \end{nrtc}
\end{nrtc}
\end{bwslide}


\begin{bwslide}
\ctitle	{PRESENTATION SERVICE}

\begin{nrtc}
\item	APPLY
    \begin{nrtc}
    \item	\emph{STRUCTURE} FROM ABSTRACT SYNTAX
    \end{nrtc}
	ON TOP OF
    \begin{nrtc}
    \item	\emph{CONTROL} FROM SESSION SERVICE
\end{nrtc}

\item	REALLY NO MORE THAN ABSTRACT SYNTAX ON TOP OF PASS-THROUGH TO SESSION
\end{nrtc}
\end{bwslide}


\begin{bwslide}
\ctitle	{PRESENTATION SERVICE (cont.)}

\begin{nrtc}
\item	IN ADDITION TO SESSION FUNCTIONALITY:
    \begin{nrtc}
    \item	CONTEXT MANAGEMENT

    \item	SYNTAX MATCHING
    \end{nrtc}
\end{nrtc}
\end{bwslide}


\begin{bwslide}
\ctitle {INFRASTRUCTURE REVISITED}

\begin{nrtc}
\item	STRENGTHS
    \begin{nrtc}
    \item	COMMON INFRASTRUCTURE

    \item	MORE TECHNICALLY COMPREHENSIVE
    \end{nrtc}

\item	WEAKNESSES
    \begin{nrtc}
    \item	COMPLEXITY

    \item	POLITICAL POLARIZATION

    \item	LACK OF PRACTICAL EXPERIENCE
    \end{nrtc}
\end{nrtc}
\end{bwslide}


\begin{bwslide}
\part*	{APPLICATION CONTEXTS}\bf

\begin{nrtc}
\item	OSI APPLICATION LAYER IS DIVIDED INTO
    \begin{nrtc}
    \item	APPLICATION SERVICE ELEMENTS (ASEs)
    \end{nrtc}
	EACH WITH A PARTICULAR FUNCTION

\item	USE OF SERVICE ELEMENTS
    \begin{nrtc}
    \item	DIVIDES RESPONSIBILITY OF ``TOTAL'' APPLICATION PROTOCOL

    \item	PROMOTES ``REUSE'' OF APPLICATION LAYER FACILITIES
    \end{nrtc}
\end{nrtc}
\end{bwslide}


\begin{bwslide}
\ctitle	{AN EXAMPLE APPLICATION ENTITY}

\vskip.5in
\diagram[p]{figureA-6}
\end{bwslide}


\begin{bwslide}
\ctitle	{APPLICATION CONTEXTS (cont.)}

\begin{nrtc}
\item	AN APPLICATION PROTOCOL IS DEFINED BY
    \begin{nrtc}
    \item	SELECTING DIFFERENT SERVICE ELEMENTS

    \item	AND DECIDING HOW THEY INTERACT WITH EACH OTHER AND THE
		PRESENTATION SERVICE 
    \end{nrtc}
    AND IS TERMED AN \emph{APPLICATION CONTEXT}

\item	EACH APPLICATION CONTEXT IS ASSIGNED AN AUTHORITATIVE NUMBER
    \begin{nrtc}
    \item	TERMED AN OBJECT IDENTIFIER
    \end{nrtc}
    WHICH UNIQUELY IDENTIFIES THE PARTICULAR APPLICATION PROTOCOL, e.g.,
\begin{quote}\smaller\begin{verbatim}
{ iso standard(0) ftam(8571) application-context(1) iso-ftam(1) }
\end{verbatim}\end{quote}
\end{nrtc}
\end{bwslide}


\begin{bwslide}
\ctitle	{(PARTIAL) OBJECT IDENTIFIER TREE}

\vskip.5in
\diagram[p]{figureA-9}
\end{bwslide}


\begin{bwslide}
\ctitle	{APPLICATION LAYER}

\vskip.5in
\diagram[p]{figureA-2}
\end{bwslide}


\begin{bwslide}
\ctitle	{APPLICATION ASSOCIATIONS}

\begin{nrtc}
\item	AN APPLICATION \emph{ASSOCIATION} IS A PRESENTATION CONNECTION WITH
    \begin{nrtc}
    \item	ADDITIONAL APPLICATION LAYER SEMANTICS
    \end{nrtc}

\item	AT PRESENT, THIS IS LITTLE MORE THAN APPLICATION LAYER NAMING
\end{nrtc}
\end{bwslide}


\begin{bwslide}
\part*	{APPLICATION ENTITIES}\bf

\begin{nrtc}
\item	AN APPLICATION PROCESS (AP) IS SOMETHING WHICH EXECUTES IN THE NETWORK

\item	THE OSI COMMUNICATIONS ASPECT OF SUCH A PROCESS IS TERMED AN
    \begin{nrtc}
    \item	APPLICATION ENTITY (AE)
    \end{nrtc}
\end{nrtc}
\end{bwslide}


\begin{bwslide}
\ctitle	{WHEN AEs COMMUNICATE}

\begin{nrtc}
\item	PEERS ARE COMPOSED OF PRECISELY THE SAME ASEs

\item	EACH ASE TALKS ONLY WITH ITS PEER
\end{nrtc}
\end{bwslide}


\begin{bwslide}
\ctitle {HOW TO KEEP TRACK OF DATA STREAMS?}

\begin{nrtc}
\item	A PRESENTATION CONTEXT IS A PAIRING OF:
    \begin{nrtc}
    \item	A SET OF DATA STRUCTURE DEFINITIONS (ABSTRACT SYNTAX), AND

    \item	RULES FOR ENCODING THOSE DATA STRUCTURES ON THE NETWORK 
		(TRANSFER SYNTAX)
    \end{nrtc}

\item	REPRESENTED BY A PRESENTATION CONTEXT IDENTIFIER (PCI)

\item	ALL USER DATA GIVEN TO PRESENTATION IS TAGGED ACCORDINGLY

\item	CONTEXTS NORMALLY DEFINED DURING CONNECTION ESTABLISHMENT

\item	BUT MIGHT BE CHANGED
    \begin{nrtc}
    \item	EXPLICITLY: BY THE APPLICATION, OR,

    \item	IMPLICITLY: BY RESYNCHRONIZATION
    \end{nrtc}
\end{nrtc}
\end{bwslide}


\begin{bwslide}
\ctitle	{A FEW OTHER DETAILS}

\begin{nrtc}
\item	MULTIPLE APPLICATION \emph{ENTITIES} MAY BE PRESENT IN A SINGLE
	APPLICATION \emph{PROCESS}

\item	CURRENT SPECIFICATION ON
    \begin{nrtc}
    \item	OSI APPLICATION LAYER STRUCTURE
    \end{nrtc}
    BEARS LITTLE RESEMBLENCE TO THE ARCHITECTURE OF EXISTING OSI
    APPLICATIONS!
\end{nrtc}
\end{bwslide}


\begin{bwslide}
\part	{APPLICATION SERVICE ELEMENTS}\bf

\begin{nrtc}
\item	ASSOCIATION CONTROL (ACSE)

\item	RELIABLE TRANSFER (RTSE)

\item	REMOTE OPERATIONS (ROSE)

\item	USE OF APPLICATION SERVICES
\end{nrtc}
\end{bwslide}


\begin{bwslide}
\ctitle	{THERE ARE A LOT OF ASEs}

\begin{nrtc}
\item	EACH APPLICATION TYPICALLY DEFINES AT LEAST ONE ASE WHICH IS SPECIFIC
	TO THE APPLICATION

\item	HOWEVER,
    \begin{nrtc}
    \item	ACSE, ROSE, and RTSE
    \end{nrtc}
    ARE USED QUITE FREQUENTLY
    \begin{nrtc}
    \item	(OFTEN TERMED \emph{COMMON} ASEs)
    \end{nrtc}
\end{nrtc}
\end{bwslide}


\begin{bwslide}
\part*	{ASSOCIATION CONTROL (ACSE)}\bf

\begin{nrtc}
\item	USED BY \emph{ALL} OSI APPLICATIONS TO PERFORM ASSOCIATION
	ESTABLISHMENT AND RELEASE

\item	ALL OSI APPLICATIONS DEFINE HOW THE ACSE IS USED

\item	ONLY THE ACSE INVOKES PRESENTATION CONNECTION ESTABLISHMENT AND
	RELEASE SERVICES
\end{nrtc}
\end{bwslide}


\begin{bwslide}
\ctitle	{BINDING SERVICE}

\begin{nrtc}
\item	AN ASSOCIATION \emph{BINDS} AN INITIATOR TO A RESPONDER
    \begin{nrtc}
    \item	TYPICALLY FOR A CONSUMER/PROVIDER MODEL
    \end{nrtc}

\item	BINDING IS TWO-STEP:
    \begin{nrtc}
    \item	MAP SERVICE ONTO AVAILABLE ENTITIES

    \item	SELECT ENTITY BASED ON COMMUNICATIONS REQUIREMENTS
    \end{nrtc}
\end{nrtc}
\end{bwslide}


\begin{bwslide}
\ctitle	{NEGOTIATION}

\begin{nrtc}
\item	THIS IS THE VERY HEART OF ASSOCIATION ESTABLISHMENT
    \begin{nrtc}
    \item	SESSION: TOKENS, SERIAL NUMBERS, etc.

    \item	PRESENTATION: CONTEXTS

    \item	APPLICATION: ASSOCIATION CONTEXT (MOOT)
    \end{nrtc}
\end{nrtc}
\end{bwslide}


\begin{bwslide}
\ctitle	{NAMING}

\begin{nrtc}
\item	APPLICATION ENTITY IDENTIFIED AS:
    \begin{nrtc}
    \item	APPLICATION PROCESS TITLE

    \item	APPLICATION ENTITY QUALIFIER
    \end{nrtc}
    TERMED AN APPLICATION ENTITY TITLE

\item	AP TITLE: A DIRECTORY DISTINGUISHED NAME

\item	AE QUALIFIER: USUALLY NONE
\end{nrtc}
\end{bwslide}


\begin{bwslide}
\ctitle	{NAMING (cont.)}

\begin{nrtc}
\item	A DISTINGUISHED NAME (DN) IS AN ASN.1 OBJECT CONSISTING OF
    \begin{nrtc}
    \item	A SEQUENCE OF ATTRIBUTE/VALUE PAIRS
    \end{nrtc}
    IMPLYING A SUPERIOR/SUBORDINATE RELATIONSHIP, e.g.,
\begin{quote}\small\begin{verbatim}
countryName            = US
organizationName       = Performance Systems International
organizationalUnitName = Operations
commonName             = nisc
commonName             = filestore
\end{verbatim}\end{quote}

\item	REFERS TO AN INFORMATION OBJECT IN THE OSI DIRECTORY
\end{nrtc}
\end{bwslide}


\begin{bwslide}
\ctitle	{EXAMPLE DIRECTORY ENTRY}

\begin{quote}\small\begin{verbatim}
c=US
    @o=Performance Systems International@ou=Operations
    @cn=nisc@cn=filestore

    objectClass= top & applicationEntity
    presentationAddress= { "", "", "ftam",
                           { 47000580fffc000000000100020123456789ab00 } }
    supportedApplicationContext= 1.0.8571.1.1

    description=
    locality=
    organizationName=
    organizationalUnitName=
    seeAlso=

    objectClass= quipuObject
    acl=
    objectClass= iSODEApplicationEntity
    execVector= iso.ftam -c
\end{verbatim}\end{quote}
\end{bwslide}


\begin{bwslide}
\ctitle	{ADDRESSING}

\begin{nrtc}
\item	SIMPLY READ THE \verb"presentationAddress" ATTRIBUTE OF THE ENTRY IN
	THE DIRECTORY
\begin{quote}\small\begin{verbatim}
PresentationAddress ::=
    SEQUENCE {
        pSelector[0]  OCTET STRING OPTIONAL,
        sSelector[1]  OCTET STRING OPTIONAL,
        tSelector[2]  OCTET STRING OPTIONAL,
        nAddresses[3] SET OF (1..MAX) OCTET STRING
    }
\end{verbatim}\end{quote}

\item	NOTE THAT THIS IS THE EXTERNAL MACHINE FORM, ALSO NEED
    \begin{nrtc}
    \item	FORM WHEN CARRIED INSIDE PROTOCOLS

    \item	FORM WHEN TEXTUALLY DESCRIBED
    \end{nrtc}
\end{nrtc}
\end{bwslide}


\begin{bwslide}
\ctitle	{A STRING ENCODING OF\\ PRESENTATION ADDRESS}

\smaller
\begin{verbatim}
<address> ::= [[[ <psel> "/" ] <ssel> "/" ] <tsel> "/" ] <naddrs>

<naddrs>  ::= <naddr> "|" <naddrs> | <naddr>

<naddr>   ::= "NS" "+" <hexstring> | <afi> "+" <idi> [ "+" <dsp> ]

<psel>    ::= <selector>
<ssel>    ::= <selector>
<tsel>    ::= <selector>

<selector>::= '"' <ia5string> '"' | "#" digits | "'" <hexstring> "'H" | ""
\end{verbatim}
\end{bwslide}


\begin{bwslide}
\ctitle	{EXAMPLES}

\begin{verbatim}
"256"/NS+a433bb93c1|NS+aa3106

#63/#41/#12/X121+234219200300

'3a'H/Janet=00002340555+CUDF+"892796"

"s"//Internet=10.0.0.6

psinet=0000000100020123456789ab00
\end{verbatim}
\end{bwslide}


\begin{bwslide}
\ctitle	{NAMING REVISITED}

\begin{nrtc}
\item	HOW TO BUILD DISTINGUISHED NAME?

\item	USERS PREFER THINGS LIKE
\begin{quote}\small\begin{verbatim}
% ftam nisc
% ftam nisc,psi,us
\end{verbatim}\end{quote}

\item	WHICH DON'T LOOK LIKE DIRECTORY DISTINGUISHED NAMES

\item	WE'LL LOOK AT A SOLUTION LATER ON
\end{nrtc}
\end{bwslide}


\begin{bwslide}
\part*	{RELIABLE TRANSFER (RTSE)}\bf

\begin{nrtc}
\item	RESPONSIBLE FOR BULK-MODE TRANSFER
    \begin{nrtc}
    \item	OPTIONALLY WITH RECOVERY
    \end{nrtc}

\item	SIMPLIFY USE OF SESSION SERVICES FOR APPLICATIONS
    \begin{nrtc}
    \item	THE SO-CALLED ``THE SEWER OF OSI''
    \end{nrtc}

\item	USES THE ACSE FOR BINDING
\end{nrtc}
\end{bwslide}


\begin{bwslide}
\ctitle	{RELIABILITY IN THE RTSE}

\begin{nrtc}
\item	SERVICE SPECIFICATION IS PROBLEMATIC SINCE RTSE, NOT ACCEPTOR,
	PROVIDES CONFIRMATION

\end{nrtc}
\vskip.5in
\diagram[p]{figureA-3}
\end{bwslide}


\begin{bwslide}
\ctitle	{AN INCONSISTENCY IN THE MODEL!}

\begin{nrtc}
\item	RTSE DEALS WITH LARGE PRESENTATION DATA VALUES

\item	BUT, PRESENTATION REQUIRES SYNCHRONIZATION OCCUR \emph{BETWEEN} DATA
	VALUES

\item	SO, RTSE SERIALIZES EACH DATA VALUE INTO AN OCTET STRING
    \begin{nrtc}
    \item	AND BREAKS THAT UP FOR SYNCHRONIZATION PURPOSES
    \end{nrtc}
\end{nrtc}
\end{bwslide}


\begin{bwslide}
\part*	{REMOTE OPERATIONS (ROSE)}\bf

\begin{nrtc}
\item	MANAGE REQUEST/REPLY INTERACTIONS
    \begin{nrtc}
    \item	(SIMILAR TO REMOTE PROCEDURE CALLS)
    \end{nrtc}

\item	AS WITH ALL CURRENT SERVICES, CO-MODE

\item	DISTINGUISHES BETWEEN
    \begin{nrtc}
    \item	INITIATOR/RESPONDER

    \item	CONSUMER/PROVIDER

    \item	INVOKER/PERFORMER
    \end{nrtc}
\item	PROBABLY THE ``BUILDING BLOCK'' FOR THE NEXT WAVE OF APPLICATIONS
\end{nrtc}
\end{bwslide}


\begin{bwslide}
\ctitle	{REMOTE OPERATIONS}

\begin{nrtc}
\item	INVOKER REQUESTS AN OPERATION

\item	PERFORMER GENERATES ONE OF THREE OUTCOMES:
    \begin{nrtc}
    \item       A \emph{RESULT}, IF THE OPERATION SUCCEEDS;

    \item       AN \emph{ERROR}, IF THE OPERATION FAILED; or,

    \item       A \emph{REJECTION}, IF THE OPERATION WAS NOT PERFORMED
    \end{nrtc}
\end{nrtc}
\end{bwslide}


\begin{bwslide}
\ctitle	{CONTEXT WITH THE RTSE}

\vskip.5in
\diagram[p]{figureA-4}
\end{bwslide}


\begin{bwslide}
\ctitle	{CONTEXT WITHOUT THE RTSE}

\vskip.5in
\diagram[p]{figureA-5}
\end{bwslide}


\begin{bwslide}
\part*	{USE OF APPLICATION SERVICES}\bf

\begin{nrtc}
\item	HOW DOES IT ALL COME TOGETHER?

\item	FINALLY, AN EXAMPLE!
\end{nrtc}
\end{bwslide}


\begin{bwslide}
\ctitle	{EXAMPLE: FILE TRANSFER}

\vskip.5in
\diagram[p]{figureA-6}
\end{bwslide}


\begin{bwslide}
\ctitle	{CAUSALITY OF EVENTS FOR}

\begin{nrtc}
\item	ASSOCIATION ESTABLISHMENT

\item	DATA TRANSFER

\item	ASSOCIATION RELEASE
\end{nrtc}
\end{bwslide}


\begin{bwslide}
\ctitle	{ASSOCIATION ESTABLISHMENT (USER)}

\begin{nrtc}
\item	USER TYPES
\begin{quote}\small\begin{verbatim}
% ftam nisc,psi,us
\end{verbatim}\end{quote}
	WHICH INVOKES FTAM USER INTERFACE
\end{nrtc}
\end{bwslide}


\begin{bwslide}
\ctitle	{PRIOR TO BINDING}

\vskip.5in
\diagram[p]{figureA-7}
\end{bwslide}


\begin{bwslide}
\ctitle	{DIRECTORY ACCESS}

\begin{nrtc}
\item	DIRECTORY ACCESS SERVICE ELEMENT (DASE) IS INVOKED TO RETRIEVE
	PRESENTATION ADDRESS

\item	DASE USES HARD-WIRED INFORMATION TO DERIVE ADDRESS OF DIRECTORY SYSTEM
	AGENT (DSA)

\item	DASE INVOKES ACSE TO ESTABLISH A DAP ASSOCIATION TO DSA

\item	USER ELEMENT MUST CONSTRUCT DISTINGUISHED NAME
\end{nrtc}
\end{bwslide}


\begin{bwslide}
\ctitle	{DIRECTORY ACCESS (cont.)}

\begin{nrtc}
\item	DASE ISSUES READ OPERATION ON ENTRY FOR \verb"presentationAddress"
	ATTRIBUTE

\item	DIRECTORY READ OPERATION IS MAPPED ONTO THE REMOTE OPERATIONS SERVICE

\item	WHEN OPERATION COMPLETES, DASE MIGHT CLOSE DAP ASSOCIATION

\item	PRESENTATION ADDRESS IS RETURNED TO USER ELEMENT
\end{nrtc}
\end{bwslide}


\begin{bwslide}
\ctitle	{AND NOW THE FUN BEGINS!}

\vskip.5in
\diagram[p]{figureA-8}
\end{bwslide}


\begin{bwslide}
\ctitle	{ASSOCIATION ESTABLISHMENT (FTAM)}

\begin{nrtc}
\item	THE FTAM ASE GENERATES A DATA OBJECT TO INITIALIZE ITS PEER

\item	THIS IS PASSED AS USER DATA TO THE ACSE FOR ASSOCIATION ESTABLISHMENT

\item	THE FTAM ASE REQUESTS PRESENTATION CONTEXTS FOR ITSELF
    \begin{nrtc}
    \item	AND A CONTEXT FOR THE ACSE
    \end{nrtc}
\end{nrtc}
\end{bwslide}


\begin{bwslide}
\ctitle	{F-INITIALIZE-request FPDU}

\smaller
\begin{verbatim}
{
   service-class {
      management-class, transfer-class,
      transfer-and-management-class
   },
   functional-units {
      read, write, limited-file-management,
      enhanced-file-management, grouping
   },
   attribute-groups { storage },
   ftam-quality-of-service no-recovery,
   contents-type-list {
      1.0.8571.5.3,    -- FTAM-3 document
      1.0.8571.5.1,    -- FTAM-1 document
      1.3.9999.1.5.9   -- NBS-9 document
   },
   initiator-identity "cheetah"
}
\end{verbatim}
\end{bwslide}


\begin{bwslide}
\ctitle	{ASSOCIATION ESTABLISHMENT (ACSE)}

\begin{nrtc}
\item	THE ACSE GENERATES A DATA OBJECT TO INITIALIZE ITS PEER

\item	THIS IS PASSED AS USER DATA DURING PRESENTATION CONNECTION
	ESTABLISHMENT 
\end{nrtc}
\end{bwslide}


\begin{bwslide}
\ctitle	{AARQ APDU}

\smaller
\begin{verbatim}
{
   application-context-name 1.0.8571.1.1,    -- iso ftam
   user-information {
      {
         direct-reference 2.1.1,
         indirect-reference 1,               -- indicates FTAM PCI
         encoding {
            single-ASN1-type {
                [3] '0370'H,
                [4] '053700'H,
                [5] '0580'H,
                [6] '00'H,
                [7] {
                    [APPLICATION 14] '28c27b0503'H,
                    [APPLICATION 14] '28c27b0501'H,
                    [APPLICATION 14] '2bce0f010509'H
                },
                [APPLICATION 22] "cheetah"
         }
      }
   }
}
\end{verbatim}
\end{bwslide}


\begin{bwslide}
\ctitle	{ASSOCIATION ESTABLISHMENT (PRESENTATION)}

\begin{nrtc}
\item	THE LOCAL PRESENTATION ENTITY GENERATES A DATA OBJECT TO INITIALIZE
	ITS PEER

\item	THIS IS PASSED AS USER DATA DURING SESSION CONNECTION ESTABLISHMENT 
\end{nrtc}
\end{bwslide}


\begin{bwslide}
\ctitle	{CP-type PPDU}

\scriptsize
\begin{verbatim}
{
   mode {
      normal-mode
   },
   normal-mode {
      context-list {
         {                -- ftam pci abstract syntax
            identifier 1,
            abstract-syntax 1.0.8571.2.1,
            transfer-syntax-list { 2.1.1 }
         },
...
\end{verbatim}
\end{bwslide}


\begin{bwslide}
\ctitle	{CP-type PPDU (cont.)}

\scriptsize
\begin{verbatim}
{
...
         {                -- FTAM-3 abstract syntax
            identifier 3,
            abstract-syntax 1.0.8571.2.4,
            transfer-syntax-list { 2.1.1 }
         },
         {                -- FTAM-1 abstract syntax
            identifier 5,
            abstract-syntax 1.0.8571.2.3,
            transfer-syntax-list { 2.1.1 }
         },
         {                -- NBS-9 abstract syntax
            identifier 7,
            abstract-syntax 1.3.999.1.2.2,
            transfer-syntax-list { 2.1.1 }
         },
         {                -- acse pci abstract syntax
            identifier 9,
            abstract-syntax 2.2.1.0.1,
            transfer-syntax-list { 2.1.1 }
         }
      },
...
\end{verbatim}
\end{bwslide}


\begin{bwslide}
\ctitle	{CP-type PPDU (cont.)}

\scriptsize
\begin{verbatim}
...
      user-data {
         complex {
            {
               identifier 9,                -- indicates ACSE PCI
               presentation-data-values {
                  single-ASN1-type {
                     [1] { 1.0.8571.1.1 },
                     [30] {
                        {
                           direct-reference 2.1.1,
                           indirect-reference 1,
                           encoding {
                              single-ASN1-type {
                                 [3] '0370'H,
                                 [4] '053700'H,
                                 [5] '0580'H,
                                 [6] '00'H,
                                 [7] {
                                    [APPLICATION 14] '28c27b0503'H,
                                    [APPLICATION 14] '28c27b0501'H,
                                    [APPLICATION 14] '2bce0f010509'H
                                 },
                                 [APPLICATION 22] "cheetah"
}  }  }  }  }  }  }  }  }  }  }
\end{verbatim}
\end{bwslide}


\begin{bwslide}
\ctitle	{ASSOCIATION ESTABLISHMENT (SESSION)}

\begin{nrtc}
\item	THE LOCAL SESSION ENTITY GENERATES A DATA OBJECT TO INITIALIZE
	ITS PEER

\item	ONCE A TRANSPORT CONNECTION IS ESTABLISHED, THIS IS PASSED AS USER DATA
\end{nrtc}
\end{bwslide}


\begin{bwslide}
\ctitle	{CONNECT SPDU}

\smaller
\begin{verbatim}
LI/           241
CODE/         CONNECT
REFERENCE/    <USER        9 140763686565746168,
               COMMON     15 170d3930303132363031313434355a,
               ADDITIONAL  0>
OPTIONS/      0x0<>
VERSION/      0x2
REQUIREMENTS/ 0x2<DUPLEX>
USER DATA/    197 3180a003800101a280a45a30100201 ...
\end{verbatim}
\end{bwslide}


\begin{bwslide}
\ctitle	{SERVICE MAPPINGS}

\scriptsize
\begin{verbatim}
A-ASSOCIATE.REQUEST
    AARQ    ---->    P-CONNECT.REQUEST
                         CP-type    ---->    S-CONNECT.REQUEST
                                                 CONNECT SPDU

                                                 T-CONNECT stuff
                                                 T-DATA.REQUEST     ---->
                                                    ...
                                                 T-DATA.INDICATION <----

                                                 ACCEPT SPDU
                         CPA-type  <----    S-CONNECT.CONFIRMATION
    AARE    <----    P-CONNECT.CONFIRMATION
A-ASSOCIATE.CONFIRMATION
\end{verbatim}
\end{bwslide}


\begin{bwslide}
\ctitle	{PARAMETER MAPPINGS}

\scriptsize
\[\begin{tabular}{|l|c|c|c|c|c|}
\hline
\begin{tabular}{c}A-ASSOCIATE\\ .REQUEST\end{tabular}&
		\begin{tabular}{c}AARQ\\ APDU\end{tabular}&
			\begin{tabular}{c}P-CONN\\ .REQ\end{tabular}&
				\begin{tabular}{c}CP-type\\ PPDU\end{tabular}&
					\begin{tabular}{c}S-CONN\\ .REQ\end{tabular}&
						\begin{tabular}{c}CN\\ SPDU\end{tabular}\\
\hline
mode&		&	\X/&	&	&	\\
\hline
application&	&	&	&	&	\\
\ \ context name&\X/&	&	&	&	\\
\hline
calling AE info&\X/&	&	&	&	\\
\hline
called AE info&	\X/&	&	&	&	\\
\hline
user-data&	\X/&	&	&	&	\\
\hline
calling P-address&
		&	P-addr&
				P-sel&
					S-addr&	
						S-sel\\
\hline
called P-address&
		&	P-addr&
				P-sel&
					S-addr&	
						S-sel\\
\hline
presentation&	&	&	&	&	\\
\ \ context&	&	\X/&	\X/&	&	\\
\ \ definition list&
		&	&	&	&	\\
\hline
default&	&	&	&	&	\\
\ \ presentation&&	\X/&	\X/&	&	\\
\ \ context name&
		&	&	&	&	\\
\hline
quality of service&
		&	\X/&	&	\X/&	\\
\hline
presentation&	&	&	&	&	\\
\ \ requirements&&	\X/&	\X/&	&	\\
\hline
session&	&	&	&	&	\\
\ \ requirements&&	\X/&	&	\X/&	\X/\\
\hline
initial serial&	&	&	&	&	\\
\ \ number&	&	\X/&	&	\X/&	\X/\\
\hline
initial token&	&	&	&	&	\\
\ \ assignment&	&	\X/&	&	\X/&	\X/\\
\hline
session connection&
		&	&	&	&	\\
\ \ identifier&	&	\X/&	&	\X/&	\X/\\
\hline
\end{tabular}\]
\end{bwslide}


\begin{bwslide}
\ctitle	{DATA TRANSFER}

\begin{nrtc}
\item	DO FTAM PRIMITIVES TO SELECT, OPEN FILE, etc.

\item	SEND FILE DATA ELEMENT USING CORRECT PRESENTATION CONTEXT:
\begin{quote}\small\begin{verbatim}
{
   complex {
      {                   -- FTAM-1 abstract syntax
         identifier 5,
         presentation-data-values { -- a single DE
            single-ASN1-type ...
         }
      }
   }
}
\end{verbatim}\end{quote}

\item	SERVICE MAPPINGS:
\begin{quote}\small\begin{verbatim}
F-DATA.REQUEST
    FADU    ---->    P-DATA.REQUEST
                         User-data    ---->    S-DATA.REQUEST
                                                   DT SPDU

                                                 T-DATA.REQUEST  ---->
\end{verbatim}\end{quote}
\end{nrtc}
\end{bwslide}


\begin{bwslide}
\ctitle	{ASSOCIATION RELEASE}

\begin{nrtc}
\item	FTAM ASE GENERATES AN \verb"F-TERMINATE-request" FPDU
\begin{quote}\small\begin{verbatim}
{}
\end{verbatim}\end{quote}

\item	WHICH IS PASSED AS USER DATA TO THE ACSE FOR ASSOCIATION RELEASE
\end{nrtc}
\end{bwslide}


\begin{bwslide}
\ctitle	{ASSOCIATION RELEASE (ACSE)}

\begin{nrtc}
\item	THE ACSE GENERATES AN \verb"RLRQ" APDU
\begin{quote}\small\begin{verbatim}
{
   reason normal,
   user-information {
      {
         indirect-reference 1,
         encoding {
            single-ASN1-type {}
         }
       }
   }
}
\end{verbatim}\end{quote}

\item	WHICH IS PASSED AS USER DATA TO THE LOCAL PRESENTATION ENTITY FOR
	CONNECTION RELEASE
\end{nrtc}
\end{bwslide}


\begin{bwslide}
\ctitle	{ASSOCIATION RELEASE (PRESENTATION)}

\begin{nrtc}
\item	THE LOCAL PRESENTATION ENTITY GENERATES AN \verb"User-Data" PPDU
\begin{quote}\small\begin{verbatim}
{
   complex {
      {
         identifier 9,
         presentation-data-values {
            single-ASN1-type {
               [0] '00'H,
               [30] {
                  {
                     indirect-reference 1,
                     encoding {
                        single-ASN1-type {}
                     }
                  }
               }
            }
         }
      }
   }
}
\end{verbatim}\end{quote}

\item	WHICH IS PASSED AS USER DATA TO THE LOCAL SESSION ENTITY FOR
	CONNECTION RELEASE
\end{nrtc}
\end{bwslide}


\begin{bwslide}
\ctitle	{ASSOCIATION RELEASE (SESSION)}

\begin{nrtc}
\item	THE LOCAL SESSION ENTITY GENERATES A \verb"FINISH" SPDU
\begin{quote}\small\begin{verbatim}
LI/           30
CODE/         FINISH
DISCONNECT/   0x1<RELEASE>
USER DATA/    25 61173015020109a010620e800100be ...
\end{verbatim}\end{quote}

\item	WHICH IS PASSED AS USER DATA TO THE TRANSPORT SERVICE
\end{nrtc}
\end{bwslide}


\begin{bwslide}
\ctitle	{SERVICE MAPPINGS}

\scriptsize
\begin{verbatim}
A-RELEASE.REQUEST
    RLRQ    ---->    P-RELEASE.REQUEST
                         User-Data  ---->    S-RELEASE.REQUEST
                                                 FINISH SPDU

                                                 T-DATA.REQUEST     ---->
                                                    ...
                                                 T-DATA.INDICATION <----

                                                 DISCONNECT SPDU
                                                 T-DISCONNECT stuff
                         User-data <----    S-RELEASE.CONFIRMATION
    RLRE    <----    P-RELEASE.CONFIRMATION
A-RELEASE.CONFIRMATION
\end{verbatim}
\end{bwslide}
