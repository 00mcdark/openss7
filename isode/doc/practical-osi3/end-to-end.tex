% run this through LaTeX with the appropriate wrapper

\dotopic{END-TO-END SERVICES}

\begin{bwslide}
\part*	{OUTLINE}\bf

\begin{description}
\item[PART I:]		CONCEPTS

\item[PART II:]		BUILDING BLOCKS

\item[PART III:]	ACHIEVING CONNECTIVITY

\item[PART IV:]		COMPARISON TO TCP/IP
\end{description}
\end{bwslide}


\begin{bwslide}
\ctitle	{A BIG ACKNOWLEDGEMENT}

\begin{nrtc}
\item	MY INTEREST IN END-TO-END SERVICES IS ONLY AS A USER,
	NOT A PROVIDER

\item	AS SUCH, I'D PREFER TO USE THEM AS A BLACK BOX

\item	UNFORTUNATELY, THIS APPROACH DOESN'T WORK IN PRACTICE
    \begin{nrtc}
    \item	THE LOWER-LAYERS AREN'T HOMOGENEOUS
    \end{nrtc}

\item	THE PRACTICAL PERSPECTIVE PRESENTED HERE IS HEAVILY INFLUENCED BY
    \begin{nrtc}
    \item	STEPHEN E.~KILLE OF UNIVERSITY COLLEGE LONDON
    \end{nrtc}

\item	AND HIS PAPER
    \begin{nrtc}
    \item	``AN INTERIM APPROACH TO USE OF NETWORK ADDRESSES''
    \end{nrtc}
\end{nrtc}
\end{bwslide}


\begin{bwslide}
\part	{CONCEPTS}\bf

\begin{nrtc}
\item	BASIC TERMINOLOGY

\item	NETWORK SERVICE

\item	TRANSPORT SERVICE
\end{nrtc}
\end{bwslide}


\begin{bwslide}
\part*	{BASIC TERMINOLOGY}\bf

\begin{nrtc}
\item	END-TO-END SERVICES RESPONSIBLE FOR
    \begin{nrtc}
    \item	DATA TRANSFER
    \end{nrtc}

\item	APPLICATION SERVICES RESPONSIBLE FOR
    \begin{nrtc}
    \item	INFORMATION TRANSFER
    \end{nrtc}
\end{nrtc}
\end{bwslide}


\begin{bwslide}
\ctitle	{BASIC TERMINOLOGY (cont.)}

\begin{nrtc}
\item	TERMINOLOGY DIFFERS BETWEEN NETWORKING COMMUNITIES
    \begin{nrtc}
    \item	WE'LL USE ``OSIFIED'' TERMINOLOGY
    \end{nrtc}

\item	A NETWORK CONSISTS OF A COLLECTION OF SUBNETWORKS CONNECTED
	BY INTERMEDIATE SYSTEMS AND POPULATED BY END-SYSTEMS

\item	DATA TRANSFER OCCURS BETWEEN TWO END-SYSTEMS,
	POTENTIALLY GOING THROUGH ONE OR MORE INTERMEDIATE-SYSTEMS
	IF THE END-SYSTEMS RESIDE ON DIFFERENT SUBNETWORKS
\end{nrtc}
\end{bwslide}


\begin{bwslide}
\ctitle	{THE NETWORK}

\vskip.5in
\diagram[p]{figureE-2}
\end{bwslide}


\begin{bwslide}
\ctitle	{END-SYSTEMs (ES)}

\begin{nrtc}
\item	CONTAIN BOTH: 
    \begin{nrtc}
    \item	THE LOWER-LAYER PROTOCOLS NECESSARY FOR DATA TRANSFER, AND

    \item	THE UPPER-LAYER PROTOCOLS NECESSARY FOR INFORMATION TRANSFER
    \end{nrtc}

\item	WHERE THE APPLICATIONS LIVE

\item	WHAT THE USERS ARE INTERESTED IN
\end{nrtc}
\end{bwslide}


\begin{bwslide}
\ctitle	{INTERMEDIATE-SYSTEMs (IS)}

\begin{nrtc}
\item	CONTAIN ONLY:
    \begin{nrtc}
    \item	THE LOWER-LAYER PROTOCOLS NECESSARY FOR DATA TRANSFER
    \end{nrtc}

\item	ULTIMATELY CONTAIN HIGHER-LAYER PROTOCOLS TO SUPPORT MANAGEMENT

\item	IN ADDITION TO PASSING ALONG APPLICATION DATA,
	INTERMEDIATE-SYSTEMS COOPERATE AMONGST THEMSELVES
    \begin{nrtc}
    \item	e.g., EXCHANGE ROUTING DATA
    \end{nrtc}
\end{nrtc}
\end{bwslide}


\begin{bwslide}
\part*	{NETWORK SERVICE}\bf

\begin{nrtc}
\item	NETWORK SERVICE IS RESPONSIBLE FOR MOVING DATA FROM ONE END-SYSTEM
	TO ANOTHER

\item	UNFORTUNATELY, THERE ARE TWO DIFFERENT VIEWS AS TO WHAT THIS MEANS:
    \begin{nrtc}
    \item	CONNECTION-ORIENTED

    \item	CONNECTIONLESS-MODE
    \end{nrtc}

\item	PERHAPS THE GREATEST ``RELIGIOUS'' ISSUE OF THE LAST DECADE
\end{nrtc}
\end{bwslide}


\begin{bwslide}
\ctitle	{CONNECTION-ORIENTED NETWORK SERVICE\\ (CONS)}

\begin{nrtc}
\item	BASED ON THE NOTION OF ``RESERVATIONS'':
    \begin{nrtc}
    \item	ON CONNECTION REQUEST, MINIMUM REQUIREMENTS ARE STATED
	\begin{nrtc}
	\item	(e.g., THROUGHPUT)
	\end{nrtc}

    \item	IF REQUEST IS GRANTED, THESE RESOURCES ARE RESERVED FOR THE
		CONNECTION'S DURATION
    \end{nrtc}

\item	CO-MODE SERVICE PRIMITIVES
    \begin{nrtc}
    \item	N-CONNECT: CONNECTION ESTABLISHMENT

    \item	N-DATA (N-DATA-ACKNOWLEDGE): DATA TRANSFER 

    \item	N-EXPEDITED-DATA: EXPEDITED DATA TRANSFER

    \item	N-DISCONNECT: CONNECTION RELEASE

    \item	N-RESET: CONNECTION RESYNCHRONIZATION
    \end{nrtc}
\end{nrtc}
\end{bwslide}


\begin{bwslide}
\ctitle	{CONS (cont.)}

\begin{nrtc}
\item	GOOD POINTS:
    \begin{nrtc}
    \item	LOW OVERHEAD FOR DATA TRANSIT

    \item	IMMUNITY FROM OTHER NETWORK TRAFFIC

    \item	ACCOUNTABILITY
    \end{nrtc}

\item	BAD POINTS:
    \begin{nrtc}
    \item	HIGH OVERHEAD FOR CONNECTION ESTABLISHMENT

    \item	QUESTIONABLE RECOVERY CHARACTERISTICS

    \item	IF RESOURCES ARE RESERVED, BUT NOT IN USE,
		NEW CONNECTION REQUESTS ARE DENIED
    \end{nrtc}
\end{nrtc}
\end{bwslide}


\begin{bwslide}
\ctitle	{CONNECTIONLESS-MODE NETWORK SERVICE\\ (CLNS)}

\begin{nrtc}
\item	BASED ON THE NOTION OF ``COME AS YOU ARE'':
    \begin{nrtc}
    \item	NO CONNECTION REQUEST, JUST SEND DATA

    \item	TRANSPORT MUST DYNAMICALLY DETERMINE IF REQUIREMENTS ARE
		BEING MET
    \end{nrtc}

\item	CL-MODE SERVICE PRIMITIVES
    \begin{nrtc}
    \item	N-UNITDATA: DATA TRANSFER
    \end{nrtc}
\end{nrtc}
\end{bwslide}


\begin{bwslide}
\ctitle	{CLNS (cont.)}

\begin{nrtc}
\item	GOOD POINTS:
    \begin{nrtc}
    \item	LESS DELAY FOR INITIAL DATA TRANSIT

    \item	POTENTIALLY MORE ROBUST WITH CHANGES IN THE NETWORK

    \item	SQUEEZES ``LAST DROP'' FROM AVAILABLE RESOURCES
    \end{nrtc}

\item	BAD POINTS:
    \begin{nrtc}
    \item	HIGHER OVERHEAD FOR DATA TRANSIT IF MULTIPLE SUBNETWORKS
		ARE INVOLVED

    \item	REQUIRES WELL-BEHAVED USERS TO PREVENT OVER-SUBSCRIPTION
    \end{nrtc}
\end{nrtc}
\end{bwslide}


\begin{bwslide}
\part*	{TRANSPORT SERVICE}

\begin{nrtc}
\item	TRANSPORT SERVICE IS RESPONSIBLE FOR MOVING DATA FROM ONE END-SYSTEM
	TO ANOTHER~---~RELIABLY
    \begin{nrtc}
    \item	(WE'RE CONSIDERING ONLY CO-MODE TRANSPORT SERVICE)
    \end{nrtc}

\item	IF CO-MODE NETWORK SERVICE IS USED, THIS IS TRIVIAL

\item	OTHERWISE, SOPHISTICATED ALGORITHMS ARE REQUIRED IN PROTOCOLS
	WHICH IMPLEMENT TRANSPORT SERVICE
\end{nrtc}
\end{bwslide}


\begin{bwslide}
\ctitle	{TRANSPORT SERVICE (cont.)}

\begin{nrtc}
\item	IMPORTANT IMPLICATION:\\
    \begin{nrtc}
    \item	AVAILABLE NETWORK SERVICE DETERMINES WHICH
		TRANSPORT PROTOCOL CAN BE USED

    \item	HOWEVER, WHEN INITIATING A CONNECTION,
		TRANSPORT SERVICE IS ACTIVE PRIOR TO NETWORK SERVICE!
    \end{nrtc}
\end{nrtc}
\end{bwslide}


\begin{bwslide}
\ctitle	{CHOICE OF NETWORK SERVICE}

\begin{nrtc}
\item	CHOICE OF NETWORK SERVICE IS ECO-POLITICAL NOT TECHNICAL
    \begin{nrtc}
    \item	EITHER APPROACH CAN BE MADE TO WORK WELL
    \end{nrtc}

\item	CO-MODE NETWORK SERVICE IS MORE SUITED TOWARDS A COMMON-CARRIER MODEL
    \begin{nrtc}
    \item	ACCOUNTABILITY AND ISOLATION
    \end{nrtc}
    THIS IS TYPIFIED BY PUBLIC DATA NETWORKS

\item	CL-MODE NETWORK SERVICE IS MORE GENERAL
    \begin{nrtc}
    \item	ADAPTABILITY AND COOPERATION
    \end{nrtc}
    THIS IS TYPIFIED BY CLOSED COMMUNITY NETWORKS

\item	HOWEVER, THE TWO APPROACHES DON'T MIX WELL
\end{nrtc}
\end{bwslide}


\begin{bwslide}
\part	{BUILDING BLOCKS}\bf

\begin{nrtc}
\item	ADDRESS FORMATS

\item	NETWORK BINDING

\item	TRANSPORT PROTOCOLS

\item	APPLICATION USE OF END-TO-END SERVICES

\item	EMULATION OF OSI END-TO-END SERVICES
\end{nrtc}
\end{bwslide}


\begin{bwslide}
\part*	{ADDRESS FORMATS}\bf

\begin{nrtc}
\item	HIERARHICALLY STRUCTURED
    \begin{nrtc}
    \item	ADDRESSING DOMAINS, SUB-DOMAINS

    \item	UNAMBIGUOUS PREFIXES
    \end{nrtc}

\item	MAIN GOAL: FACILITATE ALLOCATION

\item	NO IMPLICATIONS ON ``HOW TO GET THERE''
    \begin{nrtc}
    \item	BUT STRUCTURE MAY FACILITATE ROUTING DECISIONS
    \end{nrtc}
\end{nrtc}
\end{bwslide}


\begin{bwslide}
\ctitle	{ADDRESS FORMATS (cont.)}

\begin{nrtc}
\item	AN ADDRESSING AUTHORITY DEFINES STRUCTURE OF DOMAIN
    \begin{nrtc}
    \item	TERMED AN ABSTRACT SYNTAX
    \end{nrtc}
    AND ALSO ALLOCATES VALUES

\item	A TRANSFER SYNTAX DEFINES HOW ADDRESSES ARE ENCODED
\end{nrtc}
\end{bwslide}


\begin{bwslide}
\ctitle	{TOP-LEVEL}

\begin{nrtc}
\item	ADDRESS IS DIVIDED INTO TWO PARTS:
    \begin{nrtc}
    \item	INITIAL DOMAIN PART (IDP), AND

    \item	DOMAIN SPECIFIC PART (DSP)
    \end{nrtc}
\end{nrtc}

\diagram[p]{figureE-3}
\end{bwslide}


\begin{bwslide}
\ctitle	{TOP-LEVEL (cont.)}

\begin{nrtc}
\item	AUTHORITY AND FORMAT IDENTIFIER (AFI) DEFINES HOW
    \begin{nrtc}
    \item	IDI IS INTERPRETED, AND

    \item	HOW DSP IS FORMATTED
	\begin{nrtc}
	\item	(DECIMAL/BINARY ABSTRACT SYNTAX)
	\end{nrtc}
    \end{nrtc}

\item	INITIAL DOMAIN IDENTIFIER (IDI) SAYS WHO OWNS THE DSP
    \begin{nrtc}
    \item	MIGHT BE VARIABLE LENGTH

    \item	MIGHT HAVE (SIGNIFICANT) LEADING ZEROS
    \end{nrtc}

\item	DOMAIN SPECIFIC PART (DSP) IS JUST THAT
\end{nrtc}
\end{bwslide}


\begin{bwslide}
\ctitle	{EXAMPLE 1:\\ X.121 ADDRESS}

\begin{nrtc}
\item	AN X.121 ADDRESS MAY BE ENCODED USING
    \begin{nrtc}
    \item	AFI = 36

    \item	IDI = X.121 ADDRESS (UP TO 14~DIGITS)
    \end{nrtc}
\end{nrtc}

\diagram[p]{figureE-4}
\end{bwslide}


\begin{bwslide}
\ctitle	{EXAMPLE 2:\\ ICD ADDRESS}

\begin{nrtc}
\item	AN INTERNATIONALLY RECOGNIZED ENTITY MAY ALLOCATE ADDRESSES USING
    \begin{nrtc}
    \item	AFI = 47

    \item	IDI = INTERNATIONAL CODE DESIGNATOR (4~DIGITS)
    \end{nrtc}
\end{nrtc}

\diagram[p]{figureE-5}
\end{bwslide}


\begin{bwslide}
\ctitle	{EXAMPLE 3:\\ U.S.~GOSIP v2 ADDRESS}

\begin{nrtc}
\item	DESIGNED FOR VERY LARGE INTERNETS
    \begin{nrtc}
    \item	AFI/IDI = 470005 (US GOVT.)
    \end{nrtc}
\end{nrtc}

\diagram[p]{figureE-21}
\end{bwslide}


\begin{bwslide}
\ctitle	{EXAMPLE 4:\\ LOCAL ADDRESS}

\begin{nrtc}
\item	ANYONE MIGHT USE A ``LOCAL'' ADDRESSING FORMAT
    \begin{nrtc}
    \item	AFI = 49

    \item	IDI = NULL (0~DIGITS)
    \end{nrtc}
\end{nrtc}

\diagram[p]{figureE-6}
\end{bwslide}


\begin{bwslide}
\part*	{NETWORK BINDING}\bf

\begin{nrtc}
\item	HOW DOES DATA GO FROM ORIGINATING TO DESTINATION END-SYSTEM?
    \begin{nrtc}
    \item	i.e., HOW IS ROUTING ACCOMPLISHED?
    \end{nrtc}

\item	NETWORK SERVICE AT ORIGINATING END-SYSTEM DECIDES ``NEXT HOP''

\item	IF DESTINATION END-SYSTEM IS ON SAME SUBNETWORK,
	THEN NEXT HOP IS DESTINATION END-SYSTEM

\item	OTHERWISE, NEXT HOP IS AN INTERMEDIATE SYSTEM (ON THE SAME SUBNETWORK)
	WHICH IS ``CLOSER'' TO THE DESTINATION END-SYSTEM
\end{nrtc}
\end{bwslide}


\begin{bwslide}
\ctitle	{DETERMINING THE NEXT HOP}

\begin{nrtc}
\item	SYSTEMS MAINTAIN ROUTING TABLES CONTAINING PAIRS OF
    \begin{nrtc}
    \item	DESTINATION PREFIX, NEXT-HOP NSAP
    \end{nrtc}
    e.g.,
\begin{quote}\small\begin{verbatim}
47000580fffc00000000010001 hop1
47000580fffc00000000010002 hop2
47000580fffc0000000002     hop3
47000580                   hop4
\end{verbatim}\end{quote}

\item	NEXT HOP IS DETERMINED BY FINDING THE ENTRY WITH THE LONGEST PREFIX
	MATCH 

\item	ALLOWS FOR VERY LARGE SCALING SINCE ONLY NSAP PREFIXES ARE EXCHANGED
	BY ROUTING PROTOCOLS

\item	ONCE THE DESTINATION END-SYSTEM'S SUBNETWORK HAS BEEN REACHED,
	NEED A WAY OF DETERMINING ``WHERE IT IS'' ON A PARTICULAR
	SUBNETWORK
\end{nrtc}
\end{bwslide}


\begin{bwslide}
\ctitle	{SUBNETWORK POINT OF ATTACHMENT (SNPA)}

\begin{nrtc}
\item	A NODE (ES or IS) IS ATTACHED TO A SUBNETWORK AT A
    \begin{nrtc}
    \item	SUBNETWORK POINT OF ATTACHMENT (SNPA)
    \end{nrtc}

\item	NEED A WAY TO MAP BETWEEN A NETWORK ADDRESS
	AND ITS CORRESPONDING SNPA

\item	ES-IS PROTOCOL IS USED ON CL-MODE NETWORKS

\item	STATIC TABLES (STILL) USED ON CO-MODE NETOWRKS
    \begin{nrtc}
    \item	CONS ES-IS IS (A BIT) IMMATURE

    \item	AND REQUIRES REAL CONS~---~NOT X.25(80)
    \end{nrtc}
\end{nrtc}
\end{bwslide}


\begin{bwslide}
\ctitle	{ES-IS (IN A NUTSHELL)}

\begin{nrtc}
\item	(CLNS) ES-IS PREFERS USE OF DL-MULTICAST
    \begin{nrtc}
    \item	ES-HELLOs

    \item	IS-HELLOs
    \end{nrtc}

\item	(CONS) ES-IS PREFERS USE OF A
    \begin{nrtc}
    \item	SUBNETWORK ADDRESS RESOLUTION ENTITY (SNARE)
    \end{nrtc}

\item	WHEN ES SENDS DATA TO SUB-OPTIMAL IS
    \begin{nrtc}
    \item	IS SENDS REDIRECT
    \end{nrtc}

\item	FOR CONS, THIS OCCURS ONLY DURING CONNECTION-ESTABLISHMENT
    \begin{nrtc}
    \item	SNARE REFUSES CALL AND

    \item	RETURNS ADDRESS OF BETTER FIRST-HOP
    \end{nrtc}
\end{nrtc}
\end{bwslide}


\begin{bwslide}
\part*	{TRANSPORT PROTOCOLS}\bf

\begin{nrtc}
\item	AVAILABLE NETWORK SERVICE DETERMINES CHOICE OF TRANSPORT PROTOCOL

\item	OSI PROVIDES 5 TRANSPORT PROTOCOLS, TP0--TP4
    \begin{nrtc}
    \item	CLASSES 0--3 WORKS WITH A CO-MODE NETWORK SERVICE

    \item	CLASS 4 WORKS WITH BOTH CO/CL-MODE NETWORK SERVICES
    \end{nrtc}
\end{nrtc}
\end{bwslide}


\begin{bwslide}
\ctitle	{NETWORK CLASSES}

\begin{nrtc}
\item	``A'' --- LOW LOSS, ERRORS SIGNALLED

\item	``B'' --- ERRORS SIGNALLED

\item	``C'' --- ERRORS NOT SIGNALLED
    \begin{nrtc}
    \item	LOSS

    \item	DUPLICATION

    \item	RE-ORDERING

    \item	CORRUPTION
    \end{nrtc}
    OF DATA
\end{nrtc}
\end{bwslide}


\begin{bwslide}
\ctitle	{PROTOCOLS USING\\ CO-MODE NETWORK SERVICE}

\begin{nrtc}
\item	TP0: SIMPLE CLASS
    \begin{nrtc}
    \item	NOTHING MORE THAN TRANSPORT ADDRESSING AND SEGMENTATION

    \item	``A'' NETWORKS
    \end{nrtc}

\item	TP1: BASIC ERROR RECOVERY CLASS
    \begin{nrtc}
    \item	RECOVER FROM NETWORK RESETS\\ (MAY INVOLVE RE-ROUTING)

    \item	``B'' NETWORKS
    \end{nrtc}
\end{nrtc}
\end{bwslide}


\begin{bwslide}
\ctitle	{PROTOCOLS USING\\ CO-MODE NETWORK SERVICE (cont.)}

\begin{nrtc}
\item	TP2: MULTIPLEXING CLASS
    \begin{nrtc}
    \item	MULTIPLEX OVER A SINGLE NETWORK CONNECTION

    \item	OPTIONAL FLOW CONTROL	

    \item	``A'' NETWORKS
    \end{nrtc}

\item	TP3: ERROR RECOVERY AND MULTIPLEXING CLASS
    \begin{nrtc}
    \item	ALL OF THE ABOVE

    \item	``B'' NETWORKS
    \end{nrtc}
\end{nrtc}
\end{bwslide}


\begin{bwslide}
\ctitle	{PROTOCOLS WHICH CAN USE\\ CL-MODE NETWORK SERVICE}

\begin{nrtc}
\item	TP4: ERROR DETECTION AND RECOVERY CLASS
    \begin{nrtc}
    \item	RELIABILITY THROUGH RETRANSMISSION

    \item	``C'' NETWORKS
    \end{nrtc}
\end{nrtc}
\end{bwslide}


\begin{bwslide}
\part*	{APPLICATION USE OF END-TO-END SERVICES}\bf

\begin{nrtc}
\item	APPLICATION IDENTIFIES APPLICATION ENTITY WHICH PROVIDES
	DESIRED SERVICE
    \begin{nrtc}
    \item	e.g., AN FTAM APPLICATION IDENTIFIES A FILESTORE SERVICE
		PROVIDED BY A PARTICULAR APPLICATION ENTITY    
    \end{nrtc}

\item	THE APPLICATION ENTITY IS IDENTIFIED BY ITS DISTINGUISHED NAME IN
	THE OSI DIRECTORY
\end{nrtc}
\end{bwslide}


\begin{bwslide}
\ctitle	{STEP 1:\\ MAP DISTINGUISHED NAME\\ TO PRESENTATION ADDRESS}

\begin{nrtc}
\item	ESTABLISH ASSOCIATION TO DIRECTORY SERVICE AGENT (DSA)
	USING DIRECTORY ACCESS PROTOCOL (DAP)

\item	RETRIEVE THE \verb"presentationAddress" ATTRIBUTE FROM
	THE OBJECT WITH THE GIVEN DISTINGUISHED NAME
\end{nrtc}

\begin{quote}\small\begin{verbatim}
presentation-selector:    null
session-selector:         null
transport-selector:       0101
network-addresses:        3623421920030013
                          4700040008000141524010100000
                          47000580fffc000000000100020123456789ab00
                          490059080020004053fe00
\end{verbatim}\end{quote}
\end{bwslide}


\begin{bwslide}
\ctitle	{STEP 2:\\ DETERMINE USE OF NETWORK ADDRESSES}

\begin{nrtc}
\item	PRESENTATION ADDRESS IS GIVEN TO THE ASSOCIATION CONTROL SERVICE
	ELEMENT (ACSE), WHICH ESTABLISHES THE ASSOCIATION

\item	ACSE PASSES THE ADDRESS TO THE PRESENTATION SERVICE,
	WHICH USES THE PRESENTATION SELECTOR

\item	THE REMAINDER IS GIVEN TO THE SESSION SERVICE,
	WHICH USES THE SESSION SELECTOR

\item	THE REMAINDER IS GIVEN TO THE TRANSPORT SERVICE
\end{nrtc}
\end{bwslide}


\begin{bwslide}
\ctitle	{STEP 2 (cont.)}

\begin{nrtc}
\item	TRANSPORT SERVICE LOOKS AT EACH NETWORK ADDRESS AND MUST DECIDE
    \begin{nrtc}
    \item	WHICH MODE NETWORK SERVICE WILL BE USED FOR THIS ADDRESS
    \end{nrtc}

\item	TRANSPORT SERVICE SELECTS A TRANSPORT PROTOCOL BASED ON THE
	DERIVED NETWORK SERVICE AND THE COMMUNICATIONS QUALITY OF SERVICE (QOS)
	DESIRED BY THE APPLICATION

\item	THIS COMBINATION
    \begin{nrtc}
    \item	 (NETWORK SERVICE+TRANSPORT PROTOCOL)
    \end{nrtc}
	IS TERMED A
    \begin{nrtc}
    \item	TRANSPORT SERVICE STACK (TS-STACK)
    \end{nrtc}
\end{nrtc}
\end{bwslide}


\begin{bwslide}
\ctitle	{STILL MORE ON\\ STEP 2}

\begin{nrtc}
\item	IN MANY ENVIRONMENTS ONLY A SINGLE MODE OF NETWORK SERVICE AND A
	SINGLE TRANSPORT PROTOCOL ARE AVAILABLE 

\item	THIS IMPLIES THAT ONLY A SUBSET (OR PERHAPS NONE) OF THE
	NETWORK ADDRESSES WILL BE USABLE AT THE ORIGINATING END-SYSTEM
\end{nrtc}
\end{bwslide}


\begin{bwslide}
\ctitle	{STEP 3:\\ ORDER NETWORK ADDRESSES}

\begin{nrtc}
\item	THE NETWORK ADDRESSES ARE THEN ORDERED BY PREFERENCE

\item	PREFERENCE IS BASED BOTH ON COMMUNICATIONS-QOS AND ``CLOSENESS''
	OF NETWORK ADDRESSES

\item	FOR EXAMPLE:
    \begin{nrtc}
    \item	TWO NETWORK ADDRESSES, EACH IMPLYING A CO-MODE NETWORK
		SERVICE, MIGHT BE PRESENT

    \item	ONE OF THE NETWORK ADDRESS MIGHT BELONG TO A PRIVATE
		NETWORK, WHILST THE OTHER BELONGS TO A PDN

    \item	THE TRANSPORT SERVICE MIGHT PREFER THE PRIVATE NETWORK,
		FOR COST REASONS
    \end{nrtc}
\end{nrtc}
\end{bwslide}


\begin{bwslide}
\ctitle	{STEP 4:\\ ATTEMPT CONNECTIONS}

\begin{nrtc}
\item	FOR EACH NETWORK ADDRESS:
    \begin{nrtc}
    \item	THE APPROPRIATE TRANSPORT PROTOCOL ENGINE IS STARTED,
		AND THE NETWORK SERVICE INVOKED

    \item	ONCE A TRANSPORT CONNECTION IS ESTABLISHED,
		THE REMAINDER OF THE NETWORK ADDRESSES ARE IGNORED
    \end{nrtc}
\end{nrtc}
\end{bwslide}


\begin{bwslide}
\part*	{EMULATION OF OSI END-TO-END SERVICES}\bf

\begin{nrtc}
\item	IS IT POSSIBLE TO PROVIDE OSI APPLICATIONS IN NON-OSI NETWORKS?

\item	A SOLUTION IS OFFERED BY LAYERING
    \begin{nrtc}
    \item	THE OSI TRANSPORT \underline{SERVICE} IS VERY SIMPLE
    \end{nrtc}

\item	CAN WE BUILD TS-STACKS USING NON-OSI PROTOCOLS?
\end{nrtc}
\end{bwslide}


\begin{bwslide}
\ctitle	{SERVICE EMULATOR AT TRANSPORT}

\vskip.5in
\diagram[p]{figureE-13}
\end{bwslide}


\begin{bwslide}
\ctitle	{APPROACH:\\ TRANSPORT SERVICE\\ CONVERGENCE PROTOCOL}

\begin{nrtc}
\item	USE THE CONNECTION-ORIENTED TRANSPORT SERVICE PROVIDED BY
	THE NON-OSI PROTOCOL SUITE

\item	DEFINE A ``TSCP'' WHICH SMOOTHS OVER THE DIFFERENCES IN THE SERVICES
	OFFERED
    \begin{nrtc}
    \item	IN PRACTICE, THESE ARE QUITE SMALL
    \end{nrtc}

\item	FOR EXAMPLE, THE RFC1006 METHOD DEFINES A TSCP FOR TCP/IP NETWORKS
\end{nrtc}
\end{bwslide}


\begin{bwslide}
\ctitle	{OSI TRANSPORT SERVICES\\ ON TOP OF THE TCP}

\vskip.25in
\diagram[p]{figureE-14}
\end{bwslide}


\begin{bwslide}
\part	{ACHIEVING CONNECTIVITY}\bf

\begin{nrtc}
\item	THE REAL WORLD OF OSI

\item	INTERIM USE OF NETWORK ADDRESSES

\item	TRANSPORT BRIDGING
\end{nrtc}
\end{bwslide}


\begin{bwslide}
\ctitle	{NOW THE HARD PART}

\begin{nrtc}
\item	A LOT OF FLEXIBILITY IS AVAILABLE

\item	BUT PRACTICALLY, CAN THIS BE MADE TO WORK?
\end{nrtc}
\end{bwslide}


\begin{bwslide}
\part*	{THE REAL WORLD OF OSI}\bf

\begin{nrtc}
\item	THE ``REAL WORLD'' DEPENDS ENTIRELY WHERE YOU LIVE

\item	A COMMUNITY IS A COLLECTION OF END-SYSTEMS SHARING COMPATIBLE
	TS-STACKS AND CONNECTED TOGETHER

\item	WHAT KIND OF OSI COMMUNITIES EXIST TODAY?
\end{nrtc}
\end{bwslide}


\begin{bwslide}
\ctitle	{COMMUNITY 1:\\ INTERNATIONAL X.25}

\begin{nrtc}
\item	X.121 FORMAT ADDRESSES ARE USED

\item	NETWORK PROTOCOL IS X.25(80) WHICH DOES NOT PROVIDE TRUE
	OSI NETWORK SERVICE
    \begin{nrtc}
    \item	EVENTUALLY UPGRADING TO X.25(84)
    \end{nrtc}

\item	TP0 IS FAVORED TRANSPORT PROTOCOL

\item	TS-STACKS:
\end{nrtc}

\diagram[p]{figureE-7}
\end{bwslide}


\begin{bwslide}
\ctitle	{COMMUNITY 2:\\ PRIVATE X.25}

\begin{nrtc}
\item	SIMILAR TO INTERNATIONAL X.25 COMMUNITY,
	BUT OWNED BY A PARTICULAR ENTERPRISE
    \begin{nrtc}
    \item	e.g., THE U.K.~JOINT ACADEMIC NETWORK (JANET)    
    \end{nrtc}

\item	ADDRESSES ARE X.121-BASED, BUT ARE PRIVATELY ALLOCATED
    \begin{nrtc}
    \item	THUS THE X.121 NETWORK ADDRESS FORMAT CAN'T BE USED
    \end{nrtc}

\item	TS-STACKS:
\end{nrtc}

\diagram[p]{figureE-7}
\end{bwslide}


\begin{bwslide}
\ctitle	{COMMUNITY 3:\\ VARIANT U.S. USE OF X.25}

\begin{nrtc}
\item	X.25 TREATED AS A SUBNETWORK PROTOCOL

\item	CL-MODE NETWORK SERVICE RUN OVER THIS

\item	TS-STACKS:
\end{nrtc}

\diagram[p]{figureE-9}
\end{bwslide}


\begin{bwslide}
\ctitle	{COMMUNITY 4:\\ CONS-BASED LANS}

\begin{nrtc}
\item	CO-MODE NETWORK SERVICE OFFERRED OVER 8802 SUBNETWORK

\item	COMMONLY TERMED ``X.25 OVER ETHERNET'' (LLC2)

\item	TS-STACKS:
\end{nrtc}

\diagram[p]{figureE-10}
\end{bwslide}


\begin{bwslide}
\ctitle	{COMMUNITY 5:\\ CLNS-BASED LANS}

\begin{nrtc}
\item	CL-MODE NETWORK SERVICE OFFERRED OVER 8802 SUBNETWORK

\item	COMMONLY TERMED ``MAP/TOP LANs'' (LLC1)

\item	TS-STACKS:
\end{nrtc}

\diagram[p]{figureE-11}
\end{bwslide}


\begin{bwslide}
\ctitle	{COMMUNITY 6:\\ TCP/IP-BASED INTERNET USING RFC1006}

\begin{nrtc}
\item	RFC1006 DEFINES A MAPPING FROM THE OSI TRANSPORT SERVICE ONTO THE TCP

\item	PROBLEM: WHAT FORMAT TO USE NETWORK ADDRESS?

\item	TS-STACKS:
\end{nrtc}

\diagram[p]{figureE-12}
\end{bwslide}


\begin{bwslide}
\ctitle	{COMMUNITY 7:\\ TCP/IP-BASED LAN USING RFC1006}

\begin{nrtc}
\item	SIMILAR TO INTERNET COMMUNITY,
	BUT ON AN ISOLATED TCP/IP LAN
    \begin{nrtc}
    \item	e.g., A CAMPUS NETWORK RUNNING TCP/IP LOCALLY AND HAVING A
		CONNECTION TO A PDN
    \end{nrtc}

\item	TS-STACKS:
\end{nrtc}

\diagram[p]{figureE-12}
\end{bwslide}


\begin{bwslide}
\ctitle	{COMMUNITY INTEROPERATION}

\begin{nrtc}
\item	SO, THERE ARE (AT LEAST) SEVEN DIFFERENT COMMUNITIES IN THE OSI WORLD

\item	IDEALLY WOULD LIKE THIS INTERWORKING MATRIX:
\end{nrtc}

\diagram[p]{figureE-15}
\end{bwslide}


\begin{bwslide}
\ctitle	{COMMUNITY INTEROPERATION (cont.)}

\begin{nrtc}
\item	COMMUNITY 7 IS ISOLATED BY LACK OF CONNECTIVITY
\end{nrtc}

\diagram[p]{figureE-16}
\end{bwslide}


\begin{bwslide}
\ctitle	{COMMUNITY INTEROPERATION (cont.)}

\begin{nrtc}
\item	PRIVATE X.25 AND RFC1006--BASED COMMUNITIES NEED DIFFERENT ADDRESS
	SPACE
\end{nrtc}

\diagram[p]{figureE-17}
\end{bwslide}


\begin{bwslide}
\ctitle	{REAL WORLD CONNECTIVITY MATRIX}

\begin{nrtc}
\item	IN PRACTICE, CONS-BASED LANS DON'T INTEROPERATE WITH CONS-BASED WANS 
    \begin{nrtc}
    \item	ROUTING OF CONS-BASED SUBNETWORKS ISN'T WIDELY IMPLEMENTED
		OUTSIDE OF X.75
    \end{nrtc}
\end{nrtc}

\diagram[p]{figureE-18}
\end{bwslide}


\begin{bwslide}
\ctitle	{COMMUNITY INTEROPERATION (cont.)}

\begin{nrtc}
\item	CLNS-BASED AND CONS-BASED TS-STACKS DON'T ALWAYS INTEROPERATE
    \begin{nrtc}
    \item	IT IS NOT ENOUGH TO START WITH TP4 AND DOWN-NEGOTIATE
    \end{nrtc}
\end{nrtc}

\diagram[p]{figureE-19}
\end{bwslide}


\begin{bwslide}
\ctitle	{THE MYTH OF TRANSPORT NEGOTIATION}

\begin{nrtc}
\item	IF INITIATOR SELECTS TP4, MUST ALSO DECIDE CONS/CLNS
    \begin{nrtc}
    \item	IF CLNS IS USED, THEN MUST STAY WITH TP4

    \item	IF CLNS ISN'T USED, THEN CAN'T TALK TO CLNS-BASED LAN    
    \end{nrtc}
\end{nrtc}
\end{bwslide}


\begin{bwslide}
\part*	{INTERIM USE OF NETWORK ADDRESSES}\bf

\begin{nrtc}
\item	WANT TO ACCOMODATE ALL OSI COMMUNITIES IN OSI DIRECTORY

\item	PROBLEM: ALL ADDRESSES MUST CONFORM TO DIRECTORY DEFINED SYNTAX

\item	PROBLEM: ALL ADDRESSES MUST BE GLOBALLY UNIQUE YET LOCALLY
	INTERPRETABLE

\item	SOLUTION: KILLE'S INTERIM APPROACH
\end{nrtc}
\end{bwslide}


\begin{bwslide}
\ctitle	{CONFORMANCE TO\\ DIRECTORY DEFINED SYNTAX}

\begin{nrtc}
\item	A PROBLEM FOR THE PRIVATE X.25 AND RFC1006--BASED COMMUNITIES

\item	TAKE A PART OF THE SPACE ASSIGNED TO TELEX ADDRESSES
    \begin{nrtc}
    \item	NO ONE WILL USE TELEX AFI FOR NETWORK ADDRESSES
    \end{nrtc}

\item	SUB-DIVIDE THIS ADDRESS SPACE FOR EACH COMMUNITY, e.g.,
    \begin{nrtc}
    \item	AFI = 54

    \item	IDI = 00728722
    \end{nrtc}
\end{nrtc}

\diagram[p]{figureE-8}
\end{bwslide}


\begin{bwslide}
\ctitle	{INTERPRETATION OF ADDRESSES}

\begin{nrtc}
\item	FROM EACH NETWORK ADDRESS
    \begin{nrtc}
    \item	COMMUNITY (TS-STACK, IDENTITY OF NETWORK) MUST BE DEDUCIBLE

    \item	NETWORK-SPECIFIC INFORMATION (i.e., SNPA) MUST BE DEDUCIBLE
    \end{nrtc}
\end{nrtc}
\end{bwslide}


\begin{bwslide}
\part*	{TRANSPORT BRIDGING}\bf

\begin{nrtc}
\item	PROBLEM: SUPPOSE THE ORIGINATING END-SYSTEM DETERMINES THAT
	IT IS IN A DIFFERENT COMMUNITY THAN THE DESTINATION END-SYSTEM

\item	FROM A PURIST PERSPECTIVE:
    \begin{nrtc}
    \item	INTEROPERATION CAN NOT OCCUR!
    \end{nrtc}

\item	FROM A PRAGMATIC PERSPECTIVE:
    \begin{nrtc}
    \item	IGNORE THE CURSED MODEL AND BUILD A LEVEL-4 RELAY
    \end{nrtc}

\item	FOR AN AMUSING READ, TRY ISO/DTR10172
\end{nrtc}
\end{bwslide}


\begin{bwslide}
\ctitle	{TS-BRIDGES}

\begin{nrtc}
\item	ALTHOUGH MANY DIFFERENT TS-STACKS EXIST,
	THEY ALL PROVIDE THE SAME TRANSPORT SERVICE

\item	SO, IT IS STRAIGHT-FORWARD TO BUILD A BOX THAT:
    \begin{nrtc}
    \item	KNOWS NOTHING ABOUT TRANSPORT PROTOCOLS, BUT

    \item	KNOWS HOW TO USE THE RELATIVELY SIMPLE OSI TRANSPORT SERVICE
    \end{nrtc}

\item	A TS-BRIDGE ``COPIES'' SERVICE PRIMITIVES FROM ONE TS-STACK TO THE
	OTHER, e.g.,
    \begin{nrtc}
    \item	UPON RECEIVING A T-CONNECT.INDICATION PRIMITIVE FROM ONE
		TS-STACK,

    \item	IT ISSUES A T-CONNECT.REQUEST PRIMITIVE TO THE OTHER TS-STACK
    \end{nrtc}
\end{nrtc}
\end{bwslide}


\begin{bwslide}
\ctitle	{TS-BRIDGES (cont.)}

\vskip.5in
\diagram[p]{figureE-1}
\end{bwslide}


\begin{bwslide}
\ctitle	{THE PROBLEMS OF LEVEL-4 RELAYS}

\begin{nrtc}
\item	THE TS-BRIDGE MAINTAINS STATE AS TO THE EXISTING CONNECTIONS

\item	EACH TS-STACK PROVIDES A CHECKSUM,
	NEITHER OF WHICH IS REALLY END-TO-END
    \begin{nrtc}
    \item	(CHECKSUM AT EITHER TRANSPORT OR NETWORK SERVICE)
    \end{nrtc}

\item	DEFEATS TRANSPORT-LEVEL ENCRYPTION

\item	INTRODUCES ADDITIONAL VARIABILITY IN ROUND-TRIP TIMES

\item	END-TO-END ADDRESSES ARE NOT CARRIED TRANSPARENTLY
\end{nrtc}
\end{bwslide}


\begin{bwslide}
\ctitle	{AND WHAT ABOUT?}

\begin{nrtc}
\item	ACCOUNTING

\item	ACCESS CONTROL

\item	LOAD BALANCING

\item	CONCATENATION OF TS-BRIDGES

\item	and so on$\ldots$
\end{nrtc}
\end{bwslide}


\begin{bwslide}
\ctitle	{USE OF THE TS-BRIDGE}

\begin{nrtc}
\item	MUST NOW SUBTLY MODIFY TRANSPORT SERVICE OF ORIGINATING END-SYSTEM
    \begin{nrtc}
    \item	STEP 2: DETERMINE USE OF NETWORK ADDRESSES
    \end{nrtc}

\item	IF NO USABLE NETWORK ADDRESSES ARE AVAILABLE

\item	THEN SELECT A TS-BRIDGE WHICH SERVICES THE OSI COMMUNITY FOR ONE OF
	THE NETWORK ADDRESSES
    \begin{nrtc}
    \item	 RECALL, OSI COMMUNITY EQUALS TS-STACK PLUS CONNECTIVITY
    \end{nrtc}
\end{nrtc}
\end{bwslide}


\begin{bwslide}
\ctitle	{USE OF THE TS-BRIDGE (cont.)}

\begin{nrtc}
\item	ENCODE THE NETWORK ADDRESS AND TRANSPORT SELECTOR AS AN OCTET STRING,
    \begin{nrtc}
    \item	(USING SOME ENCODING FORMAT)
    \end{nrtc}
	CALL THIS THE NEW TRANSPORT SELECTOR

\item	USE THE NETWORK ADDRESS OF THE TS-BRIDGE FOR THE REMAINING STEPS

\item	WHEN TS-BRIDGE RECEIVES CONNECTION,
	IT SIMPLY DECODES TRANSPORT SELECTOR TO FIND ADDRESS OF
	DESTINATION END-SYSTEM
\end{nrtc}
\end{bwslide}


\begin{bwslide}
\ctitle	{TS-BRIDGE ADDRESSING\\ (CO/CL WORKSHOP FORMAT)}

\vskip.5in
\diagram[p]{figureE-20}
\end{bwslide}


\begin{bwslide}
\part	{COMPARISON TO TCP/IP}\bf

\begin{nrtc}
\item	NETWORK SERVICE

\item	TRANSPORT SERVICE
\end{nrtc}
\end{bwslide}


\begin{bwslide}
\ctitle	{COMPARISONS}

\begin{nrtc}
\item	ALL COMPARISONS ARE PARTISAN IN NATURE

\item	HOWEVER, WITHOUT BIAS OR LOSS OF GENERALITY,\\ I CAN HONESTLY STATE:
    \begin{nrtc}
    \item	THE OSI LOWER-LAYERS ARE CURRENTLY INCOHERENT
    \end{nrtc}
\end{nrtc}
\end{bwslide}


\begin{bwslide}
\part*	{NETWORK SERVICE}\bf

\begin{nrtc}
\item	THE INTERNET PROTOCOL (IP) PROVIDES A CL-NETWORK SERVICE
    \begin{nrtc}
    \item	SIMILAR TO CLNP, BUT MORE EFFICIENT
    \end{nrtc}

\item	THE LEAST COMMON DENOMINATOR, USABLE OVER BOTH WANs AND LANs
    \begin{nrtc}
    \item	BEST EFFORT DELIVERY

    \item	RELIABILITY RESPONSIBILITY OF TRANSPORT SERVICE
    \end{nrtc}

\item	32~BITS v. 20~OCTETS
\end{nrtc}
\end{bwslide}


\begin{bwslide}
\ctitle	{ARE TWO OSI NETWORK SERVICES\\ ONE TOO MANY?}

\begin{nrtc}
\item	IN A WORD: YES

\item	OSI COMMUNITIES ARE SEPERATED BY TS-STACKS AND CONNECTIVITY

\item	CONNECTIVITY ISN'T A TECHNICAL ISSUE

\item	BUT, TS-STACKS ARE, SO:
    \begin{nrtc}
    \item	IF THERE WAS A SINGLE NETWORK SERVICE,
		THEN THERE COULD BE A SINGLE TRANSPORT PROTOCOL
    \end{nrtc}
\end{nrtc}
\end{bwslide}


\begin{bwslide}
\part*	{TRANSPORT SERVICE}\bf

\begin{nrtc}
\item	THE TRANSMISSION CONTROL PROTOCOL (TCP) PROVIDES A CO-TRANSPORT
	SERVICE

\item	SEVERAL DIFFERENCES FROM THE OSI TRANSPORT SERVICE
    \begin{nrtc}
    \item	TCP IS STREAM-ORIENTED

    \item	TCP USES GRACEFUL RELEASE

    \item	TCP USES URGENT DATA
    \end{nrtc}

\item	THESE ARE DIFFERENCES, NOT PROS AND CONS
\end{nrtc}
\end{bwslide}


\begin{bwslide}
\ctitle	{COMPARISON OF PROTOCOLS}

\begin{nrtc}
\item	REALLY CAN COMPARE ONLY THE TCP AND TP4

\item	TP4 PACKET ORIENTATION PREVENTS USE OF SOPHISTICATED CONGESTION
	COLLAPSE ALGORITHMS

\item	TP4 PACKET ORIENTATION HELPS BUFFER MANAGEMENT,
	POSSIBLY MORE EFFICIENT

\item	TP4 RETRANSMISSION ALGORITHMS ARE SIMPLISTIC

\item	TP4 END-TO-END CHECKSUM IS INAPPROPRIATE
\end{nrtc}
\end{bwslide}


\begin{bwslide}
\ctitle	{TRANSPORT BRIDGING}

\begin{nrtc}
\item	UNNECESSARY IN TCP/IP WORLD
    \begin{nrtc}
    \item	COMMON NETWORK PROTOCOL

    \item	UNIFORM NETWORK ADDRESS FORMAT
	\begin{nrtc}
	\item	(ALL IN 32--BITS!)
	\end{nrtc}
    \end{nrtc}
\end{nrtc}
\end{bwslide}


\begin{bwslide}
\part*	{CONCLUSIONS}\bf

\begin{nrtc}
\item	DEPRESSING
    \begin{nrtc}
    \item	WORLD-WIDE OSI ``CAN'T HAPPEN''

    \item	THIS WILL CURTAIL USE OF WONDERFUL APPLICATIONS
    \end{nrtc}

\item	FORTUNATELY, CLOSED COMMUNITIES WILL BE RELATIVELY IMMUNE
\end{nrtc}
\end{bwslide}

