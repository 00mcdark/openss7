% -*- LaTeX -*-		(really SLiTeX)

\documentstyle[blackandwhite,landscape,oval,pagenumbers,small]{NRslides}

\font\xx=cmbx10
\font\yy=cmbx7

\raggedright

\input trademark
\let\tradeNAMfont=\relax
\let\tradeORGfont=\relax

\begin{document}

\title	{ISODE 6.0:\\ OPENLY AVAILABLE OSI}
\author	{Marshall T.~Rose\\ NYSERNet, Inc.}
\date	{December 5, 1989}
\maketitlepage


\begin{bwslide}
\part*	{AGENDA}\bf

\begin{description}
\item[PART I:]		WHAT'S THERE NOW

\item[PART II:]		WHAT'S IN PROGRESS
\end{description}
\end{bwslide}


\begin{bwslide}
\ctitle	{WHAT IS ISODE?}

\begin{nrtc}
\item	THE ISO DEVELOPMENT ENVIRONMENT

\item	AN OPENLY AVAILABLE IMPLEMENATION OF THE UPPER LAYERS OF OSI?

\item	A BASIS FOR THE TRANSITION TO OSI?

\item	A PLAYGROUND FOR ``THE PIED-PIPER OF OSI''?
\end{nrtc}
\end{bwslide}


\begin{bwslide}
\part	{WHAT'S THERE NOW}\bf

\begin{nrtc}
\item	TRANSPORT SWITCH

\item	APPLICATION ARCHITECTURE

\item	APPLICATION COOKBOOK

\item	APPLICATIONS
\end{nrtc}
\end{bwslide}


\begin{bwslide}
\part*	{CURRENT DISTRIBUTION}\bf

\begin{nrtc}
\item	STATUS: OPENLY AVAILABLE UNDER AN IMPLICIT ``HOLD HARMLESS'' CLAUSE

\item	CURRENT RELEASE: 6.0
    \begin{nrtc}
    \item	AVAILABLE JANUARY 24, 1990
    \end{nrtc}

\item	SOURCE SIZE: \~{}250K LINES OF C AND ASN.1
\end{nrtc}
\end{bwslide}


\begin{bwslide}
\ctitle	{CURRENT DISTRIBUTION (cont.)}

\begin{nrtc}
\item	DISTRIBUTION EITHER VIA POSTAL MAIL, FTP, or FTAM
    \begin{nrtc}
    \item	SOURCE: \~{}14MB

    \item	DOC: 5~VOLUME USER'S MANUAL (\~{}1000~PAGES)

    \item	DISTRIBUTION SITES: US, UK, NL, AND AU

    \item	PRICE: \~{}375~US DOLLARS
    \end{nrtc}
\end{nrtc}
\end{bwslide}


\begin{bwslide}
\ctitle	{NORTH AMERICA DISTRIBUTION}\small

\[\begin{tabular}{rl}
Postal address:&UNIVERSITY OF PENNSYLVANIA\\
&		DEPARTMENT OF COMPUTER AND INFORMATION SCIENCE\\
&		MOORE SCHOOL\\
&		ATTN: DAVID J. FARBER (ISODE DISTRIBUTION)\\
&		200 SOUTH 33RD STREET\\
&		PHILADELPHIA, PA 19104-6314\\
&		USA\\[0.2in]
Telephone:&	+1--215--898--8560\\[0.2in]
Price:&		US\$375.00 (CHECKS ONLY)
\end{tabular}\]
\end{bwslide}


\begin{bwslide}
\ctitle	{LANGUAGES AND OPERATING SYSTEMS}

\begin{nrtc}
\item	CODED ENTIRELY IN C FOR \unix/
    \begin{nrtc}
    \item	REQUIRES NO KERNEL MODIFICATIONS    
    \end{nrtc}

\item	KNOWN PORTS FOR BERKELEY \unix/ (4.x):
    \begin{nrtc}
    \item	VAXen, SUNs, Pyramids, RTs, etc.
    \end{nrtc}

\item	KNOWN PORTS FOR AT\&T \unix/ (SVR2 and SVR3):
    \begin{nrtc}
    \item	SGI, 3Bs, 386s, RT (AIX)
    \end{nrtc}
\end{nrtc}
\end{bwslide}


\begin{bwslide}
\part*	{THE TRANSPORT SWITCH}\bf

\begin{nrtc}
\item	DECIDES WHICH TS-STACK TO USE FOR A CONNECTION

\item	FOR RFC1006: TCP

\item	FOR TP0: X.25

\item	FOR TP4: 4.4BSD OSI, SunLink OSI

\item	EXPERIENCE SHOWS IT IS FAIRLY EASY TO ADD A NEW TS-STACK TO THE SWITCH
\end{nrtc}
\end{bwslide}


\begin{bwslide}
\ctitle	{TRANSPORT-SERVICE BRIDGES}

\vskip.5in
\diagram[p]{figure1}
\end{bwslide}


\begin{bwslide}
\ctitle	{TCP vs. X.25 CONNECTIVITY}

\vskip.5in
\diagram[p]{figure2}
\end{bwslide}


\begin{bwslide}
\ctitle	{CONS vs. CLNS CONNECTIVITY}

\vskip.5in
\diagram[p]{figure3}
\end{bwslide}


\begin{bwslide}
\ctitle	{POSSIBLE XNS SCENARIO}

\vskip.5in
\diagram[p]{figure4}
\end{bwslide}


\begin{bwslide}
\part*	{APPLICATION ARCHITECTURE}\bf

\begin{nrtc}
\item	A (NEARLY) COMPLETE IMPLEMENTATION OF THE UPPER LAYERS

\item	``IS'' LEVEL SINCE 5.0 RELEASE

\item	ALIGNED WITH VARIOUS NATIONAL OSI PROFILS
    \begin{nrtc}
    \item	(INFORMALLY, OF COURSE!)
    \end{nrtc}
\end{nrtc}
\end{bwslide}


\begin{bwslide}
\ctitle	{THE APPLICATION ENVIRONMENT}

\vskip.5in
\diagram[p]{figure9}
\end{bwslide}


\begin{bwslide}
\ctitle	{AN ALTERNATE ENVIRONMENT:\\ LIGHTWEIGHT PRESENTATION}

\vskip.5in
\diagram[p]{figure5}
\end{bwslide}


\begin{bwslide}
\ctitle	{AN ALTERNATE ENVIRONMENT:\\ MHS ARCHITECTURE (c.~1984)}

\vskip.5in
\diagram[p]{figure10}
\end{bwslide}


\begin{bwslide}
\part*	{THE APPLICATIONS COOKBOOK}\bf

\begin{nrtc}
\item	TOOLS TO FACILITATE DEVELOPMENT OF APPLICATIONS ARE CRITICAL

\item	IDEA IS TO DEVELOP TOOLS TO AUTOMATE USE OF OSI REMOTE OPERATIONS
	SERVICE AS A GENERAL REMOTE PROCEDURE CALL FACILITY

\item	FOR MORE DETAILS:
\begin{quote}
BUILDING DISTRIBUTED APPLICATIONS IN AN OSI FRAMEWORK
\end{quote}
APPEARING IN ConneXions, MARCH, 1988
\end{nrtc}
\end{bwslide}


\begin{bwslide}
\ctitle	{REMOTE OPERATIONS SERVICE (ROS)}

\begin{nrtc}
\item	STANDARDIZED MECHANISM FOR SPECIFYING TRANSACTIONS

\item	EMPLOYS POWER OF ASN.1

\item	USED IN MANY INTERESTING OSI APPLICATIONS
    \begin{nrtc}
    \item	MESSAGE HANDLING SYSTEMS

    \item	DIRECTORY SERVICES

    \item	NETWORK MANAGEMENT

    \item	REMOTE DATABASE ACCESS
    \end{nrtc}

\item	CURRENTLY CONNECTION-ORIENTED, BUT CONNECTIONLESS-MODE IS UNDER STUDY
\end{nrtc}
\end{bwslide}


\begin{bwslide}
\ctitle	{GENERAL ORGANIZATION}

\begin{nrtc}
\item	AT COMPILE-TIME:
    \begin{nrtc}
    \item	USE RO-SPECIFICATION TO GENERATE SUPPORT FACILITIES
    \end{nrtc}

\item	AT RUN-TIME:
    \begin{nrtc}
    \item	USE DIRECTORY SERVICES TO LOCATE/REGISTER NETWORK SERVICES
		(NEARLY THERE!)

    \item	USE ASSOCIATION CONTROL TO BIND/UNBIND APPLICATIONS

    \item	USE REMOTE OPERATIONS TO INVOKE TRANSACTIONS
    \end{nrtc}
\end{nrtc}
\end{bwslide}


\begin{bwslide}
\ctitle	{STATIC (COMPILE-TIME) ORGANIZATION}

\vskip.15in
\diagram[p]{figure11}
\end{bwslide}


\begin{bwslide}
\ctitle	{DYNAMIC (RUN-TIME) ORGANIZATION}

\vskip.15in
\diagram[p]{figure12}
\end{bwslide}


\begin{bwslide}
\ctitle	{PERFORMANCE}

\begin{nrtc}
\item	USE OF ASN.1 TOOLS LEADS TO LARGE PROCESSES:
    \begin{nrtc}
    \item	MINIMUM 300KB, DSA: 500KB, FTAM: 800KB

    \item	INITIALIZATION SPEED SUFFERS MOST OWING TO PAGING IN
    \end{nrtc}

\item	A LOT OF TIME SPENT AVOIDING BYTE COPYING

\item	SPEEDS SLOWER (BUT COMPARITIVE) TO INTERNET APPLICATIONS WHEN USING
	TCP-BASED TRANSPORT

\item	X.25 TOO SLOW FOR COMPARISON

\item	USE OF LIGHTWEIGHT PRESENTATION SOMETIMES RESULTS IN PROCESSES HALF
	AS LARGE AND TWICE AS FAST
\end{nrtc}
\end{bwslide}


\begin{bwslide}
\part*	{APPLICATIONS}\bf

\begin{nrtc}
\item	EVERYTHING BUT MHS (sigh!)
\end{nrtc}
\end{bwslide}


\begin{bwslide}
\ctitle	{CURRENT APPLICATIONS}

\begin{nrtc}
\item	FILE TRANSFER, ACCESS AND MANAGEMENT (FTAM)

\item	FTAM-FTP GATEWAY

\item	DIRECTORY SERVICES (X.500)
    \begin{nrtc}
    \item	AND WHITE PAGES ABSTRACTION
    \end{nrtc}

\item	VIRTUAL TERMINAL

\item	ISODE MISCELLANY SERVICE
    \begin{nrtc}
    \item	e.g., FINGER, QUOTE-OF-THE-DAY, etc.
    \end{nrtc}

\item	PLUS NUMEROUS ``DEMO'' PROGRAMS
    \begin{nrtc}
    \item	e.g., IMAGE SERVICE, PASSWORD LOOKUP, IDIST, etc.
    \end{nrtc}
\end{nrtc}
\end{bwslide}


\begin{bwslide}
\ctitle	{NETWORK MANAGEMENT}

\begin{nrtc}
\item	FOR BERKELEY UNIX SYSTEMS:
    \begin{nrtc}
    \item	SNMP!
    \end{nrtc}

\item	WHY?
    \begin{nrtc}
    \item	IT WORKS

    \item	CONTINUED SURVIVAL OF THE INTERNET HINGES ON ALL NODES BEING
		NETWORK MANAGEABLE
    \end{nrtc}

\item	NOT A COMPLETE PACKAGE
    \begin{nrtc}
    \item	AN AGENT WITH A MINIMAL INITIATOR

    \item	(NO NOC)
    \end{nrtc}
\end{nrtc}
\end{bwslide}


\begin{bwslide}
\ctitle	{DIRECTORY SERVICES}

\begin{nrtc}
\item	THE UCL DIRECTORY, QUIPU, HAS NOW COMPLETED ITS MAJOR DEVELOPMENT

\item	SEVERAL INTERESTING FEATURES:
    \begin{nrtc}
    \item	MEMORY, RATHER THAN DISK-BASED, ACCESS

    \item	INTERNAL SCHEDULING FOR MULTIPLE ACCESS

    \item	FLEXIBLE SEARCHING (SOUNDEX)

    \item	NOT STANDARDIZED:
	\begin{nrtc}
	\item	ACCESS CONTROL

	\item	REPLICATION

	\item	CACHING
	\end{nrtc}
    \end{nrtc}

\item	ALREADY INTEROPERABILITY TESTED AGAINST TWO (EMBRYONIC) IMPLEMENATIONS
\end{nrtc}
\end{bwslide}

\begin{bwslide}
\ctitle	{WHITE PAGES PILOT}

\begin{nrtc}
\item	A ``GRASS ROOTS'' EFFORT TO PROVIDE A WHITE PAGES SERVICE

\item	SEMI-PUBLIC INFORMATION (TELEPHONE/MAIL BOOKS)

\item	MOST SITES RESPONSIBLE FOR MAINTAINING INFORMATION

\item	ACCESS VIA NETWORK AND DIALUP LOGINS

\item	CRT-BASED INTERFACE WITH X-WINDOWS SUPPORT
\end{nrtc}
\end{bwslide}


\begin{bwslide}
\ctitle	{VIRTUAL TERMINAL}

\begin{nrtc}
\item	MITRE HAS DEVELOPED A DIS VT IMPLEMENTATION

\item	ROUGHLY EQUIVALENT TO BSD TELNET IN TERMS OF FUNCTIONALITY
    \begin{nrtc}
    \item	(BASIC CLASS, TELNET PROFILE)
    \end{nrtc}

\item	INTEROPERABILITY TESTED AGAINST THE BRIDGE/3COM VT

\item	MITRE IS WORKING ON A FORMS CLASS IMPLEMENTATION

\item	ULTIMATELY, MUST BE UPGRADED TO IS IMPLEMENTATION
\end{nrtc}
\end{bwslide}


\begin{bwslide}
\ctitle	{OTHER APPLICATIONS\\ (NOT A PART OF ISODE)}

\begin{nrtc}
\item	NETWORK MANAGEMENT

\item	ODA/ODIF

\item	MOBILE X.400 PILOT
    \begin{nrtc}
    \item	MS-DOS CLIENT SIDE ONLY PORT DONE BY HP
    \end{nrtc}
\end{nrtc}
\end{bwslide}


\begin{bwslide}
\part	{WHAT'S IN PROGRESS}\bf

\begin{nrtc}
\item	MESSAGE HANDLING SYSTEMS

\item	INTEROPERABILITY TESTING

\item	OSI-POSIX PROJECT
\end{nrtc}
\end{bwslide}


\begin{bwslide}
\part*	{MESSAGE HANDLING SYSTEMS}\bf

\begin{nrtc}
\item	UCL AND UNott ARE DEVELOPING AN X.400 TRANSPORT SYSTEM (PP)

\item	USE EXPERIENCE GAINED FROM NUMEROUS SOPHISTICATED TEXT-BASED MESSAGE
	TRANSFER SYSTEMS

\item	OWES MANY OF ITS DESIGN IDEAS TO THE UNIVERSITY OF DELAWARE MESSAGE
	SYSTEM, MMDF

\item	WILL UTILIZE DIRECTORY SERVICES
\end{nrtc}
\end{bwslide}


\begin{bwslide}
\ctitle	{TOP-LEVEL ARCHITECTURE}

\vskip.15in
\diagram[p]{figure6}
\end{bwslide}


\begin{bwslide}
\ctitle	{INTERESTING FEATURES}

\begin{nrtc}
\item	SUPPORT FOR A WIDE RANGE OF ENCODED INFORMATION TYPES 
    \begin{nrtc}
    \item	AND REFORMATTING BETWEEN THEM
    \end{nrtc}

\item	SUPPORT FOR DIFFERENT MESSAGE TRANSPORT PROTOCOLS
    \begin{nrtc}
    \item	AND CONVERSION BETWEEN THEM
    \end{nrtc}
    e.g., INCLUDES RFC987 (X.400 TO 821/822)

\item	ROBUSTNESS FOR USE IN LARGE SCALE SERVICE ENVIRONMENTS
\end{nrtc}
\end{bwslide}


\begin{bwslide}
\ctitle	{MAJOR GOALS}

\begin{nrtc}
\item	FULL X.400(84/88) SUPPORT, EXCEPT FOR X.400(88) SECURITY SERVICES

\item	PROVIDES A ``CLEAN'' INTERFACE FOR MESSAGE SUBMISSION AND DELIVERY
    \begin{nrtc}
    \item	TO SUPPORT A WIDE RANGE OF USER AGENTS,

    \item	AND APPLICATIONS OTHER THAN INTERPERSONAL MESSAGING
    \end{nrtc}

\item	QUEUE MANAGEMENT DONE VIA A ROS-BASED PROTOCOL
    \begin{nrtc}
    \item	SOPHISTICATED SCHEDULING OF MESSAGE DELIVERY

    \item	LOCAL AND REMOTE MONITORING FOR MANAGERS AND USERS

    \item	ROBUSTNESS REQUIRED TO SUPPORT HIGH LEVELS OF TRAFFIC

    \item	SUPPORT FOR ADMINISTRATIVE POLICIES ON SUBMISSION
    \end{nrtc}

\item	LIST EXPLODER AND LIST MANAGMENT    
\end{nrtc}
\end{bwslide}


\begin{bwslide}
\ctitle	{OTHER THINGS}

\begin{nrtc}
\item	TWO USER INTERFACES PLANNED
    \begin{nrtc}
    \item	MH INTERFACE

    \item	WINDOW-BASED INTERFACE
    \end{nrtc}

\item	INTEGRATION OF FAX PLANNED

\item	BETA TESTING STARTED JANUARY, 1990
\end{nrtc}
\end{bwslide}


\begin{bwslide}
\part*	{INTEROPERABILITY TESTING}\bf

\begin{nrtc}
\item	THERE IS NO SUBSTITUTE FOR PAIRWISE INTEROPERABILITY TESTING

\item	LET GROUPS SUCH AS OSInet, EUROSInet, etc., CONNECT TO HOSTS
	RUNNING X.25 AND ISODE

\item	TESTING IS UNATTENDED UNLESS PROBLEMS ARE WITH THE ISODE (gasp!)
\end{nrtc}
\end{bwslide}


\begin{bwslide}
\part*	{OSI-POSIX PROJECT}\bf

\begin{nrtc}
\item	GOAL: ACCELLERATE THE UBIQUITY OF OSI

\item	APPROACH: OPENLY AVAILABLE, COMPLETE OSI IMPLEMENTATION FOR NEXT MAJOR
	RELEASE OF BERKELEY \unix/

\item	FOR MORE DETAILS:
\begin{quote}
OSI PROTOCOLS WITHIN AN OPENLY AVAILABLE, POSIX-CONFORMANT, BERKELEY UNIX
ENVIRONMENT
\end{quote}
APPEARING IN ConneXions, OCTOBER, 1988
\end{nrtc}
\end{bwslide}


\begin{bwslide}
\diagram[p]{figure13}
\end{bwslide}


\begin{bwslide}
\diagram[p]{figure14}
\end{bwslide}


\end{document}
