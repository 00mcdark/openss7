%\begin{bwslide}
%\part	{PRESENTATION SERVICES}\bf
%\end{bwslide}


\begin{bwslide}
\part	{PRESENTATION SERVICES}\bf

%\vskip.5in
\diagram[p]{figureA-7}
\end{bwslide}


\begin{bwslide}
\ctitle {PRESENTATION LAYER\\PURPOSE}

\begin{nrtc}
\item	COMBINES
	\begin{nrtc}
	\item	STRUCTURING ASPECTS OF ABSTRACT SYNTAX
	\item	COMMUNICATIONS CONTROL OF SESSION SERVICES
	\end{nrtc}
\item	EXCHANGES DATA STRUCTURES OVER SESSION DIALOGUES
\end{nrtc}
\end{bwslide}


\begin{bwslide}
\ctitle {THE PRESENTATION LAYER}

\vskip.5in
{\tiny\diagram[p]{figureA-23}}
\end{bwslide}


\begin{bwslide}
\ctitle	{PRESENTATION}

\begin{nrtc}
\item	QUITE SIMPLE COMPARED TO SESSION SERVICE
	\begin{nrtc}
	\item	{\em ALTHOUGH IT MAY NOT APPEAR SO}
	\end{nrtc}
%\item	PRIMITIVES PREFIXED WITH {\em P---\ldots}
\end{nrtc}
\end{bwslide}


%\begin{bwslide}
%\ctitle {USERS}
%
%\begin{nrtc}
%\item	{\em PS--users} ARE APPLICATION SERVICE ELEMENTS (ASEs)
%\item	PRESENTATION VIEW:  SINGLE ENTITY IS A {\em PS--user}
%\item	APPLICATION VIEW:  MANY ASEs COOPERATE TO APPEAR AS ONE {\em PS-user}
%\end{nrtc}
%\end{bwslide}


%\begin{bwslide}
%\ctitle {\em PS-users}
%
%\begin{nrtc}
%\item	CONNECTION ESTABLISHMENT
%	\begin{nrtc}
%	\item	CALLING {\em PS-user}
%	\item	CALLED {\em PS-user}
%	\end{nrtc}
%\item	ACTION INITIATION
%	\begin{nrtc}
%	\item	REQUESTING {\em PS-user}  
%	\item	ACCEPTING {\em PS-user}
%	\end{nrtc}
%\item	CONNECTIONS ARE SYMMETRIC
%\end{nrtc}
%\end{bwslide}


%\begin{bwslide}
%\ctitle {ADDRESSING}
%
%\begin{nrtc}
%\item	USER ADDRESSED AT PRESENTATION SERVICE ACCESS POINT (PSAP)
%	\begin{nrtc}
%	\item	PRESENTATION SELECTOR
%	\item	SESSION ADDRESS
%	\end{nrtc}
%\item	SELECTOR IS A STRING OF OCTETS
%\item	SESSION ADDRESS IDENTIFIES A GIVEN PRESENTATION ENTITY
%\item	PRESENTATION SELECTOR IDENTIFIES A SPECIFIC USER OF THAT ENTITY
%\end{nrtc}
%\end{bwslide}


%\begin{bwslide}
%\ctitle {PRESENTATION PROVIDES}
%
%\begin{nrtc}
%\item	{\em PRESENTATION} CONTEXT MANAGEMENT
%\item	{\em SESSION FUNCTIONALITY}
%\end{nrtc}
%\end{bwslide}


\begin{bwslide}
\ctitle {PRESENTATION CONTEXTS}

\begin{nrtc}
\item	A CONTEXT IS A PAIRING OF:
	\begin{nrtc}
	\item	A SET OF DATA STRUCTURE DEFINITIONS (ABSTRACT SYNTAX)
	\item	RULES FOR ENCODING THOSE DATA STRUCTURES ON THE NETWORK 
		(TRANSFER SYNTAX)
	\end{nrtc}
\item	PRESENTATION CONTEXT IDENTIFIER (PCI)
	\begin{nrtc}
	\item	{\small (DIFFERENT THAN PROTOCOL CONTROL INFORMATION --- PCI)}
	\end{nrtc}
\item	USER DATA GIVEN TO PRESENTATION IS MARKED WITH A CONTEXT
	\begin{nrtc}
	\item	FOR A GIVEN ASE, e.g., ASSOCIATION CONTROL
	\end{nrtc}
\item	CONTEXTS MAY BE DEFINED OR REMOVED FOLLOWING PRESENTATION CONNECTION
	ESTABLISHMENT
\item	EVENTS MAY ALTER DEFINED CONTEXT SET (e.g., RESYNCHRONIZATION)
\end{nrtc}
\end{bwslide}


\begin{bwslide}
\ctitle {TRANSFER SYNTAXS}

\begin{nrtc}
\item	PRESENTATION NEGOTIATES TRANSFER SYNTAX ASSOCIATED WITH EACH
	ABSTRACT SYNTAX
	\begin{nrtc}
	\item	THIS IS AN EASY JOB, \underline{TODAY}!
	\end{nrtc}
\item	THE {\em BER} IS THE ONLY TRANSFER SYNTAX DEFINED TODAY
\item	{\em ASN.1} IS THE ONLY ABSTRACT SYNTAX DEFINED TODAY
\item	APPLICATION MAY NEED TO ``HINT'' AT TRANSFER SYNTAX {\em IN THE FUTURE}
\end{nrtc}
\end{bwslide}


\begin{bwslide}
\ctitle {DEFINED CONTEXT SET (DCS)}

\begin{nrtc}
\item	SET OF CONTEXTS BEING USED ON CONNECTION
\item	INITIALLY SUPPLIED BY USER
\item	NEGOTIATED BY PRESENTATION
\end{nrtc}
\end{bwslide}


%\begin{note}
%provide negotiation scenario?
%\end{note}


\begin{bwslide}
\ctitle {CONTEXT MANAGEMENT}

\begin{nrtc}
\item \underline{OPTIONAL} FUNCTIONALITY TO CHANGE {\em DCS} DURING CONNECTION
\item NEGOTIATION PROCESS DURING CONNECTION SAME AS ORIGINAL, {\em IF AVAILABLE}
\end{nrtc}
\end{bwslide}


%\begin{bwslide}
%\ctitle {FUNCTIONAL UNITS}
%
%\begin{nrtc}
%\item	KERNEL
%	\begin{nrtc}
%	\item	MANDATORY
%	\end{nrtc}
%\item	CONTEXT MANAGEMENT
%	\begin{nrtc}
%	\item	OPTIONAL
%	\end{nrtc}
%\item	CONTEXT RESTORATION
%	\begin{nrtc}
%	\item	OPTIONAL
%	\item	{\em ONLY POSSIBLE IF CONTEXT MANAGEMENT AVAILABLE}
%	\end{nrtc}
%\end{nrtc}
%\end{bwslide}


%\begin{bwslide}
%\ctitle {PRESENTATION USER DATA}
%
%\begin{nrtc}
%\item	PRESENTATION SERVICE DATA UNIT (PSDU)
%\item	COMPOSED OF:
%	\begin{nrtc}
%	\item	PRESENTATION DATA VALUES
%	\end{nrtc}
%\item	{\em EXCEPTION}: EXPEDITED DATA IS ALWAYS SENT IN THE DEFAULT CONTEXT
%\end{nrtc}
%\end{bwslide}


%\begin{bwslide}
%\ctitle {PRESENTATION DATA VALUES}
%
%\begin{nrtc}
%\item	EACH HAS AN ASSOCIATED PCI
%\item	IMPLEMENTOR MAY CHOOSE LOCAL REPRESENTATION
%\item	ORDER OF VALUES IS PRESERVED IN PSDUs
%\item	MAY CONTAIN DATA OF ANOTHER ABSTRACT SYNTAX
%	\begin{nrtc}
%	\item	BUT ONLY ONE TRANSFER SYNTAX USED
%	\end{nrtc}
%\end{nrtc}
%\end{bwslide}


%\begin{bwslide}
%\ctitle {?}
%
%THE PRESENTATION SERVICE HAS NO EXPLICIT KNOWLEDGE OF ASN.1 OR THE BER.
%BUT THE PROVIDER USES BOTH CONCEPTS INTERNALLY.
%
%PROVIDE SOME SUPPORT FOR THIS IDEA.
%\end{bwslide}


%\begin{bwslide}
%\ctitle {ACCESS TO SESSION SERVICE}
%
%\begin{nrtc}
%\item	PRESENTATION HAS A STRAIGHT FORWARD MAPPING TO THE SESSION SERVICE
%\item	SESSION PRIMITIVES HAVE CORRESPONDING PRESENTATION PRIMITIVES
%\item	PRESENTATION REQUIREMENTS IMPLY REQUIREMENTS ON SESSION FUNCTIONAL UNITS
%\item	SOME PRESENTATION CONCEPTS ARE ``PASS--THROUGH'' TO SESSION
%	\begin{nrtc}
%	\item	e.g., QUALITY OF SERVICE
%	\end{nrtc}
%\end{nrtc}
%\end{bwslide}


\begin{bwslide}
\ctitle {IMPLEMENTATION CONSIDERATIONS}

\begin{nrtc}
\item	STRONG MAPPING TO SESSION SIMPLIFIES PRESENTATION IMPLEMENTATION
\item	PRESENTATION STATE MACHINE OFTEN PROVIDED IMPLICITLY BY SESSION
\item	EXTRA WORK OVER SESSION:
	\begin{nrtc}
	\item	TRANSFORMING PSDUs INTO SSDUs \& BACK
	\end{nrtc}
\end{nrtc}
\end{bwslide}


\begin{bwslide}
\ctitle {IMPLEMENTATION CONSIDERATIONS (cont.)}

\begin{nrtc}
\item	PRESENTATION PRIMITIVES MAP EASILY TO PROCEDURE CALLS/RETURNS
\item	USING SESSION TO ENFORCE MOST RULES MOST PRESENTATION PRIMITIVES 
	ARE SIMPLE
	\begin{nrtc}
	\item	CHECK ANY SPECIFIC PRESENTATION INTERFACE POLICIES
	\item	MAP PSDU TO SSDU
	\item	INVOKE CORRESPONDING SESSION SERVICE
	\end{nrtc}
\end{nrtc}
\end{bwslide}


%\begin{bwslide}
%\ctitle {ENCODINGS}
%
%\begin{nrtc}
%\item	ASN.1 USED TO DEFINE PPDUs
%\item	THE BER IS USED TO ``SERIALIZE'' PPDUs TO \& FROM SSDUs
%\end{nrtc}
%\end{bwslide}


\begin{bwslide}
\ctitle {LIGHTWEIGHT PRESENTATION PROTOCOL}

\vskip.5in
\diagram[p]{figureA-25}
\end{bwslide}


\begin{bwslide}
\ctitle {LIGHTWEIGHT PRESENTATION (cont.)}

\begin{nrtc}
\item	INTERNET COMMUNITY CREATION
\item	STREAM--LINED PRESENTATION \& SESSION SERVICES
\item	AVOIDS CUMBERSOME OVERHEAD OF COMPLETE SERVICES
\item	INTENDED FOR USE BY SPECIFIC ASEs
\item	WILL \underline{NOT} SATISFY ALL ASE REQUIREMENTS
\end{nrtc}
\end{bwslide}


\begin{bwslide}
\ctitle {LIGHTWEIGHT PRESENTATION (cont.)}

\begin{nrtc}
\item	SUPPLIES FUNCTIONALITY REQUIRED BY:
	\begin{nrtc}
	\item	ASSOCIATION CONTROL
	\item	REMOTE OPERATIONS
	\end{nrtc}
\item	SATISFIES EXISTING OSI APPLICATIONS
	\begin{nrtc}
	\item	NETWORK MANAGEMENT ({\em ORIGINAL MOTIVATION})
	\item	DIRECTORY SERVICES
	\end{nrtc}
\item	SHOWN TO BE UP TO TWICE AS PERFORMANT
	\begin{nrtc}
	\item	(PLUS MUCH SMALLER APPLICATION PROGRAMS)
	\end{nrtc}
\end{nrtc}
\end{bwslide}


\begin{bwslide}
\ctitle {LIGHTWEIGHT PRESENTATION (cont.)}

\begin{nrtc}
\item	PROVIDES IDENTICAL APPLICATION PROGRAM INTERFACE
	\begin{nrtc}
	\item	IMPLIES APPLICATIONS CAN BE EASILY PORTED TO ``REAL'' OSI
	\end{nrtc}
\item	\underline{PROTOCOL} DOES \underline{NOT} INTEROPERATE 
		WITH ``REAL'' OSI APPLICATIONS
	\begin{nrtc}
	\item	WOULD REQUIRE AN APPLICATION LAYER GATEWAY TO AN OSI STACK
	\end{nrtc}
\end{nrtc}
\end{bwslide}


%\begin{bwslide}
%\ctitle {USER CONSIDERATIONS}
%
%\begin{nrtc}
%\item	{\em P-CONNECT}
%	\begin{nrtc}
%	\item	PASSING A DATA VALUE WITH AN ABSTRACT SYNTAX UNKNOWN TO 
%		THE RESPONDER MAY RESULT IN A REJECT
%	\end{nrtc}
%\end{nrtc}
%\end{bwslide}


\begin{bwslide}
\ctitle	{GENERAL POINTS}

PRESENTATION, SYNTAXES, AND CONTEXTS WHILE NOT VERY COMPLEX CAN PRESENT A GREAT 
DEAL OF CONFUSION.  SORTING OUT THE TERMS AND THEIR SIGNIFICANCE WILL MAKE ANY
OSI APPLICATION DEVELOPERS LIFE MUCH EASIER.  UNDERSTANDING THESE CONCEPTS IS
VITAL TO SUCCESSFUL INTEROPERABILITY TESTING.
\end{bwslide}


%\begin{bwslide}
%\ctitle {PURPOSE}
%
%\begin{nrtc}
%\item	SEMANTIC EQUIVALENCE
%	\begin{nrtc}
%	\item	SENDING ENTITY
%	\item	RECEIVING ENTITY
%	\end{nrtc}
%\end{nrtc}
%\end{bwslide}


%\begin{bwslide}
%\ctitle{PROBLEMS}
%
%\begin{nrtc}
%\item	REPRESENTATION OF PRIMATIVE VALUES
%	\begin{nrtc}
%	\item	INTEGERS
%		\begin{nrtc}
%		\item	1's OR 2's COMPLEMENT
%		\item	BIT ORDERING
%		\item	RANGE RESTRICTIONS
%		\end{nrtc}
%	\item	ENUMERATIONS, SCALARS
%	\item	CHARACTER SETS
%	\item	FLOATING POINT!
%	\end{nrtc}
%\item	REPRESENTATION OF CONSTRUCTORS
%	\begin{nrtc}
%	\item	ARRAY, RECORDS, FILES, ...
%		\begin{nrtc}
%		\item	ORDERING, PACKING, ALIGNMENT
%		\end{nrtc}
%	\item	VARIANTS, CHOICES, etc.
%	\end{nrtc}
%\end{nrtc}
%\end{bwslide}


%\begin{bwslide}
%\ctitle {REFERENCES}
%
%\begin{description}
%\item[ISO/IEC 8322:]	Basic Connection Oriented Presentation 
%							Service Definition
%\item[ISO/IEC 8323:]	Basic Connection Oriented Presentation 
%							Protocol Specification
%\end{description}
%\end{bwslide}
