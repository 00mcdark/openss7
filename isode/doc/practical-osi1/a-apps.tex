\begin{bwslide}
\part	{APPLICATIONS}
\end{bwslide}


\begin{bwslide}
\ctitle {APPLICATIONS}

\begin{nrtc}
\item	AS AN EXAMPLE OF HOW ALL THE PARTS FIT TOGETHER,
	WE WILL TAKE A QUICK LOOK AT:
	\begin{nrtc}
	\item	FILE TRANSFER, ACCESS AND MANAGEMENT (FTAM)
	\item	MESSAGE HANDLING SYSTEMS (X.400)
	\item	DIRECTORY SERVICES (X.500)
	\item	NETWORK MANAGEMENT (CMISE)
	\item	VIRTUAL TERMINAL (VT)
	\end{nrtc}
\item	AND THEN LOOK AT HOW APPLICATIONS ARE BUILT
\end{nrtc}
\end{bwslide}


%\begin{bwslide}
%\ctitle {APPLICATIONS (cont.)}
%
%\begin{nrtc}
%\item	FINALLY, WE WILL CONSIDER APPLICATION LAYER REQUIREMENTS FOR DEFINING
%	A NEW SERVICE
%\end{nrtc}
%\end{bwslide}


\begin{bwslide}
\ctitle {FILE TRANSFER, ACCESS AND MANAGEMENT (FTAM)}

\vskip.5in
\diagram[p]{figureA-19}
\end{bwslide}


\begin{bwslide}
\ctitle {THE OSI FILE SERVICE --- FTAM}

\begin{nrtc}
\item	NOT ``JUST'' FILE TRANSFER
\item	BUILDING BLOCK FOR OSI:
	\begin{nrtc}
	\item	FILESTORE TO FILESTORE TRANSFER
	\item	WORKSTATION FILE RETRIEVAL
	\item	DISKLESS WORKSTATION PROTOCOL
	\item	SPECIAL APPLICATIONS (e.g., PRINTING, SPOOLING)
	\item	REMOTE FILE ACCESS
	\end{nrtc}
\end{nrtc}
\end{bwslide}


\begin{bwslide}
\ctitle {FTAM PHILOSOPHY}

\begin{nrtc}
\item	AS WITH ALL ``OPEN SYSTEM'' SERVICES
	\begin{nrtc}
	\item	DESCRIBES A CONCEPTUAL MODEL OF THE VIRTUAL SERVICE
	\item	SPECIFIES THE SERVICE AND THE PROTOCOL
	\item	INDEPENDENT OF ACTUAL LOCAL SYSTEMS
		\begin{nrtc}
		\item	PROGRAMMING LANGUAGE BINDINGS ARE NOT SPECIFIED
		\end{nrtc}
	\end{nrtc}
\item	THE FUNDAMENTAL ABSTRACTION:
	\begin{nrtc}
	\item	THE VIRTUAL FILESTORE
	\end{nrtc}
\item	PROVIDES A CONCEPTUAL MODEL OF A FILE SERVICE ON A LOCAL SYSTEM (LOCALSTORE)
\item	DIFFICULT TASK --- EXISTING FILE SERVICE ARE QUITE DIFFERENT
\item	POTENTIALLY VERY REWARDING!
\end{nrtc}
\end{bwslide}


%\begin{bwslide}
%\ctitle {RELATIONSHIP OF THE VIRTUAL FILESTORE AND LOCALSTORE}
%
%diagram
%\end{bwslide}


\begin{bwslide}
\ctitle {FTAM --- ELEMENTS}

\begin{nrtc}
\item	A (VIRTUAL) FILESTORE IS A COLLECTION OF FILES
\item	A FILENAME IDENTIFIES EXACTLY ONE FILE IN THE FILESTORE
\item	THERE IS NO EXPLICIT RELATIONSHIP BETWEEN DIFFERENT FILES IN THE FILESTORE
	\begin{nrtc}
	\item	i.e., NO DIRECTORY STRUCTURE
	\end{nrtc}
\item	FILES HAVE
	\begin{nrtc}
	\item	ATTRIBUTES (e.g., OWNERSHIP INFORMATION)
	\item	CONTENTS (e.g., RANDOM--ACCESS RECORDS)
	\end{nrtc}
\end{nrtc}
\end{bwslide}


%\begin{bwslide}
%\ctitle {FTAM ELEMENTS --- ATTRIBUTES}
%
%\begin{nrtc}
%\item	TWO KINDS OF ATTRIBUTES ARE DEFINED
%\item	FILE ATTRIBUTES, WHICH EXIST ON A PER-FILE BASIS
%	\begin{nrtc}
%	\item	SIMULTANEOUS CLIENTS OF THE FILESTORE SEE THE SAME INFORMATION
%	\item	e.g., THE NAME OF THE FILE
%	\end{nrtc}
%\item	ACTIVITY ATTRIBUTES, WHICH EXIST ON A PER-CLIENT BASIS
%	\begin{nrtc}
%	\item	INTERACTIONS BY A CLIENT ARE NOT DIRECTLY VISIBLE TO OTHER CLIENTS
%	\item	e.g., HOW THE FILE IS BEING TRAVERSED
%	\end{nrtc}
%\item	THE CLIENT INTERACTS ON AT MOST ONE FILE
%	\begin{nrtc}
%	\item	THE ``SELECTED'' FILE
%	\end{nrtc}
%\end{nrtc}
%\end{bwslide}


%\begin{bwslide}
%\ctitle {FTAM ELEMENTS --- CONTENTS}
%
%\begin{nrtc}
%\item	TYPICALLY, FILES ARE DEFINED IN TERMS OF A ``DOCUMENT TYPE''
%\item	STATIC CHARACTERISTICS
%	\begin{nrtc}
%	\item	THE COMPOSITION OF THE FILE IN TERMS OF FILE ACCESS DATA UNITS (FADUs)
%		\begin{nrtc}
%		\item	e.g., A SEQUENTIAL COLLECTION OF RECORDS	
%		\end{nrtc}
%	\item	THE STRUCTURE OF EACH DATA UNIT (DUs)
%		\begin{nrtc}
%		\item	e.g., EACH RECORD CONTAINS A PERSONNEL RECORD
%		\end{nrtc}
%	\end{nrtc}
%\item	DYNAMIC CHARACTERISTICS
%	\begin{nrtc}
%	\item	HOW DATA UNITS ARE ENCODED ON THE NETWORK
%	\item	HOW DATA UNITS ARE REFERENCED (e.g., CURRENT POSITION)
%	\end{nrtc}
%\end{nrtc}
%\end{bwslide}


%\begin{bwslide}
%\ctitle {FTAM --- FILE ATTRIBUTES}
%
%\begin{nrtc}
%\item	4 GROUPS OF FILE ATTRIBUTES
%\item	KERNEL GROUP (REQUIRED)
%	\begin{nrtc}
%	\item	NECESSARY FOR FILE SELECTION AND BASIC FILE TRANSFER
%	\end{nrtc}
%\item	STORAGE GROUP (OPTIONAL)
%	\begin{nrtc}
%	\item	DESCRIBES THE STORAGE CHARACTERISTICS FOR THE FILE
%	\end{nrtc}
%\item	SECURITY GROUP (OPTIONAL)
%	\begin{nrtc}
%	\item	DESCRIBES THE ACCESS CONTROL MECHANISMS FOR THE FILE
%	\end{nrtc}
%\item	PRIVATE GROUP (OPTIONAL)
%	\begin{nrtc}
%	\item	A MECHANISM TO CAPTURE NON-STANDARD (PROPRIETARY) MECHANISMS
%		THAT CAN'T BE OTHERWISE REPRESENTED
%	\end{nrtc}
%\end{nrtc}
%\end{bwslide}


%\begin{bwslide}
%\ctitle {FTAM --- KERNEL GROUP}
%
%\begin{nrtc}
%\item	FILENAME:  A SEQUENCE OF STRINGS
%	\begin{nrtc}
%	\item	MAPPING TO THE LOCALSTORE NAMING CONVENTIONS IS A 
%		``LOCAL IMPLEMENTATION CHOICE''
%	\end{nrtc}
%\item	CONTENTS TYPE:  STRUCTURING INFORMATION
%\item	PERMITTED ACTIONS
%	\begin{nrtc}
%	\item	DESCRIBES THE TYPES OF DATA ACCESS THAT CAN BE PERFORMED ON THE FILE
%	\item	HOW DATA UNITS MAY BE ACCESSED (READ, WRITE, EXTEND, etc.)
%	\item	HOW THE FILE MAY BE TRAVERSED (MOVING FROM ONE DATA UNIT TO ANOTHER)
%	\end{nrtc}
%\end{nrtc}
%\end{bwslide}


%\begin{bwslide}
%\ctitle {FTAM --- STORAGE GROUP}
%
%\begin{nrtc}
%\item	STORAGE ACCOUNT:  A STRING
%	\begin{nrtc}
%	\item	ENTITY ACCRUING FILE STORAGE CHARGES
%	\end{nrtc}
%\item	IDENTITY OF USER AND THE DATE/TIME OF
%	\begin{nrtc}
%	\item	FILE CREATION
%	\item	LAST READ \& LAST MODIFICATION OF FILE CONTENTS
%	\item	LAST MODIFICATION OF FILE ATTRIBUTES
%	\end{nrtc}
%\item	FILE AVAILABILITY
%	\begin{nrtc}
%	\item	IMMEDIATE (FILE IS ``ON-LINE'')
%	\item	DEFERRED (ACCESS TO FILE MAY ENCOUNTER DELAY, e.g., AWAITING
%		ARCHIVE RETRIEVAL
%	\end{nrtc}
%\end{nrtc}
%\end{bwslide}


%\begin{bwslide}
%\ctitle {FTAM --- STORAGE GROUP (cont.)}
%
%\begin{nrtc}
%\item	FILESIZE (IN OCTETS)
%	\begin{nrtc}
%	\item	AN ESTIMATE OF THE TOTAL SIZE OF THE FILE'S CONTENTS
%	\end{nrtc}
%\item	FUTURE FILESIZE (IN OCTETS)
%	\begin{nrtc}
%	\item	A SOFT LIMIT ON THE TOTAL SIZE OF THE FILE'S CONTENTS
%	\end{nrtc}
%\end{nrtc}
%\end{bwslide}


%\begin{bwslide}
%\ctitle {FTAM --- SECURITY GROUP}
%
%\begin{nrtc}
%\item	ACCESS CONTROL (AN ACCESS CONTROL LIST) FOR EACH ELEMENT OF THE LIST:
%	\begin{nrtc}
%	\item	FILE ACTION PERMITTED
%	\item	CONCURRENCY CONSTRAINTS
%	\item	ENTITY PERMITTED TO REQUEST ACTION (OPTIONAL)
%	\item	PASSWORDS REQUIRED TO VALIDATE EACH ACTION
%	\end{nrtc}
%\item	LEGAL QUALIFICATIONS
%	\begin{nrtc}
%	\item	DEFINED THE ``LEGAL STATUS'' OF THE FILE
%	\item	MEANT TO BE USED WITH A NATIONAL PRIVACY LEGISLATION
%	\end{nrtc}
%\end{nrtc}
%\end{bwslide}


%\begin{bwslide}
%\ctitle {FTAM --- PRIVATE GROUP}
%
%\begin{nrtc}
%\item	A ``CATCH--ALL''
%\item	USE IS STRONGLY DISCOURAGED
%\end{nrtc}
%\end{bwslide}


%\begin{bwslide}
%\ctitle {FTAM --- ACTIVITY ATTRIBUTES}
%
%\begin{nrtc}
%\item	ACTIVITY ATTRIBUTES ARE ALSO DEFINED IN TERMS OF GROUPS
%	\begin{nrtc}
%	\item	KERNEL, STORAGE, AND SECURITY (NO PRIVATE GROUP, OBVIOUSLY)
%	\end{nrtc}
%\item	THESE ARE USUALLY INITIALIZED WHEN A FILE IS EITHER
%	\begin{nrtc}
%	\item	SELECTED
%	\item	OPENED FOR TRANSFER/ACCESS
%	\end{nrtc}
%\end{nrtc}
%\end{bwslide}


%\begin{bwslide}
%\ctitle {FTAM --- DOCUMENT TYPES}
%
%\begin{nrtc}
%\item	STATIC CHARACTERISTICS
%	\begin{nrtc}
%	\item	THE FILE ACCESS STRUCTURE (CONSTRAINT SET NAME)
%	\item	THE PRESENTATION STRUCTURE (ABSTRACT SYNTAX NAME)
%	\end{nrtc}
%\item	DYNAMIC CHARACTERISTICS
%	\begin{nrtc}
%	\item	THE TRANSFER STRUCTURE (TRANSFER SYNTAX NAME)
%	\item	A IDENTIFICATION STRUCTURE (ACCESS CONTEXTS)
%	\end{nrtc}
%\item	``REGISTERED'' AND REFERENCED VIA A UNIQUE IDENTIFIER
%\end{nrtc}
%\end{bwslide}


%\begin{bwslide}
%\ctitle {FTAM --- SUMMARY}
%
%\begin{nrtc}
%\item	THE VIRTUAL FILESTORE IS THE OSI ABSTRACTION OF A LOCALSTORE
%\item	FILES CONTAIN ATTRIBUTES AND STRUCTURING INFORMATION IN ADDITION
%	TO ``TYPED'' DATA
%\item	FILES ARE DISTINGUISHED BY NAME
%\item	SOME ATTRIBUTES ARE DYNAMIC, ON A PER-CLIENT BASIS
%\item	STRUCTURE IS BASED ON A HIERARCHAL MODEL
%\item	DATA A STRUCTURE ARE SEPARATE AND DISTINCT
%\item	DOCUMENT TYPES PROVIDE AN ABBREVIATED METHOD FOR REFERRING TO THE
%	FILE STRUCTURE
%\end{nrtc}
%\end{bwslide}


%\begin{bwslide}
%\ctitle {FILE TRANSFER, ACCESS AND MANAGEMENT \\ REFERENCES}
%
%\begin{description}
%\item[ISO/IEC 8571:]	File Transfer, Access and Management (Parts 1---4)
%			\begin{description}
%			\item[Part 1:] General Information
%			\item[Part 2:] Virtual Filestore Definition
%			\item[Part 3:] The Filestore Definition
%			\item[Part 4:] File Protocol Specification
%			\end{description}
%\end{description}
%\end{bwslide}


\begin{bwslide}
\ctitle {MESSAGE HANDLING SYSTEMS (MHS)\\ X.400}

\vskip.5in
\diagram[p]{figureA-20}
\end{bwslide}


\begin{bwslide}
\ctitle {MESSAGE HANDLING SYSTEMS}

\begin{nrtc}
\item	ELECTRONIC MESSAGING SERVICE
\item	PURPOSE
	\begin{nrtc}
	\item	A GENERAL PURPOSE STORE--AND--FORWARD ENVIRONMENT
	\item	AN ENVIRONMENT WHICH SUPPORTS HUMAN TO HUMAN ELECTRONIC MESSAGING
	\item	A SERVICE WHICH CAN BE USED BY OTHER APPLICATIONS
	\item	TRANSPORT STRUCTURE MESSAGES
	\end{nrtc}
\end{nrtc}
\end{bwslide}


\begin{bwslide}
\ctitle {MHS}

\begin{nrtc}
\item	ALMOST CERTAINLY TO BE THE MOST ``POPULAR'' OSI APPLICATION
\item	TECHNICALLY A VERY COMPREHENSIVE \& LARGE STANDARD
\item	DEVELOPMENT OF MHS RESULTED IN MUCH OF THE OSI UPPER LAYER DEVELOPMENT
\item	THE MOST COMPLEX OSI APPLICATION DUE TO ITS SIZE
\end{nrtc}
\end{bwslide}


%\begin{bwslide}
%\ctitle {MHS VERSIONS}
%
%\begin{nrtc}
%\item	1984 --- FIRST PASS, PROVIDES A VERY EXTENSIVE FACILITY
%\item	1988 --- SECOND PASS, REFLECTS A TREMENDOUS AMOUNT OF EXPERIENCE
%		WITH THE 1984 WORK.  A SIGNIFICANTLY MORE ``MATURE'' STANDARD
%\end{nrtc}
%\end{bwslide}


\begin{bwslide}
\ctitle {MHS PIECES}

\vskip.5in
\diagram[p]{figureA-36}
\end{bwslide}


%\begin{bwslide}
%\ctitle {MESSAGE HANDLING SYSTEMS \\ REFERENCES}
%
%\begin{description}
%\item[ISO/IEC 10021:]	Message Handling Systems (Parts 1---7)
%			\begin{description}
%			\item[Part 1:] System and Service Overview
%			\item[Part 2:] Overall Architecture
%			\item[Part 3:] Abstract Service Definition Conventions
%			\item[Part 4:] Abstract Service Definition Procedures
%			\item[Part 5:] Message Store Abstract Service Definition
%			\item[Part 6:] Protocol SPecifications
%			\item[Part 7:] Interpersonal Messaging System
%			\end{description}
%\end{description}
%\end{bwslide}


\begin{bwslide}
\ctitle {DIRECTORY SERVICES (DS)\\ X.500}

\vskip.5in
\diagram[p]{figureA-17}
\end{bwslide}


\begin{bwslide}
\ctitle {DIRECTORY SERVICES}

\begin{nrtc}
\item	PURPOSE
	\begin{nrtc}
	\item	PROVIDE A MECHANISM TO DISTRIBUTE AND RETRIEVE INFORMATION
	\item	LET APPLICATIONS MAP FROM NAMES TO ADDRESSES {\em (RECALL
		DISCUSSION OF THE DSE)}
	\item	ALLOW HUMAN USERS TO ``INTUIT'' NAMES OF OBJECTS AND THEN
		RETRIEVE INFORMATION FROM THOSE OBJECTS
		\begin{nrtc}
		\item e.g., ELECTRONIC MAIL ADDRESSES FOR OTHER USERS
		\end{nrtc}
	\end{nrtc}
\end{nrtc}
\end{bwslide}


\begin{bwslide}
\ctitle {DIRECTORY SERVICES (cont.)}

\begin{nrtc}
\item	INFORMATION IS ARRANGED HIERARCHICALLY
\item	TOP LEVEL IS IMAGINED TO BE PRIMARILY COUNTRIES
\item	RULES ABOUT WHAT APPEARS BENEATH A GIVEN POINT
	ARE LEFT TO INDIVIDUAL ADMINISTRATORS
\end{nrtc}
\end{bwslide}


\begin{bwslide}
\ctitle {THE DIRECTORY INFORMATION TREE}

\vskip.5in
\diagram[p]{figureA-37}
\end{bwslide}


\begin{bwslide}
\ctitle {HOW THE DIRECTORY IS DISTRIBUTED}

\begin{nrtc}
\item	MANY SERVERS (DSAs) COOPERATE TO PROVIDE THE HIERARCHICAL DIRECTORY
\item	USERS (DUAs) COMMUNICATE WITH SERVERS
\item	SERVERS ALSO COMMUNICATE WITH OTHER SERVERS
\item	OPERATIONS MAY BE PASSED FROM ONE SERVER TO ANOTHER FOR THE USER {\em (CHAINING)}
\item	SERVERS MAY RETURN REFERRALS TO OTHER SERVERS TO THE USER
\end{nrtc}
\end{bwslide}


\begin{bwslide}
\ctitle {DIRECTORY OPERATION}

\vskip.5in
\diagram[p]{figureA-38}
\end{bwslide}


\begin{bwslide}
\ctitle {SECURITY AND THE DIRECTORY}

\begin{nrtc}
\item	OSI APPLICATIONS MAY USE THE DIRECTORY TO PROVIDE AUTHENTICATION INFORMATION
\item	USERS MAY STORE PUBLIC ENCRYPTION KEYS IN THE DIRECTORY
\item	OTHER USERS MAY READ THOSE KEYS TO VALIDATE DIGITAL SIGNATURES AND THE LIKE
\end{nrtc}
\begin{nrtc}
\item	SIMPLE AUTHENTICATION MAY JUST STORE PASSWORDS IN THE DIRECTORY
\item	OTHER USERS MAY ONLY BE ALLOWED TO \underline{COMPARE} A VALUE TO 
		SEE IF IT MATCHES	
\end{nrtc}
\end{bwslide}


%\begin{bwslide}
%\ctitle {DIRECTORY SERVICES \\ REFERENCES}
%
%\begin{description}
%\item[ISO/IEC 9594:]	The Directory (Parts 1-8)
%			\begin{description}
%			\item[Part 1:]	Overview of Concepts, Models and Services
%			\item[Part 2:]	Models					
%			\item[Part 3:]	Abstract Service Definition	
%			\item[Part 4:]	Procedures for Distributed Operation
%			\item[Part 5:]	Protocol Specifications		
%			\item[Part 6:]	Selected Attribute Types
%			\item[Part 7:]	Selected Object Classes
%			\item[Part 8:]	Authentication Framework
%			\end{description}
%\end{description}
%\end{bwslide}


\begin{bwslide}
\ctitle {NETWORK MANAGEMENT (CMISE)}

\vskip.5in
\diagram[p]{figureA-21}
\end{bwslide}


\begin{bwslide}
\ctitle {\small COMMON MANAGEMENT INFORMATION SERVICE ELEMENT}

\begin{nrtc}
\item	PURPOSE
	\begin{nrtc}
	\item	PROVIDE A COMMON PROTOCOL FOR MANAGING OSI AND NON-OSI NETWORKS
	\item	MANAGEMENT STANDARDS ALSO DEFINE MANAGED OBJECT IN THE
		MANAGEMENT INFORMATION BASE (MIB)
	\end{nrtc}
\end{nrtc}
\end{bwslide}


%\begin{bwslide}
%\ctitle {NETWORK MANAGEMENT \\ REFERENCES}
%
%\begin{description}
%\item[ISO/IEC 9595:]	Management Information Service Definition 
%			\begin{description}
%			\item[Part 1:] Overview
%			\item[Part 2:] Common Management Information Service
%			\end{description}
%\item[ISO/IEC 9596:]	Management Information Protocol Specification
%			\begin{description}
%			\item[Part 1:] Overview
%			\item[Part 2:] Common Management Information Protocol
%			\end{description}
%\end{description}
%\end{bwslide}


\begin{bwslide}
\ctitle {VIRTUAL TERMINAL (VT)}

\vskip.5in
\diagram[p]{figureA-22}
\end{bwslide}


\begin{bwslide}
\ctitle {VIRTUAL TERMINAL}

\begin{nrtc}
\item	PURPOSE
	\begin{nrtc}
	\item	PROVIDE REMOTE TERMINAL CAPABILITY
	\item	ALLOW ACCESS TO REMOTE COMPUTER SYSTEMS ACROSS AN OSI NETWORK
	\item	ULTIMATELY ALLOW A MAPPING BETWEEN ANY APPLICATION AND ANY TERMINAL
	\end{nrtc}
\end{nrtc}
\end{bwslide}


%\begin{bwslide}
%\ctitle {VIRTUAL TERMINAL \\ REFERENCES}
%
%\begin{description}
%\item[ISO/IEC 9040:]	Virtual Terminal Service --- Basic Class
%\item[ISO/IEC 9041:]	Virtual Terminal Protocol --- Basic Class
%\end{description}
%\end{bwslide}


%\begin{bwslide}
%\ctitle {INTRODUCTION}
%
%\begin{nrtc}
%\item   LOOSELY COUPLED SYSTEMS THAT ARE BUILT USING REMOTE PROCEDURE CALLS
%        (RPC) ARE GAINING POPULARITY, e.g., NFS
%
%\item   THE OSI REMOTE OPERATIONS CONCEPT IS INTENDED TO PROVIDE THIS
%        FUNCTIONALITY FOR:
%    \begin{nrtc}
%    \item       MESSAGING
%
%    \item       DIRECTORY SERVICES
%
%    \item       NETWORK MANAGEMENT
%
%    \item       REMOTE DATABASE ACCESS
%    \end{nrtc}
%\end{nrtc}
%\end{bwslide}
%
%
%\begin{bwslide}
%\ctitle {MOTIVATION}
%
%\begin{nrtc}
%\item   MANY FEEL THAT THIS CAPABILITY MAY BE A KEY FACTOR IN THE OVERALL
%        SUCCESS OF OSI STANDARDIZATION
%
%\item   BUT, REMOTE OPERATIONS ARE SUFFICIENTLY GENERAL TO REQUIRE
%        ADDITIONAL DISCIPLINE, BEYOND THE ISO/CCITT STANDARDS,
%        FOR THEIR USE AS AN RPC MECHANISM
%\end{nrtc}
%\end{bwslide}
%
%
%\begin{bwslide}
%\ctitle {THE APPLICATIONS COOKBOOK}
%
%\begin{nrtc}
%\item   THE SET OF RULES AND LOCAL IMPLEMENTATION DECISIONS PLACED ON REMOTE
%        OPERATIONS TO MAKE THE PROBLEM MANAGEABLE:
%    \begin{nrtc}
%    \item       LANGUAGE BINDINGS (``C'')
%
%    \item       TOOLS FOR AUTOMATICALLY GENERATING PARTS OF THE
%                PROGRAMS WHICH USE REMOTE OPERATIONS
%
%    \item       A RUN-TIME ENVIRONMENT AND SOME BOILERPLATE
%
%    \item       CONVENTIONS FOR NAMING AND ADDRESSING SERVICES AND ENTITIES
%    \end{nrtc}
%\end{nrtc}
%\end{bwslide}
%
%
%\begin{bwslide}
%\part*	{\bf DEFINING A SERVICE}
%
%\begin{nrtc}
%\item	TWO ASPECT TO SERVICE DEFINITION
%\item	STATIC CHARACTERISTICS
%	\begin{nrtc}
%	\item	DEFINED IN SERVICE DEFINITION
%	\item	GENERIC TO ANY ENTITY REALIZING THE SERVICE
%	\end{nrtc}
%\item	DYNAMIC CHARACTERISTICS
%	\begin{nrtc}
%	\item	DEFINED BY SERVICE PROVIDER
%	\item	VARIES BETWEEN ENTITIES
%	\end{nrtc}
%\end{nrtc}
%\end{bwslide}
%
%
%\begin{bwslide}
%\ctitle {STATIC CHARACTERISTICS: \\ ABSTRACT SYNTAX}
%
%\begin{nrtc}
%\item	DEFINES THE DATA STRUCTURES BEING EXCHANGED BY THE SERVICE
%\item	VALUE IS AN OBJECT IDENTIFIER, e.g.:
%	\begin{quote}\small
%	ftam pci	1.0.8571.2.1
%	\end{quote}
%\item	ALL SERVICES USE AT LEAST TWO
%	\begin{nrtc}
%	\item	``PRIMARY'' PROTOCOL
%	\item	ASSOCIATION CONTROL
%	\end{nrtc}
%\item	SOME USE MORE, e.g., FTAM USES AN ABSTRACT SYNTAX FOR EACH NEGOTIATED
%	DOCUMENT TYPE
%\end{nrtc}
%\end{bwslide}
%
%
%\begin{bwslide}
%\ctitle {STATIC CHARACTERISTICS: \\ APPLICATION CONTEXT NAME}
%
%\begin{nrtc}
%\item	DEFINES THE APPLICATION SERVICE ELEMENTS (ASEs) USED BY THE SERVICE AND
%	HOW THEY INTERACT
%\item	VALUE IS AN OBJECT IDENTIFIER, e.g.:
%	\begin{quote}\small
%	iso ftam	1.0.8571.1.1
%	\end{quote}
%	INDICATES USE OF THE FTAM ASE AND THE ACSE
%\item	SOME RELATIONSHIPS ARE MORE COMPLICATED, e.g., USE PROTOCOL ``p17''
%	ON TOP OF ROSE ON TOP OF RTSE ON TOP OF ACSE
%\end{nrtc}
%\end{bwslide}
%
%
%\begin{bwslide}
%\ctitle {DYNAMIC CHARACTERISTICS: \\ APPLICATION ENTITY TITLE}
%
%\begin{nrtc}
%\item	UNIQUELY NAMES AN ENTITY IN THE NETWORK
%\item	VALUE IS A DISTINGUISHED NAME, e.g.,
%	\begin{quote}\small\begin{verbatim}
%	c=US@o=TWG@ou=Software Engineering@cn=boomer@cn=filestore
%	\end{verbatim}\end{quote}
%	(THIS IS A STRING REPRESENTATION OF A COMPLEX ASN.1 TYPE!)
%\item	SIMPLER APPROACH IS TO USE LOCAL ALIASING (``boomer-filestore'') PRIOR TO
%	RESOLUTION:
%	\begin{nrtc}
%	\item	LHS (QUALIFIER) MAPS TO INITIAL PART OF DISTINGUISHED NAME
%	\item	RHS (DESIGNATOR) MAPS TO SUFFIX
%	\end{nrtc}
%\end{nrtc}
%\end{bwslide}
%
%
%\begin{bwslide}
%\ctitle {APPLICATION ENTITY TITLE}
%
%\begin{nrtc}
%\item	FOR INITIATORS, THE DIRECTORY IS ASKED TO RETURN THE ``PRESENTATION ADDRESS''
%	ATTRIBUTE OF THIS OBJECT
%\item	CONCEPTUALLY, RESPONDERS STORE THIS INFORMATION IN THE DIRECTORY
%\end{nrtc}
%\end{bwslide}
%
%
%\begin{bwslide}
%\ctitle {DYNAMIC CHARACTERISTICS: \\ PRESENTATION ADDRESS}
%
%\begin{nrtc}
%\item	UNIQUELY IDENTIFIES THE LOCATION OF AN ENTITY
%\item	VALUE IS
%	\begin{nrtc}
%	\item	PRESENTATION SELECTOR (0+ OCTETS)
%	\item	SESSION SELECTOR (0+ OCTETS)
%	\item	TRANSPORT SELECTOR (0+ OCTETS)
%	\item	NETWORK ADDRESSES (AT LEAST 1)
%	\end{nrtc}
%\end{nrtc}
%\end{bwslide}
%
%
%\begin{bwslide}
%\ctitle {LOCAL ENVIRONMENT: \\ CURRENT METHOD}
%
%\begin{nrtc}
%\item	TRANSPORT LISTENER RESIDES ON EACH END-SYSTEM
%\item	FOR EACH SERVICE LISTED IN LOCAL DATABASE
%	\begin{nrtc}
%	\item	LISTEN ON ASSOCIATED TRANSPORT ADDRESS
%	\item	UPON RECEIVING AN INCOMING CONNECTION, TSAPD
%		INVOKES ASSOCIATED PROGRAM
%	\end{nrtc}
%\item	PROGRAM IS CALLED A DYNAMIC RESPONDER
%\end{nrtc}
%\end{bwslide}
%
%
%\begin{bwslide}
%\ctitle {LOCAL ENVIRONMENT: \\ CURRENT METHOD (cont.)}
%
%\begin{nrtc}
%\item	SOME RESPONDERS MAY WISH TO SERVICE MULTIPLE INITIATORS
%\item	DIFFERENT APPROACH:
%	\begin{nrtc}
%	\item	SERVICE NOT LISTED IN LOCAL DATABASE
%	\item	PROGRAM LISTENS ON OWN TRANSPORT ADDRESS
%	\end{nrtc}
%\item	PROGRAM IS CALLED A STATIC RESPONDER
%\end{nrtc}
%\end{bwslide}
%
%
%\begin{bwslide}
%\ctitle {LOCAL ENVIRONMENT: \\ FUTURE METHOD}
%
%\begin{nrtc}
%\item	LOCAL DATABASE CONTAINS
%	\begin{nrtc}
%	\item	SERVICE NAME
%	\item	LOCAL PROGRAM
%	\end{nrtc}
%	BUT NOT TRANSPORT ADDRESS
%\item	FOR EACH SERVICE
%	\begin{nrtc}
%	\item	TRANSPORT LISTENER LISTENS ON RANDOM TRANSPORT ADDRESS
%	\item	LISTENER REGISTERS SERVICE \& TRANSPORT ADDRESS WITH THE DIRECTORY
%	\end{nrtc}
%\item	SIMILAR APPROACH IS USED BY STATIC RESPONDERS
%\end{nrtc}
%\end{bwslide}
%
%
%\begin{bwslide}
%\ctitle {UPPER--LAYER ADDRESSING}
%
%\begin{nrtc}
%\item	TOO MUCH FREEDOM
%\item	DYNAMIC RESPONDER PREFER TO ``DISPATCH'' ON TRANSPORT SELECTOR
%\item	STATIC RESPONDERS MAY REQUIRE A UNIQUE NETWORK ADDRESS
%\end{nrtc}
%\end{bwslide}
%
%
%\begin{bwslide}
%\ctitle {UPPER--LAYER ADDRESSING}
%
%\begin{nrtc}
%\item	U.S. GOSIP CALLS FOR DISPATCHING ON PRESENTATION SELECTOR
%\item	OKAY FOR DYNAMIC RESPONDERS
%\item	STATIC RESPONDERS MAY STILL REQUIRE A UNIQUE NETWORK ADDRESS
%	\begin{nrtc}
%	\item	WITH REAL OSI NETWORK SERVICE, STATIC RESPONDERS CAN DISPATCH
%		ON TRANSPORT SELECTOR
%	\end{nrtc}
%\end{nrtc}
%\end{bwslide}
