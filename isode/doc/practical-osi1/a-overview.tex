% run this through LaTeX with the appropriate wrapper

\dotopic{1}
\begin{bwslide}
\part	{APPLICATION SERVICES}\bf
\end{bwslide}
\doparts


\begin{bwslide}
\part*	{OUTLINE}\bf

\begin{description}
\item[PART I:]		BASIC CONCEPTS 
\item[PART II:]		SESSION SERVICES
\item[PART III:]	ABSTRACT SYNTAX
\item[PART IV:]		PRESENTATION SERVICES
\item[PART V:]		APPLICATION SERVICE ELEMENTS
\item[PART VI:]		BUILDING AN APPLICATION
\end{description}
\end{bwslide}

%\begin{note}\em
%\begin{center}
%\underline{an audience survey}
%\end{center}
%
%who has heard of, is familiar with, or understands:
%\begin{quote}
%the osi model?
%
%abstract syntax notation one?
%
%remote operations in the context of OSI?
%
%sun rpc, or apollo ncs/nidl?
%\end{quote}
%
%who knows how to program under unix using:
%\begin{quote}
%the C programming language, make, shell, etc.?
%\end{quote}
%\end{note}


%\begin{note}\em
%everyone should be comfortable the material review
%\end{note}



%\begin{bwslide}
%\part	{\bf BASIC CONCEPTS}
%
%\begin{nrtc}
%\item	OPEN SYSTEMS INTERCONNECTION
%
%\item	OSI ITSELF IS QUITE SIMPLE
%
%    \begin{nrtc}
%    \item	A METHODOLOGY FOR DESCRIBING OPEN SYSTEMS
%    		WHICH IS USED IN SPECIFICATIONS WHICH DESCRIBE HOW THESE
%		SYSTEMS COMMUNICATE
%    \end{nrtc}
%
%\end{nrtc}
%\end{bwslide}

%\begin{bwslide}
%\ctitle {BASIC CONCEPTS}
%
%\begin{nrtc}
%\item	CONCERNED WITH OPEN INTERCONNECTION OF SYSTEMS
%\item	DOES NOT IMPLY OPENLY ACCESSIBLE SYSTEMS
%\end{nrtc}
%\end{bwslide}


%\begin{bwslide}
%\ctitle	{BASIC CONCEPTS (cont.)}
%
%\begin{nrtc}
%\item	AT FIRST GLANCE, OSI IS DESCRIBED IN A LANGUAGE ALL ITS OWN
%\item	OSI STANDARDS USUALLY HAVE TWO PARTS:
%	\begin{nrtc}
%	\item	A SERVICE DEFINITION
%	\item	A PROTOCOL SPECIFICATION
%	\end{nrtc}
%\end{nrtc}
%\end{bwslide}


%\begin{bwslide}
%\ctitle	{WHAT WE WILL DISCUSS}
%
%\begin{nrtc}
%\item	A PRACTICAL LOOK
%	\begin{nrtc}
%	\item	PIECES OF THE OSI MODEL
%	\item	ACTUAL SERVICES
%	\item	REAL WORLD USAGE
%	\end{nrtc}
%\end{nrtc}
%\end{bwslide}

%\begin{bwslide}
%\ctitle	{OSI SAYS WHAT TO DO, NOT HOW TO DO IT}
%
%\begin{nrtc}
%\item	IT SPECIFIES THE EXTERNAL BEHAVIOR OCCURRING BETWEEN SYSTEMS
%\item	IT DOES NOT SPECIFY HOW LOCAL SYSTEMS ARE BUILT
%\item	DOES SPECIFY
%    \begin{nrtc}
%    \item	WHAT GOES ON THE ``WIRE'' AND WHEN
%    \end{nrtc}
%
%\item	IT DOES NOT SPECIFY
%    \begin{nrtc}
%    \item	PROGRAMMING LANGUAGE BINDINGS
%    \item	OPERATING SYSTEMS BINDINGS
%    \item	APPLICATION INTERFACE ISSUES
%    \item	USER-INTERFACE ISSUES
%    \end{nrtc}
%\end{nrtc}
%\end{bwslide}


%\begin{bwslide}
%\ctitle {WHO ARE THE PLAYERS}
%
%\begin{nrtc}
%\item	THE INTERNATIONAL ORGANIZATION FOR STANDARDIZATION (ISO)
%	\begin{nrtc}
%	\item	NATIONAL STANDARDS BODIES
%	\end{nrtc}
%\item	THE INTERNATIONAL TELEGRAPH AND TELEPHONE CONSULTATIVE COMMITTEE (CCITT)
%	\begin{nrtc}
%	\item	NATIONAL TELECOM ADMINISTRATIONS
%	\end{nrtc}
%\end{nrtc}
%\end{bwslide}


%\begin{bwslide}
%\ctitle {WHO ARE THE PLAYERS (cont.)}
%
%\begin{nrtc}
%\item	THE NATIONAL INSTITUTE OF STANDARDS AND TECHNOLOGY (NIST)
%	\begin{nrtc}
%	\item	CORRESPONDS TO OTHER REGIONAL BODIES, EWOS, AOWS, etc.
%	\end{nrtc}
%\item	THE CORPORATION FOR OPEN SYSTEMS (COS)
%\item	THE TECHNICAL OFFICE PROTOCOLS USER'S GROUP \& 
%	THE MANUFACTURING AUTOMATION PROTOCOLS USER'S GROUP 
%	\begin{nrtc}
%	\item	(MAP/TOP)
%	\end{nrtc}
%\item	THE INTERNATIONAL FEDERATION OF INFORMATION PROCESSING (IFIP)
%\end{nrtc}
%\end{bwslide}


%\begin{bwslide}
%\ctitle {ISO}
%
%\begin{nrtc}
%\item	ISO DOES NOT STAND FOR INTERNATIONAL STANDARDS ORGANIZATION
%\item	DON'T CONFUSE OSI \& ISO:
%	\begin{nrtc}
%	\item	ISO IS AN ORGANIZATION
%	\item	OSI IS A COMMUNICATIONS MODEL
%	\end{nrtc}
%%\item	ONLY ISO PRODUCES ``STANDARDS'', OTHERS PRODUCE OTHER THINGS, e.g.,
%%	\begin{nrtc}
%%	\item	RECOMMENDATIONS --- CCITT
%%	\end{nrtc}
%\end{nrtc}
%\end{bwslide}


%\begin{bwslide}
%\ctitle {US STANDARDS PROCESS}
%
%\vskip.5in
%\diagram[p]{figureA-1}
%\end{bwslide}


%\begin{bwslide}
%\ctitle {WHAT ABOUT THE INTERNET COMMUNITY}
%
%\begin{nrtc}
%\item	DARPA/NSF INTERNET RESEARCH COMMUNITY IS NOTICEABLY ABSENT
%\item	DIFFERENCES IN THE COMMUNITIES
%	\begin{nrtc}
%	\item	e.g., FOCUS ON EXPERIMENTATION vs. WRITTEN CONTRIBUTIONS
%	\end{nrtc}
%\item	THIS HAS BEEN A STRATEGIC ERROR!
%\end{nrtc}
%\end{bwslide}


%\begin{bwslide}
%\ctitle {STATUS}
%
%\begin{nrtc}
%%\item	OSI STANDARDS AND VENDOR AGREEMENTS ARE REACHING STABLE STATUS
%%\item	THE U.S. GOSIP WILL PROVIDE THE INITIAL DEMAND FOR OSI IN THE U.S.
%%	AND OTHER COUNTRIES
%\item	TECHNOLOGY STILL NEEDS REFINEMENT
%	\begin{nrtc}
%	\item	MANY CURRENT OSI OFFERINGS ARE CLOSER TO EXPERIMENTS THAN
%		TO PRODUCTS
%	\item	MANY PRODUCTS ARE SPECIFIC TO MAP/TOP
%	\end{nrtc}
%\end{nrtc}
%\end{bwslide}


%\begin{bwslide}
%\ctitle {U.S. GOSIP}
%
%\begin{nrtc}
%\item	A FEDERAL INFORMATION PROCESSING STANDARD (FIPS~146)
%\item	PROPOSED TO ENABLE USERS TO SPECIFY AND PROCURE
%	\begin{nrtc}
%	\item	INTEROPERABLE
%	\item	MULTI-VENDOR
%	\item	OFF-THE-SHELF
%	\end{nrtc}
%	COMPUTER COMMUNICATIONS PRODUCTS
%\end{nrtc}
%\end{bwslide}


%\begin{note}
%protocols
%\end{note}


%\begin{note}
%services
%\end{note}


%\begin{note}
%provider/consumer, initiator/responder,  client/server
%\end{note}


\begin{bwslide}
\part	{BASIC CONCEPTS\\THE OSI MODEL}\bf

\begin{nrtc}
\item   A LAYERED ARCHITECTURE FOR COMPUTER COMMUNICATIONS

\item   STANDARDIZED IN THE INTERNATIONAL COMMUNITY

\item   NON-PROPRIETARY IN NATURE
\end{nrtc}
\end{bwslide}


\begin{bwslide}
\ctitle {UPPER-LAYER INFRASTRUCTURE}

\begin{nrtc}
\item	UPPER-LAYERS ARE EVERYTHING ABOVE TRANSPORT
\item	THE SAME UPPER-LAYERS ARE USED REGARDLESS OF THE APPLICATION,
	UNLIKE OTHER ARCHITECTURES (e.g., TCP/IP)
\item	EACH APPLICATION MAY SELECT DIFFERENT FUNCTIONALITY FROM THE UPPER LAYERS
\item	THE EMPHASIS IS ON FLEXIBILITY, TO SUPPORT MANY DIVERSE OSI APPLICATIONS
\end{nrtc}
\end{bwslide}


\begin{bwslide}
\ctitle	{OSI UPPER-LAYER INFRASTRUCTURE}

\vskip.5in
\diagram[p]{figureA-2}
\end{bwslide}


%\begin{bwslide}
%\ctitle {CORRESPONDING INTERNET \\ UPPER-LAYER INFRASTRUCTURE}
%
%\vskip.5in
%\diagram[p]{figureA-3}
%\end{bwslide}


\begin{bwslide}
\ctitle {BOTTOM LINE}

\begin{nrtc}
\item	STRENGTHS
	\begin{nrtc}
	\item	COMMON INFRASTRUCTURE
	\item	MORE TECHNICALLY COMPREHENSIVE
	\end{nrtc}
\item	WEAKNESSES
	\begin{nrtc}
	\item	COMPLEXITY
	\item	POLITICAL POLARIZATION
	\item	LACK OF PRACTICAL EXPERIENCE
	\end{nrtc}
\end{nrtc}
\end{bwslide}


\begin{bwslide}
\ctitle {THE MODEL FROM A COMMUNICATIONS VIEWPOINT}

\vskip.5in
\diagram[p]{figureA-4}
\end{bwslide}


\begin{bwslide}
\ctitle {THE MODEL FROM A COMPUTER VIEWPOINT}

\vskip.5in
\diagram[p]{figureA-5}
\end{bwslide}


\begin{bwslide}
\ctitle {LAYERING}

\vskip.5in
\diagram[p]{figureA-32}
\end{bwslide}


%\begin{bwslide}
%\ctitle {TERMINOLOGY}
%
%\begin{nrtc}
%\item	SDU (SERVICE DATA UNIT) --- USER DATA
%\item	PCI (PROTOCOL CONTROL INFORMATION) --- HEADER
%\item	PDU (PROTOCOL DATA UNIT) --- PACKET
%	\begin{nrtc}
%	\item	PDU = PCI + SDU
%	\end{nrtc}
%\item	ICI (INTERFACE CONTROL INFORMATION)~---~PROCEDURE
%\item	IDU (INTERFACE DATA UNIT)~---~CALL
%	\begin{nrtc}
%	\item	IDU = ICI + PDU
%	\end{nrtc}
%\item	SAP (SERVICE ACCESS POINT)
%\end{nrtc}
%\end{bwslide}


%\begin{bwslide}
%\ctitle {DATA TRANSIT}
%
%\vskip.5in
%\diagram[p]{figureA-14}
%\end{bwslide}


\begin{bwslide}
\ctitle {SERVICES vs. PROTOCOLS}

\vskip.5in
\diagram[p]{figureA-39}
\end{bwslide}


\begin{bwslide}
\ctitle  {SERVICES AND\\ SERVICE PRIMITIVES}

\begin{nrtc}
\item   PEERS COMMUNICATE VIA \emph{SERVICE PRIMITIVES}

\item   A PRIMITIVE IS AN ABSTRACTION
    \begin{nrtc}
    \item       NOT AN INTERFACE
    \end{nrtc}

\item   SERVICE PRIMITIVES, LIKE PROCEDURE CALLS, HAVE TYPED PARAMETERS
\end{nrtc}
\end{bwslide}


%\begin{bwslide}
%\ctitle {SERVICE}
%
%\begin{nrtc}
%\item   IN GENERAL, THERE ARE THREE KINDS OF SERVICES
%    \begin{nrtc}
%    \item       \emph{CONFIRMED}
%        \begin{nrtc}
%        \item   IN WHICH A REQUEST ALWAYS RESULTS IN A RESPONSE
%        \end{nrtc}
%
%    \item       \emph{UNCONFIRMED}
%        \begin{nrtc}
%        \item   IN WHICH NO RESPONSE IS RETURNED
%        \end{nrtc}
%
%    \item       \emph{PROVIDER-INITIATED}
%        \begin{nrtc}
%        \item   IN WHICH THE SERVICE PROVIDER INDICATES SOME SITUATION
%        \end{nrtc}
%    \end{nrtc}
%
%\item   CONFIRMATION IS UNRELATED TO RELIABILITY
%\end{nrtc}
%\end{bwslide}


\begin{bwslide}
\ctitle {SERVICE PRIMITIVES}

\begin{nrtc}
\item   EACH LAYER (OR ELEMENT) OFFERS ONE OR MORE SERVICES (VERBS)
    \begin{nrtc}
    \item       e.g., A-ASSOCIATE
    \end{nrtc}

\item   A SERVICE CONSISTS OF ONE OR MORE PRIMITIVES

\item   A CONFIRMED SERVICE HAS FOUR PRIMITIVES
    \begin{nrtc}
    \item       .REQUEST, .INDICATION, .RESPONSE, and .CONFIRMATION
    \end{nrtc}

\item   AN UNCONFIRMED SERVICE HAS TWO PRIMITIVES:
    \begin{nrtc}
    \item       .REQUEST,  and .INDICATION
    \end{nrtc}

\item   A PROVIDER-INITIATED SERVICE HAS ONE PRIMITIVE:
    \begin{nrtc}
    \item       .INDICATION
    \end{nrtc}
\end{nrtc}
\end{bwslide}


\begin{bwslide}
\ctitle {SERVICE PRIMATIVES}

\vskip.5in
\diagram[p]{figureA-24}
\end{bwslide}


%\begin{bwslide}
%\ctitle {CONFIRMED SERVICE}
%
%\vskip.5in
%\diagram[p]{figureA-40}
%\end{bwslide}


%\begin{bwslide}
%\ctitle {CONFIRMED SERVICE}
%
%\vskip.5in
%\diagram[p]{figureA-41}
%\end{bwslide}


%\begin{bwslide}
%\ctitle	{CONNECTION ORIENTED}
%
%\begin{nrtc}
%\item	AVAILABLE UPPER-LAYERS ASSUME CONNECTION ORIENTED SERVICES
%\item	CONNECTIONLESS ADDENDA BEING DEVELOPED
%\end{nrtc}
%\end{bwslide}


%\begin{bwslide}
%\ctitle {SESSION LAYER --- OVERVIEW}
%
%\begin{nrtc}
%\item	ESTABLISH, RELEASE, MANAGE TRANSPORT CONNECTIONS
%\item	NEGOTIATE AND POLICE COMMUNICATION PARAMETERS
%\end{nrtc}
%\end{bwslide}


%\begin{bwslide}
%\ctitle {ABSTRACT SYNTAX --- OVERVIEW}
%
%\begin{nrtc}
%\item	DEFINES ENCODING \& DECODING RULES
%\item	IN PRACTICE, FACILITIES USED BY
%	\begin{nrtc}
%	\item	PRESENTATION
%	\item	APPLICATION
%	\end{nrtc}
%\item	MACHINE INDEPENDENT ENCODINGS
%\item	FORMAL LANGUAGE
%\end{nrtc}
%\end{bwslide}


%\begin{bwslide}
%\ctitle {PRESENTATION LAYER --- OVERVIEW}
%
%\begin{nrtc}
%\item	ESTABLISHES (NEGOTIATES) ``SYNTACTIC'' CONVENTIONS FOR PEER
%	APPLICATION ENTITY COMMUNICATION
%\item	ENCODES \& DECODES APPLICATION AND PRESENTATION
%	LAYER STRUCTURES AND DATA UNITS
%\end{nrtc}
%\end{bwslide}


%\begin{bwslide}
%\ctitle {THE OSI APPLICATION LAYER}
%
%\begin{nrtc}
%\item   MANY STANDARD ``APPLICATION'' SERVICE ELEMENTS
%    \begin{nrtc}
%    \item       ASSOCIATION CONTROL
%
%    \item       REMOTE OPERATIONS
%
%    \item       RELIABLE TRANSFER
%
%    \item       DIRECTORY SERVICES
%    \end{nrtc}
%
%\item   ABSTRACT SYNTAX NOTATION ONE (ASN.1)\\
%        (not really a layer, more of a concept)
%
%\end{nrtc}
%\end{bwslide}


%\begin{bwslide}
%\ctitle {APPLICATION SERVICE ELEMENTS --- OVERVIEW}
%
%\begin{nrtc}
%\item	PROVIDES ``COMMON'' SERVICES IN APPLICATION LAYER
%\item	APPLICATION BUILDING BLOCKS FOR STANDARD FUNCTIONALITY
%	\begin{nrtc}
%	\item	ASSOCIATION CONTROL (ACSE)
%	\item	REMOTE OPERATIONS (ROSE)
%	\item	etc.
%	\end{nrtc}
%\end{nrtc}
%\end{bwslide}

%\begin{bwslide}
%\ctitle {APPLICATION SERVICE ELEMENTS}
%
%\begin{nrtc}
%\item   A USEFUL MECHANISM FOR DIVIDING RESPONSIBILITY OF THE ``TOTAL''
%        APPLICATION PROTOCOL
%
%\item   PROMOTES ``REUSE'' OF APPLICATION LAYER FACILITIES
%\end{nrtc}
%\end{bwslide}


%\begin{bwslide}
%\ctitle {EXAMPLE:\\ FTAM USE OF LOWER-LAYER SERVICES}
%
%\vskip.5in
%\diagram[p]{figureA-42}
%\end{bwslide}



%\begin{bwslide}
%\ctitle {APPLICATIONS --- OVERVIEW}
%
%\begin{nrtc}
%\item	THE REAL WORKER ON TOP OF THE STACK
%\item	DOES NOT INCLUDE USER INTERFACES
%\item	EXAMPLES
%	\begin{nrtc}
%	\item	MESSAGE HANDLING (X.400)
%	\item	DIRECTORY (X.500)
%	\item	FILE TRANSFER (FTAM)
%	\item	etc.
%	\end{nrtc}
%\end{nrtc}
%\end{bwslide}


%\begin{bwslide}
%\ctitle {ABOVE THE APPLICATION LAYER}
%
%\begin{nrtc}
%\item	OUTSIDE THE SCOPE OF THE OSI/RM \& STANDARDS
%\item	ENTITIES (APPLICATION PROCESSES) MAKING USE OF OSI SERVICES
%\item	EXAMPLES
%	\begin{nrtc}
%	\item	USER INTERFACES
%	\item	DATABASE APPLICATION PROCESSES
%	\end{nrtc}
%\end{nrtc}
%\end{bwslide}
