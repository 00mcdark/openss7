% Use LaTeX to process this file

\documentstyle[11pt]{article}

\begin{document}

\title{A MODEST PROPOSAL}
\author{Dwight E. Cass\\
        Automation Sciences Laboratory\\
        Northrop Research and Technology Center}
\date{May 15, 1986}
\maketitle

\begin{quote}\em
This is a transcript of the talk presented to the MAP/TOP Users' Group Meeting
on May 15, 1986 in Seattle, Washington.
\end{quote}

\bigskip

I am going to limit my comments today to one particular project which is
going on at the Corporate Research and Technology Center.  That is not to
say that Northrop is not doing other MAP and TOP related projects.  There
are a number of other opportunities currently being explored, including
an RFI which was recently released by our Aircraft Division looking to
vendor support headed to MAP and TOP.

At the Research Center, we are looking towards long range solutions to
Northrop, which is an aerospace vendor.  We have a number of unique problems
which General Motors doesn't directly address.

We have identified two specific problems that we are trying to address.  The
first problem is a lack of application protocols.  This lack has been
identified and is being addressed within the MAP and TOP community but is
not being addressed quickly enough.  The biggest problem that we have seen
is that history has shown that the application level protocol development
takes as long or significantly longer than the lower level protocols.

The second problem that we have seen is that currently we do not have MAP
and TOP for all the machines we have throughout the corporation.  One of
the biggest problems we have is with IBM right now.  Over 50\% of the
machines that we have in house are IBM equipment.  It is not adequate
to put Series Ones talking to SNA.  ({\em Applause\/})  Our customers
are the Department of Defense.  He wants us to talk electronically, TODAY!
He is not willing to wait.

The cause of these problems --- first, MAP/TOP, while unbelievably
successfully is still very very young.  There is quite a bit more work
to be done.  All the work that is going on right now that the standards
committees are developing must be embodied.  We have to continue this
level of effort.  But, Northrop has its requirements NOW!  We need to
be able to communicate with the participants electronically.  We need to
have factory and division-wide networking installed and operational in
the '88 to '89 timeframe.  We cannot consider continuing to make large
investments in SNA.  And it is simply not acceptable for Northrop to
shutdown its factory for a large period of time to replace existing
equipment and throw it out.  Finally, any solution that we take, if we
choose to do something other than a pure MAP/TOP solution must be
consistent with the directions of this body.

Two observations to be made at this point.  The first is that many of
these problems have been solved in the past, though definitely in
different environments.  There are several other proprietary and research
based protocol suites, which address many of these problems.  The second
observation is that MAP/TOP is a highly layered protocol, and indeed is
a subset of ISO.  As a matter of fact, it is a walk-through the tree
of the ISO protocols.  The key here is that layered protocols allow mixing
and matching of layers as necessary.  We are trying to emphasize here
services provided by each layer, not necessarily the implementation
of those services at the low level.  So the question that we have to ask is,
How does Northrop get communications now that would be consistent with
MAP/TOP and that would communicate electronically with the DoD?

Now, we have to look to the DoD for the solution, and the solution seems
to be TCP/IP.  This is a protocol suite installed in 1980 on the ARPAnet
worldwide internet.  It has been mandated for use for all DoD
internetworking.  It has a high emphasis on reliability and security.
It supplies currently virtual terminal service, electronic mail, file
transfer services, and a number of other less advertised capabilities.
It has wide, wide vendor support.  I can go out today and buy TCP for
every machine that I have in house, off the shelf, install it easily,
and specifically, I can buy TCP/IP for every piece of IBM hardware that I
have.

TCP/IP specifically supports a very robust transport service, as documented
in the National Research Council report comparing TCP/IP to ISO, the
functionality of these two protocols are pretty much identical.  The other
observation here is that TCP/IP, due to it's vendor support and it's
openness, has been implemented in board level products.  This allows us
to place TCP/IP in existing pieces of hardware that are fully loaded,
without making major impacts on the CPU capabilities that we have in house.

So, how do we construct this solution and stay with where MAP/TOP is going?
The only solution seems to be complementary coexistence.  We need the
best of both worlds.  This will give us the opportunity to run today,
to run tomorrow, and by the time that the Department of Defense has met
ISO, we'll be already there.  It is important to emphasize that Northrop
is looking for an evolutionary migration path to its communications problems.
It is not willing to accept throwing away and starting again.

If you are going to talk about coexistence, you are going to have to decide
where you are going to join the two protocols.  Several places come to
mind.  One of the earliest proposals was to join at the IP layer.  To do
this, of course, would require that we implement a full TP4.  It didn't
seem practical, and it didn't seem to be cost effective, besides the fact
that it prohibits us from using board level products.  We looked at
implementing at the Session layer, but the problem is that above Transport,
we start dealing with the problems of the symmetric ISO model versus the
client/server ARPA model.  The only obvious point was at the Transport
Service Access Point.  This gave us the capability of using the robust
TCP transport service while still allowing MAP freedom to do whatever
was necessary at the higher layers, isolating the applications from 
the reality of whatever they were running.

Where are we headed today?  At the Research Center we published a
proposed Transfer Service Access Point.  We then implemented this proposal
on a Berkeley 4.2 UNIX system, and have since published this proposal
within the ARPA internet community as RFC983.  RFC983 fully describes
a TP4/TSAP implementation on top of TCP.  Everything except for, of course,
quality of service is supported within the implementation, because as you
know, quality of service is not defined yet.  For addressing we have to take
TCP/IP-based addressing, however we would be more than happy to start dealing
with the addressing issue as the Standards Committees do, and currently
we have an implicit maximum TSDU size of 65 Kbytes, but that will be
relaxed in the next version.  I do have with me today, copies of the RFC
and I can get a hold of the source code for you.  This is all public
domain work.  You are all welcome to take it and if you need, work with it.

Our next step.  What we need next is to put an implementation of Session, FTAM,
and CASE on this TSAP.  One of the important things here is we can go
off and write it, but it would invalidate any hopes of compatibility between
MAP and what we're doing.  As such, we must find a version of Session, FTAM
and CASE intended for a functional MAP system --- hopefully public domain,
so we can distribute it to the ARPA internet community.  The other thing
that we will be doing is implementing this RFC on the other machines
currently within Northrop.

As a migration strategy to ISO, RFC983 is not planned solely as a migration,
but could easily form the backbone for a full migration plan.  What we are
hoping to do is to fully track what the Department of Defense does over the
next several years to yet bring us to ISO when our customers have.

In summary, what we think we found is the best of all worlds.  We can talk
to our customer today with such a capability.  We can install and perform
research, today.  We can develop MAP/TOP applications, today.  We can migrate
to ISO as our customer does.  We solved the security problem for our
communications, and reaped the benefit of all the researchers within the
Department of Defense internet community.

Thank you very much.  ({\em Applause\/})

\end{document}
