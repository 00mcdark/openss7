% run this through LaTeX with the appropriate wrapper

\chapter	{FTAM-FTP gateway}\label{ftamftp}
The FTAM/FTP gateway is an application-gateway for file service.
The gateway is actually two programs:
one which acts as an FTAM responder and an FTP client,
and the other which acts as an FTP server and an FTAM initiator.
Note that the gateway currently resides on a different location than the
standard FTAM responder and FTP server.

The implementation included runs only on Berkeley \unix/.

\section	{Implementation}\label{ftamftp:code}
If you have access to the source tree for this release,
the directories \file{ftam-ftp/} and \file{ftp-ftam/}
contains the code for the two programs.

\subsection	{The FTAM/FTP side}
The FTAM/FTP side of the gateway appears to implement the responder side of
the FTAM service,
but actually acts as an FTP client in order to provide this service.

The true destination is encoded in the user name (i.e., \verb"user@tcphost").

Note that the FTAM/FTP side is available on a different presentation address
than the FTAM service on the gateway host.
To select the FTAM/FTP side,
tell your FTAM initiator to associate with the service having ``qualifier''
\verb"ftpstore" on the gateway host.
For example, using \man ftam(1c):
\begin{quote}\small\begin{verbatim}
% ftam
ftam> set qualifier ftpstore
ftam> open gateway
user (gateway:user): user@tcphost
password (gateway:user@tcphost): 
\end{verbatim}\end{quote}

\subsubsection	{Limitations}
File information is limited to file names.
All file access rights are assumed until access is attempted;
the FTP server of the utlimate destination grants or denies action permission
at the time of file access.

Empty directories may not be recognized depending on the FTP server of the
destination machine.
This bug manifests itself when trying to remove an empty directory.

\subsection	{The FTP/FTAM side}
The FTP/FTAM side of the gateways appears to be an FTP server,
but actually acts as an FTAM initiator in order to provide this service.

The true destination is encoded in the user name (i.e., \verb"user@osihost"),
or by using the FTP SITE command.
If further accounting information is required by the true destnation,
the FTP ACCT command is used seperately and the SITE command must be used to
specify the destination.

Note that the FTP/FTAM side is available on a different port than the FTP
server on the gateway host.
To select the FTP/FTAM side,
tell your FTP client to connect to port 531 on the gateway host.
For example, using \man ftp(1c):
\begin{quote}\small\begin{verbatim}
% ftp
ftp> open gateway 531
ftp> open gateway
Name (gateway:user): user@osihost
Password:
\end{verbatim}\end{quote}

\subsubsection	{Limitations}
The FTP CD and PWD commands are not supported by the gateway
(there is no equivalent in the FTAM service and it is too difficult to emulate
at the gateway).
