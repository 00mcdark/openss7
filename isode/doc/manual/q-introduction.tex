% run this through LaTeX with the appropriate wrapper

\chapter {Overview of QUIPU}

\section {Summary}

QUIPU is a Public Domain implementation of the OSI directory as 
specified in CCITT X.500 Recommendations /
ISO 9594 for directory services 
\cite{ISO.Directory,CCITT.Directory}.
It is intended to provide an environment for experimentation and for 
pilots using standardised directory services.  QUIPU is currently aligned to
the ISO IS.  
QUIPU is also aligned to the NIST directory Implementors Guide Version 1, with 
the following exceptions:
\begin{itemize}
\item
QUIPU does not enforce the bounds constraints on attributes, filters or
APDU size.
\item 
T.61 string formatting characters are not rejected.
\item 
If a DN is supplied with no password in an unprotected simple bind,
QUIPU does not always check to see if the DN exists.
If the DSA connected to can
say authoritatively the DN does not exist, the association is rejected.
However, if a chain operation is required to check the DN, the bind is allowed.
\item 
When comparing attributes of UTCtime syntax, if the seconds field is
omitted, QUIPU does not perform the match
correctly (i.e., the seconds field in the attribute values should be ignored,
but is not).
\item 
QUIPU always supplies the optional chaining argument ``originator'' even if
the CommonArgument ``requestor'' is used.
\item
QUIPU always supplies the optional chaining argument ``target'' even if
the base object in the DAP arguments is the same.
\item 
The object class ``without an assigned object identifier'' is not recognised 
unless the ``alias'' object class is also present.
%%% This is what the standard meant to say !

\item
Non Specific Subordinate References are never followed by a QUIPU DSA,
but they are passed on correctly to the client if generated.

\end{itemize}

QUIPU is intended to provide an environment
for experimentation with standardized directory services.
It is used by the ISODE for identification of the location of OSI
applications (including QUIPU) and for provision of white and yellow
page services.
The directory abstract service and DSA abstract service
defined in \cite{CCITT.Directory,ISO.Directory} and their associated 
protocols are supported.
For replication the Internet DSP protocol is used.

Major aspects of the QUIPU implementation are:
\begin{itemize}
\item
Use of  memory structures to provide fast access.
\item
Activity scheduling within the DSA to allow for multiple accesses.
\item
General and flexible searching capabilities.
\item
Extensions to provide access control.
\item
External schema management
\item
Use of the directory to control distributed operations.
\end{itemize}

The current implementation provides a DSA, and a procedural interface to the
directory abstract service, which will enable other applications to use the
directory.
There is also a directory shell interface --- DISH.  This provides full
access to the directory abstract service, using the procedural interface.
standard distributed operations are used with both referrals and chaining
(using the directory system protocol) provided.

A full discussion of the design issues relating to QUIPU can be
found in \cite{QUIPU.Design}.

\section {Pronouncing QUIPU}

The name of the INCA directory is QUIPU.
The official pronunciation of QUIPU takes two syllables:
{\em kwip-ooo}.

\section {Why QUIPU}

QUIPU was originally developed as a part of the INCA project.
The Incas of Peru did not have writing.  Instead, they stored information on
strings, carefully knotted in a specific manner and with coloured thread, and
attached to a larger rope.
These devices were known as {\em Quipus}.
The encoding was obscure, and could only be read by selected trained people:
the {\em Quipucamayocs}.
The Quipu was a key component of Inca society, as it contained information
about property and locations throughout the extensive Inca empire.


\section {Objectives}

\subsection {General Aims}

QUIPU has a number of general aims:

\begin{itemize}
\item To produce an implementation which follows the
emerging OSI directory standards.  

\item Flexibility to enable the system to be used
for experimentation and research into problems relating to directory service.

\item Investigation of distribution and replication.

\item Pilot experimental usage.
\end{itemize}

\subsection {Technical Goals}

The major goals of the QUIPU directory service are:

\begin{itemize}
\item
Full support of the directory access protocol,
directory system protocol, and distributed operations, as 
defined in \cite{CCITT.Directory}.
\item
Support of the majority of the service elements specified in
\cite{CCITT.Directory}.
\item
Ability for interworking with other directory implementations, including
use of
referrals and chaining.
\item
Very full searching and matching capabilities, beyond the minimum
required by \cite{CCITT.Directory}.
\end{itemize}

The following are not goals:

\begin{itemize}
\item
In practice, the memory based approach has led to a quite fast lookup and
searching.
\item
The ability to handle very large volumes of data (e.g., greater than 100~MB
or 1 Million entries per DSA) is not a requirement.
\item
Substantial data robustness is not required: there is no need to employ
complex data backup techniques.
\item
Use (as opposed to provision) of authentication services.

\end{itemize}

\section {Roadmap}

This manual is split into 6 parts.  
You are reading Part I, which is a general introduction.
Part II describes a set of user interfaces (DUAs) developed as part of
QUIPU.
Part III, an administrators guide, describes how to set up both the
DUAs introduced in Part II, and how the install and manage a QUIPU directory
system agent (DSA).
Part IV is a programmers guide which discusses a procedural interface to the
directory for those of you who want to write your own DUAs.
Finally, Part V contains Appendices.

\section {QUIPU Support Address}\label{quipu:support}

If you have any problem installing QUIPU,
following the documentation
or any other QUIPU-related problems, then there are two 
discussion lists.

Comments concerning the operation of 
QUIPU should be addressed to the QUIPU support address:
\[\begin{tabular}{ll}
Internet Mailbox:&      \tt quipu-support@cs.ucl.ac.uk \\
Janet Mailbox:&    \tt quipu-support@uk.ac.ucl.cs \\
X.400 Mailbox:&	   \tt surname = quipu-support \\
& \tt ou = cs \\
& \tt Org = UCL \\
& \tt PRMD = UK.AC \\
& \tt ADMD = Gold 400 \\
& \tt C = GB
\end{tabular}\]
Or, you could look up the mailbox attribute of
\[\begin{tabular}{l}
\tt c=GB \\
\tt o=University College London \\
\tt ou=Computer Science \\
\tt cn=QUIPU-Support
\end{tabular}\]
in the directory.

There is also a discussion list for a general discussion of topics
related to QUIPU; the address is as above, but with ``quipu-support''
replaced by just ``quipu'' (e.g., quipu@cs.ucl.ac.uk).
We suggest that everybody who is intending to run QUIPU should be on this
list, as this will be used to keep you informed of what is happening.
Details of updates will also be sent to this list.

If you would like to be added to the \verb"quipu" discussion 
list, send a message to 
``quipu-request'' (e.g., quipu-request@cs.ucl.ac.uk).

\section {Acknowledgements}

QUIPU was developed at the Department of Computer Science at University
College London, under the {\ae}gis of the INCA\index{INCA} (Integrated Network
Communication Architecture) project, which is project~395
of ESPRIT\index{ESPRIT} (European Strategic Programme for Research into
Information Technology).   The partners of INCA (GEC plc\index{GEC plc},
Olivetti\index{Olivetti}, Nixdorf AG\index{Nixdorf AG},
and Modcomp GmbH\index{Modcomp GmbH}) are acknowledged for releasing this
software into the public domain.

Continued funding of QUIPU as openly available software is provided by the
Joint Network Team (JNT)\index{JNT}.

QUIPU 6.0 was implemented
primarily by Colin Robbins\index{Robbins, Colin J.}
and Alan Turland\index{Turland, Alan},
with considerable help from Marshall Rose\index{Rose, Marshall T.} of
Performance Systems International.
Mike Roe \index{Roe, Mike} implemented the authentication code, and the 
T.61 string handling code.

After QUIPU 6.0, the core development has been from Colin Robbins.
Tim Howes\index{Howes, Tim} of University of Michigan has kindly 
donated the very valuable ``TURBO'' options to QUIPU.
Alan Turland has added the asynchronous DUA interface.

Chris Moore\index{Moore, Christopher W.} of 
the Wollongong Group helped considerably in the early development
of QUIPU, and integration with ISODE.
Simon Walton\index{Walton, Simon} of University College London,
also provided much help in integrating the software with ISODE.

Steve Titcombe\index{Titcombe, Steve} of
University College London did much 
of the early work on DISH, and is now working on the management 
tools.

Paul Sharpe\index{Sharpe, Paul}, of GEC Hirst Research Laboratories
put considerable effort into the early development of SD.
This work was continued at Brunel University by
Andrew Findlay\index{Findlay, Andrew} and Damanjit Mahl\index{Mahl, Damanjit},
who have also developed the POD interface.

Paul Barker\index{Barker, Paul} of
University College London has designed and developed the DE directory
interface, funded by the COSINE project.
Paul also developed the DSA log processing scripts and has provided 
very valuable feedback throughout the project.

Whilst at UCL on secondment from CSIRO, 
Andrew Worsley\index{Worsley, Andrew} put considerable effort into 
enhancing Pepsy to provide all the extra features the QUIPU needed to be able
utilise the tool.

George Michaelson\index{Michaelson, George} of the University of Queensland,
Julian P.~Onions\index{Onions, Julian} of X-Tel Services Ltd,
Andrew Findlay of Brunel University,
Gier Pederson\index{Pederson, Gier} from the University of Oslo,
Petri Jokela\index{Jokela, Petri} of Telecom in Finland,
Juha Hein\"{a}nen\index{Hein\"{a}nen, Juha} of the Tampere University of Technology,
Peter Yee\index{Yee, Peter} from NASA and
Piete Brooks\index{Brooks, Piete} of Cambridge University,
have all run various test
versions of the system, and provided much useful feedback.


