\chapter{DE}
\label{DUA:de}

DE (which stands for {\bf D}irectory {\bf E}nquiries) is a directory user
interface primarily intended to serve as a public access user interface.  It
is a successor to, and borrows something of the style of, the {\em dsc} 
interface released in a previous version.
It is primarily aimed at the novice user, although more sophisticated users
should find that it is flexible enough to answer the majority of queries
they wish to pose.  

DE has more features than those discussed below.  However, the program 
has extensive on-line help as it is envisaged that it will often be used in
environments where neither on-line help or paper documentation will be
available.

\section {Using DE}

\subsection {Starting up}

On invoking the \pgm{de} interface, if the environment 
does not contain a terminal
type, or if the terminal type is ``dumb'', the user is prompted to enter a
terminal type.  If requested, the interface will display a list of the
terminal types supported.  However, DE is designed to work adequately 
without terminal type information, assuming a screen size of 80~x~24.

\subsection {Searching for a Person}

The interface prompts the user for input with the following four questions:

\begin{quote}\footnotesize\begin{verbatim}
Person's name, q to quit, * to list people, ? for help
:- kille
Dept name, * to list depts, <CR> to search all depts, ? for help
:- cs
Organisation name, <CR> to search `ucl', * to list orgs, ? for help
:- 
Country name, <CR> to search `gb', * to list countries, ? for help
:- 
\end{verbatim}\end{quote}

Note from the above example that it is possible to configure the interface 
so that local values are defaulted: RETURN accepts ``ucl'' for organisation,
and ``gb'' for country.  The above query returns a single result which is
displayed thus:

\begin{quote}\footnotesize\begin{verbatim}
United Kingdom
  University College London
    Computer Science
      Steve Kille
        description           Researcher into Distributed 
                              Applications and OSI
                              PP and QUIPU project leader
        telephoneNumber       +44 71-380-7294
        electronic mail       S.Kille@cs.ucl.ac.uk
        favouriteDrink        Pinta - Brakspears
        roomNumber            G24
\end{verbatim}\end{quote}

If several results are found for a single query, the user is asked to select
one from the entries matched.  For example, searching for ``jones'' in
``physics'' at ``UCL'' in ``GB'' produces the following output:

\begin{quote}\footnotesize\begin{verbatim}
United Kingdom
  University College London

Got the following approximate matches.  Please select one from the 
list by typing the number corresponding to the entry you want.

    1 Faculty of Mathematical and Physical Sciences
    2 Medical Physics and Bio-Engineering
    3 Physics and Astronomy
    4 Psychiatry
    5 Psychology
\end{verbatim}\end{quote}

Selecting ``Physics and Astronomy'' by simply typing the number 3, the
search continues, and the following is displayed:

\begin{quote}\footnotesize\begin{verbatim}
United Kingdom
  University College London
    Physics and Astronomy

Got the following approximate matches.  Please select one from the
list by typing the number corresponding to the entry you want.

     1 C L Jones     +44 71-380-7139
     2 G O Jones     +44 71-387-7050 x3468  geraint.jones@ucl.ac.uk
     3 P S Jones     +44 71-387-7050 x3483
     4 T W Jones     +44 71-380-7150
\end{verbatim}\end{quote}

In this condensed format, telephone and email information is displayed.

\subsection {Searching for other information}

Information for organisations can be found by specifying null entries for 
the person and department.

Information for departments can be found by specifying null input for
the person field.

Information about rooms and roles can be found as well as for people by, for
example, entering ``secretary'' in the person's name field.

\subsection{Interrupting}

If the user wishes to abandon a query or correct the input of a query (maybe
the user has mis-typed a name), {\em control-C} resets the interface 
so that it is
waiting for a fresh query.

\subsection{Quitting}

Type ``q'' at the prompt for a person's name.

\section {Configuration of DE}

As DE is intended as a public access dua, it is only configurable on a
system-wide basis.
DE installs help files and the \file{detailor} file into a directory 
called \file{de/} under \verb+ISODE+'s ETCDIR.

The \file{detailor} file 
contains a number of tailorable variables, of which the
following are mandatory:

\begin{description}

\item [\verb+dsa\_address+:] This is the address of the access point DSA.

\begin{quote}\small\begin{verbatim}
dsa_address:MY-TCP=2005
\end{verbatim}\end{quote}

\item [\verb+username+:] This is the username which the DUA binds to the directory
with.  It is not strictly mandatory, but you are strongly encouraged to set
this up.  It will help you to see who is connecting to the DSA.

\begin{quote}\footnotesize\begin{verbatim}
username:@c=GB@o=X-Tel Services Ltd@cn=Directory Enquiries
\end{verbatim}\end{quote}

\end{description}

You will almost certainly want to set at least some of these to suit your 
local system:

\begin{description}

\item [\verb+default\_country+:]  This is the name of the country to search by
default: e.g., ``GB''.

\begin{quote}\small\begin{verbatim}
default_country:gb
\end{verbatim}\end{quote}


\item [\verb+default\_org+:]  This is the name of the organisation to search by
default: e.g., ``University College London''

\begin{quote}\small\begin{verbatim}
default_org:University College London
\end{verbatim}\end{quote}

\item [\verb+default\_dept+:] This is the name of the department (organisational
unit) to search by default: e.g., ``Computing''.  This will usually be null
for public public access duas.

\begin{quote}\small\begin{verbatim}
default_dept:
\end{verbatim}\end{quote}

\end{description}

The following configuration options all concern the display of attributes.
The settings in the \file{detailor} file will probably be OK initially.

\begin{description}

\item [\verb+commonatt+:]  These attributes are displayed whatever type of object
is being searched for, be it an organisation, a department, or a person.

\begin{quote}\small\begin{verbatim}
commonatt:telephoneNumber
commonatt:facsimileTelephoneNumber
\end{verbatim}\end{quote}

\item [\verb+orgatt+:]  These attributes are displayed (as well as the common
attributes --- see above) if an entry for an organisation is displayed.

\begin{quote}\small\begin{verbatim}
orgatt:telexNumber
\end{verbatim}\end{quote}

\item [\verb+ouatt+:]  These attributes are displayed (as well as the common
attributes --- see above) if an entry for an organisational unit (department)
is displayed.

\begin{quote}\small\begin{verbatim}
ouatt:telexNumber
\end{verbatim}\end{quote}

\item [\verb+prratt+:]  These attributes are displayed (as well as the common
attributes --- see above) if an entry for a person, room or role is displayed.

\begin{quote}\small\begin{verbatim}
prratt:rfc822Mailbox
prratt:roomNumber
\end{verbatim}\end{quote}

\item [\verb+mapattname+:]  This attribute allows for meaningful attribute names to
be displayed to the user.  The attribute names in the quipu
oidtables may be mapped onto more user-friendly names.  This allows for 
language independence.  

\begin{quote}\small\begin{verbatim}
mapattname:facsimileTelephoneNumber fax
mapattname:rfc822Mailbox electronic mail
\end{verbatim}\end{quote}

\item [\verb+mapphone+:]  This allows for the mapping of international format phone
numbers into a local format.  It is thus possible to display local phone
numbers as extension numbers only and phone numbers in the same country
correctly prefixed and without the country code.

\begin{quote}\small\begin{verbatim}
mapphone:+44 71-380-:
mapphone:+44 71-387- 7050 x:
mapphone:+44 :0
\end{verbatim}\end{quote}

\item [\verb+greybook+:]  In the UK, we use big-endian domains in mail names.  By
setting this variable on, it is possible to display email addresses in this
order rather than the default little-endian order.

\begin{quote}\small\begin{verbatim}
greyBook:on
\end{verbatim}\end{quote}

\item [\verb+country+:]  This allows for the mapping of the 2 letter ISO country
codes (such as GB and FR) onto locally meaningful strings such as, for
english speakers, Great Britain and Germany.

\begin{quote}\small\begin{verbatim}
country:AU Australia
country:AT Austria
country:BE Belgium
\end{verbatim}\end{quote}

\end{description}

There are a number of miscellaneous variables which may be set.

\begin{description}

\item [\verb+maxPersons+:]  If a lot of matches are found, DE will display the
matches in a short form, showing email address and telephone number only.
Otherwise full entry details are displayed.  This variable allows the number
of entries which will be displayed in full to be set --- the default is 3.

\begin{quote}\small\begin{verbatim}
maxPersons:2
\end{verbatim}\end{quote}

\item [\verb+inverseVideo+:]  Prompts are by default shown in inverse video.  Unset
this variable to turn this off.

\begin{quote}\small\begin{verbatim}
inverseVideo:on
\end{verbatim}\end{quote}

\item [\verb+delogfile+:]  Searches are by default are logged to the file
\file{de.log}
in \verb+ISODE+s LOGDIR.  They can be directed elsewhere by using this
variable.

\begin{quote}\small\begin{verbatim}
delogfile:/tmp/delogfile
\end{verbatim}\end{quote}

\item [\verb+remoteAlarmTime+:]  A remote search is one where
the country and organisation name searched for not the same as
the defaults.  If the search has not completed within a configurable number
of seconds, a message is displayed warning the user that all may not be well.
The default setting is 30 seconds.
The search, however, continues until it returns or is interrupted by the
user.

\begin{quote}\small\begin{verbatim}
remoteAlarmTime:30
\end{verbatim}\end{quote}

\item [\verb+localAlarmTime+:]  As for {\em remoteAlarmTime}, except for local searches.
The default setting is 15 seconds.

\begin{quote}\small\begin{verbatim}
localAlarmTime:15
\end{verbatim}\end{quote}

\end{description}
