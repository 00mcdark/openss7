% run this through LaTeX with the appropriate wrapper

\chapter{FRED}
\label{DUA:fred}
FRED is a DUA optimised for White Pages queries; it is actually
implemented as an interface to DISH, hence the name FRED --- {\bf FR}ont 
{\bf E}nd to {\bf D}ISH.

\section	{Giving Commands to FRED}\label{fred:commands}
After invoking \pgm{fred}
you are prompted with \verb"fred> " indicating that FRED is ready.

If FRED is invoked interactively,
it will look for a file in your home directory called \file{.fredrc}.
It will execute the commands contained in this file just as if you had typed
them directly to FRED.
Following this,
you are given the \verb"fred>" prompt.

\section	{Let Your Fingers Do the Walking}
Although FRED has several commands,
the most interesting command is \verb"whois",
which performs a white pages query.

Let's begin with some simple examples and introduce the other commands along
the way.
If you already know the handle of the person you are interested in finding out
about,
just give the handle:
\begin{quote}\smaller\begin{verbatim}
fred> whois @c=US@cn=Manager
Manager (1)

Handle:        @c=US@cn=Manager
\end{verbatim}\end{quote}

\subsection	{The Alias Command}
Since handles are long strings,
FRED will automatically maintain a list of aliases of the entries you
have seen in the current session.
The alias is always a number.
When an entry is displayed,
it appears on the first line in parenthesis after the name of the object.
In the example above,
the alias is \verb"1".

To find out what aliases are currently defined,
use the \verb"alias" command:
\begin{quote}\smaller\begin{verbatim}
fred> alias
1    @c=US@cn=Manager
\end{verbatim}\end{quote}
Thus,
the previous \verb"whois" command could have been shortened to simply:
\begin{quote}\small\begin{verbatim}
fred> whois !1
Manager (1)

Handle:        @c=US@cn=Manager
\end{verbatim}\end{quote}
Each time you invoke FRED,
its list of aliases is empty.
If there are few handles which you use often,
you might wish to define them in your \file{.fredrc} file,
e.g.,
\begin{quote}\small\begin{verbatim}
alias "@c=US@o=DMD@cn=Manager"
\end{verbatim}\end{quote}
Of course,
the ordering of aliases is important.
FRED will start numbering from~1 starting with the first \verb"alias"
command.

\subsection	{Back to Searching}
Suppose however,
that you don't know the handle for the person.
In this case,
you need to specify some search parameters.
Logically,
the first step is to ascertain the organisation which the person is likely to
be associated with, e.g.,
``Performance Systems International''.
This is done as:
\begin{quote}\footnotesize\begin{verbatim}
fred> whois organization psi
Performance Systems International (1)  +1 800-836-0400 (Operations)
     aka: PSI

PSI Inc.
  Reston International Center
  11800 Sunrise Valley Drive
  Suite 1100
  Reston, VA 22091
  US

PSI Inc.
  5201 Great American Parkway
  Suite 3106
  Santa Clara, CA 95054
  US

PSI Inc.
  165 Jordan Road
  Troy, NY 12180
  US

Telephone: +1 800-836-0400 (Operations)
           +1 800-82PSI82 (Sales)
           +1 703-620-6651 (Corporate/Reston Office)
           +1 518-283-8860 (Troy Office)
           +1 408-562-6222 (Santa Clara Office)
FAX:       +1 703-620-4586
           +1 518-283-8904
           +1 408-562-6223

value-added provider of networking services

Locality:    Reston, Virginia

Name:     Performance Systems International, US (1)
Modified: Mon Jul 30 05:18:24 1990
      by: Manager, Performance Systems International,
            US (2)
\end{verbatim}\end{quote}
Second,
to search for a particular person,
you might use:
\begin{quote}\small\begin{verbatim}
fred> whois rose -area 2
Marshall Rose (3)             mrose@psi.com
     aka: mtr
     aka: Marshall T. Rose

Principal Scientist
PSI, Inc.
  POB 391776
  Mountain View, CA 94039
  US

PSI, Inc.
  5201 Great American Parkway
  Suite 3106
  Santa Clara, CA 95054
  US

Telephone: +1 415-961-3380
           +1 408-562-6222 x6221
FAX:       +1 415-961-3282
           +1 408-562-6223

Mailbox information:
  internet: mrose@psi.com
  internet: mrose@cheetah.ca.psi.com

Principal Implementor of the ISO Development Environment

Beleaguered Manager of the PSI White Pages Pilot Project

A savvyNerd according to noSauce

Locality:    Santa Clara, California

Drinks:       Iced Tea (and lots of it...)

Name:     Marshall Rose, Mountain View,
            Research and Development,
            Performance Systems International,
            US (3)
Modified: Mon Sep 24 14:43:36 1990
      by: Manager, US (5)
\end{verbatim}\end{quote}
Note the use of the alias \verb"2".
The command could also have been:
\begin{quote}\small\begin{verbatim}
fred> whois rose -area "@c=US@o=Performance Systems..."
    ...
\end{verbatim}\end{quote}
Double-quotes are used so that the DN appears as a single token to FRED.

Of course,
this two-step process,
whilst logical, is tedious.
Thus, you can combine them like this:
\begin{quote}\small\begin{verbatim}
fred> whois rose -org psi
    ...
\end{verbatim}\end{quote}
which says to look for any organisations with ``psi'' in its name.
Then, for each one,
look for something called ``rose''.

\subsection	{The Area Command}
Suppose
you want information on several persons belonging to an organisation.
You can use the \verb"area" command,
by itself,
to tell FRED where to search for subsequent commands.
For example,
\begin{quote}\small\begin{verbatim}
fred> area "@c=US@o=Performance Systems International"
\end{verbatim}\end{quote}
or simply
\begin{quote}\small\begin{verbatim}
fred> area 2
\end{verbatim}\end{quote}
Both tell FRED the default area used by the \verb"whois" command.
Of course,
you can still use the \switch"area" area with the \verb"whois" command to
override the default area.
Thus,
\begin{quote}\small\begin{verbatim}
fred> whois alan -area "@c=US@o=Columbia University"
\end{verbatim}\end{quote}
will do what you expect.

If you use the \verb"area" command without any arguments,
FRED will tell you what its default area is:
\begin{quote}\small\begin{verbatim}
fred> area
@c=US@o=Yoyodyne
\end{verbatim}\end{quote}
This indicates the default area for all commands,
{\em including\/} any subsequent \verb"area" commands.
Thus, issuing

\begin{quote}\small\begin{verbatim}
fred> area @c=US@o=Yoyodyne
@c=US@o=Yoyodyne

fred> area ou=Research
@c=US@o=Yoyodyne@ou=Research
\end{verbatim}\end{quote}
is equivalent to
\begin{quote}\small\begin{verbatim}
fred> area @c=US@o=Yoyodyne@ou=Research
@c=US@o=Yoyodyne@ou=Research
\end{verbatim}\end{quote}
because a leading ``\verb"@"''-sign was not used before \verb"ou=Research".

As you might expect,
there is a special string ``\verb".."'' which may be used to move up one level:
\begin{quote}\small\begin{verbatim}
fred> area ..
@c=US@o=Yoyodyne
\end{verbatim}\end{quote}
Combinations are possible as well,
such as
\begin{quote}\small\begin{verbatim}
fred> area ..@"o=Performance Systems International"
@c=US@o=Performance Systems International
\end{verbatim}\end{quote}
which moves up a level and then down to\\
\verb"o=Performance Systems International"

\subsection	{Getting Help}
For a brief summary of FRED commands,
type:
\begin{quote}\small\begin{verbatim}
fred> help ?
\end{verbatim}\end{quote}
This will list the commands that FRED knows about
along with a one-line summary of their function.

For help on a particular command,
type the name of the command followed by \switch"help",
e.g.,
\begin{quote}\small\begin{verbatim}
fred> alias -help
\end{verbatim}\end{quote}

If you need more help,
try
\begin{quote}\small\begin{verbatim}
fred> manual
\end{verbatim}\end{quote}
which is the same as
\begin{quote}\small\begin{verbatim}
% man fred
\end{verbatim}\end{quote}
from the shell.

\subsection	{Quitting}
To terminate FRED,
simply use:
\begin{quote}\small\begin{verbatim}
fred> quit
\end{verbatim}\end{quote}

\section {Advanced Usage}

This chapter has given a very brief overview of the basic FRED commands.
For full details you should consult~\cite{PSI.Fred}, which tells you how
to make more complex search requests, edit your own entry, and how to use
FRED to compose mail addresses using the \MH/ mail system.
