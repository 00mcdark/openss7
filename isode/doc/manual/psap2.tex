% run this through LaTeX with the appropriate wrapper

\chapter	{Presentation Services}\label{libpsap2}
The \man libpsap2(3n) library implements the presentation service.
The kernel subset of the ISO specification is currently implemented.
That is,
the library supports whatever session requirements the user wishes to
employ,
negotiates presentation contexts on connection establishments,
and utilizes abstract transfer notations to transmit data structures in a
machine-independent manner.

As with most models of OSI services,
the underlying assumption is one of a symmetric, asynchronous environment.
That is,
although peers exist at a given layer,
one does not necessary view a peer as either a client or a server.
Further,
the service provider may generate events for the service user without the
latter entity triggering the actions which led to the event.
For example,
in a synchronous environment,
an indication that data has arrived usually occurs only when the service user
asks the service provider to read data;
in an asynchronous environment,
the service provider may interrupt the service user at any time to announce
the arrival of data.

The \verb"psap2" module in this release initially uses a client/server
paradigm to start communications.
Once the connection is established,
a symmetric view is taken.
In addition,
initially the interface is synchronous;
however once the connection is established,
an asynchronous interface may be selected.

All of the routines in the \man libpsap2(3n) library are integer-valued.
They return the manifest constant \verb"OK" on success,
or \verb"NOTOK" otherwise.

\section	{Warning}
Please read the following important message.
\[\fbox{\begin{tabular}{lp{0.8\textwidth}}
\bf NOTE:&	Readers of this chapter should have an intimate understanding
		of the OSI presentation and session services.
		It is not the intent of this
		chapter to present a tutorial on these services, so novice
		users will suffer greatly if they choose to read further.
\end{tabular}}\]

\section	{Addresses}\label{psap:addresses}
Addresses at the presentation service access point are represented by the
\verb"PSAPaddr" structure.
\begin{quote}\index{PSAPaddr}\small\begin{verbatim}
struct PSAPaddr {
    struct SSAPaddr pa_addr;

#define PSSIZE  64
    int     pa_selectlen;
    char    pa_selector[PSSIZE];
};
#define NULLPA  ((struct PSAPaddr *) 0)
\end{verbatim}\end{quote}
This structure contains two elements:
\begin{describe}
\item[\verb"pa\_addr":]	the session address,
as described in Section~\ref{ssap:addresses} on page~\pageref{ssap:addresses};
and,

\item[\verb"pa\_selector"/\verb"pa\_selectlen":] the presentation selector.
\end{describe}

In Figure~\ref{getAPprovider},
an example of how one constructs the PSAP address for the LOOP provider on
host \verb"RemoteHost" is presented.
The routine \verb"is2paddr" takes a host and service,
and then consults the \man isoentities(5) file described in
Chapter~\ref{isoentities} of \volone/ to construct a presentation
address.
\begin{quote}\index{is2paddr}\small\begin{verbatim}
struct PSAPaddr *is2paddr (host, service, is)
char   *host,
       *service;
struct isoservent *is;
\end{verbatim}\end{quote}
\tagrind[tp]{grind5a-1}{Constructing the PSAP address for the LOOP provider}%
	{getAPprovider}

\subsection	{Calling Address}
Certain users of the presentation service
(e.g., the reliable transfer service)
need to know the name of the local host when they initiate a connection.
The routine \verb"PLocalHostName" has been provided for this reason.
\begin{quote}\index{PLocalHostName}\small\begin{verbatim}
char   *PLocalHostName ()
\end{verbatim}\end{quote}

\section	{Connection Establishment}
Until the connection has been fully established,
the implementation distinguishes between clients and servers,
which are more properly referred to as {\em initiators\/} and
{\em responders}, to use OSI terminology.

\subsection	{Connection Negotiation}
From the user's perspective,
there are several parameters which are negotiated by the presentation providers
during connection establishment.
Suggestions as to the values of some of these parameters are made by the user.

\subsubsection	{Session Parameters}
Consult Section~\ref{ssap:negotiation} for a list of session parameters which
are negotiated during connection establishment.

\subsubsection	{Presentation Contexts}\label{psap:context}
A {\em presentation context\/} is a binding between an abstract syntax
notation and an abstract transfer notation on a
presentation connection,
and is denoted by an integer value termed the context identifier.
The abstract syntax notation describes,
to the users of the presentation service,
the data structures being exchanged;
the abstract transfer notation describes,
to the providers of the presentation services,
the method for encoding those data structures in a machine-independent fashion.
Hence the abstract syntax notation is negotiated by the users of the
presentation service,
while the abstract transfer notation is negotiated by the providers of that
service.

When a connection is established,
the initiator suggests zero or more presentation contexts,
specifying a context identifier (an odd-valued integer),
and the abstract syntax
(a pointer to an object identifier, see Section~\ref{psap:oid} of \volone/).
The provider selects the abstract transfer notation
(in the current implementation, this is always ASN.1\cite{ISO.PP.Encoding}).
When the a {\sf P-CONNECT.INDICATION\/} event is given to the responder,
in addition 
indicating the context identifier and abstract syntax information,
the provider also indicates if it is willing to support this presentation
context.
If so, the responder decides if it will accept or reject the context.
This information is then propagated back to the initiator with
the {\sf P-CONNECT.CONFIRMATION\/} indication.

\subsection	{Server Initialization}
The \man tsapd(8c) daemon,
upon accepting a connection from an initiating host,
consults the ISO services database to determine which program
on the local system implements the desired SSAP entity.
In the case of the presentation service,
the \pgm{tsapd} program contains the bootstrap for the presentation provider.
The daemon will again consult the ISO services database to determine which
program on the system implements the desired PSAP entity.

Once the program has been ascertained,
the daemon runs the program with any arguments listed in the database.
In addition,
it appends some {\em magic arguments\/} to the argument vector.
Hence,
the very first action performed by the responder is to re-capture the PSAP
state contained in the magic arguments.
This is done by calling the routine \verb"PInit",
which on a successful return,
is equivalent to a {\sf P-CONNECT.INDICATION\/} from the presentation service
provider.
\begin{quote}\index{PInit}\small\begin{verbatim}
int     PInit (vecp, vec, ps, pi)
int     vecp;
char  **vec;
struct PSAPstart *ps;
struct PSAPindication *pi;
\end{verbatim}\end{quote}
The parameters to this procedure are:
\begin{describe}
\item[\verb"vecp":] the length of the argument vector;

\item[\verb"vec":] the argument vector;

\item[\verb"ps":] a pointer to a \verb"PSAPstart" structure, which is updated
only if the call succeeds;
and,

\item[\verb"pi":] a pointer to a \verb"PSAPindication" structure, which is
updated only if the call fails.
\end{describe}
If \verb"PInit" is successful,
it returns information in the \verb"ps" parameter,
which is a pointer to a \verb"PSAPstart" structure.
\begin{quote}\index{PSAPstart}\small\begin{verbatim}
struct PSAPstart {
    int     ps_sd;

    struct PSAPaddr ps_calling;
    struct PSAPaddr ps_called;

    struct PSAPctxlist ps_ctxlist;

    OID     ps_defctx;
    int     ps_defctxresult;

    int     ps_prequirements;

    int     ps_srequirements;
    int     ps_settings;
    long    ps_isn;

    struct SSAPref  ps_connect;

    int     ps_ssdusize;

    struct QOStype ps_qos;

#define NPDATA         10
    int     ps_ninfo;
    PE      ps_info[NPDATA];
};
\end{verbatim}\end{quote}
The elements of this structure are:\label{PSAPstart}
\begin{describe}
\item[\verb"ps\_sd":] the presentation-descriptor to be used to reference this
connection;

\item[\verb"ps\_calling":] the address of the peer initiating the connection;

\item[\verb"ps\_called":] the address of the peer being asked to respond;

\item[\verb"ps\_ctxlist":] the proposed list of presentation contexts;

\item[\verb"ps\_defctx"/\verb"ps\_defctxresult":] the default context
for the presentation connection
(and the presentation provider's response);

\item[\verb"ps\_prequirements":] the proposed presentation requirements;

\item[\verb"ps\_srequirements":] the proposed session requirements;

\item[\verb"ps\_settings":] the initial token settings;

\item[\verb"ps\_isn":] the initial serial-number;

\item[\verb"ps\_connect":] the connection identifier (a.k.a. SSAP reference)
used by the initiator;

\item[\verb"ps\_ssdusize":] the largest atomic SSDU size that can be used on
the connection (see the note on page~\pageref{SSDU:atomic});

\item[\verb"ps\_qos":] the quality of service on the connection
(see Section~\ref{tsap:qos});
and,

\item[\verb"ps\_info"/\verb"ps\_ninfo":] an array of initial data
(and the number of elements in that array).
\end{describe}
Note that the \verb"ps_info" element is allocated via \man malloc(3) and
should be released using the \verb"PSFREE" macro when no longer referenced.
The \verb"PSFREE" macro behaves as if it was defined as:
\begin{quote}\index{PSFREE}\small\begin{verbatim}
void    PSFREE (ps)
struct PSAPstart *ps;
\end{verbatim}\end{quote}
The macro frees only the data allocated by \verb"PInit",
and not the \verb"PSAPstart" structure itself.
Further,
\verb"PSFREE" should be called only if the call to the \verb"PInit"
routine returned \verb"OK".

The \verb"ps_connect" element is a \verb"SSAPref" structure,
which is passed transparently by the presentation service.
Consult the description on page~\pageref{SSAPref}.
There are two routines of interest in the \man libpsap2(3n) that deal with
these structures:
The \verb"addr2ref" routine takes a string (presumably a hostname),
determines the current UT time,
and returns a pointer to a \verb"SSAPref" structure appropriately initialized
to denote this information.
\begin{quote}\index{addr2ref}\small\begin{verbatim}
struct SSAPref *addr2ref (addr)
char   *addr;
\end{verbatim}\end{quote}
This routine might fail if it is unable to allocate a small amount of
memory.
In this event, it returns the manifest constant \verb"NULL".
The routine \verb"sprintref" can be used to return a null-terminated string
describing the SSAP reference.
\begin{quote}\index{sprintref}\small\begin{verbatim}
char   *sprintref (sr)
struct SSAPref *sr;
\end{verbatim}\end{quote}

The \verb"ps_ctxlist" element is a \verb"PSAPctxlist" structure,
which describes the presentation contexts the initiator wishes to use.
\begin{quote}\index{PSAPctxlist}\small\begin{verbatim}
struct PSAPctxlist {
    int     pc_nctx;

#define NPCTX   5

    struct PSAPcontext {
        int     pc_id;
        OID     pc_asn;
        OID     pc_atn;
        int     pc_result;
    }       pc_ctx[NPCTX];
};
#define NULLPC  ((struct PSAPctxlist *) 0)
\end{verbatim}\end{quote}
The elements of this structure are:\label{PSAPctxlist}
\begin{describe}
\item[\verb"pc\_ctx"/\verb"pc\_nctx":] the presentation contexts described
(and the number of contexts which may not exceed \verb"NPCTX" elements).
For each presentation context:
    \begin{describe}
    \item[\verb"pc\_id":] the identifier (or handle) for the context;

    \item[\verb"pc\_asn":] the abstract syntax notation to be used on the
	context;

    \item[\verb"pc\_atn":] the transfer syntax notation to be used on the
	context (this field is usually, \verb"NULLOID", only the initiator,
	when it wishes to inform the presentation service of the transfer
	syntax to use, is permitted to specify this); and,

    \item[\verb"pc\_result":] the presentation provider's response
	(codes are listed in Table~\ref{PSAPreasons}).
    \end{describe}
\end{describe}

If the call to the \verb"PInit" routine is not successful,
then a {\sf P-P-ABORT.IN\-DI\-CA\-TION\/} event is simulated,
and the relevant information is returned in a \verb"PSAPindication"
structure.\label{PSAPindication}
\begin{quote}\index{PSAPindication}\small\begin{verbatim}
struct PSAPindication {
    int     pi_type;
#define PI_DATA         0x00
#define PI_TOKEN        0x01
#define PI_SYNC         0x02
#define PI_ACTIVITY     0x03
#define PI_REPORT       0x04
#define PI_FINISH       0x05
#define PI_ABORT        0x06

    union {
        struct PSAPdata pi_un_data;
        struct PSAPtoken pi_un_token;
        struct PSAPsync pi_un_sync;
        struct PSAPactivity pi_un_activity;
        struct PSAPreport pi_un_report;
        struct PSAPfinish pi_un_finish;
        struct PSAPabort pi_un_abort;
    }   pi_un;
#define pi_data         pi_un.pi_un_data
#define pi_token        pi_un.pi_un_token
#define pi_sync         pi_un.pi_un_sync
#define pi_activity     pi_un.pi_un_activity
#define pi_report       pi_un.pi_un_report
#define pi_finish       pi_un.pi_un_finish
#define pi_abort        pi_un.pi_un_abort
};
\end{verbatim}\end{quote}
As shown, this structure is really a discriminated union
(a structure with a tag element followed by a union).
Hence, on a failure return,
one first coerces a pointer to the \verb"PSAPabort" structure contained
therein,
and then consults the elements of that structure.
\begin{quote}\index{PSAPabort}\small\begin{verbatim}
struct PSAPabort {
    int     pa_peer;

    int     pa_reason;
    int     pa_ppdu;


    int     pa_ninfo;
    char    pa_info;

#define PA_SIZE        512
    int     pa_cc;
    char    pa_data[PA_SIZE];
};
\end{verbatim}\end{quote}
The elements of a \verb"PSAPabort" structure are:
\begin{describe}
\item[\verb"pa\_peer":] if set, indicates that a user-initiated abort occurred
(a {\sf P-U-ABORT.INDICATION\/} event);
if not set, indicates that a provider-initiated abort occurred
(a {\sf P-P-ABORT.INDICATION\/} event);

\item[\verb"pa\_reason":] the reason for the provider-initiated
abort (codes are listed in Table~\ref{PSAPreasons});

\item[\verb"pa\_ppdu":] the type of presentation protocol data unit which
triggered the provider-initiated abort
(codes are listed in Table~\ref{PSAPppdus});

\item[\verb"pa\_data"/\verb"pa\_cc":] a provider-generated diagnostic string
(and the length of that string);
and,

\item[\verb"pa\_info"/\verb"pa\_ninfo":] an array of user data,
and the number of elements in that array (if \verb"pa_peer" is set).
\end{describe}
\tagtable[tp]{5a-1}{PSAP Failure Codes}{PSAPreasons}
\tagtable[tp]{5a-2}{PSAP PPDU Codes}{PSAPppdus}
Note that the \verb"pa_info" element is allocated via \man malloc(3) and
should be released using the \verb"PAFREE" macro  when no longer referenced.
The \verb"PAFREE" macro behaves as if it was defined as:
\begin{quote}\index{PAFREE}\small\begin{verbatim}
void    PAFREE (pa)
struct PSAPabort *pa;
\end{verbatim}\end{quote}
The macro frees only the data allocated in the \verb"PSAPabort" structure
and not the structure itself.

After examining the information returned by \verb"PInit" on a successful call
(and possibly after examining the argument vector),
the responder should either accept or reject the connection.
For either response,
the responder should use
the \verb"PConnResponse" routine
(which corresponds to the {\sf P-CONNECT.RESPONSE\/} action).
\begin{quote}\index{PConnResponse}\small\begin{verbatim}
int     PConnResponse (sd, status, responding, ctxlist,
                defctxresult, prequirements, srequirements,
                isn, settings, ref, data, ndata, pi)
int     sd;
struct PSAPaddr *responding;
int     status,
        prequirements,
        srequirements,
        settings,
        ndata;
long	isn;
struct PSAPctxlist *ctxlist;
int     defctxresult;
struct SSAPref *ref;
PE     *data;
struct PSAPindication *pi;
\end{verbatim}\end{quote}
The parameters to this procedure are:\label{PConnResponse}
\begin{describe}
\item[\verb"sd":] the presentation-descriptor;

\item[\verb"result":] the acceptance indicator
(either \verb"PC_ACCEPT" if the connection is to be accepted,
or \verb"PC_REJECTED" otherwise);

\item[\verb"responding":] the PSAP address of the responder (defaulting to the
called address, if not present);

\item[\verb"ctxlist":] the responder's decision as to each of the presentation
contexts suggested
(for each proposed context,
if the \verb"pc_result" element supplied by the provider is \verb"PC_ACCEPT",
which indicates that the provider is willing to support it,
the user may supply either \verb"PC_ACCEPT" or the value \verb"PC_REJECTED");

\item[\verb"defctxresult":] the response to the default context
(if the presentation provider responded with \verb"PC_ACCEPT",
the user may supply either \verb"PC_ACCEPT" or
\verb"PC_REJECTED");

\item[\verb"prequirements":] the responder's presentation requirements;

\item[\verb"srequirements":] the responder's session requirements;

\item[\verb"isn":] the initial serial-number;

\item[\verb"settings":] the initial token settings;

\item[\verb"ref":]  the connection identifier used by the responder
(consult page~\pageref{SSAPref} for a description of the \verb"SSAPref"
structure);

\item[\verb"data"/\verb"ndata":] an array of initial data
(and the number of elements in that array,
consult the warning on page~\pageref{PSAPdata});
and,

\item[\verb"pi":] a pointer to a \verb"PSAPindication" structure, which is
updated only if the call fails.
\end{describe}
If the call to \verb"PConnResponse" is successful,
and if the \verb"result" parameter is set to \verb"PC_ACCEPT",
then connection establishment has completed
and the users of the presentation service now operate as symmetric peers.
If the call is successful,
but the \verb"result" parameter is \verb"PC_REJECTED" instead,
then the connection has been rejected and the responder may exit.
Otherwise, if the call fails and the reason is not an interface error
(see Table~\ref{PSAPreasons} on page~\pageref{PSAPreasons}),
then the connection is closed.

Note that when the responder rejects the connection,
it need only supply meaningful values for the \verb"sd", \verb"status",
\verb"defctxresult", and \verb"pi" parameters.

\subsection	{Client Initialization}
A program wishing to connect to another user of presentation services calls the
\verb"PConnRequest" routine,
which corresponds to the {\sf P-CONNECT.RE\-QUEST\/} action.
\begin{quote}\index{PConnRequest}\small\begin{verbatim}
int     PConnRequest (calling, called, ctxlist, defctxname,
                prequirements, srequirements, isn, settings,
                ref, data, ndata, qos, pc, pi)
struct PSAPaddr *calling,
                *called;
int     prequirements,
        srequirements,
        settings,
        ndata;
long	isn;
struct PSAPctxlist *ctxlist;
OID     defctxname;
struct SSAPref *ref;
PE    *data;
struct QOStype *qos;
struct PSAPconnect *pc;
struct PSAPindication *pi;
\end{verbatim}\end{quote}
The parameters to this procedure are:\label{PConnRequest}
\begin{describe}
\item[\verb"calling":] the PSAP address of the initiator
(use the manifest constant verb"NULLPA" if this is not unimportant);

\item[\verb"called":] the PSAP address of the responder;

\item[\verb"ctxlist":] the list of proposed presentation contexts
(only the \verb"pc_id", \verb"pc_asn", and optionally the \verb"pc_atn"
elements should be filled in);

\item[\verb"defctxname":] the proposed default contexts;

\item[\verb"prequirements":] the presentation requirements;

\item[\verb"srequirements":] the session requirements;

\item[\verb"isn":] the initial serial-number;

\item[\verb"settings":] the initial token settings;

\item[\verb"ref":] the connection identifier used by the initiator
(consult page~\pageref{SSAPref} for a description of the \verb"SSAPref"
structure);

\item[\verb"data"/\verb"ndata":] an array of initial data
(and the number of elements in that array,
consult the warning on page~\pageref{PSAPdata});

\item[\verb"qos":] the quality of service on the connection
(see Section~\ref{tsap:qos});

\item[\verb"pc":] a pointer to a \verb"PSAPconnect" structure, which is
updated only if the call succeeds;
and,

\item[\verb"pi":] a pointer to a \verb"PSAPindication" structure, which is
updated only if the call fails.
\end{describe}
If the call to \verb"PConnRequest" is successful
(a successful return corresponds to a {\sf P-CONNECT.CONFIRMATION\/} event),
then information is returned in the \verb"pc" parameter,
which is a pointer to a \verb"PSAPconnect" structure.
\begin{quote}\index{PSAPconnect}\small\begin{verbatim}
struct PSAPconnect {
    int     pc_sd;

    struct PSAPaddr pc_responding;

    struct PSAPctxlsit pc_ctxlist;

    int     pc_defctxresult;

    int     pc_prequirements;

    int     pc_srequirements;
    int     pc_settings;
    int     pc_please;
    long    pc_isn;

    struct SSAPref pc_connect;

    int     pc_ssdusize;

    struct QOStype pc_qos;
    int     pc_result;

    struct SSAPref  pc_connect;

#define PC_SIZE         512
    int     pc_cc;
    char    pc_data[PC_SIZE];
};
\end{verbatim}\end{quote}
The elements of this structure are:\label{PSAPconnect}
\begin{describe}
\item[\verb"pc\_sd":] the presentation-descriptor to be used to reference this
connection;

\item[\verb"pc\_responding":] the responding peer's address
(which is the same as the \verb"called" parameter given to
\verb"SConnRequest");

\item[\verb"pc\_ctxlist":] the (negotiated) list of presentation contexts;

\item[\verb"pc\_defctxresult":] the response to the proposed default context;

\item[\verb"pc\_prequirements":] the (negotiated) presentation requirements;

\item[\verb"pc\_srequirements":] the (negotiated) session requirements;

\item[\verb"pc\_settings":] the (negotiated) initial token settings;

\item[\verb"pc\_please":] the tokens which the responder wants to own
(if any);

\item[\verb"pc\_isn":] the (negotiated) initial serial-number;

\item[\verb"pc\_connect":] the connection identifier used by the responder
(consult page~\pageref{SSAPref} for a description of the \verb"SSAPref"
structure);

\item[\verb"pc\_ssdusize":] the largest atomic SSDU size that can be used on
the connection (see the note on page~\pageref{SSDU:atomic});

\item[\verb"pc\_qos":] the quality of service on the connection
(see Section~\ref{tsap:qos});
and,

\item[\verb"pc\_result":] the connection response;

\item[\verb"pc\_info"/\verb"pc\_ninfo":] an array of initial data,
and the number of elements in that array.
\end{describe}
If the call to \verb"PConnRequest" is successful,
and the \verb"pc_result" element is set to \verb"PC_ACCEPT",
then connection establishment has completed and the users of the presentation
service now operate as symmetric peers.
If the call is successful,
but the \verb"pc_result" element is not \verb"PC_ACCEPT",
then the connection has been rejected;
consult Table~\ref{PSAPreasons} to determine the reason
(further information can be found in the \verb"pi" parameter).
Otherwise, if the call fails then the connection is not established and the
\verb"PSAPabort" structure of the \verb"PSAPindication" discriminated union
has been updated.

Note that the \verb"pc_info" element is allocated via \man malloc(3) and
should be released using the \verb"PCFREE" macro when no longer referenced.
The \verb"PCFREE" macro behaves as if it was defined as:
\begin{quote}\index{PCFREE}\small\begin{verbatim}
void    PCFREE (pc)
struct PSAPconnect *pc;
\end{verbatim}\end{quote}
The macro frees only the data allocated by \verb"PConnRequest",
and not the \verb"PSAPconnect" structure itself.
Further,
\verb"PCFREE" should be called only if the call to the \verb"PConnRequest"
routine returned \verb"OK".

Normally \verb"PConnRequest" returns only after a connection has succeeded or
failed.
This is termed a {\em synchronous\/} connection initiation.
If the user desires, an {\em asynchronous\/} connection may be initiated.
The routine \verb"PConnRequest" is really a macro which calls the routine
\verb"PAsynConnRequest" with an argument indicating that a connection should
be attempted synchronously.
\begin{quote}\index{PAsynConnRequest}\small\begin{verbatim}
int     PAsynConnRequest (calling, called, ctxlist,
                defctxname, prequirements, srequirements,
                isn, settings, ref, data, ndata, qos, pc,
                pi, async)
struct PSAPaddr *calling,
                *called;
int     prequirements,
        srequirements,
        settings,
        ndata,
        async;
long	isn;
struct PSAPctxlist *ctxlist;
OID     defctxname;
struct SSAPref *ref;
PE    *data;
struct QOStype *qos;
struct PSAPconnect *pc;
struct PSAPindication *pi;
\end{verbatim}\end{quote}
The additional parameter to this procedure is:
\begin{describe}
\item[\verb"async":] whether the connection should be initiated asynchronously.
\end{describe}
If the \verb"async" parameter is non-zero,
then \verb"PAsynConnRequest" returns one of four values:
\verb"NOTOK", which indicates that the connection request failed;
\verb"DONE", which indicates that the connection request succeeded;
or, either of \verb"CONNECTING_1" or \verb"CONNECTING_2", which indicates that
the connection request is still in
progress.
In the first two cases,
the usual procedures for handling return values from \verb"PConnRequest" are
employed
(i.e., a \verb"NOTOK" return from \verb"PAsynConnRequest" is equivalent to a
\verb"NOTOK" return from \verb"PConnRequest", and,
a \verb"DONE" return from \verb"PAsynConnRequest" is equivalent to a
\verb"OK" return from \verb"PConnRequest").

In the final case, when either \verb"CONNECTING_1" or
\verb"CONNECTING_2" is returned,
only the \verb"pc_sd" element of the \verb"pc" parameter has been updated;
it reflects the presentation-descriptor to be used to reference this
connection.

To determine when the connection attempt has been completed,
the routine \verb"xselect" (consult Section~\ref{acs:select} of \volone/)
should be used after calling \verb"PSelectMask".
In order to determine if the connection attempt was successful,
the routine \verb"PAsynRetryRequest" is called:
\begin{quote}\index{PAsynRetryRequest}\small\begin{verbatim}
int     PAsynRetryRequest (sd, pc, pi)
int     sd;
struct PSAPconnect *pc;
struct PSAPindication  *pi;
\end{verbatim}\end{quote}
The parameters to this procedure are:
\begin{describe}
\item[\verb"sd":] the presentation-descriptor;

\item[\verb"pc":] a pointer to a \verb"PSAPconnect" structure, which is
updated only if the call succeeds;
and,

\item[\verb"pi":] a pointer to a \verb"PSAPindication" structure, which is
updated only if the call fails.
\end{describe}
Again, one of three values are returned:
\verb"NOTOK", which indicates that the connection request failed;
\verb"DONE", which indicates that the connection request succeeded;
and, \verb"CONNECTING_1" or \verb"CONNECTING_2" which indicates that
the connection request is still in progress.

Refer to Section~\ref{tsap:async} on page~\pageref{tsap:async}
for information on how to make efficient use of the asynchronous connection
facility. 

\section	{Data Transfer}
Once the connection has been established,
a presentation-descriptor is used to reference the connection.
This is usually the first parameter given to any of the remaining routines in
the \man libpsap2(3n) library.
Further,
the last parameter is usually a pointer to a \verb"PSAPindication" structure
(as described on page~\pageref{PSAPindication}).
If a call to one of these routines fails,
then the structure is updated.
Consult the \verb"PSAPabort" element of the \verb"PSAPindication" structure.
If the \verb"pa_reason" element of the \verb"PSAPabort" structure is
associated with a fatal error,
then the connection is closed.
That is, a {\sf P-P-ABORT.INDICATION\/} event has occurred.
The \verb"PC_FATAL" macro can be used to determine this.
\begin{quote}\index{PC\_FATAL}\small\begin{verbatim}
int     PC_FATAL (r)
int     r;
\end{verbatim}\end{quote}
The most common interface error to occur is \verb"PC_OPERATION" which usually
indicates that either the user is lacking ownership of a session token
to perform an operation,
or that a session requirement required by the operation was not negotiated
during connection establishment.
For protocol purists,
the \verb"PC_OFFICIAL" macro can be used to determine if the error is an
``official'' error as defined by the specification,
or an ``unofficial'' error used by the implementation.
\begin{quote}\index{PC\_OFFICIAL}\small\begin{verbatim}
int     PC_OFFICIAL (r)
int     r;
\end{verbatim}\end{quote}

All of the remaining routines in the \man libpsap2(3n) library are identical
to their counterparts in the \man libssap(3n) library,
with these exceptions:
\begin{itemize}
\item	The final parameter to each routine is a pointer to a
	\verb"PSAPindication" structure,
	rather than a \verb"SSAPindication" structure.

\item	Any user data components are specified as an array of presentation
	elements (and the number of elements in that array),
	rather than the base of a character array (and the number of octets
	to be sent).
	Note that any data to be sent should have the \verb"pe_context"
	element set to the desired presentation context.
	By default, presentation elements are initialized with the
	default context (as represented by the manifest constant
	\verb"PE_DFLT_CTX").

\item	Asynchronous event handlers are called with pointers to \verb"PSAP"
	structures, rather tha \verb"SSAP" structures.

\item	With any indication which occurs,
	it is important to free any data which might have been allocated.
	The structures and corresponding macros are:
\[\begin{tabular}{|l|l|}
\hline
    \multicolumn{1}{|c|}{\bf Macro}&
		\multicolumn{1}{c|}{\bf Structure Pointer}\\
\hline
    \tt	PXFREE&		struct PSAPdata *\\
    \tt	PTFREE&		struct PSAPtoken *\\
    \tt	PNFREE&		struct PSAPsync *\\
    \tt	PVFREE&		struct PSAPactivity *\\
    \tt	PPFREE&		struct PSAPreport *\\
    \tt	PFFREE&		struct PSAPfinish *\\
    \tt	PRFREE&		struct PSAPrelease *\\
    \tt	PAFREE&		struct PSAPabort *\\
\hline
\end{tabular}\]
	Note that these free the user data referenced by the indication
	structures,
	and not the structures themselves.\index{PXFREE}\index{PTFREE}
	\index{PNFREE}\index{PVFREE}\index{PPFREE}\index{PFFREE}\index{PRFREE}
	\index{PAFREE}
\end{itemize}

\subsection	{Restrictions on User Data}\label{PSAPdata}
To quote the \cite{ISO.PP.Service} specification:
\[\fbox{\begin{tabular}{lp{0.8\textwidth}}
\bf NOTE:&	For all services which carry user data,
		excluding {\sf P-DATA\/} and {\sf P-TYPED-DATA},
		it may not be possible to exchange user data,
		dependent on the user data length limitation supported by
		the underlying session services.
\end{tabular}}\]

\section	{Error Conventions}
All of the routines in this library return the manifest constant \verb"NOTOK"
on error,
and also update the \verb"pi" parameter given to the routine.
The \verb"pi_abort" element of the \verb"PSAPindication" structure contains a
\verb"PSAPabort" structure detailing the reason for the failure.
The \verb"pa_reason" element of this latter structure can be given as a
parameter to the routine \verb"PErrString" which returns a null-terminated
diagnostic string.
\begin{quote}\index{PErrString}\small\begin{verbatim}
char   *PErrString (c)
int     c;
\end{verbatim}\end{quote}

\section	{Compiling and Loading}
Programs using the \man libpsap2(3n) library should include
\verb"<isode/psap2.h>".
The programs should also be loaded with \verb"-lpsap2".

\section	{An Example}
Let's consider how one might construct a source entity that resides on the
PSAP.
This entity will use a synchronous interface.

There are two parts to the program:
initialization and data transfer;
release will occur when the standard input has been exhausted.
In our example,
we assume that the routine \verb"error" results in the process being
terminated after printing a diagnostic.

In Figure~\ref{initPSsource},
the initialization steps for the source entity,
including the outer {\em C\/} wrapper,
is shown.
First, a lookup is done in the ISO services database,
and the \verb"PSAPaddr" is initialized.
The \verb"SSAPref" is initialized using the routine
\verb"addr2ref".
This routine takes a string and returns a pointer to a \verb"SSAPref"
structure which has been initialized to contain the string and the current
UTC time.
Next, for each token associated with the session requirements,
initial ownership of that token is claimed.
Finally,
the call to \verb"PConnRequest" is made.
If the call is successful,
a check is made to see if the remote user accepted the connection.
If so,
the presentation-descriptor is captured,
along with the negotiated requirements and initial token settings.

In Figure~\ref{dataPSsource} on page~\pageref{dataPSsource},
the data transfer loop is realized.
The source entity reads a line from the standard input,
and then queues that line for sending to the remote side.
When an end-of-file occurs on the standard input,
the source entity requests release and then gracefully terminates.
Although no checking is done in this example,
for the calls to \verb"PDataRequest" and \verb"PRelRequest",
on failure
a check for the operational error \verb"PC_OPERATION" should be made.
For \verb"PDataRequest",
this would occur when the data token was not owned by the user;
for \verb"PRelRequest",
this would occur when the release token was not owned by the user.
\clearpage
\tagrind[tp]{grind5a-2a}{Initializing the PSAP source entity}{initPSsource}
\clearpage
\tagrind[tp]{grind5a-2b}{Initializing the PSAP source entity (continued)}\empty
\clearpage
\tagrind[tp]{grind5a-3}{Data Transfer for the PSAP source entity}%
	{dataPSsource}

\section	{Lightweight Presentation Protocol}
To run OSI applications using the lighweight presentation protocol
defined in RFC1085,
load the program with \verb"-lpsap2-lpp" or \verb"-isode-lpp".

This is a complete implementation of RFC1085.
The following functions are available:
\begin{quote}\small\begin{verbatim}
PInit
PConnResponse
PAsynConnRequest
PAsynRetryRequest
PDataRequest
PReadRequest
PUAbortRequest
PRelRequest
PRelResponse
PSetIndications
PSelectMask
PErrSTring
\end{verbatim}\end{quote}
Note that when RFC1085 is used as a presentation backing,
only a subset of the presentation services are available:
\begin{quote}\small\sf
P-CONNECT
P-DATA
P-U-ABORT
P-P-ABORT
\end{quote}
The \man lppd(8c) daemon is used to listen for incoming connections and
dispatch the appropriate daemon.
This daemon will listen only for connections using the TCP backing.
For connections using the UDP backing,
the responder program must listen itself.
This is trivally accomplished using the \verb"isodeserver" routine
described in Section~\ref{isodeserver} on page~\pageref{isodeserver}.

\section	{For Further Reading}
The ISO specification for session services is defined in
\cite{ISO.PP.Service},
while the corresponding protocol definition is \cite{ISO.PP.Protocol}.

%%% \section	{Changes Since the Last Release}\label{psapX:changes}
%%% A brief summary of the major changes between v~\isodeprevrsn/ and
%%% v~\isodevrsn/ are now presented.
%%% These are the user-visible changes only;
%%% changes of a strictly internal nature are not discussed.

