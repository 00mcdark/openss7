% run this through LaTeX with the appropriate wrapper

\chapter	{The Transport Switch}\label{tswitch}
As of this writing,
the concept of ubiquitous OSI has not yet been realized.
In particular,
there continues to be disagreement on the transport/network protocols to be
used to provide end-to-end service.
As a result,
interworking between sites in many cases is the {\em exception},
not the rule.

In order to facility communication,
the ISODE has a powerful abstraction,
the {\em Transport Switch}.
The goal of the transport switch is to hide the complex underpinnings of the
``real world'' of OSI from the user of the transport service.
In order to make effective use of this abstraction,
it may be necessary to tailor the ISODE
(using the \man isotailor(5) file described in Chapter~\ref{isotailor}).

To explain how to use the transport switch effectively,
it is necessary to introduce some terminology.

\section	{Transport Stacks}
A {\em TS-stack\/} refers to a combination of transport protocol and network
service that is used to provide end-to-end transport service.
The ISODE, depending on how you configure it at compile-time,
supports any combination of these stacks: 
\[\begin{tabular}{rl}
\multicolumn{1}{c}{\bf Mneumoic}&
		\multicolumn{1}{c}{\bf TS-Stack}\\
\verb"tcp"&	RFC1006 over TCP/IP\\
\verb"x25"&	TP0 over X.25\\
\verb"x25-84"&	TP0 over X.25(1984)\\
\verb"bridge"&	TP0 over the TP0--bridge\\
\verb"tp4"&	TP4 over CLNP
\end{tabular}\]
Internally,
the ISODE uses ``typing'' on all network addresses to one of these choices.
From a practical perspective,
usually only the \verb"tcp", \verb"x25-84" and \verb"x25" are of interest.

The run-time tailoring variable \verb"ts_stacks" keeps track of the
TS-stacks available on your host.
This defaults to the TS-stacks that were configured at compile-time.
However, if you are sharing executables between machines with different
network attachments,
you will want to change this.
Generally,
you compile-time configure the ISODE for all stacks available at your site.
Then,
for each host you install the executables on,
you set the variable \verb"ts_stacks" accordingly.
So, suppose you have a host with both TCP and X.25 services,
but that all the other hosts at your site have only TCP.
In this case
(assuming binary compatibility between the hosts at your site),
you generate the ISODE on the host with both TCP and X.25 services configured.
When you move the executables to the TCP only hosts,
you add the line
\begin{quote}\small\begin{verbatim}
ts_stacks: tcp
\end{verbatim}\end{quote}
to the \file{isotailor} file on those hosts.

\section	{OSI Communities}
An OSI {\em community\/} is a collection of hosts which share a common TS-stack
along with basic connectivity.
Simply put,
a community consists of end-systems that all interwork with each other.
In a perfect world,
there would be but a single community.
\verb"realns".
But it isn't a perfect world,
my favorite Rock group, {\em Bangles}, broke up in September
of '{\oldstyle 89}, and I am very upset about this.
But that is another story.

There are several communities using OSI at present,
the short list is:
\[\begin{tabular}{rll}
\multicolumn{1}{c}{\bf Mneumoic}&
		\multicolumn{1}{c}{\bf Community}&
					\multicolumn{1}{c}{\bf TS-stack}\\
\verb"int-x25"&	the International X.25&	\verb"x25"\\
\verb"janet"&	the JANET in the UK&	\verb"x25"\\
\verb"internet"& the capital-I Internet&\verb"tcp"\\
\verb"localTCP"& the TCP loopback address&
					\verb"tcp"\\
\verb"IXI"&	International X.25 Interconnect&
					\verb"x25"
\end{tabular}\]
These are all termed {\em interim\/} communities as they do not use the real
OSI network service.
In order to facility communications between these communities
(and others, e.g., private LANs),
an Interim addressing scheme has been defined \cite{Interim.Addresses}
(which is included in the ISODE documentation set,
look in the directory \file{doc/interim/} in the source tree).

Thus,
the first task when running the ISODE is to determine which communities
your site belongs to.
The \verb"ts_communities" run-time tailoring variable keeps track of the
communities which your host can (in)directly reach.
This defaults to the four communities above,
plus the \verb"realNS" community.
Whilst this is the most sensible default,
this choice is probably wrong for most sites.
So,
if your host is connected to the International X.25 but not the JANET,
you add the line
\begin{quote}\small\begin{verbatim}
ts_communities: int-x25
\end{verbatim}\end{quote}
to the \file{isotailor} file on that host.

Of course,
if the host in question is connected not only to the International X.25 but
also belongs to the capital-I Internet, then the line might read
\begin{quote}\small\begin{verbatim}
ts_communities: int-x25 internet
\end{verbatim}\end{quote}
instead.
This raises an important question:
if a host wishes to contact another host,
and the two have multiple communities in common,
what preference is given when connections are attempted?
The answer is that the ordering of the communities in the
\verb"ts_communities" run-time tailoring variable tells the ISODE what the
preference is.
Thus,
in the example above,
if there were two hosts connected to both the International X.25 and the
capital-I Internet,
and one of the hosts wished to talk to the other,
the International X.25 community would be tried first.
If a connection could not be established,
then the capital-I Internet community would be tried.

There are a few common site configurations.
If the site has access to the capital-I Internet,
then usually all hosts belong to the capital-I Internet community,
e.g.,
\begin{quote}\small\begin{verbatim}
ts_communities: internet
\end{verbatim}\end{quote}
It is possible that one or more hosts at this site may have access to the
International X.25,
and exactly these hosts would have connectivity to this second community:
\begin{quote}\small\begin{verbatim}
ts_communities: internet int-x25
\end{verbatim}\end{quote}

\subsection	{Defining a new OSI community}
Another common configuration is that a site has a single host that is
connected to the International X.25.
In addition,
there is a private LAN,
running the TCP/IP protocols,
that all hosts at the site are connected to.
In this case,
you need to define a community name for your site.
This is done by describing your community name as a macro in the
\man isomacros(5) file and then enabling use of this community in the
\man isotailor(5) file.

To describe your community,
you create a macro of the form:
\begin{quote}\small\begin{verbatim}
name    TELEX+value+stack+number+
\end{verbatim}\end{quote}
where:
\begin{describe}
\item[\verb"name":] is the name of your community;

\item[\verb"value":] corresponds to the TELEX number at your site.
(This is traditionally a three-digit international code followed by a
national TELEX number.)
The combination of \verb"TELEX+value" defines an OSI address prefix which your
site uniquely administers;

\item[\verb"stack":] is either \verb"RFC-1006" or \verb"X.25(80)" indicating
the TS-stack used for the addresses;
and,

\item[\verb"number":] by convention is a two-digit decimal number from~01
to~99 (this allows you to define up to 99 communities at your site).
\end{describe}
Of course,
there are other methods for generating unique OSI addresses,
but this is the only method supported in the current implementation of the
ISODE.

\subsubsection	{Defining a new TCP-based community}
Consider the the case of an isolated LAN running the TCP/IP protocols.

First,
edit the file \file{support/macros.local} in the source area and add a line 
line like this: 
\begin{quote}\small\begin{verbatim}
nott-ether	TELEX+00738700+RFC-1006+01+
\end{verbatim}\end{quote}
where \verb"00738700" corresponds to your TELEX number.
Next,
type
\begin{quote}\small\begin{verbatim}
# ./make macros
\end{verbatim}\end{quote}
as the super-user.
This will install a new version of the \man isomacros(5) file.

Second,
in the \man isotailor(5) file, you would add this line:
\begin{quote}\small\begin{verbatim}
ts_interim: nott-ether
\end{verbatim}\end{quote}
which tells the ISODE about your new community.
You would then add \verb"nott-ether" to the beginning of the value for the
run-time tailoring variable \verb"ts_communities".
So,
on the hosts with only TCP, you would have:
\begin{quote}\small\begin{verbatim}
ts_interim:     nott-ether
ts_communities: nott-ether
\end{verbatim}\end{quote}
and on the host which also had connectivity to the International X.25,
you would have:
\begin{quote}\small\begin{verbatim}
ts_interim:     nott-ether
ts_communities: nott-ether int-x25
\end{verbatim}\end{quote}
Note that it is {\bf critical\/} that the definition of \verb"ts_interim"
occur {\bf before\/} the definition of \verb"ts_communities".

\subsubsection	{Defining a new X.25-based community}
Consider the case of a site which has access to the International X.25 and
belongs to a private X.25 network.

In this case,
the macro definition might be:
\begin{quote}\small\begin{verbatim}
psi-wan     TELEX+00738700+X.25(80)+02+
\end{verbatim}\end{quote}
So,
the tailoring variables would be
\begin{quote}\small\begin{verbatim}
ts_interim: psi-wan
ts_communities: psi-wan
\end{verbatim}\end{quote}
for those hosts which are only on the private X.25 network,
whilst
\begin{quote}\small\begin{verbatim}
ts_interim: psi-wan
ts_communities: psi-wan int-x25
\end{verbatim}\end{quote}
would be used for hosts on both the International X.25 and the private X.25
network.
Again, it is critical that the definition of \verb"ts_interim"
occur {before the definition of \verb"ts_communities".

\subsection	{Heuristic Support}
Finally,
there are some cases when the ISODE must look at a simple string and derive an
address. 
If the string encoding for presentation address defined in
\cite{String.Addresses},
then the macro defined earlier will help.
But,
if a simpler string is used,
e.g.,
\begin{quote}\small\begin{verbatim}
% ftam gonzo
\end{verbatim}\end{quote}
or
\begin{quote}\small\begin{verbatim}
% ftam 00000511160013
\end{verbatim}\end{quote}
then it would be nice to have an intelligent default to use for the community
associated with these.
There are three run-time tailoring variables provided for this purpose:
\[\begin{tabular}{ll}
\multicolumn{1}{c}{\bf if it looks like}&
		\multicolumn{1}{c}{\bf then use}\\
an OSI NSAP&	\verb"default_nsap_community"\\
a X.25 DTE&	\verb"default_x25_community"\\
a TCP address&	\verb"default_tcp_community"
\end{tabular}\]
To continue our two examples of private communities.
for the private TCP/IP LAN,
it would make sense to add
\begin{quote}\small\begin{verbatim}
default_tcp_community: nott-ether
\end{verbatim}\end{quote}
to the \file{isotailor} file on each host.
For the private X.25--based community,
it would make sense to add
\begin{quote}\small\begin{verbatim}
default_x25_community: psi-wan
\end{verbatim}\end{quote}
instead.

\section	{Transport-Service Bridges}
There is one last question which must be considered:
how can two hosts communicate if they do not have any communities in common?
If there is a third host,
which shares a community with the two other hosts,
then a {\em TS-bridge\/} can be used to achieve connectivity on
a per-connection basis.

\subsection	{Client Hosts}
For each host,
examine the value of the \verb"ts_communities" run-time variable.
For those communities which are not listed,
but which you wish that host to be able to communicate with,
you must add the definition of the appropriate TS-bridge to the
\verb"tsb_communities" run-time tailoring variable.
A definition consists of two tokens,
the name of the community that the TS-bridge can reach,
and the transport address of the TS-bridge
(using the string encoding defined in \cite{String.Addresses}),
e.g.,
\begin{quote}\smaller\begin{verbatim}
tsb_communities: internet Int-X25(80)=31344152401010+PID+03018000
\end{verbatim}\end{quote}
This says that the capital-I Internet community can be reached by contacting
the TS-bridge located on the International X.25 at the specified DTE and PID.

It is {\bf critical\/} to observe that the TS-bridge must share a common
community with the host containing this definition.
Thus,
a more instructive example is:
\begin{quote}\smaller\begin{verbatim}
ts_communities: int-x25 internet
tsb_communities: internet Int-X25(80)=31344152401010+PID+03018000
\end{verbatim}\end{quote}
which describes a host connected to the International X.25 that may wish to
talk to hosts in the capital-I Internet community.

Similarly,
one could imagine a host belonging to the capital-I Internet community using
the following definitions:
\begin{quote}\smaller\begin{verbatim}
ts_communities: internet int-x25
tsb_communities: int-x25 Internet=gonzo.twg.com+17004
\end{verbatim}\end{quote}

\subsection	{Server Hosts}
Typically,
the entity responsible for a community also runs a TS-bridge which connects
that community to the other Interim communities.
Obviously,
this host must support each of the Interim communities to be connected to the
local community.
The \verb"tsb_default_address" run-time tailoring variable is used to define
the transport address (usually containing multiple network addresses) where
the TS-bridge listens.

So, to continue with the nott-ether example,
the host on which the TS-bridge resides might have these definitions:
\begin{quote}\tiny\begin{verbatim}
ts_interim:	     nott-ether
ts_communities:      nott-ether int-x25
tsb_default_address: Nott-Ether=sheriff+17004\|Int-X25(80)=23426020017299+PID+03018000
\end{verbatim}\end{quote}
The rule is simple,
for each community named in the \verb"ts_communities" run-time tailoring
variable,
a network address is present in the run-time tailoring variable
\verb"tsb_default_address".
By convention,
for TCP-based addresses,
TCP port 17004 is used, 
whilst for X.25--based addresses,
PID 0301800 is used.

\section	{In Retrospect}
What a mess!  There's a lot to be said for the focused ``one transport
protocol, one network service'' approach used by the Internet suite of
protocols.
