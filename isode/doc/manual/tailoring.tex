% run this through LaTeX with the appropriate wrapper

\chapter	{The ISODE Tailoring File}\label{isotailor}
The file \file{isotailor} in the ISODE \verb"ETCDIR" directory
(usually \file{/usr/etc/})
contains a simple run-time configuration
mechanism for programs loaded with the \verb"-lisode" library.

The file itself is an ordinary ASCII text file containing information
regarding the known tailoring options.
Each line contains the option's name, a colon, the option's value, and a
newline.
The sharp character (`\verb"#"') at the beginning of a line indicates a
commentary line.

\section	{Tailor Variables}
The options available,
along with default values and a description of their meanings,
are now described.

\subsection	{Local Environment Tailoring}
\begin{describe}
\item[\verb"localname":] The name of the localhost.
If not set,
depending on the TCP/IP implementation you are running,
the system will be queried for this value.

\item[\verb"binpath":] Where user programs are found.

\item[\verb"sbinpath":] Where system programs are bound.

\item[\verb"etcpath":] Where configuration files are found.
\end{describe}

\vspace{0.1in}		%%% yikes!
\begin{center}\em
(continued on the next page)
\end{center}
\newpage

\subsection	{Logging Tailoring}
\begin{describe}
\item[\verb"logpath":] Where log files are found.

\item[\em xyz\/\verb"level":] The debugging level for the {\em xyz\/}
module,
defaulting to \verb"none".
\[\begin{tabular}{|l|l|}
\hline
    \multicolumn{1}{|c|}{\bf Option}&
		\multicolumn{1}{c|}{\bf Module}\\
\hline
    \tt compatlevel&	\man libicompat(3n)\\
    \tt addrlevel&	various\\
    \tt tsaplevel&	\man libtsap(3n)\\
    \tt ssaplevel&	\man libssap(3n)\\
    \tt psaplevel&	\man libpsap(3)\\
    \tt psap2level&	\man libpsap2(3n)\\
    \tt acsaplevel&	\man libacsap(3n)\\
    \tt rtsaplevel&	\man librtsap(3n)\\
    \tt rosaplevel&	\man librosap(3n)\\
\hline
\end{tabular}\]
Options are seperated by whitespace.
The debugging options available are:
\begin{describe}
\item[\verb"fatal":] fatal errors;

\item[\verb"exceptions":] exceptional events;

\item[\verb"notice":] informational notices;

\item[\verb"pdus":] {\em xyz\/}PDU printing;

\item[\verb"trace":] program tracing;

\item[\verb"debug":] full debugging
(not fully defined at this point);
and,

\item[\verb"all":] all of the above.
\end{describe}
Logging of levels other than \verb"fatal",
\verb"exceptions", or \verb"notice" are subject
to conditional compilation
(the \verb"-DDEBUG" option must be used during compilation).

\item[\em xyz\/\verb"file":] The file to be used for {\em xyz\/}PDU tracing.
The file is written in append mode.
If the filename supplied is \arg"-" (a single dash),
then the diagnostic output is used instead.
\[\begin{tabular}{|l|l|l|}
\hline
    \multicolumn{1}{|c|}{\bf Option}&
		\multicolumn{1}{c|}{\bf Default}&
		\multicolumn{1}{c|}{\bf Module}\\
\hline
    \tt compatfile&	\tt \%d.log&	\man libicompat(3n)\\
    \tt addrfile&	\tt \%d.log&	various\\
    \tt tsapfile&	\tt \%d.tpkt&	\man libtsap(3n)\\
    \tt ssapfile&	\tt \%d.spkt&	\man libssap(3n)\\
    \tt psapfile&	\tt \%d.pe&	\man libpsap(3)\\
    \tt psap2file&	\tt \%d.ppkt&	\man libpsap2(3n)\\
    \tt acsapfile&	\tt \%d.acpkt&	\man libacsap(3n)\\
    \tt rtsapfile&	\tt \%d.rtpkt&	\man librtsap(3n)\\
    \tt rosapfile&	\tt \%d.ropkt&	\man librosap(3n)\\
\hline
\end{tabular}\]
The \man getpid(2) call is used to supply the value for \verb"%d".
\end{describe}

\subsection	{Directory Services Tailoring}\label{dse:tailor}
\begin{describe}
\item[\verb"ns\_enable":] enables use of a ``user-friendly nameservice''
to perform name/address resolution.
This takes the value either \verb+"on"+ or \verb+"off"+.
If \verb+"on"+,
then an OSI Directory-based service will be use.
If the nameservice lookup fails,
the stub-directory will be used as a fallback.

\item[\verb"ns\_address":] the transport address of the nameservice.
It is specified using the ISODE ``string'' format,
e.g.,
\begin{quote}\small\begin{verbatim}
Internet=wp.psi.net+17006
\end{verbatim}\end{quote}
which indicates that the nameservice lives in the TCP/IP communications domain
on TCP port \verb"17006" at host \verb"wp.psi.net".
The nameservice is accessed via the OSI CO-mode transport service,
so other kinds of addresses (e.g., X.25 addresses can be used as well).
\end{describe}

\subsection	{Transport Switch Tailoring}
The use of these variables is more usefully described in Chapter~\ref{tswitch}.
\begin{describe}
\item[\verb"ts\_stacks":] Specifies which configured TS-stacks are
enabled.
This is useful when multiple machines (with different interfaces)
share the same executables.
Options are seperated by whitespace:
\begin{describe}
\item[\verb"tcp":] RFC1006 over TCP/IP;

\item[\verb"x25":] TP0 over X.25;

\item[\verb"cons":] TP0 over CONS;

\item[\verb"bridge":] TP0 over the TP0--bridge;

\item[\verb"tp4":] TP4 over CLNP;
and,

\item[\verb"all":] all of the above.
\end{describe}
Using this method,
the \file{isotailor} file is a normally symbolic link to
\file{/private/etc/isotailor}.

\item[\verb"ts\_interim":] Defines new OSI communities.
Each community is defined by a macro in the \file isomacros(5) file.

\item[\verb"ts\_communities":] Specifies which OSI communities are attached
(either directly or through a transport-service bridge).
Options are seperated by whitespace:
\begin{describe}
\item[\verb"int-x25":] International X.25;

\item[\verb"janet":] UK JANET;

\item[\verb"internet":] the capital-I Internet;

\item[\verb"realns":] OSI Internet (ha, ha);

\item[\verb"localTCP":] the TCP loopback address;
and,

\item[\verb"all":] all of the above OSI communities, along with those
communities defined by the \verb"ts_interim" variable;
\end{describe}
For example,
a site with an X.25 connection might be attached to the International X.25
network, but not the JANET.
Thus \verb"ts_stacks" would include ``x25'',
and \verb"ts_communities" would include ``int-x25'' but not ``janet''.

Note that the ordering of communities is important:
network addresses will be tried in the order that their respective
communities are listed with this variable.

\item[\verb"default\_nsap\_community":] the default community to be
used for NSAP addresses.

\item[\verb"default\_x25\_community":] the default community to be
used for X.25 (DTE) addresses.

\item[\verb"default\_tcp\_community":] the default community to be
used for TCP (RFC1006) addresses.
\end{describe}

\subsubsection{Transport-Service Bridge}
There are two variables that can be specified.
One is used on hosts making use of the TS-bridge,
the other is used by hosts which run the TS-bridge:
\begin{describe}
\item[\verb|tsb\_communities|]	A list of pairs of values.
The first of each value should be a community name as defined in
the \verb"ts_communities" variable.
The second value of the pair should be a presentation address using the ISODE
``string'' format.
When a call is to be placed and the network corresponds to one
of the communities given here, then a call through the TS-bridge given in
the second variable will be made automatically.

\item[\verb|tsb\_default\_address|] This variable contains a string encoded
presentation address which the TS-bridge will listen on by default.
This should normally consist of a set of network addresses with no selectors
present.
\end{describe}
Consider the case of a host with access to both the Internet and the
International X.25 network.
This host might have this entry in its \file{isotailor} file:
\begin{quote}\tiny\begin{verbatim}
tsb_default_address: Internet=sheriff+17004\|Int-X25(80)=23426020017299+PID+03018000
\end{verbatim}\end{quote}
This tells the TS-bridge to listen on two network endpoints.
Hosts in the Internet community wishing to reach the International
X.25 community would have this entry in their \file{isotailor} file:
\begin{quote}\smaller\begin{verbatim}
tsb_communities: int-x25 Internet=sheriff+17004
\end{verbatim}\end{quote}
Similarly,
hosts in the International X.25 community wishing to reach the
Internet community, would have the entry:
\begin{quote}\smaller\begin{verbatim}
tsb_communities: internet Int-X25(80)=23426020017299+PID+03018000
\end{verbatim}\end{quote}

\subsection{Interface Specific Tailoring}
Some network implementations used by ISODE require ISODE to be
tailored for their correct use.

\subsubsection{General X.25 Tailoring}
The following tailoring variables are generally applicable to
X.25 networks.
\begin{describe}
\item[\verb"x25\_local\_dte"] It is normally
necessary for ISODE to know it's local DTE
address.
This variable is used to set this.
The default is empty, i.e., do not set a calling address in call
requests.

\item[\verb"x25\_local\_pid"] It is normally
necessary for ISODE to know the X.25 protocol ID to listen on
This is specified in hex-notation,
e.g., \verb"03010100".

\item[\verb"x25\_intl\_zero"] Some Public Data Networks
require that X.121 addresses be
modified before being conveyed.
If this variable has the value \verb"on" then any addresses
with a non-local DNIC will have a leading zero appended.

\item[\verb"x25\_strip\_dnic"] If this variable has
the value \verb"on" then any address with
a local DNIC will have this removed.

\item[\verb"x25\_dnic\_prefix"] If you use either or both of the
preceding two mechanisms then you must use this variable to
inform ISODE of the local DNIC for your host.

\item[\verb"x25level":]
Defines the level of logging to be used for X.25 statistics logging.
(At present, only \verb"notice" messages are generated.)

\item[\verb"x25file":]
Defines the filename to be used for X.25 statistics logging.
\end{describe}

\subsubsection{SunLink X.25}
The following variables are currently only supported by the
SunLink X.25 interface.
They control the X.25 Facilities that are requested or accepted.
\begin{describe}
\item[\verb"reverse\_charge"]  If \verb"0" then don't
request/allow (initiator/responder) reverse charging.
If \verb"1" then request/allow reverse charging.

\item[\verb"recvpktsize"/\verb"sendpktsize"] Size of level 3 packets.
Valid values are \verb"0" (default size),
\verb"16", \verb"32", \verb"64", \verb"128",
\verb"256", \verb"512", \verb"1024" (octets in decimal).

\item[\verb"recvwndsize"/\verb"sendwndsize"] Size of level 3 window.
Valid values are \verb"0" (default window size),
\verb"1"--\verb"7" or \verb"1"--\verb"127" (decimal).

\item[\verb"recvthruput"/\verb"sendthruput"] Send/receive throughput.
\nopagebreak
\begin{quote}
\begin{tabbing}
100\=1200\hspace{2em}\=100\=\kill
\verb"0"\> default throughput \\
\verb"3"\> 75\> \verb"8"\> 2400 \\
\verb"4"\> 150\> \verb"9"\> 4800 \\
\verb"5"\> 300\> \verb"10"\> 9600 \\
\verb"6"\> 600\> \verb"11"\> 19200 \\
\verb"7"\> 1200\> \verb"12"\> 48000  \\
\end{tabbing}
\end{quote}
Values in bps decimal.

\item[\verb"cug\_req"] If \verb"0" then don't use closed user group, if
\verb"1" then use closed user group specified by
\verb"cug_index".

\item[\verb"cug\_index"] Closed user group in decimal
(\verb"00"--\verb"99").

\item[\verb"fast\_select\_type"] \verb"0" = don't use/allow fast
select.
\verb"1" = calling side --- only accept clear in response to
fast select, called side --- send clear in response to fast
select.
% is any of this fast select stuff applicable to ISODE?
\verb"2" = clear or accept is valid response to fast select.

\item[\verb"rpoa\_req"] If \verb"0" then don't request RPOA
(Recognised Private Operating Agency) transit.
If \verb"1" then request RPOA transit.

\item[\verb"rpoa"] If \verb"rpoa_req" is \verb"1" then this is
RPOA transit group in decimal (\verb"0000"--\verb"9999").
\end{describe}
% actually cug_index and rpoa seem to be ignored in the current version
See the SunLink X.25 Programmer's Manual for
further explanations of these
facilities.

\subsubsection{Camtec CCL}
There is one tailoring variable for the Camtec X.25 when used
with the socket abstraction.%
\footnote{The old device level interface is no longer supported.}
\begin{describe}
\item[\verb"x25\_outgoing\_port"] selects the physical port on the
Camtec card for outgoing calls.
It may take the value \verb"A", \verb"B" or \verb"#".
\verb"A" and \verb"B" are the X.121 WAN ports and \verb"#" is
the IEEE 802.3 (Ethernet) port.
Incoming calls will be accepted on any port.
\end{describe}

\subsubsection{Bridge X.25}
There are several tailorable variables that can specified for the
bridge connection. These are:
\begin{describe}
\item[\verb|x25\_bridge\_host|]	selects the host that runs the
tp0bridge being used. This should be a TCP accessible host.

\item[\verb|x25\_bridge\_port|]	selects the TCP port that the
tp0bridge will be listening on. The default for this is port 146 (an
internet assigned number), which should be defined in \verb|/etc/services|.

\item[\verb|x25\_bridge\_addr|]	the X.121 address of the remote host.

\item[\verb|x25\_bridge\_listen|]	the X.121 address that this host will
be listening on for incoming calls via the bridge.

\item[\verb|x25\_bridge\_pid|]	the protocol id used for listening
along with the above address. This is a set of eight hex digits.

\item[\verb|x25\_bridge\_discrim|] selects the network layer to use.
When attempting to place a call with the bridge code configured as
well as real X.25, the string selects the interface to use. If the
string is empty, the bridge will always be used. If it is set to
``--'' the bridge will not be used. If the string is anything else, it
is compared against the initial portion of the called X.121 address.
If there is a match then the bridge is used, otherwise the real
interface is called.
\end{describe}

\section	{Accessing the Tailoring File}
The tailoring file is read usually when a program attempts or accepts its
first connection.
The \verb"-lisode" library does this by calling the routine
\verb"isodetailor":
\begin{quote}\index{isodetailor}\small\begin{verbatim}
void    isodetailor (myname, wantuser)
char   *myname;
int     wantuser;
\end{verbatim}\end{quote}
The parameters to this procedure are:
\begin{describe}
\item[\verb"myname":] the name that the program was invoked with
(used by the logging package described in Chapter~\ref{logging}); and,

\item[\verb"wantuser":] if non-zero,
then a user-specific tailoring file,
with the name \verb"~/.myname_tailor",
should be consulted.
\end{describe}
Note that in order to ensure consistent logging it is {\bf critical\/} that
the call to \verb"isodetailor" be the first call made to {\em any \/} of the
ISODE routines.

To override the default location of the tailoring file,
use the routine \verb"isodesetailor":
\begin{quote}\index{isodesetailor}\small\begin{verbatim}
char   *isodesetailor (file)
char   *file;
\end{verbatim}\end{quote}
The parameter to this procedure is:
\begin{describe}
\item[\verb"file":] the filename to be used instead of the default.
Future versions of this routine might act differently.
\end{describe}
The filename is interpreted relative to the \verb"-lisode" system area.
To override this, specify a anchored pathname
(e.g., on \unix/, one which starts with \verb"/" or \verb"./").
The routine returns the name of the default tailoring file.

To set a tailoring variable from some other configuration file,
the routines \verb"isodesetvar" and \verb"isodexport" are used:
\begin{quote}\index{isodesetvar}\small\begin{verbatim}
int     isodesetvar (name, value, dynamic)
char   *name,
       *value;
int     dynamic;
\end{verbatim}
\end{quote}
When this routine is invoked,
it acts as though
\begin{quote}\small\begin{verbatim}
name: value
\end{verbatim}\end{quote}
was found in the tailoring file.
The \verb"dynamic" parameter, if non-zero,
indicates that \verb"value" may be freed if a subsequent
call to \verb"isodesetvar" is made which overrides the previous value.

The \verb"isodexport" routine is called after one or more calls to
\verb"isodesetvar",
it performs any post-processing necessary to resynchronize the tailoring
facilities.
\begin{quote}\index{isodexport}\small\begin{verbatim}
void    isodexport ()
\end{verbatim}
\end{quote}
Thus,
to read a private tailoring file,
\verb"isodesetvar" should be called for each tailoring line.
Then,
\verb"isodexport" should be called once to resynchronize things.

Finally, it may be necessary to access files in the \verb"-lisode" system area.
The routine \verb"isodefile" takes an filename and returns an anchored
pathname.
\begin{quote}\index{isodefile}\small\begin{verbatim}
char   *isodefile (file, ispgm)
char   *file;
int	ispgm;
\end{verbatim}\end{quote}
The parameters to this procedure are:
\begin{describe}
\item[\verb"file":] the filename to be expanded;
and,

\item[\verb"ispgm":] non-zero if the target file is an executable
(otherwise it is a database of some kind).
\end{describe}
This routine is actually a macro which invokes the routine \verb"_isodefile":
\begin{quote}\index{\_isodefile}\small\begin{verbatim}
char   *_isodefile (path, file)
char   *path,
       *file;
\end{verbatim}\end{quote}
The parameters to this procedure are:
\begin{describe}
\item[\verb"path":] the directory where the filename should be expanded; and,

\item[\verb"file":] the filename to be expanded.
\end{describe}

%%% \section	{Changes Since the Last Release}\label{compat:changes}
%%% A brief summary of the major changes between v~\isodeprevrsn/ and
%%% v~\isodevrsn/ are now presented.
%%% These are the user-visible changes only;
%%% changes of a strictly internal nature are not discussed.

