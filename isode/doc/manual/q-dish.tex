% Run this through LaTeX with the appropriate wrappe

\chapter{The OSI Directory}
\label{dir_model}

This chapter is designed as a quick introduction to the model of X.500
directories, to provide the reader with sufficient information to be able to
understand the following chapters.
For full coverage of X.500 readers should consult
\cite{little.black.book}.

\section {The Model}
The OSI directory is intended to support human user querying, allowing
users to find, {\em inter alia}, telephone and address information of
organisations and other users.  

It is also intended to support electronic
communication such as message handling systems and file transfer.
The directory provides name to address mapping to support, for example, OSI
presentation address look-ups. Message handling systems will be provided with 
support for user-friendly naming, security and
distribution lists.

\tagfigure[tp]{q-dit}{Example DIT}{qfig:dit}

In essence the directory is a database with certain key characteristics.
\begin{itemize}
\item
The directory is intended to be very large and highly distributed. It is
anticipated that the directory will be distributed largely on an
organisational basis.
\item
The directory is hierarchically structured, the entries being arranged in
the form of a tree called the directory information tree (DIT).
An example DIT is shown in Figure~\ref{qfig:dit}. 
Entries near the root of the tree will usually represent
objects such as countries (e.g., ``GB'') and organisations (e.g., ``University 
College London''), entries at or near the leaves
of the tree will represent people (e.g., ``Colin Robbins''), 
equipment or application processes.

\item
Read and search operations will dominate over modification operations.
\item
Temporary inconsistencies in the data are acceptable.  This greatly
facilitates the replication of data in the directory by obviating concerns
about record locking and atomic operations.
\end{itemize}


\section{Information Representation}
\label{attributes}
There are various structures needed to represent the data required in the
directory database. These are described
briefly below.

\subsection {Object Identifiers}
An important part of the OSI world is the object identifier (OID).

An OID is a hierarchy of numbers used uniquely to describe various objects
within the OSI world: for example, ``1.0.8571.1.1'' describes the FTAM
protocol; ``0.9.2342.19200300.99.1'' 
defines a QUIPU attribute type.
The strings of numbers look ``horrible'' and can be difficult to remember.
As such, QUIPU provides a mapping from OIDs to ``strings'', so that 
easy-to-remember strings can be used in place of the numeric OIDs, for
example ``iso~ftam'' or ``quipuAttributeType''.

The mappings are defined in a set of oidtables which are discussed
in Section~\ref{oidtables}. 

\subsection {Attributes}

\tagfigure[tp]{q-entry}{Structure of an Entry}{qfig:entry}

The directory holds information about various entities, such as a person or
an organisation.  The information is held in an information object,
typically referred to as an {\em entry}. 
An entry consists of a set of one of more attributes
(see Figure~\ref{qfig:entry}).

An attribute is represented by an attribute type and a set of attribute values.
An attribute with a single value is represented as follows:
\begin{quote}	
$<$attribute type$>$ ``='' $<$attribute value$>$
\end{quote}
The attribute type can either be a string (which must be in the attribute
oidtable --- see Section~\ref{oidtables}) or the OID in a dotted decimal
format.

Each attribute value is represented by a string, the format of the attribute value
depends upon the
syntax defined in the oidtables, but in general will be a string.

So
\begin{quote}\begin{verbatim}	
roomNumber = G24
2.5.4.20   = 453-5674
commonName = Colin Robbins & Colin John Robbins
photo      = {ASN} 0308207b4001488001fd...
\end{verbatim}\end{quote}
is an example of an entry, with 4 attributes. The ``commonName'' attribute
is multivalued, 
with the two values separated by the ``\&'' symbol.
Notice how the ``photo'' attribute does not have a specific string format,
so a hexadecimal ``ASN.1'' representation is used.
``2.5.4.20'' is an object identifier represented in numeric form.


\subsection {Names}
\label{quipu_name}
Within an entry, some attributes (generally one) have special importance,
and are called  distinguished attributes.  The collection of these
attributes form the relative distinguished name
(RDN).\index{RDN}

An RDN is usually represented by a single valued attribute, thus
\begin{quote}	\begin{verbatim}
commonName = Alan Turland
organization = University College London
\end{verbatim}\end{quote}
are valid RDNs. 

An RDN made up of multiple distinguished attributes uses the \verb+%+ symbol
to separate the values, for
example
\begin{quote}	\begin{verbatim}
userid = quipu % commonName = Colin Robbins
\end{verbatim}\end{quote}

A distinguished name (DN) is the sequence of RDNs that uniquely define
a node in the directory information tree (DIT).
These are represented as follows:
\begin{quote}
RDN [ ``@'' RDN \ldots]
\end{quote}
So
\begin{quote}\small\begin{verbatim}
countryName = GB @ organization = University College London 
\end{verbatim}\end{quote}
is an example of a valid DN.

In some cases an RDN or DN is specified relative to a node other than the
root in the DIT.
To be unambiguous a DN may be rooted using a leading ``@'', for example
\begin{quote}\small\begin{verbatim}
@ countryName = GB @ organization = University College London 
\end{verbatim}\end{quote}


\section{Directory User Agent}
The directory user agent (DUA) is the entity you as a user will
connect to when you interact with the directory service.
It will help you formulate your queries, wrap them up in the required
protocol, pass them to the
directory, and then show you the results obtained.

The X.500 standard defines only the protocol the DUA should use when talking
to the directory, and as such DUAs come in many varieties, but the basic
concepts are the same.

A DUA talks to the directory using the directory access protocol
(DAP)\index{DAP}.

The following chapters of this part of the manual describe several such
DUA interfaces available in the ISODE.  It is left to system
managers to decide which form of interface is appropriate at your
site, and which to install.  There now follows a brief description of
each interface:

\begin{description}
\item [DISH:]	The {\bf DI}rectory {\bf SH}ell 
		(Chapter~\ref{dish}).\\
		This interface give a user full access to the DAP, and
		as such may be complex for novice users.

\item [SID:]	{\bf S}teve's {\bf I}nterface to {\bf D}ISH 
		(Chapter~\ref{DUA:sid}).\\
		A set of scripts that make DISH usable for novice users.

\item [DE:]	{\bf D}irecory {\bf E}nquires
		(Chapter~\ref{DUA:de}).\\
		A DUA for line mode terminals

\item [FRED:]	{\bf FR}ont {\bf E}nd to {\bf D}ISH
		(Chapter~\ref{DUA:fred}).\\

		FRED provides a white pages user interface, which hides most
		of the complexity of the OSI directory.

\item [SD:]	{\bf S}creen {\bf D}irectory
		(Chapter~\ref{sd}).\\
		A curses-based DUA.

\item [POD:]	{\bf PO}pup {\bf D}irectory
		(Chapter~\ref{pod}).\\
		A DUA for the \xwindows/.

\end{description}

\section{Directory System Agent}

\tagfigure[tp]{q-dsa}{DUA/DSA Interaction}{qfig:dsa}

The directory system agent (DSA) is the entity that generates the results to
requests from a DUA.
It may handle the request itself, ask another DSA (using the directory system
protocol\index{DSP}), or advise the DUA to
contact another DSA (see Figure~\ref{qfig:dsa}).
This part of the directory is discussed more in Part III of this
manual.  Here we shall concentrate on the DUA.


\chapter{DISH}
\label{dish}

This chapter describes DISH, a directory shell interface to the directory.
It provides an interface to the directory in an similar manner
to the way that \MH/ provides an interface to a message handling systems.
As with \MH/, DISH can either be invoked as a single process with the full
repertoire of commands built in, or it can be invoked by individual shell
commands.
This latter style allows DISH to be used with others tools to provide very
flexible access to the directory.

DISH provides a very powerful interface onto the directory and gives a user
access to the full directory access protocol(DAP).
The price of such comprehensiveness is complexity.
DISH has a large number of flags, and both the syntax and volume of typing
required can seem forbidding to the novice user.
One compromise solution is to use DISH to build interfaces which are easier
to use and more intuitive.

To run DISH in interactive mode invoke \pgm{dish}, then issue any of the
commands described in Section~\ref{dish_commands}.
This mode of operation is especially useful for novice users.
The process of connecting to the directory --- binding ---
takes place automatically when the program is invoked, and an unbind
is issued when you quit.  

The arguments to DISH are the same as for bind (see 
Section~\ref{dish_bind}), with the addition of a ``-pipe'' flag
which is used to start DISH in ``shell'' mode (see Section~\ref{dish_shell})
(Note: ``dish -pipe \&'' has the same effect as the shell command ``bind'').

\section {Commands}
\label{dish_commands}

Each of the DISH commands described in this section
has a large number of valid flags.
The full names of the flags are given below; however, the shortest unique
name is sufficient to select the flag.
Similarly, when DISH is used interactively the shortest unique name is taken
for the command name (e.g., ``l'' for ``list'').
There is an exception in that ``sh'' is interpreted as ``showentry'', 
``shown'' must be typed for ``showname''.

The additional flag \tt -help\rm \ can be specified with 
every command to get limited runtime help.

\subsection{Moveto}
\label{dish_start}\index{moveto}
With nearly every command it is possible to supply the distinguished name of
the object you want to reference.
In the syntax used to describe the DISH commands this is represented by 
\begin{quote}\begin{verbatim}
	<object>.  
\end{verbatim}\end{quote}
An initial ``\tt @\rm '' in an object description specifies that the name
is relative to the root of the directory information tree (DIT), 
otherwise the name is taken as being relative to the current
position in the DIT.  
The special name component ``..'' is used to mean ``one
position up'' from the current position.  
Some examples of valid names and their interpretation relative to
how a distinguished name is built up are shown below by the following
sequence of names:

\begin{describe}

\item [\verb+@c=GB@o=University College London@ou=Computer Science+:]
\ \\
The specified object is \\
``c= GB @ o= University College London @ ou= Computer Science''.

\item [\verb+cn=Colin Robbins+:]
describes the object
cn=Colin Robbins relative to the current 
position (``c= GB @ o= University College London @ ou= Computer Science @ cn= Colin Robbins''). 
Spaces in a name will be seen as the start of a separate 
argument by the shell, so the name must be quoted.

\item [\verb+..@cn=Steve Titcombe+:]
describes the entry 
cn=Steve Titcombe at the same level
in the DIT as the current position (``c= GB @ o= University College London @
ou= Computer Science @ cn= Steve Titcombe'').

\end{describe}

Objects can also be expressed in the form of sequence numbers and nicknames.
See Sections~\ref{sequences} and \ref{quipurc}.

When you specify an object, the current position in the DIT is not changed.
To change the current position you should use the command
\begin{quote}\begin{verbatim}
moveto [-[no]pwd] [-[no]check] <position>
\end{verbatim}\end{quote}

The \tt -pwd\rm \ flag tells moveto to print the current position in the
DIT.
The \tt -nocheck\rm \ flag tells moveto NOT to check the entry exists,
normally, moveto will invoke  \verb"read" to check that the named entry exists.

The current position can also be changed with the \tt -move\rm \ flag to 
showentry and list commands (see Sections~\ref{dua:list} and ~\ref{dua:show}).


\subsection{Showentry}\index{showentry}
\label{dua:show}

\begin{quote}\begin{verbatim}
showentry [<object>] [-[no]cache] [-[no]name]
          [-[no]move]
          [<any of the read arguments>]
          [<any of the FRED arguments>]
\end{verbatim}\end{quote}

Showentry will display some or all of the attributes of the specified entry.
A combination of the  read argument flags and your \file{.quipurc} file 
described later in the chapter are used to define which attributes to show.

The \tt -noname\rm \ option tells showentry to show the distinguished 
name of the current entry as well as the attributes.

\tt -cache\rm \ is used to tell showentry to use cached information only ---
do not issue a read operation. 
This is used to see what the DUA has cached, and then get information when
the local DSA is unavailable.

\tt -nocache\rm \ is used to enforce a directory read. This is used to
update the cache directly.

\tt -move\rm \ is used to tell DISH to change the current position in the DIT
to the object specified as well as showing the requested attributes.

See Section~\ref{fred:flags} for a discussion of the flags for FRED.

\subsubsection{read flags}

These flags are used by {\em showentry}, {\em showname} and {\em search}.

\begin{quote}\begin{verbatim}
[-[no]types <attribute-type> *] [-[no]all] 
[-[no]value] [-[no]show] 
[-[no]key] [-edb]
[-proc <syntax> <process>]
\end{verbatim}\end{quote}

The \tt all\rm \ flag request that all attributes are read (default).

The \tt value\rm \ flag reads the attribute value, which is the default, 
the inverse is to read the attribute types only.

The \tt novalue\rm \ flag says print the attribute types only.

The \tt show\rm \ flag is used to make DISH show the requested
attributes (this is the default for read, but not for search).

The \tt key\rm \ flag determines whether the key (attribute type) will 
be shown along with the attribute value.

The \tt edb\rm \ flag requests that the data is displayed in ``EDB'' format,
that is, a format which can be used in QUIPU data files.  This is not such 
a nice format to present users, but is easier to manage in shell scripts.

The \tt type\rm \ flag requests that only the supplied attributes are 
read from the DSA; the inverse of this \tt -notype\rm \ does not 
prevent the attributes being read (as there is no
easy mechanism to perform this within X.500), 
but it does stop them being shown.

The \tt proc\rm \ flag is used to instruct the display routines to use an
external process to display the attribute.
For example if you have a process that displays presentation elements
representing OIDs in a particular
way, then you can use
\begin{quote}\begin{verbatim}
showentry -proc OID /usr/bin/oid_display_process.
\end{verbatim}\end{quote}
All OIDs will then be displayed using ``/usr/bin/oid\_display\_process''.
The process should read the presentation element from the standard input, and
place any textual results on the standard output.

\subsection{List}\index{list}
\label{dua:list}
\begin{quote}\begin{verbatim}
list [<object>] [-nocache] [-noshow] [-[no]move]
\end{verbatim}\end{quote}
This function displays the relative distinguished names of 
the children below your current position in the DIT.
Typically, if there are a large number of children, 
you will not be able to list them all, as a ``sizelimit'' will prevent you.
For example
\begin{quote}\small\begin{verbatim}
Dish -> list
1   commonName=Camayoc
2   commonName=Colin Robbins
3   commonName=Jess
4   commonName=Julian Onions
5   commonName=lancaster
6   commonName=PP Support
7   commonName=QUIPU Support
8   commonName=Tony Roadknight
9   commonName=Hugh Smith
10  commonName=Peter Cowen
(Limit problem)
Dish ->
\end{verbatim}\end{quote}
shows a size limit of 10.
Using the \tt -sizelimit\rm \ service control you can often increase this limit
up to some administrative limit (typically 100) set by the 
DSA you are connected to.

\tt -nocache\rm \ tells list to reread the directory --- do not use cached information.

\tt -noshow\rm \ tells list not to display the names found, but 
still construct sequences (see Section~\ref{sequences}).

\tt -move\rm \ is used to tell DISH to change the current position in the DIT
to the object specified. 

\subsection{Search}\index{search}
\label{search}

\begin{quote}\begin{verbatim}
search  [[-object] <object>] 
        [-baseobject] [-singlelevel] [-subtree]
        [[-filter] <filter>] 
        [-[no]relative] [-[no]searchaliases]
        [-[no]partial] [-hitone]
        [<any of the read arguments>]
        [-fred [-expand] [-full] [-summary] 
               [-nofredseq] [-subdisplay]]
\end{verbatim}\end{quote}

Search the DIT, starting at the object specified, for entries that match
the given filter.
When an entry is found to match the filter, only distinguished name is printed
unless a \tt-show\rm \ option is 
specified, when the attributes are displayed as well.

If no filter is specified, a default filter (which matches any object in
the DIT) is used.
This operation has a similar effect to a 
list operation, but attributes can be shown at the same time.

If no flags are given to a search command, then only one 
parameter is allowed. This is taken to be the
filter and NOT the base object as in ALL the other commands.
The flags \tt -filter\rm \ and \tt -object\rm \ are supplied to allow the
user to specify if both a filter and base object are required.
({\em Note:} only one need be flagged.)
For example, the following are all valid search commands:
\begin{quote}\begin{verbatim}
search
search <filter>
search -filter <filter>
search -object <object>
search -filter <filter> -object <object>
search <filter> -object <object>
search -filter <filter> <object>
\end{verbatim}\end{quote}
whereas 
\begin{quote}\begin{verbatim}
search <object>           (interpreted as a filter !)
search <filter> <object>
\end{verbatim}\end{quote}
are not valid.

The three flags 
\tt -baseobject\rm,  
\tt -singlelevel\rm,  and
\tt -subtree\rm \ 
are mutually exclusive.  They are used to control the depth of the 
search operation.
The \tt -baseobject\rm\ option searches only the specified object.
The \tt -singlelevel\rm \ option specifies all siblings of the 
specified object should be  searched.
The flag \tt -subtree\rm \ searches all the 
levels below and including the current position recursively.
The default is
\tt -singlelevel\rm.

For example:
\begin{quote}\small\begin{verbatim}
search -object 
   "@c=GB@o=University College London@ou=Computer Science"
   -subtree -filter "cn=colin robbins" -type telephoneNumber
   -show
\end{verbatim}\end{quote}
will search the DIT subtree below 
``c=GB @ o=University College London @ 
ou=Computer Science'' for
an exact match of ``colin robbins'', and then call showentry to 
display the {\em telephoneNumber} attribute.

The \tt -norelative\rm \ flag is used to tell \verb"showname" to print the full
distinguished name of the results; otherwise, the name relative to the current
position is printed.

The \tt -[no]searchaliases\rm \ flag directs the search as to whether aliases
encountered in the search should be dereferenced and thus searched.
This is different to the \tt -[dont]dereferencealias\rm \ service control which
defines what happens to 
aliases found when locating the base object of the search, whilst the 
\tt -[no]searchaliases\rm \ controls aliases encountered in the siblings of 
base search object.
The default is ``\tt -dereferencealias -nosearchaliases\rm ''.

The \tt -hitone\rm \ flag is used to tell DISH to consider it an error if
the search returns more than one entry.  This is particularly useful when
using DISH from shell scripts (see Section~\ref{dish_shell}).

From time to time, search will not be able to do the entire search, and will
return partial results; the \tt -nopartial\rm \ directs DISH not to print
the partial result information.  To see the full set of partial results, both
\tt -partial\rm \ and \tt -show\rm \ are needed; otherwise only a summary is
printed.
Unlike referral errors, DISH does not follow these partial references.

\subsubsection{More on Matching}

\begin{figure}[tp]
\begin{center}
\small
\begin{tabular}{|c|l|}
\hline
Class	&	Characters\\
\hline
1       &	BFPV\\
2	&	SCGJKQXZ\\
3	&	DT\\
4	&	L\\
5	&	MN\\
6	&	R\\
0	&	all others (ignored)\\
\hline
\end{tabular}
\end{center}
\caption{Soundex Character Classes}
\label{soundex:classes}
\end{figure}


When searching for an exact match (``=''),
the ``\tt *\rm '' character is used as a wildcard, and a substring
match takes place (unless only a ``\tt *\rm '' is specified, when a {\em
presence} match is used; hence any entry 
containing that attribute will be matched). 
For example, ``\verb+Colin*+'' will 
match any attribute value starting with the 
string ``\verb+Colin+'' and ``\verb+*Robbins+'' any 
attribute value ending in the 
string ``\verb+Robbins+''.
When constructing a substring filter, it should be noted that a QUIPU
DSA can resolve searches of the form \verb+string*+ significantly faster
than searches of the \verb+*string*+ or \verb+*string+ form.

Remember that a large number of the standard attributes match case
independently; for example ``x-tel'' will match ``X-Tel''.

As well as an exact equality match, the 
following types of match can be specified:
\begin{quote}\begin{verbatim}
~=     Approximately equals.
>=     Greater than or equal.
<=     Less than or equal.
\end{verbatim}\end{quote}

The Approximate match method used is DSA dependent.  A QUIPU DSA uses
a soundex based algorithm.  This works by grouping 
similar-sounding characters into classes.  The classes
and their corresponding characters are shown in
Figure~\ref{soundex:classes}.
The first character of a word is always used as the first character of
its corresponding soundex code. Adjacent similar characters are
ignored. Thus, the word ``Robbins'' has soundex
code ``R152''. Since the word
``Robens'' also has soundex code ``R152'', these two words are
approximately equal.
To match multiple words
each of the target words must appear in order in the string
to which it is being compared.  There may, however, be other items in
between the words matched.  For example:  ``Tim Howes'' would match
``Timothy Alan Howes'' since ``Tim'' matches ``Timothy,'' 
``Howes'' matches ``Howes,'' and the matched words are in the
proper order.


Filter items can be linked with {\em and} ``\tt \&\rm '', or {\em or} 
``\tt $|$\rm ''.  Brackets ``\tt( )\rm '' 
should be used to enforce the Boolean 
ordering of the expressions, otherwise the evaluation is left to right.
A {\em not} ``\tt !\rm '' can be used to specify the Boolean {\em not}
operator.

If a filter does not have a type specified, e.g., ``-filter steve'',
then an approximate match for the specified common name is assumed
(in this case ``-filter cn$\tilde{\ }$=steve'' is assumed).

A more complex example of a filter 
that searches for anybody whose is a member of staff 
or a student, who has a ``drink'' attribute,
whose name approximately matches ``steve'', but whose surname is 
not ``Kille'':
\begin{quote}\small\begin{verbatim}
cn~=steve & (userClass = staff | userClass = student) &
            (! surname = Kille) & drink = *
\end{verbatim}\end{quote}

A list of all the attributes that can be used to search for an entry
is given in 
Chapter~\ref{syntaxes}, together with a description of the type
matching allowed.

\subsubsection{FRED flags}\label{fred:flags}
These flags are used by FRED,
and are activated when the \tt -fred\rm \ flag is given.

The \tt -expand\rm \ flag indicates that any matched entry should be
displayed in full followed by children entries.

The \tt -full\rm \ flag indicates that if more than one entry is matched on a
search,
then all entries should be displayed in full.

The \tt -summary\rm \ flag indicates that a one-line display should be used for
the entry.

The \tt -subdisplay\rm \ flag indicates that a one-line display should be used
for the entry, followed by the entry's children.

The \tt -nofredseq\rm \ flag indicates that the entry (and children) should
not be added to the current sequence.

The \tt -fredlist\rm \ flag indicates that any attributes containing
DNs should be flagged when using the DA service.

\subsection {Add Entry}\index{add}
\label{dish:add}

\begin{quote}\begin{verbatim}
add [<object>] [-draft <draft> [-noedit]] 
    [-template <draft>] 
    [-newdraft] [-objectclass <objectclass>]
\end{verbatim}\end{quote}

Add is used to add entries to the DIT.
This invokes editentry
on the draft entry.
If there is not a draft entry specified, a draft entry of the specified 
objectclass (default --- thornperson \& quipuObject) is created.
The default draft file is \file{.dishdraft} in your home directory, this can be
altered with the \tt -draft\rm \ option.

When editing has finished, an add operation is sent to the directory
(the \tt -noedit\rm \ flag following a \tt -draft\rm \ stops the editor
being invoked).

\tt -newdraft\rm \ causes the current draft to be overwritten with a new
template of the appropriate objectclass.

\tt -template\rm \ is used to specify a template that should be used during
editing.

On successful completion of the add, the draft entry is renamed with a suffix of
``.old''.

\subsection{Editentry}\index{editentry}

\begin{quote}\begin{verbatim}
editentry <draft>
\end{verbatim}\end{quote}

This option is only available from the shell, and not from the DISH
program.  It is generally not invoked directly by a user, but by the
other DISH commands.

editentry invokes an editor (as defined by the users
\$EDITOR environment variable) on the current draft entry.
In a future release it is hoped to have an editor that knows about the syntax
of an entry, and thus ensures that the entry you have supplied is syntactically
correct.

\subsection {Delete Entry}\index{delete}
\begin{quote}\begin{verbatim}
delete [<object>]
\end{verbatim}\end{quote}
The specified entry is removed from the DIT (remember that 
X.500 only allows leaf
entries to be deleted).

\subsection {Modify Entry}\index{modify}
\label{dish:modify}
\begin{quote}\begin{verbatim}
modify [<object>] [-draft <draft> [-noedit]] [-newdraft] 
       [-add <attribute type>=<attribute value>]
       [-remove <attribute type>=<attribute value>]
       [<any of the showentry arguments>]
\end{verbatim}\end{quote}
Modify is used to modify existing entries in the DIT, and has two modes of
operation. The first is used to modify specific attributes and uses the {\tt
-add} and {\tt -remove} flags to add and remove single attributes and
values. Many of these can be strung together in the same command, e.g.,
\begin{quote}\smaller\begin{verbatim}
modify -add "description=new attribute" -remove "drink=Chocolate".
\end{verbatim}\end{quote}

The second method is used for altering many attributes at the same time.
Initially ``\tt showentry -all -edb -nocache\rm ''
is used to get the current DIT entry, and place a copy of it in your
.dishdraft file.  If the \tt -draft\rm \ option is 
specified then the given file is used.
After editing, any changes to the entry are sent to the directory
(the \tt -noedit\rm \ flag following a \tt -draft\rm \ stops the editor
being invoked).

The draft file is handled in the same way as with add, except that when a
new draft is created, the current values of the attributes are read from the
directory.

\subsection {ModifyRDN}\index{modifyrdn}
\begin{quote}\begin{verbatim}
modifyrdn [<object>] -name <newrdn> [-[no]delete]
\end{verbatim}\end{quote}
This is used to modify the RDN of an entry.
The RDN cannot be changed using the modify operation (remember, as
with delete X.500 only allows this operation on leaf nodes).

The \tt -nodelete\rm \ flag is used to prevent the old RDN being removed as
an attribute of the entry.


\subsection{Showname}\index{showname}
\label{showname}

\begin{quote}\begin{verbatim}
showname [<object>] [-[no]compact] [-[no]cache] [-[no]ufn]
         [<any of the read arguments>]
\end{verbatim}\end{quote}

A name can be printed either in compact form (as seen in EDB or
.dishdraft files), just showing the distinguished
name, in a UFN format (default), or by showing the distinguished
attributes one per line.

\subsection{Compare}\index{compare}
\label{dua:compare}
\begin{quote}\begin{verbatim}
compare [<object>] -attribute <attribute> [-[no]print]
\end{verbatim}\end{quote}
For example
\begin{quote}\begin{verbatim}
compare userclass=student
\end{verbatim}\end{quote}
This command has a return value of true or false (1 or 0).

If \tt -print\rm \  is specified the strings ``TRUE'' or ``FALSE'' are printed.

\subsection {Squid}\index{squid}

\begin{quote}\begin{verbatim}
squid [-sequence [<name>]] 
      [-alias <object>] [-version]
      [-user] [-syntax]
      [-fred]
\end{verbatim}\end{quote}

The command \pgm {squid} (status QUIPU DISH) with no 
parameters will inform you of the current status
of your DISH process for example:
\begin{quote}\begin{verbatim}
Connected to Armadillo at Internet=128.40.16.220+2005
Current position: @c=GB@o=Nottingham University
User name: @c=GB@o=University College London@
           ou=Computer Science@cn=Colin Robbins
Current sequence 'default'
\end{verbatim}\end{quote}

\tt -sequence\rm \ makes squid print the specified sequence.
Sequences are described in Section~\ref{sequences}.

\tt -alias\rm \ adds the given DN to the current sequence.

\tt -version\rm \ prints the version number of the DISH process and
associated ISODE and DSAP libraries.

\tt -user\rm\ prints the DN you are bound as.

\tt -syntax\rm\ prints the directory syntaxes known to DISH.

\tt -fred\rm\ must be the first argument if supplied. It causes
any Distinguished names printed by the command to be shown
using the user-friendly naming
notation rather than the DN notation.

\subsection {Bind}\index{bind}
\label{dish_bind}

\begin{quote}\begin{verbatim}
bind [-noconnect] [[-user] [<username>]] 
     [-password [<password>]] [-[no]refer]
     [-call <dsaname>]
     [-simple] [-protected] [-strong] [-noauthentication]
\end{verbatim}\end{quote}

This is used to (re)connect to the directory.  This is not normally
required, as each command will bind as necessary, but allows advanced users
to contact specified DSAs and change their username.

\tt -noconnect\rm \ is used to start DISH (and cache) without connection to a
directory.

\tt -[no]refer\rm \ is used to tell DISH whether to automatically follow
referrals issued by the DSA, by default they are followed.

To connect to the directory you need to be authenticated.
There are four flags to control the level of authentication DISH
attempts to use.
The lowest level of authentication is ``anonymous'', which is used
when you do not supply a distinguished name.  A DSA may respond with
``inappropriate authentication'' which means you should use a higher
level of authentication.
The next level of authentication is achieved by supplying a 
Distinguished Name only, with no password.  The \tt
-noauthentication\rm \ is used to tell DISH not to prompt for a password.
The next levels are ``simple'' and ``protected simple'' authentication which
use the \tt -simple\rm \ and \tt -protected\rm \ flags respectively.
These present a textual password to the
directory which the directory the uses to authenticate you.
Using \tt -simple\rm \ the password is sent across the network ``in the
clear,'' whereas \tt -protected\rm \ encrypts it; thus you are advised to
used \tt -protected\rm \ if your directory supports it.

The \tt -strong\rm \ flag tells DISH to send strong authentication
credentials to the directory. This is not fully implemented in this version
of QUIPU.

To connect to the DSA using ``simple'' or ``protected'' authentication, 
you must supply a
distinguished name (\tt -user\rm ) and password (\tt -password\rm ).
This, however, reveals your password to anybody who is looking at your 
terminal but is useful for issuing bind operations from
a shell script. To prevent this, you need to place the 
password in your \unix/ protected \file{.quipurc} (see Section~\ref{quipurc}).
Alternatively, if you do not supply a password, you will be prompted
for one (with echoing turned off).


\subsection {Unbind}\index{unbind}

\begin{quote}\begin{verbatim}
unbind [-noquit]
\end{verbatim}\end{quote}
This is used to break the connection to a DSA.
If \tt -noquit\rm \ is specified, the DISH process does not die (This is used to
maintain the cache).

To be ``friendly''
to other users, you should unbind 
when you have finished using the directory --- put an \tt unbind\rm \ in
your \file{.logout} file if you use DISH from the shell.

\subsection{FRED}
\label{dish_fred}

\begin{small}\begin{quote}\begin{verbatim}
fred [-display <name>]
     [-dm2dn [-list] [-phone] [-photo] <domain-or-mailbox>]
     [-expand [-full] <DN>]
     [-ufn [-list,][-mailbox,][-phone,]
           [-photo,][-options %x,]<name...>]
     [-ufnrc <list...>]
\end{verbatim}\end{quote}\end{small}

This is used by the FRED program to provide some functions which would
otherwise  require
complicated interactions between FRED and DISH.

\tt -display\rm\ is used to set the address of an \xwindows/ display.

\tt -dm2dn\rm\ is used to map a domain name from the DNS or a mailbox into a
distinguished name in the directory.
The \tt -list\rm\ suboption returns a list of possible matches, 
\tt -phone\rm\ returns the phone number attribute of the matched
entry, and
\tt -photo\rm\ returns the photo attribute of the matched entry.

\tt -expand\rm\ is used to see if a node has children.

\tt -ufn\rm\ is used to perform a user-friendly naming match.
The \tt -list\rm\ suboption returns a list of possible matches, 
\tt -mailbox\rm\ returns the mailbox attribute of the matched entry,
\tt -phone\rm\ returns the phone number attribute of the matched entry,
\tt -photo\rm\ returns the photo attribute of the matched entry, and
\tt -options\rm\ is used to customise the behavior of the matching algorithm.

\tt -ufnrc\rm\ is used to set the search list for user-friendly naming.

\section {Sequences}
\label{sequences}

Results returned by {\em search} and {\em list} operations 
may be long, and for 
this reason have a reference number printed beside them.
The reference number can be used as the ``object'' in any of the calls to the 
directory.  Thus
\begin{quote}\begin{verbatim}
showentry 6
\end{verbatim}\end{quote}
shows the entry labelled with the sequence number
6 by a previous {\em search} or {\em list} operation.

When you come into DISH
by default, the resultant numbers of list and search operations 
are added to a sequence called ``default''.

Two flags
\begin{quote}\begin{verbatim}
[-sequence <name>] [-nosequence]
\end{verbatim}\end{quote}
are used to control sequences.
The \tt -nosequence\rm \ flag is use to completely remove the 
concept of sequences (so no numbers are printed by list
or search).

To get a different sequence (hence renumber from 1 again), you can change
this to mysequence with a 
\begin{quote}\begin{verbatim}
-sequence mysequence
\end{verbatim}\end{quote}
flag.

The special keyword \tt reset\rm \ is used to reset the current sequence,
thus causing future operations to be renumbered from 1.
This is different to setting a new sequence, as the old sequence values are
removed.

Occasionally you will want to search an entry defined in one sequence,
putting the results in another. For this purpose the special token
\verb+result+ is recognised, thus
\begin{quote}\begin{verbatim}
search 4 -sequence xxx -sequence result yyy
\end{verbatim}\end{quote}
will search entry 4 as defined by sequence ``xxx'', but place the results in
sequence ``yyy''.

\section {Service Controls}
\label{dish_serv}

All the commands described in Sections~\ref{dua:show} 
through \ref{dua:compare} 
have additional flags to control the type of service provided.  The
advice to a novice user is let these flags take their default values. 

The flags recognised are listed below:

\begin{describe}

\item [\tt -(no)preferchain\rm :] Advise the DSA (not) to chain the operation if
required --- The DSA {\bf IS} allowed to ignore the advice!

\item [\tt -(no)chaining\rm :] (Prohibit) Allow the use of chaining.

\item [\tt -(dont)usecopy\rm :] (Prohibit) Allow the use of a cached or slave 
copy of the data.  
Note that this refers to data cached in the DSA; there are separate flags
(already described)
for data cached in the DUA.

\item [\tt -(dont)dereferencealias\rm :] (Do not) dereference aliases if found in
the path of a query.

\item [\tt -low\rm :] Flag the query as low priority.
\item [\tt -medium\rm :] Flag the query as medium (default) priority.
\item [\tt -high\rm :]  Flag the query as high priority.

\item [\tt -timelimit $n$\rm :] Set the time limit to $n$ seconds.
\item [\tt -notimelimit\rm :] Do not specify a time limit.

\item [\tt -sizelimit $n$\rm :] Set the size limit to $n$ entries.
\item [\tt -nosizelimit\rm :] Do not specify a size limit.

\item [\tt -(no)localscope\rm :] (Do not) limit operation to local scope.

\item [\tt -(no)refer\rm :] (Do not) automatically follow referrals issued by the DSA.

\item [\tt -strong\rm :]	Ask the DSA to use strong authentication when sending the
results.

\end{describe}



\section {Tailoring}

There are two levels of tailoring that control the operation of the DISH DUA.
First of all there is the system wide tailor file \file{dsaptailor}
(found in the ISODE \verb"ETCDIR" directory, usually \file{/usr/etc/}), and
then users private tailor file \file{.quipurc}.

The file \file{dsaptailor} is described in Section~\ref{dua:tailor}.

\subsection {.quipurc}
\label{quipurc}\index{.quipurc}
The file \file{.quipurc} is read by DISH on start up, and
has the following format:

\begin{quote}\begin{verbatim}
flag:  value
\end{verbatim}\end{quote}

The following flags are currently recognised:
\begin{describe}

\item [\tt username:] The name of the user to be used when binding.

\item [\tt password:] The password to use when binding.  For this reason care
should be taken to ensure you .quipurc file is not publicly readable.

\item [\tt cache\_time:] How long to keep the DISH process alive after unbinding
(specified in minutes).

\item [\tt connect\_time:] How long to keep a connection open, if there is no
activity (specified in minutes).

\item [\tt service:] A list of default service control flags 
(as defined in Section~\ref{dish_serv}).

\item [\tt sequence:] Add the supplied DN parameter to the default sequence.

\item [\tt dsap:] This can be followed by one of the \file{dsaptailor}
options
described in Section~\ref{dua:tailor}.\index{dsaptailor}

\item [\tt isode:] 
This can be followed by one of the ISODE tailor options
described in Volume~1 of the ISODE manual.

\item [\tt notype:] The argument is expected to be an attribute type.
This attribute will not be displayed by showentry by default.

\item [\tt type:] The only valid argument for this entry is ``unknown,'' and says 
that unknown attributes should be shown; by default, they are not shown.

\item [\tt certificate:] Used for strong authentication.

\item [\tt secret\_key:] Used for strong authentication.

\item [\tt $<$dish command$>$:] A list of default flags 
to be used when ``command'' is used.
For example:
\begin{quote}\begin{verbatim}
showname: -compact
\end{verbatim}\end{quote}
says that \verb|showname| should always be invoked with the 
\tt -compact\rm \ flag set.

\item [\tt $<$nickname$>$:] A nickname entry, for example
\begin{quote}\small\begin{verbatim}
cjr: "c=GB@o=X-Tel Services Ltd@cn=Colin Robbins"
\end{verbatim}\end{quote}
This can then be used in distinguished name arguments, for example
\begin{quote}\begin{verbatim}
showentry cjr
\end{verbatim} \end{quote}
will show the entry for ``Colin Robbins''.

\end{describe}


\section {Remote Management of the DSA}\label{dua:mgmt}\index{dsacontrol}
\begin{quote}\begin{verbatim}
dsacontrol [-[un]lock <entry>] [-dump <directory>] 
           [-tailor <string>] [-abort] 
           [-restart] [-info] [-refresh <entry>]
           [-resync <entry>] [-slave [<entry>]]
\end{verbatim}\end{quote}

If the DISH program is bound to the DSA as the directory manager
(See Section~\ref{dish_bind} on binding),
then it may be used (primitively) to manage the DSA.
This feature is discussed in \cite{QUIPU.Design} and Section~\ref{dsa:mgmt}
of this manual.

\tt (un)lock\rm \ used to (un)lock subtrees of the DIT held locally. This
(allows) prevents users to modify the DIT.

\tt dump\rm \ causes a copy of the local DIT to be dumped into the specified 
directory.
Note that when the DSA dumps its in-core database,
it does so relative to the possibly remote \unix/ directory from which the
DSA was started,
not the directory that DISH was invoked in.
So it is best to specify path names in full to ensure the database is dumped
where you expect it to be.

\tt tailor\rm \ is used to pass tailor command to a running DSA.  The format of
``string'' is the same as a line in the \file{quiputailor} file.
For example:
\begin{quote}\begin{verbatim}
dsacontrol -tailor "dsaplog level=all"
\end{verbatim}\end{quote}
will turn the DSA logging on full.

\tt refresh\rm \ tells the DSA to reread subtree ``entry'' from disk.
As with all dsacontrol commands, entry must be the full 
DN of the target
entry, the DISH concept of current position is not used.
Also, if relating to a disk operation, the CASE of the DN must be correct
as well.

\tt resync\rm \ tells the DSA to re-write the subtree entry to disk.

\tt slave\rm \ tell the DSA to update its SLAVE EDB files.
The keyword root is used to identify the root EDB file.
The keyword {\em shadow} is used to tell the DSA to update any entries
it has shadow copies of.
See Section~\ref{slave_update} for details.

\tt abort\rm \ tells the DSA to stop.
\tt restart\rm \ tells the DSA to stop, then restart.

\tt info\rm \ asks the DSA to return some statistics about its database.

The error messages returned from a DSA are somewhat obscure when using
dsacontrol. 
There are four things DISH may report: 
\begin{enumerate}
\item \verb+DONE+: the operation has been successfully completed; 
\item \verb+Scheduled+: the operation has been scheduled, and will be
completed as soon as possible;
\item \verb+Access rights insufficient+: you are not bound as the DSA manager;
\item \verb+Unwilling to perform+:  the DSA was unable to perform 
the task, this generally means the argument to dsacontrol was wrong.
\end{enumerate}

\section {Caching in the DUA}
\label {dua_cache}

All results returned from read and list operations are cached in memory
in the DUAs.  The cache is then used if possible to answer queries.
The cache is kept for {\em cache\_time} minutes as specified in the
\file{tailor} files.

Disk caching has not been implemented yet,
but it is intended to extend the in-core caching, 
so that the cached data can be written to disk.  When the DUAs start up, the
user will have the option to read this cache.
Similar support for private data may be provided.

\section {Running DISH from the Shell}
\label{dish_shell}

Each of the DISH commands mentioned in this chapter
can be applied directly to the shell, without having to first invoke
DISH.  This has the advantage that the result can be piped through
programs like \pgm{more} thus giving the users flexibility in how
they view the data.

Chapter~\ref{quipu:install} describes how to install this extra feature of
DISH.

Each time a new shell is invoked and a DISH command entered, a new DISH will
be invoked.  This is sometimes not required.  To prevent this from happening
an environment variable DISHPROC can be set, and should be the TCP
address to use for communication, for example:
\begin{quote}\begin{verbatim}
127.0.0.1 12345
\end{verbatim}\end{quote}
where 127.0.0.1 is the IP address, and 12345 is the TCP port.
The two example scripts show how the TCP port may be generated if required
under \verb+sh+.
For \verb+csh+ you might use
\begin{quote}\begin{verbatim}
if (${?DISHPROC} == 0) then
        setenv DISHPROC "127.0.0.1 `expr $$ + 10000`"
endif
\end{verbatim}\end{quote}

\subsection {DISHinit}
A program called \pgm{dishinit} 
\label{dishinit}
can be use to 
bind to the directory as the manager, read certain information
and create a 
default \verb".quipurc"
file for a specified user.
This is a useful utility for new databases.
If the \verb"dsaptailor" variable \verb"dishinit" is set to \verb"on",
dishinit will be invoked by DISH whenever is cannot find a 
.quipurc file.

\verb"dishinit" needs three parameters to be set in order to bind as the 
manager, and search the local subtree.
These parameters are defined at the top of the dishinit script and are:
\begin{describe}
\item [\verb+manager+:] The DN of the manager of the local DSA.
\item [\verb+password+:] The manager's password (the script should be protected 
with \unix/ file protection).
\item [\verb+position+:] The DN of the local sub-set of the DIT. This is where 
the entries for users will be looked for
\end{describe}

\subsection{Example Scripts}
Two example scripts have been supplied as part of ISODE, and can be found in
the \file{others/quipu/uips/dish} directory.
\begin{describe}
\item [\verb+dsaping+:] Contacts all the DSAs at a given level in the DIT, and
gathers some statistics.
\item [\verb+dsalist+:] Produces a list of all the known DSAs in the DIT.
\end{describe}

\subsection {Files}


When the multiple command version of DISH is compiled on a machine that does
not support sockets, name pipes are used 
to communicate with the
a background process.

The file /tmp/dish$<$pid$>$ is used by DISH to write results to the calling
processes, where the pid is the process id of the calling process.

The file /tmp/dish--$<$tty$>$--$<$uid$>$ is used by calling process to send commands to
DISH, where {\em tty} is the users terminal number, and {\em uid} their user id.


\chapter {SID}
\index{sid}
\label{DUA:sid}
SID (Steve's interface to DISH) is a directory user agent, designed to be
incredibly easy to use.
It is implemented as a simple set of scripts to the DISH DUA.

There are a lot of flags to DISH. This can make it awkward to use, and in
general shows too much directory-specific information.
FRED has a centralised view of the world, and does not give
the advantages of a shell based interface.

SID is a simple approach of giving appropriate aliases to access DISH
commands.   It is targetted for searching for people and organisations.
It may be useful ``as is'' to others, and also give an indication as to how
DISH may be used.


\section {Quickstart}


\begin{describe}
\item[WARNING] Before you start to try SID, you must have the variable
DISHPROC set in your environment.  The mechanism suggested in
Section~\ref{profile} is a good way to achieve this.  To experiment,
csh users can type the incantation:
\end{describe}
\begin{quote}\begin{verbatim}
% setenv DISHPROC "127.0.0.1 `expr $$ + 1000`"
\end{verbatim}\end{quote}


To find users within your Organisation is very easy.  
The default setup is optimised for
your particular Organisational Unit.  Simply use the command {\bf psearch}
with a single argument to identify the user looked for.  This will identify
all users with this key as a substring.  For example:

\begin{quote}\small\begin{verbatim}
% psearch steve
countryName=GB
organisation=University College London
department=Computer Science

1   cn=Steve Kille
telephoneNumber    - +44-1-380-7294 (work)
telephoneNumber    - 01-350-2888 (home)
2   cn=Stephen Titcombe
telephoneNumber    - +44-1-387-7050 x 3674/5 
telephoneNumber    - Term = 01-388-4741
3   cn=Stephen Sackin
4   cn=Steve Wilbur
telephoneNumber    - +44-1-380-7287
5   cn=Stephen Usher
telephoneNumber    - +44-1-387-7050 x 3674/5 
6   cn=Kevin Steptoe
7   cn=Tomasz Stepniak
8   cn=Stefan Penter
telephoneNumber    - +44-1-387-7050 x 3674/5 
9   cn=Stephania Loizidou
10  cn=Steve Hodges
telephoneNumber    - +44-1-387-7050 x 3674/5 
11  cn=Steve Davey
telephoneNumber    - +44-1-387-7050 x 7280 or 7289
12  cn=Steve Chan
telephoneNumber    - +44-1-387-7050 x 3674/5 
13  cn=Steven Britton
telephoneNumber    - +44-1-387-7050 x 3715
14  cn=Stephen Beckles
telephoneNumber    - +44-1-387-7050 x 3674/5 
15  cn=Steven Bacall
telephoneNumber    - +44-1-387-7050 x 3674/5 
telephoneNumber    - dd 01-387-6978 (Campbell House)
\end{verbatim}\end{quote}


Each match will have an associated number.  To show a given entry, just use
the command {\bf showentry}, with the number identifying the entry as the
single argument.  For example, to show entry 1 (Steve Kille):

\begin{quote}\smaller\begin{verbatim}
% showentry 1
c=GB@o=University College London@ou=Computer Science@cn=Steve Kille
Address            - 35 Elspeth Road
                     London
                     SW11 1DW
userClass - csresstaff
photo - (See X window, pid 598)
roomNumber         - G24
favouriteDrink     - Pinta - Brakspears
info               - Directory worker
rfc822Mailbox      - S.Kille@cs.ucl.ac.uk
textEncodedORaddress - /Pn=Stephen.Kille/Ou=cs/O=ucl/ \
                     Prmd=uk.ac/admd=gold 400/c=gb/
userid             - steve
telephoneNumber    - +44-1-380-7294 (work)
telephoneNumber    - 01-350-2888 (home)
description        - New Description
surname            - Kille
name               - Steve E. Kille
name               - Stephen Kille
name               - Steve Kille
\end{verbatim}\end{quote}

If you are running the \xwindows/ a picture will be displayed if it is present.

If you wish to search all of your Organisation, simply give a second argument 
to {\bf
psearch}.  For example, to search UCL give ``ucl'' as shown:

\begin{quote}\small\begin{verbatim}
% psearch baker ucl
Searching for People matching "baker" under:
countryName=GB
organisation=University College London

4   ou=French Lang and Lit@cn=F Baker
telephoneNumber  - 3077
5   ou=Economics@cn=V Bashkar
telephoneNumber  - 2286
6   ou=Biochemistry@cn=J Basra
telephoneNumber  - 2185
7   ou=Bartlett School@cn=H Bowker
telephoneNumber  - 7453
8   ou=Anatomy and Embryol@cn=D Becker
telephoneNumber  - 3293
9   ou=Computer Science@cn=Malcolm Booker
telephoneNumber  - +44-1-387-7050 x 3645 
10  ou=Computer Science@cn=Imtiaz Bashir
telephoneNumber  - +44-1-387-7050 x 7304 
11  ou=Computer Science@cn=Benjamin Bacarisse
telephoneNumber  - +44-1-380-7212 (work)
telephoneNumber  - 01 987 2746 (home)
12  ou=Surgery@cn=L Baker
telephoneNumber  - 82-5255
13  ou=History Of Art@cn=B A Boucher
telephoneNumber  - 7210
14  ou=Geological Sciences@cn=J R Baker
telephoneNumber  - 2382
15  ou=Secretary's Office@cn=I H Baker
telephoneNumber  - 3000
\end{verbatim}\end{quote}

\pagebreak[3]
\section {Example Usage}

A typical sequence to find someone outside your organisation is now shown:


\begin{quote}\small\begin{verbatim}
% osearch nott
Searching in country "gb" for Organisations matching "nott"
1   o=Nottingham University
telephoneNumber  - +44-0602-484848x2862
2   o=NPL
3   o=JNT
telephoneNumber  - +44-0235-445724
% psearch smith 1
Searching for People matching "smith" under:
countryName=GB
organisation=Nottingham University

4   ou=Computer Science@cn=Hugh Smith
telephoneNumber  - extn 2862 (Departmental Secretary)
% showentry 4
c=GB@o=Nottingham University@ou=Computer Science@cn=Hugh Smith
photo            - (See X window, pid 1076)
telephoneNumber  - extn 2862 (Departmental Secretary)
description      - lecturer
surname          - Smith
name             - Hugh.Smith
name             - Hugh Smith
\end{verbatim}\end{quote}



\section {Use of Nicknames}

It is convenient to have nicknames (private aliases) set up for commonly accessed parts of the
directory information tree.  These can be set up in the QUIPU profile (see
Section~\ref{profile}), which sets up the nickname ``us''.
Two nicknames used in the examples are:

\begin{describe}
\item[\verb+ucl+:]  Country GB; Organsisation University College London.
\item[\verb+cs+:] Country GB; Organsisation University College London;
Organisational Unit Computer Science.
\end{describe}

Aliases may be used as an alternative to sequence numbers.

\section {SID Commands}

The initial SID commands are:

\begin{describe}

\item[\verb+osearch+:] (Organisation Search) Search for an organisation.  The first argument is the
organisation being searched for.  For example:

\begin{quote}\begin{verbatim}
% osearch nott
\end{verbatim}\end{quote}

This searches for any organisations approximating to this string, or
containing it as a substring.

The second (optional) argument is the two letter country code (e.g., GB, US,
ES, etc).  Use {\bf clist} to find these codes.  The default country is the
previous one searched.  Example:

\begin{quote}\begin{verbatim}
% osearch psi us
\end{verbatim}\end{quote}

\item[\verb+psearch+:] (Person Search) Search for a person.  The first argument is a key to identify
the person.  Approximate and substring matches are made.
There are two ways of using psearch.  


\begin{enumerate}
\item Search an identified organisation, using the second argument to
osearch.  Example where the organisation is identified by a nickname:

\begin{quote}\begin{verbatim}
% psearch kirstein ucl
\end{verbatim}\end{quote}

Example where the organisation is identified by a sequence number, 
returned by a previous osearch command.

\begin{quote}\begin{verbatim}
% psearch steve 29 
\end{verbatim}\end{quote}


\item Move explicitly to an organistion, and then use psearch with only one
argument.  This is useful when repeated searches will be made from one point.
Example:

\begin{quote}\begin{verbatim}
% moveto  ucl
% psearch kirstein
\end{verbatim}\end{quote}

\end{enumerate}



\item[\verb+clist+:] (Country List) Used to give a list of countries, with friendly descriptions,
as well as the two-letter code.

\item[\verb+ousearch+:] (Organisational Unit Search) Search for an OU.  Analogous to psearch.  This is
appropriate for use where organisations are large, and a full search takes
too long.

\item[\verb+dlist+:] (Directory List) This uses the search command to provide a listing of directory
information, without a clutter of wild South American animals.

\end{describe}

\section {Standard DISH Commands}

The following DISH commands are useful ``as is'' for normal users, and are
considered to be a part of SID.

\begin{describe}
\item[\verb+showentry+:] Show an entry.  The single argument is typically the numeric
key returned by a search.  This will be used a lot, and it is likely to be
worthwhile setting up an alias.  The author uses ds (directory show).
Example, using a sequence number returned by a search:

\begin{quote}\begin{verbatim}
% showentry 36
\end{verbatim}\end{quote}


\item[\verb+moveto+:] Move to a defined location.  Example of moving to a nickname
defined location:

\begin{quote}\begin{verbatim}
% moveto ucl
\end{verbatim}\end{quote}


\item[\verb+modify+:] To modify your own entry, this command is used with the single
argument ``me''.  For example:
\begin{quote}\begin{verbatim}
% modify me
\end{verbatim}\end{quote}

\end{describe}


\section {QUIPU Profile}
\label {profile}
This is a suggested template for \file{.quipurc}, note that this is very
similar to the default file installed by \pgm{dishinit} described in
Section~\ref{dishinit}.


\begin{quote}\small\begin{verbatim}
username:c=GB@o=University College London@
    ou=Computer Science@cn=Steve Kille
me:c=GB@o=University College London@
    ou=Computer Science@cn=Steve Kille
password:steve

position:@c=GB@o=University College London@ou=Computer Science
notype: acl
notype: treestructure
notype: masterdsa
notype: slavedsa
notype: objectclass
notype: lastmodifiedby
notype: lastmodifiedtime
notype: userpassword
cache_time: 30
connect_time: 2

us: c=us
moveto: -pwd
showentry: -name

\end{verbatim}\end{quote}

You must set the environment variable DISHPROC.  This can be done by the
following in the \file{.login} file (csh):

\begin{quote}\begin{verbatim}
if (${?DISHPROC} == 0) then
        setenv DISHPROC "127.0.0.1 `expr $$ + 1000`"
endif
\end{verbatim}\end{quote}

