% run this through LaTeX with the appropriate wrapper

\chapter	{The ISO Aliases Database}\label{isoaliases}
The database \file{isoaliases} in the ISODE \verb"ETCDIR" directory
(usually \file{/usr/etc/})
contains a simple mapping between names (terse strings)
and values (e.g., user-friendly names and distinguished names).

The database itself is an ordinary ASCII text file containing an entry for
each locally defined alias.
Each entry contains
\begin{itemize}
\item	the alias, a simple string; and,

\item	a user-friendly name or a distinguished name.
\end{itemize}
Blanks and/or tab characters are used to seperate items.
However, double-quotes may be used to prevent separation for items containing
embedded whitspace.
The sharp character (`\verb"#"') at the beginning of a line indicates a
commentary line.

\section	{Accessing the Database}\label{isoalias}
The \man libacsap(3n) library contains the routines used to access the
database.
There is one high-level routine, \verb"alias2name"
which returns the value which corresponds to an alias in the database.
\begin{quote}\index{alias2name}\small\begin{verbatim}
char   *alias2name (name)
char   *name;
\end{verbatim}\end{quote}
The parameter to this procedure is:
\begin{describe}
\item[\verb"name":] the alias to lookup.
\end{describe}
This returns the manifest constant \verb"NULLCP" if the given alias is
not in the database.

In order to load specific aliases other than those read in the
\man isoaliases(5) file, use the routine \verb"add_alias":
\begin{quote}\index{add\_alias}\small\begin{verbatim}
int     add_alias (name, value)
char   *name,
       *value;
\end{verbatim}\end{quote}
The parameters to this procedure are:
\begin{describe}
\item[\verb"name":] the alias to enter; and,

\item[\verb"value":] its value.
\end{describe}
This returns the manifest constant \verb"NOTOK" if the given alias cannot
be added.

\section	{User-Specific Aliases}
By default a user-specific aliases database is consulted before the
system-wide aliases file.
The user-specific file is called \file{\$HOME/.isode\_aliases} in the user's
home directory.
