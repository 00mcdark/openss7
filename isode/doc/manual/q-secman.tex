% Run this file through LaTex with the appropriate wrapper.

\chapter{Security Management}
\label{secman}

\section{QUIPU Userid}

The DSA should not run under the userid of the super-user (root). The
reason for this is that no application should be run as root unless it is
absolutely necessary, just in case an undetected bug causes it to malfunction.
(A process running as an ordinary user will be halted if it starts to
misbehave; one running as root might carry on and delete important files, for
example).

Ideally, the DSA should be assigned a special userid to run under, called,
for example, ``quipu''. If this is not possible, it should run under the
userid of the DSA manager. Whichever option is taken, we shall henceforth refer
to the selected userid as ``quipu'' in the documentation.

\section{Shared Libraries}
\index{shared libraries}

If using QUIPU, in particular the DUAs, compiled which shared
libraries (under SunOS-4) care need to be taken when installing.
The ISODE Makefiles use {\em -L \$(TOPDIR) -ldsap}.  {\em TOPDIR} will
be defined as something like ``\file{../..}''.  Normally, when
installed, there will not be a \file{libdsap} in the ``\file{../..}''
directory so there is no problem.  However if a user were to install
their own \file{libdsap} in the ``\file{../..}'' directory there are
potential security problems.

The problem can be solved by relinking binaries without the
{\em -L \$(TOPDIR)} switch, once the new libraries are installed.
We are looking at ways of solving this within the ISODE Makefile.

\section{File Permissions}

The EDB files are where QUIPU stores the contents of the directory. As these
files potentially contain sensitive information, they should be readable and
writable only by the QUIPU userid.

There are two log files: one for debugging information and one for audit.
Both of these files should be readable and writable only by the QUIPU userid.
There is a potential problem here, in that ISODE's logging code usually makes
log files writable by everybody. To be on the safe side, make a separate
directory for QUIPU's logs (\file{/etc/isode/quipu-logs}, for example). Make this
directory only accessible by the QUIPU userid, and change the tailoring
parameter {\em logdir} so that the logs are written there.

\section{Discretionary Access Control}

The principles of discretionary access control are explained in the
Chapter~\ref{Security} of this manual. This section 
gives guidelines on setting the access
control lists for entries in the DIT.

\subsection{What Must be Publicly Readable}

Some attributes are used by the directory itself for routing and other
purposes. If these attributes are not publicly readable (and hence readable
by all DSAs) then the directory's internal communications may fail. 
If a DUA gives messages such as ``Unavailable'' this is one possible cause.
Note that there are many other possible causes of such failures; Network
congestion or a machine being down is the most likely explanation.

The following attributes ought to be publicly readable:

\begin{itemize}
\item
masterDSA.
\item
slaveDSA.
\item
presentationAddress.
\item
userCertificate.
\item
treeStructure.
\end{itemize}

The {\em userPassword} attribute ought to be comparable by everybody, but not
readable (unless the entry is a spot-shadowed DSA entry, in which case
the userPassword must be publically readable, see Section~\ref{DSA:ownentry}).



\section{Audit}

\subsection{Enabling Auditing}

The {\em stats} parameter in the {\em quiputailor} file controls how much audit
information is kept. It is advisable to enable recording of audit events at 
the {\em notice} level.  More detailed information is given at the 
{\em trace} level.  Both are enabled by default.

\subsection{Relating Events to Users}

Most events in the usage log contain an {\em association descriptor} instead
of the name of the user who caused the event. An association descriptor is
a (small) number which identifies a connection to QUIPU. (It is rather like
knowing which terminal line a command came in on.) To discover the user name,
it is necessary to scan back through the log to find the record for the start
of the association. This will contain the name of the user and how he
or she became authenticated.

\subsection{Format of Audit Records}

Each record in the log file is formatted as follows:

\begin{quote}\begin{verbatim}
<AuditRecord> ::= <month> "/" <day> <time> <process> 
                  <pid> "(" <userid> ")" <Event>
\end{verbatim}\end{quote}

\begin{describe}
\item [\verb+time+:]
The time of the event in hh:mm:ss format.
\item [\verb+process+:]
The name of the program (``quipu'').
\item [\verb+pid+:]
The process id of the DSA.
\item [\verb+userid+:]
The id of the user who started the DSA running. It is {\em not} the id
of the DUA which caused the event!
\item [\verb+Event+:]
is the rest of the message. The following sections describe most of the
common messages.
\end{describe}
	

\subsection{Start of an Association}

\begin{quote}\small\begin{verbatim}
<BindEvent> ::= "Bind" "(" <Integer> ")"
                 "(" <AuthType> ")" 
                 ":" <DN>
<AuthType> ::= "none" | "simple" | "protected" | 
               "strong" | "noauth"
\end{verbatim}\end{quote}

For example:

\begin{quote}\begin{verbatim}
Bind (4) (simple): c=GB@o=University College 
         London@ou=Computer Science@cn=Steve Kille
\end{verbatim}\end{quote}

This means that Steve Kille has started using association descriptor 4, and
proved his identity using simple authentication (i.e., a password).

\subsection{End of an Association}

\begin{quote}\begin{verbatim}
<UnbindEvent> ::= "Unbind" "(" <Integer> ")" <WhoBy> 
      ":" <DN> <WhoBy> ::= "(by this)" | "(by that)"
\end{verbatim}\end{quote}

For example:

\begin{quote}\begin{verbatim}
Unbind (4) (by that): c=GB@o=University College
           London@ou=Computer Science@cn=Steve Kille
\end{verbatim}\end{quote}

This means that Steve Kille's DUA has disconnected from the DSA, and descriptor
4 is left free for use by someone else. The ``(by that)'' means that the DUA,
rather than this DSA, decided to close the connection.

\subsection{DAP Operation}

\begin{quote}\begin{verbatim}
<DAPOperationEvent> ::= <OpType> "(" <Integer> ")"
                         ":" <DN>
<OpType> ::= "Read" | "Search" | "List" | "Compare" |
             "Add" | "Remove" | "Modify" | "ModifyRDN" |
             "Getedb"
\end{verbatim}\end{quote}

For example:

\begin{quote}\begin{verbatim}
Read (4): c=gb@o=Nottingham University
\end{verbatim}\end{quote}

This means that whoever is using association 4 (Steve Kille in this example)
has read the entry for Nottingham University.

\subsection{DAP Result}

\begin{quote}\begin{verbatim}
<DAPResultEvent> ::= "Result sent" "(" <Integer> ")"
<DAPErrorEvent>  ::= "Error sent"  "(" <Integer> ")"
\end{verbatim}\end{quote}

Each operation will normally be answered by either a result or an error.

\subsection{Chaining}

\begin{quote}\begin{verbatim}
<ChainingEvent> ::= "Chain" "(" <Integer> "):" <OpType>
\end{verbatim}\end{quote}

This means that the DSA has decided to contact another DSA in order to
perform an operation received previously.

Normally, nothing is logged when a chain operation is returned.
However, by increasing the logging level to ``LLOG\_DEBUG'' a
{\em Result Recieved} is logged.

\subsection{Updates}

\begin{quote}\begin{verbatim}
<UpdateEvent> ::= "Slave update" <DN> |
                  "Shadow update" <DN> 
\end{verbatim}\end{quote}

These indicate an EDB file or spot shadow entry have been updated.

\subsection{Other Events}

There are a few other less common messages that can 
be written to the audit log. The text
messages should be self-explanatory.

\subsection{Processing the Log Files}

A script called \pgm{dsastats} may be used to process the log file and
produce a report summarising the usage of a DSA.  The report shows the
following:

\begin{itemize}
\item
The operations performed by the DSA 
\item
Who has accessed the DSA
\item
Which parts of the DIT have been accessed
\end{itemize}

The report summarises the calls received by the DSA according to the degree of
authentication.  This analysis is given for remote and local access.
The report also tries to distinguish between system usage, which includes
QUIPU getedb operations, DSA probing and testing by directory system and
interface developers, and real usage.  The ability to make this
distinction rests on user
names being supplied in bind requests.  A large percentage of binds are
currently anonymous --- it is hoped that this report encourages directory
administrators to install systems such that the provision of user names
becomes the norm.

If the \tt -summary\rm \ flag is used when the script is run, a more concise report
is produced.

\subsubsection{Configuring dsastats}

The usefulness of the report relies on accurate configuration.  The key
configuration details to be noted are:

\begin{describe}
\item [\verb+Local Organisation+:] Analysis of which parts of the DIT have been
accessed is given by organisational unit for the pre-configured local
organisation.  (Note that this does not apply if the -summary flag
is used.)  For all other parts of the DIT, analysis is restricted to
being by organisation.  

\item [\verb+Local addresses+:] A file should be set up (a skeleton is provided)
which contains the leading
substrings from local DUA and DSA addresses, as they appear in the quipu.log
file.  This file, which is called {\em quipulocaladds,} might look something
like the following:

\begin{quote}\small\begin{verbatim}
LOCAL-ETHER
Janet=000012345678
Internet=128.9
# file must end with this line
\end{verbatim}\end{quote}

\item [\verb+Filtering out system usage+:] A file should be set up (again a skeleton
is provided) which contains the distinguished names of system users.  This
will probably include the names of directory software developers,
directory administrators and dsa probes.  Distinguished names can be
specified case-independently.  The file, which is called 
\file {quiputechusers},
might look something like the following:

\begin{quote}\footnotesize\begin{verbatim}
c=GB@o=X-Tel Services Ltd@cn=Camayoc
c=GB@o=University College London@cn=DSA Probe
# File must end with this line

\end{verbatim}\end{quote}

\end{describe}

\begin{figure}
\begin{footnotesize}\begin{verbatim}
Summary of calls to DSA <giant armadillo>
From  2:48:46 on 04 feb to 13:24:09 on 07 feb

No. of binds                     local  remote

Anonymous                            2       8
Unauth name DAP                      1       2
Unauth name DSP                     11     137
Simple                               8      25
Protected                            0       0

No. of operations

Reads                                2     136
Compares                            28     141
Lists                                0       4
Searches                            41     214
Modifies                             0       0
ModifyRDNs                           0       0
Adds                                 0       0
GetEDBs                                    363


System usage (calls received)

Binds by Directory technicians             169
Reads of DSA entries                         4
Other ops on DSA entries                     4
Getedb operations                          363
Spot shadows                                 0

Who has used the directory?
*Real* usage by organisation
No. users No. binds
        1       145    anonymous
        1         5    c=gb, o=x-tel services ltd
        ...

Which parts of the Directory have been accessed - real usage?
No. ops  Subtree
Local subtree
     67  c=gb@o=university college london
     14  c=gb@o=university college london@ou=computer science
      1  c=gb@o=university college london@ou=genetics and biometry
      1  c=gb@o=university college london@ou=mathematics
      ...

Other parts of the DIT
      0  root
     66  c=gb
      1  c=gb@o=brunel university
      1  c=gb@o=edinburgh university
      1  c=gb@o=glasgow university
      ...

Which parts of the Directory have been accessed - system usage?
No. ops  Subtree
      ...
\end{verbatim}
\end{footnotesize}
\caption{Example Output of dsastats}
\label{dsastat:example}
\end{figure}

\subsubsection{Running the script}

The script can be configured to find the tailor files and log files.  In
this case, the script can be run by something like the following:

\begin{quote}\small\begin{verbatim}
dsastats >usage.report
\end{verbatim}\end{quote}

The above will not be possible if you run more than 1 DSA.  The
recommendation is then that each DSA logs to a file with a name of the form
dsaname.usage.  Usage would now be:

\begin{quote}\small\begin{verbatim}
dsastats llama.usage >llama.report
\end{verbatim}\end{quote}

A report produced by this script should resemble the example
given in Figure~\ref{dsastat:example}.

To produce a more concise report, specify the -summary (or -s) flag.

\begin{quote}\small\begin{verbatim}
dsastats -s llama.usage >llama.summary
\end{verbatim}\end{quote}


