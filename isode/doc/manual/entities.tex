% run this through LaTeX with the appropriate wrapper

\chapter	{The ISODE Entities Database}\label{isoentities}
The database \file{isoentities} in the ISODE \verb"ETCDIR" directory
(usually \file{/usr/etc/})
contains a simple mapping between
appli\-cat\-ion-entity information and presentation addresses.
This database is used by the stub-directory service.

\[\fbox{\begin{tabular}{lp{0.8\textwidth}}
\bf NOTE:&	Use of this database is deprecated.
		Consult Chapter~\ref{services} on page~\pageref{services}
		for further information.
\end{tabular}}\]

The database itself is an ordinary ASCII text file containing information
regarding the known ap\-pli\-ca\-tion-entities on the network.
Each entry contains:
\begin{itemize}
\item	the object descriptor of the application-entity information,
	realized as a designator-qualifier (host/service) pair
	(the distinguished designator \verb"default" is used for a template
	entry);

\item	the object identifier of the application-entity information
	expressed in dot-notation
	(if no application-entity information is desired,
	the string ``NULL'' should be used instead);
	and,

\item	the presentation address expressed in the string format described
	in Section~\ref{addr:encodings} on page~\pageref{addr:encodings} in
	\volone/.

	Note that since double-quotes are often used in the new string format,
	it is {\bf very\/} important to quote them correctly in the 
	\man isoentities(5) file.
	Usually preceeding the first character of the address with backslash
	(``\verb"\"'') is adequate.
\end{itemize}
Blanks and/or tab characters are used to seperate items.
However, double-quotes may be used to prevent separation for items containing
embedded whitspace.
This is useful for defining the presetnation address.
The sharp character (`\verb"#"') at the beginning of a line indicates a
commentary line.

It is suggested for readability purposes that a blank line should seperate
entries. 

\subsection*	{RFC1085 Support}
Since applications using RFC1085 (the lightweight presentation protocol)
usually demultiplex on the basis of TCP or UDP port,
a further definition for the qualifier is placed in
\file{/etc/services},
one of:
\begin{quote}\small\begin{verbatim}
qualifier    portno/lpp
qualifier    portno/tcp
qualifier    portno/udp
\end{verbatim}\end{quote}
The first alternative says that the service lives on both TCP and UDP;
the second alternative says that the service lives on TCP only;
and,
the third alternative says that the service lives on UDP only.

\section	{Accessing the Database}\label{isoentity}
The \man libacsap(3n) library contains the routines used to access the
database.
The high-level routines, \verb"_str2aei",
which returns application-entity information,
and \verb"aei2addr",
which maps the ap\-pli\-ca\-tion-entity information into a presentation
address have
already been discussed in Section~\ref{acs:aei}.
Similarly, the ``old-style'' routines,
\verb"is2paddr", \verb"is2saddr", \verb"is2taddr",
which maps a host and service directly into a prsentation, session,
or transport address (respectively) are discussed in
Sections~\ref{psap:addresses}, \ref{ssap:addresses}, and~\ref{tsap:addresses}
of \voltwo/ (respectively).
Hence, for the remainder of this chapter,
we need only consider the internal form used by the library in querying the
database.

The \verb"isoentity" structure is the internal form.
\begin{quote}\index{isoentity}\small\begin{verbatim}
struct isoentity {
    OIDentifier ie_identifier;
    char    *ie_descriptor;

    struct PSAPaddr ie_addr;
};
\end{verbatim}\end{quote}
The elements of this structure are:
\begin{describe}
\item[\verb"ie\_identifier":] an object identifier used to form
application-entity information;

\item[\verb"ie\_identifier":] the corresponding object descriptor;
and,

\item[\verb"ie\_addr":] the presentation address.
\end{describe}

The routine \verb"getisoentity" reads the next entry in the database,
opening the database if necessary.
\begin{quote}\index{getisoentity}\small\begin{verbatim}
struct isoentity *getisoentity ()
\end{verbatim}\end{quote}
It returns the manifest constant \verb"NULL" on error or end-of-file.

The routine \verb"setisoentity" opens and rewinds the database.
\begin{quote}\index{setisoentity}\small\begin{verbatim}
int     setisoentity (f)
int     f;
\end{verbatim}\end{quote}
The parameter to this procedure is:
\begin{describe}
\item[\verb"f":] the ``stayopen'' indicator,
if non-zero, then the database will remain open over subsequent calls to the
library.
\end{describe}
The routine \verb"endisoentity" closes the database.
\begin{quote}\index{endisoentity}\small\begin{verbatim}
int     endisoentity ()
\end{verbatim}\end{quote}
Both of these routines return non-zero on success and zero otherwise.
