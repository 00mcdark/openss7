% run this through LaTeX with the appropriate wrapper

\chapter	{User Library}\label{libftam}
The \man libftam(3n) library implements the filestore-independent parts of
the International Standard of the OSI file service, FTAM
(which stands for File Transfer, Access and Management).
Currently supported are:
the no-recovery FTAM Quality-of-Service;
the transfer, access, management, and transfer and management service classes;
the kernel, read, write, access, limited file management,
enhanced file management, grouping, and fadu-locking functional units;
and, the kernel, storage, security, and private attribute groups.

Unlike most OSI services,
FTAM distinguishes between the {\em initiator\/} of an FTAM association and
the {\em responder}.
However,
as with most models of OSI services,
an asynchronous environment is assumed.
That is,
the service provider may generate events for the service user without the
latter triggering the actions which led to the event.
For example,
in a synchronous environment,
an indication that data has arrived usually occurs only when the service user
asks the service provider to read data;
in an asynchronous environment,
the service provider may interrupt the service user at any time to announce
the arrival of data.

The \verb"ftam" module in this release initially uses a synchronous interface;
however once the connection is established,
an asynchronous interface may be selected.

All of the routines in the \man libftam(3n) library are integer-valued.
They return the manifest constant \verb"OK" on success,
or \verb"NOTOK" otherwise.

\section	{Warning}\label{ftam:note}
Please read the following important message.
\[\fbox{\begin{tabular}{lp{0.8\textwidth}}
\bf NOTE:&	Readers of this chapter should have an intimate understanding
		of FTAM.
		It is not the intent of this
		chapter to present a tutorial on these services, so novice
		users will suffer greatly if they choose to read further.

		As previous versions of this software included an
		implementation of DIS FTAM, and the release contains an IS
		implementation, users of the \man libftam(3n) library
		are urged to re-read this chapter.
\end{tabular}}\]

\section	{Constants}
There are several important constants, described below:

\subsection	{FTAM Quality-of-Service}
\[\begin{tabular}{|l|l|}
\hline
    \multicolumn{1}{|c|}{\bf Value}&
		\multicolumn{1}{c|}{\bf FTAM-QoS}\\
\hline
    \tt FQOS\_NORECOVERY&	no recovery\\
    \tt FQOS\_CLASS1&	class 1 recovery\\
    \tt FQOS\_CLASS2&	class 2 recovery\\
    \tt FQOS\_CLASS3&	class 3 recovery\\
\hline
\end{tabular}\]
Currently, only the no-recovery FTAM-QoS is supported.

\subsection	{Service Classes}
\[\begin{tabular}{|l|l|}
\hline
    \multicolumn{1}{|c|}{\bf Value}&
		\multicolumn{1}{c|}{\bf Service Class}\\
\hline
    \tt FCLASS\_TRANSFER&	transfer\\
    \tt FCLASS\_ACCESS&		access\\
    \tt FCLASS\_MANAGE&		management\\
    \tt FCLASS\_TM&		transfer and management\\
    \tt FCLASS\_UNCONS&		unconstrained\\
\hline
\end{tabular}\]
Currently, all service classes other than the unconstrained class are
supported.

\subsection	{Functional Units}
\[\begin{tabular}{|l|l|}
\hline
    \multicolumn{1}{|c|}{\bf Value}&
		\multicolumn{1}{c|}{\bf Functional Unit}\\
\hline
    \tt FUNIT\_READ&		read\\
    \tt FUNIT\_WRITE&		write\\
    \tt FUNIT\_ACCESS&		file access\\
    \tt FUNIT\_LIMITED&		limited file management\\
    \tt FUNIT\_ENHANCED&	enhanced file management\\
    \tt FUNIT\_GROUPING&	grouping\\
    \tt FUNIT\_FADULOCK&	fadu locking\\
    \tt FUNIT\_RECOVERY&	recovery\\
    \tt FUNIT\_RESTART&		restart data transfer\\
\hline
\end{tabular}\]
Currently, all functional units other than the
recovery and restart units are supported.
Further, the grouping unit {\em must\/} be specified.

\subsection	{Attribute Groups}
\[\begin{tabular}{|l|l|}
\hline
    \multicolumn{1}{|c|}{\bf Value}&
		\multicolumn{1}{c|}{\bf Attribute Group}\\
\hline
    \tt FATTR\_STORAGE&		storage\\
    \tt FATTR\_SECURITY&	security\\
    \tt FATTR\_PRIVATE&		private\\
\hline
\end{tabular}\]
Currently, all attribute groups other than the
private group are supported.

\subsection	{State Results}
\[\begin{tabular}{|l|l|}
\hline
    \multicolumn{1}{|c|}{\bf Value}&
		\multicolumn{1}{c|}{\bf State Result}\\
\hline
    \tt FSTATE\_SUCCESS&	success\\
    \tt FSTATE\_FAILURE&	failure\\
\hline
\end{tabular}\]

\subsection	{Action Results}
\[\begin{tabular}{|l|l|}
\hline
    \multicolumn{1}{|c|}{\bf Value}&
		\multicolumn{1}{c|}{\bf Action Result}\\
\hline
    \tt FACTION\_SUCCESS&	success\\
    \tt FACTION\_TRANS&		transient error\\
    \tt FACTION\_PERM&		permanent error\\
\hline
\end{tabular}\]

\section	{Data-Structures}
There are several important data structures, described below:

\subsection	{Contents Type}
An FTAM document type is represented by the \verb"FTAMcontent" structure.
\begin{quote}\index{FTAMcontent}\small\begin{verbatim}
struct FTAMcontent {
    OID     fc_dtn;

    int     fc_id;
    int     fc_result;
};
\end{verbatim}\end{quote}
This elements of this structure are:
\begin{describe}
\item[\verb"fc\_dtn":] the document type name,
represented as an object identifier
(consult Section~\ref{psap:oid} of \volone/);

\item[\verb"fc\_id":] the presentation context identifier
(consult Section~\ref{psap:context} of \voltwo/)
associated with the document type;
and,

\item[\verb"fc\_result":] the status indicator for the presentation context
(codes are listed in Table~\ref{PSAPreasons} of \voltwo/).
\end{describe}

A list of FTAM documents types is represented by the \verb"FTAMcontentlist"
structure.
\begin{quote}\index{FTAMcontentlist}\small\begin{verbatim}
struct FTAMcontentlist {
    int     fc_ncontent;

#define NFCONT  (NPCTX - 2)
    struct FTAMcontent fc_contents[NFCONT];
};
\end{verbatim}\end{quote}
This elements of this structure are:
\begin{describe}
\item[\verb"fc\_contents"/\verb"fc\_ncontent":] the contents list
(and the number of contents in the list).
\end{describe}
A limitation of this implementation is that a fixed number of document types
may be supported,
as denoted by the manifest constant \verb"NFCONT".

\subsection	{Diagnostics}
An FTAM diagnostic is represented by the \verb"FTAMdiagnostic" structure.
\begin{quote}\index{FTAMcontent}\small\begin{verbatim}
struct FTAMdiagnostic {
    int     ftd_type;

    int     ftd_identifier;

    int     ftd_observer;
    int     ftd_source;

    int     ftd_delay;

#define FTD_SIZE        512
    int     ftd_cc;
    char    ftd_data[FTD_SIZE];
};
\end{verbatim}\end{quote}
This elements of this structure are:
\begin{describe}
\item[\verb"ftd\_type":] the diagnostic type, one of:
\[\begin{tabular}{|l|l|}
\hline
    \multicolumn{1}{|c|}{\bf Value}&
		\multicolumn{1}{c|}{\bf Meaning}\\
\hline
    \tt DIAG\_INFORM&	informative\\
    \tt DIAG\_TRANS&	transient\\
    \tt DIAG\_PERM&	permanent\\
\hline
\end{tabular}\]

\item[\verb"ftd\_identifier":] the error-identifier
(consult \verb"<isode/ftam.h>" for the list,
there are far too many to list here);

\item[\verb"ftd\_observer"/\verb"ftd\_source":] the observer and source of
the error, one of:
\[\begin{tabular}{|l|l|}
\hline
    \multicolumn{1}{|c|}{\bf Value}&
		\multicolumn{1}{c|}{\bf Meaning}\\
\hline
    \tt EREF\_NONE&	no categorization possible\\
    \tt EREF\_IFSU&	initiating file service user\\
    \tt EREF\_IFPM&	initiating file service machine\\
    \tt EREF\_SERV&	service support the file protocol machine\\
    \tt EREF\_RFPM&	responding file protocol machine\\
    \tt EREF\_RFSU&	responding file service user\\
\hline
\end{tabular}\]

\item[\verb"ftd\_delay":] the suggested delay that might be honored prior to
retrying the operation (use \verb"DIAG_NODELAY" if a delay is not
appropriate);
and,

\item[\verb"ftd\_data"/\verb"ftd\_cc":] a diagnostic string
(and the length of that string).
\end{describe}
A limitation of this implementation
is that the routines and data structures which support diagnostics have a
fixed limit as to the number of diagnostics permitted.
This is currently denoted by the manifest constant \verb"NFDIAG".

\subsection	{Charging}
A list of FTAM charges is represented by the \verb"FTAMcharging" structure.
\begin{quote}\index{FTAMcharging}\small\begin{verbatim}
struct FTAMcharging {
    int     fc_ncharge;

#define NFCHRG  5
    struct fc_charge {
        char   *fc_resource;
        char   *fc_unit;
        int     fc_value;
    }           fc_charges[NFCHRG];
};
\end{verbatim}\end{quote}
This elements of this structure are:
\begin{describe}
\item[\verb"fc\_charges"/\verb"fc\_ncharge":] the charging list
(and the number of charges in the list, which may not exceed
the manifest constant \verb"NFCHRG").
Each charge consists of:
    \begin{describe}
    \item[\verb"fc\_resource":] the resource identifier;

    \item[\verb"fc\_unit":] the charging unit;
	and,

    \item[\verb"fc\_value":] the charging value.
    \end{describe}
\end{describe}
A limitation of this implementation
is that a fixed number of charges may be supported,
as denoted by the manifest constant \verb"NFCHRG".

\subsection	{Passwords}
A list of FTAM passwords is represented by the \verb"FTAMpasswords" structure.
\begin{quote}\index{FTAMpasswords}\small\begin{verbatim}
struct FTAMpasswords {
    char   *fp_read;
    int     fp_readlen;

    char   *fp_insert;
    int     fp_insertlen;

    char   *fp_replace;
    int     fp_replacelen;

    char   *fp_extend;
    int     fp_extendlen;

    char   *fp_erase;
    int     fp_eraselen;

    char   *fp_readattr;
    int     fp_readattrlen;

    char   *fp_chngattr;
    int     fp_chngattrlen;

    char   *fp_delete;
    int     fp_deletelen;
};
\end{verbatim}\end{quote}
This elements of this structure are:
\begin{describe}
\item[\verb"fp\_read"/\verb"fp\_readlen":] the read password
(and the length of the password);

\item[\verb"fp\_insert"/\verb"fp\_insertlen":] the insert password
(and the length of the password);

\item[\verb"fp\_replace"/\verb"fp\_replacelen":] the replace password
(and the length of the password);

\item[\verb"fp\_extend"/\verb"fp\_extendlen":] the extend password
(and the length of the password);

\item[\verb"fp\_erase"/\verb"fp\_eraselen":] the erase password
(and the length of the password);

\item[\verb"fp\_readattr"/\verb"fp\_readattrlen":] the read attribute password
(and the length of the password);

\item[\verb"fp\_chngattr"/\verb"fp\_chngattrlen":] the change attribute
password (and the length of the password);
and,

\item[\verb"fp\_delete"/\verb"fp\_deletelen":] the delete password
(and the length of the password).
\end{describe}
The \verb"FPFREE" macro can be used to free the storage associated with an
\verb"FTAMpasswords" structure (without freeing the structure itself).
It behaves as if it was defined as:
\begin{quote}\index{FPFREE}\small\begin{verbatim}
void    FPFREE (fp)
struct FTAMpasswords *fp;
\end{verbatim}\end{quote}

\subsection	{Access Control}
An FTAM access control element is represented by the \verb"FTAMacelement"
structure.
\begin{quote}\index{FTAMacelement}\small\begin{verbatim}
struct FTAMacelement {
    int     fe_actions;

    struct FTAMconcurrency fe_concurrency;

    char   *fe_aet;

    struct FTAMpasswords fe_passwords;

    AEI    fe_aet;

    struct FTAMacelement *fe_next;
};
\end{verbatim}\end{quote}
The elements of this structure are:
\begin{describe}
\item[\verb"fe\_actions":] the action list,
containing the inclusive-or of any of the values shown in
Table~\ref{FTAMaccess};

\item[\verb"fe\_concurrency":] any concurrency-control constraints;

\item[\verb"fe\_aet":] the user identity;

\item[\verb"fe\_passwords"] the passwords;

\item[\verb"fe\_aet":] the application-entity title
(defined in Section~\ref{acs:aei} on page~\pageref{acs:aei} in \volone/);
and,

\item[\verb"fe\_next":] the next access element in the linked list.
\end{describe}
The \verb"FEFREE" macro can be used to free the storage associated with a
list of FTAM access elements,
including the head element on the list.
It behaves as if it was defined as:
\begin{quote}\index{FEFREE}\small\begin{verbatim}
void    FEFREE (fe)
struct FTAMacelement *fe;
\end{verbatim}\end{quote}
\tagtable[tp]{ftam-1}{FTAM Requested Access Values}{FTAMaccess}

\subsection	{Attributes}
An FTAM attribute list is represented by the \verb"FTAMattributes" structure.
\begin{quote}\index{FTAMattributes}\small\begin{verbatim}
struct FTAMattributes {
    long    fa_present;
    long    fa_novalue;

#define NFFILE  5
    int     fa_nfile;
    char   *fa_files[NFFILE];
    
    int     fa_permitted;

    OID     fa_contents;
    PE      fa_parameter;

    char   *fa_account;

    struct UTCtime fa_date_create;
    struct UTCtime fa_date_modify;
    struct UTCtime fa_date_read;
    struct UTCtime fa_date_attribute;

    char   *fa_id_create;
    char   *fa_id_modify;
    char   *fa_id_read;
    char   *fa_id_attribute;

    int     fa_availability;

    int     fa_filesize;
    int     fa_futuresize;

    struct FTAMacelement fa_control;
    char   *fa_legal;
};
\end{verbatim}\end{quote}

\newpage	%%% hack

The elements of this structure are:
\begin{describe}
\item[\verb"fa\_present":] the values present,
containing the inclusive-or of any of the values shown in
Table~\ref{FTAMattributes};

\item[\verb"fa\_novalue":] the values which are not available are not available
(using the same values listed in Table~\ref{FTAMattributes},
the value of this element is a subset of the value of the \verb"fa_present"
element);

\item[\verb"fa\_files"/\verb"fa\_nfile":] the list of filename components
(and the number of components, which may not exceed
the manifest constant \verb"NFFILE");

\item[\verb"fa\_permitted":] permitted actions,
containing the inclusive-or of any of the values shown in
Table~\ref{FTAMaccess} on page~\pageref{FTAMaccess},
in addition to any of:
\[\begin{tabular}{|l|l|}
\hline
    \multicolumn{1}{|c|}{\bf Value}&
		\multicolumn{1}{c|}{\bf Meaning}\\
\hline
    \tt FA\_PERM\_TRAV&		traversal\\
    \tt FA\_PERM\_RVTRAV&	reverse-traversal\\
    \tt FA\_PERM\_RANDOM&	random-order\\
\hline
\end{tabular}\]

\item[\verb"fa\_contents"/\verb"fa\_parameter":] the contents type;

\item[\verb"fa\_account":] the account;

\item[\verb"fa\_date\_create":] the date and time of creation;

\item[\verb"fa\_date\_modify":] the date and time of last modification;

\item[\verb"fa\_date\_read":] the date and time of last read access;

\item[\verb"fa\_date\_attribute":] the date and time of last attribute
modification;

\item[\verb"fa\_id\_create":] the identity of creator;

\item[\verb"fa\_id\_modify":] the identity of last modifier;

\item[\verb"fa\_id\_read":] the identity of last reader;

\item[\verb"fa\_id\_attribute":] the identity of last attribute modifier;

\item[\verb"fa\_availability":] file availability,
one of:
\[\begin{tabular}{|l|l|}
\hline
    \multicolumn{1}{|c|}{\bf Value}&
		\multicolumn{1}{c|}{\bf Meaning}\\
\hline
    \tt FA\_AVAIL\_IMMED&	immediate\\
    \tt FA\_AVAIL\_DEFER&	deferred\\
\hline
\end{tabular}\]

\item[\verb"fa\_filesize":] filesize;

\item[\verb"fa\_futuresize":] future filesize;

\item[\verb"fa\_control":] access control;

\item[\verb"fa\_encrypt":] encryption name;
and,

\item[\verb"fa\_legal":] legal qualification.
\end{describe}
A limitation of this implementation
is that a fixed number of filename components may be supported,
as denoted by the manifest constant \verb"NFFILE".

The routine \verb"FAFREE" can be used to free the storage associated with an
\verb"FTAMattributes" structure (without freeing the structure itself).
It is defined as:
\begin{quote}\index{FAFREE}\small\begin{verbatim}
void    FAFREE (fa)
struct FTAMattributes *fa;
\end{verbatim}\end{quote}
\tagtable[tp]{ftam-2}{FTAM Attribute Values}{FTAMattributes}

\subsection	{Concurrency}
An FTAM concurrency list is represented by the \verb"FTAMconcurrency"
structure.
\begin{quote}\index{FTAMconcurrency}\small\begin{verbatim}
struct FTAMconcurrency {
#define FLOCK_SHARED    00      /* shared */
#define FLOCK_EXCLUSIVE 01      /* exclusive */
#define FLOCK_NOTREQD   02      /* not-required */
#define FLOCK_NOACCESS  03      /* no-access */

    char    fc_readlock;
    char    fc_insertlock;
    char    fc_replacelock;
    char    fc_extendlock;
    char    fc_eraselock;
    char    fc_readattrlock;
    char    fc_chngattrlock;
    char    fc_deletelock;
};
\end{verbatim}\end{quote}
This elements of this structure are:
\begin{describe}
\item[\verb"fc\_readlock":] the concurrency constraints for the read operation;

\item[\verb"fc\_insertlock":] the concurrency constraints for the insert
operation; 

\item[\verb"fc\_replacelock":] the concurrency constraints for the replace
operation;

\item[\verb"fc\_extendlock":] the concurrency constraints for the extend
operation; 

\item[\verb"fc\_eraselock":] the concurrency constraints for the erase
operation; 

\item[\verb"fc\_readattrlock":] the concurrency constraints for the read
attribute operation;

\item[\verb"fc\_chngattrlock":] the concurrency constraints for the change
attribute operation;
and,

\item[\verb"fc\_deletelock":] the concurrency constraints for the delete
operation.
\end{describe}
The \verb"FCINIT" macro can be used to initialize an
\verb"FTAMconcurrency" structure.
It behaves as if it was defined as:
\begin{quote}\index{FCINIT}\small\begin{verbatim}
void    FCINIT (fc)
struct FTAMconcurrency *fc;
\end{verbatim}\end{quote}

\subsection	{FADU Identity}
An FADU identity is represented by the \verb"FADUidentity" structure.%
\footnote{FADU stands for File Access Data Unit;
if you didn't already know this, you shouldn't be reading this chapter.
Read Section~\ref{ftam:note} right now.}
\begin{quote}\index{FADUidentity}\small\begin{verbatim}
struct FADUidentity {
    int     fa_type;
#define FA_FIRSTLAST    0       /* first-last */
#define FA_RELATIVE     1       /* relative */
#define FA_BEGINEND     2       /* begin-end */
#define FA_SINGLE       3       /* single-name */
#define FA_NAMELIST     4       /* name-list */
#define FA_FADUNUMBER   5       /* fadu-number */

    union {
        int     fa_un_firstlast;
#define FA_FIRST        0
#define FA_LAST         1

        int     fa_un_relative;
#define FA_PREVIOUS     0
#define FA_CURRENT      1
#define FA_NEXT         2

        int     fa_un_beginend;
#define FA_BEGIN        0
#define FA_END          1

        char   *fa_un_singlename;

#define NANAME  5
        struct {
            char   *fa_un_names[NANAME];
            int     fa_un_nname;
        }       fa_un_list;

        int     fa_un_fadunumber;
    }   fa_un;
#define fa_firstlast    fa_un.fa_un_firstlast
#define fa_relative     fa_un.fa_un_relative
#define fa_beginend     fa_un.fa_un_beginend
#define fa_singlename   fa_un.fa_un_singlename
#define fa_names        fa_un.fa_un_list.fa_un_names
#define fa_nname        fa_un.fa_un_list.fa_un_nname
#define fa_fadunumber   fa_un.fa_un_fadunumber
};
\end{verbatim}\end{quote}
As shown, this structure is really a discriminated union
(a structure with a tag element followed by a union).
The elements are:
\begin{describe}
\item[\verb"fa\_type":] indicates what type of FADU identity is represented;
Depending on this type, one of the following elements is valid:
\begin{describe}
\item[\verb"fa\_firstlast":] the location, identified as first or last;

\item[\verb"fa\_relative":] the location, identified as previous,
current, or next;

\item[\verb"fa\_beginend":] the location, identified as begin or end;

\item[\verb"fa\_singlename":] the location, as identified by name;

\item[\verb"fa\_names"/\verb"fa\_nname":] the location,
as identified by a list of names (and the length of that list,
which may not exceed the manifest constant \verb"NANAME");
or,

\item[\verb"fa\_fadunumber":] the location, as identified by number.
\end{describe}
\end{describe}
A limitation of this implementation
is that a fixed number of FADU names may be supported,
as denoted by the manifest constant \verb"NANAME".

The \verb"FUFREE" macro can be used to free the storage associated with an
\verb"FADUidentity" structure (without freeing the structure itself).
It behaves as if it was defined as:
\begin{quote}\index{FUFREE}\small\begin{verbatim}
void    FUFREE (fu)
struct FADUidentity *fu;
\end{verbatim}\end{quote}

\section	{Association Establishment}
The \man libftam(3n) library distinguishes between the entity which started an
association,
the {\em initiator},
and the entity which was subsequently bound to the association,
the {\em responder}.
We sometimes term these two entities the {\em client\/} and the {\em server},
respectively.

Section~\ref{acs:addresses} of \volone/ describes the use of addressing
for the application layer.
In particular,
Figure~\ref{getFTAMentity} on page~\pageref{getFTAMentity} of \volone/
presents
an example of how one constructs the address for the File Transfer, Access
and Management (FTAM) service on host \verb"RemoteHost".

\subsection	{Responder}
The \man tsapd(8c) daemon,
upon accepting a connection from an initiating host,
consults the ISO services database to determine which program
on the local system implements the desired application context.

Once the program has been ascertained,
the daemon runs the program with any arguments listed in the database.
In addition,
it appends some {\em magic arguments\/} to the argument vector.
Hence,
the very first action performed by the responder is to re-capture the FTAM
state contained in the magic arguments.
This is done by calling the routine \verb"FInit",
which on a successful return,
is equivalent to an {\sf F-INITIALIZE.INDICATION\/} event.
\begin{quote}\index{FInit}\small\begin{verbatim}
int     FInit (vecp, vec, fts, tracing, fti)
int     vecp;
char  **vec;
struct FTAMstart *fts;
int   (*tracing) ();
struct FTAMindication *fti;
\end{verbatim}\end{quote}
The parameters to this procedure are:
\begin{describe}
\item[\verb"vecp":] the length of the argument vector;

\item[\verb"vec":] the argument vector;

\item[\verb"fts":] a pointer to an \verb"FTAMstart" structure,
which is updated only if the call succeeds;

\item[\verb"tracing":] the address of a tracing routine to be invoked when an
FTAM event occurs (consult Section~\ref{ftam:tracing});
and,

\item[\verb"fti":] a pointer to an \verb"FTAMindication" structure,
which is updated only if the call fails.
\end{describe}
If \verb"FInit" is successful,
it returns information in the \verb"fts" parameter,
which is a pointer to an \verb"FTAMstart" structure.
\begin{quote}\index{FTAMstart}\small\begin{verbatim}
struct FTAMstart {
    int     fts_sd;

    AEInfo  fts_calledtitle;
    AEInfo  fts_callingtitle;

    struct PSAPaddr fts_calledaddr;
    struct PSAPaddr fts_callingaddr;

    OID     fts_context;

    int     fts_manage;

    int     fts_class;

    int     fts_units;
    int     fts_attrs;

    PE      fts_sharedASE;

    int     fts_fqos;

    struct FTAMcontentlist fts_contents;

    char   *fts_initiator;
    char   *fts_account;
    char   *fts_password;
    int     fts_passlen;

    int     fts_ssdusize;

    struct QOStype fts_qos;
};
\end{verbatim}\end{quote}
The elements of this structure are:
\begin{describe}
\item[\verb"fts\_sd":] the association-descriptor to be used to reference
this association;

\item[\verb"fts\_calledtitle":] information on the called application-entity,
if any (consult Section~\ref{acs:aei} of \volone/);

\item[\verb"fts\_callingtitle":] information on the the calling
application-entity, if any;

\item[\verb"fts\_calledaddr":] the called presentation address
(consult Section~\ref{psap:addresses} of \voltwo/);

\item[\verb"fts\_callingaddr":] the calling presentation address;

\item[\verb"fts\_context":] the application context name;

\item[\verb"fts\_manage":] a flag indicating if presentation context
management is available for use (zero if unavailable);

\item[\verb"fts\_class":] the inclusive-or of the service classes offered;

\item[\verb"fts\_units":] the functional units;

\item[\verb"fts\_attrs":] the attribute groups;

\item[\verb"fts\_sharedASE":] shared ASE information;

\item[\verb"fts\_fqos":] the FTAM Quality-of-Service;

\item[\verb"fts\_contents":] the contents type list;

\item[\verb"fts\_initiator":] the user-identity of the initiator, if any;

\item[\verb"fts\_account":] the account, if any;

\item[\verb"fts\_password"/\verb"fts\_passlen":] the password (and its
length), if any;

\item[\verb"fts\_ssdusize":] the largest atomic SSDU size that can be used
on the underlying connection;
and,

\item[\verb"qos":] the quality of service on the connection
(see Section~\ref{tsap:qos} in \voltwo/).
\end{describe}
Note that the data contained in the structure was allocated via \man malloc(3),
and should be released by using the \verb"FTSFREE" macro when no longer
referenced.
The \verb"FTSFREE" macro,
behaves as if it was defined as:
\begin{quote}\index{FTSFREE}\small\begin{verbatim}
void    FTSFREE (fts)
struct FTAMstart *fts;
\end{verbatim}\end{quote}
The macro frees only the data allocated by \verb"FInit",
and not the \verb"FTAMstart" structure itself.
Further,
\verb"FTSFREE" should be called only if the call to the \verb"FInit"
routine returned \verb"OK".

If the call to \verb"FInit" is not successful,
then an {\sf F-P-ABORT.INDICATION\/} event is simulated,
and the relevant information is returned in an encoded \verb"FTAMindication"
structure.\label{FTAMindication}
\begin{quote}\index{FTAMindication}\small\begin{verbatim}
struct FTAMindication {
    int     fti_type;
#define FTI_FINISH      0x00
#define FTI_ABORT       0x01
#define FTI_MANAGEMENT  0x02
#define FTI_BULKBEGIN   0x03
#define FTI_BULKEND     0x04
#define FTI_ACCESS      0x05
#define FTI_READWRITE   0x06
#define FTI_DATA        0x07
#define FTI_DATAEND     0x08
#define FTI_CANCEL      0x09
#define FTI_TRANSEND    0x10

    union {
        struct FTAMfinish   fti_un_finish;
        struct FTAMabort    fti_un_abort;
        struct FTAMgroup    fti_un_group;
        struct FTAMaccess   fti_un_access;
        struct FTAMreadwrite fti_un_readwrite;
        struct PSAPdata     fti_un_data;
        struct FTAMdataend  fti_un_dataend;
        struct FTAMcancel   fti_un_cancel;
        struct FTAMtransend fti_un_transend;
    }   fti_un;
#define fti_finish      fti_un.fti_un_finish
#define fti_abort       fti_un.fti_un_abort
#define fti_group       fti_un.fti_un_group
#define fti_access      fti_un.fti_un_access
#define fti_readwrite   fti_un.fti_un_readwrite
#define fti_data        fti_un.fti_un_data
#define fti_dataend     fti_un.fti_un_dataend
#define fti_cancel      fti_un.fti_un_cancel
#define fti_transend    fti_un.fti_un_transend
};
\end{verbatim}\end{quote}
As shown, this structure is really a discriminated union
(a structure with a tag element followed by a union).
Hence, on a failure return,
one first coerces a pointer to the \verb"FTAMabort" structure contained
therein,
and then consults the elements of that structure.
\begin{quote}\index{FTAMabort}\small\begin{verbatim}
struct FTAMabort {
    int     fta_peer;

    int     fta_action;

#define NFDIAG  5
    int     fta_ndiag;
    struct FTAMdiagnostic fta_diags[NFDIAG];
};
\end{verbatim}\end{quote}
The elements of an \verb"FTAMabort" structure are:
\begin{describe}
\item[\verb"fta\_peer":] if set, indicates that a user-initiated abort occurred
(an {\sf F-U-ABORT.INDICATION\/} event);
if not set, indicates that a provider-initiated abort occurred
(an {\sf F-P-ABORT.INDICATION\/} event);

\item[\verb"fta\_action":] an action result;
and,

\item[\verb"fta\_diags"/\verb"fta\_ndiag":] any diagnostic information
(and the number of diagnostics present).
\end{describe}
For each diagnostic present (typically only one) in the \verb"FTAMabort"
structure,
information regarding the failure is present.
In particular,
the error code can be found in the
the \verb"ftd_identifier" element of each diagnostic.

After examining the information returned by \verb"FInit" on a successful call
(and possibly after examining the argument vector),
the responder should either accept or reject the association.
For either response,
the responder should use the \verb"FInitializeResponse" routine
(which corresponds to the {\sf F-INITIALIZE.RESPONSE\/} action).
\begin{quote}\index{FInitializeResonse}\small\begin{verbatim}
int     FInitializeResponse (sd, state, action, context,
        respondtitle, respondaddr, manage, class, units,
        attrs, sharedASE, fqos, contents, diag, ndiag, fti)
int     sd;
int     state,
        action,
        manage,
        class,
        units,
        attrs,
        fqos;
OID     context;
AEI     respondtitle;
struct PSAPaddr *respondaddr;
PE      sharedASE;
struct FTAMcontentlist *contents;
struct FTAMdiagnostic diag[];
int     ndiag;
struct FTAMindication *fti;
\end{verbatim}\end{quote}
The parameters to this procedure are:
\begin{describe}
\item[\verb"sd":] the association-descriptor;

\item[\verb"state":] a state result;

\item[\verb"action":] an action result;

\item[\verb"context":] the application context name
(use the manifest constant \verb"NULLOID" if the proposed name should be used);

\item[\verb"manage":] the negotiated use of presentation context management;

\item[\verb"respondtitle":] information on the responding application-entity,
if any;

\item[\verb"respondaddr":] the responding presentation address, if any;

\item[\verb"manage":] the negotiated service class;

\item[\verb"units":] the negotiated functional units
(specify both mandatory and optional functional units for the service class);

\item[\verb"attrs":] the negotiated attribute groups;

\item[\verb"sharedASE":] shared ASE information;

\item[\verb"fqos":] the negotiated FTAM Quality-of-Service;

\item[\verb"contents":] the negotiated contents type list;

\item[\verb"diag"/\verb"ndiag":] a list of diagnostics
(and the number of diagnostics in the list);
and,

\item[\verb"fti":] a pointer to an \verb"FTAMindication" structure,
which is updated only if the call fails.
\end{describe}
If the call to \verb"FInitializeResponse" is successful,
then association establishment has now been completed.
Otherwise,
if the call failed and the reason is ``fatal'',
then the association is lost.

\subsection	{Initiator}
A program wishing to associate itself with a filestore
calls the routine \verb"FInitializeRequest",
which corresponds to the user taking the {\sf F-INIT\-IAL\-IZE.REQUEST\/}
action.
\begin{quote}\index{FInitializeRequest}\small\begin{verbatim}
int     FInitializeRequest (context, callingtitle, calledtitle, 
        callingaddr, calledaddr, manage, class, units, attrs,
        sharedASE, fqos, contents, initiator, account,
        password, passlen, qos, tracing, ftc, fti)
OID     context;
AEI     calledtitle,
        callingtitle;
struct PSAPaddr *calledaddr,
                *callingaddr;
int     manage,
        class,
        units,
        attrs,
        fqos,
        passlen;
PE      sharedASE;
struct FTAMcontentlist *contents;
char   *initiator,
       *account,
       *password;
struct QOStype *qos;
IFP     tracing;
struct FTAMconnect *ftc;
struct FTAMindication *fti;
\end{verbatim}\end{quote}
The parameters to this procedure are:
\begin{describe}
\item[\verb"context":] the application context name
(use the manifest constant \verb"NULLOID" to invoke the default
file transfer context);

\item[\verb"callingtitle":] information on the the calling application-entity,
if any (consult Section~\ref{acs:aei} of \volone/);

\item[\verb"calledtitle":] information on the called application-entity,
if any;

\item[\verb"callingaddr":] the calling presentation address
(consult Section~\ref{psap:addresses} of \voltwo/);

\item[\verb"calledaddr":] the called presentation address;

\item[\verb"manage":] a flag indicating if presentation context management is
available for use (zero if unavailable);

\item[\verb"class":] the proposed service classes;

\item[\verb"units":] the proposed functional units
(specify both mandatory and optional functional units for the indicated
service class);

\item[\verb"attrs":] the proposed attribute groups;

\item[\verb"sharedASE":] shared ASE information;

\item[\verb"fqos":] the proposed FTAM Quality-of-Service;

\item[\verb"contents":] the contents type list;

\item[\verb"initiator":] the user-identity of the initiator, if any;

\item[\verb"account":] the account, if any

\item[\verb"password"/\verb"passlen":] the password (and its length), if any;

\item[\verb"qos":] the quality of service on the connection
(see Section~\ref{tsap:qos} in \voltwo/);

\item[\verb"tracing":] the address of a tracing routine to be invoked when an
FTAM event occurs (consult Section~\ref{ftam:tracing});

\item[\verb"ftc":] a pointer to an \verb"FTAMconnect" structure,
which is updated only if the call succeeds;
and,

\item[\verb"fti":] a pointer to an \verb"FTAMindication" structure,
which is updated only if the call fails.
\end{describe}
If the call to \verb"FInitializeRequest" is successful
(the {\sf F-INITIALIZE.CON\-FIR\-MA\-TION\/} event occurs),
it returns information in the \verb"ftc" parameter
which is a pointer to an \verb"FTAMconnect" structure.
\begin{quote}\index{FTAMconnect}\small\begin{verbatim}
struct FTAMconnect {
    int	    ftc_sd;

    int	    ftc_state;
    int	    ftc_action;

    AEInfo  ftc_respondtitle;

    struct PSAPaddr ftc_respondaddr;

    int	    ftc_manage;

    int	    ftc_units;
    int	    ftc_attrs;

    int	    ftc_rollback;

    struct FTAMcontentlist ftc_contents;

    int	    ftc_ndiag;
    struct FTAMdiagnostic  ftc_diags[NFDIAG];

    int	    ftc_ssdusize;

    struct QOStype ftc_qos;
};
\end{verbatim}\end{quote}
The elements of an \verb"FTAMconnect" structure are:
\begin{describe}
\item[\verb"ftc\_sd":] the association-descriptor to be used to reference
this association;

\item[\verb"ftc\_state":] a state result

\item[\verb"ftc\_action":] an action result

\item[\verb"ftc\_respondtitle":] the responding application-entity title, if
any;

\item[\verb"ftc\_respondaddr":] the responding presentation address, if any;

\item[\verb"ftc\_manage":] the negotiated use of presentation context
management;

\item[\verb"ftc\_units":] the negotiated functional units;

\item[\verb"ftc\_attrs":] the negotiated attribute groups;

\item[\verb"ftc\_rollback":] the negotiated rollback availability;

\item[\verb"ftc\_contents":] the negotiated contents type list;

\item[\verb"ftc\_diags"/\verb"ftc\_ndiag":] any diagnostic information
(and the number of diagnostics present)

\item[\verb"ftc\_ssdusize":] the largest atomic SSDU size that can be used
on the underlying connection;
and,

\item[\verb"ftc\_qos":] the quality of service on the connection
(see Section~\ref{tsap:qos} in \voltwo/).
\end{describe}
If the call to \verb"FInitializeRequest" is successful,
and the \verb"ftc_state" element is set to \verb"FSTATE_SUCCESS",
then association establishment has completed.
If the call is successful,
but the \verb"acc_result" element is not \verb"FSTATE_SUCCESS",
the the association attempt has been rejected
and the diagnostics present indicate the reason for the rejection.
Otherwise, if the call fails then the association is not established and
the \verb"FTAMabort" structure contained in the \verb"FTAMindication"
discriminated union has been updated.

Note that the data contained in the structure was allocated via \man malloc(3),
and should be released by using the \verb"FTCFREE" macro when no longer
referenced.
The \verb"FTCFREE" macro,
behaves as if it was defined as:
\begin{quote}\index{FTCFREE}\small\begin{verbatim}
void    FTCFREE (ftc)
struct FTAMconnect *ftc;
\end{verbatim}\end{quote}
The macro frees only the data allocated by \verb"FInitializeRequest",
and not the \verb"FTAMconnect" structure itself.
Further,
\verb"FTCFREE" should be called only if the call to the
\verb"FInitializeRequest" routine returned \verb"OK".

\section	{Event Handling}
Once the association has been established,
an association-descriptor is used to reference the connection.
This is usually the first parameter given to any of the remaining routines in
the \man libftam(3n) library.
Further,
the last parameter is usually a pointer to an \verb"FTAMindication" structure
(as described on page~\pageref{FTAMindication}).
If a call to one of these routines fails,
then the structure is updated.
Consult the \verb"FTAMabort" element of the \verb"FTAMindication" structure
and look at the diagnostics (typically only one) present.
When a diagnostic is reported by the provider,
the \verb"ftd_type" element indicates if the error is ``fatal'' by having the
value \verb"DIAG_PERM".
Other values, such as \verb"DIAG_TRANS" indicate non-fatal errors
(the association still exists).
The most common non-fatal error to occur is \verb"FS_GEN_WAITING" which
indicates that an indication is waiting to be read.

The \verb"FWaitRequest" routine is used to wait for an event to occur.
\begin{quote}\index{FWaitRequest}\small\begin{verbatim}
int     FWaitRequest (sd, secs, fti)
int     sd;
int     secs;
struct FTAMindication *fti;
\end{verbatim}\end{quote}
The parameters to this procedure are:
\begin{describe}
\item[\verb"sd":] the association-descriptor;

\item[\verb"secs":] the maximum number of seconds to wait for the event
(a value of \verb"NOTOK" indicates that the call should block indefinitely,
whereas a value of \verb"OK" indicates that the call should not block at all,
e.g., a polling action);
and,

\item[\verb"fti":] a pointer to an \verb"FTAMindication" structure,
which is always updated.
\end{describe}
If the call to \verb"FWaitRequest" returns the manifest constant \verb"NOTOK",
then the \verb"FTAMabort" structure contained in the \verb"FTAMindication"
parameter \verb"fti" contains the reason for the failure.
Similarly,
if the manifest constant \verb"OK" is returned,
an event is encoded in the \verb"fti" parameter,
depending on the value of the \verb"fti_type" element:
\[\begin{tabular}{|l|l|}
\hline
    \multicolumn{1}{|c|}{\bf Value}&
			\multicolumn{1}{c|}{\bf Event}\\
\hline
    \tt FTI\_FINISH&		\sf F-TERMINATE.INDICATION\\
\hline
    \tt FTI\_MANAGEMENT&	grouped management operation\\
\hline
    \tt FTI\_BULKBEGIN&		begin of grouped file transfer operation\\
\hline
    \tt FTI\_BULKEND&		end of grouped file transfer operation\\
\hline
    \tt FTI\_ACCESS&		\sf F-LOCATE.INDICATION\\
	&			\sf F-LOCATE.CONFIRMATION\\
	&			\sf F-ERASE.INDICATION\\
	&			\sf F-ERASE.CONFIRMATION\\
\hline
    \tt FTI\_READWRITE&		\sf F-READ.INDICATION\\
	&			\sf F-WRITE.INDICATION\\
\hline
    \tt FTI\_DATA&		\sf F-DATA.INDICATION\\
\hline
    \tt FTI\_DATAEND&		\sf F-DATA-END.INDICATION\\
\hline
    \tt FTI\_CANCEL&		\sf F-CANCEL.INDICATION\\
	&			\sf F-CANCEL.CONFIRMATION\\
\hline
    \tt FTI\_TRANSEND&		\sf F-TRANSFER-END.INDICATION\\
	&			\sf F-TRANSFER-END.CONFIRMATION\\
\hline
\end{tabular}\]
These are now discussed in turn.

\subsubsection	{Termination Indication}
When an event associated with association release occurs,
the \verb"fti_type" field of the \verb"fti" parameter contains the value
\verb"FTI_FINISH".
Further,
an \verb"FTAMfinish" structure is contained inside the \verb"fti" parameter,
indicating an {\sf F-TERMINATE.INDICATION\/} event.
\begin{quote}\index{FTAMfinish}\small\begin{verbatim}
struct FTAMfinish {
    int     ftf_sharedASE;
};
\end{verbatim}\end{quote}
The elements of this structure are:
\begin{describe}
\item[\verb"ftf\_sharedASE":] shared ASE information.
\end{describe}
The \verb"FTFFREE" macro can be used to free the storage associated with an
\verb"FTAMdfinish" structure (without freeing the structure itself).
It behaves as if it was defined as:
\begin{quote}\index{FTFFREE}\small\begin{verbatim}
void    FTFFREE (ftf)
struct FTAMfinish *ftf;
\end{verbatim}\end{quote}

\subsubsection	{Group Indications}
Use of the grouping function unit is required in the current implementation,
hence
when a transfer or management indication occurs,
the \verb"fti_type" field of the \verb"fti" parameter contains one of three
values:
\[\begin{tabular}{|l|l|}
\hline
    \multicolumn{1}{|c|}{\bf Value}&
			\multicolumn{1}{c|}{\bf Event}\\
\hline
    \tt FTI\_MANAGEMENT&	grouped management operation\\
    \tt FTI\_BULKBEGIN&		begin of grouped file transfer operation\\
    \tt FTI\_BULKEND&		end of grouped file transfer operation\\
\hline
\end{tabular}\]
and an \verb"FTAMgroup" structure is contained inside the \verb"fti" parameter.
\begin{quote}\index{FTAMgroup}\small\begin{verbatim}
struct FTAMgroup {
    int     ftg_threshold;

    int     ftg_flags;

    union {
        struct FTAMselect   ftg_un1_select;
        struct FTAMcreate   ftg_un1_create;
        struct FTAMclose    ftg_un1_close;
    }                   ftg_un1;
#define ftg_select      ftg_un1.ftg_un1_select
#define ftg_create      ftg_un1.ftg_un1_create
#define ftg_close       ftg_un1.ftg_un1_close

    struct FTAMreadattr ftg_readattr;

    struct FTAMchngattr ftg_chngattr;

    union {
        struct FTAMdeselect ftg_un2_deselect;
        struct FTAMdelete   ftg_un2_delete;
        struct FTAMopen     ftg_un2_open;
    } ftg_un2;
#define ftg_deselect    ftg_un2.ftg_un2_deselect
#define ftg_delete      ftg_un2.ftg_un2_delete
#define ftg_open        ftg_un2.ftg_un2_open
};
\end{verbatim}\end{quote}
The elements of this structure are:\label{FTAMgroup}
\begin{describe}
\item[\verb"ftg\_threshold":] the grouping threshold;

\item[\verb"ftg\_flags":] the indications which are present,
containing the inclusive-or of any of the values:
\[\begin{tabular}{|l|l|}
\hline
    \multicolumn{1}{|c|}{\bf Value}&
		\multicolumn{1}{c|}{\bf Meaning}\\
\hline
    \tt FTG\_BEGIN&	F-BEGIN-GROUP\\
    \tt FTG\_SELECT&	F-SELECT\\
    \tt FTG\_CREATE&	F-CREATE\\
    \tt FTG\_RDATTR&	F-READ-ATTRIB\\
    \tt FTG\_CHATTR&	F-CHANGE-ATTRIB\\
    \tt FTG\_OPEN&	F-OPEN\\
    \tt FTG\_CLOSE&	F-CLOSE\\
    \tt FTG\_DESELECT&	F-DESELECT\\
    \tt FTG\_DELETE&	F-DELETE\\
    \tt FTG\_END&	F-END-GROUP\\
\hline
\end{tabular}\]

\item[\verb"ftg\_select":]	an \verb"FTAMselect" structure;

\item[\verb"ftg\_create":]	an \verb"FTAMcreate" structure;

\item[\verb"ftg\_close":]	an \verb"FTAMclose" structure;

\item[\verb"ftg\_readattr":]	an \verb"FTAMreadattr" structure;

\item[\verb"ftg\_chngattr":]	an \verb"FTAMchngattr" structure;

\item[\verb"ftg\_deselect":]	an \verb"FTAMdeselect" structure;

\item[\verb"ftg\_delete":]	an \verb"FTAMdelete" structure;
and,

\item[\verb"ftg\_open":]	an \verb"FTAMopen" structure.
\end{describe}
The \verb"FTGFREE" macro can be used to free the storage associated with an
\verb"FTAMgroup" structure (without freeing the structure itself).
It behaves as if it was defined as:
\begin{quote}\index{FTGFREE}\small\begin{verbatim}
void    FTGFREE (ftg)
struct FTAMgroup *ftg;
\end{verbatim}\end{quote}

If the flag \verb"FTG_SELECT" is present,
then \verb"ftg_select" element contains an {\sf F-SELECT\/} event encoded in an
\verb"FTAMselect" structure:
\begin{quote}\index{FTAMselect}\small\begin{verbatim}
struct FTAMselect {
    int     ftse_state;
    int     ftse_action;

    struct FTAMattributes ftse_attrs;

    int	    ftse_access;
    struct FTAMpasswords ftse_pwds;
    struct FTAMconcurrency ftse_conctl;

    PE      ftse_sharedASE;

    char   *ftse_account;

    int	    ftse_ndiag;
    struct FTAMdiagnostic ftse_diags[NFDIAG];
};
\end{verbatim}\end{quote}
The elements of this structure are:
\begin{describe}
\item[\verb"ftse\_state":] the state result ({\sf .RESPONSE\/} only);

\item[\verb"ftse\_action":] the action result ({\sf .RESPONSE\/} only);

\item[\verb"ftse\_attrs":] file attributes;

\item[\verb"ftse\_access":] requested access,
containing the inclusive-or of any of the values
found in Table~\ref{FTAMaccess} ({\sf .REQUEST\/} only);

\item[\verb"ftse\_pwds":] access passwords ({\sf .REQUEST\/} only);

\item[\verb"ftse\_conctl":] concurrency control ({\sf .REQUEST\/} only);

\item[\verb"ftse\_sharedASE":] shared ASE information ({\sf .REQUEST\/} only);

\item[\verb"ftse\_account":] account ({\sf .REQUEST\/} only);
and,

\item[\verb"ftse\_diags"/\verb"ftse\_ndiag":] any diagnostic information
(and the number of diagnostics present, {\sf .RESPONSE\/} only).
\end{describe}
Note that the data contained in the structure was allocated via \man malloc(3),
and should be released with the \verb"FTSEFREE" macro when no longer needed.
The \verb"FTSEFREE" macro
behaves as if it was defined as:
\begin{quote}\index{FTSEFREE}\small\begin{verbatim}
void    FTSEFREE (ftse)
struct FTAMselect *ftse;
\end{verbatim}\end{quote}
The macro frees only the data allocated by \verb"FWaitRequest",
and not the \verb"FTAMselect" structure itself.
Note that the macro \verb"FTGFREE" when applied to the group,
will call \verb"FTSEFREE" as appropriate.

If the flag \verb"FTG_CREATE" is present,
then \verb"ftg_create" element contains an {\sf F-CREATE\/} event encoded in an
\verb"FTAMcreate" structure:
\begin{quote}\index{FTAMcreate}\small\begin{verbatim}
struct FTAMcreate {
    int     ftce_state;
    int     ftce_action;

    int     ftce_override;
    char   *ftce_create;
    int     ftce_crelen;

    struct FTAMattributes ftce_attrs;

    int     ftce_access;
    struct FTAMpasswords ftce_pwds;
    struct FTAMconcurrency ftce_conctl;

    PE      ftce_sharedASE;

    char   *ftce_account;

    int     ftce_ndiag;
    struct FTAMdiagnostic ftce_diags[NFDIAG];
};
\end{verbatim}\end{quote}
The elements of this structure are:
\begin{describe}
\item[\verb"ftce\_state":] the state result ({\sf .RESPONSE\/} only);

\item[\verb"ftce\_action":] the action result ({\sf .RESPONSE\/} only);

\item[\verb"ftce\_override":] the override setting, one of:
\[\begin{tabular}{|l|l|}
\hline
    \multicolumn{1}{|c|}{\bf Value}&
		\multicolumn{1}{c|}{\bf Meaning}\\
\hline
    \tt FOVER\_FAIL&		fail, if already exists\\
    \tt FOVER\_SELECT&		select, if already exists\\
    \tt FOVER\_WRITE&		zero-truncate, if already exists\\
    \tt FOVER\_DELETE&		delete, if already exists\\
\hline
\end{tabular}\]
({\sf .REQUEST\/} only);

\item[\verb"ftce\_attrs":] file attributes;

\item[\verb"ftce\_create"/\verb"ftce\_crelen":] the
create password and its length, if any ({\sf .REQUEST\/} only);

\item[\verb"ftce\_access":] requested access,
containing the inclusive-or of any of the values
found in Table~\ref{FTAMaccess} ({\sf .REQUEST\/} only);

\item[\verb"ftce\_pwds":] access passwords ({\sf .REQUEST\/} only);

\item[\verb"ftce\_conctl":] concurrency control ({\sf .REQUEST\/} only);

\item[\verb"ftce\_sharedASE":] shared ASE information ({\sf .REQUEST\/} only);

\item[\verb"ftce\_account":] account ({\sf .REQUEST\/} only);
and,

\item[\verb"ftce\_diags"/\verb"ftce\_ndiag":] any diagnostic information
(and the number of diagnostics present, {\sf .RESPONSE\/} only).
\end{describe}
Note that the data contained in the structure was allocated via \man malloc(3),
and should be released with the \verb"FTCEFREE" macro when no longer needed.
The \verb"FTCEFREE" macro
behaves as if it was defined as:
\begin{quote}\index{FTCEFREE}\small\begin{verbatim}
void    FTCEFREE (ftce)
struct FTAMcreate *ftce;
\end{verbatim}\end{quote}
The macro frees only the data allocated by \verb"FWaitRequest",
and not the \verb"FTAMcreate" structure itself.
Note that the macro \verb"FTGFREE" when applied to the group,
will call \verb"FTCEFREE" as appropriate.

If the flag \verb"FTG_RDATTR" is present,
then \verb"ftg_readattr" element contains an {\sf F-READ-ATTRIB\/} event encoded in an
\verb"FTAMreadattr" structure:
\begin{quote}\index{FTAMreadattr}\small\begin{verbatim}
struct FTAMreadattr {
    int     ftra_action;

    int     ftra_attrnames;

    struct FTAMattributes ftra_attrs;
    int     ftra_ndiag;
    struct FTAMdiagnostic ftra_diags[NFDIAG];
};
\end{verbatim}\end{quote}
The elements of this structure are:
\begin{describe}
\item[\verb"ftra\_action":] the action result ({\sf .RESPONSE\/} only);

\item[\verb"ftra\_attrnames":] the attributes to read,
containing the inclusive-or of any of the values shown in
Table~\ref{FTAMattributes} on page~\pageref{FTAMattributes}
({\sf .REQUEST\/} only);

\item[\verb"ftra\_attrs":] the attribute values
({\sf .RESPONSE\/} only);
and,

\item[\verb"ftra\_diags"/\verb"ftra\_ndiag":] any diagnostic information
(and the number of diagnostics present, {\sf .RESPONSE\/} only).
\end{describe}
Note that the data contained in the structure was allocated via \man malloc(3),
and should be released with the \verb"FTRAFREE" macro when no longer needed.
The \verb"FTRAFREE" macro
behaves as if it was defined as:
\begin{quote}\index{FTRAFREE}\small\begin{verbatim}
void    FTRAFREE (ftra)
struct FTAMreadattr *ftra;
\end{verbatim}\end{quote}
The macro frees only the data allocated by \verb"FWaitRequest",
and not the \verb"FTAMreadattr" structure itself.
Note that the macro \verb"FTGFREE" when applied to the group,
will call \verb"FTRAFREE" as appropriate.

If the flag \verb"FTG_CHNGATTR" is present,
then \verb"ftg_chngattr" element contains an {\sf F-CHANGE-ATTRIB\/} event encoded in
an \verb"FTAMchngattr" structure:
\begin{quote}\index{FTAMchngattr}\small\begin{verbatim}
struct FTAMchngattr {
    int ftca_action;

    struct FTAMattributes ftca_attrs;

    int     ftca_ndiag;
    struct FTAMdiagnostic ftca_diags[NFDIAG];
};
\end{verbatim}\end{quote}
The elements of this structure are:
\begin{describe}
\item[\verb"ftca\_action":] the action result ({\sf .RESPONSE\/} only);

\item[\verb"ftca\_attrs":] the attribute values to change
(in a {\sf .REQUEST\/}),
and possibly the new values (in a {\sf .RESPONSE\/});
and,

\item[\verb"ftca\_diags"/\verb"ftca\_ndiag":] any diagnostic information
(and the number of diagnostics present, {\sf .RESPONSE\/} only).
\end{describe}
Note that the data contained in the structure was allocated via \man malloc(3),
and should be released with the \verb"FTCAFREE" macro when no longer needed.
The \verb"FTCAFREE" macro
behaves as if it was defined as:
\begin{quote}\index{FTCAFREE}\small\begin{verbatim}
void    FTCAFREE (ftca)
struct FTAMchngattr *ftca;
\end{verbatim}\end{quote}
The macro frees only the data allocated by \verb"FWaitRequest",
and not the \verb"FTAMchngattr" structure itself.
Note that the macro \verb"FTGFREE" when applied to the group,
will call \verb"FTCAFREE" as appropriate.

If the flag \verb"FTG_OPEN" is present,
then \verb"ftg_open" element contains an {\sf F-OPEN\/} event encoded in an
\verb"FTAMopen" structure:
\begin{quote}\index{FTAMopen}\small\begin{verbatim}
struct FTAMopen {
    int     ftop_state;
    int     ftop_action;

    int     ftop_mode;

    OID     ftop_contents;
    PE      ftop_parameter;
    struct FTAMconcurrency ftop_conctl;
    PE      ftop_sharedASE;

    int     ftop_locking;

    int     ftop_ndiag;
    struct FTAMdiagnostic ftop_diags[NFDIAG];
};
\end{verbatim}\end{quote}
The elements of this structure are:
\begin{describe}
\item[\verb"ftop\_state":] the state result ({\sf .RESPONSE\/} only);

\item[\verb"ftop\_action":] the action result ({\sf .RESPONSE\/} only);

\item[\verb"ftop\_mode":] the processing mode,
containing the inclusive-or of any of the values shown in
Table~\ref{FTAMaccess} on page~\pageref{FTAMaccess}
({\sf .REQUEST\/} only);

\item[\verb"ftop\_contents"/\verb"ftop\_parameter":] the contents type;

\item[\verb"ftop\_conctl":] concurrency control;

\item[\verb"ftop\_sharedASE":] shared ASE information;

\item[\verb"ftop\_locking":] enable FADU locking ({\sf .REQUEST\/} only);
and,

\item[\verb"ftop\_diags"/\verb"ftop\_ndiag":] any diagnostic information
(and the number of diagnostics present, {\sf .RESPONSE\/} only).
\end{describe}
Note that the data contained in the structure was allocated via \man malloc(3),
and should be released with the \verb"FTOPFREE" macro when no longer needed.
The \verb"FTOPFREE" macro
behaves as if it was defined as:
\begin{quote}\index{FTOPFREE}\small\begin{verbatim}
void    FTOPFREE (ftop)
struct FTAMopen *ftop;
\end{verbatim}\end{quote}
The macro frees only the data allocated by \verb"FWaitRequest",
and not the \verb"FTAMopen" structure itself.
Note that the macro \verb"FTGFREE" when applied to the group,
will call \verb"FTOPFREE" as appropriate.

If the flag \verb"FTG_CLOSE" is present,
then \verb"ftg_close" element contains an {\sf F-CLOSE\/} event encoded in an
\verb"FTAMclose" structure:
\begin{quote}\index{FTAMclose}\small\begin{verbatim}
struct FTAMclose {
    int     ftcl_action;

    PE      ftcl_sharedASE;

    int     ftcl_ndiag;
    struct FTAMdiagnostic   ftcl_diags[NFDIAG];
};
\end{verbatim}\end{quote}
The elements of this structure are:
\begin{describe}
\item[\verb"ftcl\_action":] the action result;

\item[\verb"ftcl\_sharedASE":] shared ASE information;
and,

\item[\verb"ftcl\_diags"/\verb"ftcl\_ndiag":] any diagnostic information
(and the number of diagnostics present).
\end{describe}
Note that the data contained in the structure was allocated via \man malloc(3),
and should be released with the \verb"FTCLFREE" macro when no longer needed.
The \verb"FTCLFREE" macro
behaves as if it was defined as:
\begin{quote}\index{FTCLFREE}\small\begin{verbatim}
void    FTCLFREE (ftcl)
struct FTAMclose *ftcl;
\end{verbatim}\end{quote}
The macro frees only the data allocated by \verb"FWaitRequest",
and not the \verb"FTAMclose" structure itself.
Note that the macro \verb"FTGFREE" when applied to the group,
will call \verb"FTCLFREE" as appropriate.

If the flag \verb"FTG_DESELECT" is present,
then \verb"ftg_deselect" element contains an {\sf F-DESELECT\/} event encoded in an
\verb"FTAMdeselect" structure:
\begin{quote}\index{FTAMdeselect}\small\begin{verbatim}
struct FTAMdeselect {
    int     ftde_action;

    PE      ftde_sharedASE;

    struct FTAMcharging ftde_charges;
    int     ftde_ndiag;
    struct FTAMdiagnostic   ftde_diags[NFDIAG];
};
\end{verbatim}\end{quote}
The elements of this structure are:
\begin{describe}
\item[\verb"ftde\_action":] the action result ({\sf .RESPONSE\/} only);

\item[\verb"ftde\_sharedASE":] shared ASE information;

\item[\verb"ftde\_charges":] any charges ({\sf .RESPONSE\/} only);
and,

\item[\verb"ftde\_diags"/\verb"ftde\_ndiag":] any diagnostic information
(and the number of diagnostics present, {\sf .RESPONSE\/} only).
\end{describe}
Note that the data contained in the structure was allocated via \man malloc(3),
and should be released with the \verb"FTDEFREE" macro when no longer needed.
The \verb"FTDEFREE" macro
behaves as if it was defined as:
\begin{quote}\index{FTDEFREE}\small\begin{verbatim}
void    FTDEFREE (ftde)
struct FTAMdeselect *ftde;
\end{verbatim}\end{quote}
The macro frees only the data allocated by \verb"FWaitRequest",
and not the \verb"FTAMdeselect" structure itself.
Note that the macro \verb"FTGFREE" when applied to the group,
will call \verb"FTDEFREE" as appropriate.

If the flag \verb"FTG_DELETE" is present,
then \verb"ftg_delete" element contains an {\sf F-DELETE\/} event encoded in an
\verb"FTAMdelete" structure:
\begin{quote}\index{FTAMdelete}\small\begin{verbatim}
struct FTAMdelete {
    int     ftxe_action;
    
    PE      ftxe_sharedASE;

    struct FTAMcharging ftxe_charges;
    int     ftxe_ndiag;
    struct FTAMdiagnostic   ftxe_diags[NFDIAG];
};
\end{verbatim}\end{quote}
The elements of this structure are:
\begin{describe}
\item[\verb"ftxe\_action":] the action result ({\sf .RESPONSE\/} only);

\item[\verb"ftxe\_sharedASE":] shared ASE information;

\item[\verb"ftxe\_charges":] any charges ({\sf .RESPONSE\/} only);
and,

\item[\verb"ftxe\_diags"/\verb"ftxe\_ndiag":] any diagnostic information
(and the number of diagnostics present, {\sf .RESPONSE\/} only).
\end{describe}
Note that the data contained in the structure was allocated via \man malloc(3),
and should be released with the \verb"FTXEFREE" macro when no longer needed.
The \verb"FTXEFREE" macro
behaves as if it was defined as:
\begin{quote}\index{FTXEFREE}\small\begin{verbatim}
void    FTXEFREE (ftxe)
struct FTAMdeltee *ftxe;
\end{verbatim}\end{quote}
The macro frees only the data allocated by \verb"FWaitRequest",
and not the \verb"FTAMdelete" structure itself.
Note that the macro \verb"FTGFREE" when applied to the group,
will call \verb"FTXEFREE" as appropriate.

\subsubsection	{Access Indications}
When an event associated with file access occurs,
the \verb"fti_type" field of the \verb"fti" parameter contains the value
\verb"FTI_ACCESS".
Further,
an \verb"FTAMaccess" structure is contained inside the \verb"fti" parameter,
indicating an {\sf F-LOCATE\/} or {\sf F-ERASE\/} event.
\begin{quote}\index{FTAMaccess}\small\begin{verbatim}
struct FTAMaccess {
    int	    ftac_operation;

    int	    ftac_action;

    struct FADUidentity ftac_identity;

    int     ftac_locking;

    int     ftac_ndiag;
    struct FTAMdiagnostic   ftac_diags[NFDIAG];
};
\end{verbatim}\end{quote}
The elements of this structure are:
\begin{describe}
\item[\verb"ftac\_operation":] the operation,
one of the values shown in Table~\ref{FTAMaccessops};

\item[\verb"ftac\_action":] the action result ({\sf .CONFIRMATION\/} only);

\item[\verb"ftac\_identity":] the FADU identity
(either {\sf .INDICATION\/} and optionally on {\sf F-LOCATE.CONFIRMATION\/});

\item[\verb"ftac\_locking":] FADU locked ({\sf F-LOCATE.INDICATION\/} only);
and,

\item[\verb"ftac\_diags"/\verb"ftac\_ndiag":] any diagnostic information
(and the number of diagnostics present, {\sf .CONFIRMATION\/} only).
\end{describe}
Note that the data contained in the structure was allocated via \man malloc(3),
and should be released with the \verb"FTACFREE" macro when no longer needed.
The \verb"FTACFREE" macro
behaves as if it was defined as:
\begin{quote}\index{FTACFREE}\small\begin{verbatim}
void    FTACFREE (ftac)
struct FTAMaccess *ftac;
\end{verbatim}\end{quote}
The macro frees only the data allocated by \verb"FWaitRequest",
and not the \verb"FTAMaccess" structure itself.
\tagtable[tp]{ftam-5}{FTAM Access Operations}{FTAMaccessops}

\subsubsection	{Read/Write Indications}
When an event associated with the beginning of bulk data transfer occurs,
the \verb"fti_type" field of the \verb"fti" parameter contains the value
\verb"FTI_READWRITE".
Further,
an \verb"FTAMreadwrite" structure is contained inside the \verb"fti" parameter,
indicating an {\sf F-READ.INDICATION\/} or {\sf F-WRITE.INDICATION\/} event.
\begin{quote}\index{FTAMreadwrite}\small\begin{verbatim}
struct FTAMreadwrite {
    int     ftrw_operation;

    struct FADUidentity ftrw_identity;

    int     ftrw_context;
    int     ftrw_level;

    int     ftrw_locking;
};
\end{verbatim}\end{quote}
The elements of this structure are:
\begin{describe}
\item[\verb"ftrw\_operation":] the operation,
one of the values shown in Table~\ref{FTAMdataops};

\item[\verb"ftrw\_identity":] the FADU identity;

\item[\verb"ftrw\_context"/\verb"ftrw\_level":] the access context,
one of the values shown in Table~\ref{FTAMcontexts}
({\sf F-READ.INDICATION\/} only);
and,

\item[\verb"ftrw\_locking":] FADU locked.
\end{describe}
Note that the data contained in the structure was allocated via \man malloc(3),
and should be released with the \verb"FTRWFREE" macro when no longer needed.
The \verb"FTRWFREE" macro
behaves as if it was defined as:
\begin{quote}\index{FTRWFREE}\small\begin{verbatim}
void    FTRWFREE (ftrw)
struct FTAMreadwrite *ftrw;
\end{verbatim}\end{quote}
The macro frees only the data allocated by \verb"FWaitRequest",
and not the \verb"FTAMreadwrite" structure itself.
\tagtable[tp]{ftam-4}{FTAM Data Operations}{FTAMdataops}
\tagtable[tp]{ftam-3}{FTAM Access Contexts}{FTAMcontexts}

\subsubsection	{Data Indications}\label{ftam:data}
When an event associated with user data occurs,
the \verb"fti_type" field of the \verb"fti" parameter contains the value
\verb"FTI_DATA",
and an \verb"PSAPdata" structure
(described in Chapter~\ref{libpsap2} of \voltwo/)
is contained inside the \verb"fti" parameter,
indicating an {\sf F-DATA.INDICATION\/} event.
To distinguish between structure and contents,
FTAM uses different presentation context identifiers.
If the \verb"pe_context" field of any presentation element contains the value
\verb"PE_DFLT_CONTEXT",
then the element contains structuring information in the FTAM PCI,
and the \verb"pe_id" element should be consulted to determine the particular
structuring information being conveyed:
\[\begin{tabular}{|l|l|}
\hline
    \multicolumn{1}{|c|}{\bf Value}&
		\multicolumn{1}{c|}{\bf Meaning}\\
\hline
    \tt FADU\_NODESCR&		node descriptor data element\\
    \tt FADU\_ENTERTREE&	enter subtree data element\\
    \tt FADU\_EXITREE&		exit subtree data element\\
\hline
\end{tabular}\]
Otherwise,
the element contains file contents as defined in the PCI for the
document type being transferred.

\subsubsection	{Data End Indication}
When an event associated with the end of bulk data transfer occurs,
the \verb"fti_type" field of the \verb"fti" parameter contains the value
\verb"FTI_DATAEND".
Further,
an \verb"FTAMdataend" structure is contained inside the \verb"fti" parameter,
indicating an {\sf F-DATA-END.INDICATION\/} event.
\begin{quote}\index{FTAMdataend}\small\begin{verbatim}
struct FTAMdataend {
    int     ftda_action;

    int     ftda_ndiag;
    struct FTAMdiagnostic   ftda_diags[NFDIAG];
};
\end{verbatim}\end{quote}
The elements of this structure are:
\begin{describe}
\item[\verb"ftda\_action":] the action result;
and,

\item[\verb"ftda\_diags"/\verb"ftda\_ndiag":] any diagnostic information
(and the number of diagnostics present).
\end{describe}

\subsubsection	{Cancel Indications}
When an event associated with the cancellation of bulk data transfer occurs,
the \verb"fti_type" field of the \verb"fti" parameter contains the value
\verb"FTI_CANCEL".
Further,
an \verb"FTAMcancel" structure is contained inside the \verb"fti" parameter,
indicating an {\sf F-CANCEL\/} event.
\begin{quote}\index{FTAMcancel}\small\begin{verbatim}
struct FTAMcancel {
    int     ftcn_action;

    PE      ftcn_sharedASE;

    int     ftcn_ndiag;
    struct FTAMdiagnostic   ftcn_diags[NFDIAG];
};
\end{verbatim}\end{quote}
The elements of this structure are:
\begin{describe}
\item[\verb"ftcn\_action":] the action result;

\item[\verb"ftcn\_sharedASE":] shared ASE information;
and,

\item[\verb"ftcn\_diags"/\verb"ftcn\_ndiag":] any diagnostic information
(and the number of diagnostics present).
\end{describe}
Note that the data contained in the structure was allocated via \man malloc(3),
and should be released with the \verb"FTCNFREE" macro when no longer needed.
The \verb"FTCNFREE" macro
behaves as if it was defined as:
\begin{quote}\index{FTCNFREE}\small\begin{verbatim}
void    FTCNFREE (ftcn)
struct FTAMcancel *ftcn;
\end{verbatim}\end{quote}
The macro frees only the data allocated by \verb"FWaitRequest",
and not the \verb"FTAMcancel" structure itself.

\subsubsection	{Transfer End Indications}
When an event associated with the termination of bulk data transfer occurs,
the \verb"fti_type" field of the \verb"fti" parameter contains the value
\verb"FTI_TRANSEND".
Further,
an \verb"FTAMtransend" structure is contained inside the \verb"fti" parameter,
indicating an {\sf F-TRANSFER-END\/} event.
\begin{quote}\index{FTAMtransend}\small\begin{verbatim}
struct FTAMtransend {
    int     ftre_action;

    PE      ftre_sharedASE;

    int     ftre_ndiag;
    struct FTAMdiagnostic   ftre_diags[NFDIAG];
};
\end{verbatim}\end{quote}
The elements of this structure are:
\begin{describe}
\item[\verb"ftre\_action":] the action result ({\sf .RESPONSE\/} only);

\item[\verb"ftre\_sharedASE":] shared ASE information;
and,

\item[\verb"ftre\_diags"/\verb"ftre\_ndiag":] any diagnostic information
(and the number of diagnostics present, {\sf .RESPONSE\/} only).
\end{describe}
Note that the data contained in the structure was allocated via \man malloc(3),
and should be released with the \verb"FTREFREE" macro when no longer needed.
The \verb"FTREFREE" macro
behaves as if it was defined as:
\begin{quote}\index{FTREFREE}\small\begin{verbatim}
void    FTREFREE (ftre)
struct FTAMtransend *ftre;
\end{verbatim}\end{quote}
The macro frees only the data allocated by \verb"FWaitRequest",
and not the \verb"FTAMtransend" structure itself.

\subsection	{Asynchronous Event Handling}
Thus far the discussion on event handling has been synchronous in nature.
Some users of FTAM may wish an asynchronous interface.
The \verb"FSetIndications" routine is used to change the service associated
with an association-descriptor to or from an asynchronous interface.
\begin{quote}\index{FSetIndications}\small\begin{verbatim}
int     FSetIndications (sd, indication, fti)
int     sd;
int   (*indication) ();
struct FTAMindication  *fti;
\end{verbatim}\end{quote}
The parameters to this procedure are:
\begin{describe}
\item[\verb"sd":] the association-descriptor;

\item[\verb"indication":] the address of an event-handler routine to be
invoked when an indication occurs;
and,

\item[\verb"fti":] a pointer to an \verb"FTAMindication" structure, which is
updated only if the call fails.
\end{describe}
If the service is to be made asynchronous,
then the event handler is specified,
otherwise,
if the service is to be made synchronous,
then the handler should not be specified
(use the manifest constant \verb"NULLIFP").
The most likely reason for the call failing is \verb"FS_GEN_WAITING",
which indicates that an event is waiting for the user.

When an event-handler is invoked,
future invocations of the event-hander are blocked until it returns.
The return value of the event-handler is ignored.
Further,
during the execution of a synchronous call to the library,
the event-handler will be blocked from being invoked.

When an event occurs,
the event-handler routine is invoked with two parameters:
\begin{quote}\small\begin{verbatim}
(*indication) (sd, fti);
int     sd;
struct FTAMindication *fti;
\end{verbatim}\end{quote}
The parameters are:
\begin{describe}
\item[\verb"sd":] the association-descriptor;
and,

\item[\verb"fti":] a pointer to the \verb"FTAMindication" structure containing
the indication.
\end{describe}

\[\fbox{\begin{tabular}{lp{0.8\textwidth}}
\bf NOTE:&	The \man libftam(3n) library uses the SIGEMT signal to provide
		these services.
		Programs using asynchronous association-descriptors should NOT
		use SIGEMT for other purposes.
\end{tabular}}\]

\subsection	{Synchronous Event Multiplexing}
A user of FTAM may wish to manage multiple
association-descriptors simultaneously;
the routine \verb"FSelectMask" is provided for this purpose.
This routine updates a file-descriptor mask and associated counter for use
with \verb"xselect".
\begin{quote}\index{FSelectMask}\small\begin{verbatim}
int     FSelectMask (sd, mask, nfds, fti)
int     sd;
fd_set *mask;
int    *nfds;
struct FTAMindication  *fti;
\end{verbatim}\end{quote}
The parameters to this procedure are:
\begin{describe}
\item[\verb"sd":] the association-descriptor;

\item[\verb"mask":] a pointer to a file-descriptor mask meaningful to
\verb"xselect";

\item[\verb"nfds":] a pointer to an integer-valued location meaningful to
\verb"xselect";
and,

\item[\verb"fti":] a pointer to an \verb"FTAMindication" structure, which is updated
only if the call fails.
\end{describe}
If the call is successful, then the \verb"mask" and \verb"nfds" parameters can
be used as arguments to \verb"xselect".
The most likely reason for the call failing is \verb"FS_GEN_WAITING",
which indicates that an event is waiting for the user.

If \verb"xselect" indicates that the association-descriptor is ready for
reading,
\verb"FWaitRequest" should be called with the \verb"secs" parameter equal to
\verb"OK".
If the network activity does not constitute an entire event for the user,
then \verb"FWaitRequest" will return \verb"NOTOK" with error code
\verb"FS_PRO_TIMEOUT".

\subsection	{Tracing}\label{ftam:tracing}
Users may wish to trace FTAM events as they are processed.
The routine \verb"FHookRequest" is used to set (or unset) the user-defined
tracing routine.
\begin{quote}\index{FHookRequest}\small\begin{verbatim}
int     FHookRequest (sd, tracing, fti)
int     sd;
int   (*tracing) ();
\end{verbatim}\end{quote}
The parameters to this procedure are:
\begin{describe}
\item[\verb"sd":] the association-descriptor;

\item[\verb"tracing":] the address of a tracing routine to be invoked when an
FTAM event occurs;
and,

\item[\verb"fti":] a pointer to an \verb"FTAMindication" structure, which is updated
only if the call fails.
\end{describe}
If tracing is to be enabled,
then the tracing routine is specified,
otherwise,
if tracing is to be turned off, the manifest constant \verb"NULLIFP" should
be used.

When an event occurs,
the tracing routine is invoked with five parameters:
\begin{quote}\small\begin{verbatim}
(*tracing) (sd, event, fpdu, pe, rw)
\end{verbatim}\end{quote}
The parameters are:
\begin{describe}
\item[\verb"sd":] the association-descriptor;

\item[\verb"event":] the name of the underlying service primitive
(association control or presentation) associated with the FTAM event;

\item[\verb"fpdu":] the name of the FTAM PDU, if any;

\item[\verb"pe":] a presentation element containing the FTAM PDU
associated with the FTAM event
(described in Chapter~\ref{libpsap} of \volone/), if any;
and,

\item[\verb"rw":] a flag saying whether the event is being initiated (zero)
or being received (non-zero).
\end{describe}
The return value of the tracing routine is ignored.

One pre-defined tracing routine is available, \verb"FTraceHook"
\index{FHookRequest},
which simply appends tracing information to the log determined by
\begin{quote}\index{ftamfile}\small\begin{verbatim}
LLog *ftam_log;
\end{verbatim}\end{quote}
See Chapter~\ref{logging} in \voltwo/ for all the details.

\section	{Grouped Operations: File Transfer}
The \verb"FBulkBeginRequest" routine is used to issue a grouped file transfer
request consisting of:
\begin{itemize}
\item	{\sf F-BEGIN-GROUP.REQUEST\/};

\item	{\sf F-SELECT.REQUEST\/} or {\sf F-CREATE.REQUEST\/};

\item	optionally, {\sf F-READ-ATTRIB.REQUEST\/};

\item	optionally, {\sf F-CHANGE-ATTRIB.REQUEST\/};

\item	{\sf F-OPEN.REQUEST\/};
and,

\item	{\sf F-END-GROUP.REQUEST}.
\end{itemize}
The user initializes an \verb"FTAMgroup" structure
(described in tedious detail on page~\pageref{FTAMgroup})
and then invokes \verb"FBulkRequest".
\begin{quote}\index{FBulkRequest}\small\begin{verbatim}
int     FBulkRequest (sd, ftg, fti)
int     sd;
struct FTAMgroup *ftg;
struct FTAMindication *fti;
\end{verbatim}\end{quote}
The parameters to this procedure are:
\begin{describe}
\item[\verb"sd":] the association-descriptor;

\item[\verb"ftg":] a pointer to an \verb"FTAMgroup" structure;
and,

\item[\verb"fti":] a pointer to an \verb"FTAMindication" structure, which is
always updated.
\end{describe}
If the call to \verb"FBulkBeginRequest" is successful,
then this corresponds to the appropriate {\sf .CONFIRMATION\/} events,
the \verb"fti_type" element of the \verb"fti" parameter is set to
\verb"FTI_BULKBEGIN",
and the \verb"fti_group" element contains the results.
Note that the data contained in the structure was allocated via \man malloc(3),
and should be released with the \verb"FTGFREE" macro when no longer needed.

Otherwise if the call fails,
the \verb"FTAMabort" structure contained in the
\verb"FTAMindication" parameter \verb"fti" contains the reason for the failure.

Upon receiving an \verb"FTI_BULKBEGIN" indication,
containing the appropriate {\sf .INDICATION\/} events,
the responder is required to generate the appropriate {\sf .RESPONSE\/}
events using the \verb"FBulkBeginResponse" routine.
\begin{quote}\index{FBulkResponse}\small\begin{verbatim}
int     FBulkBeginResponse (sd, ftg, fti)
int     sd;
struct FTAMgroup *ftg;
struct FTAMindication *fti;
\end{verbatim}\end{quote}
The parameters to this procedure are:
\begin{describe}
\item[\verb"sd":] the association-descriptor;

\item[\verb"ftg":] a pointer to an \verb"FTAMgroup" structure;
and,

\item[\verb"fti":] a pointer to an \verb"FTAMindication" structure, which is
updated only if the call fails.
\end{describe}
If the call fails,
the \verb"FTAMabort" structure contained in the
\verb"FTAMindication" parameter \verb"fti" contains the reason for the failure.

The \verb"FBulkEndRequest" routine is used to issue a grouped file transfer
request consisting of:
\begin{itemize}
\item	{\sf F-BEGIN-GROUP.REQUEST\/};

\item	{\sf F-CLOSE.REQUEST\/};

\item	{\sf F-DESELECT.REQUEST\/} or {\sf F-DELETE.REQUEST\/};
and,

\item	{\sf F-END-GROUP.REQUEST}.
\end{itemize}
The user initializes an \verb"FTAMgroup" structure
(described in tedious detail on page~\pageref{FTAMgroup})
and then invokes \verb"FBulkEndRequest".
\begin{quote}\index{FBulkEndRequest}\small\begin{verbatim}
int     FBulkEndRequest (sd, ftg, fti)
int     sd;
struct FTAMgroup *ftg;
struct FTAMindication *fti;
\end{verbatim}\end{quote}
The parameters to this procedure are:
\begin{describe}
\item[\verb"sd":] the association-descriptor;

\item[\verb"ftg":] a pointer to an \verb"FTAMgroup" structure;
and,

\item[\verb"fti":] a pointer to an \verb"FTAMindication" structure, which is
always updated.
\end{describe}
If the call to \verb"FBulkEndRequest" is successful,
then this corresponds to the appropriate {\sf .CONFIRMATION\/} events,
the \verb"fti_type" element of the \verb"fti" parameter is set to
\verb"FTI_BULKEND",
and the \verb"fti_group" element contains the results.
Note that the data contained in the structure was allocated via \man malloc(3),
and should be released with the \verb"FTGFREE" macro when no longer needed.

Otherwise if the call fails,
the \verb"FTAMabort" structure contained in the
\verb"FTAMindication" parameter \verb"fti" contains the reason for the failure.

Upon receiving an \verb"FTI_BULKEND" indication,
containing the appropriate {\sf .INDICATION\/} events,
the responder is required to generate the appropriate {\sf .RESPONSE\/}
events using the \verb"FBulkEndResponse" routine.
\begin{quote}\index{FBulkEndResponse}\small\begin{verbatim}
int     FBulkEndResponse (sd, ftg, fti)
int     sd;
struct FTAMgroup *ftg;
struct FTAMindication *fti;
\end{verbatim}\end{quote}
The parameters to this procedure are:
\begin{describe}
\item[\verb"sd":] the association-descriptor;

\item[\verb"ftg":] a pointer to an \verb"FTAMgroup" structure;
and,

\item[\verb"fti":] a pointer to an \verb"FTAMindication" structure, which is
updated only if the call fails.
\end{describe}
If the call fails,
the \verb"FTAMabort" structure contained in the
\verb"FTAMindication" parameter \verb"fti" contains the reason for the failure.

\section	{File Access}
The \verb"FAccessRequest" routine is equivalent to an {\sf F-LOCATE.REQUEST\/}
or {\sf F-ERASE.REQUEST\/} action on the part of the user.
\begin{quote}\index{FAccessRequest}\small\begin{verbatim}
int     FAccessRequest (sd, operation, identity, fti)
int     sd;
int     operation;
struct FADUidentity *identity;
struct FTAMindication *fti;
\end{verbatim}\end{quote}
The parameters to this procedure are:
\begin{describe}
\item[\verb"sd":] the association-descriptor;

\item[\verb"operation":] the operation to be performed,
one of the values shown in Table~\ref{FTAMaccessops};

\item[\verb"identity":] the identity of the FADU to locate or erase;
and,

\item[\verb"fti":] a pointer to an \verb"FTAMindication" structure, which is
always updated.
\end{describe}
If the call to \verb"FAccessRequest" is successful,
then this corresponds to the appropriate {\sf .CONFIRMATION\/} event,
the \verb"fti_type" element of the \verb"fti" parameter is set to
\verb"FTI_ACCESS",
and the \verb"fti_access" element contains the results.
Note that the data contained in the structure was allocated via \man malloc(3),
and should be released with the \verb"FTACFREE" macro when no longer needed.

Otherwise if the call fails,
the \verb"FTAMabort" structure contained in the
\verb"FTAMindication" parameter \verb"fti" contains the reason for the failure.

Upon receiving an {\sf F-LOCATE.INDICATION\/}
or an {\sf F-ERASE.IN\-DI\-CA\-TION\/}
event,
the user is required to generate the appropriate {\sf .RESPONSE\/} action
using the \verb"FAccessResponse" routine.
\begin{quote}\index{FAccessResponse}\small\begin{verbatim}
int     FAccessResponse (sd, action, identity, diag, ndiag,
            fti)
int     sd;
int     action;
struct FADUidentity *identity;
struct FTAMdiagnostic diag[];
int     ndiag;
struct FTAMindication *fti;
\end{verbatim}\end{quote}
The parameters to this procedure are:
\begin{describe}
\item[\verb"sd":] the association-descriptor;

\item[\verb"action":] the action result;

\item[\verb"identity":] the identity of the FADU located
(not present for the {\sf F-ERASE.RESPONSE\/} action);

\item[\verb"diag"/\verb"ndiag":] a list of diagnostics
(and the number of diagnostics in the list);
and,

\item[\verb"fti":] a pointer to an \verb"FTAMindication" structure, which is
updated only if the call fails.
\end{describe}
If the call to \verb"AccessResponse" is successful,
then the {\sf .RESPONSE\/} event has been queued for sending to the
initiator.
Otherwise the \verb"FTAMabort" structure contained in the
\verb"FTAMindication" parameter \verb"fti" contains the reason for the failure.

\section	{Data Transfer}
Once group file transfer request has been made,
and after any file access requests,
data transfer may occur.
This occurs in three steps:
the initiator requests reading or writing of an FADU,
the source sends the data,
the transfer is then either cancelled (by either side) or terminated by the
initiator.

\subsection	{Read/Write}
The \verb"FReadWriteRequest" routine is equivalent to a {\sf
F-READ.REQUEST\/} or {\sf F-WRITE.REQUEST\/} action on the part of the user.
\begin{quote}\index{FReadWriteRequest}\small\begin{verbatim}
int     FReadWriteRequest (sd, operation, identity, context,
                level, lock, fti)
int     sd;
int     operation;
struct FADUidentity *identity;
int     context,
        level,
        lock;
struct FTAMindication *fti;
\end{verbatim}\end{quote}
The parameters to this procedure are:
\begin{describe}
\item[\verb"sd":] the association-descriptor;

\item[\verb"operation":] the operation to be performed,
one of the values shown in Table~\ref{FTAMdataops} on
page~\pageref{FTAMdataops};

\item[\verb"identity":] the identity of the FADU to be transferred;

\item[\verb"context"/\verb"level":] the access context,
one of the values shown in Table~\ref{FTAMcontexts};

\item[\verb"lock":] lock FADU;
and,

\item[\verb"fti":] a pointer to an \verb"FTAMindication" structure, which is
updated only if the call fails.
\end{describe}
If the call fails,
the \verb"FTAMabort" structure contained in the
\verb"FTAMindication" parameter \verb"fti" contains the reason for the failure.

\subsection	{Sending Data}
The FADU is transmitted as a series of presentation elements.
A group of presentation elements may be sent using the \verb"FDataRequest"
routine,
which corresponds to the {\sf F-DATA.REQUEST\/} action.
\begin{quote}\index{FDataRequest}\small\begin{verbatim}
int     FDataRequest (sd, fadus, nfadu, fti)
int     sd;
PE      fadus[];
int     nfadu;
struct FTAMindication *fti;
\end{verbatim}\end{quote}
The parameters to this procedure are:
\begin{describe}
\item[\verb"sd":] the association-descriptor;

\item[\verb"fadus"/\verb"nfadu":] the list of data elements
(and the number of data elements which may not exceed the manifest constant
\verb"NPDATA" described in Chapter~\ref{libpsap2}),
consult Section~\ref{ftam:data} on page~\pageref{ftam:data} for important
information on the use of presentation context identifiers;
and,

\item[\verb"fti":] a pointer to an \verb"FTAMindication" structure, which is
updated only if the call fails.
\end{describe}
If the call fails,
the \verb"FTAMabort" structure contained in the
\verb"FTAMindication" parameter \verb"fti" contains the reason for the failure.

When the last data element in the FADU has been sent,
the {\sf F-DATA-END.REQUEST\/} action should be performed
using the \verb"FDataEndRequest" routine.
\begin{quote}\index{FDataEndRequest}\small\begin{verbatim}
int     FDataEndRequest (sd, action, diag, ndiag, fti)
int     sd;
int     action;
struct FTAMdiagnostic diag[];
int     ndiag;
struct FTAMindication *fti;
\end{verbatim}\end{quote}
The parameters to this procedure are:
\begin{describe}
\item[\verb"sd":] the association-descriptor;

\item[\verb"action":] the action result;

\item[\verb"diag"/\verb"ndiag":] a list of diagnostics
(and the number of diagnostics in the list);
and,

\item[\verb"fti":] a pointer to an \verb"FTAMindication" structure, which is
updated only if the call fails.
\end{describe}
If the call fails,
the \verb"FTAMabort" structure contained in the
\verb"FTAMindication" parameter \verb"fti" contains the reason for the failure.

\subsection	{Canceling Transfer}
Either the initiator or the responder may cancel the transfer prior to
orderly termination of the data transfer.
The \verb"FCancelRequest",
which corresponds to the {\sf F-CANCEL.REQUEST\/} action,
is used to initiate cancellation.
\begin{quote}\index{FCancelRequest}\small\begin{verbatim}
int     FCancelRequest (sd, action, sharedASE, diag, ndiag,
                        fti)
int     sd;
int     action;
PE      sharedASE;
struct FTAMdiagnostic diag[];
int     ndiag;
struct FTAMindication *fti;
\end{verbatim}\end{quote}
The parameters to this procedure are:
\begin{describe}
\item[\verb"sd":] the association-descriptor;

\item[\verb"action":] the action result;

\item[\verb"sharedASE":] shared ASE information;

\item[\verb"diag"/\verb"ndiag":] a list of diagnostics
(and the number of diagnostics in the list);
and,

\item[\verb"fti":] a pointer to an \verb"FTAMindication" structure, which is
always updated.
\end{describe}
If the call to \verb"FCancelRequest" is successful,
then this corresponds to the {\sf F-CANCEL.CONFIRMATION\/} event,
the \verb"fti_type" element of the \verb"fti" parameter is set to
\verb"FTI_CANCEL",
and the \verb"fti_cancel" element contains the results.

Otherwise if the call fails,
the \verb"FTAMabort" structure contained in the
\verb"FTAMindication" parameter \verb"fti" contains the reason for the failure.

Upon receiving an {\sf F-CANCEL.INDICATION\/} event,
the user is required to generate the {\sf F-CANCEL.RESPONSE\/}
events using the \verb"FCancelResponse" routine.
\begin{quote}\index{FCancelResponse}\small\begin{verbatim}
int     FCancelResponse (sd, action, sharedASE, diag, ndiag,
                         fti)
int     sd;
int     action;
PE      sharedASE;
struct FTAMdiagnostic diag[];
int     ndiag;
struct FTAMindication *fti;
\end{verbatim}\end{quote}
The parameters to this procedure are:
\begin{describe}
\item[\verb"sd":] the association-descriptor;

\item[\verb"action":] the action result;

\item[\verb"sharedASE":] shared ASE information;

\item[\verb"diag"/\verb"ndiag":] a list of diagnostics
(and the number of diagnostics in the list);
and,

\item[\verb"fti":] a pointer to an \verb"FTAMindication" structure, which is
updated only if the call fails.
\end{describe}
If the call fails,
the \verb"FTAMabort" structure contained in the
\verb"FTAMindication" parameter \verb"fti" contains the reason for the failure.

\subsection	{Terminating Transfer}
After generating the {\sf F-DATA-END.REQUEST\/}
(or after receiving the {\sf F-DATA-END.RESPONSE\/}) action,
the initiator is required to perform the {\sf F-TRANSFER-END.REQUEST\/}
action using the \verb"FTransEndRequest" routine.
\begin{quote}\index{FTransEndRequest}\small\begin{verbatim}
int     FTransEndRequest (sd, sharedASE, fti)
int     sd;
PE	sharedASE;
struct FTAMindication *fti;
\end{verbatim}\end{quote}
The parameters to this procedure are:
\begin{describe}
\item[\verb"sd":] the association-descriptor;

\item[\verb"sharedASE":] shared ASE information;
and,

\item[\verb"fti":] a pointer to an \verb"FTAMindication" structure, which is
always updated.
\end{describe}
If the call to \verb"FTransEndRequest" is successful,
then this corresponds to the {\sf F-TRANSFER-END.CONFIRMATION\/} event,
the \verb"fti_type" element of the \verb"fti" parameter is set to
\verb"FTI_TRANSEND",
and the \verb"fti_transend" element contains the results.

Otherwise, if the call fails,
the \verb"FTAMabort" structure contained in the
\verb"FTAMindication" parameter \verb"fti" contains the reason for the failure.

Upon receiving an {\sf F-TRANSFER-END.INDICATION\/} event,
the responder is required to generate the {\sf F-TRANSFER-END.RESPONSE\/}
events using the \verb"FTransEndResponse" routine.
\begin{quote}\index{FTransEndResponse}\small\begin{verbatim}
int     FTransEndResponse (sd, action, sharedASE, diag, ndiag,
            fti)
int     sd;
int     action;
PE      sharedASE;
struct FTAMdiagnostic diag[];
int     ndiag;
struct FTAMindication *fti;
\end{verbatim}\end{quote}
The parameters to this procedure are:
\begin{describe}
\item[\verb"sd":] the association-descriptor;

\item[\verb"action":] the action result;

\item[\verb"sharedASE":] shared ASE information;

\item[\verb"diag"/\verb"ndiag":] a list of diagnostics
(and the number of diagnostics in the list);
and,

\item[\verb"fti":] a pointer to an \verb"FTAMindication" structure, which is
updated only if the call fails.
\end{describe}
If the call fails,
the \verb"FTAMabort" structure contained in the
\verb"FTAMindication" parameter \verb"fti" contains the reason for the failure.

\section	{Grouped Operations: File Management}
The \verb"FManageRequest" routine is used to issue a grouped management
request consisting of:
\begin{itemize}
\item	{\sf F-BEGIN-GROUP.REQUEST\/};

\item	{\sf F-SELECT.REQUEST\/} or {\sf F-CREATE.REQUEST\/};

\item	optionally, {\sf F-READ-ATTRIB.REQUEST\/};

\item	optionally, {\sf F-CHANGE-ATTRIB.REQUEST\/};

\item	{\sf F-DESELECT.REQUEST\/} or {\sf F-DELETE.REQUEST\/};
and,

\item	{\sf F-END-GROUP.REQUEST}.
\end{itemize}
The user initializes an \verb"FTAMgroup" structure
(described in tedious detail on page~\pageref{FTAMgroup})
and then invokes \verb"FManageRequest".
\begin{quote}\index{FManageRequest}\small\begin{verbatim}
int     FManageRequest (sd, ftg, fti)
int     sd;
struct FTAMgroup *ftg;
struct FTAMindication *fti;
\end{verbatim}\end{quote}
The parameters to this procedure are:
\begin{describe}
\item[\verb"sd":] the association-descriptor;

\item[\verb"ftg":] a pointer to an \verb"FTAMgroup" structure;
and,

\item[\verb"fti":] a pointer to an \verb"FTAMindication" structure, which is
always updated.
\end{describe}
If the call to \verb"FManageRequest" is successful,
then this corresponds to the appropriate {\sf .CONFIRMATION\/} events,
the \verb"fti_type" element of the \verb"fti" parameter is set to
\verb"FTI_MANAGEMENT",
and the \verb"fti_group" element contains the results.
Note that the data contained in the structure was allocated via \man malloc(3),
and should be released with the \verb"FTGFREE" macro when no longer needed.

Otherwise if the call fails,
the \verb"FTAMabort" structure contained in the
\verb"FTAMindication" parameter \verb"fti" contains the reason for the failure.

Upon receiving an \verb"FTI_MANAGEMENT" indication,
containing the appropriate {\sf .INDICATION\/} events,
the responder is required to generate the appropriate {\sf .RESPONSE\/}
events using the \verb"FManageResponse" routine.
\begin{quote}\index{FManageResponse}\small\begin{verbatim}
int     FManageResponse (sd, ftg, fti)
int     sd;
struct FTAMgroup *ftg;
struct FTAMindication *fti;
\end{verbatim}\end{quote}
The parameters to this procedure are:
\begin{describe}
\item[\verb"sd":] the association-descriptor;

\item[\verb"ftg":] a pointer to an \verb"FTAMgroup" structure;
and,

\item[\verb"fti":] a pointer to an \verb"FTAMindication" structure, which is
updated only if the call fails.
\end{describe}
If the call fails,
the \verb"FTAMabort" structure contained in the
\verb"FTAMindication" parameter \verb"fti" contains the reason for the failure.

\section	{Association Release}
The \verb"FTerminateRequest" routine is used to request the release of an
association,
and corresponds to an {\sf F-TERMINATE.REQUEST\/} action.
\begin{quote}\index{FTerminateRequest}\small\begin{verbatim}
int     FTerminateRequest (sd, sharedASE, ftr, fti)
int     sd;
PE      sharedASE;
struct FTAMrelease *ftr;
struct FTAMindication *fti;
\end{verbatim}\end{quote}
\newpage	%%% yikes!
The parameters to this procedure are:
\begin{describe}
\item[\verb"sd":] the association-descriptor;

\item[\verb"sharedASE":] shared ASE information;

\item[\verb"ftr":] a pointer to an \verb"FTAMrelease" structure, which is
updated only if the call succeeds;
and,

\item[\verb"fti":] a pointer to an \verb"FTAMindication" structure, which is
updated only if the call fails.
\end{describe}
If the call to \verb"FTerminateRequest" is successful,
then this corresponds to an {\sf F-TERMINATE.CONFIRMATION\/} event,
and it returns information in the \verb"ftr" parameter,
which is a pointer to an \verb"FTAMrelease" structure.
\begin{quote}\index{FTAMrelease}\small\begin{verbatim}
struct FTAMrelease {
    PE     ftr_sharedASE;

    struct FTAMcharging ftr_charges;
};
\end{verbatim}\end{quote}
The elements of this structure are:
\begin{describe}
\item[\verb"ftr\_sharedASE":] shared ASE information;
and,

\item[\verb"ftr\_charges":] any charges.
\end{describe}
Note that the data contained in the structure was allocated via \man malloc(3),
and should be released with the \verb"FTRFREE" macro when no longer needed.
The \verb"FTRFREE" macro
behaves as if it was defined as:
\begin{quote}\index{FTRFREE}\small\begin{verbatim}
void    FTRFREE (ftr)
struct FTAMrelease *ftr;
\end{verbatim}\end{quote}
The macro frees only the data allocated by \verb"FTerminateRequest",
and not the \verb"FTAMrelease" structure itself.
Further,
\verb"FTRFREE" should be called only if the call to the
\verb"FTerminateRequest" routine returned \verb"OK".

If the call to \verb"FTerminateRequest" is successful,
then the association has been released.
Otherwise the \verb"FTAMabort" structure contained in
the \verb"FTAMindication" parameter
\verb"fti" contains the reason for failure.

Upon receiving an {\sf F-TERMINATE.INDICATION\/} event,
the user is required to generate an {\sf F-TERMINATE.RESPONSE\/} action
using the\\	%%% hack
\verb"FTerminateResponse" routine.
\begin{quote}\index{FTerminateResponse}\small\begin{verbatim}
int     FTerminateResponse (sd, sharedASE, charging, fti)
int     sd;
PE      sharedASE;
struct FTAMcharging *charging;
struct FTAMindication  *fti;
\end{verbatim}\end{quote}
The parameters to this procedure are:
\begin{describe}
\item[\verb"sd":] the association-descriptor;

\item[\verb"sharedASE":] shared ASE information;

\item[\verb"charging":] charging information (if any);
and,

\item[\verb"fti":] a pointer to an \verb"FTAMindication" structure, which is
updated only if the call fails.
\end{describe}
If the call to \verb"FTerminateResponse" is successful,
then the association has been released.

\section	{Association Abort}
The \verb"FUAbortRequest" routine is used to request the ungraceful release
of an association,
and corresponds to an {\sf F-U-ABORT.REQUEST\/} action.
\begin{quote}\index{FUAbortRequest}\small\begin{verbatim}
int     FUAbortRequest (sd, action, diag, ndiag, fti)
int     sd;
int     action;
struct FTAMdiagnostic diag[];
int     ndiag;
struct FTAMindication *fti;
\end{verbatim}\end{quote}
The parameters to this procedure are:
\begin{describe}
\item[\verb"sd":] the association-descriptor;

\item[\verb"action":] an action result;

\item[\verb"diag"/\verb"ndiag":] a list of diagnostics
(and the number of diagnostics in the list);
and,

\item[\verb"fti":] a pointer to an \verb"FTAMindication" structure, which is
updated only if the call fails.
\end{describe}
If the call to \verb"FUAbortRequest" is successful,
then the connection is immediately closed,
and any data queued for the connection may be lost.

\section	{Error Conventions}
All of the routines in this library return the manifest constant \verb"NOTOK"
on error,
and also update the \verb"fti" parameter given to the routine.
The \verb"fti_abort" element of the \verb"FTAMindication" structure contains a
\verb"FTAMabort" structure detailing the reason for the failure.
For each diagnostic present (typically only one) in the \verb"FTAMabort"
structure,
the \verb"ftd_identifier" element of each diagnostic can be given as a
parameter to the procedure \verb"FErrString" which returns a null-terminated
diagnostic string.
\begin{quote}\index{FErrString}\small\begin{verbatim}
char   *FErrString (c)
int     c;
\end{verbatim}\end{quote}

\section	{Compiling and Loading}
Programs using the \man libftam(3n) library should include
\verb"<isode/ftam.h>".
These programs should also be loaded with \verb"-lftam" and \verb"-lisode".

\section	{An Example}
Read Section~\ref{unixftam:code} on page~\pageref{unixftam:code}.

\section	{For Further Reading}
The ISO specification for the International Standard on
File Transfer, Access, and Management is found in \cite{ISO.FTAM},
which is a four-part document.

%%% \section	{Changes Since the Last Release}\label{ftam:changes}
%%% A brief summary of the major changes between v~\isodeprevrsn/ and
%%% v~\isodevrsn/ are now presented.
%%% These are the user-visible changes only;
%%% changes of a strictly internal nature are not discussed.

