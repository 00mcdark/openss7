%run this through LaTeX with the appropriate wrapper

\chapter {Installing QUIPU}

\label{quipu:install}

This section describes how to install QUIPU, and make it operate in a basic
fashion.  This is reasonably prescriptive, as it should be possible to
install and operate a QUIPU DUA and/or DSA without too much 
knowledge about how it functions.

QUIPU comes in various separate parts of the ISODE source tree.  Only the
\man libdsap(3n) library, the DSA \pgm{ros.quipu} 
and the DUA interface DISH are 
``made'' as part of the default installation of ISODE.
This section assumes you have installed this part of ISODE.  
You should consult the \file{READ-ME} file in the
top level of the source tree to find out how to do this.
Your attention is drawn to the discussion of the \pgm{iaed} and
\pgm{dased} programs in this \file{READ-ME} file.  This can be used to
replace the \man isoentities(5n) and \man isoservices(5n) static files
with dynamic directory lookup\footnote{In fact, you are encouraged to
use these services rather that the static (potentially out of date) files.}.
\volone/ to \volfour/ of this manual describe other features of 
this installation of ISODE not specific to the OSI directory.

Before you install QUIPU
there are various compile time options you could consider setting which control
the operation of QUIPU, in particular of the DSA. 
These options are set in the file \file{h/quipu/config.h}, and are
described in the next section.
If you consider yourself a QUIPU ``novice'', then these 
are probably best left to their default values initially.
However, if you qualify as a ``large'' site you may want to consider enabling
the TURBO\_DISK options.


In the \file{others/quipu/uips/} directory of the ISODE source tree
there are various sub-directories, one for each optional user interface
(FRED, SD, POD and DE).
A version of DISH that runs directly 
from a \unix/ shell, \pgm{dishinit};
a script to create a default \file{.quipurc} file for new users, and 
\pgm{sid}: a set of scripts that utilise the shell version of DISH,
are all installed from the \file{dish} sub-directory.
The \file{manage} sub-directory contains an enhanced version of DISH
that can be used to manage alias attributes.
You should consult \file{others/quipu/uips/READ-ME}, for precise installation
details of all of these interfaces.

Each of these interfaces knows about the \verb+photo+ attribute that an entry
in the DIT can have.  In order to display the photographs, the photo
handling code must be compiled. Instructions on this can be found in
Section~\ref{dua:photo} of this manual and the file
\file{others/quipu/photo/READ-ME}.

\section{Compile Options}
\label{DSA:compile}
This section describes the options that can be set in the
\file{h/quipu/config.h} before compilation of the QUIPU code.

\begin{describe}

\item [\verb+\#define PDU\_DUMP+:]
If this is defined, and DISH is invoked with
\begin{quote}\begin{verbatim}
dish -pdu foobar
\end{verbatim}\end{quote}
then a directory \file{foobar} will be created, and 
will contain logs of all the X.500 PDUs sent to and from the DSA.
This is useful for debugging.

\item [\verb+\#define NO\_STATS+:]
If defined, the QUIPU will {\em not} produce statistical
and audit logs either the DSA or DUA.
These logs are useful to see what has been happening to your system.
If logging is allowed it can be turned off at runtime.
From the standpoint of security, it is advisable not
to select this option. Audit logs are very useful for detecting and tracing
attempts to break the security of the system.

\item [\verb+\#define CHECK\_FILE\_ATTRIBUTES+:]
If an EDB entry contains a FILE attribute, check that
the corresponding file exists at load time.  This significantly
increases the time taken to start a DSA.

\item [\verb+\#define QUIPU\_MALLOC+:]
Use a version of malloc() optimised for the memory
resident QUIPU DSA database.  It is believed to behave about 20\%
faster than the standard malloc algorithm in QUIPU's case.

\item [\verb+\#define TURBO\_DISK+:]
This option is described in the next section of this manual.

\item [\verb+\#define TURBO\_INDEX+:]
This option is described in the next section of this manual.

\item [\verb+\#define SOUNDEX\_PREFIX+:]
Consider soundex prefixes as matches.  For example, make
``fred'' match ``frederick''.  Defining this option gives
approximate matching behavior the same as in QUIPU-6.0.

\item [\verb+\#define COMPAT\_6\_0+:]
Operate in a manner compatible with QUIPU-6.0.  
For operations involving subtree searching across
multiple DSA this option is required when one of the DSAs is based on
QUIPU-6.0.  It is {\em not} required for QUIPU-6.1 and above.

\item [\verb+\#define USE\_BUILTIN\_OIDS+:] 
A DSA needs to know the OIDs
for various attribute type and object classes.  With this option, the
OIDs are built into the code (for efficiency, and to remove the table 
dependency), without it the {\em oidtables} are used.
\end{describe}

For the DISH interface, a compile option can be set to allow the use
of the GNU\index{GNU} {\em readline}\index{readline} package from the 
Free Software
Foundation.
Full details are given in the \file{quipu/dish/Makefile}.


\section{TURBO Options for Large Sites}
\index{TURBO}
This section describes two options (TURBO\_INDEX 
and TURBO\_DISK) that
can be set in the file \file{h/quipu/config.h}.  Both options are
most useful for sites with a large amount of data (e.g., thousands of
entries, or hundreds of EDB files).
The TURBO\_INDEX option is set by default.
If your DSA only holds a small amount of data you can skip this 
section.
\footnote{The TURBO\_AVL and TURBO\_LOAD options available in previous
releases of QUIPU are no longer required.}

The first option, TURBO\_INDEX, can be used to speed certain
kinds of searches by building an index based on selected
attribute types.  There are three DSA tailor file options
that are used to specify which attributes and
which portions of the tree are to be indexed.  They are
\verb"optimize_attr", \verb+index_subtree+, and \verb"index_siblings".
Section~\ref{dsa:tailor} describes these options in more
detail.  For the selected attributes and portions of the tree,
the index is searched for queries involving equality, approximate
equality, initial substring matching, and attribute existence.
{\em and} and {\em or} combinations of the above queries are also supported.
Queries involving negation, inequality, or substring matching
with no initial substring supplied are not indexed and will be
handled as usual, by a linear search of the database.

The second option, TURBO\_DISK, can be used to make modify operations
much faster, especially on large data sets.  It requires the use
of the \pgm{gdbm} library.  \pgm{Gdbm} is a
library of simple database routines
providing functionality similar to \pgm{dbm}
and \pgm{ndbm}, but without the
filesystem page size limitations of those systems.  
\pgm{Gdbm} is GNU\index{GNU}
software and is available from the Free Software Foundation.  It is
not a part of the ISODE.

Normally, when an
entry is modified, QUIPU writes the entire EDB file containing the entry
out to disk.  If the EDB file is very large, this can
take some time, especially on a heavily loaded system.  The TURBO\_DISK
option works by keeping the disk data in \pgm{gdbm} files instead of the
regular EDB files.  This way, when an update is made, only the 
affected entry need be written.  The performance increase 
is directly proportional to the size of the EDB file.
Whereas a normal update operation takes time proportional to the size
of the EDB file, with the TURBO\_DISK option it takes a small constant
amount of time.  If you don't have EDB files with
at least several hundred entries, it's probably not worth enabling
the TURBO\_DISK option.

To use the TURBO\_DISK option, you should add the following define to the 
\file{h/quipu/config.h} file:
\begin{quote}\begin{verbatim}
#define TURBO_DISK /* enable fast EDB update operations */
\end{verbatim}\end{quote}
You should also add the following to the 
\file{config/CONFIG.make} file:
\begin{quote}\begin{verbatim}
LIBGDBM=	-lgdbm
\end{verbatim}\end{quote}

If you have defined TURBO\_DISK, you will have to
convert your EDB file hierarchy into a gdbm file hierarchy
before running QUIPU.  This step
is only necessary once.  The \file{quipu/turbo} directory contains some
tools to help in this process.  The shell script \pgm{tree2dbm} will
convert an EDB file hierarchy to a \pgm{gdbm} file hierarchy.  To use it,
type \verb+tree2dbm database-directory+ where database-directory is the
directory where the EDB file hierarchy begins.  This script does a
find starting in the specified directory
for files named EDB and runs them through the
\pgm{edb2dbm} program which creates a file called \file{EDB.gdbm}.
The original EDB file is neither removed nor
molested, so you'll need roughly twice the disk space.  Alternatively, you
can run \pgm{edb2dbm} by hand on each EDB file.  The reverse operation
(converting from a \pgm{gdbm} hierarchy to an EDB hierarchy) is done
by the \pgm{synctree} shell script.  It is a good idea to run 
\pgm{synctree} out of \file{crontab} once in a while to update the
EDB file hierarchy.

Finally, it should be noted that parse errors will be reported
somewhat differently in the \file{dsap.log} file with the
TURBO\_DISK option enabled.  Since line numbers do not make much
sense in a \pgm{gdbm} file, errors will be reported based on the
Relative Distinguished Name of the offending entry.  If you get
a parse error (because of a non-printable character, for example),
the best approach is to do something like this:
\begin{quote}\begin{verbatim}
	edbcat EDB >bob
\end{verbatim}\end{quote}
Edit the file \file{bob}, locate the entry, fix the problem, then
\begin{quote}\begin{verbatim}
	edb2dbm bob
	mv bob.gdbm EDB.gdbm
\end{verbatim}\end{quote}
where this procedure assumes you are in the directory containing
the bad EDB.gdbm file.  The \pgm{edbcat} program can be found in the
\file{quipu/turbo} directory and is used to convert from \pgm{gdbm}
back to plain text EDB format.

\section{Files}
Regardless of how you install QUIPU and the ISODE,
the number of files needed to run QUIPU is quite small.

In ISODE's \verb"BINDIR" directory,
typically \file{/usr/local/bin/},
there are a few programs of interest:
\begin{describe}
\item[\verb+dish+:]	The directory shell\\
		This is discussed in Chapter~\ref{dish}.

\item[\verb+bind+:]	Shell interface to DISH\\
		There are actually several links (listed below) to a program called
		\pgm{bind}.
		These act to export the DISH interface to the \unix/
		shell.
		As such,
		you can issue commands to DISH from the shell,
		rather than running DISH directly.
\begin{quote}
\begin{tabular}{ll}
\verb+add+ &
\verb+compare+\\
\verb+delete+ &
\verb+dsacontrol+\\
\verb+list+ &
\verb+modify+\\
\verb+modifyrdn+ &
\verb+moveto+\\
\verb+search+ &
\verb+showentry+\\
\verb+showname+ &
\verb+squid+
\end{tabular}
\end{quote}

\item[\verb+fred+:]	A white pages user interface\\
		See Chapter~\ref{DUA:fred}.

\item[\verb+editentry+:]	Edit a directory entry\\
		This is a simple shell script that DISH invokes when you
		ask DISH to edit an entry in the directory.

\item[\verb+unbind+:]	Unbind from DISH\\
		This command is used to terminate DISH.
\end{describe}
In ISODE's \verb"SBINDIR" directory,
typically \file{/usr/etc/},
the DSA resides:
\begin{describe}
\item[\verb+ros.quipu+:]	The QUIPU DSA\\
			This program will be started once, for each DSA you
			are running, from rc.local.
			A script is provided to invoke this program,
			in case you need to restart it.

%%%\item[\verb+dsaconfig+:]	a configurator for Level-1 DSAs.
\end{describe}
In ISODE's \verb"ETCDIR" directory,
also typically \file{/usr/etc/},
there are a few programs and files of interest:
\begin{describe}\sloppy
\item[\verb+oidtable.at+, \verb+oidtable.gen+, \verb+oidtable.oc+:]

			These define the attribute types,
			generic object identifiers,
			and object classes known to the system.
			(An object identifier is a method used to unambiguously
			encode, among other things,
			the names of attributes and object classes.)
			These files you never deal with unless they are
			accidentally corrupted.
			
\item[\verb+dsaptailor+:]	

			This is the run-time tailor file for the DUAs.
			You will configure this file initially and then
			probably leave it alone.


\item[\verb+isoaliases+, \verb+isobjects+, \verb+isoentities+, \verb+isomacros+, \verb+isoservices+:]

			These are 
			various databases used by the ISODE.
			These files you never deal with unless they are
			accidentally corrupted.

\item[\verb+isologs+:]

			This script runs nightly under \man cron(8) to
			trim the ISODE log files, kept in ISODE's \verb"LOGDIR"
			directory,
			typically \file{/usr/tmp/}.
			This file you never deal with unless it is
			accidentally corrupted.

\item[\verb+isotailor+:]	

			This is the run-time tailor file for the ISODE.
			You will configure this file initially and then
			probably leave it alone.
\end{describe}

\chapter {Configuring a DUA}

It is suggested that you try to get a DUA operational by connecting to
a ``well-known'' DSA before you attempt
to operate a local DSA.
Or, if you have a DSA from a previous release of QUIPU --- try and connect to
that.

\section {Connecting to a DSA}
\label{dua:connect}

A DUA essentially only needs to know one thing to be able to contact a
DSA: that is the OSI network address of the DSA.
This parameter is defined in the file \file{dsaptailor}, 
together with some other parameters.
The full set of parameters is described in Section~\ref{dua:tailor}.

The \verb"dsa_address" parameter defines a local name and the network 
address of the DSA to 
initially contact. For example,
\begin{quote}\begin{verbatim}
dsa_address  vicuna  Internet=bells.cs.ucl.ac.uk+50987
\end{verbatim}\end{quote}
declares that the DSA locally referred to by the name \verb"vicuna" is 
contacted by calling the network
address \verb"Internet=bells.cs.ucl.ac.uk+50987".
The syntax of network addresses is discussed in
briefly in Section~\ref{DSA:address} and more fully in 
\voltwo/ of this manual 
and \cite{String.Addresses}.
\index{presentation addresses}

As shown, the address is preceded by a private key \verb"vicuna".  This can
be used in some DUAs (including DISH) to specifying the address of the DSA 
to contact.
If there are more than one \verb"dsa_address" entries, the first
entry will be used to supply the default DSA address.

A default \file{dsaptailor} file 
(taken from the \file{dsap/dsaptailor} file of the ISODE source tree) 
is installed as
\file{dsaptailor} in the ISODE \verb"ETCDIR" directory
(usually \file{/usr/etc/}) when
the \file{dsap} library is installed, this supplies the addresses
of various DSAs that you
may be able to access.
 
To try to connect to one of the DSAs listed in \file{dsaptailor}, 
invoke DISH, with a \verb"-call"
flag; e.g., \verb"dish -call giant" will try to contact the DSA
``giant'' (Giant Tortoise!)
running at the University of London Computer Centre (ULCC).
If the connection is successful, then the prompt \verb"Dish ->" will be returned.
If the connection fails, the
program will exit with an appropriate error
message.

If this fails
you might want to 
try connecting to some of the other registered DSAs,
for example, try \verb"dish -call alpaca", \verb"dish -call eel"
or 
\verb"dish -call anaconda".

Many of these top level DSAs do not allow anonymous connections.  If
you see the 
message \verb+inappropriate authentication+ as a result of your
connection attempt you will probably need
to supply a DN using the \verb+-username+ flag to DISH.

If you fail to contact a DSA at this point, there are likely to be lower
level problems. You should turn up the ISODE logging 
(see \voltwo/ of this manual) to see what is happening to the 
network calls.

If you invoke DISH without a \verb"-c" flag (using the default \file{dsaptailor}), it will 
try to connect to the DSA defined by the first
\verb"dsa_address" entry.

DISH is described in full in Chapter~\ref{dish} 
of this manual.

\section{Tailoring}
\label{dua:tailor}
\index{dsaptailor}

The program configuration is tailored to
allow you to change logging levels, and other parameters at
runtime.  
It is used by the QUIPU DUA procedures, and by the QUIPU User Interfaces.

The file \file{dsaptailor} is used for this purpose and
consists of single value entries (e.g., oidtable), unless otherwise
stated (e.g., dsaplog). 
Each entry has a parameter followed by its
value.
The various options are:
\begin{describe}
\item [\verb"oidtable":]
The path for the OID definition tables.
Note: It is best to have this appear as the first entry of the
\file{tailor} file, as 
other entries may contain attributes that need to be looked up in these
tables
There are three:
\begin{enumerate}
\item \file{file.gen},
which contains generic names for building OIDs;
\item \file{file.at},
which contains the OIDs for attributes; and,
\item \file{file}.oc,
which contains the OIDs for object classes.
\end{enumerate}
For example,
\begin{quote}\small\begin{verbatim}
oidtable    /usr/lib/quipu/OIDTable
\end{verbatim}\end{quote}
will direct the DSA to consult:
\begin{quote}\begin{verbatim}
/usr/lib/quipu/OIDTable.gen
/usr/lib/quipu/OIDTable.at
/usr/lib/quipu/OIDTable.oc
\end{verbatim}\end{quote}
By default this variable is set to \verb"oidtable" which refers to the tables
\file{oidtable.*} in the ISODE \verb"ETCDIR" directory.

\item [\verb"dsa\_address":]
This parameter is described in Section~\ref{dua:connect}.

\item [\verb"dsaplog":]
Tailoring for the normal logging file.
Each entry consists of one or more key/value pairs expressed as:
\begin{quote}\small\begin{verbatim}
key=value
\end{verbatim}\end{quote}
The keys are:
\begin{describe}
\item [\verb"file":]
The name of the logfile.
\item [\verb"size":]
The size in KBytes to which the logfile should be allowed to grow.
When the log has reached this size, if the ``zero'' option
below is set, then the log will be truncated; otherwise, no further
logging will take place.
\item [\verb"level":]
The levels of logging to be written to this log file.
This can be any of the following levels:
\begin{describe}
\item [\verb"fatal":]
fatal errors only.
\item [\verb"exceptions":]
serious, but hopefully temporary, errors.
\item [\verb"notice":]
general logging information.
\item [\verb"trace":]
program tracing.
\item [\verb"pdus":]
pdu tracing.
\item [\verb"debug":]
full tracing of events.
\item [\verb"all":]
log all events.
\end{describe}
For example to have all errors written to the file you will need
\begin{quote}\footnotesize\begin{verbatim}
dsaplog level=fatal level=exceptions
\end{verbatim}\end{quote}
\item [\verb"dlevel":]
Do not log the specified log level, this is the opposite of the above
entry.
\item [\verb"dflags"/\verb"sflags":]
The flags associated with the log may be set (with \verb"sflag")
or unset (with \verb"dflag"). The allowable options are:
\begin{describe}
\item [\verb"close":]
close the log after each entry.
\item [\verb"create":]
create the log file if it does not exist.
\item [\verb"zero":]
truncate the file when it gets too big.
\item [\verb"tty":]
copy the logging information to the users tty.
\end{describe}
\end{describe}
An example might be:
\begin{quote}\footnotesize\begin{verbatim}
dsaplog level=notice size=30 file=quipulog dflags=close
\end{verbatim}\end{quote}
This says log events at ``notice'' level into the file \file{quipulog}, do not
let the file grow larger than 30Kbytes; and do not close the file
after each logging message.

\item [\verb"stats":] Used to control the level of statistical logging
(parameters as for \verb+dsaplog+ above).

\item [\verb"local\_DIT":]
The argument is a distinguished name.  When some User Interfaces start, you
will be automatically moved to this position in the DIT.

\item [\verb"oidformat":]
Defines how object identifiers should be printed.
Use
\begin{quote}\small\begin{verbatim}
oidformat   short
\end{verbatim}\end{quote}

to print in short local key form, e.g.,
\begin{quote}\small\begin{verbatim}
Country
\end{verbatim}\end{quote}
or,
\begin{quote}\small\begin{verbatim}
oidformat   long
\end{verbatim}\end{quote}

to print in long object identifier form, e.g.,
\begin{quote}\small\begin{verbatim}
joint.ds.attributeType.country
\end{verbatim}\end{quote}
or,
\begin{quote}\small\begin{verbatim}
oidformat   numeric
\end{verbatim}\end{quote}

to print in numeric form, e.g.,
\begin{quote}\small\begin{verbatim}
2.5.4.6
\end{verbatim}\end{quote}

\item [\verb"photo":]
The argument has two parts, a ``terminal type'' such as
``sun'' or ``xterm''; the second is the name of the process that should
handle displaying of photographs; for example,
\begin{quote}
photo xterm Xphoto
\end{quote}
tells the DUA to call the process Xphoto to handle photograph attributes if
the user is on a terminal of type ``xterm''.
Handling photographs is described more fully in Section~\ref{dua:photo}.

\item [\verb"quipurc":]
If the argument has the value \verb"on", then a program called 
\pgm{dishinit} will be run every time a user without a \file{.quipurc}
file tries to access the directory. \pgm{dishinit} is discussed 
in Section~\ref{dishinit}.

\item [\verb"sizelimit":]
Defines the maximum number of entries a successful list or search should
return.
For example,
\begin{quote}\small\begin{verbatim}
sizelimit 20
\end{verbatim}\end{quote}
sets the DAP default service control ``sizelimit'' to be 20 entries.

\item [\verb"timelimit":]
Defines the maximum number of seconds a list or search should
return.
For example,
\begin{quote}\small\begin{verbatim}
timelimit 30
\end{verbatim}\end{quote}
sets the DAP default service control ``timelimit'' to be 30 seconds.

\item [\verb"ch\_set":]
If set to \verb+ISO8859+ this tells a DUA you have access to an ISO
8859-1 font which can be used for displaying T.61 characters, see
Section~\ref{T61String} for details.

\end{describe}

\chapter {Configuring a DSA}

This chapter discusses how to configure a QUIPU DSA.
We recommend that you get a DUA running before you try to
get a DSA working.

\section{Basic Formats and Structures}

All of the information a DSA requires is
stored on disk and is text-structured.  This includes
various files (described later), and the local DIT database itself.
A complete BNF description of the files 
is given in Appendix~\ref{bnf} on page~\pageref{bnf}.

\subsection {Entry Data Block}
\label{edb}
A key component of the directory database is the entry data block, which is
described fully in \cite{QUIPU.Design}.
Figure \ref{example_edb} shows an example EDB \index{EDB} file containing two \verb+person+
entries.

\begin{figure}
\smaller
\begin{quote}\begin{verbatim}
MASTER
19891025113003Z
CN= Colin Robbins 
CN= Colin John Robbins 
Phone= +44-1-387-7050 ext 3683 
Surname= Robbins 
Room= G10 
Userid= crobbins
userClass= csstaff
rfc822Mailbox= C.Robbins@cs.ucl.ac.uk
Photo= {FILE}crobbins.photo 
objectClass =thornPerson & quipuObject
acl=others # none # attribute # photo 
acl=self # read # attribute # photo 

CN= Steve Kille 
CN= Steve E. Kille & Stephen Kille 
Phone= +44-1-387-7050 ext 7294 
Surname= Kille 
objectClass = thornPerson & quipuObject 
Room= G24 
Userid= steve
userClass= csstaff
rfc822Mailbox= S.Kille@cs.ucl.ac.uk
\end{verbatim}\end{quote}
\caption{Example EDB File}
\label{example_edb}
\end{figure}


An EDB file contains a header, this is optionally
but typically followed by a sequence of entries.

The header consists of two lines of text, the first must contain the string
\verb"MASTER", \verb"SLAVE" or \verb"CACHE", 
which indicates whether the data in the EDB
file represents the authoritative \verb"master" data, a \verb"slave"
copy of all the data, or some \verb"cached"d entries.

The next line of the EDB 
is a string that describes the version of the EDB. Every
time the EDB is altered, the version\index{version --- of an EDB} number
should be changed, so that
\verb"SLAVE" EDBs elsewhere will be automatically updated.
When a DSA alters an EDB file, it writes the current (UTC) time in string
format as the version string.

Generally following the header are a sequence of blank line separated
entries.  The concept of a null EDB file that contains just a
header is allowed but discouraged.  However it is sometimes useful as
a temporary measure when creating a database.

An entry consists of a set of attributes, each attribute begins on a
new line of the file.
Section~\ref{attributes} discusses attributes in more detail and
Chapter~\ref{syntaxes} describes the syntaxes used by all the
attributes QUIPU recognises.

The first line of an entry is the relative distinguished name
\index{RDN} (RDN) of the entry.
The subsequent line contains the non distinguished attributes.


\subsection{Object Class Attribute}
Of all the attributes an entry may have, the \verb+objectClass+\index{objectClass attribute}
attribute is one of the most important from the configuration point of view.
It defines the set of mandatory and optional attributes that must and may be
present in the entry.
For example, the object class \verb+person+ insists that there is a \verb+surname+
attribute, and there may optionally be a \verb+telephomeNumber+ attribute.
QUIPU knows about all the standard object classes and attributes, 
those defined by COSINE/Internet \cite{Cosine.NA} and
those defined by QUIPU itself (see Appendix~\ref{naming}).
The full set of object classes and attributes a DSA knows about is
defined by the ``oidtables'' which are explained in Section~\ref{oidtables}.

An entry can belong to more that one object class.
For example, an entry representing an organisation might have the following
object class attribute:
\begin{quote}\small\begin{verbatim}
objectClass = organization & quipuNonLeafObject
\end{verbatim}\end{quote}
And a person within that organisation might use the following object class
definition:
\begin{quote}\small\begin{verbatim}
objectClass = organizationalPerson & quipuObject & thornPerson
\end{verbatim}\end{quote}
Every entry in a QUIPU DSA should belong to either the \verb"quipuObject"
or 
\verb"quipuNonLeafObject" object classes\footnote{An
entry may belong to \verb"ExternalNonLeafObject" instead IF it is
actually represented in the DIT by a non-QUIPU DSA  --- See
Section~\ref{DSA:nonquipu}.}, as this allows
attributes the DSA needs to be added to an entry.

Further, object classes posses the notion of 
\verb"class inheritance".
This means that an object class can be defined as a subclass of
a previously defined object class with additional refinements.
As a subclass,
the newly defined object inherits all the semantics of its
superclass,
in addition to having additional semantics.

For example,
the directory defines an object class called \verb"person".
This object class defines the attributes which a person in
the real world might have.
It may be useful to refine this somewhat to talk about persons
who have network access.
So, we need a new object class, e.g., \verb"netPerson".
This can be defined in a straight-forward fashion:
\begin{itemize}
\item The object class \verb"netPerson" is a subclass of the object class
\verb"person" which {\em may\/} contain an additional attribute,
\verb"netMailbox".

\item The syntax of an \verb"netMailbox" is a simple string of printable
characters which is not case sensitive when performing comparisons.
\end{itemize}

It is a QUIPU requirement that every entry that is not a leaf of the DIT 
should belong to the object class
\verb"quipuNonLeafObject". 

This class has one mandatory attribute:
\begin{describe}
\item[\verb+masterDSA+:]\index{masterDSA attribute}
			identifies the directory entity which is responsible
			for maintaining the master EDB for the children of
			this entry.
			The value is a Distinguished Name.
\end{describe}
There is typically a single master for a particular entry in the tree.
Hence, this value is usually single-valued.
When an entry is to be modified,
the directory must contact the entity responsible for the MASTER EDB for
that entry in order to perform the modification.

This class has two optional attributes:
\begin{describe}
\item[\verb+slaveDSA+:]\index{slaveDSA attribute}
			identifies any directory entities which have
			authoritative copies of the EDB for the children
			of this entry, and are prepared to resolve 
			operations on that EDB file for a remote DSA.
			The value is one or more Distinguished Names.

\item[\verb+treeStructure+:]\index{treeStructure attribute}
			identifies the object classes which may exist
			immediately below this entry.
			The value is one or more object classes.
			See Section~\ref{quipu:schema} for full details
			of how to set this attribute.
\end{describe}
Since a fundamental assumption of the directory is that reads (queries)
occur much more frequently than writes (updates),
it is common to have several entities containing authoritative copies of an
EDB.
By keeping copies locally,
queries can be answered with less latency.

\subsection {Database Structure}

All the local information held by a QUIPU directory is held in an in-core
database, this is loaded from disk when the DSA starts. 

The data on disk is held in a \unix/ tree of EDB\footnote{Throughout
this section the term EDB is used in the generic sense, and
includes \file{EDB.gdbm} files if the TURBO\_DISK compile option is used}
files that map the DIT.
At every level in the DIT for which the DSA holds data, there is a single
file called \file{EDB}\index{EDB}\index{EDB.gdbm file}.
The top level of the DIT is stored in the \unix/ directory defined by
the \verb+treedir+ variable in the \file{quiputailor} file.
For example, the setting
\begin{quote}\small\begin{verbatim}
treedir     /usr/etc/quipu-db/
\end{verbatim}\end{quote}
would define that the top level of the DIT would be found in this
\unix/ directory.
If you hold a copy of the root EDB file, it will be found here.

If an entry defined in an \file{EDB} file has
children stored locally, then the \file{EDB} file for the children 
will be found in a
sub-directory whose name is the string encoded relative
distinguished name of the entry.
For example, underneath the root, there are typically countries such
as ``c=GB''.  The data for this will be held in the file
\file{c=GB/EDB}, or to give the fullpath name
\file{/usr/etc/quipu-db/c=GB/EDB}.

This mapping continues all the way down the DIT hierarchy, so for example, if an \file{EDB} file has an entry whose RDN is
``ou= Computer Science'', then if the entry for
``ou= Computer Science'' has sibling entries and these are stored locally,
they can be found in the file \file{ou=Computer Science/EDB} 
relative to the directory that contains the EDB file with the
``ou=Computer Science'' entry.

NOTE that the case sensitivity of the sub-directory naming is one of the
few areas within QUIPU where string matching is case sensitive.
The case of the attribute type is taken from the definition of the attribute
in the oid tables, whereas the case of the attribute value is the same as
that found in the EDB file.
Spacing is also important.  There should be no spaces either side of
the ``='' sign, and only one space between each word contained in the name.

When an entry is modified, a new \file{EDB} file is rewritten to
disk.  The old \file{EDB} is renamed \file{EDB.bak} to provide
a limited backup.  

\subsection{Long Distinguished Names}
\label{EDB:mapped}
There is a problem with the above method for naming \unix/ sub-directories
with some versions of \unix/ --- particularly System~V.
In these systems directory names are limited in length.
Some versions of \unix/ will not allow space characters in filenames
(in any case they are hard to manage because of all the quoting required).

To allow for this, \verb"any" distinguished name can be given a 
``mapping name'', which will be used as the \unix/ sub-directory name.
For example the entry for ``o=University College London'', may be
mapped onto the name ``UCL'' and thus
stored in the file \file{UCL/EDB}.
The mapping names are specified in a file called
\file{EDB.map}\index{EDB.map file},
found in the same directory as the
EDB file holding the entry to be mapped.
So if the file \file{c=GB/EDB} contains an entry for 
``o=X-Tel Services Ltd'', then the file \file{c=GB/EDB.map} may be
used to map this onto ``X-Tel''.

The syntax of the \file{EDB.map} file is:
\begin{quote}\begin{verbatim}
<Distinguished Name> "#" <Mapped name>
\end{verbatim}\end{quote}
so for the example used above, the file will contain:
\begin{quote}\begin{verbatim}
o=X-Tel Services Ltd#X-Tel
\end{verbatim}\end{quote}
Only RDNs you want to map need to be in this file.
If a name is not found in the mapping file, then the long directory name will
be used. 

When a DSA needs to create a sub-directory (e.g., after an add operation)
it will use the relative distinguished name for each sub-directory,
unless the name is longer than the maximum number of allowed
characters (usually 15).  In this case, the DSA will generate a shorter
mapped name and write this to the \file{EDB.map} file.
A generated mapped name is based on the \unix/ {\em mktemp} procedure
call and produces names such as ``XTelServia01950''.

\section {Setting Up an Initial DSA}

These instructions are assuming that you are trying to set up a DSA with the
following characteristics:
\begin{itemize}
\item It is the first DSA in an Organisation.
\item It is not the first DSA in the Country.
\item It holds a copy of the root EDB.
\end{itemize}

This is found in the example:
\begin{quote}
\file{others/quipu/quipu-db/organisation} 
\end{quote}

There are two other examples which might also be used as illustrations for
national DSAs and organisational DSAs not holding a copy of the root EDB.
These are in :
\begin{quote}
\file{others/quipu/quipu-db/national}
\end{quote}
and
\begin{quote}
\file{others/quipu/quipu-db/non-root}.
\end{quote}

Note that if you are going to be running a DSA in the United States,
then you should skip this section and refer to the document
\cite{PSI.Admin},
which is provided in the ISODE documentation set
(look in the source tree area for the directory \file{doc/whitepages/admin/}).
This document describes turn-key installation mechanisms for DSAs in the
United States.

To start a DSA with one of these example databases
go into the relevant database directory and type:
\begin{quote}\begin{verbatim}
$(SBINDIR)ros.quipu -t ./quiputailor
\end{verbatim}\end{quote}

The \tt -t\rm \ flag tells QUIPU to use the \file{tailor} file \file{./quiputailor}
rather than the default file (\file{quiputailor} in the ISODE \verb"ETCDIR"
directory).

This will cause the DSA to print some logging information onto the screen,
followed by the message
\begin{quote}\begin{verbatim}
DSA c=GB@cn=toucan has started on localHost=17003
\end{verbatim}\end{quote}

\begin{figure}
\smaller
\center
\begin{quote}\begin{verbatim}
cn=Toucan
presentationAddress= localHost=17003
edbinfo= #cn=giant tortoise#
description= Demonstration DSA
description= Bird with large colourful bill.
objectClass= quipuDSA & quipuObject
manager= c=GB@o=X-Tel Services Ltd@cn=Camayoc
acl= others # compare # attributes # userPassword
userPassword = toucan
quipuVersion= quipu 6.8 #3 (trellis) of Mon Feb 4 09:26:47 GMT 1991
supportedApplicationContext= QuipuDSP & X500DSP & X500DAP
\end{verbatim}\end{quote}
\caption{Example DSA Entry}
\label{DSA:Toucan}
\end{figure}

The default setup assumes you have TCP/IP access and starts a DSA on the IP
address of the local machine (127.0.0.1).  If you do not have TCP/IP, you
will need to change the address the DSA will attempt to listen on. The
next section delves into the world of addresses!

The presentation address of the example DSA is found in the file
\begin{quote}
\file{others/quipu/quipu-db/organisation/c=GB/EDB}
\end{quote}
The first entry in this file is the entry for a DSA called
``c=GB@cn=Toucan'', which is the name of the DSA we are trying to start (as
defined by the ``mydsaname'' entry in \file{quiputailor}).
The entry is shown in Figure~\ref{DSA:Toucan}.
The attribute \verb"presentationAddress" defines the address that the DSA is
going to listen to the network on.
If you need to listen on a different address, you should change the value of
the attribute to the address you want to listen on.

Having started your DSA you should be able to connect to it by
invoking DISH. If you are using the default DSA address, and are
using the default \file{dsaptailor} file, then invoking DISH
without arguments is sufficient.
If you are not using the default address or \file{dsaptailor} file, then you
will need to edit \file{dsaptailor} in the ISODE \verb"ETCDIR" directory.
You should add an entry
\begin{quote}\begin{verbatim}
dsa_address  toucan  <presentation address>
\end{verbatim}\end{quote}
Note that \verb+<presentation address>+ should have {\em exactly} the same
value as 
the \verb+presentationAddress+ attribute in the DSAs entry in the EDB file.
Now to contact the DSA use
\begin{quote}\begin{verbatim}
dish -c toucan
\end{verbatim}\end{quote}


Once connected to the DSA, try issuing the
command:
\begin{quote}\small\begin{verbatim}
list "@c=GB@o=University College London@ou=Computer Science"
\end{verbatim}\end{quote}
You should get a list of four names 
back:
\begin{quote}\begin{verbatim}
1. commonName=Colin Robbins
2. userid=quipu%commonName=Colin Robbins
3. commonName=Steve Kille
4. commonName=Michael Roe
\end{verbatim}\end{quote}
If this happens you have a working DSA.
(The entry numbered 2 is an example of an RDN with multiple values !)

\subsection{Presentation Addresses}\index{presentation addresses}
\label{DSA:address}

This section is a brief introduction into presentation addresses, and
may be all you need to know.  For the brave, more details are given in
\voltwo/ of this manual and
\cite{String.Addresses}.

The example address used in the previous section for the demonstration
DSA was ``\verb+localHost=17003+''.  This is an address using the
TCP/IP transport stack, and is the TCP loop back address.  You should
{\em not} use this address in any DSA involved in the pilot DIT for reasons
explained at the end of the section.

If you want to use the TCP/IP transport stack (using RFC1006) there are two
possibilites for an address.  
If you are connected to the Capital-I Intenet network, you can use
addresses of the form
\begin{quote}\begin{verbatim}
Internet=128.16.5.31+17003
\end{verbatim}\end{quote}
``128.16.5.31'' is the IP address of your machine, which can usually be
found in the \file{/etc/hosts} file on your system.  You can use the
DNS name of your host instead, but the ISODE will replace this with the
IP address when possible to aid portability of addresses.
``17003'' is the TCP port number the DSA will listen on.

If you want to use TCP/IP on a local \ethernet/, that is not connected to
the Capital-I Internet, you will need to define a local network
community. 
This process is described in full in \voltwo/ of this
manual and should result in addresses of the form
\begin{quote}\begin{verbatim}
LOCAL-ETHER=128.16.5.31+17003
\end{verbatim}\end{quote}

Do {\em not} use \verb+Internet=+ addresses {\bf unless} you are connected to the Internet.


Now onto the X.25(80) community.
If you have IPSS access, you can use addresses of the form
\begin{quote}\begin{verbatim}
Int-X25(80)=23421920030045
\end{verbatim}\end{quote}
where 23421920030045 is the DTE of you host.
If you plan to have more than just a DSA (e.g., \pgm{iaed} or
\pgm{tsapd}) listening on X.25 on the same machine, we advise 
the use of a two-digit
subaddress (45 in the example).
The DTE should have the DNIC included; full details of tailoring
DNICs are given in \man isotailor(5n).
The X.25 parameters PID and CUDF can be specifed if needed, see
\cite{String.Addresses} for details.

If you do not have connectivity to the IPSS network, as in the TCP/IP
case, you will need to define a local community.
The ISODE has knowledge of two such X.25 communites built in.
In Europe, the IXI network should be addressed using an address of the form
\begin{quote}\begin{verbatim}
IXI=20433450210398
\end{verbatim}\end{quote}
In Great Britain, the Janet network should be addressed using an
address of the form
\begin{quote}\begin{verbatim}
Janet=00002100102998
\end{verbatim}\end{quote}

Your DSAs address should contain a component for every network you
have access to.  Multiple components can be linked using the
``\verb+|+'' symbol; for example:
\begin{quote}\begin{verbatim}
Internet=128.16.5.31+17003|IXI=20433450210398
\end{verbatim}\end{quote}


Using this addressing mechanism it is possible to use transport,
session and presentation selectors; however, in the context of a QUIPU
DSA they are unnecessary. Hence we advise against their use; again
details can be found in 
\cite{String.Addresses}.


Correctly defining your DSA's address is important. A paper 
\cite{QUIPU.Navigate} describes the problem in detail.  Briefly, X.500
was defined assuming a single global network. Unfortunately the world
is not yet like that: there are at least two major communities.  QUIPU
DSAs know about this, and make a careful choice between use of
referrals and chaining in an attempt to make sure all operations
suceed.  If your DSA is incorrectly addressed, other DSAs may make the
wrong assumption about your DSA, and so your DSA will not be able to contact
certain parts of the DIT, and they will not be able to contact you.
For a similar reason you should not use the ``localHost'' macro for
addreses, as other DSAs will think they can connect to the DSA in question.


\subsection {Choosing a Name for Your DSA}
\label{dsa:naming}
Every QUIPU DSA {\em must} have an entry in the DIT, hence your DSA will
need a unique {\bf distinguished} name.  This entry is used by other DSAs to
identify your DSA, and so is needed if other DSAs are going to be able
to see your DSA.  The examples are tailored to start a DSA called
``toucan'', and will be sufficient to get an example DSA started, but it
is not unique, so will not be of much use when you want to start
adding your own data, and want to connect into the global DIT.  
{\em Note} that the DN has to be unique, but not the RDN component, thus both
``cn=Wombat'' and ``c=GB @ cn=Wombat'' are allowed!

There are other two aspects to consider in choosing a name for your DSA.


Firstly, it is a QUIPU convention that DSAs should be named 
after endangered South
American wildlife, and that the entry for the DSA should contain a
description of the animal or plant in question.
Some example animal names are shown in Table~\ref{wildlife}; names of
wild plants can also be used!
A more comprehensive list of animal names 
can be found in the IUCN's ``Red Book''
\cite{IUCN.Mammal}.
This is not a just a game, there is a serious point.  The authors
believe it is wrong to name DSA ``UCL.DSA'' or similar, as this binds
the DSA to the organisation too tightly, especially at the higher
levels of the DIT.

Secondly, the entry for your DSA must be visible to other DSAs that do not know
how to contact your DSA.
So, the {\em master} copy of your DSA's own entry must be held at least one level higher up
in the DIT than the part of the DIT it holds as \verb"MASTER" data.
For example the DSA which holds
``c=GB @ o=University College London @ ou=Computer Science'', 
must be held at the
``c=GB @ o=University College London'' level 
(e.g., ``c=GB @ o=University College London @ cn=wombat'') or above.
Such naming also helps prevent loops as described more fully in
\cite{QUIPU.Design}.

Typically, you will not hold the master EDB file containing your DSA
locally.  However, you do still have control of the entry as if you
held the master copy.  This is explained in more detail in
Section~\ref{DSA:ownentry}. 

In practice DSAs should be named fairly high up the tree.
Each country should have at least two DSAs named at the root level.
Each Organisation should have at least two DSAs named at the national level.

\tagtable[tp]{q-1}{Endangered South American Wildlife}{wildlife}

You should use DISH to find out if the name you want is already taken.
For example,
if you are creating a DSA for an organisation in Great Britain,
you might use:
\begin{quote}\small\begin{verbatim}
% dish -c "Giant Tortoise"
Welcome to Dish (DIrectory SHell)
Dish -> search @c=GB -filter objectClass=dsa -nosize
\end{verbatim}\end{quote}
This will print out the list of names already in use.

\subsection{Setting up Your DSA}

Having chosen a name, you will need to tell your DSA its name, and 
make sure it can find an EDB file for its own entry.

A DSA finds it own name from the ``mydsaname'' variable defined in the file
\file{quiputailor} (see Section~\ref{dsa:tailor} for details).
An example \file"quiputailor" entry would be:
\begin{quote}\small\begin{verbatim}
mydsaname:	"c=GB@cn=a dsa name"
\end{verbatim}\end{quote}
Having read this name from \file"quiputailor", a DSA will try to find
the corresponding entry, in this case by looking in the \file{c=GB/EDB} file.

If you hold the EDB that the entry should be in, simply
make sure the entry is in that EDB and then it will be found.
If you do not hold, and do not want to hold the EDB in which your
DSA is named (e.g., your DSA is called \verb"c=GB@cn=toucan" but 
you do not hold the EDB \verb"c=GB"), then you should (normally)
supply a cached copy of the EDB which contains only the 
entry for the DSA, and not all the other entries the full EDB would have.
If you do not supply it, your DSA will have to rely on other DSAs before it
can start.

Having located its own entry, a DSA will know its network 
address by looking at the \verb+presentionAddress+ attribute,
hence it can start listening for operations.

But first it must load the database. This process is described in full
in Section~\ref{dsa:starting}. 

In its own entry it will find a \verb"edbInfo" attribute, this 
describes which EDBs the DSA is expected to hold.
The format of the \verb"edbInfo" \index{edbInfo attribute} attribute is described fully in 
Section~\ref{slave_update}, but essentially the first parameter
supplied says ``load this EDB''. Thus the attribute
\begin{quote}\begin{verbatim}
edbInfo = ##
edbInfo = c=US@o=The Wollongong Group ##
edbInfo = c=GB ##
\end{verbatim}\end{quote}
requests that the EDB files 
\begin{quote}
\file{quipu-db/EDB}
\end{quote}
(signified by no data before the first ``\verb+#+'' sign),
\begin{quote}
\file{quipu-db/c=US/o=The Wollongong Group/EDB} 
\end{quote}
and 
\begin{quote}
\file{quipu-db/c=GB/EDB}
\end{quote}
are loaded.

There are a few other important attributes your DSAs entry should have, the ``toucan''
entry in Figure~\ref{DSA:Toucan} on page~\pageref{DSA:Toucan}
gives examples of the other attributes.
They are briefly described below.
\begin{describe}

\item[\verb+description+:] A textual message describing the DSA, and the wildlife!


\item[\verb+quipuVersion+:] This should contain the version number of the
QUIPU software you are using. 
\index{quipuVersion attribute}
This can be found by using the DISH
command \verb+squid -version+.  You only have to set this once.  From
then on the DSA will manage it, updating the entry when required.


\item[\verb+manager+:] The DN of the manager of the DSA. This user will not
be blocked by
access control when modifying the local database over DAP.


\item[\verb+supportedApplicationContext+:] This should always have the value
\begin{quote}\begin{verbatim}
QuipuDSP & X500DSP & X500DAP & InternetDSP
\end{verbatim}\end{quote}
for this version of QUIPU.
It is used by remote DSAs to decide which protocols your
 DSA supports, and
thus how best to contact it.

\item[\verb+objectclass+:] This must contain ``\verb+quipuDSA+''.

\item[\verb+listenAddress+:] Sometimes, (particularly in the X.25 world, or
when using a transport service bridge), the
address a server should listen on, is not the same as the address
clients should call.  The presentationAddress attribute defines the 
address clients will always call.  If, and only if, this is not the
address the server should listen on, the address for the server should
be given in the listernAddress attribute.

\item[\verb+acl+:] A DSAs entry, like any other, can be protected by access
control lists\index{acl attribute}.  Care should be taken to make sure 
all remote DSAs can
see all the attributes they need.  The following ACL is recommended
for use in a DSAs entry.  The string ``\verb+MANAGER DN+'' should be
replaced with the distinguished name of the DSA manager as defined by
the ``manager'' attribute.
\begin{quote}\small\begin{verbatim}
acl= group # MANAGER DN # write # entry
acl= group # MANAGER DN # write # default
acl= others # read # entry
acl= others # read # default
acl= others # read # child
\end{verbatim}\end{quote}
A little explaination may help as this might seem somewhat cryptic.
The \verb+others # read # child+
line is to prevent anybody from adding children to the entry; it is
the only line referencing the children, so nobody is given write
access.
The userpassword attribute in the entry does not need protecting for
two reasons.  Firstly nothing can make use of it, self has
only got read access to the entry (as defined by the others clause).
Secondly, the password will probably be in a highly replicated EDB
file, so despite the pseudo crypting, is easily breakable.

\item[\verb+relayDSA+:]\index{DSA relay}\index{Relay DSA}\label{dsarelay} If 
your DSA is not connected to
one of the major networks (Internet, IPSS\ldots), it may from time to time
get references to a DSA that it cannot connect to directly.  For the
operation to succeed, your DSA will need to chain the operation to a
DSA that can progress the query.

This attribute is the DN of a DSA or DSAs that are connected to both your network and 
the major networks you are not connected to.
There needs to be an agreement between the managers of the two 
DSAs because the
relay DSA will be asked to perform operations on your behalf.
For example, if the DSA ``x'' has access to IPSS and Internet, but DSA ``y''
only has IPSS access, DSA ``y'' might add a relayDSA attribute to it own
entry, with the DN of DSA ``x'' as the value.
Then, when DSA ``y'' gets a reference to an Internet based DSA, it will chain
the operation the DSA ``x''.
Clearly, if every DSA chooses the same relay DSA, that DSA will soon
become overloaded and reject your association attempts with a ``busy''
error.  So some care is needed in choosing the ``right'' DSA (The
QUIPU team
recognise that there 
needs to be some form of ``relay authorisation'' and are
looking at possible solutions for future versions of QUIPU).
Section~\ref{dspchain} describes how to prevent your DSA (e.g., DSA ``x'') from
chaining operations on behalf of other DSAs.

\item[\verb+photo+:] A picture of the wildlife !

\end{describe}



Section~\ref{adding_data} describes how you can add data to your DSA by
extending the supplied textual database to include your 
own data
or by sending data to the directory via the DUA modify operations.
This may be done independently from connecting to the global directory.

\section {Tailoring}\label{dsa:tailor}

The previous sections have described how to start a basic DSA, and
connect a DUA to it.  Having done this, there are various
configuration options you can set, which are described in this section.

On startup the DSA first consults a runtime tailor file \index{quiputailor}
(the default DSA tailor file is called \file{quiputailor} in the ISODE
\verb"ETCDIR" directory, but can be
changed with a \verb"-t" parameter to \pgm{ros.quipu};
consult \man quipu(8c) for details),
which indicates such things as:

\begin{itemize}
\item
name of the DSA.
\item
location of the database.
\item
location and level of the logs that the DSA will produce.
\item
location of any other DSAs for initial bootstrap
\end{itemize}

At startup, the \man isotailor(5n) file is also 
consulted to configure the system-wide ISODE parameters.

The format is identical to the DUA \file{tailor} file
described in Section~\ref{dua:tailor}, with the addition of the
following options:
\begin{describe}
\item [\verb"mydsaname":]
The distinguished name of the DSA.
For example,
\begin{quote}\small\begin{verbatim}
mydsaname   cn=Axolotl
\end{verbatim}\end{quote}
declares this DSA to have the common name of \verb"Axolotl".
Quotes will be needed in the name contains a space characer, for example
\begin{quote}\small\begin{verbatim}
mydsaname   "c=GB@cn=Long Moustached Owl"
\end{verbatim}\end{quote}

\item [\verb"parent":]
This entry consists of a name/address pair of a parent DSA.
The DSA referenced needs to hold a master or slave copy of an EDB
higher up in the DIT than the highest locally held EDB.
For example,
\begin{quote}\small\begin{verbatim}
parent C=GB@CN=vicuna Internet=vs1.ucl.cs.ac.uk+50987
\end{verbatim}\end{quote}
\index{presentation addresses}
declares the parent DSA to be C=GB@CN=vicuna at the
indicated address.
If more than one \verb"parent" tailor entries are found, the DSA
will chose which DSA to contact.  
The first DSA in the list is taken as the master reference, with
the subsequent entries as slave references.

If your DSA holds a copy of the ROOT EDB file, this parameter should have
just one value which is a reference to the DSA holding the master copy
of the root EDB file (currently cn=Giant Tortoise).

\item [\verb"stats":]
The value has the same format as the \verb"dsaplog" entry described in
Section~\ref{dua:tailor}, and is used to control the level of
DSA statistical logging.

\item [\verb"treedir":]
Defines the directory in which the textual database is stored.
For example,
\begin{quote}\small\begin{verbatim}
treedir     /usr/etc/quipu-db/
\end{verbatim}\end{quote}
declares the directory \file{/usr/etc/quipu-db/} to contain the local
part of the directory information tree.
Be sure to remember the trailing slash ``\verb+/+''.

\item [\verb"shadow":]
When searching, often a large part of the
time is involved with chaining off to other DSAs to search aliases.
To enhance performance it
is sometimes useful to have a cached copy of the alias locally, this
tailor variable allows such attributes to be ``spot shadowed''.
The value of this variable is an attribute type.  
The attribute value associated with
this type should have DN syntax.  When the database has been loaded,
if any instance of this attribute references a DN that is not held
locally your DSA will ``spot shadow''\index{spot shadow} that entry.
That is from time
to time it will read the entry from a remote DSA, and cache the
result.

\item [\verb"optimize\_attr":]
If TURBO\_INDEX has been defined, this option
specifies the attribute types to index.  Each attribute type must
be on a separate line.  For example the lines:
\begin{quote}\begin{verbatim}
optimize_attr commonName
optimize_attr surname
\end{verbatim}\end{quote}
would arrange  to index both the commonName and surname attributes.
Only string attributes are allowed.
{\em Note}: this option must come before any
\verb"index_subtree" or
\verb"index_siblings" options.

\item [\verb"index\_subtree":]
If TURBO\_INDEX has been defined, this option
specifies the distinguished name of a subtree to index for
subtree searches.  Multiple
subtrees may be specified by multiple lines.  For example, the lines
\begin{quote}\begin{verbatim}
index_subtree "c=US@o=your org"
index_subtree "c=US@o=your org@ou=really big OU"
\end{verbatim}\end{quote}
arrange to have two indexes built, one for each subtree specified.  Be
aware that the indexes take up extra space in core, so care should be
taken to index things sparingly.

\item [\verb"index\_siblings":]
If TURBO\_INDEX has been defined, this option
specifies the distinguished name of the parent of a group of
siblings to index.  For example, the line
\begin{quote}\begin{verbatim}
index_siblings "c=US@o=your org@ou=really big OU"
\end{verbatim}\end{quote}
arranges to have an index built for one-level searches directly below
the specified entry.  Multiple indexes may be specified by multiple
lines.  Be aware that the indexes take up extra space in core, so care
should be taken to index things sparingly.

\item [\verb"optimized\_only":]
If TURBO\_INDEX has been defined, the value {\bf on}
tells the DSA to refuse any searches that do not consist entirely of 
optimized attributes and filters.  Any such non-indexed queries
will be rejected with an ``unwilling to perform'' service error.

\item [\verb"update":]
The value {\bf on} tells the DSA to update slave and cached EDB files 
when it starts up.
See Section~\ref{slave_update} for further details; by default 
this parameter is {\bf off}.

\item [\verb"searchlevel":] Defines the level below which users will be able
to search the DIT, for example, default 2 (e.g., below organisations).
If they try to search from higher up, a n``unwilling to perform'' service
error will result.

\item [\verb"lastmodified":] If the value is {\bf off}, 
the attributes \verb+lastmodifiedby+ and  
\verb+lastmodifiedtime+
will not be added by the DSA when an entry is altered.

\item[\verb"readonly":] Bring the DSA up in read only mode, that is,
prevent user modification.  Slave updates are still allowed.

\item [\verb"dspchaining":]
\label{dspchain}\index{Chaining DSP operations}
The value {\bf on} tells the DSA it is allowed to chain DSP 
requests to other DSAs 
if necessary.
The default mode of operation is to return a DSA referral, unless a
chain is needed due to a disjoint network.
The full set of issues deciding whether to use chaining or referrals
is discussed in the QUIPU design document (\cite{QUIPU.Design}), and 
\cite{QUIPU.Navigate}.

\item [\verb"adminsize":] The administrative size limit for use on search
and list operations.

\item[\verb"admintime":] The maximum time allowed to spend on a user query.

\item[\verb"cachetime":] The length of time to keep / use ``cached''
information.

\item[\verb"conntime":] The length of time to hold a unused connection open.

\item[\verb"nsaptime":] The length of time to wait before deciding a
connection cannot be established for a given NSAP.

\item[\verb"retrytime":] The length of time before deciding it is worth
attempting to connect to a DSA that could not be contacted earlier.

\item[\verb"slavetime":] The length of time between attempting to update
slave EDB files.

\item[\verb"preferdsa":] When a DSA has creates a reference to other DSAs it
tries to discriminate, and choose the ``best'' DSA to contact.
This variable gives you a handle on controlling the choice. For example the
lines
\begin{quote}\begin{verbatim}
preferDSA c=GB@cn=Vicuna
preferDSA "cn=Giant Tortoise"
\end{verbatim}\end{quote}
tell the DSA to use either c=GB@cn=Vicuna or cn=Giant Tortoise in
preference to any other DSA.  Furthermore, use the DSA c=GB@cn=Vicuna
rather than cn=Giant Tortoise if possible.

\item[\verb"cainfo":] For authentication.

\item[\verb"secretkey":] For authentication.

\item[\verb"bindwindow":] For authentication.

\item [\verb"isode":]
The argument is an \verb"isodevariable" \verb"isodevalue" pair as would 
be found in \file{isotailor}.  This is used to ``override'' isotailor
settings.

\item [\verb"authentication":]
The argument describes the minimum level of authentication required to
bind to the DSA over DAP.  The value should be one of:
\begin{describe}
\item [\verb"none":]	Anonymous binds are accepted.
\item [\verb"dn":]	A DN must be supplied.
\item [\verb"simple":]	Simple authentication must be used.
\item [\verb"protected":]	Protected simple authentication must
be used.
\item [\verb"strong":]  Strong authentication must be used.
\end{describe}
\item [\verb"reject\_prefix":]	Do not allow DAP bind operations from
DNs with the specified prefix.
\item [\verb"reject\_length":]   Do not allow DAP bind operations with
DNs having less that $n$ components.
\item [\verb"relay\_for":]	Only DSP chain operations for the
specified DSAs.
\item [\verb"accept\_prefix":]	Allow DAP binds from the given prefix ONLY.
\item [\verb"getedb\_size":]	For a getEDB operation, an EDB is
split into slices of $n$ entries, default 10.  This parameter allows
you to change this number.
\item [\verb"edbtmp\_path":]	The directory used to store temporary
fragments of EDB during a getEDB operation --- default \file{/tmp}.

\end{describe}

\subsection {Tailoring a Running DSA}
\label{dsa:mgmt}

The tailoring described above is performed when the DSA is
booted.  It is sometimes useful to alter the tailor settings
after the DSA has started without having to bring
the DSA down and reboot.  This can be done via
the  special \pgm{dsacontrol} command in the DISH program
as described in Section~\ref{dua:mgmt} on page~\pageref{dua:mgmt}.

The DUA invokes a modify operation, with a
single special attribute \verb+dSAControl+ defined in \cite{QUIPU.Design}.
The DSA recognises the special attribute, and provided you are
bound as the manager of the DSA, passes the attribute value to the
appropriate routine.



\cite{QUIPU.Design} describes this process in full.


\section {Modifying your DSA's Entry}
\label{DSA:ownentry}

A DSA manages its own entry in the DIT.
Generally the master EDB in
which your DSAs entry resides is not held by your DSA.  
For security reasons, this means it is
difficult for a DSA to modify its own entry directly, so, for example it cannot
keep its version number \index{quipuVersion attribute} up to date.
However, a  DSA holding the
master EDB knows that any QUIPU DSA\footnote{Any 
QUIPU DSA for which the version number is 6.8 or greater}
wants to manage its own entry.
To allow this to happen, the DSA holding the master EDB 
``spot shadows''\index{spot shadow}
the remote DSA entry.  That is, from
time to time it connects to the DSA in question, reads its entry,
and writes the result back into the master EDB file.  
So modify
operations on the DSA's entry can now take place in your local DSA.  This 
has the advantage that attributes such as the version number are
kept up to date.

To perform this shadowing, the DSA has to read its own entry across an
un-authenticated DSP link; thus it cannot read any attributes that
are protected by ACLs.  
So all important attributes in the DSAs entry {\em must} be publicly
readable (this includes the unused userPassword attribute).
If they are not readable the shadowing operation will fail.

To keep the database consistent with the master EDB, when a modify 
takes place, your DSA
cannot rewrite the information back to its slave copy EDB.
So it writes the entry in a file called
\file{DSA.real}\index{DSA.real}
in the top
level of your database.  If there is an inconsistency found, it is
this file that is ``trusted''.

There is also a file called \file{DSA.pseudo}\index{DSA.pseudo}, which
contains some attributes managed by the DSA, that it does not need to
make public (such as which cached EDB files it holds).
You should {\em never} need to edit this file.

There is a problem when modifying the presentation address
\index{presentation address --- changing} of a DSA. 
You must make
sure the DSA with the master EDB reads the new address, {\em before} you
move the DSA.  If not, it will always attempt to connect to the wrong
address to try to shadow the entry, and never find the new address.
(Alternatively you could use a ts\_bridge to make it look as if the 
old address still works.)
You can check the master EDB has updated by making a DAP 
bind to the DSA managing the EDB
and reading the address it has got.
If it is out of date, you should wait for it to be updated; this
usually takes place every six hours.
To summarise, the sequence of events is
\begin{enumerate}
\item Change the address attribute using a DAP modify operation.
\item Wait for the DSA with the master EDB to read the entry.
\item Restart the DSA on the new address.
\end{enumerate}


\section {Connection to Other DSAs}

\label{dsa:connect}

All QUIPU DSAs should be connected. This section describes how you should 
configure your DSA to connect to other DSAs in order
to read the data not held locally.
Section \ref{dsa:global} describes how to make sure that other DSAs 
can access your DSA.

A dynamic approach is used to bootstrap a new DSA.
The ``global master'' DSA is administered at 
the University of London Computer Center\footnote{The DSA has moved from
University College London.} (ULCC),
and other DSAs should be configured in relation to this one.

Every DSA needs to know the distinguished name and
presentation address of one
or more DSAs nearer to, or actually holding, the root EDB.  This parameter
is supplied in the \file{tailor} file (see Section~\ref{dsa:tailor})
under the name \verb"parent".
This information is used by the QUIPU DSA when it cannot find a DSA's entry
in the local database, or can find no pointers in the local database to the
whereabouts of the data.
If you hold a slave copy of the root EDB, then your 
parent should be a DSA that is ``closer'' to the master copy of the root
EDB, and probably the master itself.

The example databases are set up with default parents (defined in the file
\file{quipu-db/organisation/quiputailor}).
To see if your DSA can contact the ``parent'', connect to
the local test (root holding) DSA using DISH, and type

\begin{quote}\begin{verbatim}
list @
\end{verbatim}\end{quote}
This will list the locally held root. Now try

\begin{quote}\begin{verbatim}
list @ -dontusecopy
\end{verbatim}\end{quote}
This will try to connect to the parent DSA.

Care should be taken in choosing the parent
DSA. If all DSAs have the same parent
DSA then that DSA may become overloaded.
Typically each site will have several DSAs. One of these should
be the default parent for all the other DSAs, with only one DSA having a 
default parent outside the site.


When ``walking down'' the DIT, QUIPU needs access to information to tell it
on which DSA the next level of the DIT is stored.
To do this each nonleaf entry {\em must} have a \verb+masterDSA+\index{masterDSA attribute}
or \verb+slaveDSA+
attribute.  The value of the attribute is the distinguished name of the DSA
holding the next level of the DIT (it may be your own DSA name, if you hold
the next level as well).  If there is no such attribute, QUIPU assumes the
entry is a leaf.
If the information required is not held in the local DSA then QUIPU
chains the request to the named DSA or returns a referral 
--- depending upon the mode of operation.
(The named DSAs entry is read to establish the
address. This may mean a connection to another DSA 
in order to read the entry.)

Once you have started your DSA, you should try to connect to other DSAs 
using the DISH program (this connects to the local DSA, which will
chain requests to other DSAs).  Try listing the children of
organisations other than yourself, to find which organisations are connected,
try listing your countrys' children.

\subsection {Connection to the Global Directory}\label{dsa:global}


To enable the global directory to connect to you, you must contact the
manager of the DSA immediately above the node you want to added under.
Supply them with the entry for your DSA, and the top level 
entry of the subtree
you want to hold.
For example, to add the organisation o=foobar below c=US,
contact the manager of the DSA holding c=US as a master, giving them the
entry for c=US@o=foobar and c=US@cn=foobar\_dsa, assuming
c=US@cn=foobar\_dsa is the name of your DSA (see Section~\ref{dsa:naming}
on naming a DSA).

By convention,
to find out who administers the master EDB for a particular node,
run the DISH program and retrieve the \verb"manager" attribute
from the entry for the DSA holding that node.

You can then find a mail address by looking that person up in the directory.

The following example shows you how to get the mail address of the
manager of the c=GB subtree. 

\begin{footnotesize}\begin{quote}\begin{verbatim}
% dish
Welcome to Dish (DIrectory SHell)
Dish -> showentry @c=GB -type masterDSA
masterDSA 	- cn = Giant Tortoise
Dish -> showentry "@cn=Giant Tortoise" -type manager -edb 
manager= c=GB@o=X-Tel Services Ltd@cn=Camayoc
manager= o=Cosine@ou=Paradise@cn=Camayoc
Dish -> moveto "@c=GB@o=X-Tel Services Ltd@cn=Camayoc"
Dish -> showentry -edb
cn= Camayoc
roleOccupant= c=GB@o=X-Tel Services Ltd@cn=Colin Robbins
objectClass= organizationalRole & quipuObject
Dish -> moveto "@c=GB@o=X-Tel Services Ltd@cn=Colin Robbins"
Dish -> showentry -type rfc822mailBox
rfc822Mailbox         - C.Robbins@xtel.co.uk
Dish -> quit
%
\end{verbatim}\end{quote}\end{footnotesize}

If you want to add data to the root node of the ``global tree''.
The new entries should be sent to {\em QUIPU-support} at the address given in
Section~\ref{quipu:support}.

When you think you are connected to the ``global DIT'', 
you should test that you are.  To do this use DISH to connect 
to a DSA higher than you in the DIT (using the \verb"-call" flag as described
in Section~\ref{dua:tailor}); then see if you can navigate to your
own part of the DIT, and look at your data: if you can,  
then the DSA is connected.

\section {Connecting to a Non-QUIPU DSA}
\label{DSA:nonquipu}

QUIPU has the concept of mastering all entries in one EDB file built
into its design.  Other implementations may not share this model.
The entry represented by the non-QUIPU DSA can be 
``spot shadowed''.\index{spot shadow}
For example, if you wanted to add ``o=foobar'' to the DIT and this was
mastered by a non-QUIPU DSA, you would add an entry of the form
\begin{quote}\small\begin{verbatim}
o=foobar
objectClass= ExternalNonLeafObject & organization
subordinateReference= cn=The DSA # Internet=123.456.1.2+17003
\end{verbatim}\end{quote}
where cn=The DSA is the DN of the DSA, and
Internet=\ldots is the presentation address of the DSA.
Instead of a subordinate reference, a cross-reference or  nonspecific
subordinate reference can be given using the \verb+crossReference+
or \verb+nonSpecificSubordinateReference+ attributes respectively.

Although this is entry is in a master EDB file, QUIPU recognises that
is may not actually hold the authorative master, and contacts 
the referenced DSA as required.

\section {Adding More Data}
\label{adding_data}

Having got a DSA started, and connected to the global DIT, you should start
to add lots of data.
There are many sources of such data, and with just a relatively small amount
of effort this data can be added to the directory.

There are two ways such data can be added.
First of all you can use one of the DUA programs to bind to the DSA as the
DSA manager, and send data to the directory via the {\em add} and {\em modify} 
operations (see Sections~\ref{dish:add} and \ref{dish:modify}).
This is probably the best way to add relatively small amounts of data
or make minor changes to the data.

To add a large amount of data (i.e., during the initial creation of a 
large database), it is probably easiest to write a shell script using \unix/
tools such as \verb"awk" and \verb"grep" to create the EDB files directly.

When adding data for users it is advisable to allocate a 
userPassword\index{user password attribute}, and a suitable
ACL to protect
this password.
Chapter~\ref{Security} describes how to do this.

\subsection {More on Object Classes}
The only object class discussed so far is the quipuDSA object class.
When you start to add data, you will probably want to add information about
people, sub-divisions of your organisation, and other application entities.
This section introduces some of the more important object classes, and the
attributes they may contain.
In many cases, only the attribute type is specified, for details of typical
values and the value syntax you should read Chapter~\ref{syntaxes}.

As already described, every entry in the DIT must belong to the object
class \verb+top+, which means every entry must have an \verb+objectClass+
attribute.
Also, every entry in the QUIPU DIT should belong to either the
\verb+quipuNonLeafObject+ or \verb+quipuObject+.

The following part of this chapter describes some of the basic
object classes and the attributes implied.

\subsubsection{Person}
This is a base object class used to represent a person.

There are two mandatory attributes:
\begin{describe}
\item[1. \verb+commonName+:]
			which gives a (potentially ambiguous) name for
			the person.
			The value of this attribute is a string usually
			containing the person's first and last names; e.g.,
\small\begin{quote}\begin{verbatim}
Marshall Rose
\end{verbatim}\end{quote}
			This attribute is usually multivalued, containing
			variations on the first, middle, and last names; e.g.,
\small\begin{quote}\begin{verbatim}
Colin Robbins
Colin John Robbins
Colin J. Robbins
\end{verbatim}\end{quote}
			Generally this attribute will supply the
			distinguished attribute of the entry.
\item[2. \verb+surName+:]
			which gives the person's last name.
\end{describe}

The optional attributes are:
\begin{quote}\small\begin{verbatim}
userPassword                   seeAlso
telephoneNumber                description
\end{verbatim}\end{quote}

\subsubsection{OrganizationalPerson}
This is a subclass of the \verb+person+ person object class 
and introduces the following optional
attributes:
\begin{quote}\small\begin{verbatim}
preferredDeliveryMethod        destinationIndicator
registeredAddress              internationaliSDNNumber
x121Address                    facsimileTelephoneNumber
teletexTerminalIdentifier      telexNumber
physicalDeliveryOfficeName     postOfficeBox
postalCode                     postalAddress
title                          organizationalUnitName
streetAddress                  stateOrProvinceName
locality
\end{verbatim}\end{quote}

\subsubsection{ThornPerson}
Like \verb+OrganizationalPerson+, this is also a subclass of \verb+person+ 
and introduces the following optional
attributes:
\begin{quote}\small\begin{verbatim}
homePostalAddress              lastModifiedBy
lastModifiedTime               secretary
homePhone                      userClass
photo                          roomNumber
favouriteDrink                 info
rfc822Mailbox                  textEncodedORaddress
userid
\end{verbatim}\end{quote}

Two example \verb+ThornPerson+ entries are given in Figure~\ref{example_edb} on
page~\pageref{example_edb}.

\subsubsection{OrganizationalRole}
Entries of this class are used to represent a position or role within an
organisation.  

There is one mandatory attribute:
\begin{describe}
\item[\verb+commonName+:]	gives the name of the role.
			The value of this attribute is a string; e.g.:
\begin{quote}\small\begin{verbatim}
PostMaster
\end{verbatim}\end{quote}
\end{describe}

There are many optional attributes including:
\begin{describe}
\item[\verb+roleOccupant+:]    the distinguished name of the person 
			who fulfills the role, e.g.:
\begin{quote}\small\begin{verbatim}
c=US@o=X-Tel Services Ltd@cn=Peter Cowen
\end{verbatim}\end{quote}
\end{describe}

The other optional attributes are:
\begin{quote}\small\begin{verbatim}
seeAlso                        preferredDeliveryMethod
destinationIndicator           registeredAddress
internationaliSDNNumber        x121Address
facsimileTelephoneNumber       teletexTerminalIdentifier
telexNumber                    telephoneNumber
locality                       postOfficeBox
postalCode                     postalAddress
description                    organizationalUnitName
streetAddress                  stateOrProvinceName
physicalDeliveryOfficeName
\end{verbatim}\end{quote}


\subsubsection{Alias}
Objects of this class represent an alias to some other entry in the
DIT.  It is generally used when an entity belongs in one or more subtrees of
the DIT, and is used to ``point'' one entry at the other.

There are two mandatory attributes:
\begin{describe}
\item[\verb+commonName+:]	gives the name of the alias.
			The value of this attribute is a string; e.g.:
\begin{quote}\small\begin{verbatim}
Colin Robbins
\end{verbatim}\end{quote}

\item[\verb+aliasedObjectName+:] a pointer to another object in the directory;
			e.g.:
\begin{quote}\small\begin{verbatim}
c=GB@o=X-Tel Services Ltd@cn=Colin Robbins
\end{verbatim}\end{quote}
\end{describe}
There are no optional attributes for this object class.
An example of an Alias entry is given below:
\begin{quote}\small\begin{verbatim}
cn= QUIPU-support
aliasedObjectName= c=GB@o=University College London
		   @ou=Computer Science@cn=Incads
objectClass= quipuObject & alias & top
\end{verbatim}\end{quote}

\subsubsection{OrganizationalUnit}
The \verb+OrganisationalUnit+ object class is used to represent a unit within
organisation.
There is one mandatory attribute
\begin{describe}
\item[\verb+organizationalUnitName+:]
			which gives the name of the organisational unit.
			The value of this attribute is a string; e.g.:
\begin{quote}\small\begin{verbatim}
Research and Development
\end{verbatim}\end{quote}
\end{describe}

The optional attributes are:
\begin{quote}\small\begin{verbatim}
userPassword                   seeAlso
preferredDeliveryMethod        destinationIndicator
registeredAddress              internationaliSDNNumber
x121Address                    facsimileTelephoneNumber
teletexTerminalIdentifier      telexNumber
telephoneNumber                physicalDeliveryOfficeName
postOfficeBox                  postalCode
postalAddress                  businessCategory
searchGuide                    description
streetAddress                  stateOrProvinceName
locality
\end{verbatim}\end{quote}

\subsubsection{Organisation}
Objects of this class represent a top-level organisational entity,
such as a corporation, university, government entity, and so on.

There is one mandatory attribute:
\begin{describe}
\item[\verb+organisationName+:]
			which gives the name of the organisation.
			The value of this attribute is a string; e.g.:
\begin{quote}\small\begin{verbatim}
Performance Systems International
\end{verbatim}\end{quote}
\end{describe}

The optional attributes are:
\begin{quote}\small\begin{verbatim}
userPassword                   seeAlso
preferredDeliveryMethod        destinationIndicator
registeredAddress              internationaliSDNNumber
x121Address                    facsimileTelephoneNumber
teletexTerminalIdentifier      telexNumber
telephoneNumber                physicalDeliveryOfficeName
postOfficeBox                  postalCode
postalAddress                  businessCategory
searchGuide                    description
streetAddress                  stateOrProvinceName
locality
\end{verbatim}\end{quote}

\subsubsection	{domainRelatedObject}
If an object has some relationship to the Internet Domain Name System (DNS),
then this can be represented in the DIT using this object class.

This class has one mandatory attribute:
\begin{describe}
\item[\verb+associatedDomain+:]
			identifies the domain which corresponds to this object.
			The value is a domain string; e.g.:
\begin{quote}\small\begin{verbatim}
psi.com
xtel.co.uk
\end{verbatim}\end{quote}
\end{describe}

\subsection {Schemas}
\label{quipu:schema}

Directories should provide a very flexible tool
which enable any information to be stored.  There is a danger that schemas,
as specified in the OSI directory, will lead to Procrustean directory implementations
which impose unreasonable restrictions.  The QUIPU directory does not, per
se, place restrictions on what can be placed in a DSA.
It does, however, give control so that managers may control what is stored in the
directory.

The first aspect of structure is with respect to attributes which may be
present in an entry.  A QUIPU DSA will allow an entry to belong to one or
more object classes which are known to the DSA (stored in a table).  An
entry will typically have a small number of object classes (e.g., TOP
(implicit) + Person + Organisational Person + QUIPU Object).  The DSA will
maintain a table of mandatory and optional attributes for each object class
supported.  This will follow the guidelines of the standard or
specification identifying the object class in question.  From this
information, the DSA can determine the permitted and mandatory attributes for a
given entry by calculating the union of all the object classes of that
entry.  Free extension (i.e., the ability to store any attribute) was
rejected, as there does not appear to be a reasonable mechanism to manage
this.  However, it is straightforward for managers to create new object
classes as desired.

\tagrind[htb]{schema}{Schema Definition}{schema-py}

It is important 
to allow management control of what is permitted at a given level.
Therefore a \verb+treeStructure+
attribute may be created.  
This attribute is defined in Figure~\ref{schema-py}, with the string syntax
discussed in Section~\ref{tree_struct}.
This specifies for the level below,
what types of object are permitted.  
Each attribute value identifies a class of object which can exist at the level
below, and 
defines a set of mandatory and optional object classes.
This can be considered as defining a (private) object class implied by the
combination of these classes.

For each type of object, the attribute types permitted in the RDN are also
listed.  This is not checked in the current version of QUIPU.
The directory knows about the treeStructure attribute and will
ensure consistency.  
When creating an entry, the DSA must check that it conforms to the
treeStructure attribute of the parent entry.
When removing information from a treeStructure attribute, the
directory will  check that all of the children conform to the
modified attribute.

\subsection {Photograph Attributes}
\label{dua:photo}

The data that a DSA can hold does not have to be limited to data that can be
represented by printable strings.
Any attribute a QUIPU DSA does not know a string syntax for, will be held as
a block of ASN.1.
One such attribute is the \verb+photo+ attribute.
Photographs of people and objects are stored in the directory as a
g3fax-encoded block of ASN.1, and are best stored as ``file'' attributes
described in the next section.
There is an example of a photograph attribute in the example database.
This can be looked at by connecting to the directory with DISH
and looking at the entry ``Steve Kille, CS, UCL, GB''.


Unless you you have compiled and installed the ``photo'' code as 
described in Section~\ref{quipu:install} you will see the message
\begin{quote}\begin{verbatim}
"No display process defined"
\end{verbatim}\end{quote}

Having compiled the ``photo'' code, you will need to add a line such
as
\begin{quote}\small\begin{verbatim}
photo xterm Xphoto
\end{verbatim}\end{quote}
to your \file{dsaptailor} file (see Section~\ref{dua:tailor})
This says that if the user in logged on to a terminal
of the type \verb"xterm" then use the process \file{g3fax/Xphoto}
in the ISODE \verb"ETCDIR" directory to display the photograph.

There are display routines provided for the \xwindows/\index{X Windows},
SunView, 
Tektronix~4014, and ``dumb'' terminals.

\subsubsection{Generating Photos}

The photograph files used by QUIPU need to be two-dimensionally
g3fax-encoded.   QUIPU contains some tools to help you get your files into
the required format.  These are found in the \file{others/quipu/photo}
directory of the ISODE source tree.

Getting pictures onto your machine is a local problem that will need
discussing with your system manager.  You may have them in a different
format from that required by QUIPU.

If they are in SunView {\em pixrect} format, the tool \verb+pr2pe+ can be used
to convert them.
For other formats, there are two utilities, \verb+pbmtofax+ and \verb+faxtopbm+,
which can convert from the ``PBM''\index{PBM} format to two-dimension
fax, and vice versa.
The PBM format is that defined and used by 
Jef~Poskanzer's\index{Poskanzer, Jef} ``pbmplus package''.  This package
can convert a large range of bitmap formats into the PBM 
format\footnote{You should not confuse \verb+pbmtofax+ with the \verb+pbmtog3+
tools which are part of the pbmplus package; this uses a
one-dimensional encoding scheme}.
You will need to obtain this package to compile the \verb+pbmtofax+ 
and \verb+faxtopbm+ utilities.

{\em Note}: with this version of QUIPU, the \verb+pbmtofax+ tool should always be
used with a \verb+-old+ flag to ensure user at remote sites with 
older code can still decode and display your photograph files.

\subsection {File Attributes}
\label{file_attr}
Attributes values are generally stored in the EDB file, and loaded 
into memory when the DSA starts.
For some large attributes such a photos, this is not a sensible approach, so the
concept of a ``file'' attribute is introduced.

If an attribute value is prefixed by \verb"FILE", then the value is assumed
to be stored on disk.
For example, if the following were found in an EDB file:
\begin{quote}\small\begin{verbatim}
photo={FILE}/usr/local/pictures/steve
\end{verbatim}\end{quote}
then the value for the photo attribute would be read from the file
\begin{quote}
\verb"/usr/local/pictures/steve".
\end{quote}

The syntax of the data in the file is expected to be the same as
the string syntax, except in the case of ASN.1 and photo attributes which
are expected to be in raw ASN.1.

If there is not a file name supplied, then a default name is allocated.
The default file name is the RDN of the entry the attribute belongs to
followed by a dot ``.'', followed by the attributeType. This file is expected
to be in the same directory as the EDB file for the entry.
For example the default file for photo attribute, representing the entry
for 
``c= GB @ o= University College London @ 
ou= Computer Science @ cn= Colin Robbins''
 would be:
\begin{quote}\small\begin{verbatim}
cn=Colin Robbins.photo
\end{verbatim}\end{quote}
in the directory
\begin{quote}\small\begin{verbatim}
quipu-db/c=GB/o=University College London/ou=Computer Science
\end{verbatim}\end{quote}
or equivalent mapped file as described in Section~\ref{EDB:mapped}.

The process defined so far allows for attributes stored in an EDB in file
format to be read by a DSA.
Writing the attributes back to files if modified/added by a DUA requires
a little more.
The DSA needs to know which attributes should be stored on disk.
This information is supplied in the attribute ``oidtables'' which are 
defined in Section~\ref{oidtables}.
For example if the following entry was found in the file
\file{oidtable.at} in the ISODE \verb"ETCDIR" directory
then a \verb+photo+ attribute would always be written
back to disk.
\begin{quote}\small\begin{verbatim}
photo:  thornAttribute.7: photo : file
\end{verbatim}\end{quote}

\subsection {Attribute Inheritance}
\label{attr_inherit}
\index{inherited attribute}\index{attribute inheritance}

Attribute inheritance is a mechanism whereby an entry can get default
attribute values from its parent entry.
This can make management of common attributes much easier.
Inheritance can be used to make the in-core database significantly
smaller, which as a side effect should make the DSA run faster.    
For example, entries of the object class \verb+person+ for a particular
organisation might all have the same postal address attribute.
Using inheritance this attribute can be placed in the organisation entry
and inherited down to all \verb+person+ entries.

\tagrind[htb]{inherit}{Attribute Inheritance}{inherit-py}

The information is placed in the entry above using the \verb+inheritedAttribute+
attribute type, defined in Figure~\ref{inherit-py}. The QUIPU string syntax
to represent this is:
\begin{quote}\begin{verbatim}
<InheritedAttributeValue> ::= [ <oid> "$" ] 
                      [ "always" <InheritedList> ]
                      [ "default" <InheritedList> ]
<InheritedList> ::= "(" NEWLINE <AttributeSequence> 
                        NEWLINE ")"
\end{verbatim}\end{quote}

For example, if the following was an attribute of an entry
\begin{quote}\begin{verbatim}
inheritedAttribute = person $ always (
postalAddress= UCL $ Gower Street $ London $ WC1E 6BT
postalCode= WC1E 6BT
) default (
telephoneNumber= +44 71-387-7050
)
\end{verbatim}\end{quote}
then all entries of objectclass \verb+person+ positioned one level
below the entry would always inherit the attributes
\begin{quote}\begin{verbatim}
postalAddress= UCL $ Gower Street $ London $ WC1E 6BT
postalCode= WC1E 6BT
\end{verbatim}\end{quote}
and {\bf if} the entry does not contain a \verb+telephoneNumber+ attribute itself,
it would inherit
\begin{quote}\begin{verbatim}
telephoneNumber= +44 71-387-7050
\end{verbatim}\end{quote}

The attribute value in the inherited attribute can be left blank, for example
\begin{quote}\begin{verbatim}
inheritedAttribute = person $ always (
postalAddress
)
\end{verbatim}\end{quote}
in which case the value is taken from the same entry as the
inherited attribute itself.  This avoids duplicating attributes.

The OID representing the object class can only appear in one attribute value,
thus the following, which may seem similar to the above is {\em not}
allowed: the attributes need to be combined.
\begin{quote}\begin{verbatim}
inheritedAttribute = person $ always (
postalAddress= UCL $ Gower Street $ London $ WC1E 6BT
)
inheritedAttribute = person $ always (
postalCode= WC1E 6BT
)
inheritedAttribute = person $ default (
telephoneNumber= +44 71-387-7050
)
\end{verbatim}\end{quote}

If the OID representing the object class is omitted, then the values are
inherited to entries that themselves do not contain an object class
attribute. Thus the inherited attribute must define an object class to
inherit down (as schema checking still takes place).  For example
\begin{quote}\begin{verbatim}
inheritedAttribute = default (
ObjectClass= person & quipuobject
postalAddress= UCL $ Gower Street $ London $ WC1E 6BT
postalCode= WC1E 6BT
)
\end{verbatim}\end{quote}

Inheritance can be used for any attribute with the following exceptions:
\begin{itemize}

\item An entry cannot inherit its distinguished attributes.

\item An inherited \verb+userPassword+ attribute in not used to authenticate a user
at bind time.

\item The following system attributes 
\begin{quote}\small\begin{tabular}{ll}
objectClass &
ACL\\
masterDSA &
slaveDSA\\
aliasedObjectName &
edbInfo\\
presentationAddress &
relayDSA\\
treeStructure &
supportedApplicationContext\\
inheritedAttribute
\end{tabular}\end{quote}
can only be used in the ``default'' clause.  They can 
not appear in the ``always'' clause as there is the potential for
an inconsistency to occur.

\item You must hold at least one copy of the EDB file containing
the parent entry.

\end{itemize}

If an entry belongs to more than one object class, and more than one
of these classes is listed in the inherited attribute of the parent entry,
then only attributes from {\em one} of these will be inherited into the entry.
It is undefined which.

The inheritance described thus far only covers one level of the
DIT, whereas it is often useful to inherit attributes down the
whole subtree.  This can be achieved by inheriting the inheritance
attribute!
Consider the multivalued attribute
\begin{quote}\begin{verbatim}
iattr= organizationalUnit $ default (
iattr
) 
iattr= person $ default (
telephoneNumber
)
\end{verbatim}\end{quote}
Every \verb+person+ immediately below will inherit the telephone
number attribute.
Every \verb+organizationalUnit+ will inherit the same inheritance
attribute, thus any \verb+person+ entries below \verb+organizationalUnit+
entries will get the telephone number attribute as well.
by the same token, this works for multiple levels of
\verb+organizationalUnit+.


\section {How a DSA Starts}
\label{dsa:starting}

The section gives a brief description of how a QUIPU DSA starts.
This is intended to help people who have experienced problems in getting
their DSAs started.

First of all the tailor file \file{quiputailor} in the ISODE \verb"ETCDIR" directory
is read (unless you
specify a different tailor file using the \verb"-t" option to
\pgm{ros.quipu}).
This tells the DSA, amongst other things, its own name, and its parent
DSA(s).

The first thing a DSA need to do is find its own entry, thus it tries to
load the EDB that should contain it.
If it cannot find its own entry, then it will try connecting to the
parent DSA to read its own entry. If this fails the DSA will stop.

Having found its own entry, the DSA will check to see if it is a cached
entry read fron disk; if so, it will attempt to read a new version from the
relevant remote DSA. If successfu,l a new cache EDB file will be written.

Now the rest of the local DIT can be loaded.  Each EDB specified in the
\verb+edbInfo+ attribute \index{edbInfo attribute} is in turn loaded
from disk, followed by any cached EDB files.

As each EDB is loaded from disk, it is checked to make sure all the
attributes in the entry are allowed, as defined by the \verb+objectClass+
attribute, \index{objectClass attribute} and that the tree shape conforms to that defined by the
\verb+treeStruture+ attribute. \index{treeStructure attribute}
If any EDB fails these checks then the DSA will not start, and an appropriate
message will be logged.

Having loaded all its data, the DSA will start listening for DAP and DSP
associations.

\section {Adding More DSAs}

Many organisations will need to have more than one DSA to meet 
their requirements. This section suggests how you might arrange your
DSAs to get the most out of the system.

You will almost certainly need at least one DSA that holds a
copy of the root EDB, and your country EDB.
This DSA should also hold the master copy of your organisation's own EDB.
Subsequent DSAs that are set up should have this DSA defined 
as one of their parent DSAs.

For robustness, it is a good idea to have some other DSAs replicating 
these EDBs, so that if one local machine or DSA becomes unavailable, then
there is another copy elsewhere that can be used.
EDBs that contain addresses of other DSAs that you may want to contact
regularly should also be replicated, to prevent extra associations
having to be made just to read DSA addresses.
In short, if you know that data from a certain 
EDB is going to be accessed a lot, replicate it locally.
Replication does not have a high cost.

When setting up multiple DSAs be sure to name them as described in 
Section~\ref{dsa:naming};
ensure the DSA entries are sufficiently high in the DIT, so that other DSAs
can read the entry without having to contact the DSA concerned.

\section {Receiving EDB Updates}
\label{slave_update}
If your DSA holds a slave copy of one or more EDBs, then it can
automatically update these for you.
The current approach is very simple minded, but will be extended for future
versions.

There are two ways to ask a DSA to update its slave EDBs, either use DISH
program and issue a \verb+dsacontrol -slave+ command (see 
Section~\ref{dua:mgmt}), or set the \file"quiputailor" parameter
``update'' to {\bf on} (see Section~\ref{dsa:tailor}),
in which case the DSA will try to update its slaves
when it starts up.

To update the slaves, the DSA uses the \verb+edbinfo+ attribute that the DSAs
entry in the DIT {\em must} have.
This attribute specifies which EDBs you hold, and where to get updates of
the from.
{\em Note}: you do not necessarily have to get updates from a master EDB,  
a slave is acceptable.
In many cases this will be preferable to prevent the 
load on the master DSA being too high.
For example, a ``national'' EDB is likely to be highly replicated,
it would not be a good idea to have just one DSA
handling updates.  It would be better to have the load spread over several
DSAs.

The syntax of the attribute is \index{edbInfo attribute}
\begin{quote}\begin{verbatim}
edbinfo = EDB concerned # get from # send to
\end{verbatim}\end{quote}
It is a multivalued attribute. 

A few examples:
\begin{quote}\small\begin{verbatim}
# cn=Giant Tortoise # cn=Fruit Bat
c=US # # cn=Fruit Bat
c=US # # c=US@cn=Spectacled Bear
c=US@l=NY # cn=Fruit Bat #
\end{verbatim}\end{quote}
Note that there is no harm to using multiple \verb"eDBinfo" lines,
even if they refer to the same EDB.
These lines indicate that:
\begin{itemize}
\item	The root EDB is read from the \verb"cn=Giant Tortoise" DSA,
	further the \verb"cn=Fruit Bat" DSA is allowed to read the
	root EDB from this DSA.
	This is an important point --- A DSA has to explicitly say it will
	allow you to update from it, before it will send EDB files.
	Thus if you want to pull EDB files from a remote DSA, you will need
	to ask the manager to add your DSA to their DSAs ``send to'' field.

\item	The \verb"cn=Fruit Bat" DSA and the \verb"c=US@cn=Spectacled Bear" DSA
	are allowed to read the EDB for \verb"c=US". 
	Note that the second and third line could be combined as:
\begin{quote}\small\begin{verbatim}
c=US # # cn=Fruit Bat$c=US@cn=Spectacled Bear
\end{verbatim}\end{quote}
\item	The DSA-named \verb"cn=Fruit Bat" supplies the EDB for \verb"c=US@l=NY"
	to this DSA.
\end{itemize}

Some EDB files, such as the ROOT, and country level EDB files are highly
replicated, and having to list all the DSAs that are allowed to pull such EDBs
is too restrictive.
Thus the following syntax can be used as an abbreviation:
\begin{quote}\small\begin{verbatim}
c=US # # c=US
\end{verbatim}\end{quote}
This indicates that the c=US EDB file can be sent to DSA named at the c=US
level of the DIT, for example;
\begin{quote}\small\begin{verbatim}
c=US@cn=Wombat
\end{verbatim}\end{quote}
is allowed to pull the EDB, but the following are not:
\begin{quote}\small\begin{verbatim}
cn=Wombat
c=US@o=Foobar Inc.@cn=Wombat
c=GB@cn=Wombat
\end{verbatim}\end{quote}

Whether to update is decided upon by looking at the version string; if the
two EDB files have a different version string, then the EDB will be updated.

\section {Tables}
\label {oidtables}
Throughout QUIPU, all the Object Identifiers that need to be specified are
specified using a string representation, for example:

\[\begin{tabular}{|l|l|} \hline
String representation & Object Identifier  \\ \hline 
attributeType & 2.5.4 \\
surname & 2.5.4.4 \\
streetAddress & 2.5.4.9 \\
organizationName & attributeType.10 \\ \hline
\end{tabular}\]

A set of Object Identifier tables in the ISODE \verb"ETCDIR" directory are
used to 
provide this mapping (see also Section~\ref{isobjects} in \volone/).
In general the default table will be sufficient.
This section describes the tables for those who want to extend
the default definitions.

This basic format is used to build up the specification of general object
identifiers.  
This file is by default named \file{oidtable.gen}, and has simple
mappings from string to oid.
These strings can then be used in the definition of further oids (for example
the \verb+organizationName+ oid in the table above).

This simple table is extended to give table formats for attributes
This file is by default named \file{oidtable.at}.
Each entry in this table is assumed to represent an attribute, so
in addition to mapping strings onto oids, 
it also defines the syntax for that attribute:

\[\begin{tabular}{|l|l|l|} \hline
String representation & Object Identifier & Syntax\\ \hline
objectClass:&		attributeType.0:	&objectclass \\
aliasedObjectName:&     attributeType.1:	&dn\\
commonName,cn:&		attributeType.3:	&caseignorestring\\
searchGuide:&           attributeType.14:	&asn\\
x121Address:&           attributeType.24:	&numericstring\\
presentationAddress:& 	attributeType.29:	&presentationaddress\\
\hline
\end{tabular}\]

This says that the attribute named \verb+objectclass+ represents the oid
``attributeType.0''; using the expansion of ``attributeType'' (
as defined in the 
file \file{oidtable.gen}) thus gives the oid ``2.5.4.0''.
The syntax taken by the attribute value is \verb+objectclass+.

Similarly an \verb+aliasedObjectName+ has \verb+dn+ (DistinguishedName) syntax.
The recognised syntaxes are defined in the BNF in 
Appendix~\ref{bnf}.

After the ``syntax'', an extra optional parameter is allowed;
if this is the string ``\verb"file"'' the the relevant attribute is 
designated a file attribute; see Section~\ref{file_attr} for a
description of file attributes.


The file \file{oidtable.oc} defines the object classes\index{object
class} QUIPU knows about.
The file again maps strings onto oids.
Each string is assumed to represent an object class, and for each it
defines the hierarchy (which object classes it is a subclass of),
the mandatory attributes of the object class, and the optional attributes of
the object class.

For example the entry
\begin{quote}\small\begin{verbatim}
quipuObject:  	quipuObjectClass.2 : top : acl : \
                masterDSA, slaveDSA, treeStructure
\end{verbatim}\end{quote}
defines \verb+quipuObject+ to be an objectClass represented by the oid
``quipuObjectClass.2''.
It is a {\em subclass} of the object class \verb+top+.
An entry having this object class {\em must} have the attribute
\verb+acl+, and {\em may}
have the attributes \verb+masterDSA+, \verb+slaveDSA+ and \verb+treeStructure+
(Note: The similarity between this definition and the ASN.1 definition of
a \verb+quipuObject+ in Appendix \ref{naming}).

To allow for definitions of ``attribute sets'', there is a simple 
macro facility provided, for example, using the macro
\begin{quote}\small\begin{verbatim}
localeAttributeSet = localityName, streetAddress ...
\end{verbatim}\end{quote}
every occurrence of \verb"localeAttributeSet" will be replaced by
the right-hand side expression.


With the definition of the ``string'' form of the oid it is also possible
to specify {\em one} abbreviation for the name.
The standard tables use the following abbreviations:

\[\begin{tabular}{|l|l|}
\hline
c & countryName \\
o & organizationName \\
ou & organizationalUnitName \\
cn & commonName \\
co & friendlyCountryName \\
\hline
\end{tabular}\]

The abbreviated name may be used anywhere the full name would be allowed.


Every entry stored in the QUIPU DSA is checked against these tables to see if
the attributes supplied are allowed in the entry, to make sure the mandatory
attributes are present, and the attributes themselves have the correct syntax.

If you wish to add you own definitions to the tables you should add them to
the files 
\file{isode/dsap/oidtable.at.local},
\file{isode/dsap/oidtable.oc.local} and
\file{isode/dsap/oidtable.gen.local}
in that way, if a new set of tables is issued, your local entries will be
preserved.

\section{More Help Installing QUIPU}
Contact ``QUIPU-support''.
The mail address is given in Section~\ref{quipu:support}.

