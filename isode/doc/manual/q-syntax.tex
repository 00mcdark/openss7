% run this through LaTeX with the appropriate wrapper

\chapter {Attribute Syntaxes}
\label{syntaxes}

All attributes recognised by QUIPU have an associated ``string''
representation which is used to store the data in the DSA.
This chapter describes all the currently recognised syntaxes.

For most of the syntaxes, a BNF description is given, using the following
base description:

\begin{quote}\small\begin{verbatim}
<a> ::= any of the 52 upper and lower case IA5 letters
<d> ::= any IA5 digit 0-9
<k> ::= any of the 52 upper and lower case IA5 letters,
               IA5 digits, and "-" (hyphen)
<p> ::= any IA5 character in ASN.1 PrintableSting
<CRLF> ::= IA5 Newline
<letterstring> ::= <a> | <a> <letterstring>
<numericstring> ::= <d> | <d> <numericstring>
<keystring> ::= <k> | <k> <keystring>
<printablestring> ::= <p> | <p> <printablestring>
\end{verbatim}\end{quote}

The BNF description of attributes can be found in
Appendix~\ref{bnf}, but this does not repeat the BNF used for syntaxes
given below.

\section{Standard Syntaxes}

This section describes the Attribute Syntaxes defined by X.500.

There are many attributes that are represented by a sequence of
printable characters.  There are various ways in which the contents of the
string is restricted, each is represented by a different syntax, which are
described below.

\subsection{PrintableString}
\index{PrintableString}

\begin{center}\small
\begin{tabular}{|l|}\hline
Standard Attributes \\ \hline
	serialNumber\\
\hline
\end{tabular}
\begin{tabular}{|l|}\hline
QUIPU Attributes \\ \hline
	execVector\\
\hline
\end{tabular}
\end{center}


The printable string characters are:
\begin{quote}\begin{tabular}{l}
\verb"A" through \verb"Z"\\
\verb"a" through \verb"z"\\
\verb"0" through \verb"9"\\
\verb"'" \hspace{2ex}(apostrophe)\\
\verb"(" \hspace{2ex}(left parenthesis)\\
\verb")" \hspace{2ex}(right parenthesis)\\
\verb"+" \hspace{2ex}(plus-sign)\\
\verb"," \hspace{2ex}(comma)\\
\verb"-" \hspace{2ex}(hyphen)\\
\verb"." \hspace{2ex}(period)\\
\verb"/" \hspace{2ex}(solidus)\\
\verb":" \hspace{2ex}(colon)\\
\verb"?" \hspace{2ex}(question-mark)\\
space
\end{tabular}\end{quote}

The value can be any character listed above, with matching as for
CaseExactString.

When matching, multiple white spaces and tabs are treated as a single space
character.

Approximate matching is supported for this syntax.

The only standard attribute of this type 
is \verb+serialNumber+, this is used in the
\verb+device+ objectClass, and represents the serial number of the
device.
QUIPU defines an \verb+execVector+ attribute which is used by the
\pgm{iaed} program as the vector to pass to a program when starting it.

\subsection{CaseExactString}
\index{CaseExactString}\index{T.61 Strings}

No attributes currently use this syntax.
Matching is as for PrintableString.

\label{T61String}
The value for this syntax can be one of two types, either a printable
string, or a T.61 string represented as follows:
\begin{quote}\begin{verbatim}
<StringValue> ::= "{T.61}" <T61String> | <printablestring>
<T61String>   ::= Any character defined to be in 
                  T.61 String.
\end{verbatim}\end{quote}

If a T.61 string is used, where possible the character is displayed
using the equivalent ASCII character.
Characters in T.61 string that cannot be represented by an
equivalent ASCII character are quoted, using ``\verb+\xx+'' where
``\verb+xx+'' is the hexadecimal value of the character.

\begin{table}[tp]
\small
\[\begin{tabular}{|c|l|c|}
\hline 
Hex code & Description & Character \\
\hline 
c1 & grave accent      	& \`{o} \\
c2 & acute accent	& \'{o} \\
c3 & circumflex		& \^{o} \\
c4 & tilde		& \~{o} \\
c5 & macron		& \={o} \\
c7 & single dot		& \.{o} \\
ca & ring		& \aa \\
cb & cedilla		& \c{c} \\
cc & underscore		& \b{o} \\
cd & umlaut		& \"{o} \\
cf & caron		& \v{o} \\
a1 & inverted exclamation mark	 & !` \\
a3 & pound sign			 & \pounds \\
a4 & dollar sign		 & \$ \\
a6 & hash			 & \# \\
a7 & paragraph sign		 & \S \\
ab & left quotation mark	 & `` \\
b0 & ring			 & \aa \\
b1 & plus/minus			 & $\pm$ \\
b5 & greek letter mu		 & $\mu$\\
b6 & pilcrow			 & \P \\
b8 & division sign		 & $\div$ \\
bb & right quotation mark	 & '' \\
bc & quarter sign		 & $\frac{1}{4}$ \\
bd & half sign			 & $\frac{1}{2}$ \\
be & three-quarters sign	 & $\frac{3}{4}$ \\
bf & inverted question mark	 & ?` \\
e0 & greek letter omega		 & $\Omega$ \\
e1 & AE				 & \AE \\
e8 & L stroke			 & \L \\
e9 & O stroke			 & \O \\
ea & OE				 & \OE \\
\hline 
\end{tabular}
\]
\caption{T.61 Character Codes}
\label{T61fig}
\end{table}

However, if your terminal has the ability to display characters 
using an ISO 8859-1 font 
(for example some of the \xwindows/\index{X Windows} fonts)
then a large
number of the T.61 characters can be displayed (and so are not quoted in hex!).
You will need to tell the DUA that you have the ability to display the
ISO 8859-1 characters.  This can be achieved in one of three ways:
\begin{itemize}
\item In the system-wide dsaptailor file, you can add
\begin{quote}\begin{verbatim}
     ch_set ISO8859
\end{verbatim}\end{quote}
or
\item in your .quipurc file you can add
\begin{quote}\begin{verbatim}
     dsap: ch_set ISO8859
\end{verbatim}\end{quote}
However not all the DUA interfaces read this files (DISH does!) or
\item set the environment variable CH\_SET to ``ISO8859'', e.g., using ``csh''
\begin{quote}\begin{verbatim}
     setenv CH_SET ISO8859
\end{verbatim}\end{quote}
\end{itemize}

To enter T.61 characters is a little more tricky.
In T.61 
accented characters are represented by two octets, the first indicating the
accent and the second the base character to be accented. Note that some
combinations of accent and character do not have an equivalent in ISO 8859-1,
and hence cannot be displayed on an ISO 8859-1 device.
Table~\ref{T61fig} shows some of the characters that can be represented,
for
the accented characters only the first octect is shown, the second
octect can typically be any character, ``o'' is used in most of the
examples.

For instance,  to enter the T.61 string ``Gal\'{a}pagos Penguin,'' you
should use
\begin{quote}\begin{verbatim}
	{T.61}Gal\c2apagos Penguin
\end{verbatim}\end{quote}
The acute accent is represented by the ``\verb+\c2+'' and appears
over the following character --- ``\verb+a+'' in this case.

\subsection{CaseIgnoreString}
\index{CaseIgnoreString}
\begin{center}\small
\begin{tabular}{|l|}\hline
Standard Attributes \\ \hline
	knowledgeInformation\\
	commonName\\
	surname\\
	localityName\\
	stateOrProvinceName\\
	streetAddress\\
	organizationName\\
	organizationalUnitName\\
	title\\
	description\\
	businessCategory\\
	postalCode\\
	postOfficeBox\\
	physicalDeliveryOfficeName\\
	\hline
\end{tabular}
\begin{tabular}{|l|}\hline
COSINE/Internet Attributes \\ \hline
	userid\\
	textEncodedORaddress\\
	info\\
	favouriteDrink\\
	roomNumber\\
	userClass\\
	host\\
	documentIdentifier\\
	documentTitle\\
	documentVersion\\
	documentLocation\\
	durName\\
	wkdName\\
	protocolProfile\\
	objectID\\
	friendlyCountryName\\
	buildingName\\
\hline
\end{tabular}
\end{center}


The syntax is the same as CaseExactString, except that when matching 
``characters that differ only in their case are considered identical''.

Approximate matching is supported for this syntax.

The use of some of the attributes is now described, the more obvious ones
are omitted.

\begin{describe}
\item [\verb+knowledgeInformation+:] A description of knowledge mastered by a DSA.
\item [\verb+localityName+:] Used to identify the geographical area or locality in
which the object is physically located e.g., 
\small\begin{quote}\begin{verbatim}
London
\end{verbatim}\end{quote}
\item [\verb+stateOrProvinceName+:]
			describes the state in which
			the \verb"locality" is found; e.g., 
\small\begin{quote}\begin{verbatim}
New York
\end{verbatim}\end{quote}
\item [\verb+title+:]
An objects job title, e.g.,
\small\begin{quote}\begin{verbatim}
Technical Manager
\end{verbatim}\end{quote}

\item [\verb+businessCategory+:]
describes the business of the object, e.g.,
\small\begin{quote}\begin{verbatim}
networking
\end{verbatim}\end{quote}

This is used to find people sharing the same occupation.

\item [\verb+physicalDeliveryOfficeName+:]
			describes the geographical location
			of the physical delivery office which services the
			postal address of this object; e.g.,
\small\begin{quote}\begin{verbatim}
Troy
\end{verbatim}\end{quote}

\item [\verb+friendlyCountryName+:]
A ``nice'' name for countries, as opposed to the two letter codes enforced by
use of \verb+CountryName+, for example
\small\begin{quote}\begin{verbatim}
Great Britain
\end{verbatim}\end{quote}

\end{describe}



\subsection{CaseIgnoreList}
\index{CaseIgnoreList}

No standard attributes currently use this syntax.
 
The CaseIgnoreList syntax consists of a sequence of CaseIgnoreString
values as shown by the BNF below.
\begin{quote}\begin{verbatim}
list = <list_component> | <list_component> "$" <list>
list_component = [ "{T61}" ] <string>
\end{verbatim}\end{quote}
When the list is displayed to a user, the ``\verb+$+'' is replaced
with a newline.  Ordering is preserved.  For example, the entry in an
EDB file
\begin{quote}\begin{verbatim}
caseIgnoreAttribute= this is a $ multi line $\
	attribute definition
\end{verbatim}\end{quote}
would be shown to a user as
\begin{quote}\begin{verbatim}
caseIgnoreAttribute= this is a 
                     multi line
                     attribute definition
\end{verbatim}\end{quote}

QUIPU only allows equality matching for this syntax.

\subsection{CountryString}
\index{CountryString}
\begin{center}\small
\begin{tabular}{|l|}\hline
Standard Attributes \\ \hline
	countryName\\
\hline
\end{tabular}
\end{center}
This syntax is treated as a PrintableString, with matching rules as for
CaseIgnoreString, and the restriction that the string must 
be one of the codes defined by ISO~3166.




\subsection{IA5String}
\index{IA5String}\index{quipuVersion attribute}
\begin{center}\small
\begin{tabular}{|l|}\hline
QUIPU Attributes \\ \hline
	control\\
	quipuVersion\\
\hline
\end{tabular}
\begin{tabular}{|l|}\hline
COSINE/Internet Attributes \\ \hline
	aRecord\\
	mDRecord\\
	mXRecord\\
	nSRecord\\
	sOARecord\\
	cNAMERecord\\
\hline
\end{tabular}
\end{center}

This syntax is handled as PrintableString, except a wider range of
characters are recognised (i.e., any character in IA5 string, characters such
as NewLine are quoted using the same mechanism as described for T.61 string
in Section~\ref{T61String}), with matching rules as for CaseExactString.



\subsection{OctetString}
\index{OctetString}

No attributes use this syntax directly.

Characters that cannot be printed as an ASCII are represented using the
quoting mechanism described in Section~\ref{T61String}.

\subsection{NumericString}
\index{NumericString}
\begin{center}\small
\begin{tabular}{|l|}\hline
Standard Attributes \\ \hline
	x121Address\\
	internationaliSDNNumber\\
\hline
\end{tabular}
\end{center}
The value is simply a numeric string (digits 0 through 9 only)
\begin{quote}\begin{verbatim}
<NumericValue>	::= <numericstring>
\end{verbatim}\end{quote}

The two attributes are used thus:
\begin{describe}
\item[\verb+x121Address+:] defines the X.121 Address of an object as defined by the
CCITT Recommendation X.121.
\item[\verb+internationaliSDNNumber+:]
An International ISDN Number as defined by CCITT Recommendation E.164.
\end{describe}

\subsection{DestinationString}
\index{DestinationString}
\begin{center}\small
\begin{tabular}{|l|}\hline
Standard Attributes \\ \hline
	destinationIndicator\\
\hline
\end{tabular}
\end{center}
Behaves as a \verb+<printablestring>+, with CaseIgnoreString matching rules.

The only attribute of the type \verb+destinationIndicator+ is used to define
the addressee as required by the Public Telegram Service.

\subsection{TelephoneNumber}
\index{TelephoneNumber}
\begin{center}\small
\begin{tabular}{|l|}\hline
Standard Attributes \\ \hline
	telephoneNumber\\ 
\hline
\end{tabular}
\begin{tabular}{|l|}\hline
COSINE/Internet Attributes \\ \hline
	homePhone\\
\hline
\end{tabular}
\end{center}
\begin{center}\small\begin{tabular}{|l|}\hline
PSI Attributes \\ \hline
	mobileTelephoneNumber\\
	pagerTelephoneNumber\\
\hline
\end{tabular}
\end{center}

The value should be a string describing the phone number of the object
using the international notation; e.g.,
\begin{quote}\begin{verbatim}
+1 518-283-8860
\end{verbatim}\end{quote}
or
\begin{quote}\begin{verbatim}
+1 518-283-8860 x1234
\end{verbatim}\end{quote}
In general, the syntax is:
\begin{quote}\begin{verbatim}
"+" <country code> <national number> [ "x" <extension> ]
\end{verbatim}\end{quote}

Matching is as defined for CaseExactString, except that all space and ``--''
characters are skipped during the comparison.

\subsection{PostalAddress}
\index{PostalAddress}

\begin{center}\small
\begin{tabular}{|l|}\hline
Standard Attributes \\ \hline
	postalAddress\\
	registeredAddress\\
	homePostalAddress\\
\hline
\end{tabular}
\end{center}

\begin{quote}\begin{verbatim}
<PostalAddressValue> ::= <address_component> | 
                         <address_component> "$" <address>
<address_component> = <StringValue>
\end{verbatim}\end{quote}
For example \verb+UCL $ Gower Street $ London+.
You are limited to a maximum of 6 \verb+<address_component>+'s, each with a 
maximum of 30 characters.
(Note each \verb+<address_component>+ can include T.61 strings; see
Section~\ref{T61String} for details of this).

Note: substring matching is not supported by QUIPU for this syntax.

\subsection{DN}
\index{DN}\index{masterDSA attribute}
\begin{center}\small
\begin{tabular}{|l|}\hline
Standard Attributes \\ \hline
	aliasedObjectName\\
	member\\
	owner\\
	roleOccupant\\
	seeAlso\\
\hline
\end{tabular}
\begin{tabular}{|l|}\hline
QUIPU Attributes \\ \hline
	masterDSA\\
	slaveDSA\\
	relayDSA\\
\hline
\end{tabular}
\end{center}

\begin{center}\small\begin{tabular}{|l|}\hline
MHS Attributes \\ \hline
	mhsMessageStoreName\\
\hline
\end{tabular}
\begin{tabular}{|l|}\hline
COSINE/Internet Attributes \\ \hline
	manager\\ 
	documentAuthor\\
	secretary\\
	lastModifiedBy\\
	associatedName\\
	dITRedirect\\
\hline
\end{tabular}
\end{center}

\begin{quote}\begin{verbatim}
<DNValue> ::= <name>
\end{verbatim}\end{quote}
Distinguished names are discussed in Section~\ref{quipu_name}, the BNF is
repeated for completeness:
\begin{quote}\begin{verbatim}
<DNValue>	::= <rdn> | <rdn> "@" <DNValue>
<rdn>		::= <attribute> | <attribute> "%" <rdn>
\end{verbatim}\end{quote}

The DNs held by the attribute have the following meanings:
\begin{describe}
\item[\verb+aliasedObjectName+:] The name of an aliased object.
\item[\verb+member+:] Specifies a member of a \verb+groupOfNames+.
\item[\verb+owner+:] The name of the person responsible for the associated object.
\item[\verb+roleOccupant+:] The person who fills an organisational role.
\item[\verb+seeAlso+:] Other objects that may be of interest.
\item[\verb+masterDSA+:] The DSA holding the master EDB file.
\item[\verb+slaveDSA+:] DSAs holding slave EDB files.
\item[\verb+relayDSA+:] A DSA to relay operations to if your DSA is not
connected to a needed network community. Section~\ref{dsarelay}
describes this in more detail.
\item[\verb+manager+:] The manager of a DSA.
\item[\verb+documentAuthor+:] Author of a document.
\item[\verb+secretary+:] The name of a personal Secretary.
\item[\verb+lastModifiedBy+:] The object that last modified the referenced
object.
\item[\verb+associatedName+:] The DN associated with a DNS domain.
\end{describe}

\subsection{OID}
\index{OID}

\begin{center}\small
\begin{tabular}{|l|}\hline
Standard Attributes \\ \hline
	supportedApplicationContext\\
\hline
\end{tabular}
\begin{tabular}{|l|}\hline
MHS Attributes \\ \hline
	mhsDeliverableContentTypes\\
	mhsDeliverableEits\\
	mhsSupportedAutomaticActions\\
	mhsSupportedContentTypes\\
	mhsSupportedOptionalAttributes\\
\hline
\end{tabular}
\end{center}

\begin{quote}\begin{verbatim}
<OIDValue> ::= <oid>
\end{verbatim}\end{quote}

All the syntaxes in this section are represented by OIDs as follows:
\begin{quote}\begin{verbatim}
<oid> ::= <keystring> "." <numericoid> | 
          <keystring> | 
          <numericoid>
<numericoid> ::= <numericstring> | 
                 <numericstring> "." <numericoid>
\end{verbatim}\end{quote}
For example \verb+2.3.4.2+ or \verb+attribute.6+.

In general, the value will be a string using oidtable mappings.

\subsection{ObjectClass}
\index{objectClass attribute}
\begin{center}\small
\begin{tabular}{|l|}\hline
Standard Attributes \\ \hline
	objectClass\\
\hline
\end{tabular}
\end{center}
Although essentially an OID, a separate syntax is provided as an OID has
additional semantics when used as an object class.

\subsection{TelexNumber}
\index{TelexNumber}
\begin{center}\small
\begin{tabular}{|l|}\hline
Standard Attributes \\ \hline
	telexNumber\\
\hline
\end{tabular}
\end{center}

\begin{quote}\begin{verbatim}
<TelexNumberValue> ::= <printablestring> "$" 
                       <printablestring> "$"
                       <printablestring>
\end{verbatim}\end{quote}
This is used to represent \verb+number $ country $ answerback+, for example
\verb+007 28722 $ G $ UCLPHYS+.

\subsection{TeletexTerminalIdentifier}
\index{TeletexTerminalIdentifier}
\begin{center}\small
\begin{tabular}{|l|}\hline
Standard Attributes \\ \hline
	teletexTerminalIdentifier\\
\hline
\end{tabular}
\end{center}

\begin{quote}\begin{verbatim}
<TeletexTerminalIdentifierValue> ::= <printablestring> 
              "$" <optstr> "$"  <optstr> "$" <optstr> 
              "$" <optstr> "$" <optstr>
<optstr> ::= <printablestring> | (null)
\end{verbatim}\end{quote}
Used to represent
\begin{quote}\begin{verbatim}
terminal $ graphic $ control $ page $ misc $ private
\end{verbatim}\end{quote}

\subsection{FacsimileTelephoneNumber}
\index{FacsimileTelephoneNumber}
\begin{center}\small
\begin{tabular}{|l|}\hline
Standard Attributes \\ \hline
	facsimileTelephoneNumber\\
\hline
\end{tabular}
\end{center}
\begin{quote}\begin{verbatim}
<FacsimileTelephoneNumberValue> ::= <printablestring> 
                                [ "$" <faxparameters> ]
<faxparameters> ::= <faxparm> | <faxparm> 
                    "$" <faxparameters>
<faxparm> ::= "twoDimensional" | "fineResolution" | 
              "unlimitedLength" | "b4Length" | 
              "a3Width" | "b4Width" | "uncompressed"
\end{verbatim}\end{quote}
For example \verb"+44 602 123-4567 $ twoDimensional"


\subsection{DeliveryMethod}
\index{DeliveryMethod}
\begin{center}\small
\begin{tabular}{|l|}\hline
Standard Attributes \\ \hline
	preferredDeliveryMethod\\
\hline
\end{tabular}
\begin{tabular}{|l|}\hline
MHS Attributes \\ \hline
	mhsPreferredDeliveryMethods\\
\hline
\end{tabular}
\end{center}

\begin{quote}\begin{verbatim}
<DeliveryValue> = <pdm_component> | 
                  <pdm_component> "$" <pdm>
pdm_component = "any" | "mhs" | "physical" | "telex" | 
                "teletex" | "g3fax" | "g4fax" | "ia5" | 
                "videotex" | "telephone"
\end{verbatim}\end{quote}
For example \verb+mhs $ telephone+.

\subsection{PresentationAddress}
\index{presentation addresses}
\begin{center}\small
\begin{tabular}{|l|}\hline
Standard Attributes \\ \hline
	presentationAddress\\
\hline
\end{tabular}
\end{center}
For more details see Section~\ref{DSA:address}, \volone/ of this manual and \cite{String.Addresses}.

\subsection{Password}
\index{Password}\index{user password attribute}
\begin{center}\small
\begin{tabular}{|l|}\hline
Standard Attributes \\ \hline
	userPassword\\
\hline
\end{tabular}
\end{center}
See Section~\ref{Passwords} for a discussion of this attribute.

\subsection{Certificate}
\index{Certificate}
\begin{center}\small
\begin{tabular}{|l|}\hline
Standard Attributes \\ \hline
	userCertificate\\
	cACertificate\\
\hline
\end{tabular}
\end{center}


\begin{quote}\begin{verbatim}
<CertificateValue> ::= <Algorithm> "#" 
           <EncryptedValue> "#" <Issuer> "#" 
           <Subject> "#" <Algorithm> "#"
           <ProtocolVersion> "#" <Serial> "#" 
           <Validity> "#" <Algorithm> "#" 
           <EncryptedValue>
\end{verbatim}\end{quote}
% Give LaTeX a sporting chance of pagebreaking sensibly.
\begin{quote}\begin{verbatim}
<Algorithm> ::= <OID> "#" <AlgorithmParameters>
<AlgorithmParameters> ::= | <IntegerValue> | 
                            "{ASN}" <HexString>
<Issuer> ::= <DN>
<Subject> ::= <DN>
<ProtocolVersion> ::= <IntegerValue>
<Serial> ::= <IntegerValue>
<Validity> ::= <UTCTime> "#" <UTCTime>
<EncryptedValue> ::= <HexString> | <HexString> "-" <Digit>
\end{verbatim}\end{quote}

\subsection{CertificatePair}
\index{CertificatePair}
\begin{center}\small
\begin{tabular}{|l|}\hline
Standard Attributes \\ \hline
	crossCertificatePair\\
\hline
\end{tabular}
\end{center}

\begin{quote}\begin{verbatim}
<CertificatePairValue> ::= [<CertificateValue>] "|" 
                           [<CertificateValue>]
\end{verbatim}\end{quote}
At least one certificate should be present, although QUIPU will not enforce
this.

\subsection{CertificateList}
\index{CertificateList}
\begin{center}\small
\begin{tabular}{|l|}\hline
Standard Attributes \\ \hline
	authorityRevocationList\\
	certificateRevocationList\\
\hline
\end{tabular}
\end{center}

\begin{quote}\begin{verbatim}
<CertificateListValue> ::= <AlgorithmValue> "#" 
                <Issuer> "#" 
                <AlgorithmValue> "#" <EncryptedValue> "#"
                <UTCTime> ["#" <RevokedCertificates>]

<RevokedCertificates> ::= <AlgorithmValue> "#"
                 <EncryptedValue> 
                 [ "#" *(<RevokedCertificate> "#") ]

<RevokedCertificate> ::= <Subject> "#" 
                 <AlgorithmValue> "#"
                 <SerialNumber> "#" <UTCTime>

\end{verbatim}\end{quote}

\subsection{Guide}
\index{Guide}
\begin{center}\small
\begin{tabular}{|l|}\hline
Standard Attributes \\ \hline
	searchGuide\\
\hline
\end{tabular}
\end{center}

\begin{quote}\begin{verbatim}
<GuideValue>   ::= [<objectClass> "#"] <Criteria>
<Criteria>     ::= <CriteriaItem> | <CriteriaSet> | 
                   "!" <Criteria>
<CrtieriaSet>  ::= ["("] Criteria "@" CriteriaSet [")"] | 
                   ["("] Criteria "|" CriteriaSet [")"]
<CriteriaItem> ::= ["("] <attributetype> "$" 
                   <matchType> [")"]
<matchType>    ::= "EQ" | "SUBSTR" | "GE" | 
                   "LE" | "APPROX"
\end{verbatim}\end{quote}

Note the use of \verb+@+ for ``and'', as the \verb+&+ symbol has other meaning
here.

Some examples of Search Guide are:
\begin{quote}\begin{verbatim}
Person # commonName $ APPROX
( organization $ EQ ) @ (commonName $ SUBSTR)
( organization $ EQ ) @ ((commonName $ SUBSTR) | \
                        (commonName $ EQ))
\end{verbatim}\end{quote}

\subsection{UTCTime}
\index{UTCTime}
\begin{center}\small
\begin{tabular}{|l|}\hline
Standard Attributes \\ \hline
	lastModifiedTime\\
\hline
\end{tabular}
\end{center}
\begin{quote}\begin{verbatim}
<UTCTimeValue> ::= <printablestring>
\end{verbatim}\end{quote}
The string should be formatted using the template \verb+yymmddhhmmssz+
where 
\verb+yy+ represents the year; 
\verb+mm+ represents the month; 
\verb+dd+ represents the day; 
\verb+hh+ represents hours; 
\verb+mm+ represents minutes; 
\verb+ss+ represents seconds; 
\verb+z+  represents the timezone.

For example the string \verb+890602093221Z+ is used to represent
09:32:21 at GMT, on June~$2^{\underline{\mbox{\scriptsize nd}}}$,
{\oldstyle 1989}.

\subsection{Boolean}
\index{Boolean}

No attributes use this syntax.

\begin{quote}\begin{verbatim}
<BooleanValue> ::= <boolean>
<boolean>      ::= "TRUE" | "FALSE"
\end{verbatim}\end{quote}

\subsection{Integer}
\index{Integer}

\begin{center}\small
\begin{tabular}{|l|}\hline
MHS Attributes \\ \hline
	mhsDeliverableContentLength\\
\hline
\end{tabular}
\end{center}

\begin{quote}\begin{verbatim}
<IntegerValue> ::= <d>
\end{verbatim}\end{quote}

\subsection{AccessPoint}
\index{AccessPoint}

\begin{center}\small
\begin{tabular}{|l|}\hline
QUIPU Attributes \\ \hline
	subordinateReference\\
	crossReference\\
	nonSpecificSubordinateReference\\
\hline
\end{tabular}
\end{center}

\begin{quote}\begin{verbatim}
<AccessPointValue> ::= <DN> "#" <PresentationAddress>
\end{verbatim}\end{quote}

These attribute are used to give references to non-QUIPU DSAs and are
discussed in Section~\ref{DSA:nonquipu}.


\section{QUIPU Attribute Syntaxes}

\subsection{ACL}
\label{acl_syntax}
\index{acl attribute}\
\begin{center}\small
\begin{tabular}{|l|}\hline
QUIPU Attributes \\ \hline
	accessControlList\\
\hline
\end{tabular}
\end{center}
\begin{quote}\begin{verbatim}
<aclvalue>	::= <aclwho> "#" <aclwhat> "#" <acltype> ["#"]
<aclwho>	::= "self" | "others" | "group #" <namelist> | 
			"prefix #" <namelist>
<aclwhat>	::= "none" | "detect" | "compare" | 
                    "read" | "add" | "write"
<acltype>	::= "child" | "entry" | "default" | 
                    "attributes #" <oidlist>
<oidlist>	::= <oid> | <oidlist> "$" <oid>
\end{verbatim}\end{quote}
The use of ACL is discussed in Section~\ref{disc_acl}.

\subsection{Schema}
\label{tree_struct}
\index{Schema}\index{treeStructure attribute}
\begin{center}\small
\begin{tabular}{|l|}\hline
QUIPU Attributes \\ \hline
	treeStructure\\
\hline
\end{tabular}
\end{center}
Schema is currently represented by a single OID, and discussed fully in
Section~\ref{adding_data}.

\subsection{ProtectedPassword}
\index{ProtectedPassword}
\begin{center}\small
\begin{tabular}{|l|}\hline
QUIPU Attributes \\ \hline
	protectedPassword\\
\hline
\end{tabular}
\end{center}

\begin{quote}\begin{verbatim}
<ProtectedPasswordValue> ::= <StringValue>
\end{verbatim}\end{quote}

The password encryption mechanism is encapsulated within the matching rules
for this attribute; two values are ``equal'' if they are identically equal or
if one is an encrypted representation of the other. 

Note that ``equals'' is not transitive for this attribute:
If $a = b$ and $b = c$, testing $a = c$ may give ``incomparable'' rather than
``true'' as the result. The reason for this is that in some circumstances 
properly testing
this attribute for equality would consume an unacceptable amount of time. The
security of the encryption mechanism depends on this!

\subsection{SecurityPolicy}
\index{SecurityPolicy}
\begin{center}\small
\begin{tabular}{|l|}\hline
QUIPU Attributes \\ \hline
	entrySecurityPolicy\\
	dsaDefaultSecurityPolicy\\
	dsaPermittedSecurityPolicy\\
\hline
\end{tabular}
\end{center}

This attribute is currently handled as ``ASN.1'', see Section~\ref{ASN_syntax}.

\subsection{EdbInfo}
\index{edbInfo attribute}
\begin{center}\small
\begin{tabular}{|l|}\hline
QUIPU Attributes \\ \hline
	eDBinfo\\
\hline
\end{tabular}
\end{center}
\begin{quote}\begin{verbatim}
<EdbInfoValue> ::= <name> "#" <name> "#" <namelist> ["#"]
\end{verbatim}\end{quote}
This attribute is discussed in Section~\ref{slave_update}.

\subsection{InheritedAttribute}
\index{inherited attribute}
\begin{center}\small
\begin{tabular}{|l|}\hline
QUIPU Attributes \\ \hline
	InheritedAttribute\\
\hline
\end{tabular}
\end{center}
\begin{quote}\begin{verbatim}
<InheritedAttributeValue> ::= [ <oid> "$" ] 
                    [ "always" <InheritedList> ]
                    [ "default" <InheritedList> ]
<InheritedList> ::= "(" NEWLINE <AttributeSequence> 
                    NEWLINE ")"
\end{verbatim}\end{quote}
This attribute is discussed in Section~\ref{attr_inherit}.



\section{COSINE/Internet Attribute Syntaxes}

As well the attribute syntaxes defined by X.500, QUIPU recognises and
uses those defined by the COSINE/Internet schema document
\cite{Cosine.NA}.
This document is a sucessor to the Thorn syntaxes 
\cite{thorn-na} used in previous version of QUIPU.

\subsection{Mailbox}
\index{Mailbox}
\begin{center}\small
\begin{tabular}{|l|}\hline
COSINE/Internet Attributes \\ \hline
	otherMailbox\\
\hline
\end{tabular}
\end{center}
\begin{quote}\begin{verbatim}
<MailboxValue> ::= <printablestring> "$" <IA5String>
\end{verbatim}\end{quote}
For example \verb+internet $ quipu-support@cs.ucl.ac.uk+.

\subsection{CaseIgnoreIA5String}
\index{CaseIgnoreIA5String}
\begin{center}\small
\begin{tabular}{|l|}\hline
COSINE/Internet Attributes \\ \hline
	rfc822Mailbox\\
	domainComponent\\
	nRSTextualDescription\\
	associatedDomain\\
\hline
\end{tabular}
\end{center}

These attributes are handled as IA5Strings, but use the matching rules
for the CaseIgnoreString syntax.


\subsection{Photo}
\index{Photo}\index{ASN.1}
\begin{center}\small
\begin{tabular}{|l|}\hline
COSINE/Internet Attributes \\ \hline
	photo\\
	personalSignature\\
\hline
\end{tabular}
\end{center}
Photos are a special case of ``ASN.1'' (see Section~\ref{ASN_syntax}), but the 
output format is different.
Your attention is drawn to Section~\ref{file_attr} which discusses storing
large attributes in a separate file.

With this version of QUIPU, the attribute should be encoded as a bitstring, in a two-dimensional
G3FacsimilePage encoding
(as per recommendation T.4).
However, in the future this will migrate to an X.400 bodypart,
containing a G3FacsimilePage, with the advantage this can then be one-
or two-dimensional.
The current photo decoders will recognise the new format.

\subsection{Audio}
\index{Audio}
\begin{center}\small
\begin{tabular}{|l|}\hline
COSINE/Internet Attributes \\ \hline
	audio\\
\hline
\end{tabular}
\end{center}
The data is in the form of a u-law encoded sound file.  If you have a Sun
Microsystems Sun 4 workstation, the demonstration ``play'' utility 
can be used to play the attributes. (A modified version of ``play''
that ignores the file header is better --- such as that made available
by J.~Michael Bauer of the University of Calgary.  QUIPU-support can supply
details if required.)

This is an interim demonstration attribute; the format is
temporary and may be replaced when a suitable standard 
audio encoding mechanism is found.

\subsection{Data Quality}
\index{Data Quality syntax}
\begin{center}\small
\begin{tabular}{|l|}\hline
COSINE/Internet Attributes \\ \hline
	singleLevelQuality\\
	subtreeMinimumQuality\\
	subtreeMaximumQuality\\
\hline
\end{tabular}
\end{center}

\begin{quote}\begin{verbatim}
<dataQuality>   ::=  <compKeyword> "#" <attrQuality>
                     "#" <listQuality>
                     ["#" <description> ]
<attrQuality>   ::= <levelKeyword> "+" <compKeyword>
<listQuality>   ::= <list> "$" <list> <listQuality>
<list>          ::= <attribute> "+" <attrQuality>
<compKeyword>   ::= "NONE" | "SAMPLE" |
                    "SELECTED" | "SUBSTANTIAL" |
                    "FULL" 
<levelKeyword>  ::= "UNKNOWN" | "EXTERNAL" | 
                    "SYSTEM-MAINTAINED" | "USER-SUPPLIED" 
\end{verbatim}\end{quote}

These attributes are used to define the quality of data at a single
level of the DIT, or even the whole subtree.

\subsection{DSA Quality}
\index{DSA Quality syntax}
\begin{center}\small
\begin{tabular}{|l|}\hline
COSINE/Internet Attributes \\ \hline
	dSAQuality\\
\hline
\end{tabular}
\end{center}

\begin{quote}\begin{verbatim}
<dsaQuality> ::=  <DSAkeyword> ["#" <description>]
<DSAkeyword> ::=  "DEFUNCT" | "EXPERIMENTAL" |
                  "BEST-EFFORT" | "PILOT-SERVICE" |
                  "FULL-SERVICE"
\end{verbatim}\end{quote}

Used to define the quality of a DSA.

\section{MHS Attribute Syntaxes}

\begin{center}\small
\begin{tabular}{|l|l|}
\hline
Attribute & Syntax \\ 
\hline
mhsORaddresses & ORAddress \\
mhsDLMembers & ORName \\
mhsDLSubmitPermissions & DLSubmitPermissions \\
\hline
\end{tabular}
\end{center}

The Syntaxes shown above are not currently recognised by QUIPU,
and are thus handled as raw ASN.1 (see Section~\ref{ASN_syntax}).
However, the PP\index{PP} MHS System has QUIPU compatible syntax
handlers that can be easily added to the DISH program, and
provides a tool for handling entries of the ``distribution list''
object class.
PP is available under conditions similar to the ISODE; for details 
you should contact ``PP-Support@cs.ucl.ac.uk''.



\section{ASN.1}
\label{ASN_syntax}
As the preceding sections in this chapter have mentioned, not all syntaxes
have a string representation defined by QUIPU, so are represented by the raw
ASN.1.

An example would be:
\begin{quote}\begin{verbatim}
photo= {ASN}0308207b4001488001fd...
\end{verbatim}\end{quote}
where \verb+0308+\ldots is a hexadecimal representation (encoded using the
``Basic Encoding Rules'') of the ASN.1 defined attribute.

Attributes stored as ASN.1, will usually be matched correctly, with
the following exceptions:

\begin{itemize}
\item
There is an IMPLICIT Set in the ASN.1.  The DSA will not detect the
set, and so will not know to match components in arbitrary order.

\item
If special matching rules apply, for example, special rules to
determine equivalence of telephone numbers.  Such rules would need
to be represented by code in the DSA.
\end{itemize}

