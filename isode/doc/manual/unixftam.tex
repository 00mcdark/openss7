% run this through LaTeX with the appropriate wrapper

\chapter	{UNIX Implementation}\label{unixftam}
The File Transfer, Access, and Management (FTAM) standard is the OSI file
service.
Included in the release is a fairly complete FTAM implementation in the
context of the particular file services it offers.
It is a minimal implementation in as much as it offers only four core
services: transfer of binary files,
transfer of text files,
directory listings,
and file management.
The implementation included has been tested on both Berkeley and AT\&T
SVR2 and SVR3~\unix/.
Both the FTAM initiator and responder programs have \unix/ manual entries.

\section	{Implementation}\label{unixftam:code}
If you have access to the source tree for this release,
the directory \file{ftam2/} contains the code for the responder and initiator.

\subsection	{The Initiator}
There is currently one initiator which uses FTAM: \man ftam(1c).
Supported are:
the no-recovery FTAM-QoS;
any of 
the transfer, management, and transfer and management service classes;
the kernel, read, write, limited file management, enhanced file management, and
grouping functional units;
and, the kernel and storage attribute groups.
Only three document types are supported as of this writing:
unstructured text files (FTAM-1),
unstructured binary files (FTAM-3),
and filedirectory files (NIST-9).

The \pgm{ftam} program is an interactive FTAM initiator
which prompts the user for commands.
Generating an interrupt,
usually by typing control-C (`\verb"^C"'),
at the top-level does nothing,
but generating an interrupt twice in a row at the top-level terminates
\pgm{ftam};
generating an interrupt during additional prompting causes \pgm{ftam} to abort
the command;
typing generating an interrupt during file transfer causes the transfer to be
aborted.

\subsubsection	{Commands}
Here are the commands to \pgm{ftam}:
\begin{describe}
\item[append {\tt source destination}]
Appends to a file in the filestore.

\item[cd {\tt [dir]}]
Changes the working directory on the virtual filestore.
This requires the {\bf realstore\/} variable to be set appropriately.

\item[chgrp {\tt group file $\ldots$}]
Changes the account attribute of the named files.

\item[close]
Terminates the association with the virtual filestore.

\item[dir {\tt [file]}]
Prints a long directory listing.

\item[echo {\tt file $\ldots$}]
Simply echoes any arguments.
Useful for seeing how glob\-bed expressions will evaluate.

\item[fdir {\tt stream [file]}]
Prints a long directory listing to a file or program.
If \verb"stream" starts with a vertical bar (`\verb"|"')
then the named program is invoked;
otherwise the named file is written.

\item[fls {\tt stream [file]}]
Prints a directory listing to a file or program.
If \verb"stream" starts with a vertical bar (`\verb"|"')
then the named program is invoked;
otherwise the named file is written.

\item[get {\tt source destination}]
Retrieves a file.

\item[help {\tt [command]}]
Prints help information.
For detailed information, try ``\verb*"help ?"''.

\item[lcd {\tt [file]}]
Changes the working directory on the local system.

\item[ls {\tt [file]}]
Prints a directory listing.

\item[mkdir {\tt dir $\ldots$}]
Creates a directory.

\item[mv {\tt source destination}]
Renames a file.

\item[open {\tt host user [account]}]
Associates with the virtual filestore.

\item[put {\tt source destination}]
Stores a file.

\item[pwd]
Prints the working directories.

\item[quit]
Terminates the association with the virtual filestore and exits.

\item[rm {\tt file $\ldots$}]
Deletes a file.

\item[set {\tt variable value}]
Displays or changes variables.
For detailed information, try ``\verb*"set ?"''.

\item[status]
Shows the current status.
\end{describe}

\subsubsection	{Variables}
Here are the variables which effect \pgm{ftam}'s behavior.
\begin{describe}
\item[bell]
Rings the bell after each command terminates.
Useful for long file transfers when you want to attend to other matters and
be notified when you can type another command.
Boolean (values: {\bf on\/} or {\bf off\/}).

\item[debug]
This enables voluminous output during file transfers,
among other things.  Boolean.

\item[glob]
This enables the expansion of shell meta-characters.
Operations which perform globbing
require the {\bf realstore\/} variable to be set appropriately.
Boolean.

\item[hash]
This enables the printing of hash marks during file transfers.
Values:
\verb"off", \verb"on", \verb"total".

\item[override]
This sets the creation override mode for files being written to the virtual
filestore.
If the file being created already exists,
then one of four alternatives is taken.
Values:
\begin{describe}
\item[\verb"fail":]
the creation operation;
\item[\verb"select":]
use the existing file with its old contents and attributes;
\item[\verb"write":]
zero-truncate if it already exists, and use the existing file with its old
attributes;
and,
\item[\verb"delete":]
if it already exists, then create a new file with new attributes.
\end{describe}
This defaults to \verb"write".

\item[qualifier]
This sets the ``qualifier'' portion of the srevice which \pgm{ftam} will
associate with.
It is needed when using the current implementation of the MITRE FTAM/FTP
gateway. 
This defaults to \verb"filestore".

\item[query]
This determines if \pgm{ftam} should ask the user to confirm operations
involving globbing that expand to more than one filename.
Boolean.
This defaults to \verb"on".

\item[realstore]
Sets the type of remote realstore associated with the virtual filestore.
This is used to help \pgm{ftam} act friendlier to the user!
Values: \verb"unix", \verb"unknown".
\[\fbox{\begin{tabular}{lp{0.67\textwidth}}
\bf NOTE:&	The concept of a {\bf realstore\/} is contrary to the notion of
		open systems as it is an $N*M$ (not $N+M\/$) method.
\end{tabular}}\]

\item[trace]
This enables the tracing of FTAM PDUs.  Boolean.

\item[tracefile]
This defines the file where tracing information is appended.

\item[type]
This defines the file transfer mode to use.
Values: \verb"default", \verb"binary", and \verb"text".

\item[verbose]
This enables printing of informative diagnostics during operation.  Boolean.

\item[watch]
This enables watch mode,
something in between debug mode (too voluminous),
and verbose mode (not informative enough).  Boolean.

\item[userdn]
This defines the Distinguished Name to be used when binding to the Directory
for AE-lookup.
DN-string.

\item[{\em xyz\/}sapfile]
This defines the file where {\em xyz\/}PDU tracing information is appended.
Values: any filename, or \verb"-" for the diagnostic output.

\item[{\em xyz\/}saplevel]
This enables tracing of the {\em xyz\/} module.\\
Values: \verb"none", \verb"exceptions", \verb"notice", \verb"pdus",
\verb"trace", and \verb"debug".
\end{describe}

\subsubsection	{Options}
Here are the command line options:
\begin{describe}
\item[-a {\em acct}]
Sets the account to be used on the virtual filestore.

\item[-d]
Sets {\bf debug}.

\item[-f]
Inhibits reading of the user's \file{\$HOME/.ftamrc} file on startup.

\item[-h]
Sets {\bf hash}.

\item[-o  {\em mode}]
Sets {\bf override}.

\item[-t]
Sets {\bf trace}.

\item[-u  {\em user}]
Sets the initiator identity to be used on the virtual filestore.

\item[-v]
Sets {\bf verbose\/} (default for interactive use).

\item[-w]
Sets {\bf watch}.
\end{describe}

\subsection	{The Responder}
The \man ftamd(8c) program implements the file service.
It implements {\em filestore\/} abstractions directly on the \unix/ filesystem.
Supported are:
the no-recovery FTAM-QoS;
any of
the transfer, management, and transfer and management service classes;
the kernel, read, write, limited file management, enhanced file management, and
grouping functional units;
and, the kernel and storage attribute groups.
Only three document types are supported as of this writing:
unstructured text files (FTAM-1),
unstructured binary files (FTAM-3),
and filedirectory files (NIST-9).

\subsubsection	{Authentication}
An FTAM initiator must be listed in the \man passwd(5) file and have a
non-empty password.
Further, as with the \man ftpd(8c) daemon,
the username must not appear in the \file{ftamusers} file in the ISODE
\verb"ETCDIR" directory or in the \file{/etc/ftpusers} file.
(In fact, many of the mechanisms in \pgm{ftamd} are based on the \pgm{ftpd}
program supplied with Berkeley \unix/.)

If the username \verb"ANON" or \verb"ftp" is given,
then \pgm{ftamd} treats this as a guest access,
similar to the ``anonymous'' facility supported by the \pgm{ftpd} daemon.
An entry in the \file{/etc/passwd} file for user
\verb"ftp" must be present with a non-zero UID.
For guest access,
a \man chroot(2) to the guest home directory
is executed to restrict access to the system.
\[\fbox{\begin{tabular}{lp{0.8\textwidth}}
\bf NOTE:&	The anonymous account is inherently dangerous and should be
		avoided when possible.
		It is also inherently useful.
\end{tabular}}\]

The Berkeley UNIX version of this program runs with the effective UID of the
FTAM initiator,
but also with the real UID of the super-user.
This is necessary to change the account attribute on files
using \man chown(2).
The possible security holes have been extensively considered,
but may be incomplete.

The AT\&T UNIX version,
which lacks kernel support for this technique, acts differently.
Immediately upon association establishment,
it changes both the real and effective UID to that of the FTAM initiator.
To change the account attribute on files,
it invokes the \man chgrp(1) program.
Similarly, to create or delete directories,
it invokes either the \man mkdir(1) program or the \man rmdir(1) program.
Finally,
it is unable to change the filesize attribute to a non-zero value
if this value is smaller than the current filesize.

Finally,
on Berkeley \unix/ systems,
the \man wtmp(5) file is updated as appropriate.
(We couldn't figure out how to update \file{wtmp} under AT\&T \unix/
using the description in the SVID!)

\subsubsection	{Virtual Filestore}
Here are the file attribute mappings.
Most attributes are derived by doing a \man stat(2) on the file and then
examining the indicated field in the resulting structure.
\begin{describe}
\item[filename]
A single component, relative to the user's \file{\$HOME}.
Changing this attribute is equivalent to a \man rename(2).

\item[contents-type]
Based on the \verb"st_mode" field:
\begin{describe}
\item[NIST-9]	for directories;

\item[FTAM-1]	for regular files appearing to be textual;
and,

\item[FTAM-3]	for all other regular files.
\end{describe}
Files that are neither regular nor directories are inaccessible via this
implementation of the VFS (i.e., special files).

\item[account]
The \verb"st_gid" field according to \man group(5).
Changing this attribute is equivalent to a \man chgrp(1).

\item[date-and-time-of-creation]
The \verb"st_mtime" field.

\item[date-and-time-of-last-modification]
The \verb"st_mtime" field.

\item[date-and-time-of-last-read-access]
The \verb"st_atime" field.

\item[date-and-time-of-last-attribute-modification]
\ \\	%%% hack
The \verb"st_ctime" field.

\item[identity-of-creator]
The \verb"st_uid" field according to \man passwd(5).

\item[identity-of-last-modifier]
The \verb"st_uid" field according to \man passwd(5)
(if the value of the \verb"st_mode" field guarantees uniqueness).

\item[identity-of-last-reader]
The \verb"st_uid" field according to \man passwd(5)
(if the value of the \verb"st_mode" field guarantees uniqueness).

\item[identity-of-last-attribute-modifier]
The \verb"st_uid" field according to \man passwd(5)
(if the value of the \verb"st_mode" field guarantees uniqueness).

\item[file-availability]
Immediate.

\item[permitted-actions]
Depends on the \verb"st_mode" the as interpreted by \man access(2):
\verb"R_OK" for permission to read;
\verb"W_OK" for permission to write;
permission is always granted to read attributes;
permission is granted to change attributes if the initiator has uid equal to
the \verb"st_uid" field;
and,
permission to delete is based on writability of parent directory.

\item[filesize]
The \verb"st_size" field.

\item[future-filesize]
Not available.

\item[access-control]
Not available.

\item[encryption-name]
Not available.

\item[legal-qualifications]
Not available.

\item[private-use]
Not available.
\end{describe}

The activity attribute mappings are straight-forward.
The read action corresponds to reading UNIX files.
The insert, replace, extend, and erase actions correspond to writing
UNIX files.
Concurrency control is supported for reading and writing,
but not for reading or changing attributes, or for deleting files.
