% run this through LaTeX with the appropriate wrapper

\section{Models}

\subsection{Access Control}

QUIPU uses access control lists to represent access rights; the subjects
are DUAs (represented by DNs) and the objects are entries in the DIT,
attributes of entries, and the child-of relation for each entry (all represented
by DNs, and, in the case of attributes, the attribute type as well).

Furthermore, objects can be {\em containers} that hold other objects. To access
an object within an container, it is necessary to have access rights both to
the object and the container that holds it. In particular, entries in the DIT
are containers. The attributes of the entry and the list of its children
are contained within the entry. The children themselves are {\em not} contained
within the parent.

The possible levels of access are as follows: {\em none}, {\em detect}, 
{\em compare}, {\em read}, {\em add} and {\em write}.

\subsection{Security Domains}

The DIT is a global database, maintained by many separate organisations.  
It is possible for the manager of DSA to change its interpretation of the 
access control rules, in order to  obtain access fraudulently to information
held by other DSAs. 

As a result, data in a DSA may need to be protected from the managers of other
DSAs, as well as from users. This means that special checks must be performed
on DSP operations that come from untrusted DSAs.

In order to decide to trust a DSA, it is necessary to be able to authenticate it
and to know that it should be trusted. The current version of QUIPU 
can do neither of these, and
so assumes that {\em all} DSAs are untrusted.


\section{Representation in the DIT}

\subsection{Simple Authentication}

DUAs which use simple authentication have their password stored in the
{\em userPassword} attribute of their entry.

\subsection{Protected Simple Authentication}

QUIPU represents both passwords (the $K_{A}$s) and authenticators by the
ASN.1 type {\em ProtectedPassword} shown if Figure~\ref{protected-py}.

When this structure represents a password,
{\em algorithm} indicates which one-way function is being used and
{\em password} is the (unencrypted) password. The other fields are
not supplied.

When this structure represents an authenticator, {\em password}
is the hash value 
\begin{math}
f_{1}(K_{A}, t_{1}, q_{1})
\end{math},
where $t_{1}$ is {\em time1}, $q_{1}$ is {\em random1} and 
$K_A$ is the password. {\em algorithm} is not supplied.

A relation $\geq$ can be defined on the set of passwords and
authenticators as follows: If $a$ is a password, $b$ is an authenticator,
and $a$ matches $b$, then $a \geq b$. (Also, $a \geq a$). It can be shown
that this relation
is a partial ordering of the set, justifying our use of the symbol ``$\geq$''.
We could have defined an attribute syntax for which the above relation corresponded
to ``greater than or equal to'' without violating any of the implied semantics
of X.500. However, the DAP {\em compare} operation cannot be used to test 
``greater than or equal to'', only ``equal to'', and we wanted an easy
way to ask the 
directory to check this relation. To achieve this, we made the relation
correspond to ``equals'' for the {\em ProtectedPassword} attribute syntax.

\tagrind{protected}{ProtectedPassword}{protected-py}

\subsection{Access Control Lists}

The access control list for an entry is held in that entry's
{\em accessControlList} attribute.

\subsection{Security Policies}

The {\em entrySecurityPolicy} attribute of an entry is used to indicate the
amount of care that should be taken to preserve the integrity and
confidentiality of that entry. While the {\em accessControlList} indicates
who should have access to the entry, the {\em entrySecurityPolicy} indicates
which steps should be taken to prevent unauthorized access.

The {\em dsaDefaultSecurityPolicy} attribute of a DSA indicates which
precautions the DSA will take when dealing with entries that do not have
an {\em entrySecurityPolicy} attribute.

The {\em dsaPermittedSecurityPolicy} attribute of a DSA indicates which
security policies the DSA is prepared to enforce. A DSA may not support a
security policy either because it lacks the necessary software or
because its manager wishes to forbid use of that policy.

The security policy attribute syntax is not yet implemented.

\subsection{Labels}

Security labels are one means of enforcing rule-based access control
(sometimes referred to as mandatory access control). QUIPU does not
provide mandatory access control in any form.

\section{Distributed Operations}
\label{des-auth}

\subsection{(Protected) Simple Authentication}

If the responding DSA holds the initiator's entry, then it may check the
password directly. Otherwise, the responder will formulate a DSP compare
operation to check it. If the result is {\em true}, the bind will be
accepted.

This approach can only be used to check a DAP bind; if it were used in DSP,
there would be a danger of livelock. Hence, QUIPU will not attempt to
verify the password in a DSP bind.

\subsection{Strong Authentication}

In strong authentication, it is necessary to build a certification path
for the originator that will be believed by the recipient. From the standpoint
of the protocol, the originator builds the certification path and sends it to
the recipient. However, the originator does not know for certain which CA's
the recipient trusts, and so cannot be sure of constructing a valid path.

QUIPU solves this problem by treating the presented certification path
as a hint. The recipient tries to build an acceptable certification path
out of the components of the presented path and the certificates it has cached.

\subsection{Restricting Read Access}

To maintain the confidentiality of data, results from read, search etc. must
not be passed to another DSA unless that DSA has rights to them.
The current version of QUIPU solves this problem in the manner described below.

When a DUA requests private data via an untrusted DSA, QUIPU checks whether
any of the requested information can be seen by the DUA but not by the DSA.
If it can, the security error ``inappropriate authentication'' is sent to
the untrusted DSA. (Otherwise, the data can be safely sent over DSP).

QUIPU DSAs will interpret this error if it is sent by another QUIPU DSA 
to mean that they are not trusted, and
will pass a referral back to the DUA. (Other DSAs may behave differently;
X.500 ought to be clarified in this area).

The DUA will chase the referral, and repeat its query directly to the DSA
holding the data. The DSA will then give the DUA the results.

\subsection{Restricting Write Access}

To maintain the integrity of the data, requests to modify data over DSP must
not be accepted unless they are signed or can be performed by anyone.

QUIPU solves this problem as in the read access case, 
by returning a security error.
When strong authentication is added to QUIPU, modify requests will be
accepted over DSP provided that they are signed by the DUA.

As an optimisation, QUIPU DSAs never chain modify requests unless they are
signed. (The operation will probably be rejected, so it is quicker to
give a referral to the DUA immediately, rather than trying to chain.)

\subsection{Caching}
\label{dsp-acl}

DSAs try to improve performance by caching the results of DSP operations.
Caching interferes with access control in two ways:

\begin{enumerate}
\item
The cached data may be incomplete.
The DUA that requested it may not have had
rights to all the data, or the DSA that held the data may not have been
prepared to send all of it over DSP.
\item
Caching can prevent access control.
The original data may have been subject to access controls which the holder of
the cache is unaware of.
\end{enumerate}

QUIPU has a special solution that only works between QUIPU DSAs (using the
QUIPU DSP application context): the ACL is sent along with the data across
DSP.

QUIPU DSAs will only cache data if they have the ACL to go with it.
This ensures 
that the same access controls will be applied to a cache as to the master copy.
It also enables a DSA to tell if the cache is complete. The required
information 
is checked against the ACL in the cache; if any of it is not publicly readable,
then it will not have been returned over DSP and the cache cannot be used.

\subsection{Replicated Data}

The mechanism originally envisaged for replicating data was ``reliable
ROS'':  remote operations carried by X.400(88). DSAs could use the security
mechanisms in X.400(88) to provide both integrity and confidentiality
for the EDB updates.
This would make the slave updates the only place in QUIPU where cryptographic 
techniques are used to provide confidentiality, as opposed to integrity. The
provision of confidentiality here is justified in view of the sensitivity of
entire entry data blocks; many of them will contain user's passwords, for
example.

However, X.400(88) is not yet widespread, and hence cannot be used as the
primary means of replicating data. In the interim, the {\em getEDB} mechanism
is used. This provides neither confidentiality nor integrity, and is
not even subject to access control (as DSP is unauthenticated).

