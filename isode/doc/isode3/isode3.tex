% -*- LaTeX -*-		(really SLiTeX)

\documentstyle[blackandwhite,landscape,oval,pagenumbers,small]{NRslides}

\font\xx=cmbx10
\font\yy=cmbx7

\raggedright

\input trademark
\let\tradeNAMfont=\relax
\let\tradeORGfont=\relax

\begin{document}

\title	{ISODE:\\ OPENLY AVAILABLE OSI\\ FOR TCP/IP NETWORKS}
\author	{Marshall T.~Rose\\ The Wollongong Group, Inc.}
\date	{September 28, 1988}
\maketitlepage


\begin{bwslide}
\part*	{AGENDA}\bf

\begin{description}
\item[PART I:]		CURRENT DISTRIBUTION

\item[PART II:]		WHAT'S PLANNED
\end{description}
\end{bwslide}


\begin{bwslide}
\ctitle	{WHAT IS ISODE?}

\begin{nrtc}
\item	THE ISO DEVELOPMENT ENVIRONMENT

\item	AN OPENLY AVAILABLE IMPLEMENATION OF THE UPPER LAYERS OF OSI?

\item	A BASIS FOR THE TRANSITION TO OSI?

\item	A PLAYGROUND FOR ``THE PIED-PIPER OF OSI''?
\end{nrtc}
\end{bwslide}


\begin{bwslide}
\part	{CURRENT DISTRIBUTION}\bf

\begin{nrtc}
\item	STATUS: OPENLY AVAILABLE UNDER AN IMPLICIT ``HOLD HARMLESS'' CLAUSE

\item	CURRENT RELEASE: 4.0
    \begin{nrtc}
    \item	AVAILABLE JULY 24, 1988
    \end{nrtc}
\end{nrtc}
\end{bwslide}


\begin{bwslide}
\ctitle	{CURRENT DISTRIBUTION (cont.)}

\begin{nrtc}
\item	DISTRIBUTION EITHER VIA POSTAL MAIL OR ARPAnet FTP
    \begin{nrtc}
    \item	SOURCE: \~{}9.25MB

    \item	DOC: 5~VOLUME USER'S MANUAL (\~{}800~PAGES)

    \item	DISTRIBUTION SITES: US, UK, NL, AND AU

    \item	PRICE: \~{}350~US DOLLARS
    \end{nrtc}
\end{nrtc}
\end{bwslide}


\begin{bwslide}
\ctitle	{NORTH AMERICA DISTRIBUTION}\small

\[\begin{tabular}{rl}
Postal address:&UNIVERSITY OF PENNSYLVANIA\\
&		DEPARTMENT OF COMPUTER AND INFORMATION SCIENCE\\
&		MOORE SCHOOL\\
&		ATTN: DAVID J. FARBER (ISODE DISTRIBUTION)\\
&		200 SOUTH 33RD STREET\\
&		PHILADELPHIA, PA 19104-6314\\
&		USA\\[0.2in]
Telephone:&	+1--215--898--8560\\[0.2in]
Price:&		US\$350.00 (CHECKS ONLY)
\end{tabular}\]
\end{bwslide}


\begin{bwslide}
\ctitle	{LANGUAGES AND OPERATING SYSTEMS}

\begin{nrtc}
\item	CODED ENTIRELY IN C FOR \unix/
    \begin{nrtc}
    \item	REQUIRES NO KERNEL MODIFICATIONS    
    \end{nrtc}

\item	KNOWN PORTS FOR BERKELEY \unix/ (4.2 and 4.3):
    \begin{nrtc}
    \item	VAXen, SUNs, Pyramids, RTs, etc.
    \end{nrtc}

\item	KNOWN PORTS FOR AT\&T \unix/ (SVR2 and SVR3):
    \begin{nrtc}
    \item	SGI, 3Bs, 386s, RT (AIX)
    \end{nrtc}

\item	MS-DOS (CURRENTLY CLIENT SIDE ONLY)
    \begin{nrtc}
    \item	EARLY PORT DONE BY HP IN THE UK

    \item	NEGOTIATING AVAILABILITY OF CODE
    \end{nrtc}
\end{nrtc}
\end{bwslide}


\begin{bwslide}
\part*	{APPLICATION ARCHITECTURE}\bf

\begin{nrtc}
\item	A (NEARLY) COMPLETE IMPLEMENTATION OF THE UPPER LAYERS

\item	CURRENTLY IS LEVEL (FINALLY!)

\item	ALIGNED WITH THE U.S.~GOSIP
\end{nrtc}
\end{bwslide}


\begin{bwslide}
\ctitle	{THE APPLICATION ENVIRONMENT}

\vskip.5in
\diagram[p]{figure9}
\end{bwslide}


\begin{bwslide}
\ctitle	{AN ALTERNATE ENVIRONMENT:\\ MHS ARCHITECTURE (c.~1984)}

\vskip.5in
\diagram[p]{figure10}
\end{bwslide}


\begin{bwslide}
\ctitle	{THE TRANSPORT SWITCH}

\begin{nrtc}
\item	DECIDES WHICH TS-STACK TO USE FOR A CONNECTION

\item	FOR TP0:
    \begin{nrtc}
    \item	TCP (SOCKETS)

    \item	X.25 (SEVERAL INTERFACES, MOSTLY SOCKETS)
    \end{nrtc}

\item	FOR TP4:
    \begin{nrtc}
    \item	TWG's PROPRIETARY WIN/LLS (TLI)

    \item	SunLink OSI (EVENT SOCKETS)
    \end{nrtc}

\item	EXPERIENCE SHOWS IT IS FAIRLY EASY TO ADD A NEW TS-STACK TO THE SWITCH
\end{nrtc}
\end{bwslide}


\begin{bwslide}
\part*	{WHERE IN USE}\bf

\begin{nrtc}
\item	HARD TO TELL HOW MANY COPIES ARE IN USE (DUE TO AVAILABILITY VIA
	ARPAnet FTP)

\item	AT LAST COUNT, ABOUT 350~DIFFERENT SITES USING ISODE

\item	IN ADDITION TO SITES IN NORTH AMERICA:
    \begin{nrtc}
    \item	WESTERN EUROPE

    \item	MIDDLE EAST (ISRAEL)

    \item	SOUTH PACIFIC (AUSTRALIA, NEW ZEALAND)

    \item	ASIA (SOUTH KOREA, JAPAN)
    \end{nrtc}
\end{nrtc}
\end{bwslide}


\begin{bwslide}
\ctitle	{PROJECTS}

\begin{nrtc}
\item	THREE PILOT PROJECTS IN OSI INFRASTRUCTURE IN EUROPE
    \begin{nrtc}
    \item	A NATIONAL PROJECT IN THE UK

    \item	A NATIONAL PROJECT IN WEST GERMANY (DFN)

    \item	A PROJECT FOR RARE (THE EUROPEAN ACADEMIC COMMUNITY)
    \end{nrtc}

\item	IN USE BY DIFFERENT CONFORMANCE TESTING ORGANIZATIONS
    \begin{nrtc}
    \item	THE CORPORATION FOR OPEN SYSTEMS IN THE US

    \item	THE NATIONAL COMPUTER CENTRE IN THE UK
    \end{nrtc}

\item	IN USE AT NARDAC LABORATORY (U.S.~NAVY)

\item	ENDORSED BY THE NSF (DNCRI)
\end{nrtc}
\end{bwslide}


\begin{bwslide}
\part*	{THE APPLICATIONS COOKBOOK}\bf

\begin{nrtc}
\item	TOOLS TO FACILITATE DEVELOPMENT OF APPLICATIONS ARE CRITICAL

\item	IDEA IS TO DEVELOP TOOLS TO AUTOMATE USE OF OSI REMOTE OPERATIONS
	SERVICE AS A GENERAL REMOTE PROCEDURE CALL FACILITY

\item	FOR MORE DETAILS:
\begin{quote}
BUILDING DISTRIBUTED APPLICATIONS IN AN OSI FRAMEWORK
\end{quote}
APPEARING IN ConneXions, MARCH, 1988
\end{nrtc}
\end{bwslide}


\begin{bwslide}
\ctitle	{REMOTE OPERATIONS SERVICE (ROS)}

\begin{nrtc}
\item	STANDARDIZED MECHANISM FOR SPECIFYING TRANSACTIONS

\item	EMPLOYS POWER OF ASN.1

\item	USED IN MANY INTERESTING OSI APPLICATIONS
    \begin{nrtc}
    \item	MESSAGE HANDLING SYSTEMS

    \item	DIRECTORY SERVICES

    \item	NETWORK MANAGEMENT

    \item	REMOTE DATABASE ACCESS
    \end{nrtc}

\item	CURRENTLY CONNECTION-ORIENTED, BUT CONNECTIONLESS-MODE IS UNDER STUDY
\end{nrtc}
\end{bwslide}


\begin{bwslide}
\ctitle	{GENERAL ORGANIZATION}

\begin{nrtc}
\item	AT COMPILE-TIME:
    \begin{nrtc}
    \item	USE RO-SPECIFICATION TO GENERATE SUPPORT FACILITIES
    \end{nrtc}

\item	AT RUN-TIME:
    \begin{nrtc}
    \item	USE DIRECTORY SERVICES TO LOCATE/REGISTER NETWORK SERVICES
		(NEARLY THERE!)

    \item	USE ASSOCIATION CONTROL TO BIND/UNBIND APPLICATIONS

    \item	USE REMOTE OPERATIONS TO INVOKE TRANSACTIONS
    \end{nrtc}
\end{nrtc}
\end{bwslide}


\begin{bwslide}
\ctitle	{STATIC (COMPILE-TIME) ORGANIZATION}

\vskip.15in
\diagram[p]{figure11}
\end{bwslide}


\begin{bwslide}
\ctitle	{DYNAMIC (RUN-TIME) ORGANIZATION}

\vskip.15in
\diagram[p]{figure12}
\end{bwslide}


\begin{bwslide}
\part	{WHAT'S PLANNED}\bf

\begin{nrtc}
\item	APPLICATIONS

\item	OSI-POSIX PROJECT
\end{nrtc}
\end{bwslide}


\begin{bwslide}
\part*	{APPLICATIONS}\bf

\begin{nrtc}
\item	UPPER LAYERS FLESHED OUT AND STABLE

\item	NOW TIME TO FINISH UP APPLICATIONS
\end{nrtc}
\end{bwslide}


\begin{bwslide}
\ctitle	{CURRENT APPLICATIONS}

\begin{nrtc}
\item	FILE TRANSFER, ACCESS AND MANAGEMENT (FTAM)

\item	MITRE FTAM-FTP GATEWAY

\item	DIRECTORY SERVICES (X.500) IN EARLY BETA

\item	ISODE MISCELLANY SERVICE
    \begin{nrtc}
    \item	e.g., FINGER, QUOTE-OF-THE-DAY, etc.
    \end{nrtc}

\item	PLUS NUMEROUS ``DEMO'' PROGRAMS
    \begin{nrtc}
    \item	e.g., IMAGE SERVICE, PASSWORD LOOKUP, etc.
    \end{nrtc}
\end{nrtc}
\end{bwslide}


\begin{bwslide}
\ctitle	{DIRECTORY SERVICES}

\begin{nrtc}
\item	THE UCL DIRECTORY, QUIPU, IS NEARING COMPLETION OF BETA DEVELOPMENT

\item	SEVERAL INTERESTING FEATURES:
    \begin{nrtc}
    \item	MEMORY, RATHER THAN DISK-BASED, ACCESS

    \item	INTERNAL SCHEDULING FOR MULTIPLE ACCESS

    \item	FLEXIBLE SEARCHING (SOUNDEX)

    \item	ACCESS CONTROL (NOT STANDARDIZED)
    \end{nrtc}
\end{nrtc}
\end{bwslide}


\begin{bwslide}
\ctitle	{DIRECTORY SERVICES (cont.)}

\begin{nrtc}
\item	FOR NAME/ADDRESS RESOLUTION, ISODE USES A 
    \begin{nrtc}
    \item	``HIGHER-PERFORMANCE'' NAMESERVICE
    \end{nrtc}
    BUILT ON TOP OF QUIPU SINCE
    \begin{nrtc}
    \item	CONNECTION-ORIENTED OVERHEAD AND

    \item	PROTOCOL COMPLEXITY
    \end{nrtc}
    ARE TOO HIGH FOR THE ``SIMPLE'' FUNCTIONALITY NEEDED BY MOST APPLICATIONS

\item	AT WOLLONGONG, CHRIS MOORE WILL BE HOSTING A PILOT PROJECT TO
	ACCELERATE DIRECTORY IMPLEMENTATION AND TESTING IN THE US
    \begin{nrtc}
    \item	ALSO, SITES IN THE UK AND AU WILL PARTICIPATE
    \end{nrtc}
\end{nrtc}
\end{bwslide}


\begin{bwslide}
\ctitle	{MESSAGE HANDLING SYSTEMS}

\begin{nrtc}
\item	UCL AND UNott ARE DEVELOPING AN X.400 TRANSPORT SYSTEM (PP)

\item	USE EXPERIENCE GAINED FROM NUMEROUS SOPHISTICATED TEXT-BASED MESSAGE
	TRANSFER SYSTEMS

\item	OWES MANY OF ITS DESIGN IDEAS TO THE UNIVERSITY OF DELAWARE MESSAGE
	SYSTEM, MMDF

\item	WILL UTILIZE DIRECTORY SERVICES
\end{nrtc}
\end{bwslide}


\begin{bwslide}
\ctitle	{INTERESTING FEATURES}

\begin{nrtc}
\item	SUPPORT FOR A WIDE RANGE OF ENCODED INFORMATION TYPES 
    \begin{nrtc}
    \item	AND REFORMATTING BETWEEN THEM
    \end{nrtc}

\item	SUPPORT FOR DIFFERENT MESSAGE TRANSPORT PROTOCOLS
    \begin{nrtc}
    \item	AND CONVERSION BETWEEN THEM
    \end{nrtc}
    e.g., INCLUDES RFC987 (X.400 TO 821/822)

\item	ROBUSTNESS FOR USE IN LARGE SCALE SERVICE ENVIRONMENTS
\end{nrtc}
\end{bwslide}


\begin{bwslide}
\ctitle	{MAJOR GOALS}

\begin{nrtc}
\item	FULL X.400(84/88) SUPPORT, EXCEPT FOR X.400(88) SECURITY SERVICES

\item	PROVIDES A ``CLEAN'' INTERFACE FOR MESSAGE SUBMISSION AND DELIVERY
    \begin{nrtc}
    \item	TO SUPPORT A WIDE RANGE OF USER AGENTS,

    \item	AND APPLICATIONS OTHER THAN INTERPERSONAL MESSAGING
    \end{nrtc}

\item	QUEUE MANAGEMENT DONE VIA A ROS-BASED PROTOCOL
    \begin{nrtc}
    \item	SOPHISTICATED SCHEDULING OF MESSAGE DELIVERY

    \item	LOCAL AND REMOTE MONITORING FOR MANAGERS AND USERS

    \item	ROBUSTNESS REQUIRED TO SUPPORT HIGH LEVELS OF TRAFFIC

    \item	SUPPORT FOR ADMINISTRATIVE POLICIES ON SUBMISSION
    \end{nrtc}

\item	LIST EXPLODER AND LIST MANAGMENT    
\end{nrtc}
\end{bwslide}


\begin{bwslide}
\ctitle	{VIRTUAL TERMINAL}

\begin{nrtc}
\item	MITRE HAS BEEN DEVELOPING A VT IMPLEMENTATION

\item	ROUGHLY EQUIVALENT TO BSD TELNET IN TERMS OF FUNCTIONALITY

\item	BEING INTEROPERABILITY TESTED AGAINST THE BRIDGE/3COM VT
\end{nrtc}
\end{bwslide}


\begin{bwslide}
\part*	{OSI-POSIX PROJECT}\bf

\begin{nrtc}
\item	GOAL: ACCELLERATE THE UBIQUITY OF OSI

\item	APPROACH: OPENLY AVAILABLE, COMPLETE OSI IMPLEMENTATION FOR NEXT MAJOR
	RELEASE OF BERKELEY \unix/

\item	FOR MORE DETAILS:
\begin{quote}
OSI PROTOCOLS WITHIN AN OPENLY AVAILABLE, POSIX-CONFORMANT, BERKELEY UNIX
ENVIRONMENT
\end{quote}
APPEARING IN ConneXions, OCTOBER, 1988
\end{nrtc}
\end{bwslide}


\begin{bwslide}
\diagram[p]{figure13}
\end{bwslide}


\begin{bwslide}
\diagram[p]{figure14}
\end{bwslide}


\begin{bwslide}
\part*	{ISODE 5.0}\bf

\begin{nrtc}
\item	WILL INCLUDE
    \begin{nrtc}
    \item	FILE TRANSFER, ACCESS AND MANAGEMENT: FINAL

    \item	FTAM-FTP GATEWAY: FINAL

    \item	DIRECTORY SERVICES: STABLE

    \item	MESSAGE HANDLING: EARLY BETA

    \item	VIRTUAL TERMINAL: NEARLY STABLE
    \end{nrtc}

\item	AVAILABLE MID-JANUARY, 1989!
\end{nrtc}
\end{bwslide}


\end{document}
