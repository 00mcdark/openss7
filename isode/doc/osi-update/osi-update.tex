% run this through SLiTeX

\documentstyle[blackandwhite,landscape,oval,pagenumbers,plain,small]{NRslides}

\input trademark 

\raggedright

\begin{document}

\title	{RECENT DEVELOPMENTS WITH\\ OSI IMPLEMENTATIONS}
\author	{Marshall T.~Rose\\The Wollongong Group}
\date	{January 13, 1988}
\maketitlepage


\begin{bwslide}
\part*	{AGENDA}\bf

\begin{description}
\item[PART I:]	OSI PROTOCOLS WITHIN AN OPENLY AVAILABLE, POSIX CONFORMANT,
		BERKELEY UNIX ENVIRONMENT

\item[PART II:]	EXPERIMENTAL OSI-BASED NETWORK

\item[PART III:]TRANSPORT-LEVEL BRIDGES
\end{description}
\end{bwslide}


\begin{bwslide}
\part	{OSI PROTOCOLS WITHIN AN\\
	OPENLY AVAILABLE\\ POSIX CONFORMANT\\ BERKELEY UNIX ENVIRONMENT}
\end{bwslide}


\begin{note}\em
if i've left out any 
\begin{quote}
buzzwords\\
jargon\\
marketing hype\\
\end{quote}
please let me know and i'll add them!

also, henceforth ``users'' means ``u.s.~govt. users''
\end{note}


\begin{bwslide}
\ctitle	{STANDARD DISCLAIMER}

\begin{nrtc}
\item	THE VIEWS PRESENTED HERE ARE NOT NECESSARILY THOSE OF:
\begin{quote}
TWG, DoD, MITRE, NBS, U.C.~BERKELEY, UCL, UWISC, THE U.S.~GOVT,
OR ANY OTHER ACRONYM, AGENCY, OR ORGANIZATION
\end{quote}

\item	I APOLOGIZE ONLY TO THOSE WHOM I HAVE UNINTENTIONALLY OFFENDED
\end{nrtc}
\end{bwslide}


\begin{bwslide}
\ctitle	{FUNDAMENTAL PREMISES:\\ NETWORKING}

\begin{nrtc}
\item	OSI/ISO WILL EVENTUALLY DOMINATE COMPUTER COMMUNICATIONS

\item	THE U.S. GOVERNMENT OSI PROFILE (GOSIP) WILL BE THE INITIAL SET OF
	GUIDELINES FOR PROCUREMENT OF OSI FOR USERS
\end{nrtc}
\end{bwslide}


\begin{bwslide}
\ctitle	{GOSIP}

\begin{nrtc}
\item	A (SOON-TO-BE) FEDERAL INFORMATION PROCESSING STANDARD

\item	PROPOSED TO ENABLE USERS TO SPECIFY AND PROCURE
	\begin{nrtc}
	\item	INTEROPERABLE

	\item	MULTI-VENDOR

	\item	OFF-THE-SHELF
	\end{nrtc}
	COMPUTER COMMUNICATIONS PRODUCTS

\item	THE \dod/:
    \begin{nrtc}
    \item	IS ADOPTING GOSIP AS A CO-STANDARD WITH TCP/IP

    \item	INTENDS (IN APPROX.~TWO YEARS) TO SPECIFY GOSIP AS THE 
		\underbar{ONLY} STANDARD FOR NON-PROPRIETARY, INTEROPERABLE
		COMPUTER COMMUNICATIONS
    \end{nrtc}
\end{nrtc}
\end{bwslide}


\begin{bwslide}
\ctitle	{FUNDAMENTAL PREMISES:\\ OPERATING SYSTEMS}

\begin{nrtc}
\item	THE \unix/ FAMILY WILL DOMINATE OPERATING SYSTEMS

\item	THE EMERGING IEEE \unix/-BASED PORTABLE OPERATING SYSTEM
	STANDARD (POSIX) WILL BE THE BASELINE FOR THESE SYSTEMS

\item	ANOTHER FIPS IS UNDER DEVELOPMENT TO BE THE INITIAL SET OF
	GUIDELINES FOR PROCUREMENT OF OPERATING SYSTEMS FOR USERS
\end{nrtc}
\end{bwslide}


\begin{bwslide}
\ctitle	{POSIX}

\begin{nrtc}
\item	CURRENTLY POSIX SPECIFIES ONLY THE \unix/ KERNEL INTERFACE
    \begin{nrtc}
    \item	INFLUENCED MOSTLY BY AT\&T \unix/ (SVID) WITH SOME BERKELEY
		ENHANCEMENTS
    \end{nrtc}

\item	WORK IS UNDERWAY ON A SHELL AND TOOLS STANDARD

\item	A STANDARD INTERFACE FOR NETWORKING IS NOTABLY MISSING
\end{nrtc}
\end{bwslide}


\begin{bwslide}
\ctitle	{A MODEST OBSERVATION}

\begin{nrtc}
\item	TCP/IP BECAME WIDESPREAD AFTER IT WAS INCLUDED IN BERKELEY \unix/

\item	QUESTIONS:
    \begin{nrtc}
    \item	CAN WE PUT A REFERENCE VERSION OF THE OSI PROTOCOLS INTO
		BERKELEY \unix/?

    \item	CAN WE MAKE BERKELEY \unix/ POSIX COMPLIANT?

    \item	CAN WE EXTEND POSIX TO DEFINE AN INTERFACE TO NETWORK SERVICES?

    \item	CAN WE MAKE THE WORK OPENLY AVAILABLE AND HAVE IT READY FOR
		4.4\bsd/~\unix/?
    \end{nrtc}

\item	ANSWER: YES

\item	THIS SHOULD RESULT IN ACCELERATING THE UBIQUITY OF OSI
\end{nrtc}
\end{bwslide}


\begin{bwslide}
\ctitle	{EXPLANATION}

\begin{nrtc}
\item	A LARGE NUMBER OF THE PIECES ARE ALREADY OPENLY AVAILABLE

\item	SO, THE WORK CONSISTS MAINLY OF:
    \begin{nrtc}
    \item	FILLING IN THE GAPS

    \item	INTEGRATING THE COMPONENTS

    \item	TESTING THE SYSTEM\\ (INTEROPERABILITY AND CONFORMANCE)
    \end{nrtc}

\item	THIS MODEST AMOUNT OF WORK SHOULD RESULT IN ACCELERATING THE UBIQUITY
	OF OSI
\end{nrtc}
\end{bwslide}


\begin{bwslide}
\ctitle	{APPROACH:\\ OSI PROTOCOLS}

\begin{nrtc}
\item	AN IMPLEMENTATION OF THE OSI UPPER-LAYERS (ISODE) IS ALREADY AVAILABLE

\item	OTHER ORGANIZATIONS HAVE DEVELOPED OR PLAN TO DEVELOP:
    \begin{nrtc}
    \item	THE LOWER LAYERS

    \item	SOME OSI APPLICATIONS
    \end{nrtc}

\item	MOST STANDARDS HAVE PROGRESSED FROM DRAFT (DIS) TO FINAL (IS) STATUS
\end{nrtc}
\end{bwslide}


\begin{bwslide}
\diagram[p]{figure1}
\end{bwslide}


\begin{bwslide}
\diagram[p]{figure2}
\end{bwslide}


\begin{bwslide}
\ctitle	{THE WORK PLAN}

\begin{nrtc}
\item	UPGRADE ISODE AND OTHER OSI APPLICATIONS TO FINAL (IS) STATUS

\item	INTEGRATE OTHER OSI APPLICATIONS INTO ISODE

\item	PERFORM INTEROPERABILITY TESTING ON OSInet

\item	PERFORM CONFORMANCE TESTING WITH COS
\end{nrtc}
\end{bwslide}


\begin{bwslide}
\ctitle	{APPROACH:\\ POSIX COMPLIANCE}

\begin{nrtc}
\item	MINOR WORK TO MODIFY THE BERKELEY \unix/ KERNEL TO SUPPORT THE POSIX
	STANDARD

\item	PERFORM CONFORMANCE TESTING WITH NBS

\item	ISODE AND OSI APPLICATIONS WILL BE CONVERTED TO USE THE POSIX
	INTERFACE AS APPLICABLE
\end{nrtc}
\end{bwslide}


\begin{bwslide}
\ctitle	{APPROACH:\\ POSIX NETWORK SERVICE}

\begin{nrtc}
\item	A /usr/group COMMITTEE WAS FORMED OVER A YEAR AGO

\item	U.C.~BERKELEY (AND FRIENDS) WILL EXAMINE THE OUTPUT OF THIS
	GROUP AND EITHER:
    \begin{nrtc}
    \item	ADOPT THIS INTERFACE (IF ACCEPTED BY POSIX), OR

    \item	SUBMIT A NEW DRAFT PROPOSAL TO THE POSIX COMMITEE
    \end{nrtc}
\end{nrtc}
\end{bwslide}


\begin{bwslide}
\ctitle	{SCHEDULE}

\begin{nrtc}
\item	WOULD YOU BELIEVE 18~CALENDAR-MONTHS?

\item	ACTUALLY 120~MAN-MONTHS%
	\footnote{You may have read Brooks' {\em The Mythical Man-Month}.}
\end{nrtc}
\end{bwslide}


\begin{bwslide}
\part	{EXPERIMENTAL\\ OSI-BASED NETWORK}
\end{bwslide}


\begin{bwslide}
\ctitle	{MOTIVATION}

\begin{nrtc}
\item	GOAL: WANT TO SPEED DEVELOPMENT OF AND EXPERIMENTATION WITH
	LOWER-LAYER ISO PROTOCOLS, e.g.,
    \begin{nrtc}
    \item	TP4, CLNP, ES-IS

    \item	IS-IS
    \end{nrtc}

\item	AND WITH NETWORK MANAGEMENT, e.g., NETWORK LAYER SUPPORT FOR CMIS


\item	ASIDE: IN ADDITION TO AREAS SUCH AS PERFORMANCE TUNING, etc.,
	ALSO INTERESTED IN PROMOTING INTEROPERABILITY TESTING AMONGST
	VARIOUS IMPLEMENTATIONS
\end{nrtc}
\end{bwslide}


\begin{bwslide}
\ctitle	{REQUIREMENTS}

\begin{nrtc}
\item	A ``TYPICAL'' DATAGRAM SERVICE
    \begin{nrtc}
    \item	POSSIBLE PACKET LOSS, CORRUPTION, DUPLICATION, AND
		RE-ORDERING, etc.
    \end{nrtc}

\item	OFFERED OVER A HETEROGENEOUS COLLECTION OF SUBNETS
    \begin{nrtc}
    \item	MULTIPLE PATHS, VARYING LINK AND MEDIA CHARACTERISTICS, etc.
    \end{nrtc}

\item	WHICH IS WELL-USED (OVER-SUBSCRIBED).
    \begin{nrtc}
    \item	CONGESTION, VARIABLE DELAY,  etc.
    \end{nrtc}

\item	IN SHORT, WE WANT A RICH LOWER-LAYER INFRASTRUCTURE
    \begin{nrtc}
    \item	e.g., A NATIONAL CLNP-BASED INTERNET
    \end{nrtc}
\end{nrtc}
\end{bwslide}


\begin{bwslide}
\ctitle	{OBSERVATION}

\begin{nrtc}
\item	WHERE HAVE WE SEEN ONE OF THOSE?
    \begin{nrtc}
    \item	$\ldots$ THE DARPA/NSF INTERNET!
    \end{nrtc}

\item	THE INTERNET MEETS ALL THE REQUIREMENTS BUT ONE:
    \begin{nrtc}
    \item	IT IS IP-BASED RATHER THAN CLNP-BASED
    \end{nrtc}

\item	SO, WHAT IS NEEDED IS A WAY TO EMULATE A CLNP-BASED NETWORK
	ON TOP OF THE EXISTING DARPA/NSF INTERNET
\end{nrtc}
\end{bwslide}


\begin{bwslide}
\ctitle	{EON:\\ AN EXPERIMENTAL\\ OSI-BASED NETWORK}

\begin{nrtc}
\item	AN RFC HAS BEEN SUBMITTED BY UWISC AND TWG DESCRIBING:
    \begin{nrtc}\em
    \item	USE OF THE DARPA/NSF INTERNET AS A SUBNETWORK FOR
		EXPERIMENTATION WITH THE OSI NETWORK LAYER
    \end{nrtc}

\item	PARTICIPATING IP-NODES FORM A LOGICAL ISO NETWORK
    \begin{nrtc}
    \item	A NODE PARTICIPATES AS AN IS, ES, OR BOTH

    \item	SEVERAL LOGICAL ISO SUBNETS CAN EXIST ON THE DARPA/NSF INTERNET
    \end{nrtc}

\item	IT IS NON-DESTRUCTIVE IN THE SENSE THAT IT DOES NOT AFFECT THE
	EXISTING IP-BASED CONNECTIVITY (CORE GATEWAYS, etc.)
\end{nrtc}
\end{bwslide}


\begin{bwslide}
\ctitle	{EON DEFINES PROCEDURES FOR}

\begin{nrtc}
\item	ENCAPSULATION OF NPDUs

\item	FORMATION AND MAPPING OF SNPA-ADDRESSES

\item	USE OF SUBNET MULTICASTING IN CLNL

\item	DISSEMINATION OF TOPOLOGICAL INFORMATION	
\end{nrtc}
\end{bwslide}


\begin{bwslide}
\ctitle	{SCHEDULE}

\begin{nrtc}
\item	EON IS NEW, THE RFC, ALTHOUGH SUBMITTED, HASN'T BEEN RELEASED YET
	
\item	BUT, BY APRIL, UWISC AND TWG EXPECT TO BE PARTICIPATING IN INTERNET
	EXPERIMENTS
\end{nrtc}
\end{bwslide}



\begin{bwslide}
\part	{TRANSPORT-LEVEL BRIDGES}
\end{bwslide}


\begin{bwslide}
\ctitle	{MOTIVATION}

\begin{nrtc}
\item	THERE ARE MANY TCP/IP NETWORKS TODAY, THERE WILL BE MORE TOMMORROW

\item	BY THE TIME OSI/OSI BECOMES A WORTHWHILE OPERATION ALTERNATIVE,
	THERE WILL BE MANY MORE TCP/IP NETWORKS THAN THERE ARE TODAY!

\item	PREDICTION: AT THAT TIME, TCP/IP NETWORKS WILL
	OFFER A MIX OF SERVICES:
    \begin{nrtc}
    \item	SUCH AS FTAM AND X.400, IN ADDITION TO FTP AND SMTP
    \end{nrtc}

\item	FURTHER PREDICATION: THIS MIX WILL PROLIFERATE TO PERMEATE
	BOTH TCP/IP AND OSI/ISO NETWORKS
\end{nrtc}
\end{bwslide}


\begin{bwslide}
\ctitle	{OBSERVATION}

\begin{nrtc}
\item	GIVEN THE ASSUMPTION ABOVE, IT SHOULD BE NOTED THAT:
    \begin{nrtc}
    \item	THE TWO COMMUNITIES ARE USING THE SAME APPLICATIONS,
		AND

    \item	ONLY THE UNDERLYING NETWORK AND TRANSPORT PROTOCOLS WILL
		DIFFER BETWEEN THE TWO
    \end{nrtc}

\item	THIS LEADS US TO POSTULATE AN INTERESTING COEXISTENCE
	STRATEGY:
    \begin{nrtc}
    \item	LET'S RUN ISO APPLICATIONS BETWEEN THE TWO COMMUNITIES
    \end{nrtc}
\end{nrtc}
\end{bwslide}


\begin{bwslide}
\ctitle	{TRANSPORT-LEVEL BRIDGES}

\begin{nrtc}
\item	IDEA: OFFER THE SAME TRANSPORT SERVICE INTERFACE IN BOTH
	COMMUNITIES (THE ISO TRANSPORT SERVICE)
    \begin{nrtc}
    \item	USE RFC1006 TO OFFER THE ISO TRANSPORT SERVICE ON TOP OF
		THE TCP
    \end{nrtc}

\item	INTRODUCE A TRANSPORT ENTITY CALLED THE ``TS-BRIDGE''

\item	THE TS-BRIDGE ``COPIES'' SERVICE PRIMITIVES FROM ONE COMMUNITY TO THE
	OTHER, e.g.,
    \begin{nrtc}
    \item	UPON RECEIVING A T-CONNECT.INDICATION PRIMITIVE FROM ONE
		NETWORK,

    \item	IT ISSUES A T-CONNECT.REQUEST PRIMITIVE TO THE OTHER NETWORK
    \end{nrtc}

\item	THE TS-BRIDGE MAINTAINS STATE AS TO THE EXISTING CONNECTIONS
	(AND AS SUCH IS A SINGLE POINT OF FAILURE)
\end{nrtc}
\end{bwslide}


\begin{note}\em
perhaps these should be called ``gateways'' instead of ``ts-bridges''

well, they are *so* simple that ``gateway'' really seems to be an overloaded
term in this circumstance$\ldots$

in fact, simplicity is one reason why this approach was chosen over a
network-level solution
\end{note}


\begin{bwslide}
\ctitle	{TRANSPARENT USE OF TS-BRIDGES}

\begin{nrtc}
\item	BY JUDICIOUS USE OF DIRECTORY SERVICES, SELECTION OF THE
	TS-BRIDGE CAN BE MADE TRANSPARENT ON BOTH ENDPOINTS

\item	CONSIDER A ``TYPICAL'' PRESENTATION ADDRESS:
\[\begin{tabular}{ll}
network address:&	CLNP 4700050017000008002000405301\\
transport selector:&	1\\
session selector:&	``FTAM''\\
presentation selector:&	null
\end{tabular}\]

\item	A SLIGHTLY DIFFERENT ENTRY IS RETURNED FOR HOSTS IN THE
	OPPOSITE COMMUNITY:
\[\begin{tabular}{ll}
network address:&	ts-bridge's network address\\
transport selector:&	\begin{tabular}[t]{ll}
			network address:&
				CLNP 47 $\ldots$\\
			transport selector:&	 1
			\end{tabular}\\
session selector:&	``FTAM''\\
presentation selector:&	null
\end{tabular}\]
\end{nrtc}
\end{bwslide}


\begin{bwslide}
\ctitle	{ANOTHER PROBLEM SOLVED:\\ ISO CONS versus CLNS}

\begin{nrtc}
\item	IN GENERAL, THE TS-BRIDGE SHOWS HOW TO PERFORM
	``IMPEDANCE MATCHING'' BETWEEN TWO PROTOCOLS WHICH OFFER THE
	SAME SERVICE INTERFACE, e.g., OUR USE IS:
    \begin{nrtc}
    \item	PROTOCOLS: TP4/CLNP AND TP0/TCP

    \item	SERVICE: ISO TRANSPORT SERVICE
    \end{nrtc}

\item	THIS IS SUSPICIOUSLY SIMILAR TO THE ISO CONS vs. CLNS PROBLEM:
    \begin{nrtc}
    \item	PROTOCOLS: TP4/CLNP AND TP0/X.25

    \item	SERVICE: ISO TRANSPORT SERVICE
    \end{nrtc}

\item	THE TS-BRIDGE WILL ALSO WORK IN THIS ENVIRONMENT WITHOUT
	MEANINGFUL LOSS OF GENERALITY:
    \begin{nrtc}
    \item	EXPEDITED DATA IS NEGOTIATED AWAY, AND

    \item	USER DATA ON CONNECTION PRIMITIVE IS DISREGARDED
    \end{nrtc}
\end{nrtc}
\end{bwslide}


\end{document}
