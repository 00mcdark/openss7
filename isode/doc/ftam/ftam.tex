% run this through SLiTeX

\documentstyle
    [blackandwhite,landscape,oval,pagenumbers,plain,small,tgrind]{NRslides}

\makeatletter
\def\@maketitle{%
    \newpage
    \null
    \setbox\z@=\vbox{%
	\ \vskip .75em
	\begin{center}
	    {\huge\bf \@title \par}%	%%% was \Large
	    \vskip .5em
	    {\Large\bf			%%% was \large
		\lineskip=.25em 
		\begin{tabular}[t]{c}
		    \@author 
		\end{tabular}
		\par
	    }%
	    \vskip .25em		%%% was .5em
	    {\Large\bf \@date}%		%%% was \large
	\end{center}
	\par
	\vskip .75em
    }%
    \if@ovaltitle
	\title@ht=\ht\z@	\title@wd=\wd\z@
	\title@@ht=\ht\z@	\title@@wd=\wd\z@
	\divide\title@@ht by2	\divide\title@@wd by2
	\unitlength=1sp
    \fi
    \box\z@
    \if@ovaltitle
	\vskip -\title@ht\unitlength
	{\centering
	    \begin{picture}(\title@wd,\title@ht)
		\put(\title@@wd,\title@@ht){\oval(\title@wd,\title@ht)}
	    \end{picture}
	\par}
    \fi
}
\makeatother

\raggedright

\begin{document}

\title	{FILE TRANSFER, ACCESS\\ AND MANAGEMENT}
\author	{Marshall T.~Rose}
\date	{September 1, 1987}
\maketitlepage


\begin{bwslide}
\ctitle	{INTRODUCTION}

\begin{nrtc}
\item	FTAM - FILE TRANSFER, ACCESS AND MANAGEMENT

\item	THE OPEN SYSTEMS FILE SERVICE

\item	CURRENTLY A {\bf DRAFT INTERNATIONAL STANDARD}\\
	SOON TO ACHIEVE FULL STANDARD STATUS
\end{nrtc}
\end{bwslide}


\begin{bwslide}
\ctitle	{THE OPEN SYSTEMS FILE SERVICE}

\begin{nrtc}
\item	NOT ``JUST'' FILE TRANSFER

\item	THE BASIC BUILDING BLOCK FOR OSI
    \begin{nrtc}
    \item	FILESTORE TO FILESTORE TRANSFER

    \item	WORKSTATION FILE RETRIEVAL

    \item	DISKLESS WORKSTATION PROTOCOL

    \item	SPECIAL APPLICATIONS (e.g., PRINTING, SPOOLING)

    \item	REMOTE DATABASE ACCESS
    \end{nrtc}
\end{nrtc}
\end{bwslide}


\begin{note}\em
as one of the OSI applications,
FTAM is exciting
as we get to see how all of the lower layer services are used

because of its scope,
FTAM is challenging
as it has a very large operating charter
(simple file transfer, database access, remote filesystems, and so on)

it is a largish undertaking:
the spec is about 350~pages,
the ISODE implementation (which isn't complete) is about 40K~lines of C code
(about 25\% machine-generated, thank goodness)
\end{note}


\begin{note}\em
an audience survey, who has:

heard of FTAM?

read the FTAM specification?

*understood* the FTAM specification?

heard of association control (sometimes incorrectly called CASE)?

heard of unix and the C programming language?
\end{note}


\begin{bwslide}
\part*	{OUTLINE}

\begin{nrtc}\bf
\item	THE VIRTUAL FILESTORE

\item	THE FILE SERVICE

\item	THE FILE PROTOCOL

\item	ISSUES IN IMPLEMENTING THE VIRTUAL FILESTORE

\item	ISSUES IN IMPLEMENTING A CLIENT OF THE VIRTUAL FILESTORE

\item	FTAM STATUS
\end{nrtc}
\end{bwslide}

% run this through SLiTeX with the appropriate wrapper

\begin{bwslide}
\part	{THE VIRTUAL FILESTORE}

\begin{nrtc}\bf
\item	PHILOSOPHY

\item	FILE ATTRIBUTES

\item	ACTIVITY ATTRIBUTES

\item	DOCUMENT TYPES
\end{nrtc}
\end{bwslide}


\begin{note}\em
this section corresponds roughly to iso/dis 8571/2,
but the concepts are explained in almost an entirely different way

this description is done via successive refinement:
concepts are introduced and continuously expanded

hopefully,
this is less intimidating than the way the standard presents things$\ldots$
\end{note}


\begin{bwslide}
\part*	{PHILOSOPHY}\bf

\begin{nrtc}
\item	AS WITH ALL ``OPEN SYSTEM'' SERVICES
    \begin{nrtc}
    \item	DESCRIBES A CONCEPTUAL MODEL OF THE VIRTUAL SERVICE

    \item	SPECIFIES THE SERVICE AND THE PROTOCOL\\
		INDEPENDENT OF ACTUAL LOCAL SYSTEMS
	\begin{nrtc}
	\item	PROGRAMATIC INTERFACES ARE NOT SPECIFIED
	\end{nrtc}
    \end{nrtc}

\item	THE FUNDAMENTAL ABSTRACTION: THE VIRTUAL FILESTORE

\item	A CONCEPTUAL MODEL OF A FILE SERVICE ON A LOCAL SYSTEM (LOCALSTORE)

\item	DIFFICULT TASK~---~EXISTING FILE SERVICES ARE QUITE DIFFERENT

\item	POTENTIALLY VERY REWARDING!
\end{nrtc}
\end{bwslide}


\begin{bwslide}
\ctitle	{RELATIONSHIP OF THE VIRTUAL FILESTORE\\ AND LOCALSTORE}

\vskip.5in
\diagram[p]{figure1}
\end{bwslide}


\begin{note}\em
why a virtual filestore?

it is unacceptable to choose any existing real filestore as the basis
for the file service (all lack one thing or another)

hence, it is desirable to devise a model which can reasonably express any
existing real filestore.
\end{note}


\begin{bwslide}
\ctitle	{ELEMENTS}\bf

\begin{nrtc}
\item	A (VIRTUAL) FILESTORE IS A COLLECTION OF FILES

\item	A FILENAME IDENTIFIES EXACTLY ONE FILE IN THE FILESTORE

\item	THERE IS NO EXPLICIT RELATIONSHIP BETWEEN DIFFERENT FILES IN THE
	FILESTORE
    \begin{nrtc}
    \item	i.e., NO DIRECTORY STRUCTURE (A {\bf BIG} MISTAKE)
    \end{nrtc}

\item	FILES HAVE
    \begin{nrtc}
    \item	ATTRIBUTES (e.g., OWNERSHIP INFORMATION)

    \item	CONTENTS (e.g., RANDOM-ACCESS RECORDS)
    \end{nrtc}
\end{nrtc}
\end{bwslide}


\begin{bwslide}
\ctitle	{ELEMENTS -- ATTRIBUTES}

\begin{nrtc}
\item	TWO KINDS OF ATTRIBUTES ARE DEFINED

\item	FILE ATTRIBUTES, WHICH EXIST ON A PER-FILE BASIS
    \begin{nrtc}
        \item	SIMULTANEOUS CLIENTS OF THE FILESTORE SEE THE SAME INFORMATION

	\item	e.g., THE NAME OF THE FILE
    \end{nrtc}

\item	ACTIVITY ATTRIBUTES, WHICH EXIST ON A PER-CLIENT BASIS
    \begin{nrtc}
    \item	INTERACTIONS BY A CLIENT ARE NOT DIRECTLY VISIBLE TO OTHER
		CLIENTS

    \item	e.g., HOW THE FILE IS BEING TRAVERSED
    \end{nrtc}

\item	THE CLIENT INTERACTS ON AT MOST ONE FILE
    \begin{nrtc}
    \item	THE ``SELECTED'' FILE
    \end{nrtc}
\end{nrtc}
\end{bwslide}


\begin{bwslide}
\ctitle	{ELEMENTS -- CONTENTS}

\begin{nrtc}
\item	TYPICALLY, FILES ARE DEFINED IN TERMS OF A ``DOCUMENT TYPE''

\item	STATIC CHARACTERISTICS
    \begin{nrtc}
    \item	THE COMPOSITION OF THE FILE IN TERMS OF FILE ACCESS DATA
		UNITS (FADUs)\\
		e.g., A SEQUENTIAL COLLECTION OF RECORDS

    \item	THE STRUCTURE OF EACH DATA UNIT (DUs)\\
		e.g., EACH RECORD CONTAINS A PERSONNEL RECORD
    \end{nrtc}

\item	DYNAMIC CHARACTERISTICS
    \begin{nrtc}
    \item	HOW DATA UNITS ARE ENCODED ON THE NETWORK

    \item	HOW DATA UNITS ARE REFERENCED (e.g., CURRENT POSITION)
    \end{nrtc}
\end{nrtc}
\end{bwslide}


\begin{bwslide}
\part*	{FILE ATTRIBUTES}\bf

\begin{nrtc}
\item	FOUR GROUPS OF FILE ATTRIBUTES

\item	KERNEL GROUP (REQUIRED)
    \begin{nrtc}
    \item	NECESSARY FOR FILE SELECTION AND BASIC FILE TRANSFER
    \end{nrtc}

\item	STORAGE GROUP (OPTIONAL)
    \begin{nrtc}
    \item	DESCRIBES THE STORAGE CHARACTERISTICS FOR THE FILE
    \end{nrtc}

\item	SECURITY GROUP (OPTIONAL)
    \begin{nrtc}
    \item	DESCRIBES THE ACCESS CONTROL MECHANISMS FOR THE FILE
    \end{nrtc}

\item	PRIVATE GROUP (OPTIONAL)
    \begin{nrtc}
    \item	A MECHANISM TO CAPTURE NON-STANDARD (PROPRIETARY)
		MECHANISMS THAT CAN NOT BE OTHERWISE REPRESENTED
    \end{nrtc}
\end{nrtc}
\end{bwslide}


\begin{note}\em
definitions of types is rather loose at this point in the presentation;
e.g., ``string'' is usually an asn.1 graphicstring

the emphasis at the moment is on the concept,
not on the actual abstract data type
\end{note}


\begin{bwslide}
\ctitle	{KERNEL GROUP}

\begin{nrtc}
\item	FILENAME: A SEQUENCE OF STRINGS
    \begin{nrtc}
    \item	MAPPING TO THE LOCALSTORE NAMING CONVENTIONS IS A
		``LOCAL IMPLEMENTATION CHOICE''
    \end{nrtc}

\item	CONTENTS TYPE: STRUCTURING INFORMATION
    \begin{nrtc}
    \item	THE FILE STRUCTURE (A {\bf LOT} MORE LATER)
    \end{nrtc}
	
\end{nrtc}
\end{bwslide}


\begin{bwslide}
\ctitle	{STORAGE GROUP}

\begin{nrtc}
\item	STORAGE ACCOUNT: A STRING
    \begin{nrtc}
    \item	ENTITY ACCRUING FILE STORAGE CHARGES
    \end{nrtc}	

\item	IDENTITY OF USER AND THE DATE/TIME OF
    \begin{nrtc}
    \item	FILE CREATION

    \item	LAST READ AND LAST MODIFICATION OF FILE CONTENTS

    \item	LAST MODIFICATION OF FILE ATTRIBUTES
    \end{nrtc}

\item	FILE AVAILABILITY
    \begin{nrtc}
    \item	IMMEDIATE (FILE IS ``ON-LINE'')

    \item	DEFFERRED (ACCESS TO FILE MAY ENCOUNTER DELAY,
		e.g., AWAITING ARCHIVE RETRIEVAL)
    \end{nrtc}
\end{nrtc}
\end{bwslide}


\begin{bwslide}
\ctitle	{STORAGE GROUP (cont.)}

\begin{nrtc}
\item	PERMITTED ACTIONS
    \begin{nrtc}
    \item	DESCRIBES THE TYPES OF DATA ACCESS THAT CAN BE PERFORMED ON
		THE FILE

    \item	HOW DATA UNITS MAY BE ACCESSED
		(READ, WRITE, EXTEND, etc.)

    \item	HOW THE FILE MAY BE TRAVERSED
		(MOVING FROM ONE DATA UNIT TO ANOTHER)
    \end{nrtc}

\item	FILESIZE (IN OCTETS)
    \begin{nrtc}
    \item	AN ESTIMATE OF THE TOTAL SIZE OF THE FILE'S CONTENTS
    \end{nrtc}
	

\item	FUTURE FILESIZE (IN OCTETS)
    \begin{nrtc}
    \item	A SOFT LIMIT ON THE TOTAL SIZE OF THE FILE'S CONTENTS
    \end{nrtc}
\end{nrtc}
\end{bwslide}


\begin{bwslide}
\ctitle	{SECURITY GROUP}

\begin{nrtc}
\item	ACCESS CONTROL (AN ACCESS CONTROL LIST)\\
	FOR EACH ELEMENT OF THE LIST:
    \begin{nrtc}
    \item	FILE ACTIONS PERMITTED

    \item	ENTITY PERMITTED TO REQUEST ACTION (OPTIONAL)

    \item	PASSWORD REQUIRED TO VALIDATE ACTION
    \end{nrtc}

\item	ENCRYPTION NAME
    \begin{nrtc}
    \item	DEFINES HOW FILE WAS ENCRYPTED

    \item	FILES ARE TRANSFERRED IN ENCRYPTED FORM

    \item	REQUIRES A REGISTRATION AUTHORITY TO BE ESTABLISHED
    \end{nrtc}

\item	LEGAL QUALIFICATIONS
    \begin{nrtc}
    \item	DEFINES THE ``LEGAL STATUS'' OF THE FILE

    \item	MEANT TO BE USED WITH A NATIONAL PRIVACY LEGISLATION
    \end{nrtc}
\end{nrtc}
\end{bwslide}


\begin{bwslide}
\ctitle	{PRIVATE GROUP}

\begin{nrtc}
\item	A ``CATCH-ALL''

\item	USE IS STRONGLY DISCOURAGED
\end{nrtc}
\end{bwslide}


\begin{bwslide}
\part*	{ACTIVITY ATTRIBUTES}\bf

\begin{nrtc}
\item	ACTIVITY ATTRIBUTES ARE ALSO DEFINED IN TERMS OF GROUPS\\
	KERNEL, STORAGE, AND SECURITY (NO PRIVATE GROUP, OBVIOUSLY)

\item	THESE ARE USUALLY INITIALIZED WHEN A FILE IS EITHER
    \begin{nrtc}
    \item	SELECTED

    \item	OPENED FOR TRANSFER/ACCESS
    \end{nrtc}
\end{nrtc}
\end{bwslide}


\begin{note}\em
relationship of actions is rather loose at this point in the presentation;
selection, open, transfer/access

the next section will formalize these terms
\end{note}


\begin{bwslide}
\ctitle	{KERNEL GROUP ACTIVITY ATTRIBUTES}

\begin{nrtc}
\item	ACTIVE CONTENTS TYPE
    \begin{nrtc}
    \item	THE CONTENTS TYPE
    \end{nrtc}

\item	CURRENT ACCESS REQUEST
    \begin{nrtc}
    \item	THOSE PERMITTED ACTIONS WHICH ARE REQUESTED BY THE CLIENT

    \item	CONTENTS: READ, INSERT, REPLACE, EXTEND, ERASE

    \item	ATTRIBUTES: READ, CHANGE, AND DELETE FILE
    \end{nrtc}

\item	CURRENT LOCATION

\item	CURRENT PROCESSING MODE
    \begin{nrtc}
    \item	ACTIONS ON THE CONTENTS WHICH THE CLIENT WISHES TO PERFORM
		(READ, INSERT, REPLACE, EXTEND, ERASE, LOCATE)
    \end{nrtc}
	
\item	CURRENT APPLICATION ENTITY TITLE
    \begin{nrtc}
    \item	A GLOBAL IDENTIFIER FOR THE ENTITY PROVIDING THE FILE SERVICE
		(A FILESTORE OR FILESERVER)
    \end{nrtc}
\end{nrtc}
\end{bwslide}


\begin{bwslide}
\ctitle	{STORAGE GROUP ACTIVITY ATTRIBUTES}

\begin{nrtc}
\item	CURRENT ACCOUNT
    \begin{nrtc}
    \item	THE CLIENT'S ACCOUNT WHEN THE FILE SERVICE WAS INITIATED
		(MAY BE CHANGED WHEN A FILE IS SELECTED)
    \end{nrtc}

\item	CURRENT ACCESS CONTEXT
    \begin{nrtc}
    \item	HOW THE FILE STRUCTURE IS COMMUNICATED ON THE NETWORK
		(MUCH MORE ON THIS LATER)
    \end{nrtc}

\item	CURRENT CONCURRENCY CONTROL
    \begin{nrtc}
    \item	HOW SIMULTANEOUS CLIENTS INTERACT WHEN ACCESSING THE FILE

    \item	FOR EACH ACTION: SHARED, EXCLUSIVE, NOT REQUIRED, NO ACCESS
    \end{nrtc}
\end{nrtc}
\end{bwslide}


\begin{bwslide}
\ctitle	{SECURITY GROUP ACTIVITY ATTRIBUTES}

\begin{nrtc}
\item	ACTIVE LEGAL QUALIFICATION

\item	CURRENT INITIATOR IDENTITY
    \begin{nrtc}
    \item	THE CLIENT'S IDENTITY WHEN THE FILE SERVICE WAS INITIATED
    \end{nrtc}

\item	CURRENT ACCESS PASSWORDS
    \begin{nrtc}
    \item	THE ACCESS LIST WHICH APPLIES TO THE CLIENT
    \end{nrtc}
\end{nrtc}
\end{bwslide}


\begin{bwslide}
\part*	{DOCUMENT TYPES}\bf

\begin{nrtc}
\item	STATIC CHARACTERISTICS
    \begin{nrtc}
    \item	THE FILE ACCESS STRUCTURE (CONSTRAINT SET NAME)

    \item	THE PRESENTATION STRUCTURE (ABSTRACT SYNTAX NAME)
    \end{nrtc}

\item	DYNAMIC CHARACTERISTICS
    \begin{nrtc}
    \item	THE TRANSFER STRUCTURE (TRANSFER SYNTAX NAME)

    \item	A IDENTIFICATION STRUCTURE (ACCESS CONTEXTS)
    \end{nrtc}

\item	``REGISTERED'' AND REFERENCED VIA A UNIQUE IDENTIFIER
\end{nrtc}
\end{bwslide}


\begin{bwslide}
\ctitle	{FILE ACCESS STRUCTURE}

\begin{nrtc}
\item	ANY FILE'S CONTENT CAN BE DESCRIBED AS A TREE

\item	EACH NODE IN THE TREE CONTAINS
    \begin{nrtc}
    \item	A DESCRIPTOR (A NAME AND DISTANCE TO PARENT)

    \item	OPTIONALLY, A DATA UNIT (DEFINED BY THE PRESENTATION STRUCTURE)

    \item	OPTIONALLY, CHILDREN (OTHER NODES)
    \end{nrtc}

\item	THE ROOT NODE DEFINES THE ``STARTING'' POINT FOR THE FILE

\item	NEED A WAY TO LIMIT THE COMPLEXITY OF THE TREE
\end{nrtc}
\end{bwslide}


\begin{bwslide}
\ctitle	{CONSTRAINT SETS}

\begin{nrtc}
\item	DEFINES THE STRUCTURE OF THE TREE AND HOW ACTIONS ON THE FILE
	(e.g., WRITE, ERASE) CHANGE THE STRUCTURE AND POSITION

\item	SEVERAL KINDS
    \begin{nrtc}
    \item	UNSTRUCTURED

    \item	SEQUENTIAL FLAT

    \item	ORDERED FLAT

    \item	ORDERED FLAT WITH UNIQUE NAMES

    \item	ORDERED HIERARCHICAL

    \item	GENERAL HIERARCHICAL

    \item	GENERAL HIERARCHICAL WITH UNIQUE NAMES
    \end{nrtc}
\end{nrtc}
\end{bwslide}


\begin{bwslide}
\ctitle	{EXAMPLE: UNSTRUCTURED CONSTRAINT SET}

\vskip.5in
\diagram[p]{figure2}
\end{bwslide}


\begin{note}\em
a unnamed root node with a data unit but no children

file can be transferred as a whole, or extended

access to a part is not permitted
\end{note}


\begin{bwslide}
\ctitle	{EXAMPLE: SEQUENTIAL FLAT CONSTRAINT SET}

\vskip.5in
\diagram[p]{figure3}
\end{bwslide}


\begin{note}\em
a two-level tree:\\
a root with no data unit but with zero or more children;
and,
each child has a data unit but no children

data unit is identified based on position in the file (relation to siblings)

insertions occur at end of file

erase at root node to empty file

ordered-flat differs by naming each child and identifying data units based
on the name
\end{note}


\begin{bwslide}
\ctitle	{EXAMPLE: GENERAL HIERARCHICAL CONSTRAINT SET}

\vskip.5in
\diagram[p]{figure4}
\end{bwslide}


\begin{note}\em
hierarchical: a tree of arbitrary structure

at a given level, nodes have the same ``type'' of name

insert as sister (sibling):\\
the data unit becomes the next child visited when using preorder traversal
(not valid for the root node, obviously)

insert as child:\\
the data unit becomes the first child visited when using preorder traversal

note difference between fadu and data(du)
\end{note}


\begin{bwoverlay}
\ctitle	{EXAMPLE: GENERAL HIERARCHICAL CONSTRAINT SET}

\vskip.5in
\diagram[p]{figure4a}
\end{bwoverlay}


\begin{bwoverlay}
\ctitle	{EXAMPLE: GENERAL HIERARCHICAL CONSTRAINT SET}

\vskip.5in
\diagram[p]{figure4b}
\end{bwoverlay}


\begin{bwoverlay}
\ctitle	{EXAMPLE: GENERAL HIERARCHICAL CONSTRAINT SET}

\vskip.5in
\diagram[p]{figure4c}
\end{bwoverlay}


\begin{bwslide}
\ctitle	{PRESENTATION STRUCTURE}

\begin{nrtc}
\item	STRUCTURE OF EACH DATA UNIT IS DEFINED USING ABSTRACT SYNTAX NOTATION
	ONE (ASN.1)

\item	SPECIFICATION CAN BE SIMPLE, e.g., A STRING OF OCTETS

\item	OR COMPLEX, e.g., A PERSONNEL RECORD
\end{nrtc}
\end{bwslide}


\begin{bwslide}
\ctitle	{TRANSFER STRUCTURE}

\begin{nrtc}
\item	DATA UNITS ARE COMPOSED OF ``DATA ELEMENTS''

\item	EACH DATA ELEMENT MAPS DIRECTLY TO A ``WRITE'' TO THE NETWORK

\item	A TRANSFER STRUCTURE IS SAID TO BE ``SELF-DELIMITING'' IF EACH
	DATA UNIT MAPS TO EXACTLY ONE DATA ELEMENT

\item	OTHERWISE, A 1:n RATIO IS USED AS AN EFFICIENCY ``HACK''
    \begin{nrtc}
    \item	DATA ELEMENTS ARE CONCATENATED TO FORM A SINGLE DATA UNIT
    \end{nrtc}
\end{nrtc}
\end{bwslide}


\begin{bwslide}
\ctitle	{IDENTIFICATION STRUCTURE}

\begin{nrtc}
\item	ACCESS CONTEXT:
	AN ALGORITHM FOR DEFINING A ASPECT OF THE FILE STRUCTURE

\item	ACTIONS TAKEN IN THE CONTEXT OF THE CURRENT POSITION\\ (i.e., NODE)

\item	RECURSIVE ACTIONS PERFORMED IN PREORDER TRAVERSAL

\item	IMPORTANT DISTINCTION
    \begin{nrtc}
    \item	ALL DATA UNITS~---~THE NODE'S DATA UNIT AND DATA BELONGING
		TO ALL CHILDREN OF THE NODE

    \item	SINGLE DATA UNIT~---~THE NODE'S DATA UNIT (IGNORE CHILDREN)
    \end{nrtc}

\item	SEVERAL DEFINED (OF COURSE)
\end{nrtc}
\end{bwslide}


\begin{bwslide}
\ctitle	{EXAMPLE: ACCESS CONTEXTS}

\vskip.5in
\diagram[p]{figure4}
\end{bwslide}


\begin{note}\em
unstructured single data unit (US):\\
transmit the node's data

unstructured all data units (UA):\\
transmit all data in the fadu

flat single data unit (FS):\\
transmit the single node's name and all data in the fadu

flat one level data units (FL):\\
transmit names/data from all nodes at a given level having data

flat all data units (FA):\\
transmit names/data from all nodes having data

hierarchical no data units (HN):\\
transmit name, data, and structure

hierarchical all data units (HA):\\
transmit name, data, and structure
\end{note}


\begin{bwslide}
\part*	{SUMMARY}\bf

\begin{nrtc}
\item	THE VIRTUAL FILESTORE IS THE OPEN SYSTEMS ABSTRACTION OF A LOCALSTORE

\item	FILES CONTAIN ATTRIBUTES AND STRUCTURING INFORMATION IN ADDITION TO
	``TYPED'' DATA

\item	FILES ARE DISTINGUISHED BY NAME

\item	SOME ATTRIBUTES ARE DYNAMIC, ON A PER-CLIENT BASIS

\item	STRUCTURE IS BASED ON A HIERARCHICAL MODEL

\item	DATA AND STRUCTURE ARE SEPARATE AND DISTINCT

\item	DOCUMENT TYPES PROVIDE AN ABBREVIATED METHOD FOR REFERRING TO THE
	FILE STRUCTURE
\end{nrtc}
\end{bwslide}

% run this through SLiTeX with the appropriate wrapper

\begin{bwslide}
\part	{THE FILE SERVICE}

\begin{nrtc}\bf
\item	MODEL OF OPERATION

\item	REGIMES AND SERVICES
\end{nrtc}
\end{bwslide}


\begin{bwslide}
\part*	{MODEL OF OPERATION}\bf

\begin{nrtc}
\item	FILE SERVICE INITIATOR AND RESPONDER

\item	REGIMES

\item	SERVICE PRIMITIVES AND COMMON PARAMETERS
\end{nrtc}
\end{bwslide}


\begin{bwslide}
\ctitle	{FILE SERVICE INITIATOR AND RESPONDER}

\begin{nrtc}
\item	A CLIENT-SERVER MODEL IS USED
    \begin{nrtc}
    \item	THE PROTOCOL EXCHANGE IS ASYMMETRIC

    \item	NOTE: CLIENT COULD BE ANOTHER FILESTORE
    \end{nrtc}

\item	AN INITIATOR IS A USER ENTITY WHICH REQUESTS THE FILE SERVICE

\item	A RESPONDER IS A USER ENTITY WHICH IMPLEMENTS THE VIRTUAL FILESTORE

\item	THIS PAIRING OF USERS IS TERMED AN FTAM ACTIVITY

\item	THE PROVIDER IS THE FILE SERVICE ABSTRACTION,
	IT IMPLEMENTS THE FILE SERVICE BY USING THE FILE PROTOCOL
\end{nrtc}
\end{bwslide}


\begin{bwslide}
\ctitle	{REGIMES}

\begin{nrtc}
\item	THE ASSOCIATION BETWEEN THE TWO ENTITIES PROGRESS THROUGH A NUMBER
	OF STAGES, TERMED ``REGIMES''

\item	A REGIME DETERMINES WHICH COMPONENTS OF THE FILE SERVICE MAY BE
	REQUESTED

\item	REGIMES NEST IN AN ORDERLY FASHION
\end{nrtc}
\end{bwslide}


\begin{note}\em
following are a lot of standard osi modeling mechanisms$\ldots$
\end{note}


\begin{bwslide}
\ctitle	{SERVICE PRIMITIVES AND COMMON PARAMETERS}

\begin{nrtc}
\item	THE INITIATOR AND RESPONDER COMMUNICATE VIA SERVICE PRIMITIVES

\item	A PRIMITIVE IS AN ABSTRACTION (NOT AN INTERFACE)

\item	IN GENERAL, THERE ARE THREE KINDS OF SERVICES
    \begin{nrtc}
    \item	CONFIRMED, IN WHICH A REQUEST ALWAYS RESULTS IN A RESPONSE

    \item	UNCONFIRMED, IN WHICH NO RESPONSE IS RETURNED

    \item	PROVIDER-INITIATED,
		IN WHICH THE SERVICE PROVIDER INDICATES SOME ABNORMAL
		CONDITION
    \end{nrtc}

\item	CONFIRMATION IS UNRELATED TO RELIABILITY

\item	SERVICE PRIMITIVES, LIKE PROCEDURE CALLS, HAVE TYPED PARAMETERS
\end{nrtc}
\end{bwslide}



\begin{bwslide}
\ctitle	{SERVICE PRIMITIVES}

\begin{nrtc}
\item	A SERVICE CONSISTS OF ONE OR MORE PRIMITIVES

\item	EACH PRIMITIVE IS PREFIXED WITH ``F-''

\item	A CONFIRMED SERVICE HAS FOUR PRIMITIVES
    \begin{nrtc}
    \item	.REQUEST, .INDICATION, .RESPONSE, and .CONFIRMATION
    \end{nrtc}

\item	AN UNCONFIRMED SERVICE HAS TWO PRIMITIVES:
    \begin{nrtc}
    \item	.REQUEST,  and .INDICATION
    \end{nrtc}

\item	A PROVIDER-INITIATED SERVICE HAS ONE PRIMITIVE:
    \begin{nrtc}
    \item	.INDICATION
    \end{nrtc}
\end{nrtc}
\end{bwslide}


\begin{bwslide}
\ctitle	{EXAMPLE: CONFIRMED SERVICE}

\vskip.5in
\diagram[p]{figure5}
\end{bwslide}


\begin{bwslide}
\ctitle	{COMMON PARAMETERS TO SERVICE PRIMITIVES}

\begin{nrtc}
\item	STATE RESULT: INDICATES IF A REGIME CHANGE IS SUCCESSFUL

\item	ACTION RESULT: INDICATES IF A SERVICE PRIMITIVE IS SUCCESSFUL

\item	DIAGNOSTICS: PROVIDES DETAILED INFORMATION ON THE FAILURE
	(IF ANY) OF A CONFIRMED SERVICE

\item	CHARGING: A RESOURCE, CHARGING UNIT, and CHARGE VALUE\\
	(INTERPRETATION IS UNDEFINED BY FTAM)

\item	IDENTITY OF FILE ACCESS DATA UNIT (FADU)

\item	PLUS: FILE ATTRIBUTES, REQUESTED ACCESS, etc.
\end{nrtc}
\end{bwslide}



\begin{bwslide}
\part*	{REGIMES AND SERVICES}\bf

\begin{nrtc}
\item	AS NOTED EARLIER,
	THE INNER-MOST REGIME DETERMINES WHICH SERVICE PRIMITIVES
	(AND HENCE SERVICES) ARE ACCESSIBLE

\item	THERE ARE FOUR REGIMES
    \begin{nrtc}
    \item	APPLICATION ASSOCIATION

    \item	FILE SELECTION

    \item	FILE OPEN

    \item	DATA TRANSFER
    \end{nrtc}
\end{nrtc}
\end{bwslide}


\begin{bwslide}
\ctitle	{NESTED REGIMES}

\vskip.5in
\diagram[p]{figure6}
\end{bwslide}


\begin{note}\em
as regimes and services are so hopelessly interwined,
we now bounce back-and-forth between the two in this part of the presentation
\end{note}


\begin{bwslide}
\ctitle	{APPLICATION ASSOCIATION REGIME}

\begin{nrtc}
\item	FTAM REGIME ESTABLISHMENT SERVICE
    \begin{nrtc}
    \item	WHEN TWO APPLICATIONS ARE BOUND BY AN ASSOCIATION,
		AN FTAM REGIME IS ESTABLISHED    
    \end{nrtc}

\item	FTAM REGIME TERMINATION SERVICE

\item	FTAM REGIME ABORT SERVICE

\item	DURING REGIME ESTABLISHMENT,
	PARAMETERS REGARDING THE USE OF THE FILE SERVICE ARE MANDATED OR
	NEGOTIATED
    \begin{nrtc}
    \item	SERVICE LEVEL (RELIABLE/USER-CORRECTABLE)

    \item	SERVICE CLASS

    \item	FUNCTIONAL UNITS

    \item	ATTRIBUTE GROUPS (KERNEL, STORAGE, etc.)
    \end{nrtc}
\end{nrtc}
\end{bwslide}


\begin{note}\em
observation: having defined a massive service,
we now frantically seek ways to delimit what actually gets used!
\end{note}


\begin{bwslide}
\ctitle	{SERVICE CLASS}

\begin{nrtc}
\item	FTAM SUPPORTS MANY SERVICES\\
	NEED A WAY TO CHOOSE A SUBSET OF THE SERVICES A INITIATOR DESIRES

\item	FIVE SERVICES CLASSES
    \begin{nrtc}
    \item	FILE TRANSFER

    \item	FILE ACCESS

    \item	FILE MANAGEMENT

    \item	FILE TRANSFER AND MANAGEMENT

    \item	UNCONSTRAINED
    \end{nrtc}

\item	THE SERVICE CLASS IS SELECTED BY THE INITIATOR DURING CONNECTION
	ESTABLISHMENT
\end{nrtc}
\end{bwslide}


\begin{bwslide}
\ctitle	{FUNCTIONAL UNITS}

\begin{nrtc}
\item	FUNCTIONAL UNITS, WHICH ARE NEGOTIABLE, PROVIDE A WAY TO FURTHER
	DELIMIT THE SERVICES NEEDED BY AN INITIATOR

\item	A FUNCTIONAL UNIT DEFINES WHICH SERVICES ARE AVAILABLE DURING
	THE LIFETIME OF THE FTAM REGIME

\item	THE SERVICE LEVEL AND CLASS OFTEN MANDATE CERTAIN FUNCTIONAL UNITS
\end{nrtc}
\end{bwslide}


\begin{bwslide}
\ctitle	{FUNCTIONAL UNITS (cont.)}

\begin{nrtc}
\item	KERNEL: REGIME ESTABLISHMENT/TERMINATION, FILE SELECTION/DESELECTION

\item	READ: FILE OPEN/CLOSE, READ BULK DATA

\item	WRITE: FILE OPEN/CLOSE, WRITE BULK DATA

\item	FILE ACCESS: LOCATE/ERASE FADU

\item	LIMITED FILE MANAGEMENT: CREATE/DELETE FILES, READ ATTRIBUTES

\item	ENHANCED FILE MANAGEMENT: CHANGE ATTRIBUTES

\item	GROUPING: BEGIN/END A COLLECTION OF REQUESTS

\item	RECOVERY: RECOVER PREVIOUS REGIME, CHECKPOINTING

\item	RESTART: RESTART DATA TRANSFER, CHECKPOINTING
\end{nrtc}
\end{bwslide}




\begin{bwslide}
\ctitle	{FILE SELECTION REGIME}

\begin{nrtc}
\item	WHEN A FILE IS BOUND TO FTAM REGIME,
	THE FILE SELECTION REGIME IS ESTABLISHED BY EITHER
    \begin{nrtc}
    \item	FILE SELECTION SERVICE:
		A FILE IS SELECTED, IF IT ALREADY EXISTS

    \item	FILE CREATION SERVICE:
		A FILE IS (OPTIONALLY) CREATED AND THENCE SELECTED
    \end{nrtc}

\item	ONCE A FILE IS SELECTED,
	FILE MANAGEMENT FUNCTIONS MAY BE PERFORMED BY
    \begin{nrtc}
    \item	READ ATTRIBUTE SERVICE

    \item	CHANGE ATTRIBUTE SERVICE
    \end{nrtc}

\item	AFTER ANY FILE MANAGEMENT FUNCTIONS,
	THE FILE MAY BE OPENED FOR TRANSFER AND/OR ACCESS
\end{nrtc}
\end{bwslide}


\begin{bwslide}
\ctitle	{FILE SELECTION REGIME (cont.)}

\begin{nrtc}
\item	THE FILE SELECTION REGIME IS TERMINATED BY EITHER
    \begin{nrtc}
    \item	FILE DESELECTION SERVICE:
		THE FILE IS SIMPLY UNBOUND FROM THE FTAM REGIME

    \item	FILE DELETION SERVICE:
		THE FILE IS REMOVED FROM THE FILESTORE, AND HENCE UNBOUND
    \end{nrtc}
\end{nrtc}
\end{bwslide}


\begin{bwslide}
\ctitle	{FILE OPEN REGIME}

\begin{nrtc}
\item	FILE OPEN SERVICE
    \begin{nrtc}
    \item	WHEN A FILE IS TO BE TRANSFERRED OR ACCESSED,
		THE FILE OPEN REGIME IS ESTABLISHED    
    \end{nrtc}

\item	THIS BINDS THE STRUCTURE OF THE FILE

\item	ONCE A FILE IS OPENED,
	FILE ACCESS FUNCTIONS MAY BE PERFORMED
    \begin{nrtc}
    \item	LOCATE FADU SERVICE

    \item	ERASE FADU SERVICE
    \end{nrtc}

\item	AFTER ANY FILE ACCESS FUNCTIONS,
	DATA TRANSFER MAY OCCUR

\item	THE FILE CLOSE SERVICE TERMINATES THE FILE OPEN REGIME
\end{nrtc}
\end{bwslide}


\begin{bwslide}
\ctitle	{DATA TRANSFER REGIME}

\begin{nrtc}
\item	FINALLY, WHEN DATA IS TO BE ACTUALLY TRANSFERRED,
	THE DATA TRANSFER REGIME IS ESTABLISHED

\item	THIS INVOKES A ``BULK DATA'' TRANSFER MECHANISM FOR FADUs
    \begin{nrtc}
    \item	READ BULK DATA SERVICE

    \item	WRITE BULK DATA SERVICE

    \item	DATA UNIT TRANSFER SERVICE

    \item	END OF DATA TRANSFER SERVICE

    \item	END OF TRANSFER SERVICE

    \item	CANCEL DATA TRANSFER SERVICE
    \end{nrtc}
\end{nrtc}
\end{bwslide}


\begin{bwslide}
\ctitle	{THE GROUPING SERVICE}

\begin{nrtc}
\item	TYPICALLY MANY FILE OPERATIONS HAVE THREE ACTIONS
    \begin{nrtc}
    \item	ACQUIRE THE FILE FOR DATA TRANSFER

    \item	PERFORM THE DATA TRANSFER

    \item	RELEASE THE FILE
    \end{nrtc}

\item	THE FIRST AND LAST ACTIONS CAN EACH BE VIEWED AS BEING INDIVISIBLE

\item	GROUPING PERMITS PRIMITIVES TO BE ``BUNDLED TOGETHER''
	IN ORDER TO IMPLEMENT ONE OF THESE TWO ACTIONS

\item	GROUPING IS MANDATED BY MOST FILE CLASSES
\end{nrtc}
\end{bwslide}


\begin{bwslide}
\ctitle	{THE GROUPING SERVICE (cont.)}

\begin{nrtc}
\item	THE TYPICAL ``ACQUIRE THE FILE'' GROUP:
    \begin{nrtc}
    \item	F-BEGIN-GROUP

    \item	F-SELECT

    \item	F-OPEN

    \item	F-END-GROUP
    \end{nrtc}

\item	THE TYPICAL ``RELEASE THE FILE'' GROUP:
    \begin{nrtc}
    \item	F-BEGIN-GROUP

    \item	F-CLOSE

    \item	F-DESELECT

    \item	F-END-GROUP
    \end{nrtc}
\end{nrtc}
\end{bwslide}


\begin{note}\em
more complicated groups will be examined later on

this should pretty much illustrate the distinction between state results and
actions results:
\begin{quote}
if a state-change operation fails, they all fail

otherwise if an operation fails, processing continues
\end{quote}

grouping is constrained to certain commonly used combinations
(setup and cleanup)
\end{note}


\begin{note}\em
not discussed due to time constraints:

service level: reliable, user-correctable

recover service for regime recreation

checkpoint service for mark insertion

restart service for transfer restoration

we're pressed for time,
so no examples here, later on during the implementation part of the talk,
different applications will be sketched
\end{note}


\begin{bwslide}
\part*	{SUMMARY}\bf

\begin{nrtc}
\item	THE FILE SERVICE EXISTS BETWEEN AN INITIATOR, RESPONDER, AND PROVIDER

\item	THE FILE SERVICE PROGRESSES THROUGH A NUMBER OF NESTED REGIMES,
	WHICH DETERMINE WHICH PARTS OF THE SERVICE MAY BE INVOKED

\item	THE SERVICE IS FURTHER LIMITED BY NEGOTIATION OF SERVICE ELEMENTS

\item	ALL SERVICES CENTER ON A SELECTED FILE WHICH IS TRANSFERRED,
	ACCESSED, OR MANAGED

\item	THE SERVICES ARE SUFFICIENTLY GENERAL TO SUPPORT A WIDE RANGE OF
	FILE ACTIVITIES
\end{nrtc}
\end{bwslide}

% run this through SLiTeX with the appropriate wrapper

\begin{bwslide}
\part	{THE FILE PROTOCOL}

\begin{nrtc}\bf
\item	ELEMENTS OF PROCEDURE

\item	DEFINITION AND ENCODING OF DATA UNITS

\item	FTAM USE OF LOWER-LAYER SERVICES

\item	EXAMPLES
\end{nrtc}
\end{bwslide}


\begin{bwslide}
\part*	{ELEMENTS OF PROCEDURE}%%%\bf

\begin{nrtc}
\item	THE FILE SERVICE PROVIDER EXECUTES THE FILE PROTOCOL

\item	THE PROVIDER IS ACTUALLY TWO PEER ENTITIES

\item	ASSOCIATION CONTROL IS USED TO MANAGE THE END-TO-END ASSOCIATION
	BETWEEN FILE USERS

\item	PRESENTATION SERVICES ARE USED TO EXCHANGE DATA IN A
	MACHINE-INDEPENDENT FASHION

\item	COMMITMENT, CONCURRENCY AND RECOVERY (CCR) SERVICES CAN ALSO BE USED
	FOR THE FILE TRANSFER CLASS

\item	ALL DATA UNITS (FPDUs and FADUs) ARE EXPRESSED IN TERMS OF
	ABSTRACT SYNTAX NOTATION ONE (ASN.1)
\end{nrtc}
\end{bwslide}


\begin{bwslide}
\ctitle	{FILE SERVICE REQUESTS}

\begin{nrtc}
\item	THE VALIDITY OF THE REQUEST IS VERIFIED
    \begin{nrtc}
    \item	i.e., CHECK NEGOTIATED FUNCTIONAL UNITS, INNER-MOST REGIME,
		INTERNAL STATE, and so on
    \end{nrtc}

\item	THE PARAMETERS OF THE REQUEST ARE ENCODED IN A
	FILE PROTOCOL DATA UNIT (FPDU)

\item	THE FPDU IS GIVEN TO THE PRESENTATION PROVIDER FOR DELIVERY
	TO THE REMOTE SYSTEM

\item	THE PROVIDER UPDATES ITS INTERNAL STATE
\end{nrtc}
\end{bwslide}


\begin{bwslide}
\ctitle	{ON RECEIPT OF A FILE PROTOCOL DATA UNIT}

\begin{nrtc}
\item	THE VALIDITY OF THE FPDU IS VERIFIED
    \begin{nrtc}
    \item	i.e., CHECK NEGOTIATED FUNCTIONAL UNITS, INNER-MOST REGIME,
		INTERNAL STATE,	and so on
    \end{nrtc}

\item	THE PARAMETERS OF THE FPDU ARE ENCODED IN A SERVICE .INDICATION
	OR .CONFIRMATION EVENT

\item	THE EVENT IS GIVEN TO THE FILE SERVICE USER

\item	THE PROVIDER UPDATES ITS INTERNAL STATE

\item	FADUs (DETERMINED BY PRESENTATION CONTEXT) ARE GIVEN DIRECTLY TO THE
	USER
\end{nrtc}
\end{bwslide}


\begin{bwslide}
\part*	{DEFINITION AND ENCODING OF DATA UNITS}\bf

\begin{nrtc}
\item	TWO KINDS OF DATA UNITS ARE EXCHANGED IN THE FILE SERVICE

\item	FILE PROTOCOL DATA UNITS (FPDUs) ARE EXCHANGED WITHIN THE
	FILE SERVICE PROVIDER

\item	FILE ACCESS DATA UNITS (FADUs) ARE EXCHANGED BY THE USERS OF THE
	FILE SERVICE
\end{nrtc}
\end{bwslide}


\begin{bwslide}
\ctitle	{ABSTRACT SYNTAX NOTATION ONE (ASN.1)}

\begin{nrtc}
\item	ABSTRACT SYNTAX NOTATION ONE (ASN.1) IS USED TO DESCRIBE THE STRUCTURE
	AND ENCODING OF DATA UNITS

\item	ASN.1 IS A DATA STRUCTURE DESCRIPTION LANGUAGE AND AN ENCODING
	SPECIFICATION
    \begin{nrtc}
    \item	IT IS USED TO DESCRIBE DATA STRUCTURES INDEPENDENT OF A
		GIVEN MACHINE'S INTERNAL REPRESENTATION

    \item	IT ALSO DEFINES HOW TO UNIVERSALLY ENCODE THOSE STRUCTURES
		AS THEY ARE TRANSMITTED FROM ONE MACHINE TO ANOTHER
    \end{nrtc}
\end{nrtc}
\end{bwslide}


\begin{bwslide}
\part*	{FTAM USE OF LOWER-LAYER SERVICES}\bf

\begin{nrtc}
\item	ASSOCIATION CONTROL

\item	PRESENTATION SERVICES

\item	SESSION SERVICES

\item	COMMITMENT, CONCURRENCY AND RECOVERY
\end{nrtc}
\end{bwslide}


\begin{bwslide}
\ctitle	{FTAM USE OF LOWER-LAYER SERVICES (cont.)}

\vskip.5in
\diagram[p]{figure7}
\end{bwslide}


\begin{bwslide}
\ctitle	{ASSOCIATION CONTROL}

\begin{nrtc}
\item	ASSOCIATION CONTROL IS USED BY
    \begin{nrtc}
    \item	FTAM REGIME ESTABLISHMENT SERVICE: A-ASSOCIATE

    \item	FTAM REGIME TERMINATION SERVICE: A-RELEASE

    \item	FTAM REGIME ABORT SERVICE: A-(U-)ABORT, A-P-ABORT
    \end{nrtc}

\item	NOTE THAT ASSOCIATION CONTROL MAPS DIRECTLY ONTO PRESENTATION
	SERVICES
    \begin{nrtc}
    \item	A PART OF THE APPLICATION LAYER
		(SO-CALLED COMMON APPLICATION SERVICE ENTITY)
    \end{nrtc}
\end{nrtc}
\end{bwslide}


\begin{bwslide}
\ctitle	{ADDRESSES AND APPLICATION ENTITY TITLES}

\begin{nrtc}
\item	INITIATOR PROVIDES
    \begin{nrtc}
    \item	DESCRIPTION OF FILE SERVICE DESIRED,
		e.g., ``gremlin-filestore''
    \end{nrtc}
        
\item	AND (SOMEHOW) PERFORMS TWO MAPPINGS
    \begin{nrtc}
    \item	DESCRIPTOR TO APPLICATION ENTITY TITLE PROVIDING SERVICE:
		CURRENTLY AN OBJECT IDENTIFIER

    \item	AET TO PRESENTATION ADDRESS:
		CURRENTLY P-SELECTOR, S-SELECTOR, T-SELECTOR, AND A LIST OF
		NETWORK ADDRESSES
    \end{nrtc}

\item	IN THE FUTURE, DIRECTORY SERVICES ARE USED
\end{nrtc}
\end{bwslide}


\begin{bwslide}
\ctitle	{PRESENTATION SERVICES}

\begin{nrtc}
\item	PRESENTATION SERVICES ARE USED BY THE REMAINING FTAM REGIMES: P-DATA

\item	FURTHER
    \begin{nrtc}
    \item	THE FILE OPEN SERVICE MAY REQUIRE: P-ALTER-CONTEXT

    \item	THE CANCEL DATA SERVICE REQUIRES: P-RESYNCHRONIZE

    \item	THE CHECKPOINT SERVICE REQUIRES: P-SYNC-MINOR

    \item	THE RESTART SERVICE REQUIRES: P-ALTER-CONTEXT
    \end{nrtc}

\item	IN ADDITION, ASSOCIATION CONTROL REQUIRES:
	P-CONNECT, P-RELEASE, P-U-ABORT, P-P-ABORT
\end{nrtc}
\end{bwslide}


\begin{bwslide}
\ctitle	{PRESENTATION CONTEXTS}

\begin{nrtc}
\item	ASSOCIATION CONTROL PCI (PRESENTATION CONTEXT INFORMATION)

\item	FTAM PCI

\item	IF THE PRESENTATION CONTEXT MANAGEMENT SERVICE IS UNAVAILABLE, THEN
    \begin{nrtc}
    \item	FTAM REQUESTS A CONTEXT FOR EACH DOCUMENT TYPE THAT MIGHT BE
		EXCHANGED
    \end{nrtc}

\item	FTAM PROVIDES BOTH THE ABSTRACT SYNTAX AND TRANSFER SYNTAX
	OF EACH CONTEXT
\end{nrtc}
\end{bwslide}


\begin{bwslide}
\ctitle	{COMMITMENT, CONCURRENCY AND RECOVERY}

\begin{nrtc}
\item	IF THE FILE TRANSFER CLASS IS SELECTED, AS A USER OPTION,
	THE ISO COMMITMENT, CONCURENCY, AND RECOVERY PROTOCOL CAN BE USED

\item	NEEDED FOR ATOMIC TRANSFER OF FILES (BUT NOT REALLY NEEDED FOR
	RESUMPTION OF FILE TRANSFER)

\item	PERSONAL OPINION
    \begin{nrtc}
    \item	A TREMENDOUS ``OVERKILL'' FOR ATOMIC FILE TRANSFER

    \item	NOT REALLY WELL-DEFINED AT THIS POINT
    \end{nrtc}
\end{nrtc}
\end{bwslide}


\begin{bwslide}
\ctitle	{SESSION SERVICES}

\begin{nrtc}
\item	THE FILE PROVIDER DOES NOT USE SESSION SERVICES DIRECTLY

\item	HOWEVER MOST PRESENTATION SERVICES MAP DIRECTLY ONTO SESSION SERVICES

\item	HENCE: AT LEAST S-CONNECT, S-DATA, S-RELEASE, S-U-ABORT, AND S-P-ABORT
	ARE REQUIRED

\item	AND OPTIONALLY: S-TYPED-DATA, S-RESYNCHRONIZE and S-SYNC-MINOR ARE
	ALSO REQUIRED
\end{nrtc}
\end{bwslide}


\begin{bwslide}
\part*	{EXAMPLES}\small

\begin{verbatim}
wrote F-INITIALIZE-request, context 1
{
   {
      service-class transfer-and-management-class,
      functional-units { read, write, limited-file-management,
                         enhanced-file-management },
      attribute-groups { storage, security },
      contents-type-list {
         { document-types { 1.0.8571.6.3, 1.17.3.6.1, 1.17.3.6.8 } }
      },
      initiator-identity "ANON",
      filestore-password { "mrose" }
   }
}
\end{verbatim}
\end{bwslide}


\begin{bwslide}\small
\begin{verbatim}
wrote AARQapdu, context 9
{
   protocolVersion { version1 },
   calledAEtitle 1.17.4.3.1,
   applicationContextName 1.0.8571.2.1,
   userInformation {
      data-value-identifier { indirect-reference 1 },
      encodings {
         single-ASN1-type {
            [3] '03'H,
            [4] '0136'H,
            [5] '06c0'H,
            [7] {
               [0] {
                  [APPLICATION 7] '28c27b0603'H,
                  [APPLICATION 7] '39030601'H,
                  [APPLICATION 7] '39030608'H
               }
            },
            [APPLICATION 4] "ANON",
            [APPLICATION 6] { "mrose" }
         }
      }
   }
}
\end{verbatim}
\end{bwslide}


\begin{bwslide}\small
\begin{verbatim}
wrote CPppdu
{
   { nonx410mode },
   [2] {
      [2] TRUE,
      [3] {
         { 1, 1.0.8571.1.1, { 1.0.8825 } },
         { 3, 1.0.8571.2.4, { 1.0.8571.3.4 } },
         { 5, 1.17.3.2.0, { 1.17.3.3.0 } },
         { 7, 1.17.3.2.2, { 1.17.3.3.0 } },
         { 9, 1.0.8650.2.1, { 1.0.8825 } }
      },
      [4] { 1.0.8571.1.1, 1.0.8825 },
      [5] { version-1 },
      {
         {
            {
               data-value-identifier { indirect-reference 9 },
               encodings {
                  single-ASN1-type {
                     [0] '0780'H,
                     [1] { 1.17.4.3.1 },
                     [3] { 1.0.8571.2.1 },
\end{verbatim}
\end{bwslide}


\begin{bwslide}\small
\begin{verbatim}
                     [4] {
                        [UNIVERSAL 8] {
                           1,
                           [0] {
                              [0] {
                                 [3] '03'H,
                                 [4] '0136'H,
                                 [5] '06c0'H,
                                 [7] {
                                    [0] {
                                       [APPLICATION 7] '28c27b0603'H,
                                       [APPLICATION 7] '39030601'H,
                                       [APPLICATION 7] '39030608'H
                                    }
                                 },
                                 [APPLICATION 4] "ANON",
                                 [APPLICATION 6] { "mrose" }
                              }
                           }
                        }
                     }
                  }
               }
            }
         }
      }
   }
}
\end{verbatim}
\end{bwslide}


\begin{bwslide}\small
\begin{verbatim}
---> (: dump of SPDU 0xb8404, errno=0xffffffff mask=0x409f
---> LI/ 281
---> CODE/ CONNECT
---> REFERENCE/ <USER "gremlin", COMMON  "870601045854Z", ADDITIONAL 0>
---> OPTIONS/ 0x0<>
---> TSDU/ INITIATOR: 65528, RESPONDER: 65528
---> VERSION/ 0x1
---> ISN/ 1
---> REQUIREMENTS/ 0x22<DUPLEX,RESYNC>
---> USER DATA/ 225 bytes
---> )
\end{verbatim}
\end{bwslide}


\begin{bwslide}
\part*	{SUMMARY}\bf

\begin{nrtc}
\item	THE FILE SERVICE PROVIDER IS A ``STATE MACHINE'' COMPOSED OF TWO
	PEERS EXECUTING THE FILE PROTOCOL

\item	IN ADDITION TO RESOURCES ON THEIR HOST SYSTEMS,
	THEY USE THE ASSOCIATION CONTROL AND PRESENTATION SERVICES

\item	ASN.1 IS USED TO DEFINE AND ENCODE THE DATA UNITS WHICH ARE EXCHANGED
\end{nrtc}
\end{bwslide}

% run this through LaTeX with the appropriate wrapper

\chapter       {Boilerplate for Responders}\label{cook:responder}
Let's consider how to build a responder which is also a performer.
In Chapter~\ref{cook:discipline},
two forms for a responder were identified:
{\em dynamic},
in which each incoming association caused the instantiation of a new responder,
and,
{\em static},
in which each incoming association was given to a pre-existing single
instantiation of a responder.
The dynamic responder can be thought of as simply being a special case of a
static responder.
Hence,
we will describe how one builds a static responder in this chapter.

If you have access to the source tree for this release,
the directory \file{others/lookup/} contains the boilerplate described herein.

\section	{Static Responder}
There are three areas for the boilerplate:
association management, operation response, and, error handling.

Before proceeding however,
let's consider what an \verb"#include" file, say \verb"ryresponder.h",
might look like.
First, the standard \man librosy(3n) definitions are included,
along with the definitions for the daemon logging package.
Next, a \verb"dispatch" structure is defined
along with the boilerplate routines.
The \verb"dispatch" structure will be used by the boilerplate to invoke a
user-supplied routine that will respond to an invocation.

\newpage
\tgrindfile{ryresponder-h}

\subsection	{Association Management}
Association management is performed precisely the way it is outlined in
Section~\ref{acs:server} of \volone/.
This is implemented by the routine \verb"ryresponder":
\begin{quote}\index{ryresponder}\small\begin{verbatim}
int     ryresponder (argc, argv, myservice, dispatches, ops,
                start, stop)
int     argc;
char  **argv,
       *myservice;
struct dispatch *dispatches;
struct RyOperation *ops;
IFP     start,
        stop;
\end{verbatim}\end{quote}
The parameters to this procedure are:
\begin{describe}
\item[\verb"argc"/\verb"argv":] the argument vector (and its length)
that the program was invoked with;

\item[\verb"myservice":] the non-host portion of the application-entity
information;

\item[\verb"dispatches":] a pointer to a \verb"dispatch" table;

\item[\verb"ops":] a pointer to a \verb"RyOperation" table;

\item[\verb"start":] the address of a routine to decide if incoming
associations should be accepted
(use \verb"NULLIFP" if associations should always be accepted);
and,

\item[\verb"stop":] the address of a routine to note that an association
requests termination or has been terminated
(use \verb"NULLIFP" if associations termination is unremarkable).
\end{describe}
The function of this routine is straight-forward though tedious.
First, \verb"myname" is initialized to the name that the program was invoked
with.
Next, a debug flag is possibly set and the daemon logging package is
initialized,
and the responder's application-entity information is computed.
Finally,
each operation is registered with the \verb"RyDispatch" routine,
and the \verb"start" and \verb"stop" routines are remembered.

The routine \verb"isodeserver" is then called to manage any associations.
As a result,
the \verb"ros_init" routine will be informed of new associations,
the \verb"ros_work" routine will be informed of network activity,
and the \verb"ros_lose" routine will be advised if network listening fails.

The \verb"ros_init" routine is also straight-forward.
First, \verb"AcInit" is called to recapture the ACSE-state,
then the user-defined association acceptance routine is called (if any).
This routine should return either \verb"ACS_ACCEPT" if the association is to
be accepted,
or either 
\begin{quote}\tt
ACS\_TRANSIENT, ACS\_PERMANENT,\\
\end{quote}
otherwise
(these latter two codes are discussed in Table~\ref{AcSAPreasons} on
page~\pageref{AcSAPreasons} of \volone/).
The routine \verb"AcAssocResponse" is then called to deal with the incoming
association.
The arguments are strictly boilerplate:
they will work unaltered for most applications.
If the association was accepted,
then the routine \verb"RoSetService" is used to tell the remote operations
library to use the presentation service as the underlying service.

The routine \verb"ros_work" is called when activity occurs on an association,
and is somewhat complex.
The routine sets a global return vector using \man setjmp (3)
and then calls \verb"RyWait" to poll for the next operation-related event.
This usually results in one of the operations registered earlier being
dispatched immediately,
and then \verb"RyWait" will return \verb"NOTOK"
with an error condition of \verb"ROS_TIMER" indicating that there is no
more network activity pending.
Otherwise, the routine \verb"ros_indication" is called to handle
extraordinary conditions on the association.
If some error occurred during the handling of an invocation,
use of the routine \verb"ros_adios" will cause control to return to the
\verb"setjmp" call.
In this case,
several things happen.
First,
the user-defined association termination routine routine is called (if any).
This routine should note that the association is now abruptly terminated.
Next,
the \verb"AcUAbortRequest" routine is called to make sure that the association
is (ungracefully) released.
Following this,
the \verb"RyLose" routine is called to expunge any information regarding
queued operations from the run-time environment.
Finally, \verb"NOTOK" is returned to \verb"isodeserver",
which causes the association to be removed from the list of current
associations.

The \verb"ros_indication" routine is used to handle uncommon events for an
association: user-rejections, provider-rejections, and association termination.
In all three cases,
the event is logged.
In the case of the initiator requesting that the association be terminated,
several things happen.
First,
the user-defined association termination routine is called (if any).
This routine should return either \verb"ACS_ACCEPT" if the association is to
be released,
or \verb"ACS_REJECT" if the termination is to be refused.
The routine \verb"AcRelResponse" is then called to deal with the request to
terminate the association.
If the termination was accepted,
then control returns to the \verb"setjmp" call in \verb"ros_work",
which finalizes things.

The \verb"ros_lose" routine is simple:
the error condition is logged.

\tgrindfile{ryresp-assoc}
\newpage

\subsection	{Operation Response}
When an operation is invoked,
its dispatch routine is called from the routine \verb"RyWait"
as described in Section~\ref{librosy:register} on
Page~\pageref{librosy:register}.
The boilerplate for a dispatch routine is fairly uniform.
A check is made to see if the invocation is linked to a previous invocation,
and the invocation is logged.
Any user-rejections are performed by the routine \verb"ureject" which is a
simple wrapper for the \verb"RyDsUReject" routine.
Next, the operation is attempted.
If it succeeded (as denoted by \verb"won"),
then a result is allocated, and initialized and returned to the initiator by
the \verb"RyDsResult" routine.
Otherwise,
an error is selected,
and
its parameter is allocated, initialized and returned to the initiator by the
\verb"error" routine which is a simple wrapper for the \verb"RyDsError"
routine.
Finally,
the argument is freed and the handler returns.

\tgrindfile{ryresp-invoke}
\newpage

\subsection	{Error Handling}
These routines for the most part are all straight-forward.
\begin{describe}
\item[\verb"ros\_adios":] used to report a ROS error and terminate;

\item[\verb"ros\_advise":] used to report a ROS error;

\item[\verb"acs\_advise":] used to report an ACS error;

\item[\verb"adios":] used to report an error and terminate;

\item[\verb"advise":] used to report an error;
and,

\end{describe}
\pgm{pepy} generate errors are normally caught by the routine
\verb"PY_advise"\index{PY\_advise} by using the \verb"-a" switch to
\pgm{pepy}. The \man librosy(3n) routines use these to build up error
message when \pgm{pepy} routines fail.

\tgrindfile{ryresp-error}
\newpage

\subsection	{An Example}
An example of a responder written using this boilerplate is
shown in Section~\ref{passwd:responder} on page~\pageref{passwd:responder}.

% run this through LaTeX with the appropriate wrapper

\chapter       {Boilerplate for Initiators}\label{cook:initiator}
Let's consider how to build an initiator which is also an invoker.
In Chapter~\ref{cook:discipline},
two forms for an initiator were identified:
{\em interactive}, and {\em embedded}.
The interactive initiator can be thought of as simply being a special case of
an embedded initiator.
Hence,
we will start by describing the embedded initiator and then describe the
additional structure added to form an interactive initiator.

If you have access to the source tree for this release,
the directory \file{others/lookup/} contains the boilerplate described herein.

\section	{Embedded Initiator}
An embedded initiator is characterized as automatically managing an association
and invoking operations as required.
There are four areas: association establishment, operation invocation,
association release, and error handling.

\subsection	{Association Establishment}
The application-entity information and presentation address for the desired
service are computed,
along with the application context and default presentation context
information for the service.
In addition,
a session reference identifier is chosen.

The routine \verb"AcAssocRequest" is called to establish an association for
the service.
The arguments are strictly boilerplate:
they will work unaltered for most applications.
If the association is established,
then the routine \verb"RoSetService" is used to tell the remote operations
library to use the presentation service as its underlying service.

\tgrindfile{ryinit-estab}
\newpage

\subsection	{Operation Invocation}
We'll consider how an operation is invoked using both the synchronous and
asynchronous interfaces.

\subsubsection	{Synchronous Invocation}
First, the argument (if any) for the operation is allocated and initialized.
Then the macro \verb"op_MODULE_operation" is called,
which is really a call to the \verb"RyOperation" routine described in
Section~\ref{ryoperation} on page~\pageref{ryoperation}.
One of three values is expected to be returned.

If the manifest constant \verb"NOTOK" is returned,
then some error has occured prior to invoking the operation.
The most common error is the association being abruptly terminated due to
network failure.

If the manifest constant \verb"OK" is returned,
then the responder replied with either a result or an error for the invocation.
The variable \verb"response" is consulted to determine if a result is
present, or if not, which error is present.
For each case,
the \verb"out" variable is cast to the appropriate variable,
the application-specific code is executed,
and then the structure is freed.

If the manifest constant \verb"DONE" is returned,
then the responder indicated that it wished to terminate the association.
Since the association was established in a way which prohibited this behavior,
this return is unlikely.

\tgrindfile{ryinit-invoke}
\newpage

\subsubsection	{ASynchronous Invocation}

\tgrindfile{ryinit-async}
\newpage


\subsection	{Association Release}
Terminating the assocation is simple:
the routine \verb"AcRelRequest" is called.

\tgrindfile{ryinit-release}
\newpage

\subsection	{Error Handling}
These routines for the most part are all straight-forward.
\begin{describe}
\item[\verb"ros\_adios":] used to report a ROS error and terminate;

\item[\verb"ros\_advise":] used to report a ROS error;

\item[\verb"acs\_adios":] used to report an ACS error and terminate;

\item[\verb"acs\_advise":] used to report an ACS error;

\item[\verb"adios":] used to report an error and terminate;
and,

\item[\verb"advise":] used to report an error.
\end{describe}
It is assumed that there is a definition of the form
\begin{quote}\small\begin{verbatim}
char   *myname = "myname";
\end{verbatim}\end{quote}
for use by the \verb"advise" routine.

\tgrindfile{ryinit-error}
\newpage

\section	{Interactive Initiator}
Now, let's build on these routines to write an initiator which is interactive:
the user runs a program and interactively directs the invocation of
operations.

\subsection	{Include File}
Let's consider what an \verb"#include" file, say \verb"ryinitiator.h",
might look like.
First, the standard \man librosy(3n) definitions are included,
then a \verb"dispatch" structure is defined,
along with the boilerplate routines.
The \verb"dispatch" structure will be used by the boilerplate to invoke a
user-supplied routine that will invoke the operation.

\tgrindfile{ryinitiator-h}
\newpage

\subsection	{Worker Routines}
Now, let's consider the routines which implement the interactive initiator.
There are only two: \verb"ryinitiator" and \verb"getline".

The \verb"ryinitiator" routine does most of the work:
\begin{quote}\index{ryinitiator}\small\begin{verbatim}
int     ryinitiator (argc, argv, myservice, mycontext,
                mypci, ops, dispatches, quit)
int     argc;
char  **argv,
       *myservice,
       *mycontext,
       *mypci;
struct RyOperation ops[];
struct dispatch *dispatches;
IFP    quit;
\end{verbatim}\end{quote}
The parameters to this procedure are:
\begin{describe}
\item[\verb"argc"/\verb"argv":] the argument vector (and its length)
that the program was invoked with;

\item[\verb"myservice":] the non-host portion of the application-entity
information;

\item[\verb"mycontext":] the application context name of the service;

\item[\verb"mypci":] the abstract syntax of the service;

\item[\verb"ops":] the operations defined for the service;

\item[\verb"dispatches":] a pointer to a \verb"dispatch" table;
and,

\item[\verb"quit":] a routine to call when the association is to be
terminated.
\end{describe}
The function of this routine is straight-forward though tedious.
First, \verb"myname" is initialized to the name that the program was invoked
with.
Next, a rudimentary argument check is done,
and the application-entity information and presentation address for the desired
service are computed.
This is followed by computing the application context and default
presentation context information for the service,
along with a session reference identifier.
Finally, some diagnostic information may be printed out if the program will
operate in interactive mode.

The routine \verb"AcAssocRequest" is called to establish an association for
the service.
If successful,
the routine \verb"RoSetService" is used to tell the remote operations library
to use the presentation service as the underlying service.

The interactive loop is then entered:
a line is read from the standard input (the \verb"getline" routine is called),
it is broken into components,
a search is performed on the dispatch table to find a matching keyword,
and then finally the user-supplied routine is called.
When end-of-file is reached
(or if the user-supplied routine returns the manifest constant \verb"DONE"),
the assocation is released.

If additional arguments were present on the invocation line,
then the named operation is invoked instead of entering an interactive loop.

\tgrindfile{ryinitiator-c}
\newpage

\subsection	{An Example}
An example of an interactive initiator written using this boilerplate is
shown in Section~\ref{passwd:initiator} on page~\pageref{passwd:initiator}.
However,
a brief exposition of the \verb"dispatch" structure and possible
\verb"help" and \verb"quit" routines are shown here.

\tgrindfile{ryinit-example}

% run this through SLiTeX with the appropriate wrapper

\begin{bwslide}
\part	{FTAM STATUS\\ AS OF 1 JUNE 1987}

\begin{nrtc}\bf
\item	STATUS OF THE FTAM STANDARD

\item	USER PROFILES

\item	CONFORMANCE TESTING
\end{nrtc}
\end{bwslide}


\begin{note}\em
strong disclaimer: this section of the presentation is a limited snapshot of
the state of the world as of june first

everything will have changed by june second
\end{note}


\begin{bwslide}
\part*	{STATUS OF THE FTAM STANDARD}\bf

\begin{nrtc}
\item	FTAM IS CURRENTLY A DRAFT INTERNATIONAL STANDARD (DIS)
	AS OF OCTOBER, 1986

\item	THE DRAFT OF THE FTAM INTERNATIONAL STANDARD (IS) WILL PROBABLY
	BE AVAILABLE BY THE FOURTH QUARTER, 1987

\item	MOST RELATED STANDARDS IN PLACE PRIOR TO THEN
\end{nrtc}
\end{bwslide}


\begin{bwslide}
\ctitle	{THE FINAL FTAM STANDARD\\ WILL PROBABLY:}

\begin{nrtc}
\item	INCLUDE MORE TUTORIAL INFORMATION!

\item	SIMPLIFY THE RELIABLE AND USER-CORRECTABLE SERVICE LEVELS INTO ONE
	(USER-VISIBLE) SERVICE LEVEL

\item	MAKE THE SERVICE CLASS NEGOTIABLE

\item	SIMPLIFY THE SYNTAX OF FADUs SOMEWHAT (FA, DU and DE MAPPINGS)

\item	CORRECT A NUMBER OF BUGLETS AND TYPOS
\end{nrtc}
\end{bwslide}


\begin{bwslide}
\part*	{USER PROFILES}\bf

\begin{nrtc}
\item	MOST U.S. PROFILES SEEM TO HAVE ALIGNED WITH
	NBS IMPLEMENTATION AGREEMENTS FOR OSI PROTOCOLS
    \begin{nrtc}
    \item	GOSIP

    \item	MAP/TOP

    \item	COS
    \end{nrtc}

\item	EUROPEANS SEEM VERY INTERESTED AS WELL
    \begin{nrtc}
    \item	SPAG

    \item	CEN/CENELEC

    \item	CEPT
    \end{nrtc}

\item	STANDARDIZATION OF FUNCTIONAL PROFILES IS OCCURING BETWEEN THESE
	GROUPS
\end{nrtc}
\end{bwslide}


\begin{bwslide}
\part*	{CONFORMANCE TESTING}\bf

\begin{nrtc}
\item	THE NATIONAL PHYSICAL LABORATORY IN THE UK IS BUILDING AN
	FTAM (AND EMBEDDED SESSION) TESTING SYSTEM

\item	THE NATIONAL COMPUTER CENTRE (UK) WILL RUN THEIR FTAM CONFORMANCE
	TESTING SERVICE BASED ON THIS SYSTEM

\item	THE NCC IS MAKING THIS FTAM TESTING SYSTEM AVAILABLE TO COS AND
	VARIOUS EUROPEAN TESTING SERVICES

\item	HENCE, THERE SHOULD BE HARMONIZATION OF CONFORMANCE TESTING BETWEEN
	THE US AND EUROPE
\end{nrtc}
\end{bwslide}


\begin{note}\em
to summarize:

things may be finalized by mid-1988
\end{note}


\begin{bwslide}
\part*	{ACKNOWLEDGEMENTS}

\begin{quote}\em
this presentation is based on experiences in implementing the
ISO Development Environment (ISODE) at NRTC,
an openly available implementation of the upper-layers of OSI

others have made significant contributions to the content and quality of this
presentation,
notably:
\end{quote}

\begin{nrtc}\em
\item	at UCL: Steve Kille

\item	at BRL: Mike Muuss

\item	at NPL: John Pavel

\item	at NRTC: John L.~Romine

\item	at NMA: Einar A.~Stefferud
\end{nrtc}

also, UNIX is a trademark of at\&t bell laboratories
\end{bwslide}

\end{document}
