% run this through SLiTeX

\documentstyle
    [blackandwhite,landscape,oval,pagenumbers,small]{NRslides}

\input trademark

\def\tradeORGfont{\rm}
\def\tradeNAMfont{\rm}

\def\tcptpgw/{TCP $\Longleftrightarrow$ TP MAGIC-BOX}

\raggedright

\begin{document}

\title	{RECENT DEVELOPMENTS IN\\ MIGRATING TOWARD ISO}
\author	{Marshall T.~Rose\\
	Computer Science Laboratory\\
	Northrop Research and Technology Center}
\date	{October 20, 1986}
\maketitlepage


\begin{bwslide}
\part*	{OUTLINE}\bf

\begin{nrtc}
\item	BACKGROUND
\item	TOWARD A SOLUTION
\item	A DIGRESSION ON THE INTEROPERABILITY OF APPLICATIONS
\item	A MIGRATION STRATEGY
\item	THE FIRST STEP: ISODE
\end{nrtc}
\end{bwslide}


\begin{note}\em
i've given a few variants of this talk in the past,
so may cover it too quickly!
\end{note}


\begin{bwslide}
\part	{BACKGROUND}\bf
\begin{nrtc}
\item	THE ISO INTERPRETATION OF OSI IS GAINING POPULARITY
    \begin{nrtc}
    \item	VENDORS (e.g., COS)
    \item	USER GROUPS (e.g., MAP/TOP)
    \item	INTERNATIONAL COMMUNITY
    \end{nrtc}
\end{nrtc}
\end{bwslide}


\begin{bwslide}
\ctitle	{INFORMAL DEFINITION OF TERMS}

\begin{nrtc}
\item	OSI

\item	DDN, ARPA, ISO

\item	SUITE, STACK, WORLD

\item	BLACK-BOX

\item	GATEWAY, MAGIC-BOX
\end{nrtc}
\end{bwslide}


\begin{bwslide}
\ctitle	{$\ \ \ \ \ \ $ A BIG PROBLEM}
\vskip-0.5in
\diagram[p]{figure1}
\end{bwslide}



\begin{note}\em
note that ``development'' means design and implementation
\end{note}


\begin{bwslide}
\ctitle	{MORE PROBLEMS}

\begin{nrtc}
\item	NEED ISO EXPERTISE AND MATURITY (AT ALL LEVELS)
    \begin{nrtc}
    \item	BUT CAN'T WAIT~---~HAVE REQUIREMENTS NOW!

    \item	AS USUAL, DEVELOPMENT OF APPLICATIONS LAG BEHIND OTHER LAYERS
    \end{nrtc}

\item	MIGRATION PREFERABLE TO STARTING FROM SCRATCH
    \begin{nrtc}
    \item	MANY NEW, MAJOR INVESTMENTS BEING MADE IN CURRENT\\
		TECHNOLOGY (e.g., NSFnet, NASA's PROPOSED INTERNET, etc.)

    \item	MANY EXISTING SYSTEMS WORK ACCEPTABLY AND/OR HAVE A HEAVY
		RE-IMPLEMENTATION COST
    \end{nrtc}
\end{nrtc}
\end{bwslide}


\begin{bwslide}
\ctitle	{OBSERVATIONS}

\begin{nrtc}
\item	MANY OF THESE PROBLEMS HAVE PREVIOUSLY BEEN SOLVED, IN\\ DIFFERENT
	SETTINGS, WITHOUT LOSS OF GENERALITY
    \begin{nrtc}
    \item	SOME OF THESE SOLUTIONS ARE QUITE MATURE\\
		(STABLE, ROBUST, AND ENJOY A HIGH DEGREE OF VENDOR SUPPORT)
    \end{nrtc}

\item	THE ISO STACK IS A STRONGLY LAYERED ARCHITECTURE, WHICH\\ FACILITATES
	APPLICATION-INDEPENDENCE OF UNDERLYING\\ PROTOCOLS
    \begin{nrtc}
    \item	SERVICES ARE IMPORTANT,\\ IMPLEMENTATIONS ARE NOT
    \end{nrtc}
\end{nrtc}
\end{bwslide}


\begin{bwslide}
\part	{TOWARD A SOLUTION}\bf

\begin{nrtc}
\item	USE A VENDOR-PROPRIETARY SOLUTION?
    \begin{nrtc}
    \item	HOW MANY TIMES WILL SYSTEMS HAVE TO BE REWRITTEN?

    \item	HOW MANY INTERIM SOLUTIONS WILL THE VENDORS SELL US?
    \end{nrtc}

\item	DEVELOP A MIGRATION STRATEGY!
    \begin{nrtc}
    \item 	ENSURE THAT ANY WORK STARTED TODAY WILL EASILY MIGRATE TO
		TOTAL ISO SOLUTIONS, AS THEY BECOME AVAILABLE

    \item	ALLOW FOR THE CO-EXISTENCE OF CURRENTLY OPERATIONAL SYSTEMS
    \end{nrtc}
\end{nrtc}
\end{bwslide}


\begin{bwslide}
\ctitle	{A SOLUTION}

\begin{nrtc}
\item	USE TCP/IP AS THE MIGRATION VEHICLE
    \begin{nrtc}
    \item	OFFER ISO SERVICES ON TOP OF THE DDN PROTOCOL SUITE

    \item	DEVELOP ISO APPLICATIONS NOW IN A LARGE ENVIRONMENT

    \item	NO (OR MINIMAL) RECODING LATER
    \end{nrtc}

\item	A FEW ADVANTAGES OF TCP/IP
    \begin{nrtc}
    \item	ROBUST, MATURE, ETC.

    \item	VENDOR SUPPORT

    \item	LARGE BODY OF EXPERTISE

    \item	SIMILAR ARCHITECTURE
    \end{nrtc}
\end{nrtc}
\end{bwslide}


\begin{note}\em
assume everyone already knows about tcp/ip here$\ldots$

if not, we're in big trouble
\end{note}


\begin{bwslide}
\ctitle	{PHILOSOPHY}

\begin{nrtc}
\item	COMPLEMENTARY CO-EXISTENCE:
    \begin{nrtc}
    \item	UTILIZE TCP/IP FUNCTIONALITY NOT CURRENTLY IN ISO\\
		(ROUTING, ETC.)

    \item	GAIN EXPERIENCE IN THE NUMEROUS EXISTING TCP/IP WORLDS

    \item	UTILIZE ISO FUNCTIONALITY AS IT BECOMES AVAILABLE
    \end{nrtc}

\item	DEVELOP APPLICATIONS IN AN \underline{EVOLUTIONARY},
	NOT \underline{REVOLUTIONARY}, FASHION

\item	WANT TO BE CONSISTENT WITH ISO'S DIRECTION,
	BUT WANT TO GET WORK DONE NOW
\end{nrtc}
\end{bwslide}


\begin{note}\em
aside on arpa applications:

\begin{nrtc}
\item	most are 15+ years old (sans domains)
\item	each application ``rolls its own'' syntax
\item	the netascii legacy
\end{nrtc}  
\end{note}


\begin{bwslide}
\ctitle	{$\ \ \ \ \ \ $ WHERE TO JOIN THEM?}
\vskip-0.5in
\diagram[p]{figure2}
\end	{bwslide}


\begin{bwslide}
\ctitle	{COMPARISON OF THE TCP AND TP4}

\begin{nrtc}
\item	THE TCP IS STREAM-ORIENTED, THE TP4 IS PACKET-ORIENTED

\item	THE TCP COALESCES SIMULTANEOUS CONNECTIONS

\item	THE TCP HAS AN ``URGENCY'' CONCEPT, THE TP4 HAS ``EXPEDITED''

\item	THE TCP HAS A GRACEFUL CLOSE
\end{nrtc}
\end{bwslide}


\begin{bwslide}
\ctitle	{APPROACH}

\begin{nrtc}
\item	USE \underline{INTERFACE TRANSLATION} ABOVE TCP/IP
    \begin{nrtc}
    \item	USE A WRAPPER TO MAKE THE NATIVE TCP INTERFACE APPEAR TO BE
		THE TP4 INTERFACE

    \item	SAME SERVICE OFFERED TO USERS

    \item	ENTIRELY DIFFERENT IMPLEMENTATION OF THOSE SERVICES
    \end{nrtc}
\end{nrtc}
\end{bwslide}


\begin{bwslide}
\ctitle	{ISO TRANSPORT SERVICES ON TOP OF THE TCP}

\diagram[p]{figure5}
\end{bwslide}


\begin{bwslide}
\ctitle	{SUMMARY OF THE MAGIC-BOX PROTOCOL}

\begin{nrtc}
\item	OBSERVATIONS
    \begin{nrtc}
    \item	ALL THE REALLY HARD PARTS ARE ALREADY DONE BY THE\\ TCP
		(i.e., THE MAJORITY OF THE TRANSPORT PROTOCOL\\ FUNCTIONALITY)

    \item	THE TRANSPORT INTERFACING REMAINS TO BE DONE
    \end{nrtc}

\item	USES AN EFFICIENT PACKETIZATION PROTOCOL\\
	(GOING THE OTHER WAY IS A LOT HARDER)

\item	QUALITY OF SERVICE~---~FOR FURTHER STUDY
\end{nrtc}
\end{bwslide}


\begin{bwslide}
\ctitle	{ISSUE: MANAGEMENT OF THE ADDRESS SPACE}

THE CLEVER APPROACH:
\begin{small}
\[\begin{tabular}{rlc}
	$<$NSAP ID$>$&		$\longleftrightarrow$&	$<$IP address$>$\\
	$<$TSAP selector, SSAP selector, PSAP selector$>$&
				$\longleftrightarrow$&	$<$TCP port$>$
\end{tabular}\]
\end{small}

\begin{nrtc}
\item	SUGGESTS THAT THE TP CAN BE RUN DIRECTLY ABOVE THE DDN IP PROTOCOL
\end{nrtc}
\end{bwslide}



\begin{bwslide}
\ctitle	{MANAGEMENT OF THE ADDRESS SPACE (cont.)}

\begin{nrtc}
\item	THE TCP PORT SPACE IS TOO LIMITED

\item	THE SIMPLE SOLUTION:\\
	USE A SINGLE HARD-WIRED MAGIC-BOX PORT FOR THE TCP
\end{nrtc}
\end{bwslide}


\begin{bwslide}
\ctitle	{ISSUE: EXPEDITED DATA}

THREE WAYS TO TRY IT:
\begin{nrtc}
\item	ONE TCP CONNECTION\\
	BORDERLINE COMPLIANCE

\item	ONE TCP CONNECTION WITH URGENCY TO SIGNAL EXPEDITED DATA

\item	TWO TCP CONNECTIONS, ONE WITH BETTER IP QOS\\
	COMPLICATED PROTOCOL NEEDED TO GUARANTEE COMPLIANCE
\end{nrtc}
\end{bwslide}


\begin{bwslide}
\ctitle	{EXPEDITED DATA (cont.)}

\begin{nrtc}
\item	NOT ALL TCP IMPLEMENTATIONS CORRECTLY HANDLE URGENCY IN THE
	DEGENERATE CASES

\item	NOT ALL IP IMPLEMENTATIONS ACTUALLY IMPLEMENT QOS

\item	THE SIMPLE SOLUTION:\\
	USE A SINGLE CONNECTION SINCE THIS IS THE LEAST COMPLEX CHOICE
\end{nrtc}
\end{bwslide}


\begin{bwslide}
\ctitle	{COMPARISON TO OTHER APPROACHES}

\begin{nrtc}
\item	THE ARCHIVAL REFERENCE: [PGREE86]

\item	PROTOCOL TRANSLATION: \tcptpgw/ [IGROE86]
    \begin{nrtc}
    \item	ANALYZE ESMs FOR EACH
    \item	IDENTIFY SUBSET OF COMMON SERVICES
    \item	BUILD ESM FOR MAGIC-BOX
    \end{nrtc}
\end{nrtc}
\end{bwslide}


\begin{bwslide}
\ctitle	{WHAT IS THE PRACTICAL VALUE?}

\begin{nrtc}
\item	STILL NO COMMONALITY FOR APPLICATIONS
    \begin{nrtc}
    \item	DDN APPLICATIONS STILL WANT TCP SERVICES\\
		SO CAN'T RUN DDN STUFF IN THE ISO WORLD

    \item	ISO APPLICATIONS STILL WANT ISO SERVICES\\
		SO CAN'T RUN ISO STUFF IN THE DDN WORLD
    \end{nrtc}

\item	ONE WORLD HAS TO IMPLEMENT THE OTHER WORLD'S STACK
\end{nrtc}
\end{bwslide}


\begin{note}\em
but, isn't this criticism also true of our work?

yes.
\end{note}


\begin{bwslide}
\part	{A DIGRESSION ON THE INTEROPERABILITY OF APPLICATIONS}\bf

CAN WE DO EITHER OF THESE?
\begin{nrtc}
\item	ACHIEVE INTEROPERABILITY BETWEEN SIMILAR APPLICATIONS\\ (e.g., MAIL)

\item	MOVE AN APPLICATION FROM ONE PROTOCOL SUITE TO ANOTHER
\end{nrtc}
\end{bwslide}


\begin{bwslide}
\ctitle	{ABSTRACT VIEW OF AN ENTITY}

\diagram[p]{figure7}
\end{bwslide}


\begin{bwslide}
\ctitle	{APPROACH \#1: BUILD AN APPLICATION MAGIC-BOX}

PROBLEM: SERVICES OFFERED USUALLY VARY DRAMATICALLY

\vspace{0.25in}
\diagram[p]{figure3}
\vspace{0.25in}

E.G., MAIL, CONSIDER [SKILL86]
\end{bwslide}


\begin{note}\em
The acid test is moving data through the magic-box and back again w/o loss
of information

padlipsky: ``sometimes when you try to turn an apple into an orange you get
back a lemon''
\end{note}


\begin{bwslide}
\ctitle	{APPROACH \#2: MIGRATE THE APPLICATION}

PROBLEM: SERVICES REQUIRED USUALLY VARY DRAMATICALLY

\vspace{0.25in}
\diagram[p]{figure4}
\end{bwslide}


\begin{bwslide}
\ctitle	{THE RECURRING THEME}

GENERAL UTILITY REQUIRES THAT PROTOCOL CONVERSION OCCUR AT EVERY LAYER
IN WHICH THE SUITES CAN BE CONNECTED
\end{bwslide}


\begin{bwslide}
\ctitle	{THE RECURRING THEME (cont.)}

SO TO INTEROPERATE MAIL (FOR EXAMPLE), WE NEED ONE OF:
\begin{nrtc}
\item	SMTP IN BOTH WORLDS

\item	P1 IN BOTH WORLDS

\item	SMTP AND P1 RUNNING IN THE \tcptpgw/\\
	(REALLY AN APPLICATION RELAY)
\end{nrtc}
IN ADDITION TO THE \tcptpgw/
\end{bwslide}


\begin{note}\em
all three choices are hard from above (services offered)

all three choices are hard from below (services required)
\end{note}


\begin{bwslide}
\ctitle	{DOES INTERFACE TRANSLATION HELP?}

\begin{nrtc}
\item	BOTH GIVE THE SAME END-RESULT

\item	\underline{INTERFACE} TRANSLATION REQUIRES SIMILAR FUNCTIONALITY
	BETWEEN THE TWO \underline{SERVICES} IN QUESTION

\item	\underline{PROTOCOL} TRANSLATION REQUIRES SIMILAR FUNCTIONALITY\\
	BETWEEN THE TWO \underline{PROTOCOLS} IN QUESTION

\item	HENCE, OPTIMALITY DEPENDS ON CONTEXT
\end{nrtc}
\end{bwslide}


\begin{note}\em
in other words,
neither approach makes applications interoperate

there is no free lunch!
\end{note}


\begin{bwslide}
\ctitle	{BENEFITS IN OUR CONTEXT}

\begin{nrtc}
\item	SHORT-TERM: EASY TO IMPLEMENT

\item	MEDIUM-TERM:
    \begin{nrtc}
    \item	ANY FUTURE WORK IS DONE IN ONE STACK, BUT WILL RUN IN BOTH
		WORLDS

    \item	APPLICATION DESIGNERS CAN USE AN ISO-APPLICATIONS\\ FRAMEWORK
		IN THE NUMEROUS EXISTING TCP/IP WORLDS
    \end{nrtc}

\item	LONG-TERM: PROVIDES THE BASIS FOR A MIGRATION STRATEGY
\end{nrtc}
\end{bwslide}


\begin{bwslide}
\part	{A MIGRATION STRATEGY}\bf

\begin{nrtc}
\item	THREE PHASES FROM THE DDN SUITE TO THE ISO SUITE

\item	ASSUMES AN EXISTING (AND HOPEFULLY) EXTENSIVE TCP/IP\\
	INTERNET IN PLACE

\item	REQUIRES ALL NEW HOSTS TO SPEAK TCP/IP UNTIL PHASE THREE

\end{nrtc}
\end{bwslide}


\begin{bwslide}
\ctitle	{PHASE ONE:\\ BUILD ISO DEVELOPMENT ENVIRONMENT}

\begin{nrtc}
\item	BEGIN WORKING ON ISO APPLICATIONS

\item	MAGIC-BOX OFFERS TP4 SERVICE

\item	DEVELOP DDN/ISO USER AGENTS
    \begin{nrtc}
    \item	COMMON USER-INTERFACE

    \item	USE EITHER DDN OR ISO APPLICATION SERVICE, AS AVAILABLE

    \item	NAME(DIRECTORY) SERVICE DETERMINES CHOICE OF STACK
    \end{nrtc}
\end{nrtc}
\end{bwslide}


\begin{note}\em
for example, the symbolics filesystem interface
\end{note}


\begin{bwslide}
\ctitle	{PHASE TWO:\\ EXPERIMENT WITH MIGRATION ENGINES}

\begin{nrtc}
\item	START USING HOSTS WITH BOTH ISO AND DDN STACKS

\item	TEST APPLICATIONS IN A ``PURE'' ISO ENVIRONMENT

\item	DO IP-LEVEL ROUTERS TO FORM TWO LOGICAL INTERNETS
\end{nrtc}
\end{bwslide}


\begin{bwslide}
\ctitle	{PHASE THREE:\\ DEPLOY MIGRATION ENGINES}

\begin{nrtc}
\item	RELATIVELY INEXPENSIVE (AT FIRST) TO KEEP SOME DDN-ONLY HOSTS

\item	USER AGENTS BEGIN TO SPEAK ISO ONLY

\item	NEW HOSTS CAN BE ISO ONLY
\end{nrtc}
\end{bwslide}


\begin{bwslide}
\ctitle {LAN--BASED MIGRATION TO NATIVE ISO}
\vskip-0.5in
\diagram[p]{figure6}
\end{bwslide}


\begin{note}\em
so, our plan is to attack things from the top,
while others attack from the bottom$\ldots$
\end{note}


\begin{bwslide}
\part	{THE FIRST STEP:\\ ISODE}\bf

\begin{nrtc}
\item	AN OPENLY AVAILABLE ISO DEVELOPMENT
	ENVIRONMENT HAS BEEN IMPLEMENTED AT NRTC

\item	CODED ENTIRELY IN C
\end{nrtc}
\end{bwslide}


\begin{bwslide}
\ctitle	{OPERATING ENVIRONMENTS}

\begin{nrtc}
\item	4.2\bsd/ \unix/
\item	SVR2 AT\&T \unix/ WITH AN EXCELAN \exos/~8044 TCP/IP PACKAGE
\item	\vax//\vms/ RELEASE 4.4 WITH AN \exos/ CARD (UNDER DEVELOPMENT)
\item	\pcdos/ WITH THE MIT PC-IP SOFTWARE (UNDER NEGOTIATION)
\end{nrtc}

\end{bwslide}


\begin{bwslide}
\ctitle	{SOFTWARE}

\begin{nrtc}
\item	TRANSPORT: IMPLEMENTS VERSION~2 OF THE MAGIC-BOX PROTOCOL

\item	SESSION: BCS, BAS, BSS, EXPEDITED

\item	PRESENTATION: ASN.1 ENCODING


\item	APPLICATION: 
    \begin{nrtc}
    \item	ROS (REMOTE OPERATIONS)

    \item	RTS (RELIABLE TRANSFER)

    \item	ASN.1 SPECIFICATION PARSER FOR THE AUTOMATIC GENERATION OF
		APDU PARSERS
    \end{nrtc}

\item	PLANNED FOR THE NEXT RELEASE:
    \begin{nrtc}
    \item	MAP/TOP VERSION~3.0 COMPATIBILITY\\
		(WHEN THAT STABILIZES)

    \item	APPLICATION SERVICE ELEMENTS (ASE) SUPPORT

    \item	ISO PRESENTATION PROTOCOL
    \end{nrtc}
\end{nrtc}
\end{bwslide}


\begin{bwslide}
\ctitle	{PERFORMANCE OBSERVATIONS}

\begin{nrtc}
\item	ALTHOUGH NOT PRODUCTION SOFTWARE,
	CODED WITH AN EYE\\ TOWARD EFFICIENCY

\item	INITIAL BENCHMARKING SUGGESTS THROUGHPUT RATES VERY CLOSE TO RAW TCP
	FOR BOTH TRANSPORT AND SESSION ECHO AND SINK ENTITIES

\item	AT THE APPLICATION INTERFACE,
	PERFORMANCE IS ONLY 10\%-12\% WORSE THAN RAW TCP

\item	RESULTS PRIMARILY DUE TO MINIMIZED BYTE-COPYING BETWEEN\\
	LAYERS
\end{nrtc}
\end{bwslide}


\begin{bwslide}
\ctitle	{FOR FURTHER READING}

\begin{nrtc}
\item	REQUEST FOR COMMENTS 983 [DCASS86]

\item	ISO TRANSPORT SERVICES ON TOP OF THE TCP\\
	COMPUTER NETWORKS AND ISDN SYSTEMS JOURNAL (TO APPEAR)

\item	REQUEST FOR COMMENTS 987 [SKILL86]\\
	MAPPING BETWEEN X.400 AND RFC822

\item	PROTOCOL CONVERSION [PGREE86]\\
	IEEE TRANSACTIONS ON COMMUNICATION\\
	VOLUME 34, NUMBER 3, MARCH 1986

\item	CONVERSION BETWEEN THE TCP AND ISO TRANSPORT$\ldots$
	[IGROE86]\\
	IEEE JOURNAL ON SELECTED AREAS IN COMMUNICATIONS\\
	VOLUME 4, NUMBER 2, MARCH 1986

\item	MOVING FROM DOD TO ISO PROTOCOLS: A FIRST STEP [MWITT86]\\
	ACM COMPUTER COMMUNICATIONS REVIEW\\
	VOLUME 16, NUMBER 2, APRIL/MAY 1986
\end{nrtc}
\end{bwslide}


\end{document}
