% -*- LaTeX -*-		(really SLiTeX)

\begin{bwslide}
\part	{REVIEW OF\\ BACKGROUND MATERIAL}\bf

\begin{nrtc}
\item	THE OSI MODEL

\item	THE UPPER-LAYER ARCHITECTURE

\item	SERVICES AND SERVICE PRIMITIVES
\end{nrtc}
\end{bwslide}


\begin{note}\em
everyone in the audience should be comfortable the material in this modest
review (and perhaps even bored, which is fine)
\end{note}


\begin{bwslide}
\part*	{THE OSI MODEL}\bf

\begin{nrtc}
\item	A LAYERED ARCHITECTURE FOR COMPUTER COMMUNICATIONS

\item	STANDARDIZED IN THE INTERNATIONAL COMMUNITY

\item	NON-PROPRIETARY IN NATURE
\end{nrtc}
\end{bwslide}


\begin{bwslide}
\ctitle	{THE MODEL FROM A COMMUNICATIONS VIEWPOINT}

\vskip.5in
\diagram[p]{figure1}
\end{bwslide}


\begin{bwslide}
\ctitle	{THE MODEL FROM A COMPUTER VIEWPOINT}

\vskip.5in
\diagram[p]{figure2}
\end{bwslide}


\begin{bwslide}
\ctitle	{(OBLIGATORY SLIDE SHOWING)\\ THE 7--LAYER STACK}

\vskip.5in
\diagram[p]{figure3}
\end{bwslide}


\begin{bwslide}
\part*	{THE UPPER-LAYER ARCHITECTURE}\bf

\begin{nrtc}
\item	BY ``UPPER-LAYER'' WE MEAN EVERYTHING ABOVE TRANSPORT:
    \begin{nrtc}
    \item	THE APPLICATION-SPECIFICS OF HOW THE NETWORK IS USED
    \end{nrtc}

\item	UNLIKE OTHER ARCHITECTURES THE SAME UPPER-LAYERS ARE USED
	REGARDLESS OF THE APPLICATION

\item	WHAT DIFFERS IS THE ACTUAL FUNCTIONALITY USED BY THE APPLICATION
\end{nrtc}
\end{bwslide}


\begin{note}\em
it's not clear at this point the effect of connectionless-mode operation on
the upper-layer architecture
\end{note}


\begin{bwslide}
\ctitle	{THE UPPER-LAYER ARCHITECTURE (cont.)}

\vskip.15in
\diagram[p]{figure4}
\end{bwslide}


\begin{bwslide}
\ctitle	{THE OSI APPLICATION LAYER}

\begin{nrtc}
\item	MANY STANDARD SERVICE ELEMENTS
    \begin{nrtc}
    \item	ASSOCIATION CONTROL

    \item	REMOTE OPERATIONS

    \item	RELIABLE TRANSFER

    \item	COMMITMENT, CONCURRENCY AND RECOVERY

    \item	DIRECTORY SERVICES
    \end{nrtc}

\item	ABSTRACT SYNTAX NOTATION ONE (ASN.1)\\
	(really a concept not an element, per se)

\item	THE DISTINCTION WILL BE DISCUSSED LATER ON
\end{nrtc}
\end{bwslide}


\begin{bwslide}
\ctitle	{APPLICATION USE OF UPPER-LAYER SERVICES}

\vskip.5in
\diagram[p]{figure5}
\end{bwslide}


\begin{bwslide}
\ctitle	{APPLICATION SERVICE ELEMENTS}

\begin{nrtc}
\item	A USEFUL MECHANISM FOR DIVIDING RESPONSIBILITY OF THE ``TOTAL''
	APPLICATION PROTOCOL

\item	PROMOTES ``REUSE'' OF APPLICATION LAYER FACILITIES
\end{nrtc}
\end{bwslide}


\begin{bwslide}
\ctitle	{EXAMPLE:\\ FTAM USE OF LOWER-LAYER SERVICES}

\vskip.5in
\diagram[p]{figure24}
\end{bwslide}


\begin{bwslide}
\part*	{SERVICES AND\\ SERVICE PRIMITIVES}\bf

\begin{nrtc}
\item	PEERS COMMUNICATE VIA \emph{SERVICE PRIMITIVES}

\item	A PRIMITIVE IS AN ABSTRACTION
    \begin{nrtc}
    \item	NOT AN INTERFACE
    \end{nrtc}

\item	SERVICE PRIMITIVES, LIKE PROCEDURE CALLS, HAVE TYPED PARAMETERS
\end{nrtc}
\end{bwslide}


\begin{bwslide}
\ctitle	{SERVICES vs. PROTOCOLS}

\vskip.5in
\diagram[p]{figure23}
\end{bwslide}


\begin{bwslide}
\ctitle	{SERVICE}

\begin{nrtc}
\item	IN GENERAL, THERE ARE THREE KINDS OF SERVICES
    \begin{nrtc}
    \item	\emph{CONFIRMED}
	\begin{nrtc}
	\item	IN WHICH A REQUEST ALWAYS RESULTS IN A RESPONSE
	\end{nrtc}

    \item	\emph{UNCONFIRMED}
	\begin{nrtc}
	\item	IN WHICH NO RESPONSE IS RETURNED
	\end{nrtc}

    \item	\emph{PROVIDER-INITIATED}
	\begin{nrtc}
	\item	IN WHICH THE SERVICE PROVIDER INDICATES SOME SITUATION
	\end{nrtc}
    \end{nrtc}

\item	CONFIRMATION IS UNRELATED TO RELIABILITY
\end{nrtc}
\end{bwslide}


\begin{bwslide}
\ctitle	{SERVICE PRIMITIVES}

\begin{nrtc}
\item	EACH LAYER (OR ELEMENT) OFFERS ONE OR MORE SERVICES
    \begin{nrtc}
    \item	e.g., A-ASSOCIATE
    \end{nrtc}

\item	A SERVICE CONSISTS OF ONE OR MORE PRIMITIVES

\item	A CONFIRMED SERVICE HAS FOUR PRIMITIVES
    \begin{nrtc}
    \item	.REQUEST, .INDICATION, .RESPONSE, and .CONFIRMATION
    \end{nrtc}

\item	AN UNCONFIRMED SERVICE HAS TWO PRIMITIVES:
    \begin{nrtc}
    \item	.REQUEST,  and .INDICATION
    \end{nrtc}

\item	A PROVIDER-INITIATED SERVICE HAS ONE PRIMITIVE:
    \begin{nrtc}
    \item	.INDICATION
    \end{nrtc}
\end{nrtc}
\end{bwslide}


\begin{bwslide}
\ctitle	{EXAMPLE: CONFIRMED SERVICE}

\vskip.5in
\diagram[p]{figure6}
\end{bwslide}


\begin{bwslide}
\ctitle	{EXAMPLE: CONFIRMED SERVICE (cont.)}

\vskip.5in
\diagram[p]{figure30}
\end{bwslide}
