% -*- LaTeX -*-		(really SLiTeX)

\begin{bwslide}
\part	{WHAT NOW?}\bf

\begin{nrtc}
\item	COMPARISON TO SUN RPC/XDR

\item	COMPARISON TO APOLLO NCS
\end{nrtc}
\end{bwslide}


\begin{bwslide}
\ctitle	{GUIDELINES}

\begin{nrtc}
\item	NOT TRYING TO SAY WHICH IS BETTER

\item	MERELY TRYING TO COMPARE AND CONTRAST

\item	REFERENCE DOCUMENT IS USED AS BASELINE\\
	(SOME INFORMATION MAY BE DATED)
\end{nrtc}
\end{bwslide}


\begin{bwslide}
\part*	{COMPARISON TO\\ SUN RPC/XDR}\bf

\begin{nrtc}
\item	ALTHOUGH NOT THE FIRST RPC SYSTEM DEPLOYED,
	CERTAINLY THE FIRST ``POPULARIZATION'' OF AN RPC SYSTEM

\item	SUN RPC/XDR IS BEST (UN)KNOWN FOR MAKING NFS POSSIBLE
\end{nrtc}
\end{bwslide}


\begin{bwslide}
\ctitle	{REFERENCE DOCUMENT}

\begin{nrtc}
\item	REMOTE PROCEDURE CALL PROGRAMMING GUIDE
    \begin{nrtc}
    \item	VERSION: REVISION B OF 17 FEBRUARY 1986

    \item	SOURCE: SMI DOCUMENTATION SET
    \end{nrtc}
\end{nrtc}
\end{bwslide}


\begin{bwslide}
\ctitle	{SYNTAX CHARACTERISTICS}

\begin{nrtc}
\item	NO FORMAL ABSTRACT SYNTAX, PER SE
    \begin{nrtc}
    \item	APPLICATION PROTOCOL DEFINES DATA STRUCTURES EXCHANGED

    \item	INITIALLY, NO STUB COMPILER\\ (THERE IS ONE NOW)
    \end{nrtc}

\item	SERIALIZATION METHOD IS CALLED XDR, EXTERNAL DATA REPRESENTATION
    \begin{nrtc}
    \item	CANONICAL FORM WITH IMPLICIT TAGS

    \item	MOST QUANTITIES PADDED TO 32--BIT BOUNDARIES

    \item	ANY DATA TYPE CAN BE SERIALIZED
    \end{nrtc}
\end{nrtc}
\end{bwslide}


\begin{bwslide}
\ctitle	{PROTOCOL CHARACTERISTICS}

\begin{nrtc}
\item	PROTOCOL IS SIMPLE REQUEST/REPLY INTERACTION
    \begin{nrtc}
    \item	BY CONVENTION, EACH APPLICATION HAS A ``NULL'' PROCEDURE    
    \end{nrtc}

\item	MULTIPLE TRANSPORT PROTOCOLS SUPPORTED
    \begin{nrtc}
    \item	ALTHOUGH UDP (DoD USER DATAGRAM PROTOCOL) IS THE MOST COMMON

    \item	THIS IMPACTS, e.g., THE SIZE OF ARGUMENTS THAT CAN BE PASSED
		IN A REQUEST
    \end{nrtc}

\item	SUPPORT FOR BROADCAST MEDIA
\end{nrtc}
\end{bwslide}


\begin{bwslide}
\ctitle	{BINDING CHARACTERISTICS}

\begin{nrtc}
\item	SERVICE IS IDENTIFIED BY
    \begin{nrtc}
    \item	PROGRAM NUMBER (32--BITS)

    \item	VERSION NUMBER (32--BITS)
    \end{nrtc}

\item	MAPPING OF SERVICE TO NETWORK ADDRESS IS DONE THROUGH PORT MAPPER

\item	SUPPORT FOR DIFFERENT AUTHENTICATION SCHEMES
\end{nrtc}
\end{bwslide}


\begin{bwslide}
\part*	{COMPARISON TO\\ APOLLO NCS}\bf

\begin{nrtc}
\item	ANOTHER ENTRY INTO THE POPULAR RPC MARKET

\item	EMPHASIZES OBJECT-BASED ABSTRACTIONS
\end{nrtc}
\end{bwslide}


\begin{bwslide}
\ctitle	{REFERENCE DOCUMENT}

\begin{nrtc}
\item	NETWORK COMPUTING SYSTEM: A TECHNICAL OVERVIEW
    \begin{nrtc}
    \item	VERSION: FEBRUARY 1987 (DOCUMENT 002402--322)

    \item	SOURCE: APOLLO WHITE PAPER
    \end{nrtc}
\end{nrtc}
\end{bwslide}


\begin{bwslide}
\ctitle	{SYNTAX CHARACTERISTICS}

\begin{nrtc}
\item	A FORMAL SYNTAX IS USED
    \begin{nrtc}
    \item	NETWORK INTERFACE DEFINITION LANGUAGE (NIDL)

    \item	ANY DATA TYPE CAN BE DESCRIBED

    \item	NIDL COMPILER PRODUCES ``C'' AND ``PASCAL'' BINDINGS
    \end{nrtc}

\item	SERIALIZATION (APOLLO CALLS IT \emph{MARSHALLING}) IS BASED ON THE
    \begin{nrtc}
    \item	``RECEIVER MAKES IT RIGHT''
    \end{nrtc}
    PRINCIPLE
\end{nrtc}
\end{bwslide}


\begin{bwslide}
\ctitle	{PROTOCOL CHARACTERISTICS}

\begin{nrtc}
\item	A LIGHTWEIGHT TRANSACTION PROTOCOL IS USED
    \begin{nrtc}
    \item	CAN DISTINGUISH (NON-)IDEMPOTENT OPERATIONS
    \end{nrtc}

\item	MULTIPLE TRANSPORT PROTOCOLS SUPPORTED
    \begin{nrtc}
    \item	EMPHASIZING USE OF THE BSD SOCKET ABSTRACTION
    \end{nrtc}

\item	SUPPORT FOR MULTI-TASKING
\end{nrtc}
\end{bwslide}


\begin{bwslide}
\ctitle	{BINDING CHARACTERISTICS}

\begin{nrtc}
\item	SERVICE (OBJECT) IS IDENTIFIED BY \emph{UNIQUE ID} OR
	\emph{INTERFACE NAME} 
    \begin{nrtc}
    \item	REPLICATION AND CONSISTENCY IS CONSIDERED
    \end{nrtc}

\item	MAPPING OF SERVICE TO NETWORK ADDRESS IS DONE THROUGH LOCATION BROKER

\item	SUPPORT FOR DIFFERENT AUTHENTICATION SCHEMES

\item	SUPPORT FOR AUTHORIZATION PLANNED
\end{nrtc}
\end{bwslide}
