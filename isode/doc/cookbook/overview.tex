% run this through LaTeX

\input lcustom

\documentstyle[12pt,tgrind]{article}

\begin{document}

\title{Building Distributed\\ Applications in an\\ OSI Framework}
\author{Marshall T.~Rose\\ The Wollongong Group}
\date{April 27, 1988}
\maketitle

Included in this tutorial are a copy of the presentation notes,
a draft of {\em The Applications Cookbook},
and this brief overview.

The tutorial is divided into six parts:
\begin{itemize}
\item	Review of Background Material\\
This presents an elementary review of the Open Systems Interconnection Model.
In particular,
we focus on the upper-layer architecture and the service elements found in the
application layer.
Following this,
a brief discussion of OSI modeling nomenclature takes place.

\item	A Model for Distributed Applications\\
This presents a formal model for how distributed applications are organized.
The fundamental concept is that of the {\em abstract data type\/} which
seperates the ``what'' from the ``how''.
The notion of an {\em operation\/} is then introduced as the means by which
this level of indirection can be introduced.
Following this,
a brief discussion of associations takes place.
Finally,
some design guidelines for applications built using this model are considered.

\item	Underlying Services\\
This presents the underyling facilities which OSI makes available to the
distributed application:
{\em abstract syntax notation one\/} (ASN.1),
which provides a means for describing data structures in a machine-independent
fashion;
the {\em remote operations\/} service,
which provides the rules for requesting actions to be performed elsewhere in
the network;
and,
the {\em binding\/} service,
which provides the mechanisms for establishing associations.
Given these facilities,
the problem of organizing them into a solution is considered.

\item	Static Facilities\\
This presents the toolkit used for building distributed applications.
First,
a review of remote operations specifications takes place.
Then,
the three tools,
a stub generator, a structure generator, and an element parser are discussed.

\item	Dynamic Facilities\\
This presents the support libraries used for building distributed applications.
First,
the run-time environment is discussed,
then boilerplate for initiators and responders are considered.
Finally,
the administrative details of defining a new service are examined.

\item	What Now?\\
A comparison is made to two ``popular'' rpc systems.
\end{itemize}

Throughout the tutorial,
the ISO network management specification is used as an example.
Starting on the next page,
the actual source used for the examples is shown.

\vspace{0.25in}
{\raggedleft /mtr\par}
{\raggedright Palo Alto, California\\
April, {\oldstyle\number\year}\par}

\newpage
\tgrindfile{cmip-bind}

\newpage
\tgrindfile{cmip-rosy}

%%% \newpage
%%% \tgrindfile{cmip-posy}

%%% \newpage
%%% \tgrindfile{cmip-pepy}

\end{document}
