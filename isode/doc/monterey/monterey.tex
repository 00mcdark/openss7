% run this through SLiTeX

\documentstyle
    [blackandwhite,landscape,oval,pagenumbers,small,plain]{NRslides}

\input trademark
\def\tradeORGfont{\rm}
\def\tradeNAMfont{\rm}

\raggedright

\begin{document}

\title	{PROTOCOL ADVANCES 3:\\ ISO DEVELOPMENT ENVIRONMENT}
\author	{Stephen E.~Kille\\ University College, London\\[0.15in]
	Ronald G.~Minnich\\ University of Delaware\\[0.15in]
	Marshall T.~Rose\\ Northrop Corporation}
\date	{March 17, 1987}
\maketitlepage


\begin{bwslide}
\part*	{AGENDA}\bf

\begin{nrtc}
\item	ISODE: INTRODUCTION AND STATUS REPORT (ROSE)

\item	PUTTING ISODE TO WORK (KILLE)

\item	ISODE AT THE UNIVERSITY OF DELAWARE (MINNICH)

\item	A STRATEGY FOR CONVERGENCE WITH ISO\\
	(ANOTHER SERMON FROM MT.~ROSE?)
\end{nrtc}
\end{bwslide}


\begin{bwslide}
\part	{ISODE:\\ INTRODUCTION AND\\ STATUS REPORT}
\end{bwslide}


\begin{bwslide}
\ctitle	{ISODE}

\begin{nrtc}
\item	AN OPENLY AVAILABLE ISO DEVELOPMENT ENVIRONMENT

\item	ISO APPLICATION, PRESENTATION, SESSION, AND LAYERED TRANSPORT

\item	CODED ENTIRELY IN C

\item	OPERATING SYSTEMS
    \begin{nrtc}
    \item	4.2\bsd/ \unix/

    \item	SVR2 AT\&T \unix/ WITH AN EXCELAN \exos/~8044 TCP/IP PACKAGE

    \item	\vms/ AND \pcdos/ (STILL) UNDER DEVELOPMENT
    \end{nrtc}
\end{nrtc}
\end{bwslide}


\begin{bwslide}
\ctitle	{MOTIVATION}

\begin{nrtc}
\item	WANT TO BE CONSISTENT WITH ISO'S DIRECTION,
	BUT WANT TO GET WORK DONE NOW
    \begin{nrtc}
    \item	MANY NEW, MAJOR INVESTMENTS BEING MADE IN CURRENT TECHNOLOGY
		(e.g., NSFnet, NASA's NEW INTERNET, etc.)
    \end{nrtc}

\item	CURRENTLY, TCP/IP HAS SEVERAL ADVANTAGES OVER TP4/IP:
    \begin{nrtc}
    \item	WORKING IS-IS (GATEWAY-GATEWAY) PROTOCOL

    \item	MATURITY

    \item	VENDOR SUPPORT

    \item	LARGE BODY OF EXPERTISE
    \end{nrtc}

\item	WOULD LIKE TO WORK IN AN ISO ENVIRONMENT,
	BUT WILL USE TCP/IP's STRENGTHS TO DO SO NOW
\end{nrtc}
\end{bwslide}


\begin{bwslide}
\ctitle	{THE APPLICATION ENVIRONMENT}

\vskip.15in
\diagram[p]{figure1}
\end{bwslide}


\begin{note}\em
other ASEs: RTSE, CCR, and so on

presentation: manage presentation contexts~---~abstract syntax and transfer

session: manage tokens, activities, checkpointing, and so on

about 35K lines of code
\end{note}


\begin{bwslide}
\ctitle	{AN ALTERNATE ENVIRONMENT:\\ MHS ARCHITECTURE (c.~1984)}

\vskip.15in
\diagram[p]{figure2}
\end{bwslide}


\begin{bwslide}
\ctitle	{APPLICATIONS IN PROGRESS}

\begin{nrtc}
\item	FTAM - FILE TRANSFER, ACCESS AND MANAGEMENT (NRTC)

\item	MHS - MESSAGE HANDLING SYSTEM (UCL)

\item	DS - DIRECTORY SERVICES (UCL)
\end{nrtc}
\end{bwslide}


\begin{bwslide}
\ctitle	{PERFORMANCE OBSERVATIONS}

\begin{nrtc}
\item	THE 5-P PRINCIPLE:\\
	PROPER PLANNING PREVENTS POOR PERFORMANCE

\item	INITIAL BENCHMARKING SUGGESTS THROUGHPUT RATES VERY CLOSE TO RAW TCP
	FOR BOTH TRANSPORT AND SESSION ECHO AND SINK ENTITIES

\item	AT THE APPLICATION INTERFACE (ABOVE ACSE/ROSE),
	THROUGHPUT IS ONLY 10\%-12\% WORSE THAN RAW TCP FOR DATA TRANSFER

\item	RESULTS PRIMARILY DUE TO MINIMIZED BYTE-COPYING BETWEEN LAYERS
\end{nrtc}
\end{bwslide}


\begin{bwslide}
\ctitle	{WHERE NEXT?}

\begin{nrtc}
\item	VALIDATE/TEST AGAINST PURE ISO\\
	(SOMEBODY ELSE'S IMPLEMENTATION)

\item	SYNCHRONIZE WITH GOSIP SPECIFICATION

\item	EXPAND SOME MODULES AS NEW APPLICATIONS REQUIRE

\item	CONVERGENCE WORK (DESCRIBED LATER)
\end{nrtc}
\end{bwslide}


\begin{bwslide}
\ctitle	{AVAILABILITY INFORMATION}

\begin{nrtc}
\item	VERSION 2 AVAILABLE MARCH 15, 1987

\item	USPS: SEND TAPE AND PREPAID MAILER TO:
\begin{small}
    \[\begin{tabular}{l}
	NORTHROP RESEARCH AND TECHNOLOGY CENTER\\
	ATTN: AUTOMATION SCIENCES LABORATORY (0330/T30)\\
	ONE RESEARCH PARK\\
	PALOS VERDES PENINSULA, CA  90274\\
	USA\\
    \end{tabular}\]
\end{small}
    \begin{nrtc}
    \item	ADD 3 POUNDS AND 1--1/2 INCHES FOR DOCUMENTATION SET

    \item	SEND ONLY POSTAGE, NO MONEY

    \item	TELCO: 213--544--5393
    \end{nrtc}

\item	ANONYMOUS FTP: HOST louie.udel.edu, FILE portal/isode-2.tar
\end{nrtc}
\end{bwslide}


\begin{bwslide}
\part	{PUTTING ISODE TO WORK}\large\bf

\vskip-0.5in
\[\begin{tabular}[t]{c}\large\bf
    Stephen E.~Kille\\
    Department of Computer Science\\
    University College, London
\end{tabular}\]
\end{bwslide}


\begin{note}\em
with credits to:

\begin{nrtc}
\item	at UCL:\\
	George G.~Michaelson, Stephen E.~Easterbrook, Thomas Woo

\item	at the Department of Computer Science, Nottingham University:\\
	Julian P.~Onions
\end{nrtc}
\end{note}


\begin{bwslide}
\ctitle	{OVERVIEW}

\begin{nrtc}
\item	DISCUSSION OF WORK AT UCL USING ISODE

\item	WORK ON DISTRIBUTED APPLICATIONS

\item	EMPHASIS ON MESSAGE HANDLING AND DIRECTORY SERVICES

\item	FOCUS ON WORK ALIGNED WITH INTERNATIONAL STANDARDS

\item	AIM TO DESCRIBE HOW ISODE FACILITATES THIS WORK
\end{nrtc}
\end{bwslide}


\begin{bwslide}
\ctitle	{OSI INFRASTRUCTURE}

\begin{nrtc}
\item	UCL RUNS A WIDE VARIETY OF UNIX SYSTEMS

\item	NEEDS OSI ENVIRONMENT WHICH CAN OPERATE ON ALL OF THESE,
	FOR WORKING IN BOTH LANs AND WANs

\item	CURRENTLY USE TCP/IP OVER THE LAN,
	AS THIS IS THE ONLY PROTOCOL COMMON TO ALL OF THE MACHINES IN QUESTION

\item	TP0/X.25 WILL BE USED FOR WAN ACCESS

\item	TP4 MAY ALSO BE USED (IF WE HAVE TO)
\end{nrtc}
\end{bwslide}


\begin{bwslide}
\ctitle	{ADVANTAGES OF TP0/X.25}

\begin{nrtc}
\item	THE PREFERRED EUROPEAN APPROACH

\item	EFFICIENT UTILIZATION OF PTT X.25 SERVICES

\item	UTILIZATION OF X.25 HARDWARE TO REDUCE CPU LOAD

\item	WILL ALLOW FOR EXTENSIVE TESTING OF ISODE AGAINST OTHER OSI
	IMPLEMENTATIONS AT UCL AND ELSEWHERE

\item	MIGRATION TO USE OF X.25 OVER IEEE~802 LLC
\end{nrtc}
\end{bwslide}


\begin{bwslide}
\ctitle	{ABSTRACT SYNTAX NOTATION 1 (ASN.1)}

\begin{nrtc}
\item	REPRESENTATION CURRENTLY USED BY ALL OSI APPLICATIONS

\item	RICH, EXTENSIBLE SYNTAX

\item	USEFUL FOR SPECIFICATION OF NEW PROTOCOLS
    \begin{nrtc}
    \item	CLEAR TO READ SPECIFICATIONS

    \item	NOT TIED TO MACHINE-ORIENTED STRUCTURES AND RESTRICTIONS
    \end{nrtc}
\end{nrtc}
\end{bwslide}


\begin{bwslide}
\ctitle	{REMOTE OPERATIONS SERVICE (ROS)}

\begin{nrtc}
\item	STANDARDIZED MECHANISM FOR SPECIFYING TRANSACTIONS

\item	MAKES FULL POWER OF ASN.1 AVAILABLE

\item	USED IN MANY INTERESTING OSI APPLICATIONS
    \begin{nrtc}
    \item	MESSAGING

    \item	DIRECTORY SERVICES

    \item	NETWORK MANAGEMENT

    \item	REMOTE DATABASE ACCESS
    \end{nrtc}
\end{nrtc}
\end{bwslide}


\begin{bwslide}
\ctitle{WHY ISODE}

\begin{nrtc}
\item	FULL AND UP-TO-DATE IMPLEMENTATION OF OSI LAYERS

\item	REMOTE OPERATIONS SERVICE

\item	ASN.1 ELEMENT HANDLING

\item	COMPILER FOR DECODING ASN.1 (PEPY)

\item	FLEXIBILITY TO USE DIFFERENT TRANSPORT SERVICES

\item	GOOD PERFORMANCE
\end{nrtc}
\end{bwslide}


\begin{bwslide}
\ctitle{RARE DIRECTORY SERVICES}

\begin{nrtc}
\item	R\'{E}SEAUX ASSOCI\'{E}S POUR LA RECHERCHE EUROP\'{E}ENNE (RARE)

\item	TRANSLATION: EUROPEAN ACADEMIC RESEARCH NETWORK

\item	AN ASSOCIATION OF THE VARIOUS NATIONAL RESEARCH NETS

\item	WISH TO PROVIDE EARLY DIRECTORY SERVICES, UTILIZING
	A CENTRAL DATABASE (LIKE THE ARPA ``WHOIS'')

\item	CONTAINS FACILITY, PROJECT, AND PERSON DATA

\item	THE DATA IS MORE VALUABLE THAN THE INITIAL SERVICE

\item	DATA STANDARD FORMAT IS DESIRED
\end{nrtc}
\end{bwslide}


\begin{bwslide}
\ctitle{RARE ASN.1 STRUCTURE}

\begin{nrtc}
\item	UCL HAS SPECIFIED A FIRST VERSIONS OF THE DATA STRUCTURE

\item	ASN.1 USED

\item	CAN REPRESENT DETAILED STRUCTURE, WHICH WILL BE USEFUL IN
	LATER DISTRIBUTED DIRECTORY SERVICES

\item	PEPY (ISODE) MADE ASN.1 VERIFICATION STRAIGHTFORWARD IN THE
	DESIGN PHASE

\item	ALLOWED EASY IMPLEMENTATION OF ``PRETTY PRINTER'' AND
	ENCODING OF TEST DATA

\item	IS LIKELY TO BE USED TO SUPPORT THE WIDER INTRODUCTION OF
	THIS FORMAT FOR ENCODING AND DECODING
\end{nrtc}
\end{bwslide}


\begin{bwslide}
\ctitle{NRS LOOKUP PROTOCOL}

\begin{nrtc}
\item	NAME REGISTRATION SCHEME (NRS) IS A DATABASE OF THE HOSTS
	(DOMAINS) IN THE UK ACADEMIC COMMUNITY.

\item	CURRENTLY CONTAINS ABOUT 1000 HOSTS, AND IS GROWING RAPIDLY

\item	THE CENTRALIZED DATABASE HAS DISTRIBUTED MANAGEMENT AND
	IS WIDELY REPLICATED

\item	NRS LOOKUP PROTOCOL SPECIFIES A LIGHTWEIGHT TRANSACTION
	OVER X.25, TO ENABLE LOOKUP OF INFORMATION IN THE NRS

\item	THE PACKET FORMATS ARE SPECIFIED IN ASN.1

\item	HANDLES BOTH CURRENT ``COLOURED BOOK'' APPLICATIONS AND
	PLANNED OSI APPLICATIONS
\end{nrtc}
\end{bwslide}


\begin{bwslide}
\ctitle{NRS LOOKUP PROTOCOL IMPLEMENTATION}

\begin{nrtc}
\item	IMPLEMENTATION DONE IN PARALLEL WITH FINAL WORK ON SPECIFICATION

\item	3RD YEAR STUDENT PROJECT (3 MONTHS)

\item	PEPY FOUND A NUMBER OF ERRORS IN THE PROTOCOL SPECIFICATION

\item	PEPY ALLOWED A FULL PROTOCOL DECODER TO BE BUILT WITH MINIMUM EFFORT

\item	EARLY RESULTS SUGGEST GOOD PERFORMANCE
\end{nrtc}
\end{bwslide}


\begin{bwslide}
\ctitle{MESSAGE HANDLING}

\begin{nrtc}
\item	UCL AND NOTTINGHAM UNIVERSITY ARE DEVELOPING AN X.400 SYSTEM (PP)

\item	OWES MANY OF ITS DESIGN IDEAS TO MMDF (THE CSNET MESSAGE SYSTEM)

\item	FLEXIBLE HANDLING OF MULTI-MEDIA

\item	PROTOCOL AND FORMAT CONVERSION

\item	UTILIZATION WITH DIRECTORY SERVICES

\item	MAY BE DISTRIBUTED WITH LATER VERSIONS OF ISODE
\end{nrtc}
\end{bwslide}


\begin{bwslide}
\ctitle{CURRENT USE OF ISODE IN PP}

\begin{nrtc}
\item	MOST OF THE EARLY WORK HAS NOT NEEDED ISODE

\item	PEPY CAN HANDLE P1 AND P2

\item	INTEGRATION OF P1 AND ISODE'S RELIABLE TRANSFER SERVICE (RTS) HAS
	STARTED AND HAS BEEN SATISFYINGLY STRAIGHTFORWARD
\end{nrtc}
\end{bwslide}


\begin{bwslide}
\ctitle{PLANNED USE OF ISODE IN PP}

\begin{nrtc}
\item	RFC987 WILL BE IMPLEMENTED, TO PROVIDE FULL MAPPINGS
	BETWEEN X.400 AND RFC822 MAIL

\item	QUEUE MANAGEMENT WILL UTILIZE A ROS PROTOCOL
    \begin{nrtc}
    \item	THIS WILL PROVIDE HIGH FUNCTIONALITY, AND ALLOW FOR REMOTE
		MANAGEMENT
    \end{nrtc}

\item	A SYSTEM FOR MANAGING LOCAL LISTS WILL BE SPECIFIED IN ROS
\end{nrtc}
\end{bwslide}


\begin{bwslide}
\ctitle{UCL DIRECTORIES}

\begin{nrtc}
\item	HAVE ESTABLISHED DATABASE OF UCL USERS IN CENTRAL DATABASE
    \begin{nrtc}
    \item	DATA SEEMS TO BE A PRE-REQUISITE OF A DIRECTORY SERVICE!
    \end{nrtc}

\item	ACCESS USING ROS TO ALLOW INFORMATION TO BE UPDATED AND QUERIED

\item	GENERATION OF LOCAL MAIL TABLES IS AUTOMATED FROM THIS DATABASE

\item	ALL UCL PASSWORD FILES ARE MANAGED FROM THIS DATABASE
  \begin{nrtc}\small
    \item	MUCH HARDER THAN WE THOUGHT

    \item	ALL PASSWORD FILE RELATED TOOLS (passwd(1), chsh(1), etc.)
		EMULATED OVER ROS

    \item	SOME EXTENSIONS
		(E.G., ABILITY TO CHANGE PASSWORDS ON ALL MACHINES)

    \item	PASSWORD FILES PULLED BY MACHINES USING ROS

    \item	UCL SPECIFIC (SIMPLE) APPROACH TO AUTHENTICATION
    \end{nrtc}

\item	CURRENT SYSTEM IS SEEN AS A STEPPING STONE TO FULL OSI
	DIRECTORY SERVICES
\end{nrtc}
\end{bwslide}


\begin{bwslide}
\ctitle{WORK AT UCL TO EXTEND ISODE}

\begin{nrtc}
\item	TOOLS TO FACILITATE DEVELOPMENT OF APPLICATIONS ARE SEEN AS CRITICAL

\item	CURRENTLY EXTENDING PEPY TO PERFORM ENCODING AS WELL AS DECODING

\item	WILL EXTEND PEPY TO SUPPORT ROS IN AN AUTOMATIC MANNER,
	MIXING ENCODING AND DECODING FUNCTIONS

\item	ENABLE ROS TO BE USED IN A MANNER VERY LIKE
	SOME REMOTE PROCEDURE CALL APPROACHES
\end{nrtc}
\end{bwslide}


\begin{bwslide}
\part	{ISODE AT\\ THE UNIVERSITY OF DELAWARE}\bf

\vskip-0.5in
\[\begin{tabular}[t]{c}\large\bf
    Ronald G.~Minnich\\
    Dept. of Electrical Engineering\\
    University of Delaware
\end{tabular}\hskip1em plus.17fil
\begin{tabular}[t]{c}\large\bf
    David J.~Farber\\
    Dept. of Electrical Engineering\\
    Dept. of Computer and Information Sciences\\
    University of Delaware
\end{tabular}\]
\end{bwslide}


\begin{bwslide}
\ctitle	{STANDARD DISCLAIMER}

\begin{nrtc}
\item	ISODE HAS JUST COME INTO USE AT UDEL AS OF 1987 

\item	WE KNOW ABOUT ISO BUT HAVE NEVER USED ISO OR ISODE~---
    \begin{nrtc}
    \item	WE ARE NOT ALONE IN THAT; THAT IS WHY ISODE EXISTS
    \end{nrtc}
\end{nrtc}
\end{bwslide}


\begin{bwslide}
\ctitle	{ISODE AFFECTS THREE AREAS}

\begin{nrtc}
\item	EXPERIMENTAL COMPUTER NETWORK RESEARCH

\item	SYSTEMS PROGRAMMING

\item	APPLICATIONS PROGRAMMING
\end{nrtc}
\end{bwslide}


\begin{bwslide}
\ctitle	{IMPLICATIONS OF ISO FOR OUR NETWORK RESEARCH}

\begin{nrtc}
\item	IF THE ISO MODEL IS THE FUTURE,
	THEN xxxNET HAD BETTER SUPPORT IT EFFECTIVELY

\item	WE HAVE SEEN A SIMILAR PHENOMENON WITH UNIX AND COMPUTER ARCHITECTURE
\end{nrtc}
\end{bwslide}

\begin{bwslide}
\ctitle	{EXPERIMENTAL COMPUTER NETWORK RESEARCH}

\begin{nrtc}
\item	MEMNET[DELP86]~---~LARGE PHYSICALLY DISTRIBUTED MEMORY CONNECTED BY A
	TOKEN RING
    \begin{nrtc}
    \item	IT IS UNDER CONSTRUCTION NOW AND SIMULATIONS PREDICT VERY HIGH
		THROUGHPUT

    \item	WE ARE CONSIDERING PUTTING ISODE ON TOP OF MEMNET;
		MEMNET BECOMES THE TRANSPORT

    \item	WE CONSIDER THIS A GOOD TEST OF MEMNET'S CAPABILITIES AS WELL
		AS BEING POTENTIALLY WORTHWHILE IN AND OF ITSELF
    \end{nrtc}

\item	NOAHNET[PARULKAR86]~---~FLOOD NETWORK
    \begin{nrtc}
    \item	WHAT IMPLICATIONS DOES THE ISO MODEL HAVE FOR NOAHNET?
    \end{nrtc}
\end{nrtc}
\end{bwslide}


\begin{bwslide}
\ctitle	{ISO FOR SYSTEMS PROGRAMMING}

\begin{nrtc}
\item	WE NEED TO LEARN HOW TO USE IT FOR `HOUSEKEEPING'\\
	(e.g. KERNAL DATA STRUCTURE MONITORS, FONT LOADERS,\\
	MAN PAGE PROGRAMS, etc.)

\item	WHERE WE HAVE BEEN USING THE UNSTRUCTURED ``PIPE''-LIKE CHANNEL
	PROVIDED BY TCP/IP, WE NOW USE THE HIGHER-LEVEL ISO CONSTRUCTS

\item	A LOGICAL CONSEQUENCE OF THE ABOVE IS THAT WE WILL NOT NEED AS
	MANY CUSTOM PROTOCOLS (e.g., rdump, rman, etc.)
	AND WILL THUS (WE HOPE) HAVE LESS CONFUSION WHEN AN 'r'-PROGRAM
	BREAKS (AND THEY HAVE~---~EXAMPLE ON REQUEST)

\item	ISODE AND PEPY MAKE THE PROCESS MUCH EASIER
\end{nrtc}
\end{bwslide}


\begin{note}\em
example slides here...
\end{note}

\begin{bwslide}
\ctitle	{WHERE DO I SIGN?}

\begin{nrtc}
\item	BUT: WE HAVE TO CLIMB A STEEP LEARNING CURVE

\item	MUCH BIGGER BUT: WE HAVE TO CONVINCE OTHERS TO CLIMB IT TOO
\end{nrtc}
\end{bwslide}


\begin{bwslide}
\ctitle	{ISO FOR THE APPLICATIONS PROGRAMMER}

\begin{nrtc}
\item	MAIL SYSTEMS

\item	MULTI-MEDIA SYSTEMS (E.G. NETWORKED APA DISPLAYS)

\item	NETWORK MONITORING PROGRAMS[AMER87]

\item	IMAGE PROCESSING

\item	SPEECH PROCESSING

\item	NETWORKED PCs~---~USING PCs TO HELP MANAGE VAX RESOURCES
\end{nrtc}
\end{bwslide}


\begin{bwslide}
\ctitle	{TWO PROBLEMS}
\begin{nrtc}
\item	CONVINCING VERY BUSY PEOPLE TO TAKE THE TIME TO LEARN IT

\item	ONCE THEY LEARN IT, DO THEY FIND IT BOTH USABLE AND USEFUL? 
\end{nrtc}

SO FAR, AT UDEL, THERE IS NOT ENOUGH TIME TO TELL
\end{bwslide}


\begin{bwslide}
\ctitle	{CONCLUSIONS}

\begin{nrtc}
\item	 EXPERIMENTAL NETWORKS AT UDEL WILL SUPPORT ISODE
    \begin{nrtc}
    \item	BOTH AS A TEST OF ISODE\\ AND AS A TEST OF THE NETWORK
    \end{nrtc}

\item	WE ARE PLANNING TO USE ISODE FOR SYSTEM PROGRAMS THAT 
	PREVIOUSLY WOULD HAVE BEEN IMPLEMENTED IN THE 'r'-PROGRAM STYLE OR
	THAT COULD NOT HAVE BEEN EASILY WRITTEN AT ALL

\item	WE ARE ENCOURAGING OTHER RESEARCHERS AT UDEL TO USE ISODE FOR THEIR
	APPLICATIONS
\end{nrtc}
\end{bwslide}


\begin{bwslide}
\part	{A STRATEGY FOR CONVERGENCE WITH ISO}\bf

\vskip-0.5in
\[\begin{tabular}[t]{c}\large\bf
    Marshall T.~Rose\\
    Computer Science Laboratory\\
    Northrop Research and Technology Center
\end{tabular}\]
\end{bwslide}


\begin{bwslide}
\ctitle	{THE PROBLEM}

\begin{nrtc}
\item	TCP/IP IS HERE NOW AND IT WORKS

\item	ISO IS INEVITABLE!

\item	HOW DO WE GET TO THERE FROM HERE?
\end{nrtc}
\end{bwslide}


\begin{bwslide}
\ctitle	{PREMISES}

\begin{nrtc}
\item	START WITH AN EXISTING TCP/IP INTERNET

\item	ADD SOME ISO-ONLY HOSTS/NETWORKS

\item	MAKE NO MODIFICATIONS TO ISO-ONLY HOSTS
	(AND MINIMIZE CHANGES TO TCP/IP HOSTS)
\end{nrtc}
\end{bwslide}


\begin{bwslide}
\ctitle	{DESIRED INTEROPERABILITY}

\begin{nrtc}
\item	WANT SERVICES BETWEEN END-SYSTEMS AT HIGHER-LEVELS

\item	AVOID APPLICATION-LEVEL GATEWAYS

\item	IMPLIES INTEROPERABILITY AT TRANSPORT LAYER AND ABOVE

\item	CONVERGE ON HIGHER-LEVELS IN THE ISO SUITE

\item	NEEDED: VIRTUAL TRANSPORT SERVICE
\end{nrtc}
\end{bwslide}


\begin{bwslide}
\ctitle	{TCP TRANSPORT SERVICE}

\vspace{0.25in}
\diagram[p]{figure3a}
\end{bwslide}


\begin{bwslide}
\ctitle	{ISO TRANSPORT SERVICE}

\vspace{0.25in}
\diagram[p]{figure3b}
\end{bwslide}


\begin{bwslide}
\ctitle	{A CONVERGENCE STRATEGY}

\begin{nrtc}
\item	NEED TWO THINGS:
    \begin{nrtc}
    \item	HIGHER-LEVEL ISO SERVICES FOR TCP/IP HOSTS

    \item	ISO-IP ENCAPSULATION ON DDN-IP
    \end{nrtc}
\end{nrtc}
\end{bwslide}


\begin{bwslide}
\ctitle	{VIRTUAL ISO TRANSPORT SERVICE}

\vspace{0.25in}
\diagram[p]{figure3c}
\end{bwslide}


\begin{bwslide}
\ctitle	{ISO DEVELOPMENT ENVIRONMENT}

\begin{nrtc}
\item	PROVIDES HIGHER-LEVEL ISO SERVICES FOR TCP/IP HOSTS

\item	A MAGIC-BOX OFFERS TP4 SERVICE OVER TCP (RFC983)

\item	GAIN EXPERIENCE WITH THE ISO SUITE
\end{nrtc}
\end{bwslide}


\begin{bwslide}
\ctitle	{ISO TRANSPORT SERVICES ON TOP OF THE TCP}

\diagram[p]{figure4}
\end{bwslide}


\begin{bwslide}
\ctitle	{MIXED ISO TRANSPORT SERVICE}

\vspace{0.25in}
\diagram[p]{figure3d}
\end{bwslide}


\begin{bwslide}
\ctitle	{DUAL-IP ``GATEWAYS''}

\begin{nrtc}
\item	NEED TWO MORE THINGS:
    \begin{nrtc}
    \item	PUT A REAL TP4 AND ISO-IP IN THE HYBRID HOST

    \item	WITH ISO-IP ENCAPSULATED IN DDN-IP
    \end{nrtc}
\end{nrtc}
\end{bwslide}


\begin{bwslide}
\ctitle	{HYBRID HOST}

\diagram[p]{figure5}
\end{bwslide}


\begin{bwslide}
\ctitle	{THE BIG PICTURE}

\vspace{0.25in}
\diagram[p]{figure6}
\end{bwslide}


\begin{bwslide}
\ctitle	{VIRTUAL TRANSPORT SERVICES (REVIEW)}

\vspace{0.25in}
\diagram[p]{figure7}
\end{bwslide}


\begin{note}\em
to migrate: just stop buying tcp/ip when both of these are done

you need an application gateway to go from tcp/ip-only to iso-only
(essential for mail, probably not needed otherwise)
\end{note}


\end{document}
