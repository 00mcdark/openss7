% run this through SLiTeX with the appropriate wrapper

\dotopic	{NETWORK MANAGEMENT (SNMP)}

\begin{bwslide}
\part*	{OUTLINE}\bf

\begin{description}
\item[PART I:]		INTRODUCTION TO NETWORK MANAGEMENT

\item[PART I:]		CORE ROUTINES

\item[PART II:]		AGENT IMPLEMENTATION

\item[PART III:]	EXPORTING MIB MODULES

\item[PART IV:]		PROTOTYPING NOC APPLICATIONS
\end{description}
\end{bwslide}


\begin{bwslide}
\ctitle	{WHAT? SNMP IN AN OSI PACKAGE!}

\begin{nrtc}
\item	CONTINUED SURVIVAL OF THE INTERNET HINGES ON ALL NODES
	BECOMING NETWORK MANAGABLE:
    \begin{nrtc}
    \item	4BSD/ISODE SNMP PACKAGE

    \item	FREELY DISTRIBUTED WITH UPCOMING RELEASES OF BERKELEY UNIX
    \end{nrtc}

\item	ALSO, GOOD EXAMPLE OF USE OF ASN.1 TOOLS

\item	AND CMIP HAS QUITE A FEW PROBLEMS, e.g.,
    \begin{nrtc}
    \item	OO-HOOPLAH, FILTERING/SCOPING

    \item	FULL ASN.1

    \item	EVENT-HANDLING

    \item	CO-MODE TRANSPORT
    \end{nrtc}
    (BASICALLY, NOT TRACTABLE TECHNOLOGY)
\end{nrtc}
\end{bwslide}


\begin{bwslide}
\part	{A WHIRLWIND INTRODUCTION TO\\ NETWORK MANAGEMENT}\bf

\begin{nrtc}
\item	SEVERAL MANAGED NODES, EACH WITH AN AGENT

\item	AT LEAST ONE MANAGEMENT STATION

\item	A MANAGEMENT PROTOCOL
\end{nrtc}
\end{bwslide}


\begin{bwslide}
\ctitle	{FUNCTIONAL COMPOSITION}

\begin{nrtc}
\item	EXCHANGE OF MANAGEMENT INFORMATION:
    \begin{nrtc}
    \item	 PROTOCOL AND OBJECTS
    \end{nrtc}

\item	ADMINISTRATIVE FRAMEWORK:
    \begin{nrtc}
    \item	AUTHENTICATION AND AUTHORIZATION
    \end{nrtc}
\end{nrtc}
\end{bwslide}


\begin{bwslide}
\ctitle	{MANAGED NODES}

\vskip.5in
\diagram[p]{figureS-21}
\end{bwslide}


\begin{bwslide}
\ctitle	{MANAGEMENT STATIONS}

\begin{nrtc}
\item	CONTAIN
    \begin{nrtc}
    \item	MANAGEMENT PROTOCOL

    \item	MANAGEMENT APPLICATIONS
    \end{nrtc}

\item	GOOD IDEA: SHIFT BURDEN TO STATIONS FROM NODES
    \begin{nrtc}
    \item	SCALABILITY

    \item	MINIMIZE IMPACT ON ``USEFUL'' WORK
    \end{nrtc}
\end{nrtc}
\end{bwslide}


\begin{bwslide}
\ctitle	{MANAGEMENT PROTOCOL}

\begin{nrtc}
\item	REMOTE DEBUGGING PARADIGM~---~MANIPULATE ``VARIABLES''
    \begin{nrtc}
    \item	MONITORING: READ OPERATION

    \item	CONTROL: WRITE OPERATION
    \end{nrtc}

\item	TRAVERSAL OPERATION: DETERMINE SUPPORTED OBJECTS
	(e.g, TABLE EXPLORATION)

\item	TRAP OPERATION: REPORT EXTRAORDINARY EVENTS
\end{nrtc}
\end{bwslide}


\begin{bwslide}
\ctitle	{SIMPLE NETWORK MANAGEMENT PROTOCOL}

\begin{nrtc}
\item	OPERATIONS:
    \begin{nrtc}
    \item	GET

    \item	POWERFUL GET-NEXT

    \item	SET

    \item	TRAP
    \end{nrtc}

\item	REQUEST-RESPONSE PROTOCOL
    \begin{nrtc}
    \item	EXCEPT THAT TRAPS ARE UNCONFIRMED
    \end{nrtc}
\end{nrtc}
\end{bwslide}


\begin{bwslide}
\ctitle	{MANAGEMENT OBJECTS}

\begin{nrtc}
\item	DEFINED USING (WELL-DEFINED SUBSET OF) ASN.1
\begin{quote}\small\begin{verbatim}
sysDescr OBJECT-TYPE
        SYNTAX  DisplayString (SIZE (0..255))
        ACCESS  read-only
        STATUS  mandatory
        ::= { system 1 }
\end{verbatim}\end{quote}

\item	PROTOCOL CARRIES BOTH
    \begin{nrtc}
    \item	 OBJECT TYPE

    \item	 OBJECT INSTANCE
    \end{nrtc}

\item	IN SNMP, INSTANCE IS APPENDED TO OID OF TYPE
    \begin{nrtc}
    \item	NON-TABULAR OBJECTS HAVE \verb".0"

    \item	EACH TABULAR OBJECT DEFINES FORM OF INSTANCE
    \end{nrtc}
\end{nrtc}
\end{bwslide}


\begin{bwslide}
\ctitle	{OBJECT TREE}

\vskip.5in
\diagram[p]{figureS-22}
\end{bwslide}


\begin{bwslide}
\ctitle	{OBJECT ORDERING}

\vskip.5in
\diagram[p]{figureS-23}
\end{bwslide}


\begin{bwslide}
\ctitle	{ADMINISTRATIVE FRAMEWORK}

\begin{nrtc}
\item	AUTHENTICATION: COMMUNITIES
    \begin{nrtc}
    \item	USER/COMMUNITY PAIRS
    \end{nrtc}
    COMMUNITY DEFINES AUTHENTICATION METHOD

\item	AUTHORIZATION: VIEWS
    \begin{nrtc}
    \item	VIEW/OPERATION PAIRS
    \end{nrtc}
    VIEW DEFINES ACCESS METHOD
\end{nrtc}
\end{bwslide}


\begin{bwslide}
\ctitle	{OVERVIEW}

\vskip.5in
\diagram[p]{figureS-26}
\end{bwslide}


\begin{bwslide}
\ctitle	{EXAMPLE:\\ ``OUT OF THE BOX'' MANAGEMENT}

\vskip.5in
\diagram[p]{figureS-27}
\end{bwslide}


\begin{bwslide}
\ctitle	{EXAMPLE:\\ CROSS-ENTERPRISE MANAGEMENT}

\vskip.5in
\diagram[p]{figureS-28}
\end{bwslide}


\begin{bwslide}
\part	{CORE ROUTINES}\bf

\begin{nrtc}
\item	MOSY

\item	RUN-TIME LIBRARY

\item	ENCODE/DECODE ROUTINES
\end{nrtc}
\end{bwslide}


\begin{bwslide}
\part*	{MOSY}\bf

\begin{nrtc}
\item	MANAGED OBJECT SYNTAX-COMPILER (YACC-based)

\item	READS MIB MODULES DEFINED ACCORDING TO RFC1155
	AND PRODUCES FLAT FILE
\end{nrtc}
\end{bwslide}


\begin{bwslide}
\ctitle	{EXAMPLE}

\begin{quote}\small\begin{verbatim}
% mosy smi.my
RFC1155-SMI identifiers: internet directory mgmt
            experimental private enterprises

RFC1155-SMI types: ObjectName ObjectSyntax SimpleSyntax
            ApplicationSyntax NetworkAddress IpAddress
            Counter Gauge TimeTicks Opaque

% mosy mib.my
RFC1156-MIB identifiers: mib system interfaces at ip
            icmp tcp udp egp

RFC1156-MIB objects: sysDescr sysObjectID sysUpTime
            ifNumber ifTable ifEntry

    ...

% cat smi.defs mib.defs > objects.defs
\end{verbatim}\end{quote}
\end{bwslide}


\begin{bwslide}
\ctitle	{EXAMPLE (cont.)}

\begin{quote}\small\begin{verbatim}
sysObjectID OBJECT-TYPE
        SYNTAX  OBJECT IDENTIFIER
        ACCESS  read-only
        STATUS  mandatory
        ::= { system 2 }
\end{verbatim}\end{quote}
BECOMES
\begin{quote}\small\begin{verbatim}
sysObjectID   system.2  ObjectID   read-only   mandatory
\end{verbatim}\end{quote}
\end{bwslide}


\begin{bwslide}
\ctitle	{NEW DEVELOPMENT:\\ CONCISE MIB MODULES}

\begin{quote}\small\begin{verbatim}
sysObjectID OBJECT-TYPE
        SYNTAX  OBJECT IDENTIFIER
        ACCESS  read-only
        STATUS  mandatory
        DESCRIPTION
                "The vendor's authoritative identification
                 of the network management subsystem
                 contained in the entity.  This value is
                 allocated within the SMI enterprises subtree
                 (1.3.6.1.4.1) and provides an easy and
                 unambiguous means for determining `what kind
                 of box' is being managed."
        ::= { system 2 }
\end{verbatim}\end{quote}
\end{bwslide}


\begin{bwslide}
\ctitle	{NEW DEVELOPMENT:\\ COMPILED FLAT FILES}

\begin{nrtc}
\item	INVOKE \verb"snmpi" PROGRAM:
\begin{quote}\small\begin{verbatim}
% snmpi -f objects.defs
snmpi> compile
snmpi: 1035 objects written to objects.defs
\end{verbatim}\end{quote}

\item	ALMOST AN ORDER OF MAGNITUDE FASTER TO READ
\end{nrtc}
\end{bwslide}


\begin{bwslide}
\part*	{RUN-TIME LIBRARY}\bf

\begin{nrtc}
\item	DEAL WITH
    \begin{nrtc}
    \item	OBJECT SYNTAXES

    \item	OBJECT TYPES

    \item	OBJECT INSTANCES
    \end{nrtc}

\item	LOAD WITH \verb"-lisnmp -lisode"
\end{nrtc}
\end{bwslide}


\begin{bwslide}
\ctitle	{INITIALIZATION}

\begin{quote}\small\begin{verbatim}
#include <isode/snmp/objects.h>

...

if (readobjects ("objects.defs") == NOTOK)
    error ("readobjects: %s", PY_pepy);
\end{verbatim}\end{quote}
\end{bwslide}


\begin{bwslide}
\ctitle	{OBJECT SYNTAXES}

\begin{quote}\small\begin{verbatim}
typedef struct object_syntax {
    char   *os_name;      /* syntax name */

    IFP     os_encode;    /* data -> PE */
    IFP     os_decode;    /* PE -> data */
    IFP     os_free;      /* free data */

    IFP     os_parse;     /* str -> data */
    IFP     os_print;     /* data -> tty */

    ...
}               object_syntax, *OS;
#define NULLOS  ((OS) 0)
\end{verbatim}\end{quote}
\end{bwslide}


\begin{bwslide}
\ctitle	{OBJECT SYNTAXES (cont.)}

\[\tt\begin{tabular}{|r|l|}
\hline
\multicolumn{1}{|c|}{\bf syntax}&
		\multicolumn{1}{|c|}{\bf C structure}\\
\hline
INTEGER&    		integer\\
OCTET STRING&		struct qbuf *\\
OBJECT IDENTIFIER&	struct OIDentifier *\\
NULL&			char *\\
IpAddress&		struct sockaddr\_in *\\
NetworkAddress&		struct sockaddr\_in *\\
Counter&		u\_long\\
Gauge&			u\_long\\
TimeTicks&		u\_long\\
DisplayString&		struct qbuf *\\
ClnpAddress&		struct sockaddr\_osi *\\
\hline
\end{tabular}\]
\end{bwslide}


\begin{bwslide}
\ctitle	{DEFINING A NEW SYNTAX}

\hrule\vskip.15in
\begin{tgrind}\scriptsize
\let\linebox=\relax
\def\_{\ifstring{\char'137}\else\underline{\ }\fi}
\input figureS-1\relax
\end{tgrind}
\end{bwslide}


\begin{bwslide}
\ctitle	{DEFINING A NEW SYNTAX (cont.)}

\hrule\vskip.15in
\begin{tgrind}\scriptsize
\let\linebox=\relax
\def\_{\ifstring{\char'137}\else\underline{\ }\fi}
\input figureS-2\relax
\end{tgrind}
\end{bwslide}


\begin{bwslide}
\ctitle	{DEFINING A NEW SYNTAX (cont.)}

\hrule\vskip.15in
\begin{tgrind}\scriptsize
\let\linebox=\relax
\def\_{\ifstring{\char'137}\else\underline{\ }\fi}
\input figureS-3\relax
\end{tgrind}
\end{bwslide}


\begin{bwslide}
\ctitle	{OBJECT TYPES}

\begin{quote}\smaller\begin{verbatim}
typedef struct object_type {
    char   *ot_text;      /* descriptor */
    OID     ot_name;      /* identifier */

    OS      ot_syntax;    /* SYNTAX */

    int     ot_access;    /* ACCESS */
#define OT_NONE         0x00
#define OT_RDONLY       0x01
#define OT_WRONLY      0x03
#define OT_RDWRITE      (OT_RDONLY | OT_WRONLY)

    int     ot_status;    /* STATUS */
#define OT_OBSOLETE     0x00
#define OT_MANDATORY    0x01
#define OT_OPTIONAL     0x02
#define OT_DEPRECATED   0x03

    IFP     ot_getfnx;    /* get method */
    caddr_t ot_info;      /*   .. for object */

    IFP     ot_setfnx;    /* set method */
    caddr_t ot_save;      /*   .. for object */
     ...
}               object_type, *OT;
#define NULLOT  ((OT) 0)
\end{verbatim}\end{quote}
\end{bwslide}


\begin{bwslide}
\ctitle	{SUPPORT ROUTINES}

\begin{quote}\small\begin{verbatim}
OT    name2obj ( text2oid ("ipNextHop.0.0.0.0") );

OT    text2obj ("ipNextHop");

OID   text2oid ("1.3.6.1");
OID   text2oid ("internet");
OID   text2oid ("iso.3.6.1");

char *oid2ode (oid);
\end{verbatim}\end{quote}
\end{bwslide}


\begin{bwslide}
\ctitle	{OBJECT INSTANCES}

\begin{quote}\small\begin{verbatim}
typedef struct object_instance {
    OID     oi_name;      /* instance OID */
    OT      oi_type;      /* prototype */
}               object_instance, *OI;
#define NULLOI  ((OI) 0)
\end{verbatim}\end{quote}
\end{bwslide}


\begin{bwslide}
\ctitle	{SUPPORT ROUTINES}

\begin{quote}\small\begin{verbatim}
OI      name2inst ( text2oid ("ipNextHop.0.0.0.0") );

OI      next2inst (oid);

OI      text2inst (text);
\end{verbatim}\end{quote}
\end{bwslide}


\begin{bwslide}
\part*	{ENCODE/DECODE ROUTINES}\bf

\begin{nrtc}
\item	ALL GENERATED USING PEPSY
\end{nrtc}
\end{bwslide}


\begin{bwslide}\small
\begin{verbatim}
% pepsy -A -f -h snmp.py
SNMP types: Message PDUs GetRequest-PDU GetNextRequest-PDU GetResponse-PDU
     SetRequest-PDU PDU Trap-PDU VarBind VarBindList ObjectName ObjectSyntax
     NetworkAddress IpAddress TimeTicks ClnpAddress DisplayString

% cc -c SNMP_tables.c
  ...
\end{verbatim}
\end{bwslide}


\begin{bwslide}\small
\begin{verbatim}
struct type_SNMP_PDU {
    integer    request__id;

    integer    error__status;
#define int_SNMP_error__status_noError  0
#define int_SNMP_error__status_tooBig   1
#define int_SNMP_error__status_noSuchName       2
#define int_SNMP_error__status_badValue 3
#define int_SNMP_error__status_readOnly 4
#define int_SNMP_error__status_genErr   5

    integer    error__index;

    struct type_SNMP_VarBindList *variable__bindings;
};
int     free_SNMP_PDU ();
\end{verbatim}
\end{bwslide}


\begin{bwslide}
\part	{AGENT IMPLEMENTATION}\bf

\begin{nrtc}
\item	INITIALIZATION

\item	MAIN LOOP

\item	NON-TABULAR ACCESS

\item	TABULAR ACCESS
\end{nrtc}
\end{bwslide}


\begin{bwslide}
\ctitle	{AGENT FEATURES}

\begin{nrtc}
\item	SUPPORT FOR SNMP OVER
    \begin{nrtc}
    \item	UDP, CLTS, COTS
    \end{nrtc}

\item	MIBs: MIB-II, UNIX MIB, SMUX, EVAL

\item	PROXY: SNMP/UDP

\item	TRAPS: SNMP/UDP

\item	VIEWS
\end{nrtc}
\end{bwslide}


\begin{bwslide}
\ctitle	{NEW DEVELOPMENT:\\ SNMPC, SNMPT}

\begin{nrtc}
\item	SNMPC: COMPOSITE AGENT
    \begin{nrtc}
    \item	NO INTERNET-STANDARD MIB

    \item	USED FOR MANAGING AGENTS(!)
    \end{nrtc}

\item	SNMPT: TRAP SINK
\end{nrtc}
\end{bwslide}


\begin{bwslide}
\part*	{INITIALIZATION}\bf

\begin{nrtc}
\item	A USER-LEVEL PROCESS, SO DO ALL THE USUAL DAEMON STUFF

\item	READ \verb"snmpd.defs" in \verb"ETCDIR"

\item	INSTANTIATE LINKAGE TO ACCESS ROUTINES

\item	READ CONFIGURATION FILE \verb"snmpd.rc" in \verb"ETCDIR"

\item	START LISTENING ON CONFIGURED STACKS:
    \begin{nrtc}
    \item	UDP

    \item	COTS

    \item	CLTS
    \end{nrtc}
\end{nrtc}
\end{bwslide}


\begin{bwslide}
\ctitle	{LINKAGE}

\begin{nrtc}
\item	FOR EACH OBJECT SUPPORTED
\begin{quote}\small\begin{verbatim}
#define ipForwarding    0
...
extern  int    o_ip ();

...

OT    ot;

if (ot = text2obj ("ipForwarding")) {
    ot -> ot_getfnx = o_ip;
    ot -> ot_info = (caddr_t) ipForwarding;
}
...
\end{verbatim}\end{quote}
\end{nrtc}
\end{bwslide}


\begin{bwslide}
\ctitle	{CONFIGURATION FILE}

\begin{quote}\small\begin{verbatim}
community       public

community       system          0.0.0.0         readWrite           1.17.2.1
view            1.17.2.1        system unix

community       wp2             0.0.0.0         readOnly            1.17.2.2
proxy           1.17.2.2        rfc1157         Internet=wp2+161    secret

trap            traps           192.33.4.20

logging         file=snmpd.log  size=50
logging         slevel=fatal    slevel=exceptions  slevel=notice
logging         sflags=close    sflags=create      sflags=zero

variable        sysContact      "Marshall Rose <mrose@psi.com>"
variable        sysLocation     "Troy machine room"
variable        sysServices     72

variable        interface lo0   ifType=24       ifSpeed=0
variable        interface le0   ifType=6
\end{verbatim}\end{quote}
\end{bwslide}


\begin{bwslide}
\part*	{MAIN LOOP}\bf

\begin{nrtc}
\item	WAIT FOR A MESSAGE

\item	BUMP \verb"quantum" (LOGICAL REQUEST-ID)
\end{nrtc}
\end{bwslide}


\begin{bwslide}
\ctitle	{DESERIALIZE ASN.1 STRUCTURE}

\hrule\vskip.15in
\begin{tgrind}\scriptsize
\let\linebox=\relax
\def\_{\ifstring{\char'137}\else\underline{\ }\fi}
\input figureS-4\relax
\end{tgrind}
\end{bwslide}


\begin{bwslide}
\ctitle	{DESERIALIZE ASN.1 STRUCTURE (cont.)}

\hrule\vskip.15in
\begin{tgrind}\scriptsize
\let\linebox=\relax
\def\_{\ifstring{\char'137}\else\underline{\ }\fi}
\input figureS-5\relax
\end{tgrind}
\end{bwslide}


\begin{bwslide}
\ctitle	{TRANSLATE TO SNMP MESSAGE}

\hrule\vskip.15in
\begin{tgrind}\scriptsize
\let\linebox=\relax
\def\_{\ifstring{\char'137}\else\underline{\ }\fi}
\input figureS-6\relax
\end{tgrind}
\end{bwslide}


\begin{bwslide}
\ctitle	{USUAL CHECKS}

\begin{nrtc}
\item	RIGHT VERSION, COMMUNITY, ACCESS, etc.

\item	LOOP THROUGH LIST OF VARIABLES
    \begin{nrtc}
    \item	FIND PROTOTYPE

    \item	INVOKE ACCESS ROUTINE
    \end{nrtc}
\end{nrtc}
\begin{quote}\small\begin{verbatim}
int     result = (*ot -> ot_getfnx) (oi, v, offset)
OI      oi;
struct type_SNMP_VarBind *v;
int    offset;
\end{verbatim}\end{quote}
\end{bwslide}


\begin{bwslide}
\ctitle	{BINDING STRUCTURE}

\begin{quote}\small\begin{verbatim}
struct type_SNMP_VarBind {
    OID     name;

    PE      value;
};
\end{verbatim}\end{quote}
\end{bwslide}


\begin{bwslide}
\part*	{NON-TABULAR ACCESS}\bf

\begin{nrtc}
\item	DETERMINE WHICH LEAF OBJECT IS BEING REFERENCED

\item	DETERMINE INSTANCE, BASED ON OPERATION
    \begin{nrtc}
    \item	GET: MUST HAVE ``\verb".0"'' INSTANCE-ID

    \item	POWERFUL GET-NEXT: EITHER
	\begin{nrtc}
	\item	INSTANCE-ID ABSENT: APPEND ``\verb".0"''

	\item	INSTANCE-ID PRESENT: FLAG NEXT OBJECT
	\end{nrtc}
    \end{nrtc}
\end{nrtc}
\end{bwslide}


\begin{bwslide}
\ctitle	{EXAMPLE: NON-TABULAR ACCESS}

\vskip.5in
\diagram[p]{figureS-23}
\end{bwslide}


\begin{bwslide}
\ctitle	{NON-TABULAR ACCESS}

\hrule\vskip.15in
\begin{tgrind}\scriptsize
\let\linebox=\relax
\def\_{\ifstring{\char'137}\else\underline{\ }\fi}
\input figureS-7\relax
\end{tgrind}
\end{bwslide}


\begin{bwslide}
\ctitle	{NON-TABULAR ACCESS (cont.)}

\hrule\vskip.15in
\begin{tgrind}\scriptsize
\let\linebox=\relax
\def\_{\ifstring{\char'137}\else\underline{\ }\fi}
\input figureS-8\relax
\end{tgrind}
\end{bwslide}


\begin{bwslide}
\ctitle	{NON-TABULAR ACCESS (cont.)}

\hrule\vskip.15in
\begin{tgrind}\scriptsize
\let\linebox=\relax
\def\_{\ifstring{\char'137}\else\underline{\ }\fi}
\input figureS-9\relax
\end{tgrind}
\end{bwslide}


\begin{bwslide}
\ctitle	{ENCODING INSTANCE VALUES}

\begin{nrtc}
\item	FIND PROTOTYPE AND OBJECT SYNTAX

\item	CALL \verb"os_encode" ROUTINE

\item	UPDATE \verb"value" in \verb"struct type_SNMP_VarBind"

\item	RETURN \verb"OK" OR SNMP ERROR CODE
\end{nrtc}
\end{bwslide}


\begin{bwslide}
\part*	{TABULAR ACCESS}\bf

\begin{nrtc}
\item	BUILD DATA STRUCTURE FROM SOURCE\\ (e.g., KERNEL)

\item	DETERMINE WHICH LEAF OBJECT IS BEING REFERENCED
\end{nrtc}
\end{bwslide}


\begin{bwslide}
\ctitle	{EXAMPLE: TABULAR ACCESS}

\vskip.5in
\diagram[p]{figureS-23}
\end{bwslide}


\begin{bwslide}
\ctitle	{GET OPERATION}

\begin{nrtc}
\item	MUST HAVE CORRECT LENGTH OF INSTANCE-ID
    \begin{nrtc}
    \item	EACH TABLE DEFINITION SAYS WHAT INSTANCE-ID LOOKS LIKE
    \end{nrtc}

\item	LOOKUP DATA STRUCTURE
\end{nrtc}
\end{bwslide}


\begin{bwslide}
\ctitle	{POWERFUL GET-NEXT OPERATION}

\begin{nrtc}
\item	IF INSTANCE-ID PRESENT, BUT WRONG LENGTH, THEN NORMALIZE

\item	IF INSTANCE-ID ABSENT, USE FIRST ENTRY IN DATA STRUCTURE

\item	OTHERWISE, FIND ``NEXT'' ENTRY IN DATA STRUCTURE
\end{nrtc}
\end{bwslide}


\begin{bwslide}
\ctitle	{TABULAR ACCESS}

\hrule\vskip.15in
\begin{tgrind}\scriptsize
\let\linebox=\relax
\def\_{\ifstring{\char'137}\else\underline{\ }\fi}
\input figureS-10\relax
\end{tgrind}
\end{bwslide}


\begin{bwslide}
\ctitle	{TABULAR ACCESS (cont.)}

\hrule\vskip.15in
\begin{tgrind}\scriptsize
\let\linebox=\relax
\def\_{\ifstring{\char'137}\else\underline{\ }\fi}
\input figureS-11\relax
\end{tgrind}
\end{bwslide}


\begin{bwslide}
\ctitle	{TABULAR ACCESS (cont.)}

\hrule\vskip.15in
\begin{tgrind}\scriptsize
\let\linebox=\relax
\def\_{\ifstring{\char'137}\else\underline{\ }\fi}
\input figureS-12\relax
\end{tgrind}
\end{bwslide}


\begin{bwslide}
\ctitle	{TABULAR ACCESS (cont.)}

\hrule\vskip.15in
\begin{tgrind}\scriptsize
\let\linebox=\relax
\def\_{\ifstring{\char'137}\else\underline{\ }\fi}
\input figureS-13\relax
\end{tgrind}
\end{bwslide}


\begin{bwslide}
\ctitle	{TABULAR ACCESS (cont.)}

\hrule\vskip.15in
\begin{tgrind}\scriptsize
\let\linebox=\relax
\def\_{\ifstring{\char'137}\else\underline{\ }\fi}
\input figureS-29\relax
\end{tgrind}
\end{bwslide}


\begin{bwslide}
\ctitle	{TABULAR ACCESS (cont.)}

\hrule\vskip.15in
\begin{tgrind}\scriptsize
\let\linebox=\relax
\def\_{\ifstring{\char'137}\else\underline{\ }\fi}
\input figureS-14\relax
\end{tgrind}
\end{bwslide}


\begin{bwslide}
\ctitle	{NEW DEVELOPMENT:\\ SETs}

\hrule\vskip.15in
\begin{tgrind}\scriptsize
\let\linebox=\relax
\def\_{\ifstring{\char'137}\else\underline{\ }\fi}
\input figureS-30\relax
\end{tgrind}
\end{bwslide}


\begin{bwslide}
\part	{EXPORTING MIB MODULES}\bf

\begin{nrtc}
\item	USER-PROCESS WORKS OK (NOT GREAT) WHEN REALIZING THE
	INTERNET-STANDARD MIB

\item	MAY WANT TO HAVE OTHER PROCESSES MANAGE PARTS OF THE MIB
	(e.g., MAIL QUEUE)

\item	BUT, COMMUNICATION WITH OTHER PROCESSES IS PROBLEMATIC
\end{nrtc}
\end{bwslide}


\begin{bwslide}
\ctitle	{ARCHITECTURE}

\begin{nrtc}
\item	CHOOSE BETWEEN
    \begin{nrtc}
    \item	REQUEST-RESPONSE MODEL

    \item	CACHE-AHEAD MODEL
    \end{nrtc}

\item	SNMP MULTIPLEXING (SMUX) PROTOCOL USES REQUEST-RESPONSE APPROACH
    \begin{nrtc}
    \item	SINCE A SMUX PEER CAN ALWAYS FRONT-END A CACHE-AHEAD APPROACH
    \end{nrtc}

\item	NOTE THAT USE OF THE SMUX IS \emph{STRICTLY} A LOCAL MATTER
\end{nrtc}
\end{bwslide}


\begin{bwslide}
\ctitle	{SMUX FIREWALL}

\vskip.5in
\diagram[p]{figureS-24}
\end{bwslide}


\begin{bwslide}
\part*	{PROTOCOL}\bf

\begin{nrtc}
\item	AGENT LISTENS FOR SMUX ASSOCIATIONS

\item	SMUX PEER REGISTERS SUBTREES

\item	SMUX PEER MIGHT ISSUE TRAPS

\item	AGENT ASKS PEERS TO HELP IN PERFORMING
    \begin{nrtc}
    \item	GET, POWERFUL GET-NEXT, and SET
    \end{nrtc}
    OPERATIONS

\item	USES CO-MODE TRANSPORT SERVICE
    \begin{nrtc}
    \item	AT PRESENT, MAPPING ONTO THE TCP IS ONLY DEFINED
    \end{nrtc}
\end{nrtc}
\end{bwslide}


\begin{bwslide}
\ctitle	{SMUX PROTOCOL}

\hrule\vskip.15in
\begin{tgrind}\scriptsize
\let\linebox=\relax
\def\_{\ifstring{\char'137}\else\underline{\ }\fi}
\input figureS-15\relax
\end{tgrind}
\end{bwslide}


\begin{bwslide}
\ctitle	{SMUX PROTOCOL (cont.)}

\hrule\vskip.15in
\begin{tgrind}\scriptsize
\let\linebox=\relax
\def\_{\ifstring{\char'137}\else\underline{\ }\fi}
\input figureS-16\relax
\end{tgrind}
\end{bwslide}


\begin{bwslide}
\ctitle	{SMUX PROTOCOL (cont.)}

\hrule\vskip.15in
\begin{tgrind}\scriptsize
\let\linebox=\relax
\def\_{\ifstring{\char'137}\else\underline{\ }\fi}
\input figureS-17\relax
\end{tgrind}
\end{bwslide}


\begin{bwslide}
\ctitle	{SMUX PROTOCOL (cont.)}

\hrule\vskip.15in
\begin{tgrind}\scriptsize
\let\linebox=\relax
\def\_{\ifstring{\char'137}\else\underline{\ }\fi}
\input figureS-18\relax
\end{tgrind}
\end{bwslide}


\begin{bwslide}
\ctitle	{SMUX PROTOCOL (cont.)}

\hrule\vskip.15in
\begin{tgrind}\scriptsize
\let\linebox=\relax
\def\_{\ifstring{\char'137}\else\underline{\ }\fi}
\input figureS-19\relax
\end{tgrind}
\end{bwslide}


\begin{bwslide}
\ctitle	{SMUX PROTOCOL (cont.)}

\hrule\vskip.15in
\begin{tgrind}\scriptsize
\let\linebox=\relax
\def\_{\ifstring{\char'137}\else\underline{\ }\fi}
\input figureS-20\relax
\end{tgrind}
\end{bwslide}


\begin{bwslide}
\ctitle	{TRICKY THINGS}

\begin{nrtc}
\item	REGISTRATION
    \begin{nrtc}
    \item	SUBTREE/PRIORITY PAIR IS UNIQUE

    \item	SUBTREE MOUNTING EFFECT
    \end{nrtc}
\end{nrtc}
\end{bwslide}


\begin{bwslide}
\ctitle	{EXAMPLE: SUBTREE MOUNTING EFFECT}

\vskip.5in
\diagram[p]{figureS-25}
\end{bwslide}


\begin{bwslide}
\ctitle	{TRICKY THINGS (cont.)}

\begin{nrtc}
\item	MAPPING SNMP MESSAGES TO SMUX MESSAGES
    \begin{nrtc}
    \item	ONE-TO-ONE OR ONE-TO-MANY

    \item	USE OF REQUEST-ID
    \end{nrtc}

\item	POWERFUL GET-NEXT OPERATOR

\item	SETs
\end{nrtc}
\end{bwslide}


\begin{bwslide}
\ctitle	{NEW DEVELOPMENT:\\ SETs}

\begin{nrtc}
\item	PROTOCOL CHANGED TO SIMPLE TWO-PHASE COMMIT

\item	PHASE I: AGENT SENDS ONE OR MORE \verb"SetRequest-PDU"
    \begin{nrtc}
    \item	EACH SMUX PEER SENDS BACK \verb"GetResponse-PDU"
    \end{nrtc}

\item	PHASE II: AGENT SENDS \verb"SoutPDU"
    \begin{nrtc}
    \item	IF COMMIT, PEER PERFORMS OPERATION AND SENDS BACK
		\verb"GetResponse-PDU" 

    \item	IF ROLLBACK, PEER DE-COMMITS
    \end{nrtc}
\end{nrtc}
\end{bwslide}


\begin{bwslide}
\part	{PROTOTYPING NOC APPLICATIONS}\bf

\begin{nrtc}
\item	CODING NOC APPLICATIONS IN C IS TIRESOME

\item	REALLY NEED A RAPID-PROTOTYPING SYSTEM
\end{nrtc}
\end{bwslide}


\begin{bwslide}
\ctitle	{TWO THINGS ARE NEEDED}

\begin{nrtc}
\item	AN APPLICATIONS FRAMEWORK FOR THE USER

\item	AUTOMATIC GENERATION OF NETWORK MANAGEMENT PDUs
    \begin{nrtc}
    \item	AND THE GLUE BETWEEN THEM    
    \end{nrtc}
\end{nrtc}
\end{bwslide}


\begin{bwslide}
\ctitle	{APPLICATIONS FRAMEWORK:\\ AWK}

\begin{nrtc}
\item	A PATTERN SCANNING AND PROCESSING LANGUAGE

\item	USEFUL FOR PROTOTYPING TEXT-HANDLING APPLICATIONS

\item	INTERPRETIVE LANGUAGE

\item	MODIFY GNU AWK FOR
    \begin{nrtc}
    \item	SNMP-CAPABLE GAWK
    \end{nrtc}
\end{nrtc}
\end{bwslide}


\begin{bwslide}
\part*	{MODIFICATIONS TO AWK}\bf

\begin{nrtc}
\item	READ COMPILED-MIB MODULE

\item	RECOGNIZE LEAF OBJECTS
    \begin{nrtc}
    \item	TABULAR AND NOT
    \end{nrtc}
	AS SPECIAL VARIABLES

\item	CURRENTLY ASSUME READ-ONLY STATUS
\end{nrtc}
\end{bwslide}


\begin{bwslide}
\ctitle	{VARIABLES}

\begin{nrtc}
\item	FOR SIMPLE VARIABLES:
\begin{quote}\small\begin{verbatim}
sysDescr

sysDescr[value]
\end{verbatim}\end{quote}

\item	FOR TABULAR VARIABLES:
\begin{quote}\small\begin{verbatim}
ifDescr[value]

for (i in ifDescr)
    printf "%d: %s\n", ifIndex, ifDescr;

for (i in ipRouteDest, "130.117")
    printf "%s: %s\n", ipRouteDest, ipRouteNextHop;
\end{verbatim}\end{quote}
	USING POWERFUL GET-NEXT OPERATOR
\end{nrtc}
\end{bwslide}

\begin{bwslide}
\ctitle	{VARIABLES (cont.)}

\begin{nrtc}
\item	WHEN VARIABLE APPEARS IN AN ``RVALUE'',
	USE THE SNMP TO RETRIEVE VALUE

\item	ADD A FEW NEW VARIABLES TO HELP OUT
    \begin{nrtc}
    \item	\verb"DIAGNOSTIC"

    \item	\verb"ERROR" (IF ANY) IN LAST SNMP OPERATION

    \item	\verb"AGENT" ADDRESS

    \item	\verb"COMMUNITY" NAME

    \item	\verb"RETRIES" FOR EACH SNMP OPERATION

    \item	\verb"TIMEOUT" SECONDS BETWEEN RETRIES
    \end{nrtc}
\begin{quote}\small\begin{verbatim}
% gawk 'BEGIN { print "sysDescr: ", sysDescr, DIAGNOSTIC; }'
\end{verbatim}\end{quote}
\end{nrtc}
\end{bwslide}


\begin{bwslide}
\ctitle	{NEW DEVELOPMENT:\\ COMMUNITY DATABASE}

\begin{nrtc}
\item	USED TO DERIVE COMMUNITY NAME BASED ON AGENT ADDRESS

\item	MAPPING FILE FORMAT
\begin{quote}\small\begin{verbatim}
<IP-address> [ <IP-address mask> ] <community>
\end{verbatim}\end{quote}
e.g.,
\begin{quote}\small\begin{verbatim}
130.117.1    255.255.255.0         "public"
\end{verbatim}\end{quote}

\item	IF \verb"COMMUNITY" VARIABLE STARTS WITH A SLASH,
	NAMED MAPPING FILE IS READ

\item	FIRST ENTRY SATIFISYING
$$\mbox{(AGENT IP-address)} \land \mbox{(IP-address mask)}\ \equiv\ 
\mbox{(IP-address)}$$
IS CHOSEN

\item	DEFAULT ENTRY (LAST IN FILE) SHOULD BE:
\begin{quote}\small\begin{verbatim}
0.0.0.0      0.0.0.0               "public"
\end{verbatim}\end{quote}
\end{nrtc}
\end{bwslide}


\begin{bwslide}
\ctitle	{DATA TYPING}

\begin{nrtc}
\item	INTEGER: number

\item	OCTET STRING: string (\verb|"%02x: ... :%02x"|)

\item	OBJECT IDENTIFIER: string (\verb|"%u. ... .%u"|)

\item	and so on$\ldots$
\end{nrtc}
\end{bwslide}


\begin{bwslide}
\part*	{TWO EXAMPLES}\bf

\begin{nrtc}
\item	ROUTING TABLE

\item	ROUTE TRACING
\end{nrtc}
\end{bwslide}


\begin{bwslide}
\ctitle	{ROUTING TABLE}

\vspace{0.5in}
\smaller
\begin{verbatim}
hasunix = unixNetstat == 1;

printf "Routing tables\n";
printf "%-15s %-15s %-8s %-6s %-10s %s\n",
        "Destination",
              "Gateway",
                    hasunix ? "Flags" : "Type",
                         "Refcnt",
                              "Use",
                                   "Interface";
\end{verbatim}
\end{bwslide}


\begin{bwslide}
\ctitle	{ROUTING TABLE (cont.)}

\vspace{0.5in}
\smaller
\begin{verbatim}
didone = 0;
for (i in ipRouteDest) {
    didone = 1;

    printf "%-15s %-15s %-8s %-6s %-10s %s (#%d)\n",
            ipRouteDest == "0.0.0.0" ? "default" : ipRouteDest,
                  ipRouteNextHop,
                        hasunix ? rt_flags(unixIpRouteFlags[i]) \
                                : rt_type(ipRouteType),
                             hasunix ? unixIpRouteRefCnt[i] : "",
                                  hasunix ? unixIpRouteUses[i] : "",
                                        ifDescr[ipRouteIfIndex],
                                        ipRouteIfIndex;
}
if (!didone && DIAGNOSTIC)
    printf "ipRoutingTable: %s\n", DIAGNOSTIC;
\end{verbatim}
\end{bwslide}


\begin{bwslide}
\ctitle	{ROUTING TABLE (cont.)}

\vspace{0.5in}
\smaller
\begin{verbatim}
% gawk -f mib.routes  (actually "s-netstat -r")
Routing tables
Destination     Gateway         Flags    Refcnt Use        Interface
default         192.52.180.3    UG       5      63813      le0 (#1)
127.0.0.0       127.0.0.1       U        22     31804      lo0 (#2)
192.52.180.0    192.52.180.1    U        7      167235     le0 (#1)
\end{verbatim}
\end{bwslide}


\begin{bwslide}
\ctitle	{ROUTE TRACING}

\vspace{0.5in}
\smaller
\begin{verbatim}
BEGIN {
        printf "from %s to %s:\n", AGENT, DEST;
        tried[AGENT] = 1;
        INVALID = 2; DIRECT = 3;

        split (DEST, dest, ".");

        if (dest[1] < 128)
            net = dest[1];
        else
            if (dest[1] < 192)
                net = dest[1] "." dest[2];
            else
                net = dest[1] "." dest[2] "." dest[3];
\end{verbatim}
\end{bwslide}


\begin{bwslide}
\ctitle	{ROUTE TRACING (cont.)}

\vspace{0.5in}
\smaller
\begin{verbatim}
        while (DEST != AGENT) {
            dr = 0;
            gotit = 0;
            for (i in ipRouteMask, net) {
                if ((type = ipRouteType) == INVALID)
                    continue;

                split(ipRouteMask, mask, ".");
                mask = bit_and(dest[1],mask[1]) "." \
                       bit_and(dest[2],mask[2]) "." \
                       bit_and(dest[3],mask[3]) "." \
                       bit_and(dest[4],mask[4]);
                if (mask == ipRouteDest) {
                    hop = ipRouteNextHop;
                    gotit = 1;
                    break;
                }
            }
\end{verbatim}
\end{bwslide}


\begin{bwslide}
\ctitle	{ROUTE TRACING (cont.)}

\vspace{0.5in}
\smaller
\begin{verbatim}
            if (!gotit) {
                if ((hop = ipRouteNextHop[addr = DEST]) \
                        && (type = ipRouteType[addr]) == INVALID)
                    hop = 0;
                if (!hop \
                        && (hop = ipRouteNextHop[addr = net]) \
                        && (type = ipRouteType[addr]) == INVALID)
                    hop = 0;
                if (!hop && (hop = ipRouteNextHop[addr = "0.0.0.0"])) {
                    if ((type = ipRouteType[addr]) == INVALID)
                        hop = 0;
                    else
                        dr = 1;
                }
            }
\end{verbatim}
\end{bwslide}


\begin{bwslide}
\ctitle	{ROUTE TRACING (cont.)}

\vspace{0.5in}
\smaller
\begin{verbatim}
            if (hop) {
                printf "    via %-15s metric %2d%s\n",
                        hop, ipRouteMetric1[addr],
                        dr ? " (default route)" : "";
            }
            else {
                printf "\nno path to %s from %s\n%s\n", 
                        DEST, AGENT, DIAGNOSTIC;
                exit(1);
            }

            if (type == DIRECT) {
                printf "\tdirect route.\n";
                exit(0);
            }

            if (hop in tried)
                printf "\nrouting loop!\n";
            else
                tried[hop] = 1;
            AGENT = hop;
        }

        printf "\tdone.\n";
      }
\end{verbatim}
\end{bwslide}


\begin{bwslide}
\part*	{NEW DEVELOPMENT:\\ BULK TABLE RETRIEVAL}\bf

\begin{nrtc}
\item	IDEA IS TO RETRIEVE BIG TABLES QUICKLY

\item	PARALLEL ALGORITHM DEVELOPED BY:
    \begin{nrtc}
    \item	ROSE, McCLOGHRIE, DAVIN
    \end{nrtc}

\item	PRESENT IN \verb"snmpi" PROGRAM:
\begin{quote}\small\begin{verbatim}
% snmpi
snmpi> bulk -alg2 column1 ... columnN
\end{verbatim}\end{quote}
\end{nrtc}
\end{bwslide}


\begin{bwslide}
\ctitle	{THE SERIAL ALGORITHM}

\begin{nrtc}
\item	SIMPLISTIC APPROACH
\begin{quote}\small\begin{verbatim}
get-next (ipRouteDest)            -> ipRouteDest.0.0.0.0
get-next (ipRouteDest.0.0.0.0)    -> ipRouteDest.a.b.c.d
    ...
get-next (ipRouteDest.e.f.g.h)    -> ipRouteIfIndex.0.0.0.0
get-next (ipRouteIfIndex.0.0.0.0) -> ipRouteIfIndex.a.b.c.d
    ...
get-next (ipRouteMask.e.f.g.h)    -> ipNetToMediaIfIndex.1.a.b.c.d
\end{verbatim}\end{quote}

\item	SMARTER MANAGERS ASK FOR SEVERAL COLUMNS IN EACH REQUEST
\begin{quote}\small\begin{verbatim}
get-next (ipRouteDest, ipRouteIfIndex, ..., ipRouteMask)
\end{verbatim}\end{quote}
\end{nrtc}
\end{bwslide}


\begin{bwslide}
\ctitle	{THE PARALLEL ALGORITHM:\\ PHASE I}

\begin{nrtc}
\item	GOAL: RETRIEVE INSTANCES OF A SINGLE COLUMN:

\item	INVOKE SEVERAL \verb"get-next"s WITH SEVERAL OPERANDS

\item	NAMING SPACE IS DIVIDED UP BY GENERATING SUB-IDENTIFIER SUFFIXES,
	e.g.,
\begin{quote}\small\begin{verbatim}
get-next (ipRouteNextHop, ipRouteNextHop.63, ipRouteNextHop.127,
          ipRouteNextHop.159, ipRouteNextHop.192)
\end{verbatim}\end{quote}

\item	USE ADAPTIVE RETRANSMISSION STRATEGY TO
    \begin{nrtc}
    \item	MAXIMIZE THROUGHPUT

    \item	AVOID OVERLOADING AGENT
    \end{nrtc}
\end{nrtc}
\end{bwslide}


\begin{bwslide}
\ctitle	{PHASE I (cont.)}

\begin{nrtc}
\item	ALGORITHM KEEPS TRACK OF
    \begin{nrtc}
    \item	INVOCATIONS OUTSTANDING

    \item	OBJECT IDENTIFIERS BOUNDS ACTIVE IN EACH
    \end{nrtc}
    (SO INCOMING ORDER OF RESULTS IS UNIMPORTANT!)

\item	FOR EACH RESULT
    \begin{nrtc}
    \item	IF RESULT IS OUTSIDE ITS BOUNDS, IT IS REMOVED

    \item	OTHERWISE, NAME AND VALUE ARE STORED

    \item	AND THE BOUNDS IS SPLIT INTO TWO\\
		(DEPENDING ON OBSERVED PERFORMANCE OF AGENT AND NETWORK)
    \end{nrtc}

\item	WHEN ALL BOUNDS HAVE BEEN REMOVED, START SECOND PHASE
\end{nrtc}
\end{bwslide}


\begin{bwslide}
\ctitle	{THE PARALLEL ALGORITHM:\\ PHASE II}

\begin{nrtc}
\item	GOAL: RETRIEVE INSTANCES OF OTHER COLUMNS

\item	INVOKE SEVERAL \verb"get"s WITH SEVERAL OPERANDS

\item	INSTANCE-IDENTIFIERS ARE TAKEN FROM THE PHASE I RESULTS

\item	INITIAL PROTOTYPE IMPLEMENTATION ACHIEVES A SPEED-UP OF ROUGHLY AN
	ORDER OF MAGNITUDE OVER SERIAL ALGORITHM
\end{nrtc}
\end{bwslide}


\begin{bwslide}
\ctitle	{FOR FURTHER READING}

\begin{nrtc}
\item	4BSD/ISODE SNMP Roadmap

\item	How to export a MIB module from a BSD UNIX daemon using the
	4BSD/ISODE SMUX API

\item	SNMP MUX Protocol and MIB

\item	The Simple Book: An Introduction to Management of TCP/IP-based
	internets\\
	Marshall T.~Rose\\
	Prentice-Hall, 1990 (ISBN 0--13--812611--9)

\item	Bulk Table Retrieval with the SNMP\\
	RFC1187
\end{nrtc}
\end{bwslide}
