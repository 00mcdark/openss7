% run this through SLiTeX

\documentstyle[blackandwhite,landscape,oval,pagenumbers,plain,small]{NRslides}

\raggedright

\input trademark

\begin{document}

\title	{FOCUS ON OSI FOR\\ NETWORKED APPLICATIONS}
\author	{Marshall T.~Rose\\ The Wollongong Group}
\date	{November 18, 1987}
\maketitlepage


\begin{note}\em
all of the original work reported herein was done at the Northrop Research
and Technology Center
\begin{quote}
in particular, the support of Dr.~Stephen J.~Lukasik,\\
Vice President~---~Technology is gratefully acknowledged
\end{quote}

other individuals contributing to this work include:
\begin{nrtc}\em
\item	at UCL: Steve Kille

\item	at NPL: John Pavel

\item	at Nott: Julian Onions

\item	at NMA: Einar Stefferud
\end{nrtc}
\end{note}


\begin{bwslide}
\ctitle	{FUNDAMENTAL PREMISES}

\begin{nrtc}
\item	OSI/ISO WILL DOMINATE COMPUTER COMMUNICATIONS

\item	EXISTING INVESTMENTS (IN NON-OSI TECHNOLOGY) SHOULD BE PROTECTED

\item	BUT REALISTICALLY, EQUIPMENT HAS A HALF-LIFE OF 5 YEARS, SO$\ldots$

\item	NEW INVESTMENTS SHOULD EITHER BE OSI OR HAVE A HIGH DEGREE OF
	OSI-COMPATIBILITY
\end{nrtc}
\end{bwslide}


\begin{bwslide}
\ctitle	{WHAT IS OSI/ISO?}

\begin{nrtc}
\item	A LAYERED ARCHITECTURE FOR COMPUTER COMMUNICATIONS

\item	STANDARDIZED IN THE INTERNATIONAL COMMUNITY

\item	NON-PROPRIETARY IN NATURE
\end{nrtc}
\end{bwslide}


\begin{bwslide}
\ctitle	{FROM A COMMUNICATIONS VIEWPOINT}

\vskip.5in
\diagram[p]{figure1}
\end{bwslide}


\begin{bwslide}
\ctitle	{FROM A COMPUTER VIEWPOINT}

\vskip.5in
\diagram[p]{figure2}
\end{bwslide}


\begin{bwslide}
\ctitle	{(OBLIGATORY SLIDE SHOWING)\\ THE 7--LAYER STACK}

\vskip.5in
\diagram[p]{figure3}
\end{bwslide}


\begin{bwslide}
\ctitle	{THE UPPER-LAYER ARCHITECTURE}

\vskip.15in
\diagram[p]{figure4}
\end{bwslide}


\begin{bwslide}
\ctitle	{THE OSI APPLICATION LAYER}

\begin{nrtc}
\item	MANY STANDARD SERVICE ELEMENTS
    \begin{nrtc}
    \item	ASSOCIATION CONTROL

    \item	REMOTE OPERATIONS

    \item	RELIABLE TRANSFER

    \item	COMMITMENT, CONCURRENCY AND RECOVERY

    \item	DIRECTORY SERVICES

    \item	ABSTRACT SYNTAX NOTATION ONE\\
		(really a concept not an element, per se)
    \end{nrtc}
\end{nrtc}
\end{bwslide}


\begin{bwslide}
\ctitle	{FTAM USE OF UPPER-LAYER SERVICES}

\vskip.5in
\diagram[p]{figure5}
\end{bwslide}


\begin{bwslide}
\ctitle	{OSI SERVICES IN NON-OSI ENVIRONMENTS}

\begin{nrtc}
\item	START BUILDING AN OSI ENVIRONMENT ON TOP OF EXISTING ENVIRONMENTS

\item	BUILD SELECTED NEW APPLICATION SYSTEMS WITH OSI

\item	MIGRATE EXISTING APPLICATIONS TO AN OSI FRAMEWORK AS THE TECHNOLOGY
	BECOMES AVAILABLE

\item	CONTINUE RUNNING SELECTED SYSTEMS ``AS IS''

\item	PROOF OF CONCEPT: DECnet/ISO, STREAMS/TLI, ISODE
\end{nrtc}
\end{bwslide}


\begin{bwslide}
\ctitle	{SERVICE EMULATOR AT TRANSPORT}

\vskip.5in
\diagram[p]{figure6}
\end{bwslide}


\begin{bwslide}
\ctitle	{AN EXAMPLE:\\ ISO TRANSPORT SERVICES ON TOP OF DoD TCP}
\vskip-0.1in
\diagram[p]{figure7}
\end{bwslide}


\begin{bwslide}
\ctitle	{ISODE: THE ISO DEVELOPMENT ENVIROMENT}

\begin{nrtc}
\item	AN OPENLY AVAILABLE IMPLEMENTATION OF THE OSI UPPER-LAYERS

\item	CODED ENTIRELY IN C

\item	OPERATING SYSTEMS: BERKELEY AND AT\&T \unix/
    \begin{nrtc}
    \item	REQUIRES NO KERNEL MODIFICATIONS
    \end{nrtc}

\item	ALIGNED WITH THE U.S. GOSIP

\item	AT THE APPLICATION INTERFACE (ABOVE ACSE/ROSE),
	THROUGHPUT IS ONLY 10\%-12\% WORSE THAN RAW TCP FOR DATA TRANSFER
\end{nrtc}
\end{bwslide}


\begin{bwslide}
\ctitle	{THE APPLICATION ENVIRONMENT}

\vskip.15in
\diagram[p]{figure8}
\end{bwslide}


\begin{bwslide}
\ctitle	{AN ALTERNATE ENVIRONMENT:\\ MHS ARCHITECTURE (c.~1984)}

\vskip.15in
\diagram[p]{figure9}
\end{bwslide}


\begin{bwslide}
\ctitle	{TRANSPORT SERVICES}

\begin{nrtc}
\item	CURRENTLY A TP0 TRANSPORT SERVICE IS USED
    \begin{nrtc}
    \item	OVER X.25 (FOR EUROPEANS, et. al.)

    \item	OVER TCP (FOR DEFENSE DATA NETWORK)
    \end{nrtc}

\item	TCP-BASED SERVICE IS INDISTINGUISHABLE FROM A CONNECTION-ORIENTED
	NETWORK SERVICE

\item	WORK WILL START SOON ON INTEGRATING A NATIVE TP4
\end{nrtc}
\end{bwslide}


\begin{bwslide}
\ctitle	{WHERE NEXT?}

\begin{nrtc}
\item	UPGRADE TO FINAL (IS) SPECIFICATIONS

\item	``COOKED'' SUPPORT FOR REMOTE OPERATIONS

\item	INTEGRATION OF:
    \begin{nrtc}
    \item	MHS

    \item	DIRECTORY SERVICES

    \item	VIRTUAL TERMINAL
    \end{nrtc}
	IMPLEMENTATIONS DONE AT OTHER SITES

\item	INTEROPERABILITY/CONFORMANCE TESTING
\end{nrtc}
\end{bwslide}


\begin{bwslide}
\ctitle	{AVAILABILITY INFORMATION}

\begin{nrtc}
\item	VERSION 3 AVAILABLE OCTOBER 14, 1987

\item	USPS: SEND CHECK OR INVOICE FOR \$200 US DOLLARS TO:
    \[\begin{tabular}{l}
	ISODE DISTRIBUTION\\
	DEPARTMENT OF ELECTRICAL ENGINEERING\\
	UNIVERSITY OF DELAWARE\\
	NEWARK, DE  19716\\[0.25in]
	TELCO: 302--451--1163
    \end{tabular}\]

\item	DISTRIBUTION CONTAINS:
    \begin{nrtc}
    \item	1600bpi TAR TAPE

    \item	3 VOLUME DOCUMENTATION SET
    \end{nrtc}
\end{nrtc}
\end{bwslide}


\begin{bwslide}
\ctitle	{FOCUS ON OSI FOR NETWORKED APPLICATIONS}

\begin{nrtc}
\item	TOOLS TO FACILITATE DEVELOPMENT OF APPLICATIONS ARE CRITICAL

\item	IDEA IS TO DEVELOP TOOLS TO AUTOMATE USE OF OSI REMOTE OPERATIONS
	SERVICE AS A GENERAL REMOTE PROCEDURE CALL FACILITY

\item	ECMA TR/31: REMOTE OPERATIONS -- CONCEPTS, NOTATION AND
	CONNECTION-ORIENTED MAPPINGS (SECTIONS 1--4)
\end{nrtc}
\end{bwslide}


\begin{bwslide}
\ctitle	{ABSTRACT SYNTAX NOTATION 1 (ASN.1)}

\begin{nrtc}
\item	UNIVERSAL LANGUAGE TO DESCRIBE INFORMATION OBJECTS WITH STRONG TYPING

\item	RICH, EXTENSIBLE SYNTAX

\item	USEFUL FOR SPECIFICATION OF NEW PROTOCOLS
    \begin{nrtc}
    \item	CLEAR TO READ SPECIFICATIONS

    \item	NOT TIED TO MACHINE-ORIENTED STRUCTURES AND RESTRICTIONS
    \end{nrtc}

\item	REPRESENTATION CURRENTLY USED BY ALL OSI APPLICATIONS
\end{nrtc}
\end{bwslide}


\begin{bwslide}
\ctitle	{REMOTE OPERATIONS SERVICE (ROS)}

\begin{nrtc}
\item	STANDARDIZED MECHANISM FOR SPECIFYING TRANSACTIONS

\item	EMPLOYS FULL POWER OF ASN.1

\item	USED IN MANY INTERESTING OSI APPLICATIONS
    \begin{nrtc}
    \item	MESSAGING

    \item	DIRECTORY SERVICES

    \item	NETWORK MANAGEMENT

    \item	REMOTE DATABASE ACCESS
    \end{nrtc}
\end{nrtc}
\end{bwslide}

\begin{bwslide}
\ctitle	{GENERAL ORGANIZATION}

\begin{nrtc}
\item	AT COMPILE-TIME:
    \begin{nrtc}
    \item	USE RO-SPECIFICATION TO GENERATE SUPPORT FACILITIES
    \end{nrtc}

\item	AT RUN-TIME:
    \begin{nrtc}
    \item	USE DIRECTORY SERVICES TO LOCATE/REGISTER NETWORK SERVICES

    \item	USE ASSOCIATION CONTROL TO BIND/UNBIND APPLICATIONS

    \item	USE REMOTE OPERATIONS TO INVOKE TRANSACTIONS
    \end{nrtc}
\end{nrtc}
\end{bwslide}


\begin{bwslide}
\ctitle	{STATIC (COMPILE-TIME) ORGANIZATION}

\vskip.15in
\diagram[p]{figure10}
\end{bwslide}


\begin{bwslide}
\ctitle	{DYNAMIC (RUN-TIME) ORGANIZATION}

\vskip.15in
\diagram[p]{figure11}
\end{bwslide}


\begin{bwslide}
\ctitle	{CURRENT STATUS}

\begin{nrtc}
\item	STATIC FACILITIES
    \begin{nrtc}
    \item	ALL TOOLS/LIBRARIES ARE DEVELOPED AND IN VARYING STAGES OF
		BETA TESTING
    \end{nrtc}

\item	DYNAMIC FACILITIES
    \begin{nrtc}
    \item	MOST LIBRARIES ARE ALREADY WELL-TESTED

    \item	``REAL'' (DYNAMIC) DIRECTORY SERVICES IS TOO IMMATURE
		FOR SERIOUS IMPLEMENTATION
    \end{nrtc}

\item	AN ``APPLICATIONS COOKBOOK'' IS PLANNED BUT NOT YET WRITTEN
\end{nrtc}
\end{bwslide}


\end{document}
