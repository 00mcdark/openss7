% -*- LaTeX -*-

\newpage
\section	{Activity}
The focus on the pilot during the first-half of {\oldstyle 1990\/} has been
three-fold:
user-interface development;
bringing new sites online;
and,
routine maintenance of the software.

\subsection	{User-Interface Development}
There have been two advances in user-interface technology for the pilot:
a new search query language,
and a window-based interface to the White Pages.

\subsubsection	{User-Friendly Naming Scheme}
The original search syntax used by \pgm{fred} is based on the well-known WHOIS
service.
Although this has proven useful,
there has been a need for a syntax which is both more natural for human users
and has more expressive power.
A user-friendly naming scheme has been defined to meet these needs.

Names appear as ordered components,
e.g.,
\begin{quote}\small\begin{verbatim}
marshall rose, psi
kille, cs, ucl, gb
L. Eagle, "Sue, Grabbit, and Runn", Oxford
\end{verbatim}\end{quote}
However,
components may be missing,
e.g.,
\begin{quote}\small\begin{verbatim}
kille, ucl, gb
kille, cs, ucl, gb
\end{verbatim}\end{quote}
and still yield the same distinguished name.

The algorithm,
developed by Steve Kille,
and implemented by Colin Robbins
(both of University College London),
uses imprecise matching and assigns ``goodness'' levels to matches.
Users are queried for assistance on questionable matches in order to focus the
search.

The \pgm{fred} interface program has been modified to use this new algorithm,
as an optional user facility.
In the future,
use of the algorithm will become the default.

\subsubsection	{An Interface from X}
Although the \pgm{fred} interface appears adequate for dumb-terminal
applications,
a window-based interface to the white pages has much appeal.
PSI has developed a proprietary (yet easily licensable) interface between the
\xwindows/ and the OSI Directory.
The interface, \pgm{xwp}, implements the White Pages abstraction using a
windowing paradigm for user-interaction.

When \pgm{xwp} starts,
it's top-level window appears as:
% xwd | xwdtopbm | pnmflip -lr | pnminvert > figure1.pbm
\[\begin{minipage}\columnwidth
    \input figure1
    \centerline{\box\graph}
    \vskip\baselineskip
\end{minipage}\]
To search,
click on the \boxit{whois} button,
fill-in a name,
e.g.,
\[\begin{minipage}\columnwidth
    \input figure2
    \centerline{\box\graph}
    \vskip\baselineskip
\end{minipage}\]
and hit return.

Depending on the number of matches found,
the user may be presented with a list of possibilities,
e.g.,
\[\begin{minipage}\columnwidth
    \input figure3
    \centerline{\box\graph}
    \vskip\baselineskip
\end{minipage}\]

\newpage	%%%
Clicking on any of them will yield additional information,
e.g.,
\[\begin{minipage}\columnwidth
    \input figure4
    \centerline{\box\graph}
    \vskip\baselineskip
\end{minipage}\]

\newpage	%%%
To browse,
simply click on a line in the top-level window,
which \pgm{xwp} maintains as it learns about things in the White Pages,
e.g.,
\[\begin{minipage}\columnwidth
    \input figure5
    \centerline{\box\graph}
    \vskip\baselineskip
\end{minipage}\]

\subsection	{Maintenance}
There is little to report in regards to maintenance.
The QUIPU 6.0 release,
the baseline for the pilot participants is relatively stable.
The heavier usage encountered by the Level-0 DSAs has uncovered other problems.
These DSAs run a later revision of the software.

\newpage	
\subsection	{Numbers}
As a part of the pilot project,
PSI maintains the DIT for:
\begin{quote}\small\begin{verbatim}
countryName=US
organizationName=Internet
localityName=North America
\end{verbatim}\end{quote}

\subsubsection	{US}
The allocation of participating sites in the US portion of the pilot is:
\[\begin{tabular}{|r|rl|}
\hline
\multicolumn{1}{|c|}{\bf Type}&
			\multicolumn{2}{c|}{\bf Number}\\
\hline
University&		23&	\\
Corporate&		13&	\\
Government&		 7&	\\
Non-profit&		 6&	\\
Remote&			 5&	\\
\cline{2-3}
&			\bf 54&	\bf total\\
\hline
\end{tabular}\]
All organizations run their own DSAs with the exception of ``remote''
organizations;
PSI maintains the DMD for these organizations as a courtesy.

In addition,
three of the participating organizations run their own Level-2 DSAs:
\[\begin{tabular}{|c|c|}
\hline
\multicolumn{1}{|c|}{\bf \# of DSAs}&
			\multicolumn{1}{|c|}{\bf \# of DMDs}\\
\hline
3&		1\\
5&		2\\
\hline
\end{tabular}\]
As of this writing,
there were 224186 entries mastered and available under the \verb"c=US" subtree.
However,
15 (30\%) of the DSAs were either IP-unreachable or unavailable
(a disturbingly high figure, discussed in greater detail in
Section~\ref{uniformity}).

\newpage	%%%
\subsubsection	{Internet}
Two subtrees were maintained under the experimental \verb"o=Internet" subtree
of the DIT,
one for RFC documents,
the other for FYI documents.
A typical entry might appear as:
\begin{quote}\small\begin{verbatim}
RFC1160 (1)

The Internet Activities Board

Author: Vinton Cerf, Corporation for National Research Initiatives,
          US (2)

Version of: May 90

Information: Cerf, V.

Name:     RFC1160, RFC Documents,
            Internet (1)
Modified: Fri May 25 18:36:38 1990
      by: Manager, US (3)
\end{verbatim}\end{quote}
Note that if the author of a document has an associated distinguished name
then this is present in the entry.

As of this writing,
all RFCs and FYIs were present under this experimental portion of the DIT.

\subsubsection	{North America}
The portion of the DIT is automatically maintained,
containing organizational entries under both \verb"c=CA" and \verb"c=US".
This allows a user to search for a top-level organization in either country.
As the user-friendly naming scheme can accomplish this without this subtree,
it is likely to be discontinued in the future.

\subsection	{North American Directory Forum}
In early {\oldstyle 1990},
a group of electronic mail carriers formed an informal organization
with the purpose of realizing a public North American Directory Service.
PSI has been providing expert participation in the Forum,
as an organizational member,
in order to leverage the experiences of the pilot towards this goal.

\newpage
\section	{Problems thus far}
As of this writing,
the pilot has now been underway for over a year.
Not surprisingly,
several defects have been uncovered.

\subsection	{Uniformity of Service}\label{uniformity}
The largest problem is uniformity of service.
As noted earlier,
Not all DSAs are consistently available,
and of those which are available,
some are sparsely populated
(of the~54 participating organizations,
18~are sparse).
Thus,
even if a DSA is available,
it is quite possible that it does not contain the information it should.

This problem is one which is inherent in the approach taken by the
pilot,
and is administrative,
not technical, in nature.
Fundamentally,
the problem with an X.500 Directory is that it requires the construction of a
{\em new\/} infrastructure: data must be loaded into the Directory.

In contrast,
consider other work, e.g., \pgm{netaddress} (Droms at Bucknell)
or \pgm{netfind} (Schwartz at UC/Boulder):
\begin{itemize}
\item	The \pgm{netaddress} program knows about the various white pages
services in the Internet
(e.g., whois, finger, profile, etc.),
and simply contacts those services,
issues domain-specific queries and then filters the output into a uniform
presentation for the user. 

\item	The \pgm{netfind} program maintains a database,
incrementally built by scanning the ``Organization:'' fields in public netnews
messages,
and maintain relationships between domain and organization names.
When \pgm{netfind} is invoked,
it consults the database and issues \pgm{finger} commands as appropriate.
\end{itemize}
Both of these approaches {\em leverage\/} existing infrastructure and do not
require explicit site administrator action~---~at the target site~---~in order 
to provide the service.
Although such services are generally limited to persons with computer accounts,
they cleverly avoid the problem of building new infrastructure.

Although some steps have been taken by the pilot sponsors
(e.g., a script is run each evening to find DSAs which are unavailable),
the administrators of unavailable DSAs seem little motivated to fix the
problems.%
\footnote{This suggests that the pilot administrators should consider removing
DMDs which are largely sparse or unavailable.}

The solution to these problems is one of demand:
if applications can be provided which use the White Pages,
then the White Pages service will be in larger demand,
and site administrators will be more likely to populate and maintain their
DSAs.
In brief,
one must rely on the superior quality-of-service of the X.500 approach to
motivate the community to take-on the added expenses of building a new
infrastructure.

Unfortunately,
this may very well be the proverbial chicken-and-egg problem:
coarse statistics monitoring shows little use of the White Pages service,
presumably because little information is stored there!

\subsection	{Speed of Search}
There is an impression that search speed is too slow.  Steve Kille
has suggested that a DSA can search 2K-entries/sec on a Sun-4/330 for a
typical filter.

Presumably this is a ``simple matter of coding''
UCL has begun working on this,
but it will probably take additional programmer resources to tune the code
appropriately.

\subsection	{User-Interfaces}
Many sites have expressed a wish for interfaces for the PC and the MAC.
A PC interface should probably run over both IP and Novell.
Similarly,
a MAC interface should probably run over both IP and Appletalk.

This is also a simple matter of coding,
although the architecture for such a system should be a split-DUA model:
the code on the PC or MAC should be talk to a real DUA on a UNIX box,
so the PC and MAC code would not require the ISODE.

It has also been suggested that even \pgm{fred} is too slow and that perhaps
it too could benefit from a split-DUA model,
using a lightweight protocol based on fast-IPC or shared-memory.

\subsection	{Access Control}
Administrative limits severely hinder browsing capability,
but are needed to (crudely) address external security concerns.
The proper solution is to devise a means of differentiating between local and
remote users so as have differing administrative limits.

\subsection	{Attribute Inheritance}
EDB files, for real institutions, are huge, owing to the repeated presence of
the same attributes in each entry.
Attribute inheritance is needed.
This should also help for ACLs.

UCL has added attribute inheritance,
and this will likely go into extensive use later this year.

\newpage
\section	{Documents}
Since the publication of the previous status report,
these have been produced:
\begin{itemize}
\item	{\em Realizing the White Pages using the OSI Directory Service},
	by Rose,
	33~pages.

This describes the refinements made to the X.500 standards in order to achieve
a working system.

\item	{\em An X Window System Interface to the White Pages},
	by Rose,
	36~pages.

This presents a PSI-proprietary user-interface to the Directory.
\end{itemize}
In addition,
independent of the work on the pilot,
one other documents have been produced by the ISODE/QUIPU effort which
is germane to white pages:
\begin{itemize}
\item	{\em Using the OSI Directory to achieve User Friendly Naming},
	by Kille,
	24~pages.

This introduces the notion of simple, user-oriented naming strings,
and how the OSI Directory Service may be used to map these into distinguished
names.
\end{itemize}
