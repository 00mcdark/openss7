% -*- LaTeX -*-

\newpage
\section	{Introduction}
In July of {\oldstyle 1989},
NYSERNet started a pilot project offering a White Pages service using OSI
technology.
Since that time,
NYSERNet and PSI have been extending and evaluating the service in response to
experience gained during the operation of the pilot.

The service is interesting in three respects:
it is a large,
distributed information service involving administration by multiple
organizations;
it is the first production-quality field test of the OSI Directory (X.500);
and,
it is the first large scale production application of OSI technology on top of
the popular Internet (TCP/IP) suite of protocols.

The purpose of this document is to describe the ``refinements'' made to the
base X.500 Recommendations in order to produce a workable system.
It must be {\bf stressed\/} that none of these refinements are contrary to
either the spirit or letter of the X.500 protocols.
Rather,
if one begins with the assumption that one has a conformant X.500
implementation,
then these refinements consist of fleshing out the ``for further study'' and
``local matter'' clauses of the X.500 Recommendations.
Further,
one should note that a system implementing these refinements will still be
able to interwork,
albeit with less functionality,
with conformant X.500 implementations that do not implement these refinements.

For the remainder of this note,
the reader is assumed to have a basic understanding of X.500 technology.

The purpose in presenting this information is to advance understanding of
these grey areas by describing operational experiences in offering a Public
Directory Service using X.500.
Having stated what this note is about,
it is also appropriate to state what this note is not about:
this note is not meant to provide any implementation-specific information.
That is,
this note considers only the externally visible aspects of a Directory Service
which incorporates these refinements.
As such,
these refinements should be applicable to {\bf any\/} conformant X.500
implementation.

This note begins by reviewing the need for a White Pages Service and presents
a basic model for offering such a service using the X.500 protocols.
Next,
a brief overview of the technology used to realize this service is presented.
Following this,
this note is divided into two major sections:
first,
issues relating to the Directory System
(e.g., refinements in the operation of a DSA or use of the DSP)
are discussed;
second,
issues relating to a Directory User
(e.g., refinements in the operation of a DUA or use of the DAP)
are considered.

\newpage
\subsection	{For Further Reading}
There is very little ``new'' information presented in this note.
Rather,
it is a compilation of several sources,
organized to focus on the refinements necessary to produce a workable system
using X.500 technology.
Interested readers might wish to consult:
\begin{itemize}
\item	\cite{WPP.Intro},
	for an introduction to the NYSERNet/PSI White Pages Pilot Project;

\item	\cite{WPP.Report-1},
	for the latest status report (as of this writing);

\item	\cite{WPP.User,WPP.XWP},
	for information on user interfaces;

\item	\cite{WPP.Admin},
	for information on administering a DMD within the pilot
	(i.e., a PRDMD);

\item	\cite{QUIPU.Manual}
	for programmatic information on the underlying X.500 implementation.
\end{itemize}
All of these documents (and many others) were consulted when this note was
written.

\newpage
\subsection	{Managing Infrastructural Information}
A natural function of computer networks is to form the {\em infrastructure\/}
between the users they interconnect.
For example,
the electronic mail service offered by computer networks provides a means for
users to collaborate towards some common goal.
In the simplest cases,
this collaboration may be solely for the dissemination of information.
In other cases,
two users may work on joint research project,
using electronic mail as their primary means of communication.

Most network services are based on the implicit assumption that each user can
supply {\em infrastructural information} to 
facilitate information transfers through the network.
For example,
electronic mail services expect that an originator can supply 
addressing information 
for all the intended recipients.
It is not necessarily the task of electronic mail, per se,
to provide this infrastructural information to the user.

This model works fine in small environments,
particularly those where infrastructural information is not difficult to 
obtain and remember.
However,
the model does not scale well.
Consider the case when the membership of a network consists of hundreds of
thousands of users belonging to thousands of organizations.
It is no longer reasonable for a single user to provide this information,
except in very limited circumstances.
Further,
it is likely that some of the information changes frequently,
due to personnel and other resource movement.
The goal of a {\em white pages\/} service is to 
provide the necessary information, and to mask the complexity of the
infrastructural information.

\subsection	{A Public Directory Service}
In our context,
a public directory service allows a user to ascertain information
about a person whose name is (approximately) known.
In providing such a service using technology based on the X.500
Recommendations,
it is necessary to view this somewhat more concretely.

In particular,
we assume that the information on a person is represented as an entry in the
OSI Directory.
As such,
the ``handle'' associated with each person is a Directory Distinguished Name
(DN).
However,
we further assume that a user need not know the value of this name.
Rather,
the user provides commonly known naming information about a person,
and the user interface to the Directory somehow determines the correct DN,
and then presents publically readable information from the entry associated
with that DN.

In the NYSERNet/PSI White Pages Pilot Project,
this first step is assumed to be an iterative process:
first,
``areas'' of the Directory likely to contain the entry are identified,
usually by searching;
and,
second,
in each of those areas,
a search for the entry is initiated.
In a business context,
the area corresponds to an organizational object in the Directory.
In contrast,
for a personal context,
the area might correspond to a locality.
In the pilot project,
only the organizational case has been extensively used.
Nonetheless,
experience with the pilot software indicates that the search model is
sufficiently general to support either context.

\subsection	{An Implementation}
To implement a White Pages service using the OSI Directory,
three things are needed:
\begin{itemize}
\item	an OSI infrastructure,
	this is provided by the ISODE,
	an openly available implementation of the upper-layers of OSI;

\item	an implementation of the OSI Directory,
	this is provided by the ISODE implementation of the Directory,
	QUIPU;
	and,

\item	a White Pages abstraction,
	provided by an administrative discipline along with at least one user
	interface through which the service is accessed.
\end{itemize}
It is important to distinguish between the White Pages {\em service\/} and the
OSI Directory {\em technology} as defined in
\cite{ISO.Directory,CCITT.Directory}.
The White Pages abstraction is provided both by a focused use of the
underlying OSI Directory technology
and by special user interfaces.
Of course,
many might view the sole focus of the OSI Directory is to provide a white
pages service.
As such,
the abstraction can be seen simply as the administrative and operational
agreements necessary to use X.500.

The ISODE (pronounced {\em I-SO-D-E\/}),
or {\em ISO Development Environment}, is a collection of library
routines and programs that implements an extensive set of OSI upper-layer
services\cite{Open.Book}.%
\footnote{It is an unfortunate historical coincidence that the first three
letters of the ISODE are ``ISO''.
This is not an acronym for the International Organization for Standardization,
but rather three letters which,
when pronounced in English,
produce a pleasing sound.}
In brief,
the ISODE implementation of the upper-layers of OSI is interesting in four
respects:
it provides extensive automatic tools for the development of OSI applications;
it supports OSI applications on top of both OSI and TCP/IP-based networks;
it provides a novel approach to the problems of OSI coexistence
and transition;
and,
it is openly available (non-proprietary).

The ISODE implementation of the OSI Directory,
QUIPU\cite{QUIPU.Directory},
was donated by University College London.
QUIPU was originally developed as a part of the INCA
(Integrated Network Architecture) project
(under the auspices of the ESPRIT initiative of the EEC).
The Inca of Peru did not have writing.
Instead,
they stored information on strings,
carefully knotted in a specific manner with colored thread,
and attached to a larger rope.
Such a device was known as a {\em Quipu\/}
(pronounced {\em kwip-ooo}).
The encoding was obscure,
and could only be read by selected trained people:
the {\em Quipucamayocs}.
The Quipu was a key component of Inca society,
as it contained information about property and locations throughout
the extensive Inca empire.

QUIPU is a complete implementation of the OSI Directory,
based on the {\oldstyle 1988\/} CCITT recommendations\cite{CCITT.Directory}
and ISO standards\cite{ISO.Directory}.
As with the entire ISODE,
QUIPU is coded in the {\em C\/} programming language\cite{C.Language} and runs
on the \unix/ operating system.

The QUIPU Directory User Agent (DUA) supports a {\em C\/} language procedural
interface to access the full Directory Access Service.
The interface is relatively straight-forward,
mapping ``programmer-friendly'' procedure calls to the formal service.
Code to manipulate ASN.1 structures is automatically generated using
facilities provided by the ISODE. 

The DUA interface can be used directly at the programmatic level,
or exported from a interface process called \verb"dish"~---~the DIrectory
SHell.

The QUIPU Directory System Agent (DSA) is memory-based:
it uses the native \unix/ file-system to provide stable-storage between
reboots, but otherwise maintains all data in program memory.
As might be expected,
providing that the DSA avoids paging,
execution of the lookup and search facilities of the Directory can be realized
in a timely fashion.
Naturally, when an update operation occurs,
the copy on disk is updated and a journal entry written before the update is
acknowledged.
The disk copy is stored in a textual format to facilitate examination.
As this copy is read only once~---~when the DSA starts,
typically when \unix/ goes multi-user~---~the cost of such a strategy is
believed to be relatively small if properly implemented and tuned.

The DSA supports both Directory distributed operations (DSA-DSA) and the
Directory Abstract Service (DUA-DSA),
along with the full range of standard Directory object classes and attributes
types.
Further,
a large number of other attribute types have been defined,
both to facilitate experimentation and support the White Pages service.
Most notably these attribute types were necessary to support automatic use of
replication of information in the Directory.

Having now briefly introduced the implementation used in the pilot project,
we now describe the refinements which were necessary to offer a working system.

\newpage
\section	{Directory System Issues}
The system-related refinements used in the pilot project are all artifacts of
the QUIPU design\cite{QUIPU.Design}.
These are best described in terms of the object classes and associated
attribute types which are manipulated by a QUIPU DSA.
However,
before these are introduced it is necessary to discuss two topics:
the textual convention used for writing DNs,
and how the DSAs navigate the DIT.

\subsection	{Writing Distinguished Names}
Although the textual notation used to write DNs should be unimportant.
For expository (and local) purposes it is important to have a concise notation.

A Distinguished Name is written as an ordered series of RDNs separated by
an \verb"`@'"-sign with the most significant RDN appearing at the left;
e.g.,
\begin{quote}\small\begin{verbatim}
c=US@o=NYSERNet Inc.
\end{verbatim}\end{quote}
refers to an entry with an RDN of \verb"o=NYSERNet Inc."
whose parent has an RDN of \verb"c=US".
In turn,
this parent entry is an immediate child of the root.
To avoid any potential ambiguity,
one usually prefixes a \verb"`@'"-sign to a string when referring to a fully
qualified Distinguished Name;
e.g.,
\begin{quote}\small\begin{verbatim}
@c=US@o=NYSERNet Inc.
\end{verbatim}\end{quote}
always refers to the same entry regardless of context.
Finally,
if an RDN consists of multiple attributes,
then these are separated by a \verb"`%'"-sign;
e.g.,
\begin{quote}\small\begin{verbatim}
@l=North America@o=NYSERNet Inc.%l=US
\end{verbatim}\end{quote}

Tables~\ref{attribute-std} and~\ref{attribute-new}
on pages~\pageref{attribute-std} and~\pageref{attribute-new},
respectively,
list the attribute types commonly used in the pilot project,
any abbreviations, and the syntax associated with each attribute's value.
In the interests of brevity,
the abbreviated form is used when writing DNs.

\subsection	{DIT Navigation}
When a DUA requests some action from a DSA
(e.g., to read an entry),
the DSA may not have that information resident.
In this case,
the DSA has a choice:
it may either contact another DSA which is ``closer'' to the information and
propagate the request
(i.e., {\em chaining\/}),
or it may return information about this ``closer'' DSA to the DUA,
and let the DUA re-issue its request (i.e., {\em referral\/}).
Of course,
when DSAs communicate between themselves,
they may also chain or refer requests.

The key issue is to understand what the term ``resident'' means with respect
to information held by a DSA.
In order to improve performance,
DSAs {\em employ\/} both replication and caching of information in the
Directory.
Hence,
to explain residency of information,
some additional terminology must be introduced.

\subsubsection	{Entry Data Blocks}
First,
we think of the entire global Directory tree as consisting of discrete units
of information termed {\em entry data blocks},
or simply {\em blocks}.
A block consists of a small portion of the tree:
the immediate children of a particular node.
Thus the phrase ``this DSA holds a copy of the block for \verb"c=US"'' means
that the DSA in question has information about the {\em attributes\/} of 
the nodes which exist immediately below the \verb"c=US" entry,
such as
\begin{quote}\small\begin{verbatim}
c=US@o=NYSERNet Inc.
\end{verbatim}\end{quote}
The phrase does not imply that the DSA has any information, whatsoever,
about {\em entries\/} beneath these immediately subordinate nodes,
e.g.,
\begin{quote}\small\begin{verbatim}
c=US@o=NYSERNet Inc.@ou=Research and Development
\end{verbatim}\end{quote}
There are three kinds of blocks that a DSA might hold:
\begin{itemize}
\item	a {\em slave\/} copy refers to complete, authoritative information of
a block:
the DSA knows about all of the entries and all of the children of the node
corresponding to that particular block.
Slave copies are regularly updated by contacting an {\em upstream\/} DSA,
and comparing timestamps associated with the two copies of the block.

\item	a {\em cache\/} copy refers to possibly partial information that a DSA
has received during its execution lifetime:
the DSA may know about some of the children and possibly some of their attributes.
When a DSA chains a request for some information and the information returns
through the DSA,
that DSA may decide to cache that information for some short period of time
(e.g., 30~minutes).

\item	the term {\em master\/} copy is self-evident.
Only a single DSA may hold the master copy for any portion of the tree.
\end{itemize}

With these terms now described,
the {\em residency requirement\/} is simply enumerated as:
\[\begin{tabular}{|r|l|}
\hline
\multicolumn{1}{|c|}{\bf Operation Requested}&
		\multicolumn{1}{|c|}{\bf Copy Required for Residency}\\
\hline
read, compare&	master, slave, or cache\\
list, search&	master, or slave\\
update&		master\\
\hline
\end{tabular}\]
(Actually,
a cached copy might be used to satisfy a list request,
depending on the service controls used with the operation.)

Hence,
while the pilot relies primarily on replication (slave blocks) to speed
queries,
updates still rely on a centralized entity (containing the master block) being
available. 
(This is the best compromise can be taken without making the system
tremendously more complex,
e.g.,
introducing distributed database technology.)

Finally,
note that the EDB approach is much coarser than a
``distributed entry'' capability.
Although distributing across several DSAs the attribute and children
information for a single entry may be administrative attractive,
it is felt to be technically intractable at this time.

\subsubsection	{Use of Presentation Address}
Only within a single OSI community can homogeneous interworking be achieved.
It is important to realize that the value of a white pages service increases
as its scope increases.
As such,
it is important to appreciate that OSI applications are implemented on a wide
range of end-to-end protocol combinations.
Sadly,
these are largely incompatible (e.g., TP4/CLNP and TP0/X.25).
In order to provide homogeneous service to the user,
the DSAs must be able to compare two OSI presentation addresses and determine
if an association can be established directly between the two.
This need arises in two cases.

First,
a DSA wishes to perform an operation and realizes that it must either chain or
refer.
The DSA compares the presentation address of the application entity
on whose behalf it is performing the operation,
with the presentation address of a DSA which is closer to the information.
If the two addresses belong to the same ``community''
(i.e., are compatible),
then the DSA knows that it is safe to return a referral.
If not,
then the DSA realizes that,
barring service controls to the contrary,
it should chain.

Second,
a DSA might wish to contact other DSAs.
In this case,
the DSA sees if the presentation address of the other DSA is in one of the
communities of the local end-system.
If so,
the DSA knows that it is possible to chain operations to that DSA.
Otherwise,
in the absence of transport-level bridging,
the DSA knows that it can not reach the other DSA.

To further complicate matters,
Directory service may be offered by application entities using non-standard
end-to-end protocol combinations
(e.g., RFC1006/TCP \cite{TSAP.on.TCP}).
There must be a means whereby the addresses associated with such entities can
be stored in the Directory and subsequently interpreted by entities with
knowledge of these non-standard communities.
A mechanism for achieving this is defined in \cite{Interim.Addresses}.

Finally,
the textual notation used to write presentation addresses should be
unimportant.
For expository (and local) purposes it is important to have a concise notation.
\cite{String.Addresses} defines such a notation.

\subsection	{Objects}
The abstract syntax of the classes and attributes described in this note are
largely taken from the THORN naming architecture\cite{thorn-na},
which was developed in the RARE community.
These are optional extensions to what is defined in the X.500 recommendations.

\subsubsection	{Object Class: friendlyCountry}
In order to provide more intuitive identification of sovereign nations,
a new object class,
subordinate to the \verb"country" class,
has been added.
This has but a single mandatory attribute:
\begin{describe}
\item[friendlyCountryName:]
			gives a user-friendly rendition of the
			country's name (hence the term \verb"friendlyCountry").
\end{describe}

\subsubsection	{Object Class: domainRelatedObject}
In order to experiment with mappings from user's Internet mailboxes into DNs,
a means for inverting the Internet Domain Name System (DNS) is needed.
A new object class, \verb"domainRelatedObject", is used for this purpose.
There is one mandatory attribute:
\begin{describe}
\item[associatedDomain:]
			identifies a portion of the DNS which is conceptually
			located at an entry of this class.
\end{describe}

\subsubsection	{Object Class: quipuObject}
In order to provide access control to the Directory,
an \verb"accessControlList" attribute type,
consisting of zero or more positive-access rules,
has been defined.
This is a attribute type is mandatory of objects within the \verb"quipuObject"
class.
Further,
all entries occuring in an EDB must be of this class.

In addition,
there are two optional attribute types.
These are maintained automatically by the DSA:
\begin{describe}
\item[lastModifiedBy:]
			identifies the Directory entity which last modified
			this entry.

\item[lastModifiedTime:]
			identifies the time at which this entry was last
			modified.
\end{describe}

\subsubsection	{Object Class: quipuNonLeafObject}
In order to provide navigational information for DSAs,
all non-leaf objects are within the \verb"quipuNonLeafObject".
There is one mandatory attribute type:
\begin{describe}
\item[masterDSA:]
			identifies the Directory entity which is responsible
			for maintaining the MASTER EDB for the children of
			this entry.
\end{describe}
In addition,
there are two optional attribute types:
\begin{describe}
\item[slaveDSA:]
			identifies DSA(s) which also have authoritative
			information on the immediate children of a non-leaf
			entry.

\item[treeStructure:]
			if present, lists those object classes which the
			immediate children are allowed to take.
\end{describe}

\subsubsection	{Object Class: quipuDSA}
It is necessary to introduce an additional object class,
\verb"quipuDSA",
to identify those DSAs
which implement these refinements.
There are several mandatory attribute types:
\begin{describe}
\item[eDBinfo:]
			Indicates how the DSA participates when propagating
			authoritative EDB information.
			Each attribute lists the DN of a non-leaf entry,
			along with (optionally) the DN of the upstream DSA
			which provides the associated EDB to this DSA,
			and with (optionally) a list of DNs of downstream DSAs
			to which this DSA provides the associated EDB.
			(The dual of the \verb"masterDSA" and \verb"slaveDSA"
			attribute types.)

\item[userPassword:]
			a string containing a password used by the DSA to
			authenticate itself to other DSAs.
			This attribute is not used for DSP authentication
			as it introduces a livelock problem.
			(When strong authentication matures,
			DSP authentication will be used instead.)

\item[manager:]
			the DN of an entry corresponding to the ``super-user''
			for this DSA.

\item[quipuVersion:]
			a simple string relating the version of the software
			being run by this DSA.
\end{describe}
along with the \verb"acl" attribute.
There is also an optional one:
\begin{describe}
\item[description:]
			a textual description of the DSA.
\end{describe}
along with the \verb"lastModifiedBy" and \verb"lastModifiedTime" attribute
types.

A DSA which implements these refinements also has the value
\verb"quipuDSP" for its \verb"supportedApplicationContext" attribute.
In addition to offering all of the DSP services on this application context,
there is an additional operation,
\verb"getEDB",
which is used by a downstream DSA to request EDB information from an upstream
DSA,
if that EDB information has changed.

When a \verb"quipuDSA" must contact another DSA,
and several choices are available,
it favors those DSAs of the \verb"quipuDSA" class.
For example,
 caching can not occur without an understanding of the access control
policy being applied to an entry.

\subsection	{Naming DSAs}
If a \verb"quipuDSA" does not have information resident to satisfy a request,
it identifies,
to its local knowledge,
the furthest non-leaf entry along the path to the desired entry.
The \verb"masterDSA" and \verb"slaveDSA" attributes of this entry are
consulted to derive the DNs of DSAs which are more likely to be able to
perform the desired operation.

Obviously,
in order to resolve the \verb"presentationAddress" attribute of the entries
corresponding to the closer DSAs,
additional lookup in the Directory is needed.
As such,
a DSA should be at least one level-higher in the DIT than any EDB for which it
holds authoritative information.

To see why this is so,
consider a DSA named:
\begin{quote}\small\begin{verbatim}
c=US@o=NYSERNet Inc.@cn=losing
\end{verbatim}\end{quote}
which,
according to the entry for
\begin{quote}\small\begin{verbatim}
c=US@o=NYSERNet Inc.
\end{verbatim}\end{quote}
is a \verb"masterDSA",
and that there are no \verb"slaveDSA"s for that entry.
Now suppose another DSA,
called ``naive'',
with knowledge only of \verb"c=US",
wishes to satisfy a request for a (possibly distant) child of this
organization.
In order to satisfy the request,
the ``naive'' DSA must know the address of the ``losing'' DSA,
which can obviously not be determined.

\subsection	{DSA Management}
In order to provide a management capability for these refinements,
a new attribute type, \verb"control", is defined for use over the DAP.
When a DUA is bound as the DSA's manager,
it may issue the modify operation to add a value of type \verb"control".
This is interpreted by the DSA rather than being applied to the DIT.

The operations of interest are:
\begin{itemize}
\item	abort DSA;

\item	restart DSA;

\item	refresh EDB from local stable-storage;

\item	rewrite EDB to local stable-storage;

\item	mark EDB as read-only (lock EDB);

\item	unlock EDB;

\item	initiate EDB update procedure from upstream DSA
	(either for all EDBs or a particular EDB).
\end{itemize}
For example,
if it is necessary to make a substantive change to an EDB for which the DAP is
unappropriate,
a DSA manager tells the DSA to lock the EDB,
edits the EDB on local stable-storage,
tells the DSA re-read the EDB from local stable-storage,
and then tells the DSA to unlock the EDB.

Intead,
if a DUA issues a read for the \verb"control" attribute of an entry,
then the value returned consists of a string:
\begin{quote}\scriptsize\begin{verbatim}
%d Master entries (in %d EDBs), %d Slave entries (in %d EDBs), %d Cached entries
\end{verbatim}\end{quote}
giving some very basic information about the EDBs/entries resident in the DSA.

\newpage
\section	{Directory User Issues}
An important task in any pilot project is to objectively focus the scope of
the project:
projects with ill-defined goals tend to fail.
In the context of the pilot project,
an explicit decision was made to focus solely on data relating to persons and
organizations.
Thus, although the pilot project software supports the full range of Directory
objects and attributes, the objects supported in the pilot project are limited
to:
\begin{quote}\small\begin{tabular}{l}
organizations\\ organizational units\\ organizational roles (e.g., ``Chair of
the Department'')\\ localities (for those individuals not belonging to an
organization)\\ persons
\end{tabular}\end{quote}
In addition to supporting the appropriate attribute types for these objects,
some additional attributes were defined.  For example, users can store a
facsimile image of themselves as their \verb"photo" attribute.

The list of standard attribute types supported in the pilot is shown in
Table~\ref{attribute-std}.
Table~\ref{attribute-new} shows additional attribute types which have been
defined for use with the pilot.
\tagtable[tp]{INT-1}{Supported Standard Attribute Types}{attribute-std}
\tagtable[tp]{INT-2}{Additional Attribute Types}{attribute-new}

\subsection	{Objects}
In the pilot project,
all of the objects described above are modeled by the standard the
corresponding object class,
with one exception.

\subsubsection	{Object Class: pilotPerson}
The \verb"pilotPerson" object class is used to refer to a person in the pilot
project.
An entry belonging to this class may have any of several additional attributes:
\begin{describe}
\item[favouriteDrink:]
			which is a string describing the user's favorite drink.

\item[homePhone:]
		which is a string describing the phone number of the object
		using the international notation; e.g., ``+1 518-283-8860''.

\item[homePostalAddress:]
			which describes how physical mail is addressed to the
			person's home.

\item[info:]
			which is additional, textual information about the
			person.

\item[mobileTelephoneNumber:]
			which is a string describing the user's mobile
			number (e.g., for a cellular phone).

\item[otherMailbox:]
			which is the user's computer mail address
			in various domains.
			The string syntax of this attribute's value is special:
\begin{quote}\small\begin{verbatim}
<domain> $ <mailbox>
\end{verbatim}\end{quote}
			e.g., ``internet \$ mrose@psi.com''.
The current list of mail domains are:
applelink, bitnet, compuserve, genie, internet, mcimail, nasamail,
preferred, and uucp.

\item[pagerTelephoneNumber:]
			which is a string describing the user's pager number.

\item[photo:]
			which is a facsimile bitmap of the user's face.

\item[rfc822Mailbox:]
			which is the user's computer mail address in the
			Internet.

\item[roomNumber:]
			which is a string describing where the person resides
			at the location.

\item[secretary:]
			which is the Distinguished Name of the user's
			administrative support.

\item[userClass:]
			which describe's the user's classification; e.g.,
			``staff''.

\item[userid:]
			which is the user's login name; e.g., ``mrose''.
\end{describe}
Thus,
Table~\ref{pilot-person} shows all the attributes that might be associated
with a person.
\tagtable[tp]{INT-3}{Attribute Types for the pilotPerson Object Class}%
	{pilot-person}

\subsubsection	{Other Object Classes}
For completeness sake,
Tables~\ref{friendlyCountry-attributes}
through~\ref{organizationalRole-attributes} starting on
page~\pageref{friendlyCountry-attributes} list the attributes supported for
the standard object classes in use by the pilot.
\tagtable[tp]{INT-4}{Attribute Types for the friendlyCountry Object Class}%
	{friendlyCountry-attributes}
\tagtable[tp]{INT-5}{Attribute Types for the organization Object Class}%
	{organization-attributes}
\tagtable[tp]{INT-6}{Attribute Types for the organizationalUnit Object Class}%
	{organizationalUnit-attributes}
\tagtable[tp]{INT-7}{Attribute Types for the locality Object Class}%
	{locality-attributes}
\tagtable[tp]{INT-8}{Attribute Types for the alias Object Class}%
	{alias-attributes}
\tagtable[tp]{INT-9}{Attribute Types for the organizationalRole Object Class}%
	{organizationalRole-attributes}

\subsection	{Naming}
In the pilot project,
the structure of the DIT is straight-forward:
at the root,
countries and country-level DSAs reside;
in \verb"c=US",
organizations and organizational-level DSAs reside.
Beyond this,
the administrators of the \verb"c=US" DMD are not concerned with how
individual organizations are internally structured.

In addition to this simple structure,
to aid searching in North America,
there is also a top-level locality,
\verb"l=North America",
which contains aliases to all organizations in the \verb"c=US" and \verb"c=CA"
DMDs,
e.g.,
\begin{quote}\small\begin{verbatim}
@l=North America@o=NYSERNet Inc.%l=US
\end{verbatim}\end{quote}
which has an \verb"aliasedObjectName" attribute value of
\begin{quote}\small\begin{verbatim}
@c=US@o=NYSERNet Inc.
\end{verbatim}\end{quote}

The tasks of the country-level DSAs
(there are currently three for \verb"c=US")
are:
\begin{itemize}
\item	provide authoritative information on the root and \verb"c=US"
	for replication purposes to organizational-level DSAs;
	and,

\item	act as a conduit for chaining to/from foreign DMDs.
\end{itemize}
The former task is to allow high replication of top-level information to speed
searching.
The latter task is to permit a means of harmonizing between different
end-to-end protocol combinations.

The RDN for organizations is the full name.
However,
to aid searching,
other name forms are allowed as attribute values.
For example,
the entry 
\begin{quote}\small\begin{verbatim}
@c=US@o=Performance Systems International
\end{verbatim}\end{quote}
might have an additional attribute value of ``PSI'' which is not used in its
RDN.
Because the Directory searches on the basis of attribute values and not RDNs,
a user supplying a search string of ``PSI'' will find (at least) the right
entry.

\subsection	{Searching}
The single most important aspect of the user interface is to relate the naming
architecture to a search algorithm.
The interaction model is simple.
The user provides the interface with:
\begin{enumerate}
\item	(optionally) the kind of object being looked for;

\item	naming information about where the object resides;
	and,

\item	(optionally) qualifying information on the object.
\end{enumerate}
Based on the kind of object being looked for,
the user interface constructs a ``basic'' filter:
\[\begin{tabular}{|r|l|}
\hline
\multicolumn{1}{|c|}{\bf What}&
		\multicolumn{1}{|c|}{\bf Basic Filter}\\
\hline
default&	\tt cn$\approx$X $\lor$ sn$\approx$X $\lor$ mail=X@*\\
organization&	\tt o$\approx$X\\
unit&		\tt ou$\approx$X\\
role&		\tt cn$\approx$X\\
locality&	\tt l$\approx$X\\
person&		\tt sn$\approx$X $\lor$ mail=X@*\\
(or) person&	\tt cn$\approx$X\\
\hline
\end{tabular}\]
where ``$\approx$'' indicates either wildcard, imprecise, or exact matching,
based on user-preference,
and ``$=$'' indicates exact matching.
If wildcarding is desired,
but the user does not explicitly specify the substrings,
the string ``\verb"*X*"'' is used.

After the basic filter is constructed,
if the user supplied qualifying information on the object,
an ``info'' filter is appended to the basic filter using logical-AND.
\[\begin{tabular}{|r|l|}
\hline
\multicolumn{1}{|c|}{\bf What}&
		\multicolumn{1}{|c|}{\bf Info Filter}\\
\hline
organization&	\tt businessCategory=X\\
unit&		\tt businessCategory=X\\
role&		\tt title=X\\
person&		\tt title=X\\
\hline
\end{tabular}\]
Next the depth of search must be determined:
\[\begin{tabular}{|r|l|}
\hline
\multicolumn{1}{|c|}{\bf What}&
		\multicolumn{1}{|c|}{\bf Depth}\\
\hline
default&	whole-subtree\\
organization&	single-level\\
unit&		whole-subtree\\
role&		whole-subtree\\
locality&	single-level\\
person&		whole-subtree\\
\hline
\end{tabular}\]
Finally,
a search is performed with these service controls:
\begin{quote}\small\begin{verbatim}
-preferchaining
-dontdereferencealias
-notimelimit
-nosizelimit
-nosearchaliases
-types rfc822Mailbox telephoneNumber aliasedObjectName
-values
\end{verbatim}\end{quote}
The last two lines are probably somewhat cryptic.
They indicate that for all matched entries,
the DSA should return any values associated with those three attribute types.

The task of the user interface is simple:
\begin{itemize}
\item	if no matches are found,
	it reports this to the user;

\item	if exactly one match is found,
	then it reads this entry and displays its to the user;
	otherwise,

\item	if more than one match is found,
	the user interface displays each entry in a concise one-line format
	containing either the mailbox or telephone number.
	This aids the user in determining which entry might be the one
	actually desired.
	The user can then direct the interface to expand that particular entry.
\end{itemize}
If an entry is read,
then the service controls are:
\begin{quote}\small\begin{verbatim}
-preferchaining
-dontdereferencealias
-notimelimit
-nosizelimit
\end{verbatim}\end{quote}

All that need be explained now is how the ``baseobject'' for the search is
determined.
Recall that the user might have supplied something called
\begin{quote}
``naming information where the user resides''
\end{quote}
to the interface.
For each kind of object,
the user interface maintains a default area (DN) used for searching.
These are set by the system administrator,
e.g.,
\[\begin{tabular}{|r|l|}
\hline
\multicolumn{1}{|c|}{\bf What}&
		\multicolumn{1}{|c|}{\bf Base}\\
\hline
default&	\tt @c=US@o=NYSERNet Inc.\\
organization&	\tt @l=North America\\
unit&		\tt @c=US@o=NYSERNet Inc.\\
role&		\tt @c=US@o=NYSERNet Inc.\\
locality&	\tt @c=US\\
person&		\tt @c=US@o=NYSERNet Inc.\\
\hline
\end{tabular}\]
in order to provide sensible searching.
The user may override this by supplying additional naming information.

So,
the user interface decides if the additional information was for an
organization, organizational unit, or locality,
constructs a basic filter, and performs a search.
\begin{itemize}
\item	if no matches are found,
	it reports this to the user;

\item	if exactly one match is found,
	then it uses that DN as the baseobject for the search;
	otherwise,

\item	if more than one match is found,
	the user interface displays each entry in a concise one-line format
	containing either the mailbox or telephone number.
	This aids the user in determining which entry might be the one
	actually desired.
	The user can then direct the interface to continue searching in the
	appropriate areas.
\end{itemize}
Note that prior to using one of these area entries as a baseobject,
the user interface sees if the search returned an \verb"aliasedObjectName"
attribute for the area.
If so,
then the value of this attribute is used for the baseobject.
This is used,
for example,
when the areas being searched are under \verb"l=North America".

\subsection	{Browsing}
To browse the White Pages,
a listing capability is needed.
Rather than use the Directory list operation,
it is usually more direct to perform a single-level search looking for any
entry which has
\begin{quote}\small\tt
objectClass$\neq$dSA
\end{quote}

\subsection	{Searching Revisited}
Actually,
there are a couple of additional searching capabilities provided by the White
Pages.

Although its use is discouraged
interfaces should be able to accept a DN as typed by a user for either naming
or area information.

If the naming information given by the user appears to be an Internet-style
mailbox,
then an entirely different search algorithm is used.
Suppose the user enters ``\verb"X"'' in the form:
\begin{quote}\small\begin{verbatim}
local@domain
\end{verbatim}\end{quote}
and either there is wildcarding present in \verb"domain",
or the user specifies area information.
In this case,
a search filter is constructed,
one of
\begin{quote}\small\begin{verbatim}
rfc822Mailbox=*@domain
\end{verbatim}\end{quote}
if no \verb"local" portion was given in ``\verb"X"'';
\begin{quote}\small\begin{verbatim}
rfc822Mailbox=local@*
\end{verbatim}\end{quote}
if no \verb"domain" portion was given in ``\verb"X"'';
or,
\begin{quote}\small\tt
rfc822Mailbox=X $\lor$ otherMailbox="internet \$ X"
\end{quote}
otherwise,
and a whole-subtree search is done.

Otherwise,
the user does not specify area information and there is no wildcarding present
in \verb"domain",
the user interface performs an special search to identify those areas likely
to contain the mailbox.
For each DN in the list of areas,
the interface uses that DN as a base object for a whole-subtree search using
the filter:
\begin{quote}\small\tt
rfc822Mailbox=X $\lor$ otherMailbox="internet \$ X"
\end{quote}
So,
how is this list of areas generated?
The user-interface repeatedly performs single-level searches looking for
entries which have an \verb"associatedDomain" value equal to the domain in
question.
If a match is not found,
then a sub-domain is stripped off,
and the search is tried again.
If one or matches are found,
the routine recurses and searching begins anew.
In this fashion,
the routine finds the deepest entries in the DIT associated information about
a (sub-)domain of the DNS.

\newpage
\section	{Acknowledgements}
The NYSERNet/PSI White Pages Pilot Project builds on the work of many others:
\begin{itemize}
\item	At the core,
	the ISODE provides the programmatic infrastructure for OSI
	applications.
	So many people have contributed to the ISODE that it is presumptuous
	to attempt to list them.

\item	The QUIPU directory from University College London,
	designed by Stephen E.~Kille and programmed primarily by
	Colin J.~Robbins, form the heart of the pilot's functionality.

\item	Although NYSERNet and PSI have devoted considerable resources to the
	hardening of QUIPU,
	it should be noted that UCL has been responsible for the vast majority
	of improvements and fixes.

\item	The white pages abstraction was designed primarily at NYSERNet with
	the help of several experts: Kille, Robbins, and Einar Stefferud.

\item	Geoffrey S.~Goodfellow was kind enough to be the alpha tester for the
	white pages abstraction.

\item	The white pages administrators have been patient as the software
	twitches.

\item	Finally,
	it should be noted that none of this would be possible if it
	weren't for the excellent end-to-end services provided by the Internet
	suite of protocols (namely the TCP and the IP).
\end{itemize}

\newpage
\section*	{Appendix: Differences from NIST OIW Agreements}
Although QUIPU attempts to be faithful to various Directory implementation
profiles,
it is not entirely successful.
QUIPU is believed to be aligned with with the NIST OIW Agreements on the
Directory,
with these exceptions:
\begin{itemize}
\item	QUIPU doesn't enforce the bounds constraints on attributes, filters or 
APDU size.

\item	QUIPU does not reject T.61 string formatting characters.

\item	QUIPU does not check to if the value of the \verb"countryName"
attribute is defined in ISO3166.

\item	If a DN is supplied with no password in an unprotected simple bind,
then QUIPU does not check to see if the DN exists.

\item	When comparing attribute of UTCtime syntax,
if the seconds field is omitted,
then QUIPU does not perform the match correctly.
(i.e.,
the seconds field in the attribute values should be ignored, but are not).

\item	QUIPU always supplies the optional Chaining argument \verb"originator"
even if the CommonArgument \verb"requestor" is used.

\item	QUIPU always supplies the optional Chaining argument \verb"target"
even if the base object in the DAP arguments is the same.
\end{itemize}
