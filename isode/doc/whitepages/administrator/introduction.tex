% run this through LaTeX with the appropriate wrapper

\chapter	{Introduction}
This document is {\em The Administrator's Guide\/} for the
PSI White Pages Pilot.
The goal of \theguide/ is to provide a site administrator
with enough information to be able to participate in the White Pages
Pilot.
In practical terms,
this means that \theguide/ provides information on how
to install and maintain an OSI Directory at your site.

The OSI Directory is used to provide the white pages service.
\theguide/ is not intended as a tutorial nor a detailed description
of the OSI Directory.
However, as administrative responsibilities and procedures are
outlined,
the appropriate Directory concepts will be introduced.%
\footnote{Actually this paragraph is a lie:
you can't administer the white pages without knowing what you're
doing,
and you don't know what you're doing unless you understand both
the basics of the OSI Directory and how the pilot project software
implements the Directory Service.
As such,
this entire chapter is really a thinly-veiled primer on these
two topics.}

Your comments are welcome!
The OSI Directory is a new, complex technology.
Although \theguide/ attempts to be straight-forward it probably doesn't
succeed all the time.
If you have comments on this document, send them to the Internet
mailbox
\begin{quote}\begin{verbatim}
wpp-camayocs@uu.psi.com
\end{verbatim}\end{quote}
so that \theguide/ can be improved.

\newpage

\section*	{Related Documentation}
The white paper
{\em An Introduction to a NYSERNet White Pages Pilot Project\/}
introduces the goals and phases of the pilot project.
Administrators should familiarize themselves with this document
before proceeding.

There is also a document which you should provide to the user community at
your organization.
It is called
{\em PSI White Pages Pilot Project: User's Handbook},
or simply \thebook/.

The OSI Directory standard is defined in \cite{ISO.Directory,CCITT.Directory}.

The research note {\em The Design of QUIPU\/} \cite{QUIPU.Design}
describes the design of the software used for the pilot project,
whilst {\em Volume Five\/} of the ISODE User's Manual \cite{QUIPU.Manual},
henceforth termed \volfive/,
describes the implementation of the software.
\theguide/ is intended to replace these later two documents as the
primary reference for administrators of the pilot project.

PostScript versions of these documents are available via anonymous FTP
from host \verb"uu.psi.com" (\verb"[136.161.128.3]"):
\begin{quote}\small\begin{verbatim}
% cd /usr/src/local/
% ftp uu.psi.com
Connected to uu.psi.com.
220 uu.psi.com FTP server ready
Name (uu.psi.com:user): anonymous
Password (uu.psi.com:anonymous): guest
331 Guest login ok, send ident as password.
230 Guest login ok, access restrictions apply.
ftp> binary
200 Type set to I.
ftp> cd pub/isode
ftp> get isode-ps.tar.Z
ftp> get pilot-ps.tar.Z
ftp> quit
% uncompress < isode-ps.tar.Z | tar xf -
% uncompress < pilot-ps.tar.Z | tar xf -
% rm isode-ps.tar.Z pilot-ps.tar.Z
\end{verbatim}\end{quote}

\newpage
\section	{Informational Model}
From a theoretical standpoint,
the internals of the white pages service can be viewed from 
four different perspectives.
We begin with the Informational Model which describes the service in terms of
information objects.
The Directory's representation of an information object,
typically called an {\em entry},
contains information about a person, a place, an organization, etc.
Each entry consists of one or more attributes.

Each attribute consists of a type,\index{attribute types}
indicating what kind of attribute it is,
and one or more values
(one of which is termed the {\em distinguished value\/}).
Attribute values are structured using a data definition language
called  Abstract Syntax Notation One (ASN.1).
As such,
different programs using the Directory will interpret information in the same
way.
In addition,
the Directory will perform type-checking on the values in order
to keep things consistent.

\subsection	{Naming}
One of the attributes of an entry is particularly special:
it is referred to as the {\em Relative Distinguished Name\/} (RDN) of
the entry.
The RDN is formed by taking the name of the attribute and its
distinguished value.
For example,
if the attribute in question was called \verb"countryName" and it had
a distinguished value of \verb"US", then we might say that the RDN
for the entry was \verb"countryName=US".
Of course,
this is strictly a ``user-friendly'' notation:
the Directory uses a concise binary format for representing an RDN.
Fortunately,
the pilot project software allows simple textual strings to be used in their
place and converts back and forth accordingly.

In the OSI Directory,
information is primarily organized according to a hierarchical tree
structure.
The top of the tree is termed the {\em root},
and has no explicit name.
To find the  name of an object,
termed its {\em Distinguished Name\/} (DN),
one concatenates the RDNs found when traversing the tree by starting
at the root and proceeding directly to the object's entry.

For purposes of discussion,
we write a Distinguished Name as an ordered series of RDNs separated by
an \verb"`@'"-sign with the most significant RDN appearing at the left;
e.g.,
\begin{quote}\smaller\begin{verbatim}
countryName=US@organizationName=Performance Systems International
\end{verbatim}\end{quote}
refers to an entry with an RDN of
\verb"organizationName=Performance Systems International"
whose parent has an RDN of \verb"countryName=US".
In turn,
this parent entry is an immediate child of the root.

To avoid any potential ambiguity when using an interface to the Directory
such as \man fred(1c) or \man dish(1c),
one prefixes a \verb"`@'"-sign to a string when referring to a fully
qualified Distinguished Name;
e.g.,
\begin{quote}\small\begin{verbatim}
@countryName=US@organizationName=Performance Systems International
\end{verbatim}\end{quote}
always refers to the same entry regardless of context.
Note that this is a convention only for interface programs such as these.

As a rule,
unless searching,
text before the \verb"`='"-sign is not case sensitive,
neither is text after the \verb"`='"-sign.

Of course, names like \verb"countryName",
while more friendly than binary strings,
are still rather long,
so there are several abbreviations that the pilot project software permits.
Table~\ref{attribute-types} lists the attribute types supported by
the pilot project, any abbreviations, and the syntax associated with
each attribute's value.
Throughout \theguide/,
the semantics of these attributes will be introduced.
When using this document later on,
refer to the Index to quickly find a particular attribute you are interested
in.
\tagtable[tp]{INT-1}{Supported Attribute Types}{attribute-types}
\tagtable[tp]{INT-1b}{Supported Attribute Types (cont.)}\empty

The entities which access the Directory on behalf of a user
are termed {\em Directory User Agents\/} (DUAs).
The entities which provide the Directory service are termed
{\em Directory System Agents\/} (DSAs).
Both kinds of entities are identified by their Distinguished Name.
That is,
the DSAs and DUAs are not only network processes,
they are also objects which have entries in the Directory.

\subsection	{How Entries and Objects are Described}\label{edb:format}
It is now time to explain the notation used to describe entries.
This notation is termed the ``EDB format'',
and is named after the disk-resident data structures used by the
pilot project software.

An entry is textually described in an ASCII formatted file.
The format of the file is simple:
\begin{itemize}
\item	Each entry consists of two or more lines followed by a blank line.

\item	Each line consists of an attribute type/value pair using
	the familiar
\begin{quote}\small\begin{verbatim}
name=value
\end{verbatim}\end{quote}
notation,
such as:
\begin{quote}\small\begin{verbatim}
c=US
\end{verbatim}\end{quote}

\item	If multiple values are present for an entry,
	then either multiple lines may be used or the values may
	be separated by the \verb"`&'"-sign.
	The first value occurring is the distinguished value.
	Hence,
\begin{quote}\small\begin{verbatim}
co=US
co=United States of America
\end{verbatim}\end{quote}
and
\begin{quote}\small\begin{verbatim}
co=US & United States of America
\end{verbatim}\end{quote}
are identical:
both say that the attribute \verb"friendlyCountryName" has two values:
the distinguished value is \verb"US" and an alternate value is
\verb"United States of America".

\item	Most attributes take textual strings for their value.
	However, for a large number of attributes,
	the actual characters used are limited to a subset of the
	``printable'' characters: 
\begin{quote}\begin{tabular}{l}
\verb"A" through \verb"Z"\\
\verb"a" through \verb"z"\\
\verb"0" through \verb"9"\\
\verb"'" (apostrophe)\\
\verb"(" (left parenthesis)\\
\verb")" (right parenthesis)\\
\verb"+" (plus-sign)\\
\verb"," (comma)\\
\verb"-" (hyphen)\\
\verb"." (period)\\
\verb"/" (solidus)\\
\verb":" (colon)\\
\verb"?" (question-mark)\\
space
\end{tabular}\end{quote}
	In particular, note that the characters \verb"`='", \verb"`&'",
	\verb"`$'" and \verb"`#'" are not allowed.

	If it makes sense to use a particular character, e.g., \verb"`@'"
	in a mail address,
	then this usage is usually permitted.

	If it is necessary to characters from the restricted set,
	proceed the value with \verb"{T.61}", e.g.,
\begin{quote}\small
\begin{verbatim}
{T.61}whatever$you$want
\end{verbatim}\end{quote}

	Always use the appropriate case (upper or lower) when entering
	values.
	Although the Directory supports imprecise matching for searching,
	direct lookup is made on the basis of an exact match.

	When strings are read by the pilot project software,
	it strips leading and trailing blanks and also compresses multiple
	blanks to a single blank space.

\item	Another common kind of attribute value is a Distinguished
	Name, which is specified using the DN notation described earlier.
	Note that a DN is always interpreted in absolute form when it is
	written in an EDB file.
	Hence, the leading \verb"`@'"-sign is never written.

\item	Of course,
	there are a few attributes with very special syntaxes,
	when these attributes are introduced,
	the associated notation will be discussed.
\end{itemize}
Note that the first line for an entry gives the object's RDN.
This line should be repeated later on in the entry,
this causes it to be included in the attributes for the object.
(The pilot software includes this attribute automatically,
but it is a good practice regardless.)

An EDB file contains all of the child objects of a node in the tree.
The first line of the file consists of a single keyword, one of:
\begin{describe}
\item[MASTER:]	indicating that this EDB file is the original source of the
		information;

\item[SLAVE:]	indicating that this EDB file is a copy of the information
		from an authoritative source;
		and,

\item[CACHE:]	indicating that this EDB file is a partial, unauthoriative
		copy of the information.
\end{describe}
The second line of the file contains the creation date of the information
in the file.
This is expressed as a \verb"UniversalTime".
Put simply,
the format used is:\label{UniversalTime}
\begin{quote}\small\begin{verbatim}
yyyymmddhhmmssZ
\end{verbatim}\end{quote}
as in:
\begin{quote}\small\begin{verbatim}
19890509213614Z
\end{verbatim}\end{quote}
which is May 9$^{\underline{\mbox{\scriptsize th}}}$, {\oldstyle 1989},
21:36:14 hours at UT.

Hence,
the format of an EDB file is:
\begin{quote}\small\begin{verbatim}
<keyword>
yyyymmddhhmmssZ
<entries>
\end{verbatim}\end{quote}
(remember that each entry ends with a blank line).
Figure~\ref{edb-example} shows an example of a simple EDB file
containing two entries.
\tagfigure[tp]{INT-1}{An Example of an EDB file}{edb-example}

\subsection	{Object Classes}\index{object classes}\index{attribute types, objectClass}
One attribute of an entry,
its \verb"objectClass",
determines what kind of object this entry corresponds to (e.g., a person).
The value of this attribute indicates what types of attributes the
entry {\em must\/} and {\em may\/} contain.
For example,
if the value of \verb"objectClass" indicates that the entry corresponds
to a person,
then it would make sense for that entry to have a \verb"surName"
attribute.
On the other hand,
if the value of \verb"objectClass" indicates that the entry corresponds
to an organization,
then a \verb"surName" attribute would be inappropriate.

The OSI Directory is flexible in that it defines several common
types of objects,
and then allows users to define their own objects.
Further,
object definition is based on the notion of {\em class inheritance}.
This means that an object class can be defined as a ``subclass'' of
a previously defined object class with additional refinements.
As a subclass,
the newly defined object ``inherits'' all the semantics of its
superclass,
in addition to having additional semantics.

For example,
the Directory defines an object class called \verb"person".
This object class defines the attributes which a person in
the real world might have.
It may be useful to refine this somewhat to talk about persons
who have Internet access.
So, we need a new object class, e.g., \verb"internetPerson".
This can be defined in a straight-forward fashion:
\begin{quote}
The object class \verb"internetPerson" is a subclass of the object class
\verb"person" which {\em may\/} contain an additional attribute,
\verb"internetMailbox".

The syntax of an \verb"internetMailbox" is a simple string of printable
characters which is not case sensitive when performing comparisons.
\end{quote}
Thus,
when entries are defined with an \verb"objectClass" attribute which contains
the value \verb"internetPerson":
\begin{itemize}
\item	they {\em must\/} contain whatever attributes an entry with
	\verb"objectClass" \verb"person" {\em must\/} contain;

\item	they {\em may\/} contain whatever attributes an entry with
	\verb"objectClass" \verb"person" {\em may\/} contain;
	and,

\item	they {\em may\/} contain the \verb"internetMailbox" attribute.
\end{itemize}

As was seen in Figure~\ref{edb-example} on page~\pageref{edb-example},
the \verb"objectClass" attribute is multi-valued.
For our purposes,
object class values fall into one of two categories:
\begin{itemize}
\item	the primary class,
	of which each object has one
	(this is the class that inherited all the properties from the
	superclasses);
	and,

\item	one or more superclasses.
\end{itemize}
In order to limit the focus of the pilot project,
only a few primary classes are supported.
Further,
of those classes,
not all optional attributes are supported.%
\footnote{This limitation is specific to the pilot project;
the pilot project software can support the entire range of object classes
and attribute types.}
Before discussing the primary classes and their class-specific attributes,
there are a few attributes which are required for every object.
These are introduced by looking at a few key superclasses.

\subsubsection	{The top objectClass}\index{object classes, top}
The Directory requires that every object have \verb"top" as a super
class.
Only one attribute is required for members of this class:
\begin{describe}
\item[objectClass:]\index{attribute types, objectClass}
			which defines all the classes that this object belongs
			to.
\end{describe}

\subsubsection	{The quipuObject objectClass}\index{object classes, quipuObject}
In addition,
the pilot project software requires that every object also have the object
class \verb"quipuObject" as a superclass.
Only one attribute is required for members of this class:
\begin{describe}
\item[accessControlList:]\index{attribute types, accessControlList}\index{attribute types, acl (accessControlList)}
			which defines how users of the Directory may
			access the entry.
\end{describe}
The \verb"accessControlList" attribute uses a special notation to enter
its value in the EDB file:
\begin{quote}\small\begin{verbatim}
<syntax> ::= <who> "#" <action> "#" <what>

<who> ::=       "group" "#" DN
            |   "prefix" "#" DN
            |   "others"
            |   "self"

<action> ::=    "none"
            |   "detect"
            |   "compare"
            |   "read"
            |   "add"
            |   "write"

<what> ::=      "child"
            |   "entry"
            |   "default"
            |   "attributes" "#" <attribute-list>

<attribute-list> ::=
                attribute-name
            |   attribute-name "$" <attribute-list>
\end{verbatim}\end{quote}
The \verb"<who>" part indicates who the rule applies to.
Recall that when a user connects to the Directory,
the user {\em binds\/} on behalf of a Distinguished Name.
This DN is used to determine which rules apply to the user:
\begin{describe}
\item[group:]	indicates a DN which has an \verb"objectClass" attribute
		of \verb"groupOfNames" or \verb"organizationalRole",
		or it indicates a particular DN, which is the most
		common case.

\item[prefix:]	indicates all DNs which start with the indicated
		prefix.  Thus, access may be granted to all users
		at or below a particular part of the tree.

\item[self:]	indicates the DN which corresponds to this object.

\item[others:]	the default case.
\end{describe}
The \verb"<what>" part indicates what part of the entry the rule applies
to:
\begin{describe}
\item[child:]	the access control rule applies to children immediately below
		the entry.

\item[entry:]	the access control rule applies to the entry itself.

\item[attributes:] the access control rule applies to exactly those
		attributes listed.
		Multiple attributes pertaining to the same rule are
		separated by the \verb"`$'"-sign.

\item[default:]	the default rule for attributes.
\end{describe}
The \verb"<action>" part indicates the actions which are permitted.
The semantics of this part vary, depending on the \verb"<what>" part.
However,
the \verb"<action>" values are ordered from least- to most-significant;
e.g., having ``read'' permission implies both ``compare'' and ``detect''
permission.
\begin{describe}
\item[none:]
\begin{tabular}[t]{rp{3.35in}}
for entries:&	 	any knowledge of the entry is hidden\\
for attributes:&	any knowledge of the attribute is hidden\\
for children:&		downwards progress is blocked
\end{tabular}

\item[detect:]
\begin{tabular}[t]{rp{3.35in}}
for entries:& determination of the entry's existence is permitted\\
for attributes:& determination of the attribute's existence is permitted\\
for children:& the existence of children is admitted, but downwards progress is still blocked\\
\end{tabular}

\item[compare:]
\begin{tabular}[t]{rp{3.35in}}
for entries:& the RDN may be compared\\
for attributes:& the values of the attribute may be compared\\
for children:& exactly specified RDNs may be matched\\
\end{tabular}

\item[read:]
\begin{tabular}[t]{rp{3.35in}}
for entries:& the RDN may be read\\
for attributes:& the attribute values may be read\\
for children:& the child information may be listed, and downward searches are permitted\\
\end{tabular}

\item[add:]
\begin{tabular}[t]{rp{3.35in}}
for entries:& new attributes may be added\\
for attributes:& new attribute values may be added\\
for children:& new children may be added\\
\end{tabular}

\item[write:]
\begin{tabular}[t]{rp{3.35in}}
for entries:& the RDN may be changed and existing attributes may be added\\
for attributes:& values may be modified or removed\\
for children:& children may be removed\\
\end{tabular}
\end{describe}
The only exception to these rules is the \verb"accessControlList" attribute
itself: access control rules may be added only to the extent access is
allowed by the user.

The default access control list is
\begin{quote}\small\begin{verbatim}
acl=
\end{verbatim}\end{quote}
which simply means ``read'' access for everything and also
\begin{quote}\small\begin{verbatim}
self # write # entry
self # write # default
\end{verbatim}\end{quote}

There are two optional attributes.
These are maintained automatically by the Directory:
\begin{describe}
\item[lastModifiedBy:]\index{attribute types, lastModifiedBy}
			identifies the Directory entity which last modified
			this entry.
			The value is a Distinguished Name.

\item[lastModifiedTime:]\index{attribute types, lastModifiedTime}
			identifies the time at which this entry was last
			modified.
			The value is a \verb"UniversalTime",
			the syntax of which was described earlier on
			page~\pageref{UniversalTime}.
\end{describe}

\subsubsection	{The quipuNonLeafObject objectClass}\index{object classes, quipuNonLeafObject}
If an object is permitted children,
the pilot project software requires that that object also have
\verb"quipuNonLeafObject" as a superclass.

This class has one mandatory attribute:
\begin{describe}
\item[masterDSA:]\index{attribute types, masterDSA}
			identifies the Directory entity which is responsible
			for maintaining the MASTER EDB for the children of
			this entry.
			The value is a Distinguished Name.
\end{describe}
There is typically a single MASTER for a particular entry in the tree.
Hence, this value is usually single-valued.
When an entry is to be modified,
the Directory must contact the entity responsible for the MASTER EDB for
that entry in order to perform the modification.

This class has two optional attributes:
\begin{describe}
\item[slaveDSA:]\index{attribute types, slaveDSA}
			identifies any Directory entities which have
			authoritative copies of the EDB for the children
			of this entry.
			The value is one or more Distinguished Names.

\item[treeStructure:]\index{attribute types, treeStructure}
			identifies the object classes which may exist
			immediately below this entry.
			The value is one or more object classes.
\end{describe}
Since a fundamental assumption of the Directory is that reads (queries)
occur much more frequently than writes (updates),
it is common to have several entities containing authoritative copies of an
EDB.
By keeping copies locally,
queries can be answered with less latency.

\subsubsection	{The domainRelatedObject objectClass}\index{object classes, domainRelatedObject}
If an object has some relationship to the Internet Domain Name System (DNS),
the pilot project software requires that that object also have
\verb"domainRelatedObject" as a superclass.

This class has one mandatory attribute:
\begin{describe}
\item[associatedDomain:]\index{attribute types, associatedDomain}
			identifies the domain which corresponds to this object.
			The value is a domain string, e.g.,
\begin{quote}\small\begin{verbatim}
psi.com
\end{verbatim}\end{quote}
\end{describe}
There is typically a single MASTER for a particular entry in the tree.
Hence, this value is usually single-valued.
When an entry is to be modified,
the Directory must contact the entity responsible for the MASTER EDB for
that entry in order to perform the modification.

Now we look at the primary object classes.
Keep in mind that these all have both \verb"top" and \verb"quipuObject" as
superclasses.
Further,
the primary classes which are permitted children all have
\verb"quipuNonLeafObject" as a superclass.

\subsubsection	{The friendlyCountry objectClass}\index{object classes, friendlyCountry}
Objects of this class represent a sovereign nation.
For the purposes of the PSI White Pages Pilot,
there is only one of these,
although the pilot project is attached to a world-wide Directory service.

Table~\ref{friendlyCountry-attributes} summarizes the attributes types for
a \verb"friendlyCountry" object.
There are two mandatory attributes:
\begin{describe}
\item[countryName:]\index{attribute types, countryName}\index{attribute types, c (countryName)}
			which gives the name of the country.
			The value of this attribute is a two-letter ISO~3166
			code; e.g.,
\begin{quote}\small\begin{verbatim}
US
\end{verbatim}\end{quote}

\item[friendlyCountryName:]\index{attribute types, friendlyCountryName}\index{attribute types, co (friendlyCountryName)}
			which gives a user-friendly rendition of the
			country's name (hence the term \verb"friendlyCountry").
			The value of this attribute is a string; e.g.,
\begin{quote}\small\begin{verbatim}
United States of America
\end{verbatim}\end{quote}
\end{describe}
\tagtable[tp]{INT-2}{Attribute Types for the friendlyCountry Object Class}%
	{friendlyCountry-attributes}

In addition,
there is one attribute that may be present:
\begin{describe}
\item[description:]\index{attribute types, description}
			which is a simple textual description;
			e.g.,
\begin{quote}\small\begin{verbatim}
the land of the free and the home of the brave
\end{verbatim}\end{quote}
\end{describe}

Putting everything together,
one could describe an entry for an object of class \verb"friendlyCountry"
thusly:
\begin{quote}\small\begin{verbatim}
c=US
masterDSA= cn=Alpaca#
slaveDSA= cn=Fruit Bat#
slaveDSA= cn=Giant Tortoise# 
acl= others # read # entry
acl= group # c=US@cn=Manager # write # entry
acl= others # read # default
acl= group # c=US@cn=Manager # write # default
treeStructure= quipuNonLeafObject & organization & locality
treeStructure= quipuDSA & alias & organizationalRole
c= US
co= USA & US & United States of America
objectClass= top & quipuObject & quipuNonLeafObject
objectClass= country & friendlyCountry
\end{verbatim}\end{quote}
These lines indicate that:
\begin{itemize}
\item	The RDN for this object is \verb"c=US"
	(read ``country name is US'').

\item	The user-friendly name for this country takes on one of three values
\begin{quote}\small\begin{verbatim}
USA
US
United States of America
\end{verbatim}\end{quote}

\item	There are two DSAs which have SLAVE copies of the EDB for this
	object, one is called \verb"cn=Giant Tortoise" and the other
	is called \verb"cn=Fruit Bat".
	(We'll discuss how these colorful names are chosen later on.)

\item	The DSA responsible for maintaining the MASTER copy of the EDB
	is called \verb"cn=Alpaca".
	Note that all three of these DSAs live directly under the ROOT of
	the Directory tree.

\item	Everyone is allowed to read the entry and all of its attributes,
	and only \verb"c=US@cn=Manager" is allowed to change anything.

\item	The \verb"treeStructure" attribute says that the children immediately
	below this object:
    \begin{itemize}
     \item	may have children themselves (\verb"quipuNonLeafObject"),
		in which case, the objects under \verb"c=US" are either
		\verb"organization" or \verb"locality" objects; or,

     \item	needn't have children, being a DSA
		(of object class \verb"quipuDSA")
		or an \verb"organizationalRole";
		or,

    \item	may be an \verb"alias".
    \end{itemize}

\item	The official name of this country is \verb"US".

\item	This object belongs to five object classes:
    \begin{itemize}
    \item	\verb"top" and \verb"quipuObject" because membership is
		mandatory;

    \item	\verb"quipuNonLeafObject" so that children may reside
		under the entry;

    \item	\verb"friendlyCountry" which is the primary object class; and,

    \item	\verb"country" which is a superclass of \verb"friendlyCountry".
    \end{itemize}
\end{itemize}

\subsubsection	{The organization objectClass}\index{object classes, organization}
Objects of this class represent a top-level organizational entity,
such as a corporation, university, government entity, and so on.
There is an \verb"organization" object for each organization participating in
the PSI White Pages Pilot.

Table~\ref{organization-attributes} summarizes the attributes types for
an \verb"organization" object.
There is one mandatory attribute:
\begin{describe}
\item[organizationName:]\index{attribute types, organizationName}\index{attribute types, o (organizationName)}
			which gives the name of the organization.
			The value of this attribute is a string; e.g.,
\begin{quote}\small\begin{verbatim}
Performance Systems International
\end{verbatim}\end{quote}
\end{describe}
\tagtable[tp]{INT-3}{Attribute Types for the organization Object Class}%
	{organization-attributes}

There are several attributes that may be present,
which are divided into three groups:
physical address, telecommunications address, and miscellaneous information.

The first group describes the physical address of the object:
\begin{describe}
\item[postalAddress:]\index{attribute types, postalAddress}
			which describes how physical mail is addressed to the
			object.
			The syntax of this attribute's value is special:
			it consists of~1 to~6 fields, seperated by the
			``\verb"$"''-sign, each field being from~1 to~30
			characters long; e.g.
\begin{quote}\tiny\begin{verbatim}
Performance Systems International $ 11800 Sunrise Valley Drive $ Reston, VA  22091
\end{verbatim}\end{quote}

\item[registeredAddress:]\index{attribute types, registeredAddress}
			which defines how registered physical mail is
			addressed to the object.
			The syntax is identical to that of the
			\verb"postalAddress" attribute.

\item[streetAddress:]\index{attribute types, streetAddress}
			which is a string describing where the object
			physically resides
			(i.e., the street name, place, avenue, and building
			number); e.g.,
			object
\begin{quote}\small\begin{verbatim}
165 Jordan Road
\end{verbatim}\end{quote}
			This need have no relationship to the object's
			postal address.

\item[postOfficeBox:]\index{attribute types, postOfficeBox}
			which is a string describing the box at which the
			object will receive physical postal delivery; e.g.,
\begin{quote}\small\begin{verbatim}
1010
\end{verbatim}\end{quote}

\item[physicalDeliveryOfficeName:]\index{attribute types, physicalDeliveryOfficeName}
			which is a string describing the geographical location
			of the physical delivery office which services the
			postal address of this object; e.g.,
\begin{quote}\small\begin{verbatim}
Troy
\end{verbatim}\end{quote}

\item[stateOrProvinceName:]\index{attribute types, stateOrProvinceName}\index{attribute types, st (stateOrProvinceName)}
			which is a string describing the state in which
			the \verb"locality" is found; e.g.,
\begin{quote}\small\begin{verbatim}
NY
\end{verbatim}\end{quote}

\item[postalCode:]\index{attribute types, postalCode}
			which is a string containing the ZIP code; e.g.,
\begin{quote}\small\begin{verbatim}
12180
\end{verbatim}\end{quote}
or
\begin{quote}\small\begin{verbatim}
94043-2112
\end{verbatim}\end{quote}

\item[localityName:]\index{attribute types, localityName}\index{attribute types, l (localityName)}
			which is a string describing the geographical
			area containing the \verb"streetAddress"; e.g.,
\begin{quote}\small\begin{verbatim}
Troy, New York
\end{verbatim}\end{quote}
\end{describe}
The second optional group describes telecommunications addressing information
for the object.
\begin{describe}
\item[telephoneNumber:]\index{attribute types, telephoneNumber}
		which is a string describing the phone number of the object
		using the international notation; e.g.,
\begin{quote}\small\begin{verbatim}
+1 518-283-8860
\end{verbatim}\end{quote}
or
\begin{quote}\small\begin{verbatim}
+1 518-283-8860 x1234
\end{verbatim}\end{quote}
In general, the syntax is:
\begin{quote}\smaller\begin{verbatim}
+<<country code>> <<national number>> [x<<extension>>]
\end{verbatim}\end{quote}

\item[facsimileTelephoneNumber:]\index{attribute types, facsimileTelephoneNumber}
		which is a string describing the fax number of the object
		using the international notation; e.g.,
\begin{quote}\small\begin{verbatim}
+1 518-283-8904
\end{verbatim}\end{quote}

\item[telexNumber:]\index{attribute types, telexNumber}
		which is defines the TELEX address of the object in a
		three-field format:
\begin{quote}\small\begin{verbatim}
number $ country $ answerback
\end{verbatim}\end{quote}
		e.g.,
\begin{quote}\small\begin{verbatim}
650 103 7390 $ US $ MCI UW
\end{verbatim}\end{quote}
\end{describe}
The final optional group contains one miscellaneous attribute:
\begin{describe}
\item[description:]\index{attribute types, description}
			which is a simple textual description;
			e.g.,
\begin{quote}\small\begin{verbatim}
Not-for-profit organization providing network services
\end{verbatim}\end{quote}
\end{describe}
Figure~\ref{organization-edb} shows an entry for an \verb"organization" object.
\tagfigure[tp]{INT-2}{An Example of an Entry for the organization Object Class}%
	{organization-edb}

\subsubsection	{The organizationalUnit objectClass}\index{object classes, organizationalUnit}
Objects of this class represent a unit within an organization,
such as a division or department.
Each organization participating in the PSI White Pages Pilot decides
how many organizational units it wishes to maintain as subordinates to
its organization entry.

Table~\ref{organizationalUnit-attributes} summarizes the attributes types for
an \verb"organizationalUnit" object.
There is one mandatory attribute:
\begin{describe}
\item[organizationalUnitName:]\index{attribute types, organizationUnitName}\index{attribute types, ou (organizationUnitName)}
			which gives the name of the organizational unit.
			The value of this attribute is a string; e.g.,
\begin{quote}\small\begin{verbatim}
Research and Development
\end{verbatim}\end{quote}
\end{describe}
\tagtable[tp]{INT-4}{Attribute Types for the organizationalUnit Object Class}%
	{organizationalUnit-attributes}

There are several attributes that may be present.
They belong to the physical address, telecommunications address,
and miscellaneous information groups described earlier.

Figure~\ref{organizationalUnit-edb} shows an entry for an
\verb"organizationalUnit" object.
\tagfigure[tp]{INT-3}%
	{An Example of an Entry for the organizationalUnit Object Class}%
	{organizationalUnit-edb}

\subsubsection	{The locality objectClass}\index{object classes, locality}
Objects of this class represent a geographical area,
such as a state, county, or city.
The sponsors of the PSI White Pages Pilot maintain such an entry for
the State of New York.
This is used to contain objects corresponding to individuals who are
associated with organizations that are not participating in the pilot
project.

Table~\ref{locality-attributes} summarizes the attributes types for
a \verb"locality" object.
There is one mandatory attribute:
\begin{describe}
\item[localityName:]\index{attribute types, localityName}\index{attribute types, l (localityName)}
			which gives the name of the locality.
			The value of this attribute is a string; e.g.,
\begin{quote}\small\begin{verbatim}
NY
\end{verbatim}\end{quote}
or
\begin{quote}\small\begin{verbatim}
Rensselaer County
\end{verbatim}\end{quote}
\end{describe}
\tagtable[tp]{INT-5}{Attribute Types for the locality Object Class}%
	{locality-attributes}

There is only one attribute that may be present,
it  belongs to the miscellaneous information group described earlier.

Figure~\ref{sony-edb} shows the entry for the State of New York,
whilst Figure~\ref{locality-edb} shows an entry for a subordinate
\verb"locality" object.
\tagfigure[tp]{INT-4}{The Entry for The State of New York}{sony-edb}
\tagfigure[tp]{INT-5}{An Example of an Entry for the locality Object Class}%
		{locality-edb}

\subsubsection	{The pilotPerson objectClass}\index{object classes, pilotPerson}
Objects of this class represent a person participating in the pilot project.

Table~\ref{pilotPerson-attributes} summarizes the attributes types for
a \verb"pilotPerson" object.
There are two mandatory attributes:
\begin{describe}
\item[commonName:]\index{attribute types, commonName}\index{attribute types, cn (commonName)}
			which gives a (potentially ambiguous) name for
			the person.
			The value of this attribute is a string usually
			containing the person's first and last names; e.g.,
\begin{quote}\small\begin{verbatim}
Marshall Rose
\end{verbatim}\end{quote}
			This attribute is usually multi-valued, containing
			variations on the first, middle, and last names; e.g.,
\begin{quote}\small\begin{verbatim}
Colin Robbins
Colin John Robbins
Colin J. Robbins
\end{verbatim}\end{quote}
			For purposes of the pilot project, the distinguished
			value of the attribute usually contains only the
			person's first and last names.
			(See Section~\ref{naming:people} on 
			page~\pageref{naming:people} for a discussion of this.)

\item[surName:]\index{attribute types, surName}
			which gives the person's last name.
			The value of this attribute is a string; e.g.,
\begin{quote}\small\begin{verbatim}
Rose
\end{verbatim}\end{quote}
\end{describe}
\tagtable[tp]{INT-6}{Attribute Types for the pilotPerson Object Class}%
	{pilotPerson-attributes}

There are several attributes that may be present,
which are divided into five groups:
physical address, telecommunication information, 
computer environment,
miscellaneous information,
and home information.
The first two groups have been previously described.

The third group describes information relating to the person's
computer environment:
\begin{describe}
\item[rfc822Mailbox:]\index{attribute types, rfc822Mailbox}
			which is the user's computer mail address,
			e.g., 
\begin{quote}\small\begin{verbatim}
mrose@psi.com
\end{verbatim}\end{quote}

\item[otherMailbox:]\index{attribute types, otherMailbox}
			which is the user's computer mail address
			in various domains.
			The syntax of this attribute's value is special:
\begin{quote}\small\begin{verbatim}
<domain> $ <mailbox>
\end{verbatim}\end{quote}
			e.g., 
\begin{quote}\small\begin{verbatim}
internet $ mrose@psi.com
\end{verbatim}\end{quote}
The current list of mail domains are:
\[\begin{tabular}{l}
applelink\\
bitnet\\
compuserve\\
genie\\
internet\\
mcimail\\
nasamail\\
preferred\\
uucp
\end{tabular}\]

\item[userid:]\index{attribute types, userid}
			which is the user's login name; e.g.,
\begin{quote}\small\begin{verbatim}
mrose
\end{verbatim}\end{quote}

\item[userClass:]\index{attribute types, userClass}
			which describe's the user's classification; e.g.,
\begin{quote}\small\begin{verbatim}
staff
\end{verbatim}\end{quote}
\end{describe}
The next optional group contains a few miscellaneous attributes:
\begin{describe}
\item[description:]\index{attribute types, description}
			which is a simple textual description;
			e.g.,
\begin{quote}\small\begin{verbatim}
Principal Implementor of the ISO Development Environment
\end{verbatim}\end{quote}

\item[info:]\index{attribute types, info}
			which is additional information about the object;
			e.g.,
\begin{quote}\small\begin{verbatim}
It's nearly as good as BIND
\end{verbatim}\end{quote}

\item[businessCategory:]\index{attribute types, businessCategory}
			which describes the person's business,
			e.g.,
\begin{quote}\small\begin{verbatim}
networking
\end{verbatim}\end{quote}

\item[title:]\index{attribute types, title}
			which is the person's job title,
			e.g.,
\begin{quote}\small\begin{verbatim}
Senior Scientist
\end{verbatim}\end{quote}

\item[roomNumber:]\index{attribute types, roomNumber}
			which is a string describing where the object resides
			at the location, e.g.,
\begin{quote}\small\begin{verbatim}
Building T-30
\end{verbatim}\end{quote}

\item[userPassword:]\index{attribute types, userPassword}
			which is a string containing the object's
			password in the Directory.  This is used,
			for example, when the user wants to update
			the entry.
			The password is kept in the clear
			(the security ramifications of this are
			discussed later on); e.g.,
\begin{quote}\small\begin{verbatim}
secret
\end{verbatim}\end{quote}

\item[mobileTelephoneNumber:]\index{attribute types, mobileTelephoneNumber}\index{attribute types, mobile (mobileTelephoneNumber)}
			which is a string describing the user's mobile
			number (e.g., for a cellular phone).

\item[pagerTelephoneNumber:]\index{attribute types, pagerTelephoneNumber}\index{attribute types, pager (pagerTelephoneNumber)}
			which is a string describing the user's pager number.

\item[favouriteDrink:]\index{attribute types, favouriteDrink}\index{attribute types, drink (favouriteDrink)}
			which is a string describing the user's favorite drink.

\item[secretary:]\index{attribute types, secretary}
			which is the Distinguished Name of the user's
			secretary.

\item[seeAlso:]\index{attribute types, seeAlso}
			which is a Distinguished Name pointing to another
			entry related to the user (perhaps in a different
			role).

\item[photo:]\index{attribute types, photo}
			which is a facsimile bitmap of the user's face.
\end{describe}
The final optional group contains a few attributes about the person at home:
\begin{describe}
\item[homePostalAddress:]\index{attribute types, homePostalAddress}
			which describes how physical mail is addressed to the
			person's home.
			The syntax of this attribute's value is special:
			it consists of~1 to~6 fields, seperated by the
			``\verb"$"''-sign, each field being from~1 to~30
			characters long.

\item[homePhone:]\index{attribute types, homePhone}
		which is a string describing the phone number of the object
		using the international notation.
\end{describe}
Figure~\ref{pilotPerson-edb} shows an entry for a \verb"pilotPerson" object.
\tagfigure[tp]{INT-6}{An Example of an Entry for the pilotPerson Object Class}%
	{pilotPerson-edb}

\subsubsection	{The alias objectClass}\index{object classes, alias}
Objects of this class represent an alias to some other entry in the
PSI White Pages Pilot.
Each organization participating in the pilot must define at least two of
these:
\begin{itemize}
\item	a manager responsible for the organization's portion of the Directory
	tree;
	and,

\item	a postmaster responsible for the organization's electronic mail.
\end{itemize}

Table~\ref{alias-attributes} summarizes the attributes types for
an \verb"alias".
There are two mandatory attributes:
\begin{describe}
\item[commonName:]\index{attribute types, commonName}\index{attribute types, cn (commonName)}
			which gives the name of the alias.
			The value of this attribute is a string; e.g.,
\begin{quote}\small\begin{verbatim}
PostMaster
\end{verbatim}\end{quote}

\item[aliasedObjectName:]\index{attribute types, aliasedObjectName}
			which is a pointer to another object in the Directory;
			e.g.,
\begin{quote}\tiny\begin{verbatim}
c=US@o=Performance Systems International@ou=Operations@cn=Wengyik yeong
\end{verbatim}\end{quote}
\[\fbox{\begin{tabular}{lp{0.4\textwidth}}
\bf NOTE:&	The value of this attribute must not be another alias:
		each alias must point to a non-aliased entry!
\end{tabular}}\]
\end{describe}
There are no optional attributes for this object class.
Figure~\ref{alias-edb} shows an entry for an \verb"alias" object.
\tagtable[tp]{INT-7}{Attribute Types for the alias Object Class}%
	{alias-attributes}
\tagfigure[tp]{INT-7}{An Example of an Entry for the alias Object Class}%
	{alias-edb}

\subsubsection	{The organizationalRole objectClass}\index{object classes, organizationalRole}
Objects of this class represent a position filled by a person.

Table~\ref{organizationalRole-attributes} summarizes the attributes types for
a \verb"organizationalRole".
There is one mandatory attribute:
\begin{describe}
\item[commonName:]\index{attribute types, commonName}\index{attribute types, cn (commonName)}
			which gives the name of the role.
			The value of this attribute is a string; e.g.,
\begin{quote}\small\begin{verbatim}
Manager
\end{verbatim}\end{quote}
\end{describe}
There are several attributes that may be present,
which are divided into three groups:
physical address, telecommunication information, 
and miscellaneous information.
The first two groups have been previously described.
The final optional group contains two miscellaneous attributes:
\begin{describe}
\item[description:]\index{attribute types, description}
			which is a simple textual description;
			e.g.,
\begin{quote}\small\begin{verbatim}
Manager of the Troy Office
\end{verbatim}\end{quote}

\item[roleOccupant:]\index{attribute types, roleOccupant}
			which is a pointer to another object in the Directory,
			referring to the person who currently fills the role.
			e.g.,
\begin{quote}\tiny\begin{verbatim}
c=US@o=Performance Systems International@ou=Administration@cn=Kimberly Brown
\end{verbatim}\end{quote}
\end{describe}
Figure~\ref{organizationalRole-edb} shows an entry for a \verb"organizationalRole"
object.
\tagtable[tp]{INT-9}{Attribute Types for the organizationalRole Object Class}%
	{organizationalRole-attributes}
\tagfigure[tp]{INT-14}%
	{An Example of an Entry for the organizationalRole Object Class}%
	{organizationalRole-edb}

\subsubsection	{The quipuDSA objectClass}\index{object classes, quipuDSA}
Objects of this class represent a Directory System Agent (DSA) running 
the pilot project software.
Each organization participating in the PSI White Pages Pilot typically
runs at least one DSA.

Table~\ref{quipuDSA-attributes} summarizes the attributes types for
a \verb"quipuDSA".
There are several mandatory attributes:
\begin{describe}
\item[commonName:]\index{attribute types, commonName}\index{attribute types, cn (commonName)}
			which gives the name of the DSA,
			by convention this is the name of an endangered species
			in South America; e.g.,
\begin{quote}\small\begin{verbatim}
Fruit Bat
\end{verbatim}\end{quote}

\item[eDBinfo:]\index{attribute types, eDBinfo}
			which indicates how the DSA participates when
			propagating authoritative EDB information
			(this is the dual
			of the \verb"masterDSA" and \verb"slaveDSA"
			attribute types discussed earlier).
			The syntax of this attribute's value is special:
\begin{quote}\small\begin{verbatim}
<name> # <upstream> # <downstream>
\end{verbatim}\end{quote}
The \verb"<name>" field is a Distinguished Name referring to the object
containing the EDB.
By convention, if absent this refers to the ROOT (unnamed) EDB.
The \verb"<upstream>" field is the Distinguished Name of a DSA which is
considered to be ``closer'' to the MASTER of this EDB.
It may be the name of the actual DSA containing the MASTER EDB,
but it might also be the name of an intermediate DSA (containing a SLAVE copy).
If this field is empty,
then this DSA does not pull the EDB from another DSA
(this needn't mean that it holds the MASTER EDB,
though usually this is the case).
The \verb"<downstream>" field is the Distinguished Name of a DSA which is
allowed to copy the EDB from this DSA.
If this field is empty,
then this DSA does not allow other DSAs to copy the EDB.
If the \verb"<downstream>" field is to contain more than one DSA,
either enter multiple \verb"eDBinfo" lines or separate the Distinguished
Names of the downstream DSAs with the \verb"`$'"-sign.

A few examples:
\begin{quote}\small\begin{verbatim}
# cn=Giant Tortoise # cn=Fruit Bat
c=US # # cn=Fruit Bat
c=US # # c=US@cn=Spectacled Bear
c=US@l=NY # cn=Fruit Bat #
\end{verbatim}\end{quote}
Note that there is no harm to using multiple \verb"eDBinfo" lines,
even if they refer to the same EDB.
These lines indicate that:
\begin{itemize}
\item	The ROOT EDB is read from the \verb"cn=Giant Tortoise" DSA,
	further the \verb"cn=Fruit Bat" DSA is allowed to read the
	ROOT EDB from this DSA.

\item	The \verb"cn=Fruit Bat" DSA and the \verb"c=US@cn=Spectacled Bear" DSA
	are allowed to read the EDB for \verb"c=US". 
	Note that the second and third line could be combined as:
\begin{quote}\small\begin{verbatim}
c=US # # cn=Fruit Bat$c=US@cn=Spectacled Bear
\end{verbatim}\end{quote}
While more concise,
this format is not recommended as it is probably more difficult to read.

\item	The DSA named \verb"cn=Fruit Bat" supplies the EDB for \verb"c=US@l=NY"
	to this DSA.
\end{itemize}

\item[presentationAddress:]\index{attribute types, presentationAddress}
			which is the OSI presentation address that the DSA
			resides at.
			The syntax of this attribute's value is special:
\begin{quote}\small\begin{verbatim}
'0101'H/Internet=<quad>+<port>
\end{verbatim}\end{quote}
or
\begin{quote}\small\begin{verbatim}
'0101'H/Internet=<quad-1>+<port>|Internet=<quad-2>+<port>
\end{verbatim}\end{quote}
if the DSA is capable of listing on multiple IP addresses.%
\footnote{Actually, the rules for forming OSI presentation addresses are
{\em much\/} more complicated;
however, for the purposes of the pilot project,
this should suffice.}
The \verb"<quad>" field is a 32-bit IP address expressed in the
usual quad notation (e.g., \verb"130.117.128.2").
The \verb"<port>" field is a decimal TCP port which isn't likely to
be used on your system.
The value \verb"17003" is recommended for the purposes of the pilot project.

A simple example:
\begin{quote}\small\begin{verbatim}
'0101'H/Internet=130.117.128.2+17003
\end{verbatim}\end{quote}
Note that multiple values for the \verb"presentationAddress" attribute type
should not be used:
multiple IP addresses (up to~4) can be specified using a single attribute value.

\item[manager:]\index{attribute types, manager}
			which is a pointer to another object in the Directory
			of the person who manages this DSA;
			e.g.,
\begin{quote}\small\begin{verbatim}
c=US@o=Performance Systems International@cn=Manager
\end{verbatim}\end{quote}

\item[userPassword:]\index{attribute types, userPassword}
			which is a string containing the DSA's
			password in the Directory.  This is used
			when the manager of the DSA wants to
			change the entry for the DSA itself
			(as opposed to the entries held by the DSA).
			The password is kept in the clear
			(the security ramifications of this are
			discussed later on); e.g.,
\begin{quote}\small\begin{verbatim}
secret
\end{verbatim}\end{quote}

\item[quipuVersion:]\index{attribute types, quipuVersion}
			which is a simple string relating the version of
			the pilot project software being run by this DSA;
			e.g.,
\begin{quote}\small\begin{verbatim}
Version 5.0
\end{verbatim}\end{quote}
\end{describe}
There are two optional attributes:
\begin{describe}
\item[description:]\index{attribute types, description}
			which, by convention, is multi-valued.
			The distinguished value is the wildlife description
			of the \verb"commonName" attribute whilst the other
			values describe the role that the DSA plays in
			the pilot project;
			e.g.,
\begin{quote}\small\begin{verbatim}
The Mischevous Varmint
A MASTER DSA for l=NY under c=US
A SLAVE DSA for ROOT and c=US
\end{verbatim}\end{quote}

\item[supportedApplicationContext:]\index{attribute types, supportedApplicationContext}
			which defines the application ports offerred by this
process.
			The current values are:
\begin{quote}\small\begin{verbatim}
x500DSP & x500DAP & quipuDSP
\end{verbatim}\end{quote}
\end{describe}
Figure~\ref{quipuDSA-edb} shows an entry for a \verb"quipuDSA" object.
\tagtable[tp]{INT-8}{Attribute Types for the quipuDSA Object Class}%
	{quipuDSA-attributes}
\tagfigure[tp]{INT-8}{An Example of an Entry for the quipuDSA Object Class}%
	{quipuDSA-edb}

\section	{Functional Model}
The Functional Model describes the service in terms of entities.
As noted earlier,
two kinds of entities exist:
{\em Directory User Agents\/} (DUAs),
which operate on behalf of a user,
and
{\em Directory System Agents\/} (DSAs),
which provide the service.

The key aspect of the functional model is deciding how information
is transferred through the system.
When a DUA requests some action from a DSA
(e.g., to read an entry),
the DSA may not have that information resident.
In this case,
the DSA has a choice:
it may either contact another DSA which is ``closer'' to the information
and propagate the request (this is called {\em chaining\/}),
or it may return information about this ``closer'' DSA to the DUA,
and let the DUA re-issue its request
(this is called {\em referral\/}).
WHen DSAs communicate between themselves,
they may also chain or refer requests.

For the purposes of the pilot,
it is important to understand what the term ``resident'' means with respect
to information.
This residency requirement is:
\[\begin{tabular}{|l|l|}
\hline
\multicolumn{1}{|c|}{\bf Operation Requested}&
		\multicolumn{1}{|c|}{\bf EDB Required for Residency}\\
\hline
read, compare&	MASTER, SLAVE, or CACHE\\
list, search&	MASTER, or SLAVE\\
update&		MASTER\\
\hline
\end{tabular}\]
Note that after an update occurs,
the DSA actually performing the update (the one holding the MASTER EDB)
will write the in-core EDB structure to disk prior to returning the result
of the operation.
Of course,
once a MASTER DSA updates its copy of the EDB,
there is still the question of how DSAs with SLAVE copies receive this
new information.
The MASTER DSA does not explicitly inform them.
A polling model is used instead.


At present,
a SLAVE DSA will reload its copy of an EDB when it starts,
or when its \verb"manager" issues a management command directing it to do so.
Later on,
a script will be discussed which performs this task.
The frequency at which the script runs (or how often a DSA reboots)
determines how out of date a SLAVE copy of an EDB can become.

Prior to requesting a copy of an EDB,
a DSA checks to see if the timestamp has changed.
If so,
the EDB is updated.
As noted earlier,
the timestamp is the second line of an EDB file.
Whenever a modification is made to an entry in an EDB file,
the MASTER DSA for that EDB will update the timestamp accordingly.
Similarly,
whenever a text editor is used to modify an EDB file,
the user should update the timestamp.
The actual value is unimportant:
when a DSA examines the timestamp,
it looks only for a change;
it does not consider whether the EDB is newer or older.

Thus, while replication in the pilot is important to speed queries,
updates still rely on a centralized entity being available.
(This is the best compromise which can be taken without making the system
tremendously more complex.)

\section	{Organizational Model}
The Organizational Model describes the service in terms of the policy
defining interactions between entities and the information they hold.
That is,
this model describes how portions of the Directory tree map onto the DSAs.
This includes issues of replication and access control.
A {\em Directory Management Domain\/} (DMD) defines a portion of the
Directory Tree and how it is managed.

For the purposes of the pilot project,
DSAs are divided into three categories:
\begin{describe}
\item[Level-0:]	These DSAs are run by the sponsors of the PSI White
		Pages Pilot.  Their goal is to be highly-available
		authoritative servers.

\item[Level-1:]	Each organization participating in the pilot project runs
		a Level-1 DSA.%
\footnote{The pilot project sponsors may run the Level-1 DSA on behalf of an
organization so requested.
This is discouraged, but may be necessary for resource constrained sites.}
The Level-1 DSA for a participating organization holds authoritative
information about that organization object.

\item[Level-2:]	As the size of the organization's subtree requires,
an organization runs one or more Level-2 DSA to hold a portion of that subtree.
Typically,
a Level-2 DSA is run for each top-level organizational unit in the
participating organization.
\end{describe}
Figure~\ref{dmd-topology} shows the top-level topology of the pilot project DMD.
Each participating organization is represented by an \verb"organization"
object ($O_1$ through $O_n$) under the \verb"c=US" tree.
Underneath each \verb"c=US@o=O_i" object,
one or more objects of class \verb"organizationalUnit" are placed.
The structure beneath each \verb"c=US@o=O_i@ou=U_j" object
is decided by the participating organization.
The pilot project sponsors suggest that only \verb"pilotPerson" objects
be placed underneath the organizational units.
\tagfigure[tp]{INT-9}{Topology of the Pilot Project Directory Management Domain}%
	{dmd-topology}

Now that the general structure is outlined,
how does this portion of the Directory tree map onto DSAs?

\subsection	{Level-0 DSAs}
There are two Level-0 DSAs.
Both reside directly under the ROOT.

The first Level-0 DSA, \verb"cn=Alpaca",
contains a copy of the ROOT EDB which it obtains directly from the
DSA holding the MASTER copy of the ROOT EDB.
In addition,
\verb"cn=Alpaca" is a MASTER DSA for \verb"c=US".
It propagates this EDB to the second Level-0 DSA,
and to the Level-1 DSAs of all participating organizations
(e.g., each DSA holding the MASTER of the \verb"c=US@o=O_i" EDB).
In addition,
these Level-1 DSAs propagate a copy of their EDB for \verb"c=US@o=O_i"
to this Level-0 DSA.
Finally,
this DSA receives a copy of the EDB for \verb"c=US@l=NY".

Figure~\ref{alpaca-edb} shows the entry corresponding to this DSA.
A single participating organization is shown in this entry.
\tagfigure[tp]{INT-10}{The Entry for the Level-0 DSA, cn=Alpaca}{alpaca-edb}

The second Level-0 DSA, \verb"cn=Fruit Bat",
contains a copy of the ROOT and \verb"c=US" EDBs which it obtains from
\verb"cn=Alpaca".
It may propagate these EDBs to any Level-1 DSA
(in case \verb"cn=Alpaca" is unavailable for an extended period of time).
In addition,
these Level-1 DSAs
propagate a copy of their MASTER EDB to this Level-0 DSA.
Finally,
\verb"cn=Fruit Bat" is a MASTER DSA for \verb"c=US@l=NY",
which it propagates to \verb"cn=Alpaca".

Figure~\ref{fruitbat-edb} shows the entry corresponding to this DSA.
As with the \verb"cn=Alpaca" DSA,
a single participating organization is shown.
\tagfigure[tp]{INT-11}{The Entry for the Level-0 DSA, cn=Fruit Bat}%
	{fruitbat-edb}

Level-0 DSAs perform chaining when handling requests from other DSAs.

\subsection	{Level-1 DSAs}
Each participating organization runs a DSA which contains the MASTER EDB
for that organization.
This DSA resides directly under \verb"c=US".

This DSA receives a copy of the ROOT and \verb"c=US" EDBs from
one of the Level-0 DSAs (usually \verb"cn=Alpaca").
It also propagates its MASTER copy of the \verb"c=US@o=O_i" EDB to
both Level-0 DSAs.

In addition,
this DSA propagates its EDB and a copy of the ROOT and \verb"c=US" EDBs
to all Level-2 DSAs run by this organization.
Similarly,
each of these Level-2 DSAs propagate {\em all\/} of the EDBs they
MASTER (e.g., \verb"c=US@o=O_i@ou=U_j") to this Level-1 DSA.

Level-1 DSAs perform chaining when handling requests from other DSAs.

Figure~\ref{spectacledbear-edb} shows the entry corresponding to a Level-1 DSA,
which might be named \verb"c=US@cn=Spectacled Bear".
A single subordinate Level-2 DSA is shown in this entry.
\tagfigure[tp]{INT-12}{The Entry for a Level-1 DSA}{spectacledbear-edb}

\subsection	{Level-2 DSAs}
Each participating organization runs zero or more Level-2 DSAs as the size of
their organizational units require.
These DSAs reside directly under \verb"c=US@o=O_i".

\[\fbox{\begin{tabular}{lp{0.8\textwidth}}
\bf NOTE:&	At present, the pilot sponsors discourage the use of Level-2
		DSAs.

		Nevertheless,
		it is important to structure an organization so that it 
		contains organizational units.
		In this fashion,
		when a Level-1 DSA is too large to run on an available system,
		a Level-2 DSA can be used to reduce the memory requirements
		on the system running the Level-1 DSA.
\end{tabular}}\]

Level-2 DSAs receive a copy of the \verb"c=US@o=O_i" EDB from their
superior Level-1 DSA,
along with SLAVE copies of the ROOT and \verb"c=US" EDBs.
They also propagate all MASTER copies of the EDBs they hold to this Level-1
DSA. 

Level-1 DSAs perform referrals when handling requests from other DSAs.

Figure~\ref{toucan-edb} shows the entry corresponding to a Level-2 DSA,
which might be named \verb"c=US@o=Performance Systems International@cn=Toucan".
\tagfigure[tp]{INT-13}{The Entry for a Level-2 DSA}{toucan-edb}

\subsection	{DUAs}
In terms of the pilot project,
DUAs conceptually exist at the same level as Level-2 DSAs.
When a DUA associates with a DSA to access the Directory service,
chaining is preferred over referrals,
so that the associated DSA may cache the results to any query made by the DUA.
This will slightly enhance the performance for other DUAs which later
associate with this particular DSA.

The preference to be used when associating with a DSA is:
\begin{itemize}
\item	``closest'' Level-2 DSA; followed by

\item	Level-1 DSA; followed by

\item	\verb"cn=Fruit Bat"; followed by

\item	\verb"cn=Alpaca".
\end{itemize}

\section	{Security Model}
The Security Model describes the service in terms of authentication
and authorization.
For the purposes of the pilot project,
``simple'' authentication is performed (i.e., passwords).
The pilot project supports an anonymous DN, termed the NULL DN,
which may be used for public, read-only access.
Otherwise,
users are required to bind to the Distinguished Name which corresponds
to them.

Note that an important feature of the DMD topology presented in the
Organizational Model is that information on users
(and hence their passwords and any other sensitive information)
is kept only on Level-1 and Level-2 DSAs.
Since,
in general,
these DSAs are run by each participating organization,
each organization may safeguard the disk-resident EDB files accordingly.

Because the ROOT, \verb"c=US", and \verb"c=US@o=O_i" EDBs are widely
replicated,
entries contained therein should not rely on \verb"self" access control
for modification.
That is,
the entries contained in this EDBs should:
\begin{itemize}
\item	if of object class \verb"quipuDSA", have a \verb"manager"
	attribute which resolves to an entry which is not propagated outside
	of the associated organization; e.g.,
\begin{quote}\small\begin{verbatim}
manager= c=US@o=O_i@cn=Manager
\end{verbatim}\end{quote}
which is an object of classe \verb"alias" to
\begin{quote}\small\begin{verbatim}
c=US@o=O_i@ou=Development@cn=FirstName LastName
\end{verbatim}\end{quote}
	and,

\item	regardless of the object class,
	always use \verb"group" access to specify an entry which is not
	propagated outside of the associated organization; e.g.,
\begin{quote}\scriptsize\begin{verbatim}
acl= others # read # entry
acl= group # c=US@o=O_i@cn=Manager # write # entry
acl= others # read # default
acl= group # c=US@o=O_i@cn=Manager # write # default
acl= others # compare # attributes # accessControlList$userPassword
acl= group # c=US@o=O_i@cn=Manager # write # attributes # accessControlList$userPassword
\end{verbatim}\end{quote}
\end{itemize}
Note that the object associated with the \verb"manager" attribute of a DSA
is effectively the ``super user'' for all EDBs which that DSA MASTERs.
As such,
the \verb"userPassword" associated with the corresponding object should be 
protected to the greatest extent practical.

Since there is no hope of protecting the passwords for a DSA or any object
residing in the ROOT, \verb"c=US", and possibly \verb"c=US@o=O_i" EDBs,
this policy effectively revokes any privileges assocated with the passwords of
the entries contained in these EDBs.

\section	{Running your DMD}
The pilot sponsors prefer that you administer your own DMD by running the
pilot software on one of your own systems,
doing maintenance,
and so on.
You should now continue with Chapter~\ref{installation}.

However,
if your site is a member of PSInet,
then the pilot sponsors will administer the DMD for you,
if you lack the facilities to do so.
In this case,
you will be required to provide EDB files to the pilot sponsors.
Contact 
\begin{quote}\small\begin{verbatim}
wpp-manager@uu.psi.com
\end{verbatim}\end{quote}
to find out the full details.

For now, 
note that entries are formatted as described above,
but with a few read-only modifications:
\begin{itemize}
\item	The \verb"accessControlList" attribute is simply:
\begin{quote}\small\begin{verbatim}
acl=
\end{verbatim}\end{quote}

\item	The \verb"userPassword" attribute is not present.
\end{itemize}

