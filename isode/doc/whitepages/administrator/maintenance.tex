% run this through LaTeX with the appropriate wrapper

\chapter	{Maintenance}
Now that your Level-1 DSA has successfully joined the pilot project DMD,
you must maintain your portion of the Directory tree.

\section	{A file you should know about}
The file \file{quiputailor} file in the \file{quipu/wildlife/} directory
contains runtime configuration for your Level-1 DSA.
It was automatically created earlier.

Section~13.3 of \volfive/ discusses the options available for run-time
tailoring.
Typically,
you will not need to edit this file.

\section	{Nightly Maintenance}
One of the last tasks performed when you Level-1 DSA joined the pilot project
DMD was to direct your system to run a shell script, \file{nightly.sh}.
This script performs two tasks:
\begin{itemize}
\item	it mails your logs to the pilot project sponsors
	(this is only a temporary measure to aid our understanding of
	how the software is behaving);
	and,

\item	it cycles your logs.
\end{itemize}
You might wish to modify this script so that you are also informed of the
activities of your Level-1 DSA.
Actually, the logs record only crude information.
During the course of the pilot project,
the pilot software might be upgraded to provide more meaningful information.
This will be examined as experience is gained.

\subsection	{Logs}
The pilot software does a lot of logging.
There are two logs which are generated,
the first,
\file{dsap.log},
contains information on general DSA activity,
whilst the second,
\file{stats.log},
contains statistical information on the DSA.

\subsection	{Limiting the size of Logs}
If your Level-1 DSA is particularly busy,
it may generate large logs.
As such,
you might find it desirable to limit the maximum size that a log may grow to.
Since two logs are generated,
one for DSA activity and the other for statistics,
you will need to make two edits.

Look at the \file{quiputailor} file in the \file{quipu/wildlife/} directory.
There should be four lines similar to this:
\begin{quote}\small\begin{verbatim}
# minimal logging
dsaplog	level=exceptions dflags=tty file=dsap.log

# full statistics
stats	level=all dflags=tty file=stats.log
\end{verbatim}\end{quote}
Add the string \verb"size=100" at the end of the two lines,
e.g.,
\begin{quote}\small\begin{verbatim}
# minimal logging
dsaplog	level=exceptions dflags=tty file=dsap.log size=100

# full statistics
stats	level=all dflags=tty file=stats.log size=100
\end{verbatim}\end{quote}
This will limit the size of each log to 100~Kbytes.
If you wish other limits,
change the value \verb"100" accordingly.

\subsection	{Reading Logs}
This is currently a black art.
When the sponsors of the pilot project have mastered this,
\theguide/ will be updated accordingly.
In the meantime,
if you figure something out,
share it with the \verb"wpp-camayocs" list.
Have fun.

\section	{Adding Entries}
Now comes the fun part:
entering data into the Directory.
In general,
there are two kinds of activities:
small, incremental changes are best made using \man dish(1c).
However,
for the wholesale entry of massive amounts of data,
the easiest way is to run your favorite text editor and create EDB files
manually. 
(In a future release of the pilot project software,
more management tools will be available to automate this process somewhat.)

Note that if you edit the EDB files directly,
you {\bf must\/} tell your DSA to re-read these files after you are done
editing.
This is accomplished by either killing and restarting the DSA
or using the \switch"refresh" option to the \pgm{dish} command
\verb"dsacontrol".

To aid the process,
a number of templates for the objects you might add are found in the directory
\file{quipu/templates/}:
\begin{quote}\begin{tabular}{rl}
\file{alias}&	\verb"alias" object\\
\file{dsa}&	Level-2 DSA\\
\file{person}&	\verb"pilotPerson" object\\
\file{role}&	\verb"organizationalRole" object\\
\file{unit}&	\verb"organizationalUnit" object
\end{tabular}\end{quote}
Each of these files contains editing instructions.

In the \file{quipu/} source directory,
there is a program called \pgm{testedb},
which can be used to check EDB files for correctness:
\begin{quote}\small\begin{verbatim}
% cd quipu/
% ./make testedb
% ./testedb < EDB
\end{verbatim}\end{quote}
The \pgm{testedb} program will find the vast majority of errors in an EDB file.
It can not however,
find errors due to schema violation
(i.e., not conforming to the \verb"treeStructure" attribute of the EDB's
parent).
If your DSA fails to boot properly,
the log will indicate:
\begin{quote}\small\begin{verbatim}
DSA Halted
\end{verbatim}\end{quote}
In this case,
invoke the DSA interactively to determine the cause of the problem,
e.g.,
\begin{quote}\small\begin{verbatim}
% $(SBINDIR)ros.quipu -t ./quiputailor
Schema error in entry ending line 16...
*** Attribute error ***
<<DN of entry in error>>
Attribute type objectClass - Constrain violation
File ...wildlife/c=US/o=O_i/EDB not loaded
FATAL ERROR: DSA Halted
\end{verbatim}\end{quote}

\subsection	{Using Dish}
If you use the first approach,
then your Level-1 DSA will automatically update the database directory.
Thus,
all you need be able to do is run one of the user interfaces.

First, identify yourself to \pgm{dish} as the manager of the DSA holding the
entries you want to modify:
\begin{quote}\smaller\begin{verbatim}
% dish -c "wildlife name" -user "c=US@o=O_i@cn=Manager" -simple
Welcome to Dish (DIrectory SHell)
Enter password for "c=US@o=O_i@cn=Manager":
Dish ->
\end{verbatim}\end{quote}
You can now use the \verb"add" and \verb"modify" commands as appropriate.
For the \verb"add" command,
it is suggested you start with one of the supplied templates, e.g.,
\begin{quote}\small\begin{verbatim}
Dish -> add ou=Corporate -template $(ETCDIR)quipu/templates/unit
\end{verbatim}\end{quote}
will create a new organizational unit under the current node.

The only tricky part is when objects of class \verb"organizationalUnit" or
\verb"dsa" are added.

\section	{Adding organizationalUnits}
When an organizational unit is added,
you must also modify the entries for the DSAs holding MASTER or SLAVE copies
of the subordinates of the organizational unit.
\[\fbox{\begin{tabular}{lp{0.8\textwidth}}
\bf NOTE:&	At the present time, the pilot sponsors strongly recommend
		against adding Level-2 DSAs.

		A Level-2 DSA should be added only when a Level-1 DSA is too
		large to run on an available system.  In this case,
		a Level-2 DSA can be used to reduce the memory requirements
		on the system running the Level-1 DSA.
\end{tabular}}\]

If the organizational unit is to be mastered by your Level-1 DSA,
then the procedure is straight-forward:
First,
create a directory in your \file{wildlife/c=US/o=O\_i/} directory with
the name of the organizational unit, e.g.,
\begin{quote}\smaller\begin{verbatim}
wildlife/c=US/o=Performance Systems International/ou=Corporate
\end{verbatim}\end{quote}
Second,
create an \file{EDB} file in this \unix/ directory containing information on
the entries in that organizational unit.
Third,
create an entry for that organizational unit in the Directory,
e.g., by running \pgm{dish}, moving to your organization's entry,
typing:
\begin{quote}\small\begin{verbatim}
Dish -> add ou=Corporate -template $(ETCDIR)quipu/templates/unit
\end{verbatim}\end{quote}
and then following the editing instructions in the file.

\section	{Inherited Attributes}
In order to make your EDB files (and in-core Directory) smaller,
you can use attribute inheritance.

There are a lot of possibilities,
rather than list them all,
here are the smallest number of steps to get the biggest savings:
\begin{enumerate}
\item	Look at each \verb"quipuNonLeafObject" in your part of the DIT
(e.g., all the organizational units).

\item	For each of these,
add an attribute like this:
\begin{quote}\smaller\begin{verbatim}
iattr= $ default (
acl= self # write # entry
acl= others # read # entry
acl= self # write # default
acl= others # read # default
acl= self # write # attributes # userPassword
acl= others # compare # attributes # userPassword
streetAddress
postOfficeBox
physicalDeliveryOfficeName
stateOrProvinceName
postalCode
postalAddress
telephoneNumber
facsimileTelephoneNumber
objectClass= top & quipuObject & person & thornPerson & pilotPerson
)
\end{verbatim}\end{quote}
If the entry doesn't have some of these attributes listed above,
then don't include those attributes in the \verb"iattr" list.

\item	For each immediate child of these entries,
if the child has an objectClass value of \verb"pilotPerson",
remove {\bf all\/} the objectClass values.
Next,
if the child entry has a different value for any of the other attributes in
the \verb"iattr" list in the parent,
then that value should be present in the child,
otherwise remove it.  
\end{enumerate}

One interesting variation on this is to specify the inherited attributes only
once, rather than in each \verb"quipuNonLeafObject"
\begin{enumerate}
\item	In your organization's entry add an attribute like this:
\begin{quote}\smaller\begin{verbatim}
iattr= quipuNonLeafObject $ default (
iattr
)
iattr= $ default (
acl= self # write # entry
acl= others # read # entry
acl= self # write # default
acl= others # read # default
acl= self # write # attributes # userPassword
acl= others # compare # attributes # userPassword
streetAddress
postOfficeBox
physicalDeliveryOfficeName
stateOrProvinceName
postalCode
postalAddress
telephoneNumber
facsimileTelephoneNumber
objectClass= top & quipuObject & person & thornPerson & pilotPerson
)
\end{verbatim}\end{quote}
If the entry for your organization,
doesn't have some of these attributes listed above,
then don't include those attributes in the second \verb"iattr" list.

\item	Then, for any \verb"person" {\em anywhere\/} subordinate to your
organization,
if the child has an objectClass value of \verb"pilotPerson",
remove {\bf all\/} the objectClass values.
Next,
if the child entry has a different value for any of the other attributes in
the \verb"iattr" list in the parent,
then that value should be present in the child,
otherwise remove it.  
\end{enumerate}
This optimization assumes a fairly homogeneous relationship in your
organization, which is often the case.

\section	{Adding a Level-2 DSA}
There are three aspects to adding a Level-2 DSA:
first, the entry for your organization and Level-1 DSA must be modified,
and an entry for your Level-2 DSA must be created;
second,
the Level-2 DSA must be configured;
and,
third,
parts of the Directory tree mastered by the Level-1 DSA may be moved over to
be mastered by the Level-2 DSA.

First, you must choose a name for your new Level-2 DSA.
Since there will probably be more Level-2 DSAs then endangered species of
South American Wildlife,
you do not have to use a wildlife name for a Level-2 DSA.
Choose something associated with your organization or state.

For the purposes of the pilot project,
the name of each Level-2 DSA taks the form:
\begin{quote}\small\begin{verbatim}
c=US@o=O_i@cn=wildlife name
\end{verbatim}\end{quote}
As usual,
you will have to pick a ``sanitized'' name that will be used for the \unix/
directory which will contain the database for your Level-2 DSA.

To remain consistent with the discussion on configuring a Level-1 DSA,
we'll call the name of the DSA \verb"wildlife name" and the directory will be
called \file{wildlife/}.

\subsection	{Modifying the Level-1 DSA}
To the entry for your Level-1 DSA,
you will need to add these lines:
\begin{quote}\small\begin{verbatim}
eDBinfo= # # c=US@o=O_i@cn=wildlife name
eDBinfo= c=US # # c=US@o=O_i@cn=wildlife name
eDBinfo= c=US@o=O_i # # c=US@o=O_i@cn=wildlife name
\end{verbatim}\end{quote}
This says that your Level-1 DSA provide copies of the ROOT, \verb"c=US" and
your organization's EDBs to your Level-2 DSA.

This addition is done using the \verb"modify" command to \pgm{dish}:
\begin{quote}\small\begin{verbatim}
Dish -> modify "@c=US@cn=wildlife name"
\end{verbatim}\end{quote}

Note that even though your Level-2 DSA will contain a slave copy of the EDB
for your organization,
you do not add a \verb"slaveDSA" attribute to your organization's entry to
reflect this.
In order to contact your Level-2 DSA,
it is necessary to find its \verb"presentationAddress" attribute by asking the
directory.
Since the entry for your Level-2 DSA is kept beneath your organization's
entry, anyone asking for information about your Level-2 DSA would already have
information on your organization!

Finally,
you need to add an entry for your Level-2 DSA to the EDB for your organization:
\begin{quote}\small\begin{verbatim}
Dish -> add "cn=wildlife name" -template $(ETCDIR)quipu/templates/dsa
\end{verbatim}\end{quote}
The template file contains these editing instructions:
\begin{enumerate}
\item	Change each occurrence of \verb"O_i" to your organization's name; e.g.,
\begin{quote}\small\begin{verbatim}
Performance Systems International
\end{verbatim}\end{quote}

\item	Change each occurrence of \verb"wildlife name" to the common name of
	your DSA; e.g.,
\begin{quote}\small\begin{verbatim}
beeblebrox
\end{verbatim}\end{quote}

\item	For each organizational unit, \verb"U_j", this DSA will master,
add a line:
\begin{quote}\small\begin{verbatim}
eDBinfo = c=US@o=O_i@ou=U_j # # c=US@cn=level-1 DSA
\end{verbatim}\end{quote}
where \verb"c=US@cn=level-1 DSA" is the name of your Level-1 DSA.

\item	Change the value of the \verb"presentationAddress" attribute to
	contain the IP address of the host running the Level-2 DSA,
	and select an unused TCP port at this IP address
	(port~17010 is suggested for Level-2 DSAs);
	e.g.,
\begin{quote}\small\begin{verbatim}
'0101'H/Internet=130.117.118.3+17010
\end{verbatim}\end{quote}
If other Level-2 DSAs are to be run on this host,
it is suggested that ascending port numbers, starting at 17011, be assigned.
However,
running multiple Level-2 DSAs on a single host is not recommended.

\item	Change the value of the \verb"description" attribute for your
	Level-2 DSA accordingly.
	The first value should be the wildlife description.
	Note that you should fully explain the meaning of the Level-2 DSA's
	common name.
	Another description value should be added for each organizational
	unit mastered by this DSA, e.g.,
\begin{quote}\small\begin{verbatim}
Master DSA for U_j under O_i
\end{verbatim}\end{quote}
\end{enumerate}

\subsection	{Configuring a Level-2 DSA}
Configuring a Level-2 DSA is currently a pain as \pgm{dsaconfig} is not
currently used for this task.

Start by copying the database directory for your Level-2 DSA:
\begin{quote}\small\begin{verbatim}
# cd quipu/
# cp -r level-1-dsa wildlife
# chmod 700 wildlife
# find wildlife -exec chown daemon {} \;
# find wildlife -exec chgrp daemon {} \;
# su daemon
# cd wildlife/
\end{verbatim}\end{quote}

\subsubsection	{Editing the DSA tailoring file}
Now edit the \file{quiputailor} file in the \file{quipu/wildlife/} directory.
There are three things to do:
\begin{enumerate}
\item	Change the \verb"mydsaname" variable to reflect the Distinguished Name
	of the DSA.
	For example:
\begin{quote}\small\begin{verbatim}
mydsaname       "c=US@o=O_icn=wildlife name"
\end{verbatim}\end{quote}
	becomes
\begin{quote}\smaller\begin{verbatim}
mydsaname       "c=US@o=Performance Systems International@cn=beeblebrox"
\end{verbatim}\end{quote}

\item	Change the \verb"logdir" variable to reflect the \unix/ directory where
QUIPU log files are to reside.
	For example:
\begin{quote}\small\begin{verbatim}
logdir  $(ETCDIR)quipu/wildlife/
\end{verbatim}\end{quote}
	becomes
\begin{quote}\small\begin{verbatim}
logdir  $(ETCDIR)quipu/beeblebrox/
\end{verbatim}\end{quote}
(Note the trailing slash.)

\item	Change the \verb"treedir" variable to reflect the \unix/ directory
where the DSA's database resides.

	For example:
\begin{quote}\small\begin{verbatim}
treedir $(ETCDIR)quipu/wildlife/
\end{verbatim}\end{quote}
	becomes
\begin{quote}\small\begin{verbatim}
treedir $(ETCDIR)quipu/beeblebrox/
\end{verbatim}\end{quote}
(Note the trailing slash.)
\end{enumerate}

Section~13.3 of \volfive/ discusses the options available for run-time
tailoring.
You will have no need of editing this file.

\subsubsection	{Editing the DSA startup file}
Now edit the \file{startup.sh} file in the \file{quipu/wildlife/} directory.
There are two things to do:
\begin{enumerate}
\item	Change the \verb"W" variable to reflect the wildlife name of the DSA.
	For example:
\begin{quote}\small\begin{verbatim}
W=wildlife
\end{verbatim}\end{quote}
	becomes
\begin{quote}\small\begin{verbatim}
W="Beeblebrox"
\end{verbatim}\end{quote}

\item	Change the \verb"D" variable to reflect the \unix/ directory where
the DSA's database resides.
	For example:
\begin{quote}\small\begin{verbatim}
D=$(ETCDIR)quipu/wildlife
\end{verbatim}\end{quote}
	becomes
\begin{quote}\small\begin{verbatim}
D=$(ETCDIR)quipu/beeblebrox
\end{verbatim}\end{quote}
\end{enumerate}

\subsubsection	{Building an Initial Database}
The directory database you created with the \pgm{cp} command earlier has done
virtually all the work for you.
Now all you need do is edit each EDB file to initially mark each as a
SLAVE copy.
A simple way of doing this is:
\begin{quote}\small\begin{verbatim}
# find . -name EDB -a -exec vi {} \;
\end{verbatim}\end{quote}
which will run \pgm{vi} on each EDB file.
If the first line of this file says \verb"MASTER",
change it to \verb"SLAVE".
Otherwise the first line should say \verb"SLAVE"
(if the first line of the EDB file says \verb"CACHE",
then contact a \camayoc/ for assistance.)

\subsubsection	{Testing the Level-2 DSA}
At this point,
your Level-2 DSA should be configured and you should start and test it:
\begin{quote}\small\begin{verbatim}
# $(SBINDIR)ros.quipu -t ./quiputailor &
\end{verbatim}\end{quote}
If your DSA is configured properly,
it will print out something like:
\begin{quote}\small\begin{verbatim}
-- '0101'H/Internet=130.117.128.3+17010 --
DSA Started
\end{verbatim}\end{quote}
If your Level-2 DSA does not boot for some reason,
consult Section~\ref{dsa:failure} on page~\pageref{dsa:failure}.

You should now try connecting to the Level-2 DSA.
\begin{quote}\small\begin{verbatim}
% dish -c "wildlife name"
Welcome to Dish (DIrectory SHell)
Dish ->
\end{verbatim}\end{quote}
indicates that the DUA connected to your Level-1 DSA.
Otherwise consult Section~\ref{dua:failure} on
page~\pageref{dua:failure} and try to debug the problem.

\subsubsection	{Editing the DUA tailoring file}
Once your Level-2 DSA is operational,
you should edit the file \file{dsaptailor} in the ISODE \verb"ETCDIR"
directory so that your DUAs 
will know about this DSA.
This is done by adding this line
\begin{quote}\smaller\begin{verbatim}
dsa_address "wildlife name"      '0101'H/Internet=aaa.bbb.ccc.ddd+port
\end{verbatim}\end{quote}
{\em after\/} the \verb"dsa_address" line for your Level-1 DSA
and then making these edits:
\begin{enumerate}
\item	Substitute the common name of your DSA for \verb"wildlife name"; e.g.,
\begin{quote}\small\begin{verbatim}
Beeblebrox
\end{verbatim}\end{quote}

\item	Change the IP address and TCP port number to correspond to the
	OSI presentation address you defined earlier in the entry for the
	DSA;
	e.g.,
\begin{quote}\small\begin{verbatim}
'0101'H/Internet=130.117.118.3+17010
\end{verbatim}\end{quote}
\end{enumerate}

Now run the \man dish(1c) program again,
telling it to connect to your Level-1 DSA.
\begin{quote}\small\begin{verbatim}
% dish -c "wildlife name"
Welcome to Dish (DIrectory SHell)
Dish ->
\end{verbatim}\end{quote}
indicates that the DUA connected to your Level-1 DSA.
Otherwise consult Section~\ref{dua:failure} on
page~\pageref{dua:failure} and try to debug the problem.

Now look around the Directory tree using \pgm{dish}.
Descend to \verb"c=US@o=O_i" to check on your own entries.
A good test to run is to try and bind to your own entry,
but to do so by dereferencing the alias for the Manager of your DMD:
\begin{quote}\small\begin{verbatim}
Dish -> bind -user "c=US@o=O_i@cn=Manager"
Enter password for "c=US@o=O_i@cn=Manager": 
Dish -> 
\end{verbatim}\end{quote}
Indicates that you are now bound to the directory as that DN.
Instead, if you see:
\begin{quote}\small\begin{verbatim}
Dish -> bind -user "c=US@o=O_i@cn=Manager"
Enter password for "c=US@o=O_i@cn=Manager": 
Security Error - check name and password
\end{verbatim}\end{quote}
then either you may have entered the DN or password wrong.
Try again.
If not,
or if you encounter some other problem,
contact a \camayoc/ for assistance.

\subsection	{Moving portions of the Directory Tree}
For each organizational unit, \verb"ou=U_j", the Level-2 DSA will master,
you now need to do two things:
\begin{itemize}
\item	tell your Level-1 DSA that it no longer masters the EDB;
	and,

\item	tell your Level-2 DSA that it now masters the EDB.
\end{itemize}
The first step is done as follows.
Begin by using \pgm{dish} to bind to the Level-1 DSA:
\begin{quote}\smaller\begin{verbatim}
% dish -c "Level-1 DSA" -user "c=US@o=O_i@cn=Manager" -simple
\end{verbatim}\end{quote}
Next:
\begin{enumerate}
\item	Modify the entry for the organizational unit which is held by your
Level-1 DSA.
Change the \verb"masterDSA" attribute of the entry corresponding to that
unit from:
\begin{quote}\small\begin{verbatim}
masterDSA= c=US@cn=Level-1 DSA
\end{verbatim}\end{quote}
to:
\begin{quote}\small\begin{verbatim}
masterDSA= c=US@o=O_i@cn=wildlife name
\end{verbatim}\end{quote}
You will also need to add a \verb"slaveDSA" attribute to this entry:
\begin{quote}\small\begin{verbatim}
slaveDSA= c=US@cn=Level-1 DSA
\end{verbatim}\end{quote}

These additions are done using \pgm{dish}:
\begin{quote}\small\begin{verbatim}
Dish -> modify "@c=US@o=O_i@ou=U_j"
\end{verbatim}\end{quote}

\item	Modify the entry for your Level-1 DSA.
Change the line that says:
\begin{quote}\small\begin{verbatim}
eDBinfo= c=US@o=O_i@ou=U_j # #
\end{verbatim}\end{quote}
to:
\begin{quote}\small\begin{verbatim}
eDBinfo= c=US@o=O_i@ou=U_j # c=US@o=O_i@cn=wildlife name #
\end{verbatim}\end{quote}
This says that your Level-1 DSA will receive copies of the EDB for each
organizational unit held by your Level-2 DSA.

This change is done using the \verb"modify" command to \pgm{dish}:
\begin{quote}\small\begin{verbatim}
Dish -> modify "@c=US@cn=wildlife name"
\end{verbatim}\end{quote}

\item	Lock the Level-1 DSA's copy of the EDB file:
\begin{quote}\small\begin{verbatim}
Dish -> dsacontrol -lock "c=US@o=O_i@ou=U_j"
\end{verbatim}\end{quote}

\item	Edit the EDB file kept in the {\bf Level-1\/} DSA's directory
database by changing the first line from \verb"MASTER" to \verb"SLAVE".
(If the first line of this file does not say \verb"MASTER",
then you are editing the {\bf wrong\/} directory database.)

\item	Unlock the Level-1 DSA's copy of the EDB file:
\begin{quote}\small\begin{verbatim}
Dish -> dsacontrol -refresh "c=US@o=O_i@ou=U_j"
Dish -> dsacontrol -unlock "c=US@o=O_i@ou=U_j"
\end{verbatim}\end{quote}
\end{enumerate}

The second step is done as follows.
Begin by using \pgm{dish} to bind to the Level-2 DSA:
\begin{quote}\smaller\begin{verbatim}
% dish -c "Level-2 DSA" -user "c=US@o=O_i@cn=Manager" -simple
\end{verbatim}\end{quote}
Next:
\begin{enumerate}
\item	Modify the entry for the for organizational unit which is to be
mastered by your Level-2 DSA.
Change the \verb"masterDSA" attribute of the entry corresponding to that
unit from:
\begin{quote}\small\begin{verbatim}
masterDSA= c=US@cn=Level-1 DSA
\end{verbatim}\end{quote}
to:
\begin{quote}\small\begin{verbatim}
masterDSA= c=US@o=O_i@cn=wildlife name
\end{verbatim}\end{quote}
You will also need to add a \verb"slaveDSA" attribute to this entry:
\begin{quote}\small\begin{verbatim}
slaveDSA= c=US@cn=Level-1 DSA
\end{verbatim}\end{quote}

These additions are done using \pgm{dish}:
\begin{quote}\small\begin{verbatim}
Dish -> modify "@c=US@o=O_i@ou=U_j"
\end{verbatim}\end{quote}

\item	Modify the entry for your Level-2 DSA.
Add this line:
\begin{quote}\small\begin{verbatim}
eDBinfo= c=US@o=O_i@ou=U_j # # c=US@cn=Level-1 DSA
\end{verbatim}\end{quote}
This says that your Level-1 DSA will receive copies of the EDB for each
organizational unit held by your Level-2 DSA.

This change is done using the \verb"modify" command to \pgm{dish}:
\begin{quote}\small\begin{verbatim}
Dish -> modify "@c=US@o=O_i@cn=wildlife name"
\end{verbatim}\end{quote}

\item	Lock the Level-2 DSA's copy of the EDB file:
\begin{quote}\small\begin{verbatim}
Dish -> dsacontrol -lock "@c=US@o=O_i@ou=U_j"
\end{verbatim}\end{quote}

\item	Edit the EDB file kept in the {\bf Level-2\/} DSA's directory
database by changing the first line from \verb"SLAVE" to \verb"MASTER".
(If the first line of this file does not say \verb"SLAVE",
then you are editing the {\bf wrong\/} directory database.)

\item	Unlock the Level-2 DSA's copy of the EDB file:
\begin{quote}\small\begin{verbatim}
Dish -> dsacontrol -unlock "@c=US@o=O_i@ou=U_j"
\end{verbatim}\end{quote}
\end{enumerate}
You should now reboot your Level-2 DSA and then try to connect to it using
\pgm{dish}.
Once this is successfully,
you should reboot your Level-1 DSA and also use \pgm{dish} to connect to it.

When you restart the Level-2 DSA,
it will try to update its ROOT,
\verb"c=US",
and \verb"o=O_i" EDB files from your Level-1 DSA.
The first two should be the same as what your Level-2 is running,
so no update will take place.
The third will be different however,
so you should see a file \file{c=US/o=O\_i/EDB.bak} created.

When you restart the Level-1 DSA,
in addition to trying to reload its ROOT and \verb"c=US" EDB files from the
Level-0 DSAs,
it will try to reload the EDB file for each organizational unit mastered by
the Level-2 DSA.
Since these will be the same,
initially no update will take place.

\subsubsection	{Editing the DUA tailoring file}
You now edit the \file{dsaptailor} file one more time.
Move the \verb"dsa_address" line for your Level-2 DSA above the line for your
Level-1 DSA.
This will tell your DUAs to contact the Level-2 DSA by default,
rather than the Level-1 DSA.

\subsubsection	{System Administration}
Once everything checks out,
its time to restart the DSA in the background.
Use \pgm{dish} to abort the DSA and then run the \file{startup.sh} script:
\begin{quote}\small\begin{verbatim}
% $(ETCDIR)quipu/wildlife/startup.sh
\end{verbatim}\end{quote}
Take a look at the log files it creates and once you're satisfied
that it is operational,
use \pgm{dish} one last time before considering things up and running.

Finally,
it's time for the last bit of system administration:
\begin{enumerate}
\item	Add an entry to the file \file{/etc/rc.local}:
\begin{quote}\smaller\begin{verbatim}
if [ -d $(ETCDIR)quipu/wildlife ]; then
   $(ETCDIR)quipu/wildlife/startup.sh & \
                                (echo -n ' wildlife') > /dev/console
fi
\end{verbatim}\end{quote}
in the section where the network servers are started.
If your \file{rc.local} file starts \man tsapd(8c),
then place this entry after the one which starts \pgm{tsapd}.

\item	Edit the file \file{quipu/wildlife/nightly.sh},
by looking for these three lines
\begin{quote}\smaller\begin{verbatim}
W="dsa name from dsaptailor, e.g., Beeblebrox"
D="wildlife directory, e.g., $(ETCDIR)quipu/beeblebrox"
\end{verbatim}\end{quote}
and editing them appropriately.

\item	Based on the time that the \verb"c=US" manager gave you for your
Level-1 DSA, add one hour and modify the \file{crontab} file according; e.g.,
\begin{quote}\small\begin{verbatim}
0 5 * * * $(ETCDIR)quipu/wildlife/nightly.sh
\end{verbatim}\end{quote}
If the directory database for the Level-2 DSA is owned by a user-ID other
than \verb"root" (e.g., \verb"daemon"),
then instead the line should look something like this:
\begin{quote}\small\begin{verbatim}
0 5 * * * su daemon < $(ETCDIR)quipu/wildlife/nightly.sh
\end{verbatim}\end{quote}
\end{enumerate}

Congratulations!
Your Level-2 DSA has now joined the pilot DMD.

\section	{Miscellaneous Topics}
Here is information on a wide range of topics,
arranged in  no particular order.

\subsection	{Moving a Level-1 DSA}
For various reasons you might need to move your DSA from one host to another.
In OSI terminology,
you need to change the presentation address of the DSA.
The steps to do this are:
\begin{enumerate}
\item	Modify the \verb"presentationAddress" attribute of your DSA using
\pgm{dish}.
Since your DSA masters its own EDB entry,
you should now send a message to the \verb"c=US" manager which indicates the
new address of your DSA.
The manager will tell the master DSA for \verb"c=US" to contact your DSA
(at it's old address to get the new information).

\item	Now wait a couple of days for this new information to propagate.
This is important to avoid a transient service outage.

\item	Stop the DSA and move its hierarchy over to the new host.

\item	Start the DSA on the new host and verify that it is working okay.

\item	Edit your \file{dsaptailor} file to reflect the new addrss.

\item	Don't forget to edit \file{/etc/rc.local} and \file{/usr/lib/crontab}
on both systems.
\end{enumerate}

%%% \subsection	{Running a SLAVE Level-1 DSA}
%%% {\em to be supplied$\ldots$}

%%% register at c=US
%%% copy edb files
%%% change masters to slaves
%%% for each master add
%%%     eDBinfo= dn # master #		to slave DSA entry
%%%     eDBinfo= dn # # slave		to master entry
%%%     slaveDSA= slave			to dn entry

\subsection	{A Final Word on DSAs and Knowledge Information}
It is important to appreciate that a DSA may hold knowledge
(have local \file{EDB} files)
even though the DIT does not indicate this
(no corresponding \verb"slaveDSA" attribute for that portion of the tree).
This is a feature.

For example,
if you wish to speed access to certain parts of the tree for your users,
then to the entry of the DSA which MASTERs that information,
you add this attribute: 
\begin{quote}\small\begin{verbatim}
eDBinfo = interesting_EDB # # slave_DSA_name
\end{verbatim}\end{quote}
to the entry of the DSA which is to have a copy,
you add this attribute:
\begin{quote}\small\begin{verbatim}
eDBinfo = interesting_EDB # master_DSA_name #
\end{verbatim}\end{quote}
And you do {\bf not\/} add any \verb"slaveDSA" attribute to the entry
corresponding that EDB.

This configuration has the effect that any user contacting the slave DSA,
will find that information local,
but, since this DSA is not listed in the \verb"slaveDSA" attribute,
then other DSAs won't bother it asking for information.

\subsection	{Naming People}\label{naming:people}
As noted earlier,
entries in the Directory are uniquely named by their
Relative Distinguished Name (RDN).
In the pilot software,
the RDN is represented as the first line of each entry in the \file{EDB} file
for its immediate parent.
Thus,
within an \file{EDB} file,
all RDNs must be unique.

The simplest way of doing this is to use:
\begin{quote}\small\begin{verbatim}
cn=FirstName LastName
\end{verbatim}\end{quote}
as the RDN.
However,
in organizations with large numbers of people,
this may not be sufficient to be unique.
So, there are four alternative strategies:
\begin{enumerate}
\item	Use
\begin{quote}\small\begin{verbatim}
cn=FirstName LastName
\end{verbatim}\end{quote}
	whenever possible.
	However,
	whenever ambiguity occurs, use either
\begin{quote}\small\begin{verbatim}
cn=FirstName MiddleInitial LastName
\end{verbatim}\end{quote}
	or
\begin{quote}\small\begin{verbatim}
cn=FirstName MiddleName LastName
\end{verbatim}\end{quote}
	Note that,
	the shorter forms should also be included to aid in searching.
	Hence,
	the first three lines of an entry might look like:
\begin{quote}\small\begin{verbatim}
cn=FirstName MiddleName LastName
cn= FirstName MiddleInitial LastName
cn= FirstName LastName
\end{verbatim}\end{quote}
	The first value is used for the RDN,
	and the other two for searching.

\item	Always use
\begin{quote}\small\begin{verbatim}
cn=FirstName MiddleName LastName
\end{verbatim}\end{quote}
	as the RDN for all entries,
	and include the shorter forms whenever possible.

\item	Always generate a uniquely constructed string
\begin{quote}\small\begin{verbatim}
cn=FML1
\end{verbatim}\end{quote}
and include whatever real naming information is available for searching
purposes:
\begin{quote}\small\begin{verbatim}
cn= FirstName MiddleName LastName
cn= FirstName MiddleInitial LastName
cn= FirstName LastName
\end{verbatim}\end{quote}
The only caveat with this approach is that each time the EDB file is
generated, it is best not to change the RDNs for entries which previously
existed.

\item	Use a multi-valued RDN ({\em deus ex machina\/} formed by a
\verb"commonName" attribute and some other distinguishing attribute:
\begin{quote}\small\begin{verbatim}
cn=FirstName MiddleName LastName%userid=Lastname
\end{verbatim}\end{quote}
The \verb"`%'"-sign is used to concatenate attributes when forming an RDN.
Hence,
if the first line of an entry is
\begin{quote}\small\begin{verbatim}
cn=Marshall Rose%userid=mrose
\end{verbatim}\end{quote}
then the entry's RDN really has two parts:
the most significant part is a \verb"commonName" attribute,
and the next significant part is a \verb"userid" attribute.
Other good choices besides \verb"userid" are things like \verb"localityName".

Of course,
the \verb"`%'"-notation can be used only on the first line of an entry,
as it is used only for RDNs.
\end{enumerate}
In all cases,
regardless of the actual RDN chosen,
it is strongly recommended to include as many alternate forms as possible,
in order to aid searching.

\subsection	{Installing the Software on other hosts}
You may wish to install the interfaces to the Directory,
on other systems,
whilst running a DSA on a single host.
If the hardware/software configuration of the new hosts are the same as the
initial host,
then you can simply install the compiled binaries.
Usually this is done by mounting the source hierarchy over the network,
logging in to the new host,
and using:
\begin{quote}\small\begin{verbatim}
# ./make inst-all inst-quipu
# (cd others/quipu; ./make inst-pilot)
\end{verbatim}\end{quote}
Then,
you must copy over two files from the ISODE \verb"ETCDIR" directory on the
original host: \file{dsaptailor} and \file{fredrc}.

If, however, you wish to change the software configuration,
then after moutning the source hierarchy,
you must clean the existing binaries in the source hierarchy,
then select the new configuration files and go through the generation and
installation process:
\begin{quote}\small\begin{verbatim}
% ./make distribution
% ./make once-only all all-quipu
% (cd others/quipu; ./make pilot)
# ./make inst-all inst-quipu
# (cd others/quipu; ./make inst-pilot)
\end{verbatim}\end{quote}
Next, you must then copy over the \file{dsaptailor} and \file{fredrc} files
from the ISODE \verb"ETCDIR" directory on the original host.
