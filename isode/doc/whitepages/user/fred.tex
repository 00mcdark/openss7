% run this through LaTeX with the appropriate wrapper

\chapter	{The White Pages User Interface}
The \man fred(1c) program is the ``simple'' interface for the white pages
service.
This is actually a small program which talks to the ``complicated'' interface,
a program called \pgm{dish}.
The \pgm{fred} program should be able to handle most of the queries you wish
to make.
However,
it is purposefully limited in its power.
As experienced users will find,
\pgm{fred} is a great pair of crutches when you need help walking,
but if you want to sprint, then you need a good pair of sneakers,
called \pgm{dish},
instead.

\section	{Invoking Fred}
When your local \camayoc/ told you about the white pages pilot,
you should have been given three bits of information:
\begin{itemize}
\item	your handle (a Directory Distinguished Name);

\item	a password for the handle (so you can keep your own entry up to date);
	and,

\item	the location of the \pgm{fred} program.
\end{itemize}

\subsection	{Fred resides locally}
In most cases,
the \pgm{fred} program will reside on a machine which you have a login on.
This will be a \unix/ machine.

Normally,
the \unix/ directory containing \pgm{fred} is already in your search path,
e.g., \file{/usr/local/bin/}.
Type the command:
\begin{quote}\small\begin{verbatim}
% which fred
\end{verbatim}\end{quote}
If the \pgm{which} command says something like:
\begin{quote}\small\begin{verbatim}
/usr/local/bin/fred
\end{verbatim}\end{quote}
then \pgm{fred} is already in your search path:
simply type
\begin{quote}\small\begin{verbatim}
% fred
\end{verbatim}\end{quote}
and skip to Section~\ref{fred:commands} on page~\pageref{fred:commands}.

Otherwise,
if \pgm{which} says
\begin{quote}\small\begin{verbatim}
fred not found in path
\end{verbatim}\end{quote}
then take a look at the \file{.cshrc} or \file{.profile} file in your home
directory to see if this is so.
You should see a line such as:
\begin{quote}\small\begin{verbatim}
set path=(. /bin /usr/ucb /usr/bin /usr/local/bin)
\end{verbatim}\end{quote}
in your \file{.cshrc} file,
or some lines similar to
\begin{quote}\small\begin{verbatim}
PATH=.:/bin:/usr/ucb:/usr/bin:/usr/local/bin
export PATH
\end{verbatim}\end{quote}
in your \file{.profile} file.
Verify that the name of the \unix/ directory containing \pgm{fred} is listed.
If not,
edit the file to make it so.
Then,
tell your shell to re-read the appropriate file,
e.g.,
either
\begin{quote}\small\begin{verbatim}
% source ~/.cshrc
\end{verbatim}\end{quote}
from the {\em C\/} shell,
or
\begin{quote}\small\begin{verbatim}
% . $HOME/.profile
\end{verbatim}\end{quote}
from the Bourne shell.

As they say in the trade,
``you are now ready to rock and roll''.
Simply type
\begin{quote}\small\begin{verbatim}
% fred
\end{verbatim}\end{quote}
to name that tune.
Now skip to Section~\ref{fred:commands} on page~\pageref{fred:commands}.

\subsection	{Fred resides remotely}
If for some reason you do not have a login on the \unix/ machine where
\pgm{fred} lives,
there are three other ways to use the \pgm{fred} program.
These are described in order of descending preference.

\subsubsection	{via Guest account}
Your \camayoc/ might provide an anonymous login to a machine where
\pgm{fred} resides.
In this case,
you just use TELNET to the machine,
using the information provided by the \camayoc/
who will have already initialized the environment.
For example:
\begin{quote}\small\begin{verbatim}
% telnet wp.psi.net
login: fred

Welcome to the PSI White Pages Pilot Project

  accessing service, please wait...

fred> 
\end{verbatim}\end{quote}
and away you go.
Now skip to Section~\ref{fred:commands} on page~\pageref{fred:commands}.

If you are accessing the White Pages from an \XW/ display,
and if your TELNET connection is to a host running \XW/ client software,
then you might see something like this instead:
\begin{quote}\small\begin{verbatim}
% telnet wp.psi.net
login: fred

Welcome to the PSI White Pages Pilot Project

If you want X window access, please enter your DISPLAY name,
otherwise, if you do not wish to use X, enter "none"

DISPLAY (default=myhost.psi.com:0.0)=
\end{verbatim}\end{quote}
If you intend to use the \xwindows/,
then you should make sure that the machine you have a TELNET connection to is
allowed to access your \XW/ display.
This is usually accomplished by adding a line such as
\begin{quote}\small\begin{verbatim}
xhost +wp.psi.net
\end{verbatim}\end{quote}
to your \XW/ start-up file.

\subsubsection	{via WHOIS}
The white pages service might also be available via the WHOIS network port.
In this case,
you use the \pgm{whois} program found on your machine,
usually with a special option,
e.g.,
\begin{quote}\small\begin{verbatim}
% whois -h wp.psi.net "query"
\end{verbatim}\end{quote}
Your \camayoc/ will provide the details.
There may even be a command called \pgm{whitepages} which shortens type-in
somewhat,
e.g.,
\begin{quote}\small\begin{verbatim}
% whitepages "query"
\end{verbatim}\end{quote}
Note that a key disadvantage of this approach is that it provides read-only
access to the white pages service.
You can never update your own entry using this technique.

\subsubsection	{via mail}
Finally,
the white pages service might also be available via electronic mail.
In this case,
you use the \pgm{mail} program found on your machine,
and send a message to a special address,
e.g.,
\begin{quote}\small\begin{verbatim}
whitepages@wp.psi.net
\end{verbatim}\end{quote}
Your query is placed in either the \verb"Subject:" field or the body of the
message.
Your \camayoc/ will provide the details.

In addition to providing read-only access to the white pages service,
this approach has the added disadvantage of not being interactive.

\section	{Giving commands to Fred}\label{fred:commands}
After invoking \pgm{fred},
you are prompted with ``\verb"fred> "'' indicating that \pgm{fred} is ready.
(Actually,
this is only true if \pgm{fred} is invoked from a \unix/ machine.
If invoked via WHOIS or mail,
then no prompts are given.)

If \pgm{fred} is invoked interactively,
it will look for a file in your home directory called \file{.fredrc}.
It will execute the commands contained in this file just as if you had typed
them directly to \pgm{fred}.
Following this,
you are given the ``\verb"fred> "'' prompt.

\section	{Let your fingers do the walking}
Although \pgm{fred} has several commands,
the most interesting command is \verb"whois",
which performs a white pages query.

If the value of the \verb"namesearch" variable is \verb"friendly",
then Kille's user-friendly name notation is used.
(Setting variables is discussed a little later on in
Section~\ref{setting:variables} on page~\pageref{setting:variables}.)
Kille's notation is ordered but untyped,
with components separated by commas.
Typical names include:
\begin{quote}\small\begin{verbatim}
rose, psi
kille, cs, ucl, gb
L. Eagle, "Sue, Grabbit and Runn", GB
\end{verbatim}\end{quote}
Note that you don't have to know all the components,
just list what you know, left-to-right,
starting with the person's name.
The user-friendly searching algorithm will usually figure out what you mean.

This is the preferred syntax as it is the most intuitive.
So,
you simply issue the \verb"whois" command followed by a name
and \pgm{fred} will do the rest,
e.g.,
\begin{quote}\small\begin{verbatim}
fred> whois schoffstall, psi
\end{verbatim}\end{quote}
In fact,
if you are only a casual user,
this is all you need to know.
You can stop reading now!

Otherwise,
there is an older searching procedure,
that is rather mechanical.
Let's begin with some simple examples and introduce the other commands along
the way.
If you already know the handle of the person you're interested in finding out
about,
just give the handle:
\begin{quote}\smaller\begin{verbatim}
fred> whois @c=US@cn=Manager
Manager (1)

Name:          Manager, US
\end{verbatim}\end{quote}

\subsection	{The Alias Command}
Since handles are long strings,
\pgm{fred} will automatically maintain a list of aliases of the entries you
have seen in the current session.
The alias is always a number.
When an entry is displayed,
it appears on the first line in parenthesis after the name of the object.
In the example above,
the alias is \verb"1".

To find out what aliases are currently defined,
use the \verb"alias" command:
\begin{quote}\smaller\begin{verbatim}
fred> alias
1    @c=US@cn=Manager
\end{verbatim}\end{quote}
Thus,
the previous \verb"whois" command could have been shortened to simply:
\begin{quote}\small\begin{verbatim}
fred> whois !1
    ...
\end{verbatim}\end{quote}
Each time you invoke \pgm{fred},
its list of aliases is empty.
If there are few handles which you use often,
you might wish to define them in your \file{.fredrc} file,
e.g.,
\begin{quote}\small\begin{verbatim}
alias "@c=US@o=Performance Systems International@cn=Manager"
\end{verbatim}\end{quote}
Of course,
the ordering of aliases is important.
\pgm{fred} will start numbering from~1 starting with the first \verb"alias"
command.

\subsection	{Back to Searching}
Suppose however,
that you don't know the handle for the person.
In this case,
you need to specify some search parameters.
Logically,
the first step is to ascertain the organization which the person is likely to
be associated with, e.g.,
``Performance Systems International''.
This is done as:
\begin{quote}\small\begin{verbatim}
fred> whois organization psi
Performance Systems International (1)    +1 800-836-0400 (Operations)
     aka: PSI

PSI Inc.
  Reston International Center
  11800 Sunrise Valley Drive
  Suite 1100
  Reston, VA 22091
  US

PSI Inc.
  5201 Great American Parkway
  Suite 3106
  Santa Clara, CA 95054
  US

PSI Inc.
  165 Jordan Road
  Troy, NY 12180
  US

Telephone: +1 800-836-0400 (Operations)
           +1 800-82PSI82 (Sales)
           +1 703-620-6651 (Corporate/Reston Office)
           +1 518-283-8860 (Troy Office)
           +1 408-562-6222 (Santa Clara Office)
FAX:       +1 703-620-4586
           +1 518-283-8904
           +1 408-562-6223

value-added provider of networking services

Locality:    Reston, Virginia

Name:     Performance Systems International, US (1)
Modified: Mon Jul 30 05:18:24 1990
      by: Manager, Performance Systems International,
            US (2)
\end{verbatim}\end{quote}
Second,
to search for a particular person,
you might use:
\begin{quote}\small\begin{verbatim}
fred> whois rose -area 2
Marshall Rose (111)                                 mrose@psi.com
     aka: mtr
     aka: Marshall T. Rose

Principal Scientist
PSI, Inc.
  POB 391776
  Mountain View, CA 94039
  US

PSI, Inc.
  5201 Great American Parkway
  Suite 3106
  Santa Clara, CA 95054
  US

Telephone: +1 415-961-3380
           +1 408-562-6222 x6221
FAX:       +1 415-961-3282
           +1 408-562-6223

Mailbox information:
  internet: mrose@psi.com
  internet: mrose@cheetah.ca.psi.com

Principal Implementor of the ISO Development Environment

Beleaguered Manager of the PSI/NYSERNet White Pages Pilot Project

A savvyNerd according to noSauce

Locality:    Santa Clara, California

Drinks:       Iced Tea (and lots of it...)
Picture:      /usr/etc/g3fax/Xphoto invoked

Name:     Marshall Rose, Mountain View,
            Research and Development,
            Performance Systems International,
            US (111)
Modified: Mon Sep 24 14:43:36 1990
      by: Manager, US (4)
\end{verbatim}\end{quote}
Note the use of the alias \verb"2".
The command could also have been:
\begin{quote}\small\begin{verbatim}
fred> whois rose -area "@c=US@o=Performance Systems International"
    ...
\end{verbatim}\end{quote}
Double-quotes are used so that the DN appears as a single token to \pgm{fred}.

Of course,
this two-step process,
whilst logical, is tedious.
Thus, you can combine them like this:
\begin{quote}\small\begin{verbatim}
fred> whois rose -org psi
    ...
\end{verbatim}\end{quote}
which says to look for any organizations with ``psi'' in its name.
Then, for each of these,
look for something called ``rose''.

\subsection	{The Area Command}
Suppose
you want information on several persons belonging to an organization.
You can use the \verb"area" command,
by itself,
to tell \pgm{fred} where to search for subsequent commands.
For example,
\begin{quote}\small\begin{verbatim}
fred> area "@c=US@o=Performance Systems International"
\end{verbatim}\end{quote}
or simply
\begin{quote}\small\begin{verbatim}
fred> area 2
\end{verbatim}\end{quote}
both tell \pgm{fred} the default area used by the \verb"whois" command.
Of course,
you can still use the \switch"area" area with the \verb"whois" command to
override the default area.
Thus,
\begin{quote}\small\begin{verbatim}
fred> whois alan -area "@c=US@o=Columbia University"
\end{verbatim}\end{quote}
will do what you expect.

If you use the \verb"area" command without any arguments,
\pgm{fred} will tell you what its default area is:
\begin{quote}\small\begin{verbatim}
fred> area
@c=US@o=Yoyodyne
\end{verbatim}\end{quote}
This indicates the default area for all commands,
{\em including\/} any subsequent \verb"area" commands.
Thus,
issuing:
\begin{quote}\small\begin{verbatim}
fred> area @c=US@o=Yoyodyne
@c=US@o=Yoyodyne

fred> area ou=Research
@c=US@o=Yoyodyne@ou=Research
\end{verbatim}\end{quote}
is equivalent to
\begin{quote}\small\begin{verbatim}
fred> area @c=US@o=Yoyodyne@ou=Research
@c=US@o=Yoyodyne@ou=Research
\end{verbatim}\end{quote}
because a leading \verb"`@'"-sign was not used before \verb"ou=Research".

As you might expect,
there is a special string ``\verb".."'' which may be used to move up one level:
\begin{quote}\small\begin{verbatim}
fred> area ..
@c=US@o=Yoyodyne
\end{verbatim}\end{quote}
Combinations are possible as well,
such as:
\begin{quote}\small\begin{verbatim}
fred> area ..@"o=Performance Systems International"
@c=US@o=Performance Systems International
\end{verbatim}\end{quote}
which moves up a level and then down to
\verb"o=Performance Systems International".

\subsection	{Getting Help}
For a brief summary of \pgm{fred} commands,
type:
\begin{quote}\small\begin{verbatim}
fred> help ?
\end{verbatim}\end{quote}
This will list the commands that \pgm{fred} knows about
along with a one-line summary of their function.

For help on a particular command,
type the name of the command followed by \switch"help",
e.g.,
\begin{quote}\small\begin{verbatim}
fred> alias -help
\end{verbatim}\end{quote}

If you need more help,
try
\begin{quote}\small\begin{verbatim}
fred> manual
\end{verbatim}\end{quote}
which is the same as
\begin{quote}\small\begin{verbatim}
% man fred
\end{verbatim}\end{quote}
from the shell.
If that's not enough,
contact your local \camayoc/.

\subsection	{Reporting problems}
To report something to your local white pages manager,
simply use the \verb"report" command.
You will be prompted for some text,
which \pgm{fred} will send to the appropriate mailbox.

\subsection	{Quitting}
To terminate \pgm{fred},
simply use:
\begin{quote}\small\begin{verbatim}
fred> quit
\end{verbatim}\end{quote}

\subsection	{Setting Variables}\label{setting:variables}
\pgm{fred} contains a few variables that may be manipulated to modify its
behavior: 
\begin{describe}
\item[debug:]	debug \pgm{fred}

\item[manager:]	mail-address of local white pages manager

\item[namesearch:] type of named use for matching,
		either \verb"fullname", \verb"surname", or \verb"friendly"

\item[pager:]	program to use for terminal pagination

\item[phone:]	display phone numbers in one-liner

\item[query:]	confirm two-step operations

\item[server:]	IP-address of directory assistance server

\item[soundex:]	use soundex for matching when no wildcards are present

\item[timelimit:]	maximum number of seconds to spend searching

\item[verbose:]	verbose interaction

\item[watch:]	watch dialogue with \pgm{dish}
\end{describe}
To view or change settings, use the \verb"set" command:
\begin{quote}\small\begin{verbatim}
fred> set query
\end{verbatim}\end{quote}
will display the current value of the \verb"query" variable,
whilst
\begin{quote}\small\begin{verbatim}
fred> set query on
\end{verbatim}\end{quote}
will change the variable accordingly.

\subsection	{via WHOIS or mail}
If you are accessing the white pages via WHOIS or mail,
then only the \verb"whois", \verb"area", \verb"help", and \verb"manual"
commands are available.
If your command to \pgm{fred} does not start with one of these keywords,
then it is assumed that your input is a list of arguments to the
\verb"whois" command.

Note that it is not possible to use aliases with the \verb"whois" command
since your \pgm{fred} session is not interactive.

\section	{More on Searching}
The full syntax to the \verb"whois" command is:
\begin{quote}\small\begin{verbatim}
whois input-field [record-type] [area-designator] [output-control]
\end{verbatim}\end{quote}
Only the {\em input-field\/} component need be present.
The components may appear in any order.

\subsection	{Input Field}
This component tells the white pages who or what to look for.
\begin{describe}
\item[name NAME {\rm or} .NAME]
\item[surname NAME]
\item[fullname NAME]
gives the name of the target.

Searching for names follows these rules:
\begin{enumerate}
\item	If the ``\verb"*"''-sign appears at the beginning and/or end of the
name, 
then wildcard-style matching is used:
the ``\verb"*"''-sign matches zero or more characters at the beginning
or end of a name.

\item	Otherwise,
if soundex has been enabled (set the variable \verb"soundex" to \verb"on"),
then imprecise matching occurs according to a Soundex algorithm.

\item	Otherwise,
if searching is to occur for a person's surname
(either the keyword \verb"surname" is used
or the value of the variable \verb"namesearch" is \verb"surname"),
then a case-insensitive match is used.%
\footnote{For compatibility with the older WHOIS service,
if \verb"surname" searching is used,
then a case-insensitive match is also performed on the local-part of the
entry's electronic-mail address}.

\item	Finally,
as a last resort,
\pgm{fred} will force a rather liberal wildcard-style match
(e.g., ``\verb"NAME"'' is treated as ``\verb"*NAME*"'').
\end{enumerate}
For compatibility with the WHOIS service,
an input field of ``\verb"NAME."'' is equivalent to ``\verb"NAME*"''
(i.e., a partial match for names having the given prefix).
Similarly,
an input field of ``\verb"*NAME"'' is equivalent to ``\verb"NAME expand"''.
Thus, to have wildcard matching at the beginning of the name,
use two ``\verb"*"''-signs,
e.g., ``\verb"**inc"'' matches names ending in ``\verb"inc"''.
(A terrible hack, but that's the price one pays to be consistent with the
WHOIS service.)

\item[handle HANDLE {\rm or} !HANDLE]
gives the Distinguished Name (DN) of the target.
Instead of a DN,
an alias for the DN may be used.

\item[mailbox STRING] gives the mailbox of the target.
(The mailbox search is more properly a Yellow Pages function.
It is included solely for the backwards compatibility with the WHOIS service.
As such,
use of mailbox matching is not recommended.)
\end{describe}
If a keyword is not given,
then \pgm{fred} attempts to intuit which kind of input field is being provided.
In most cases,
\pgm{fred} will treat the input field as a name,
unless it contains the \verb"`@'"-sign,
which makes it either a handle or a mailbox.

\subsection	{Record Type}
The {\em record type\/} is a single keyword,
one of:
\begin{describe}
\item[person:]		a person

\item[organization:]	an organization

\item[unit:]		a division under an organization

\item[role:]		a role within an organization

\item[locality:]	a geographic locality

\item[dsa:]		a white pages server
\end{describe}
If one of these keywords is not present,
then \pgm{fred} will not know what kind of entry you are looking for and
this can result in inefficient searches.
If you know what kind of entry you're looking for,
be kind and tell \pgm{fred}.

If you are searching for a person with a particular title,
then rather than using \verb"person",
use the \switch"title" switch.
For example,
\begin{quote}\small\begin{verbatim}
whois rose -title scientist
\end{verbatim}\end{quote}
looks for someone named \verb"rose" who is a scientist,
whilst
\begin{quote}\small\begin{verbatim}
whois -title operator
\end{verbatim}\end{quote}
looks for anyone who is an \verb"operator".
Naturally,
these searches are carried out relative to the appropriate area.

\subsubsection	{More on the Area Command}
If the \verb"area" command is given two arguments then this defines the
default area for a particular record type.
For example,
\begin{quote}\small\begin{verbatim}
fred> area unit @c=US@o=Yoyodyne
\end{verbatim}\end{quote}
indicates that searches for organizational units should,
by default,
occur under the indicated area.

It is now time to confess to a little white lie told earlier.
If the \verb"area" command is invoked without arguments,
it doesn't just print one line of information,
it prints several:
\begin{quote}\small\begin{verbatim}
fred> area
                     default area @c=US@o=Yoyodyne
area for record-type organization @c=US
area for record-type         unit @c=US@o=Yoyodyne
area for record-type     locality @c=US@l=NY
area for record-type       person @c=US@o=Yoyodyne
area for record-type          dsa @c=US
area for record-type         role @c=US@o=Yoyodyne
\end{verbatim}\end{quote}
The first line tells what the default area is.
The remaining lines tell what the default area is for particular kinds of
searches.

These values are normally set by your local \camayoc/ in a system-wide file
read by \pgm{fred} during initialization.
The name of this file is probably \file{/usr/etc/fredrc},
check with your system administrator.
Since this file is read before your own \file{.fredrc} file,
you can automatically override these settings if you choose.

So,
what use does \pgm{fred} make of the record type you supply?
At the moment,
\pgm{fred} employs some rather simple heuristics.
If a default area is declared for a \verb"organization",
\verb"unit", \verb"locality", or \verb"dsa",
then searches for objects of this type will look {\em only\/} one-level deep.
This is an artifact of the rules used by the sponsors of the pilot project
when they defined the overall structure of the Directory tree.
In practical terms,
this severely limits the search to the areas where it is possible for these
objects to exist.
This means that searching is faster since parts of the tree,
believed to never contain entries of this type,
are never searched.

Note however that in order to take advantage of this heuristic,
not only must a default area be declared for the record type you are searching
for,
but you must also tell the \verb"whois" command what the record type is.
While this might seem obvious:
\begin{quote}\small\begin{verbatim}
fred> whois organization psi
\end{verbatim}\end{quote}
will,
given the default areas defined above,
invoke the search heuristic and result in a ``lightning fast'' search.
However,
\begin{quote}\small\begin{verbatim}
fred> whois psi
\end{verbatim}\end{quote}
will result in a slow search since \pgm{fred} has no way of knowing that
\verb"psi" isn't someone's name unless {\em you\/} tell it.

\subsection	{Area Designator}
The {\em area designator\/} takes one of two forms.
The most common form is one of the switches:
\begin{quote}\begin{tabular}{l}
\switch"org" (short for \switch"organization"),\\
\switch"unit", or,\\
\switch"locality"
\end{tabular}\end{quote}
followed by a name.
For example,
\begin{quote}\small\begin{verbatim}
fred> whois rose -org psi
\end{verbatim}\end{quote}
which was introduced earlier,
says to look for any organization with ``psi'' in its name.
Then, for each of these,
look for something called ``rose''.

In the second form,
the area designator consists of the switch \switch"area" followed by a location
in the white pages,
either a Distinguished Name or an alias for a DN.
If an area designator is not given,
then the area for the search is chosen as follows:
\begin{enumerate}
\item	If a {\em record type\/} was given for the search
	(always a good idea),
	and if a default area was declared for that record type,
	then the appropriate area is used.

	This also allows \pgm{fred} to use the special searching
	heuristic based on the way the Directory tree is structured
	under the pilot project.

\item	Otherwise,
	the default area is used.
\end{enumerate}

\subsection	{Output Control}
Any combination of the following keywords may appear in the
{\em output control\/} component:
\begin{describe}
\item[expand:]		give a detailed listing and show children of matched
			entries.

\item[full:]		give a detailed listing, even on ambiguous matches
			(the default is to give a detailed listing only
			if a single match is found.)

\item[summary:]		give a one-line listing, even on unique matches
			(the default is to give a one-line listing only
			if a multiple matches are found.)

\item[subdisplay:]	give a one-line listing and show children of matched
			entries (the default is to not show any children.)
\end{describe}

\section	{Editing your Entry}
Finally,
you should keep your own entry as accurate and current as possible.
This is your responsibility in order to add value to the white pages service.
	
The \verb"edit" command is used to examine and modify your entry in the white
pages.

\subsection	{Who are You?}
Of course,
the first step is to tell \pgm{fred} who you are so that it can find your
entry.
Recall that your local \camayoc/ indicated your handle and password earlier.
You use the \verb"thisis" command to tell the whitepages who you are,
e.g.,
\begin{quote}\tiny\begin{verbatim}
fred> thisis "@c=US@o=Performance Systems International@cn=Manager"
Enter password for "@c=US@o=Performance Systems International@cn=Manager": secret
\end{verbatim}\end{quote}
You can place this command in your \file{.fredrc} file to automatically tell
\pgm{fred} about your entry in the white pages.
Note however that if you do so,
then \pgm{fred} will also prompt you for a password whenever it starts.
To get around this you might choose to place the password in your
\file{.fredrc} file.
e.g.,
\begin{quote}\small\begin{verbatim}
thisis "@c=US@o=Performance Systems International@cn=Manager" secret
\end{verbatim}\end{quote}
However,
you must now protect your \file{.fredrc} accordingly.
Entering:
\begin{quote}\small\begin{verbatim}
% chmod 0600 $HOME/.fredrc
\end{verbatim}\end{quote}
will do the trick.

\subsection	{Editing the Entry}
Once the \pgm{fred} knows about your entry in the white pages,
to edit your entry, use the \verb"edit" command:
\begin{quote}\small\begin{verbatim}
fred> edit
\end{verbatim}\end{quote}
This will invoke the editor defined by your \verb"$EDITOR" shell variable on a
template file.
(If you do not have the shell variable \verb"$EDITOR" set,
then you will be prompted for the name of an editor to use.)
Edit the template file accordingly and then exit the editor.
You will be asked if you want to update your entry with the new template.
If so,
the white pages will be informed of the changes.

It is a good idea to take a look at your entry after editing it.
To simplify type-in,
there is a special option to the \verb"whois" command which directs it to
display the entry for you:
\begin{quote}\small\begin{verbatim}
fred> whois !me
\end{verbatim}
\end{quote}
which isn't great grammar, but gets the job done.

Appendix~\ref{person:attributes} on page~\pageref{person:attributes} lists the
syntax and semantics of the attributes that you may have in your entry.
