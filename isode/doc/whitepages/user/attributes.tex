% run this through LaTeX with the appropriate wrapper

\chapter	{Attributes for Persons in the Pilot Project}%
		\label{person:attributes}
Here is the list of the attributes which may be present for a person
represented in the PSI White Pages Pilot Project.

There are two mandatory attributes:
\begin{describe}
\item[commonName:]
			which gives a (potentially ambiguous) name for
			the person.
			The value of this attribute is a string usually
			containing the person's first and last names; e.g.,
\begin{quote}\small\begin{verbatim}
Marshall Rose
\end{verbatim}\end{quote}
			This attribute is usually multi-valued, containing
			variations on the first, middle, and last names; e.g.,
\begin{quote}\small\begin{verbatim}
Colin Robbins
Colin John Robbins
Colin J. Robbins
\end{verbatim}\end{quote}
			For purposes of the pilot project, the distinguished
			value of the attribute should contain only the
			person's first and last names.

\item[surName:]
			which gives the person's last name.
			The value of this attribute is a string; e.g.,
\begin{quote}\small\begin{verbatim}
Rose
\end{verbatim}\end{quote}
\end{describe}
There are several attributes that may be present,
which are divided into five groups:
physical address, telecommunication information, 
computer environment,
miscellaneous information,
and home information.

The first group describes the physical address of the object:
\begin{describe}
\item[postalAddress:]
			which describes how physical mail is addressed to the
			object.
			The syntax of this attribute's value is special:
			it consists of~1 to~6 fields, seperated by the
			``\verb"$"''-sign, each field being from~1 to~30
			characters long; e.g.
\begin{quote}\tiny\begin{verbatim}
Performance Systems International $ 11800 Sunrise Valley Drive $ Reston, VA  22091
\end{verbatim}\end{quote}

\item[registeredAddress:]
			which defines how registered physical mail is
			addressed to the object.
			The syntax is identical to that of the
			\verb"postalAddress" attribute.

\item[roomNumber:]
			which is a string describing where the object resides
			at the location, e.g.,
\begin{quote}\small\begin{verbatim}
Building T-30
\end{verbatim}\end{quote}

\item[streetAddress:]
			which is a string describing where the object
			physically resides
			(i.e., the street name, place, avenue, and building
			number); e.g.,
			object
\begin{quote}\small\begin{verbatim}
165 Jordan Road
\end{verbatim}\end{quote}
			This need have no relationship to the object's
			postal address.

\item[postOfficeBox:]
			which is a string describing the box at which the
			object will receive physical postal delivery; e.g.,
\begin{quote}\small\begin{verbatim}
1010
\end{verbatim}\end{quote}

\item[physicalDeliveryOfficeName:]
			which is a string describing the geographical location
			of the physical delivery office which services the
			postal address of this object; e.g.,
\begin{quote}\small\begin{verbatim}
Troy
\end{verbatim}\end{quote}

\item[stateOrProvinceName:]
			which is a string describing the state in which
			the \verb"locality" is found; e.g.,
\begin{quote}\small\begin{verbatim}
New York
\end{verbatim}\end{quote}

\item[postalCode:]
			which is a string containing the ZIP code; e.g.,
\begin{quote}\small\begin{verbatim}
12180
\end{verbatim}\end{quote}
or
\begin{quote}\small\begin{verbatim}
94043-2112
\end{verbatim}\end{quote}

\item[localityName:]
			which is a string describing the geographical
			area containing the \verb"streetAddress"; e.g.,
\begin{quote}\small\begin{verbatim}
Troy, New York
\end{verbatim}\end{quote}
\end{describe}
The second optional group describes telecommunications addressing information
for the object.
\begin{describe}
\item[telephoneNumber:]
		which is a string describing the phone number of the object
		using the international notation; e.g.,
\begin{quote}\small\begin{verbatim}
+1 518-283-8860
\end{verbatim}\end{quote}

\item[facsimileTelephoneNumber:]
		which is a string describing the fax number of the object
		using the international notation; e.g.,
\begin{quote}\small\begin{verbatim}
+1 518-283-8904
\end{verbatim}\end{quote}

\item[telexNumber:]
		which is defines the TELEX address of the object in a
		three-part string:
\begin{quote}\small\begin{verbatim}
number $ country $ answerback
\end{verbatim}\end{quote}
		e.g.,
\begin{quote}\small\begin{verbatim}
650 103 7390 $ US $ MCI UW
\end{verbatim}\end{quote}
\end{describe}
The third group describes information relating to the person's
computer environment:
\begin{describe}
\item[rfc822Mailbox:]
			which is the user's computer mail address,
			e.g., 
\begin{quote}\small\begin{verbatim}
mrose@psi.com
\end{verbatim}\end{quote}

\item[otherMailbox:]
			which is the user's computer mail address
			in various domains.
			The syntax of this attribute's value is special:
\begin{quote}\small\begin{verbatim}
<domain> $ <mailbox>
\end{verbatim}\end{quote}
			e.g., 
\begin{quote}\small\begin{verbatim}
internet $ mrose@psi.com
\end{verbatim}\end{quote}
			or
\begin{quote}\small\begin{verbatim}
uucp $ uupsi!mrose
\end{verbatim}\end{quote}

\item[userid:]
			which is the user's login name; e.g.,
\begin{quote}\small\begin{verbatim}
mrose
\end{verbatim}\end{quote}

\item[userClass:]
			which describe's the user's classification; e.g.,
\begin{quote}\small\begin{verbatim}
staff
\end{verbatim}\end{quote}
\end{describe}
The next optional group contains a few miscellaneous attributes:
\begin{describe}
\item[description:]
			which is a simple textual description;
			e.g.,
\begin{quote}\small\begin{verbatim}
Principal Implementor of the ISO Development Environment
\end{verbatim}\end{quote}

\item[info:]
			which is additional information about the object;
			e.g.,
\begin{quote}\small\begin{verbatim}
It's nearly as good as BIND
\end{verbatim}\end{quote}

\item[businessCategory:]
			which describes the person's business,
			e.g.,
\begin{quote}\small\begin{verbatim}
networking
\end{verbatim}\end{quote}

\item[title:]
			which is the person's job title,
			e.g.,
\begin{quote}\small\begin{verbatim}
Senior Scientist
\end{verbatim}\end{quote}

\item[userPassword:]
			which is a string containing the object's
			password in the Directory.  This is used,
			for example, when the user wants to update
			the entry.
			The password is kept in the clear; e.g.,
\begin{quote}\small\begin{verbatim}
secret
\end{verbatim}\end{quote}

\item[mobileTelephoneNumber:]
			which is a string describing the user's mobile
			number (e.g., for a cellular phone).

\item[pagerTelephoneNumber:]
			which is a string describing the user's pager number.

\item[favouriteDrink:]
			which is a string describing the user's favorite drink.

\item[secretary:]
			which is the Distinguished Name of the user's
			administrative support.

\item[seeAlso:]
			which is a Distinguished Name pointing to another
			entry related to the user (perhaps in a different
			role).

\item[photo:]
			which is a facsimile bitmap of the user's face.
\end{describe}
The final optional group contains a few attributes about the person at home:
\begin{describe}
\item[homePostalAddress:]
			which describes how physical mail is addressed to the
			person's home.
			The syntax of this attribute's value is special:
			it consists of~1 to~6 fields, seperated by the
			``\verb"$"''-sign, each field being from~1 to~30
			characters long.

\item[homePhone:]
		which is a string describing the phone number of the object
		using the international notation.
\end{describe}
