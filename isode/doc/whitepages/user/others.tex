% run this through LaTeX with the appropriate wrapper

\chapter	{Other Programs of Interest}
In addition to the \pgm{fred} program,
there are some other programs which may be available,
depending on whether your system administrator has installed them.

\section	{Faces}
When \pgm{fred} and \pgm{dish} display the entry for someone,
they check to see if there is a photograph associated with the user.
This is stored in facsimile format in the \verb"photo" attribute for the entry.
If a photograph is present,
then a file called \file{/usr/etc/dsaptailor} is consulted for directives
indicating how the picture should be displayed
(check with your system administrator for the exact pathname of this file).

If you are running the X windows system,
then your administrator may have installed a program called \pgm{Xphoto}.
If your terminal type is \verb"xterm",
then this program will be automatically invoked whenever \pgm{fred} or
\pgm{dish} wish to display someone's photo.

If you want your picture to be kept in the OSI directory,
ask your system administrator if this is possible.

\subsection	{xwho}
If your system is running the \man rwhod(8c) daemon,
and you are running the X windows system,
then you might want to run the \pgm{xwho} program to see who is logged in on
your local network.
Consult the \man xwho(1c) manual page for more details.

\subsection	{xface}
If you use the \MH/ system to read your mail,
then you might want to run the \pgm{xface} program in the background.
Whenever \MH/ display a message for you,
it will ask \pgm{xface} to display the picture of the person who sent the
message.

\section	{Mail Composition}
If you use the \MH/ system to send your mail,
then you can use the White Pages automatically to lookup the electronic mail
addresses of the people you are sending messages to.

Rather than specifying an address,
you can specify a name by bracketing a White Pages query between
``\verb"<<"'' and ``\verb">>"'' using the user-friendly naming syntax,
 e.g.,
\begin{quote}\small\begin{verbatim}
To: << rose, psi, us >>
\end{verbatim}\end{quote}
At the \whatnow/ prompt,
you can say \verb"whom" to have the names expanded into addresses.
Alternately, the \verb"send" option can be used as well.
For each query appearing between ``\verb"<<"'' and ``\verb">>"'',
the DA-server will be asked to perform a White Pages resolution.
All matches are printed and you are asked to select one.
If one is not selected,
you return to \whatnow/ level.

Note that expansion can occur only if \verb"whom" or \verb"send" is invoked
interactively. 
If the \verb"push" option is used instead,
then the expansion will fail because MH will be unable to query 
you to select/confirm the right entry to use for the substitution.

Finally,
to enable this feature,
you must add a line
\begin{quote}\small\begin{verbatim}
da-server: hostname
\end{verbatim}\end{quote}
to your \profile/,
where \verb"hostname" is the domain-name or IP-address of the machine running
\pgm{dad}.
Your system administrator can tell you what \verb"hostname" to use.
