% run this through LaTeX

\input lcustom

\documentstyle[11pt,sfwmac,titlepage]{article}

\advance\textwidth by0.75in
\advance\oddsidemargin by-0.375in
\advance\evensidemargin by-0.375in

\pagestyle{reprint}
\def\reprintext{{\scriptsize\sf
	Reprinrted from
	{\em Computer Networks and ISDN Systems}, 12(3) {\oldstyle 1986}}}
\let\reprintstring=\empty

\begin{document}

\title{OSI Transport Services\\ on top of the TCP}
\author{Marshall T.~Rose\and Dwight E.~Cass}
\date{\vskip-1.25em
	Northrop Research and Technology Center\\
	One Research Park\\
	Palos Verdes Peninsula, CA  90274\\
	USA\vskip1.5em
	\ifdraft \versiondate/\\ \tt Draft Version \versiontag/\else \today\fi}
\maketitle

\begin{abstract}
The ARPA Internet community has a well-developed,
mature set of transport and internetwork protocols,
which are quite successful in offering network and transport services to
end-users.
The protocols provide ``open system interconnection'' (OSI) for users,
and a robust competitive market is supplied by a growing number of vendors.
The CCITT and ISO have defined various network, transport, session, presentation,
and application recommendations and standards which have been adopted by the
international community and numerous vendors.
To the largest extent possible,
it is desirable to offer the higher level services (i.e., session and above)
in networks based on the ARPA protocols,
without disrupting existing facilities.
It is also desirable to develop expertise with ISO and CCITT protocols while
utilizing a robust set of transport services.
Finally,
given the probable, long-term dominance of the international standards,
it is desirable to design a graceful transition strategy from networks based
on the ARPA protocols to OSI-based networks.

In this paper,
we suggest an exercise in interoperability which achieves all three of these
goals.
Our solution is practical in the sense that it:
can be easily implemented in a cost-effective manner in the short-term;
permits application-designers to work with an OSI applications-framework,
while utilizing existing robust transport and network protocols;
and,
provides the basis for a far-reaching migration strategy to the OSI protocols.
\end{abstract}

\newpage\thispagestyle{empty}
\tableofcontents
\footnotetext[0]{\hskip -\parindent
This document (version \versiontag/)
was \LaTeX set \today\ with \fmtname\ v\fmtversion.
This document is also referenced as NRTC Technical Paper \#700.
This work was supported in part by Northrop's Independent Research and
Development program.}

\makeatletter
\let\s@ction=\section
\def\section{\let\section=\s@ction \banner\section}
\makeatother

% run this through LaTeX with the appropriate wrapper

\section	{Introduction and Motivation}
To promote the use of open systems environments in the office,
technical, and manufacturing environments,
two influential user groups were formed,
the Manufacturing Automation Protocol group (MAP),
sponsored by General Motors Corporation,
and, the Technical and Office Protocols group (TOP),
sponsored by Boeing Computer Services.
In order to act more effectively,
the Society of Manufacturing Engineers (SME)
consolidated the two user groups into MAP/TOP under SME sponsorship.
MAP/TOP is championing the use of open systems interconnection (OSI),
primarily in the United States.
The widespread adoption of open standards is not a new idea,
for example,
the Department of the Defense has,
since the mid-{\oldstyle 70\/}'s been mandating the use of what is now called
the Defense Data Network (DDN) protocol suite for its computer-communication
applications.

MAP/TOP has received support from both the public and private sectors,
particularly from users who have suffered for years with myriads of
computer-based systems which simply do not talk to each other
(even if all those systems are purchased from a single vendor).
This has influenced the Department of Defense to clearly state,
through a number of recent programs and proposals,
that successful contractors must be able to communicate electronically,
not only internally, but also with the customer (\dod/),
preferably using the OSI protocols.

This leaves most defense contractors and many others in the following
situation:
With the use of open standards, many benefits can be achieved.
For example,
the use of open standards makes it possible to solve the so-called
{\em islands of automation\/} problem as described above.
However,
although the work currently being done in the MAP/TOP community promises a
major breakthrough in electronic communications,
MAP/TOP is generally not expected to achieve widespread use for at least
five years.%
\footnote{This problem can not be understated:
for example,
in the international committees,
there is no consensus to the routing issues at the network layer.
Some insiders suggest that three years of work will be required before
the non-trivial routing problems are resolved (at the committee level).}
Contractors planning to do business with the Department of Defense cannot
afford the luxury of waiting five years before starting to solve their
communications problems.
Nor can they simply ``scrap'' everything and start from ``scratch'' when
MAP/TOP has materialized.
Hence, we find ourselves in a dilemma.

\subsection	{Understanding the Problems}
Thus far,
we have isolated two problems:
first, we want to be running MAP/TOP applications now;
and,
second, if we are using other computer-communication technologies now,
it would be advantageous to migrate from our current approaches to the
MAP/TOP approach.

A key point in understanding the thrust of MAP/TOP,
is that MAP/TOP bases itself on the OSI protocol suite.
In many senses,
MAP/TOP represents an effort to ``prune the OSI tree'' in order to provide
vendors with a subset of the protocols which can be implemented in a timely
manner.
For example,
MAP/TOP currently limits itself to local area networking.
Further,
as new needs arise,
MAP/TOP influences the OSI community towards the development of protocols to
meet those needs.
Hence,
for the remainder of this paper,
we will use the term ``OSI protocol suite'',
understanding that it includes MAP/TOP,
without any loss in generality.

While the last several years of work in OSI have resulted in a
consensus regarding the transport layer and below,
application protocols have not enjoyed this degree of attention
(with perhaps the notable exception of MHS\cite{MHS}).
Further,
experience in the field repeatedly demonstrates that the development of
standard application protocols takes as long, or {\em much longer},
than protocols at other layers.
Finally,
considering that there is still a great deal of basic research to be
done in a number of application protocols areas,
such as distributed information bases and distributed robotic control,
it becomes apparent that the problem of application protocol development
is significant.

\subsection	{Finding the Solutions}
To solve these problems,
we begin by considering two observations.
First,
many of these problems have been solved in the past by others.
There are several other proprietary and open protocol suites,
which address many of these problems.
Indeed,
the last 15~years of the literature describe several successful attempts
at application (and other) protocols.
Second,
we note that MAP/TOP is based on the OSI protocol suite,
which in turn is based on a layered architecture.
The key here is that layered architectures allow us to
``mix and match'' modules with similar requirements at a given layer as
necessary to achieve an optimal solution.
In all cases,
the layered approach emphasizes the {\em services\/} provided at a layer,
not the {\em implementation\/} of those services.

Let us now continue with the problem at hand:
how can we achieve a communications strategy consistent with the
direction of MAP/TOP,
{\em and\/} allow us to communicate electronically with the Department of
Defense?
In this paper,
we suggest using the Defense Data Network (DDN) protocol suite,
commonly referred to as TCP/IP,
as the solution of choice.

During the course of this paper,
it will become clear that the approach we suggest is not limited to
defense contractors only.
Quite the contrary,
we present several arguments which suggest that the
immediate adoption of the DDN protocol suite as a migration path
towards MAP/TOP and OSI is an optimal strategy.
We base these arguments in part on the maturity and stability of the DDN
protocols;
on the multi-vendor support that TCP/IP enjoys;
and on
the widespread use of TCP/IP as an interconnection method among various
proprietary solutions.

In using the notion of layering to solve the problems,
we observe that there are two fundamental approaches which can be used:
\begin{itemize}
\item	A {\em protocol translation\/} approach, as suggested by
	Groenbaek\cite{TCP.convert.ISO} and others;
	and,

\item	an {\em interface translation\/} approach, which we suggest in this
	paper.
\end{itemize}
We argue that our approach is superior since it accomplishes the same
end-results as the protocol translation approach,
but is much simpler and less expensive to implement.
Further,
our approach has two important advantages:
it allows us to develop OSI standard applications directly on hosts using the
DDN protocols,
and any such applications will be able to ``run'' in either environment.

\subsection	{What's in a Name?}
Before proceeding,
the authors feel it important to clarify some terminology used in this paper.
Throughout this work,
we refer to both ``DDN'' and ``OSI'' as different protocol suites.
Although this is currently the dominant interpretation,
there is an alternate perspective, which the authors and others subscribe to.

This alternate view considers OSI as a technique for describing a
com\-put\-er-com\-mun\-i\-cat\-ions architecture in terms of abstractions
(e.g., layering, entities, services, protocols, access points, and so on).
As such, it follows that there are varying interpretations of OSI,
each with a reference model described in terms of these abstractions.
For example,
the Defense Data Network (DDN) protocol suite can be thought of as one
interpretation of OSI;
similarly,
the protocols developed by the International Standards Organization (ISO),
can be thought of as an alternate interpretation of OSI.
That is,
it is sensible for one to discuss
the difference between the DDN and ISO protocol suites,
even though the two suites are based on different reference models.
The commonality the two protocol suites share is derived from the abstraction
technique used to construct their respective models.

In order to maximize the readability of this work,
we have used the consensus interpretation throughout the paper.
It is the opinion of the authors however that the alternate perspective is
more powerful.

% run this through LaTeX with the appropriate wrapper

\section	{Background}
Having suggested that the use of TCP/IP is beneficial in the short- and
medium-term for the development of OSI-based application protocols,
and further,
that a strategy utilizing the DDN protocols towards these ends can also prove
to be a valuable tool during the transition from TCP/IP-based networks to
OSI-based networks,
it behooves us to present a terse definition of these terms.

\subsection	{The DDN Protocol Suite}
The \dod/ Transmission Control Protocol (TCP)\cite{TCP} provides
end-to-end transport services for \dod/ military standard networking
applications.
The TCP presents a full-duplex connection service to two end-peers,
while addressing the issues of reliability when using a non-reliable datagram
service.
The TCP implements a three-way handshake for connection establishment,
uses a windowing scheme for flow-control purposes,
and has a graceful termination phase.
To access the underlying network,
the TCP utilizes the services of the \dod/ Internet Protocol (IP)\cite{IP}.

The IP is a datagram, or connectionless, service and includes provision for
service specification, fragmentation/reassembly, and security information.
The IP makes no assumption regarding the reliability of the communications
subnet, and uses a simple checksum mechanism.
This requires the higher layer protocols (e.g., the TCP) to handle end-to-end
reliability.
This degree of freedom permits the IP an unprecedented amount of flexibility
in dealing with different subnetwork technologies,
and in connecting those subnetworks efficiently.

The DDN protocol suite is based on the ARPAnet Reference Model (ARM).
Interested readers should consult \cite{Internet.Architecture,ARM} for a
perspective on this model.
For our purposes,
it is instructive to note that there is an underlying {\em client/server\/}
paradigm for services.
Speaking in broad generalities:
a {\em server\/} listens on a well-known {\em port\/}
(e.g., file transfer activities occur over TCP port~21),
awaiting a connection.
At some later time,
a {\em client\/} will connect to the server.
After some application-specific negotiations,
the client begins to make requests of the server,
which in turn makes various responses.
These transactions
continue until either the client or server determine that no further
transactions remain and the connection should be closed.

Further,
servers in the ARPAnet Reference Model tend to be stateless in nature.
That is,
there is no inherent state information retained between connection sessions.
As a result,
connections can and generally do exist in parallel between several clients and
several servers
(each of the latter being instantiated when a client connects to the
well-known port).
Of course,
there may be side-effects between different instantiations
(e.g., when one file server removes a file,
this usually results in other servers being unable to retrieve that file),
but there is no state information which ``lives'' between instantiations of a
given server.

Finally,
the ARM is not a seven-layer architecture:
applications reside directly above the TCP
and have implicit session and presentation mechanisms
(i.e., they are application-specific).

The DDN protocol suite is quite mature,
having stabilized about {\oldstyle 1980}
(the fundamental work leading to TCP/IP having been done since the early
{\oldstyle 1970\/}'s).
TCP/IP enjoys a wide vendor support base and user population.
Typically, vendor support from TCP/IP to other technologies
(e.g., IBM's SNA\cite{SNA}) is available,
and many vendors offer board-level implementations at the transport layer.
As of April, {\oldstyle 1986},
in the ARPA Internet,
most probably the largest unclassified TCP/IP internetwork,
it was estimated that there were 2400~hosts residing on 400~networks
connected via 120~gateways.
The same source\cite{IP.Requirements} estimates that the ARPA Internet is
growing at the rate of approximately 10\% a month.

The services available in the DDN protocol suite are wide and varied,
supporting everything from cross-network debugging,
to network measurement and host measurement,
to voice protocols,
and so on (consult \cite{Assigned.Numbers} for a full list).
Perhaps the three most notable services are:
SMTP\cite{SMTP}, the Simple Mail Transfer Protocol,
which provides store-and-forward service for text messages;
FTP\cite{FTP}, the File Transfer Protocol,
which provides remote file access;
and,
TELNET\cite{TELNET}, which provides network virtual terminal access.

\subsection	{The OSI Protocol Suite}
Offering services similar to the TCP is the OSI Transport
Service\cite{ISO.TP.Service},
henceforth called TP,
which addresses the same general issues.%
\footnote{This paper references the ISO specifications rather than
the CCITT recommendations.
The differences between these parallel standards are quite small,
and can be ignored, with respect to this paper, without loss of generality.
To provide the reader with the relationships:
\[\begin{tabular}{lll}
	Session protocol&	\cite{ISO.SP.Protocol}&
						\cite{CCITT.SP.Protocol}\\
	Transport protocol&	\cite{ISO.TP.Protocol}&
						\cite{CCITT.TP.Protocol}\\
	Transport service&	\cite{ISO.TP.Service}&
						\cite{CCITT.TP.Service}
\end{tabular}\]}
In contrast though,
the reference model for Open Systems Interconnection (OSI)\cite{OSI},
is based on a {\em user/provider\/} paradigm for services.
That is,
instead of viewing an asymmetric horizontal relationship between peers
(as with the servers and clients in the ARM),
the peers are viewed as being completely symmetric in behavior.
Instead of a server listening on a well-known port
when a connection is established,
the {\em provider\/} fires an {\sf INDICATION\/} event for the entity at the
next highest layer.
This entity, called the {\em user}, catches the event and acts accordingly.
Although this might be implemented by having a user acting as a server
``listen'' by hanging on the {\sf INDICATION\/} event,
from the perspective of the OSI reference model,
all users remain in the {\sf IDLE\/} state until they generate a
{\sf REQUEST\/} or accept an {\sf INDICATION}.

The services under consideration and being implemented for the OSI protocol
suite are also wide and varied.
Perhaps the most notable services are:
FTAM\cite{ISO.FTAM}, the File Transfer, Access, and Management Protocol;
and,
MHS\cite{MHS}, the suite of protocols for Message Handling Systems,
originally developed by IFIP in {\oldstyle 1979}.

\subsection	{Comparison of Services}
Several studies have been performed comparing the DDN and OSI protocol
suites
(e.g., \cite{NRC.Report}, which concluded that the TCP and the TP were
functionally equivalent).
As discussed by \cite{TCP.convert.ISO},
the services offered by the two protocols can be divided into three areas:
connection establishment, data transfer, and connection release.
Although we disagree with the bias with which the comparison is presented,
the results are essentially correct:
the two protocols offer services of functional equivalence with very few
differences.
For our purposes,
it is important to note:
\begin{itemize}
\item	Simultaneous connection requests by both ends of the connection
result in two connections being established in the TP,
while in the TCP a single connection is established.
Experience shows this difference to be of little practical value in
a large number of applications using a connection-oriented transport service.

\item	The TCP does not have an expedited data transfer mechanism,
although an urgent pointer is available.
This paper will suggest a method for implementing an expedited data stream on
top of the TCP so that OSI-based applications can naively utilize this
mechanism.

\item	The TP does not have a graceful ``drain-and-close'' mechanism for
connection release.
Most user profiles of the TP
(e.g., the National Bureau of Standards profile of TP4\cite{NBS.TP})
require the inclusion of an orderly release facility in the TP,
based on the corresponding session-level mechanism.
As will be made clear later,
this lack of functionality in TP is unimportant.
\end{itemize}

\subsection	{A Digression on the Interoperability of Applications}
Ideally,
in order to preserve our investment in existing tools
we would like to do one of two things:
\begin{enumerate}
\item	achieve interoperability between similar applications in each protocol
	suite;
	or,
\item	migrate an existing application from one protocol suite to the other.
\end{enumerate}
In practice,
neither of these goals can be realized without a significant investment.

In the first case,
although both the DDN protocol suite and the OSI protocol suite have,
for example,
a``mail'' application,
they are quite different.
That is, they provide electronic mail services,
but the type of services available and the implementation of those services
is vastly different.
Hence,
in order to make mail in the DDN protocol suite interoperate with mail in the
OSI protocol suite,
a special-purpose gateway is required.
(To grasp the complexity of this issue,
interested readers should consult \cite{ARPA.MHS} which describes the
operation of such a gateway for mail applications.)

Given that the first alternative is not practical,
what about the second?
Unfortunately,
this too is not straight-forward.
As a brief review,
let us compare the two {\em protocol stacks\/} which comprise the
DDN and OSI protocol suites.
These are presented in Figure~\ref{stacks}.
\tagfigure{2-1}{Comparison of the Protocol Stacks}{stacks}

Although there is great similarly until we reach the the transport layer,
at this point,
the stacks diverge.
The figure emphasizes that
each application in the DDN protocol suite has its own implicit session and
presentation mechanisms,
whereas in the OSI protocol suite these exist explicitly as layers.
In short,
in the DDN protocol suite,
the services required by the applications are exactly those which are offered
by the transport service;
but,
in the OSI protocol suite,
the services required by the applications and those provided by the transport
service are not the same.
Hence,
although the {\em services offered by the transport layer\/} are quite similar
between the two stacks,
the {\em services required by the applications\/} are quite different.

% run this through LaTeX with the appropriate wrapper

\section	{The Approach}
From the viewpoint of OSI-applications development,
we should note that the best of all solutions to the problem sketched earlier
requires a complementary co-existence:
\begin{itemize}
\item	we want to utilize mature TCP/IP functionality and stability which is
	not currently present in the immature OSI protocol suite;

\item	we want to utilize new OSI functionality as it becomes available;
	and,

\item	we want to develop OSI-based applications now in an {\em evolutionary},
	not {\em revolutionary} fashion.
\end{itemize}
These criteria suggest motivations for a migration strategy.
In migrating from the DDN protocol suite to the OSI protocol suite,
there are several strategic choices to be made.

For the lack of a better term,
we will call the object which joins the two protocol suites a {\em magic box}.
We favor the use of this term instead of {\em gateway},
since the former term invokes less inference on the part of the reader.
To choose where the magic box should ``live'',
we make three observations:
\begin{enumerate}
\item	Joining at the session layer (or above) introduces too much
dependency on the ARPAnet client/server model, as discussed in the
introduction.
This argues for the magic box to live no higher than the transport
level.

\item	Joining at the network layer (or below) foregoes use of the TCP and
requires complete implementation of the TP on top of the \dod/ IP.
It also prevents the use of the efficient board-level products which
implement the TCP.
This argues for the magic box to live no lower than the service
access points at the top of the transport layer.

\item	In general,
the construction of magic boxes at any location other than the transport
layer is problematic at best;
this will be illustrated later on
(in Section~\ref{comparison} on page~\pageref{comparison}).
\end{enumerate}
Given these arguments,
the OSI Transport Service Access Point (TSAP) is the ideal place for the magic
box to be constructed:
it allows us to take advantages of the main strengths of the DDN protocol
suite while providing OSI defined services at the layers above.
Figure~\ref{tsap.model} outlines the relationships we propose.
\tagfigure{3-1}{OSI Transport Services on top of the TCP}{tsap.model}

Our approach is to utilize exactly one TCP port (number~102) with
TS-peers running on this port to interpret the protocol data units (PDUs)
on the connection and dispatch them accordingly.
This is not nearly as difficult as it sounds,
since the TCP performs all the multiplexing,
and provides each client/server pair with exactly one connection to manage.
As a result, the dispatch function is trivial (i.e., null).
Currently,
we use a very simple mapping between TSAP IDs and services
(a shortcoming described later).

\subsection	{Summary of the Protocol}\label{protocol}
The external behavior of our magic box is to use the services offered by the
TCP
and co-operate with other magic boxes,
to provide the services defined for the TP.
For our purposes,
we define the {\em services offered\/} by a protocol
as its service primitives.
The services offered by the TCP and the TP are summarized in
Tables~\ref{tcp.services} and~\ref{tp.services} respectively
(readers should consult the authoritative works \cite{TCP} and
\cite{ISO.TP.Service} for more detailed information).
For purposes of exposition,
we use the term {\em TS-user\/} to denote an entity utilizing the transport
services,
and {\em TS-provider\/} to denote an entity providing those services.
\tagtable{3-1}{TCP Service Primitives Offered}{tcp.services}
\tagtable{3-2}{TP Service Primitives Offered}{tp.services}

Since both the TCP and the TP primarily offer a virtual-circuit service,
it is not surprising that
all the ``hard parts'' of the TP are also done by the TCP
(e.g., three-way handshake, choice of initial sequence number, windowing,
multiplexing, and so on).
This leaves us with the task of devising a protocol,
running above the TCP,
which performs whatever tasks are necessary to map one flavor of service
primitives into the other.

Despite the symmetry of the TP,
it is useful to consider the magic-box protocol with the perspective of a
client/server model.
We view two entities, denoted {\em TS-peers}, as implementing this protocol.
One of these TS-peers,
the TSAP server, begins by LISTENing on TCP port~102.
When the other peer, the TSAP client, successfully connects to this port,
the protocol begins.

A client TS-peer decides to connect to the TCP port on the service host
when the client's TS-user issues the {\sf T-CONNECT.REQUEST\/} action.
One of the parameters to this action specifies the TSAP ID (identifier) of
the remote TS-user.
The client uses the TSAP ID to ascertain the IP address of the server,
and attempts to open a connection to TCP port~102 at that IP address.
If not successful,
the client fires the {\sf T-DISCONNECT.INDICATION\/} event for the TS-user.

Although the TCP offers a byte-stream,
the magic-box protocol packetizes the bytes into units which have the
identical syntax (format) as data units in the TP (TPDUs).
These packets are termed TPKTs to avoid any confusion with the term TPDU.
When the connection is opened,
the client fills in a request-connection TPKT, sends it to the server, and
awaits a response.

The server, upon receipt of the TPKT,
validates the contents of the TPKT
(checking the version number, verifying the type code, and so on).
If the packet is invalid,
the server sends a request-disconnection TPKT,
closes the connection, and goes back to the listen state.
Otherwise,
the server examines the TSAP ID of the TS-user with which the remote TS-user
wants to communicate.
If the specified TS-user can be located and started,
the server starts this TS-user by firing the {\sf T-CONNECT.INDICATION\/}
event.
Otherwise a request-disconnection TPKT is sent
(and the server closes the connection and goes back to the LISTEN state).

The TS-user receiving the event will now respond with one of two actions,
either {\sf T-CONNECT.RESPONSE\/} or {\sf T-DISCONNECT.REQUEST}.
Depending on the response,
the server sends either a request-disconnection or a confirm-connection
TPKT back to the client.
The server then enters the SYMMETRIC PEER state.

When the client receives the reply TPKT from the server,
it performs the usual validation.
If the packet received was a request-disconnection TPKT,
the client fires the {\sf T-DISCONNECT.INDICATION\/} event,
and closes the connection.
Otherwise,
it fires the {\sf T-CONNECT.CONFIRMATION\/} event
and enters the SYMMETRIC PEER state.

Once both sides have reached the SYMMETRIC PEER state,
the protocol is completely symmetric and the notion of whether the TS-peer
started as a client or server is lost.
Both TS-peers act in the following fashion:
if the TCP indicates that data can be read,
the TS-peer reads the TPKT, and validates the contents.
\begin{itemize}
\item	A request-disconnection TPKT results in the
{\sf T-DISCONNECT.INDICATION\/} event being fired, and the TCP connection
being closed.

\item	A data TPKT or expedited data TPKT results in {\sf T-DATA.INDICATION\/}
or {\sf T-EXPEDITED DATA.INDICATION\/} event (respectively) being fired.

\item	Invalid TPKTs result in the {\sf T-DISCONNECT.INDICATION\/} event
being fired for the TS-user,
a request-disconnection TPKT being sent,
and the connection being gracefully closed.

\item	An error on the TCP connection also results in the
{\sf T-DISCONNECT.INDICATION\/} event being fired.
\end{itemize}
As expected,
the {\sf T-DATA.REQUEST}, the {\sf T-EXPEDITED DATA.REQUEST},
and the {\sf T-DISCONNECT.REQUEST\/} actions on the part of the TS-user
result in the appropriate TPKT being sent to the remote TS-peer.

In the interests of brevity,
many parts of the protocol were simplified or omitted from the discussion
above.
For example,
when a request-disconnect TPKT is sent,
it contains a code indicating why the disconnect was initiated.
The precise protocol (including state machines) is presented in
\cite{TSAP.on.TCP.old}.

\subsection	{Design Decisions}
The previous section discussed the protocol in simplistic detail.
We should now consider certain nuances in the protocol's design and behavior.

\subsubsection	{Packet Format}
The choice of packet format for the magic box protocol was made rather arbitrarily:
the TP format for TPDUs was chosen as it was suitable for our needs.
Although a few fields are ignored,
this introduces a very small amount of additional overhead.
For example,
on a request-connection TPKT (the worse case),
there are 6 octets of ``zero on output, ignore on input'' fields.
Considering that the packet overhead is fixed,
requiring that implementations allocate an additional 6 octets is not
unreasonable.
As experience is gained,
some of these fields (e.g., the class field) may be used in future versions
of the protocol.

\subsubsection	{Quality of Service}
In our proposal, this is left ``for further study''.
We expect a future version of the protocol to attempt to map the TP QOS
parameters into the DDN IP precedence and security parameters.

\subsubsection	{Administration of Address Space}
It is tempting to define a straight-forward mapping between the OSI and
TCP/IP addressing domains.
For example,
the OSI network service access point identifier (NSAP ID) could be mapped
into a DDN IP address,
and a combination of the service access point selectors of the higher layers
could be mapped into a TCP port number.
Unfortunately there is no straight-forward mapping for
the OSI session and presentation service access point selectors,
as each application in the DDN protocol suite has its own implicit session
and presentation mechanism.
One solution is to view the mappings as:
\[\begin{tabular}{rlc}
	$<$NSAP ID$>$&
			$\longleftrightarrow$&	$<$IP address$>$\\
	$<$TSAP  selector, SSAP selector, PSAP selector$>$&
			$\longleftrightarrow$&	$<$TCP port$>$
\end{tabular}\]
This approach is particularly interesting because it suggests that the
TP can be run directly above the DDN IP protocol.
However, this suggestion is not necessarily a benefit
(again see Section~\ref{comparison}).

However,
the TCP port space cannot accommodate the space of OSI higher layer selectors.
The TCP supports a port space denoted by small integers,
represented as unsigned 16--bit quantities.
Further, any port larger than~1023 is reserved for the use of clients,
and ports larger than ~511 are reserved for the use of local servers.
This leaves about 510~ports for well-known (pre-assigned) services,
most of which are already in use by existing services.

\subsubsection	{Expedited Data}
The largest difference between the services offered by the TCP and the TP is
the expedited data service offered by the TP.
We initially experimented with three approaches which could be used in
implementing expedited data with the TCP:
\begin{itemize}
\item	Use a single connection without the TCP URGENT\\[0.1in]
All data TPKTs and expedited data TPKTs are placed on the same TCP connection.
As a result,
there is no actual ``priority'' associated with the
{\sf EXPEDITED DATA.REQUEST\/} action other than it
eventually resulting in the {\sf EXPEDITED DATA.INDICATION} event occuring in
the future.
Furthermore,
we are guaranteed that once any expedited data is sent,
it will arrive before any data sent in the future arrives.
This is true to the letter, though perhaps not to the spirit of the TP.

\item	Use a single connection with the TCP URGENT\\[0.1in]
All data TPKTs and expedited data TPKTs are placed on the same TCP connection.
However,
expedited data TPKTs are sent with the URGENT bit set.
The receiving TCP could signal the TS-provider that URGENT information was
available on the connection.
The TS-provider could then buffer all further TPKTs until the TPKT
containing the URGENT pointer was received.
After this TPKT had been handled,
the buffered (non-expedited) data could be acted upon.
Although this is more true to the spirit of the TP,
it requires some complex buffer management and may not be implementable on
single-thread implementations of the TCP
(e.g., some implementations for \msdos/ on the PC).

\item	Use two connections\\[0.1in]
A third approach is to open two connections.
During connection establishment,
after making a TCP connection to the server,
the client starts a PASSIVE, non-blocking TCP open.
The address of a TCP port associated with the open is then passed
in the TSAP ID of the request-connection TPKT.
Immediately before the server sends the confirm-connection TPKT,
it connects to the indicated port.
Upon success,
it passes in the confirm-connection TPKT the TCP port it used to make
the second connection (thus introducing a secondary correctness check).
This requires no buffer management on the part of the TS-provider,
and can be implemented even in single-thread implementations of the TCP.
The disadvantage of our approach is that it may violate the letter of the TP!
Since two connections are used,
the paths in the subnet taken by the packets may differ,
and it is possible that non-expedited data sent after expedited data may
actually arrive earlier.
\end{itemize}
Version~1 of the protocol used the third approach to implement the expedited
data service.
However, in order to guarantee the semantics of the service,
it became apparent that a complicated buffering scheme was required.
Rather than introduce additional complexity,
in version~2 of the protocol we opted for the first approach.
This requires no buffering and implements the semantics correctly,
albeit at the boundaries of the service specification.

\subsection	{Work To Date}
Work to date has reached two milestones:

\subsubsection	{RFC983}
In April of {\oldstyle 1986},
the DDN Network Information Center issued RFC983,
entitled {\it ISO Transport Services on top of the TCP\/}\cite{TSAP.on.TCP.old}.%
\footnote{\cite{TSAP.on.TCP.old} specifies version~1 of the protocol.
Based on our experiences,
we are currently running version~2,
which consists of four minor changes to the original protocol.
A new RFC, describing these changes, is undergoing review.
Appendix~\ref{changes} summarizes these changes.}
This document presents our model of operation and gives a formal description
of the protocol described in the Section~\ref{protocol}.

It is important to understand exactly what RFC983 intends.
RFC983 does not specify how one constructs an interface to the OSI TSAP.
It indicates how one might build such an interface on top of the TCP.
That is,
given the abstract service definitions for the TP,
instructions are given as to how those can be mapped on to the services
provided by the TCP.
From our perspective, a proper implementation of RFC983 exhibits the
following properties:  
\begin{enumerate}
\item	it has the TSAP interface that you want on your host;
	and,
\item	it uses the protocol defined in RFC983.
\end{enumerate}
RFC983 has no intention of specifying an ``OSI protocol''.
Rather it specifies a DDN-style protocol which provides OSI services.
It is the intent of RFC983 to permit standard OSI protocols to run on top of
the TCP.
It is not the intent to build OSI-like protocols for the DDN.  

From our experience, we agree with John Leong of CMU that:
\begin{quote}\em
``In general,
it is important for one to produce good generic protocol interface design so
that a particular protocol implementation or even the protocol itself can
easily be replaced without affecting the code in the upper or lower layer.''
\end{quote}
It is the intent of RFC983 to be true to this tenet.

\subsubsection	{Prototype Implementation}
To test our ideas,
we constructed a prototype implementation of RFC983 under Berkeley \unix/.
The implementation supports both OSI-style asynchronous {\sf INDICATION}
events,
and also DDN-style synchronous events.
We intend this software to play a key role in the migration strategy which is
discussed in the next section.

\subsection	{Comparison to Other Approaches}\label{comparison}
\cite{Protocol.Conversion} makes a thorough investigation of the many
issues involved in protocol conversion and complementing,
and it is unnecessary for us to summarize those findings here.
Instead, let us concentrate on previous work devoted to implementing the magic
boxes between two different protocol suites,
and in particular between the DDN and OSI protocols.
Further,
it has been our claim throughout this paper that building a magic box at any
location other than the transport layer is not particularly useful.
In this section,
we will consider this argument.

In \cite{TCP.convert.ISO}
it is proposed that conversion between the DDN and OSI protocol suites
occur inside the transport layer instead of at the transport service access
point.
After an analysis of the state machines associated with the TCP and the TP,
and the identification of a common subset of services,
a state machine for a gateway between the two is derived.
The resulting gateway algorithm is no more complicated than the most
complicated of the two original machines.
It is then suggested that this approach is practical as a short-term solution
to the TCP/IP and OSI interoperability problem.

\subsubsection	{Our Analysis of this Work}
Unfortunately,
in our opinion,
the existence of such a gateway is of limited {\em practical\/} value.
Although the gateway does achieve an end-to-end transport capability,
with one TS-user utilizing a TP interface and the other TS-user utilizing a
TCP interface,
there is no common session, presentation, and application layer.
That is,
although entities above the transport layer can have the potential to
communicate between the two suites,
no useful work can occur
until the DDN ``world'' implements the higher-layer OSI protocols,
or the OSI ``world'' implements the higher-layer DDN protocols,
or both.
\cite{TCP.convert.ISO} suggests that the DDN domain immediately implement
the FTAM on top of the TCP, and so on.
In the authors opinion,
this is not a practical short-term solution.
Furthermore,
it appears to violate the widely-held notion of separation of knowledge
between layers.

It does however illustrate a recurring theme in work wherein conversion
between different protocol suites is reported
(e.g., \cite{SNA.convert.XNS},
which discusses the interconnection of SNA and XNS networks).
The theme, of course, is that general utility requires that
protocol conversion occur at every layer in which the two suites can be
connected.
For example,
when considering the application of electronic mail,
building a TCP/TP gateway is not useful unless there is either:
\begin{itemize}
\item	an SMTP implementation running in both the TCP and TP domains;
	or,
\item	an X.400 implementation running in both the TCP and TP domains;
	or,
\item	an SMTP/X.400 (or more precisely SMTP/P1\cite{MHS.P1}) gateway
	running in the TCP/TP gateway.
\end{itemize}
Each of these three alternatives is expensive to implement,
and in addition, the first two are politically unacceptable to the parties doing
the implementing.
All three require that everything above the transport layer,
except for the application itself (i.e., session and presentation),
be implemented as well.

In comparing our approach with previous work,
we see that ultimately both approaches have the same end-result:
the same transport service is offered.
However,
our approach is significantly easier to implement.
Neither approach makes the problem of achieving interoperability between
applications in the two protocol suites any easier:
the only way to achieve interoperability is to implement special-purpose
gateways for each pair of related applications.
However,
the approach we suggest has two important advantages over the protocol
translation approach, once we are willing to acknowledge that we intend to
migrate to one of the protocol suites (i.e., from the DDN to the OSI protocol
suite):
first, we can develop and gain experience with OSI applications now;
and, second,
any new applications we develop can run in either environment.
This has the important implication that from this point on,
that any future applications we develop will be guaranteed to run in the
protocol suites available to us
(that is, no new work will have to be done when we migrate from one protocol
suite to the other).

In the next section,
we will consider the value of our approach in the medium-term,
when we want to transition from the DDN protocol suite to the OSI protocol
suite.

% run this through LaTeX with the appropriate wrapper

\section	{Migration Strategies}
Let us now consider the larger question of how a migration
strategy based on the approach we've suggested might be devised.
In practice,
successful migration strategies are based on careful planning and a
step-by-step progression toward the final goal,
wherein each step consists of changing only one logical portion of the
environment.
This tends to isolate the effect of unexpected consequences,
and also to minimize the degree of ``culture shock'' suffered by users of the
system.
In our approach,
we suggest three phases which are not clearly distinct.
We imagine all three to be on-going projects in which the first phase enjoys
the most emphasis initially,
and as time progresses,
emphasis shifts to later phases.

An interesting exception to our strategy should be briefly mentioned.
Although there is little utility in having a host support both the DDN and
OSI protocols for virtual terminal applications or file transfer applications,
the store-and-forward nature of electronic mail suggests that a gateway
between MHS and DDN mail is essential during a transition period.
Fortunately,
work is proceeding in this area along a separate line
(interested readers should consult \cite{ARPA.MHS}).

Finally,
note that an underlying assumption of our migration strategy,
is that any new hosts introduced into our internet must support TCP/IP
until the end of the final phase.
Although it is preferable that these hosts also support the OSI protocol suite,
this is not mandatory.

\subsection	{Phase One: Development Environment for OSI Applications}
Phase one of our migration strategy has the goal of building an OSI
application development environment on a DDN foundation.
Our intention is then to build OSI applications and gain experience using them,
and to do so within the robust and mature DDN network community.

To accomplish this,
we make use of the OSI TSAP on top of the TCP.
We will also need
implementations of the OSI session layer\cite{ISO.SP.Protocol,ISO.SP.Service},
and the OSI presentation layer\cite{ISO.PP.Syntax,ISO.PP.Encoding},
and the common Application Service Elements Kernel
(CASE)\cite{ISO.CASE.Service}.
During the integration of this software,
it will be important to track the work currently being done in the area
of programming language interfaces,
in order to utilize these applications in later phases.

Finally,
as new applications are developed during this phase,
we will also need to develop user agents.
As an application and its corresponding user agent software is brought
on-line,
users of applications in the DDN protocol suite will be encouraged to try the
new OSI equivalents instead.
Although progress will be slow at first,
this has several advantages:
first, it helps prepare users for the final phase,
in which only OSI applications are available;
second, it allows us to make use of the vast wealth of application and user
experience to be found in the many existing communities that use the DDN
protocol suite.

Of course,
there is an important side-effect that this phase introduces:
properly designed user-interfaces for network services
(e.g., file transfer) should be able to distinguish between the DDN and OSI
services which could be offered by the remote host.
For example,
a user-interface to a file transfer facility,
could select either an FTAM or FTP client,
depending on the protocols supported by the service host.
In either case,
a TCP/IP underpinning is used to support these services,
and the user is naive with respect to the actual application protocols used.

\subsection	{Phase Two: Experiment with Migration Engines}
Phase two of our migration strategy consists of putting in the field several
computers which have both the DDN and the OSI protocol suites implemented in
them.
We term these hosts {\em migration engines}.%
\footnote{Readers should not be misled by the use of this colorful term.
A migration engine is simply any host having both the DDN and OSI protocol
suites resident.
In many cases,
installing additional software in the host operating system is sufficient to
meet this criterion.}
Although it is tempting to consider these migration engines as potential
gateways, particularly application gateways,
between the DDN and OSI protocol suites,
considering the many reasons previously discussed,
this is highly undesirable.
Rather,
we use these migration engines to test whether the applications developed in
our DDN-style environment will function correctly in an OSI-style environment.

Of course,
once a migration engine is in place,
it allows us to experiment with performing gatewaying at the internet layer
(e.g., as suggested by \cite{TCP.convert.ISO}).
This has the potential of permitting additional connectivity between networks
using the DDN protocols and those using the OSI protocols.
Although this will not achieve interoperability between the DDN and OSI
applications (as was discussed in the previous section),
it does achieve interoperability between OSI applications running in DDN-style
networks and OSI-style networks.

Initially,
the number of migration engines available from different vendors
will be rather small, and the implementations of the OSI protocols on them
will be immature and most likely inefficient.
This is to be expected.
As the expertise with the technology is gained,
we can expect these deficiencies to be lessened.
As the OSI implementations improve,
more and more OSI applications developed in phase one can be migrated.

\subsection	{Phase Three: Deploy Migration Engines}
Phase three of our migration strategy consists of an upgrade or replacement
of existing computers with hosts speaking both protocol suites.
Once we find that vendors are able to supply OSI capability with robust and
mature characteristics,
then we can begin to field the migration engines throughout our DDN-style
network.
During phase three,
the majority of users will employ user agents for OSI instead of DDN
applications,
since all hosts will be supporting the OSI user agents
(even those which only support the DDN protocol suite).

Finally,
at the end of phase three,
the ratio of hosts speaking only the DDN protocol suite to hosts speaking at
least the OSI protocol suite will be very low,
and the requirement that hosts speak the DDN protocol suite can be lifted.
It is critical to a smooth transition however,
that this requirement not be lifted prematurely.
Recall that,
given the wide range of OSI applications now implemented
and the availability of the software developed during phase one,
it will be relatively inexpensive to maintain support of the DDN protocol
suite.

\subsection	{Work To Date}
We have implemented an ISO Development Environment (ISODE) at the Northrop
Research and Technology Center.
The version~1.0 distribution was released in September, {\oldstyle 1986}.
Although ISODE is not proprietary,
it was not placed in the public domain in order to include a ``hold
harmless'' clause in the release.
However, for all intents and purposes, the release is openly available.

ISODE currently runs on native Berkeley 4.2~\unix/,
and on AT\&T SVR2~\unix/ with an Excelan \exos/ card
(a board-level that implements the TCP).
Current modules in the release include the TSAP described in the previous
section, an OSI Basic Combined Subset (BCS) session,
an ASN.1 encoding mechansism,
a parser which takes an ASN.1 specification and produces a program fragment
which recognizes the corresponding APDUs,
and an implementation of the ECMA Remote Operations Services
(ROS)\cite{ECMA.ROS}.

For information on receiving a copy of the ISODE distribution,
consult Appendix~\ref{distribution}.

% run this through LaTeX with the appropriate wrapper

\section	{Conclusions}
In this paper we have suggested that it is better to perform
{\em interface translation\/} rather than {\em protocol translation\/}
when one is interested in migrating between two protocol suites.
In our method,
which uses such an interface translation approach,
we implement the interface to the OSI Transport Services on top of the TCP.
This has the additional advantage of facilitating the development of OSI
applications in a robust and mature network environment,
and in allowing us to avoid any additional work in the future when we migrate.
In short,
we are able to make use of a complementary co-existence between the two
suites,
utilizing the best of both.
Our fundamental assumption in doing this is that the lower levels of the OSI
protocol suite will become fully supported as we follow our
migration strategy.

Furthermore,
we have discussed the difficulties inherent in providing interoperability at
the application level,
and concluded that, as a part of a migration strategy,
building the special-purpose gateways required for each pair of related
applications is not a practical approach.
Neither the protocol translation or the interface translation approach is of
any benefit in building these gateways.

Finally,
we have demonstrated that the DDN protocol suite,
because of both its maturity and closeness to to the OSI suite at the TSAP
provides an excellent migration vehicle for those users in need of
immediate electronic communications.
In view of the many new major investments being made in TCP/IP networks
(e.g., by the NSF and NASA),
we feel that our approach,
which emphasizes {\em evolution\/} rather than {\em revolution\/},
is a useful solution.

% run this through LaTeX with the appropriate wrapper

\section*	{Acknowledgements}
The authors would like to thank 
Vinton G.~Cerf of the Corporation for National Research Initiatives,
David J.~Farber of the University of Delaware,
Einar A.~Stefferud of Network Management Associates,
Stephen S.~Wolff of the National Science Foundation,
and the anonymous referrees.
Their insightful comments have resulted in a much clearer paper.


\appendix
% run this through LaTeX with the appropriate wrapper

\section	{Distribution Mechanics}\label{distribution}
The ISODE distribution is available in two forms:
\begin{itemize}
\item	Anonymous FTP\\
If you can FTP to the ARPA Internet,
use anonymous FTP to the ARPAnet host louie.udel.edu (10.0.0.96) and retrieve
the file portal/isode-1.tar.
This is a 2.0MB \man tar(1) image.
Alternately,
the file \verb"portal/isode-1.tar.Z" can be retrieved which contains a
\pgm{compress}'d version of the \pgm{tar} file,
and is about 0.8MB in size.

\item	Magnetic tape\\
Otherwise, send a magnetic tape with a self-addressed mailing label to:
\[\begin{tabular}{ll}
Postal address:&	Northrop Research and Technology Center\\
&			Attn: Automation Sciences Laboratory (0330/T30)\\
&			One Research Park\\
&			Palos Verdes Peninsula, CA  90274\\
&			USA\\[0.1in]
Telephone:&		+1--213/544--5393\\[0.1in]
Internet Mailbox:&	\tt Bug-ISODE@NRTC.NORTHROP.COM
\end{tabular}\]
The lab will send back the tape in \man tar(1) format at 1600bpi,
along with a copy of the {\em User's Manual}.
\end{itemize}

% run this through LaTeX with the appropriate wrapper

\section	{Changes to RFC983}\label{changes}
The current version, version~2, of the protocol described in this paper
incorporates four minor changes to RFC983\cite{TSAP.on.TCP.old}.
These are:
\begin{itemize}
\item	``Infinite'' length TSDUs are supported\\
The original protocol supported transport service data unit (TSDU) lengths
of no more than approximately 65K octets.
This restriction was removed.
From the performance standpoint,
it turns out that a TSDU with length on the order of 65K octets (or less) can
be handled very efficiently by most implementations.
If a user of transport services requires a larger TSDU size,
the current protocol will support this,
but any implementations of the protocol will most likely do this less
efficiently.

\item	More correct use TSAP IDs\\
The original protocol mistakenly placed entities other than OSI session
on top of the TSAP.
This has been corrected.

\item	Negotiation of Expedited Data\\
The original protocol mistakenly aborted the connection establishment attempt
if there was a mismatch in the proposed use of expedited data.
This has been corrected:
the use of expedited data is negotiated downwards, as per the ISO specification.

\item	Handling of Expedited Data\\
As described earlier,
RFC983 tried to manage two TCP connections,
one for expedited TSDUs and the second for all other traffic.
Although intuitively natural,
this method could not guarantee the semantics of the expedited data service
in the TP without a complex buffer management scheme.
Hence, currently only a single connection is employed.
Owing to the properties of the TCP,
the expedited data semantics of the TP are handled properly.
\end{itemize}

Although all of these changes were rather small,
regretably version~2 of the protocol is incompatible with the original
protocol found in RFC983.


\bibliography{bcustom,networking}
\bibliographystyle{alpha}

\showsummary

\end{document}
