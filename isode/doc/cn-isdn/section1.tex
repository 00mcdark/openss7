% run this through LaTeX with the appropriate wrapper

\section	{Introduction and Motivation}
To promote the use of open systems environments in the office,
technical, and manufacturing environments,
two influential user groups were formed,
the Manufacturing Automation Protocol group (MAP),
sponsored by General Motors Corporation,
and, the Technical and Office Protocols group (TOP),
sponsored by Boeing Computer Services.
In order to act more effectively,
the Society of Manufacturing Engineers (SME)
consolidated the two user groups into MAP/TOP under SME sponsorship.
MAP/TOP is championing the use of open systems interconnection (OSI),
primarily in the United States.
The widespread adoption of open standards is not a new idea,
for example,
the Department of the Defense has,
since the mid-{\oldstyle 70\/}'s been mandating the use of what is now called
the Defense Data Network (DDN) protocol suite for its computer-communication
applications.

MAP/TOP has received support from both the public and private sectors,
particularly from users who have suffered for years with myriads of
computer-based systems which simply do not talk to each other
(even if all those systems are purchased from a single vendor).
This has influenced the Department of Defense to clearly state,
through a number of recent programs and proposals,
that successful contractors must be able to communicate electronically,
not only internally, but also with the customer (\dod/),
preferably using the OSI protocols.

This leaves most defense contractors and many others in the following
situation:
With the use of open standards, many benefits can be achieved.
For example,
the use of open standards makes it possible to solve the so-called
{\em islands of automation\/} problem as described above.
However,
although the work currently being done in the MAP/TOP community promises a
major breakthrough in electronic communications,
MAP/TOP is generally not expected to achieve widespread use for at least
five years.%
\footnote{This problem can not be understated:
for example,
in the international committees,
there is no consensus to the routing issues at the network layer.
Some insiders suggest that three years of work will be required before
the non-trivial routing problems are resolved (at the committee level).}
Contractors planning to do business with the Department of Defense cannot
afford the luxury of waiting five years before starting to solve their
communications problems.
Nor can they simply ``scrap'' everything and start from ``scratch'' when
MAP/TOP has materialized.
Hence, we find ourselves in a dilemma.

\subsection	{Understanding the Problems}
Thus far,
we have isolated two problems:
first, we want to be running MAP/TOP applications now;
and,
second, if we are using other computer-communication technologies now,
it would be advantageous to migrate from our current approaches to the
MAP/TOP approach.

A key point in understanding the thrust of MAP/TOP,
is that MAP/TOP bases itself on the OSI protocol suite.
In many senses,
MAP/TOP represents an effort to ``prune the OSI tree'' in order to provide
vendors with a subset of the protocols which can be implemented in a timely
manner.
For example,
MAP/TOP currently limits itself to local area networking.
Further,
as new needs arise,
MAP/TOP influences the OSI community towards the development of protocols to
meet those needs.
Hence,
for the remainder of this paper,
we will use the term ``OSI protocol suite'',
understanding that it includes MAP/TOP,
without any loss in generality.

While the last several years of work in OSI have resulted in a
consensus regarding the transport layer and below,
application protocols have not enjoyed this degree of attention
(with perhaps the notable exception of MHS\cite{MHS}).
Further,
experience in the field repeatedly demonstrates that the development of
standard application protocols takes as long, or {\em much longer},
than protocols at other layers.
Finally,
considering that there is still a great deal of basic research to be
done in a number of application protocols areas,
such as distributed information bases and distributed robotic control,
it becomes apparent that the problem of application protocol development
is significant.

\subsection	{Finding the Solutions}
To solve these problems,
we begin by considering two observations.
First,
many of these problems have been solved in the past by others.
There are several other proprietary and open protocol suites,
which address many of these problems.
Indeed,
the last 15~years of the literature describe several successful attempts
at application (and other) protocols.
Second,
we note that MAP/TOP is based on the OSI protocol suite,
which in turn is based on a layered architecture.
The key here is that layered architectures allow us to
``mix and match'' modules with similar requirements at a given layer as
necessary to achieve an optimal solution.
In all cases,
the layered approach emphasizes the {\em services\/} provided at a layer,
not the {\em implementation\/} of those services.

Let us now continue with the problem at hand:
how can we achieve a communications strategy consistent with the
direction of MAP/TOP,
{\em and\/} allow us to communicate electronically with the Department of
Defense?
In this paper,
we suggest using the Defense Data Network (DDN) protocol suite,
commonly referred to as TCP/IP,
as the solution of choice.

During the course of this paper,
it will become clear that the approach we suggest is not limited to
defense contractors only.
Quite the contrary,
we present several arguments which suggest that the
immediate adoption of the DDN protocol suite as a migration path
towards MAP/TOP and OSI is an optimal strategy.
We base these arguments in part on the maturity and stability of the DDN
protocols;
on the multi-vendor support that TCP/IP enjoys;
and on
the widespread use of TCP/IP as an interconnection method among various
proprietary solutions.

In using the notion of layering to solve the problems,
we observe that there are two fundamental approaches which can be used:
\begin{itemize}
\item	A {\em protocol translation\/} approach, as suggested by
	Groenbaek\cite{TCP.convert.ISO} and others;
	and,

\item	an {\em interface translation\/} approach, which we suggest in this
	paper.
\end{itemize}
We argue that our approach is superior since it accomplishes the same
end-results as the protocol translation approach,
but is much simpler and less expensive to implement.
Further,
our approach has two important advantages:
it allows us to develop OSI standard applications directly on hosts using the
DDN protocols,
and any such applications will be able to ``run'' in either environment.

\subsection	{What's in a Name?}
Before proceeding,
the authors feel it important to clarify some terminology used in this paper.
Throughout this work,
we refer to both ``DDN'' and ``OSI'' as different protocol suites.
Although this is currently the dominant interpretation,
there is an alternate perspective, which the authors and others subscribe to.

This alternate view considers OSI as a technique for describing a
com\-put\-er-com\-mun\-i\-cat\-ions architecture in terms of abstractions
(e.g., layering, entities, services, protocols, access points, and so on).
As such, it follows that there are varying interpretations of OSI,
each with a reference model described in terms of these abstractions.
For example,
the Defense Data Network (DDN) protocol suite can be thought of as one
interpretation of OSI;
similarly,
the protocols developed by the International Standards Organization (ISO),
can be thought of as an alternate interpretation of OSI.
That is,
it is sensible for one to discuss
the difference between the DDN and ISO protocol suites,
even though the two suites are based on different reference models.
The commonality the two protocol suites share is derived from the abstraction
technique used to construct their respective models.

In order to maximize the readability of this work,
we have used the consensus interpretation throughout the paper.
It is the opinion of the authors however that the alternate perspective is
more powerful.
