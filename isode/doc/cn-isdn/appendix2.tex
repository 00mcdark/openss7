% run this through LaTeX with the appropriate wrapper

\section	{Changes to RFC983}\label{changes}
The current version, version~2, of the protocol described in this paper
incorporates four minor changes to RFC983\cite{TSAP.on.TCP.old}.
These are:
\begin{itemize}
\item	``Infinite'' length TSDUs are supported\\
The original protocol supported transport service data unit (TSDU) lengths
of no more than approximately 65K octets.
This restriction was removed.
From the performance standpoint,
it turns out that a TSDU with length on the order of 65K octets (or less) can
be handled very efficiently by most implementations.
If a user of transport services requires a larger TSDU size,
the current protocol will support this,
but any implementations of the protocol will most likely do this less
efficiently.

\item	More correct use TSAP IDs\\
The original protocol mistakenly placed entities other than OSI session
on top of the TSAP.
This has been corrected.

\item	Negotiation of Expedited Data\\
The original protocol mistakenly aborted the connection establishment attempt
if there was a mismatch in the proposed use of expedited data.
This has been corrected:
the use of expedited data is negotiated downwards, as per the ISO specification.

\item	Handling of Expedited Data\\
As described earlier,
RFC983 tried to manage two TCP connections,
one for expedited TSDUs and the second for all other traffic.
Although intuitively natural,
this method could not guarantee the semantics of the expedited data service
in the TP without a complex buffer management scheme.
Hence, currently only a single connection is employed.
Owing to the properties of the TCP,
the expedited data semantics of the TP are handled properly.
\end{itemize}

Although all of these changes were rather small,
regretably version~2 of the protocol is incompatible with the original
protocol found in RFC983.
