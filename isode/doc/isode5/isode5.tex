% -*- LaTeX -*-		(really SLiTeX)

\documentstyle[blackandwhite,landscape,oval,pagenumbers,small]{NRslides}

\font\xx=cmbx10
\font\yy=cmbx7

\raggedright

\input trademark
\let\tradeNAMfont=\relax
\let\tradeORGfont=\relax

\begin{document}

\title	{ISODE 5.0:\\ OPENLY AVAILABLE OSI}
\author	{Marshall T.~Rose\\ NYSERNet, Inc.}
\date	{May 14, 1989}
\maketitlepage


\begin{bwslide}
\part*	{AGENDA}\bf

\begin{description}
\item[PART I:]		CURRENT DISTRIBUTION

\item[PART II:]		WHAT'S PLANNED
\end{description}
\end{bwslide}


\begin{bwslide}
\ctitle	{WHAT IS ISODE?}

\begin{nrtc}
\item	THE ISO DEVELOPMENT ENVIRONMENT

\item	AN OPENLY AVAILABLE IMPLEMENATION OF THE UPPER LAYERS OF OSI?

\item	A BASIS FOR THE TRANSITION TO OSI?

\item	A PLAYGROUND FOR ``THE PIED-PIPER OF OSI''?
\end{nrtc}
\end{bwslide}


\begin{bwslide}
\part	{CURRENT DISTRIBUTION}\bf

\begin{nrtc}
\item	STATUS: OPENLY AVAILABLE UNDER AN IMPLICIT ``HOLD HARMLESS'' CLAUSE

\item	CURRENT RELEASE: 5.0
    \begin{nrtc}
    \item	AVAILABLE MARCH 28, 1989
    \end{nrtc}

\item	SOURCE SIZE: \~{}200K LINES OF C AND ASN.1
\end{nrtc}
\end{bwslide}


\begin{bwslide}
\ctitle	{CURRENT DISTRIBUTION (cont.)}

\begin{nrtc}
\item	DISTRIBUTION EITHER VIA POSTAL MAIL OR ARPAnet FTP
    \begin{nrtc}
    \item	SOURCE: \~{}10.5MB

    \item	DOC: 5~VOLUME USER'S MANUAL (\~{}900~PAGES)

    \item	DISTRIBUTION SITES: US, UK, NL, AND AU

    \item	PRICE: \~{}365~US DOLLARS
    \end{nrtc}
\end{nrtc}
\end{bwslide}


\begin{bwslide}
\ctitle	{NORTH AMERICA DISTRIBUTION}\small

\[\begin{tabular}{rl}
Postal address:&UNIVERSITY OF PENNSYLVANIA\\
&		DEPARTMENT OF COMPUTER AND INFORMATION SCIENCE\\
&		MOORE SCHOOL\\
&		ATTN: DAVID J. FARBER (ISODE DISTRIBUTION)\\
&		200 SOUTH 33RD STREET\\
&		PHILADELPHIA, PA 19104-6314\\
&		USA\\[0.2in]
Telephone:&	+1--215--898--8560\\[0.2in]
Price:&		US\$365.00 (CHECKS ONLY)
\end{tabular}\]
\end{bwslide}


\begin{bwslide}
\ctitle	{LANGUAGES AND OPERATING SYSTEMS}

\begin{nrtc}
\item	CODED ENTIRELY IN C FOR \unix/
    \begin{nrtc}
    \item	REQUIRES NO KERNEL MODIFICATIONS    
    \end{nrtc}

\item	KNOWN PORTS FOR BERKELEY \unix/ (4.2 and 4.3):
    \begin{nrtc}
    \item	VAXen, SUNs, Pyramids, RTs, etc.
    \end{nrtc}

\item	KNOWN PORTS FOR AT\&T \unix/ (SVR2 and SVR3):
    \begin{nrtc}
    \item	SGI, 3Bs, 386s, RT (AIX)
    \end{nrtc}
\end{nrtc}
\end{bwslide}


\begin{bwslide}
\part*	{APPLICATION ARCHITECTURE}\bf

\begin{nrtc}
\item	A (NEARLY) COMPLETE IMPLEMENTATION OF THE UPPER LAYERS

\item	CURRENTLY IS LEVEL (FINALLY!)

\item	ALIGNED WITH THE U.S.~GOSIP
\end{nrtc}
\end{bwslide}


\begin{bwslide}
\ctitle	{THE APPLICATION ENVIRONMENT}

\vskip.5in
\diagram[p]{figure9}
\end{bwslide}


\begin{bwslide}
\ctitle	{AN ALTERNATE ENVIRONMENT:\\ MHS ARCHITECTURE (c.~1984)}

\vskip.5in
\diagram[p]{figure10}
\end{bwslide}


\begin{bwslide}
\ctitle	{THE TRANSPORT SWITCH}

\begin{nrtc}
\item	DECIDES WHICH TS-STACK TO USE FOR A CONNECTION

\item	FOR TP0:
    \begin{nrtc}
    \item	TCP (SOCKETS)

    \item	X.25 (SEVERAL INTERFACES, MOSTLY SOCKETS)
    \end{nrtc}

\item	FOR TP4:
    \begin{nrtc}
    \item	TWG's PROPRIETARY WIN/LLS (TLI)

    \item	SunLink OSI (EVENT SOCKETS)
    \end{nrtc}

\item	EXPERIENCE SHOWS IT IS FAIRLY EASY TO ADD A NEW TS-STACK TO THE SWITCH
\end{nrtc}
\end{bwslide}


\begin{bwslide}
\part*	{THE APPLICATIONS COOKBOOK}\bf

\begin{nrtc}
\item	TOOLS TO FACILITATE DEVELOPMENT OF APPLICATIONS ARE CRITICAL

\item	IDEA IS TO DEVELOP TOOLS TO AUTOMATE USE OF OSI REMOTE OPERATIONS
	SERVICE AS A GENERAL REMOTE PROCEDURE CALL FACILITY

\item	FOR MORE DETAILS:
\begin{quote}
BUILDING DISTRIBUTED APPLICATIONS IN AN OSI FRAMEWORK
\end{quote}
APPEARING IN ConneXions, MARCH, 1988
\end{nrtc}
\end{bwslide}


\begin{bwslide}
\ctitle	{REMOTE OPERATIONS SERVICE (ROS)}

\begin{nrtc}
\item	STANDARDIZED MECHANISM FOR SPECIFYING TRANSACTIONS

\item	EMPLOYS POWER OF ASN.1

\item	USED IN MANY INTERESTING OSI APPLICATIONS
    \begin{nrtc}
    \item	MESSAGE HANDLING SYSTEMS

    \item	DIRECTORY SERVICES

    \item	NETWORK MANAGEMENT

    \item	REMOTE DATABASE ACCESS
    \end{nrtc}

\item	CURRENTLY CONNECTION-ORIENTED, BUT CONNECTIONLESS-MODE IS UNDER STUDY
\end{nrtc}
\end{bwslide}


\begin{bwslide}
\ctitle	{GENERAL ORGANIZATION}

\begin{nrtc}
\item	AT COMPILE-TIME:
    \begin{nrtc}
    \item	USE RO-SPECIFICATION TO GENERATE SUPPORT FACILITIES
    \end{nrtc}

\item	AT RUN-TIME:
    \begin{nrtc}
    \item	USE DIRECTORY SERVICES TO LOCATE/REGISTER NETWORK SERVICES
		(NEARLY THERE!)

    \item	USE ASSOCIATION CONTROL TO BIND/UNBIND APPLICATIONS

    \item	USE REMOTE OPERATIONS TO INVOKE TRANSACTIONS
    \end{nrtc}
\end{nrtc}
\end{bwslide}


\begin{bwslide}
\ctitle	{STATIC (COMPILE-TIME) ORGANIZATION}

\vskip.15in
\diagram[p]{figure11}
\end{bwslide}


\begin{bwslide}
\ctitle	{DYNAMIC (RUN-TIME) ORGANIZATION}

\vskip.15in
\diagram[p]{figure12}
\end{bwslide}


\begin{bwslide}
\ctitle	{PERFORMANCE}

\begin{nrtc}
\item	USE OF ASN.1 TOOLS LEADS TO LARGE PROCESSES:
    \begin{nrtc}
    \item	MINIMUM 300KB, DSA: 500KB, FTAM: 800KB

    \item	INITIALIZATION SPEED SUFFERS MOST OWING TO PAGING IN
    \end{nrtc}

\item	A LOT OF TIME SPENT AVOIDING BYTE COPYING

\item	SPEEDS SLOWER (BUT COMPARITIVE) WHEN USING TCP

\item	X.25 TOO SLOW FOR COMPARISON
\end{nrtc}
\end{bwslide}


\begin{bwslide}
\part*	{APPLICATIONS}\bf

\begin{nrtc}
\item	EVERYTHING BUT MHS
\end{nrtc}
\end{bwslide}


\begin{bwslide}
\ctitle	{CURRENT APPLICATIONS}

\begin{nrtc}
\item	FILE TRANSFER, ACCESS AND MANAGEMENT (FTAM)

\item	FTAM-FTP GATEWAY

\item	DIRECTORY SERVICES (X.500)

\item	VIRTUAL TERMINAL

\item	ISODE MISCELLANY SERVICE
    \begin{nrtc}
    \item	e.g., FINGER, QUOTE-OF-THE-DAY, etc.
    \end{nrtc}

\item	PLUS NUMEROUS ``DEMO'' PROGRAMS
    \begin{nrtc}
    \item	e.g., IMAGE SERVICE, PASSWORD LOOKUP, IDIST, etc.
    \end{nrtc}
\end{nrtc}
\end{bwslide}


\begin{bwslide}
\ctitle	{DIRECTORY SERVICES}

\begin{nrtc}
\item	THE UCL DIRECTORY, QUIPU, HAS NOW COMPLETED ITS MAJOR DEVELOPMENT

\item	SEVERAL INTERESTING FEATURES:
    \begin{nrtc}
    \item	MEMORY, RATHER THAN DISK-BASED, ACCESS

    \item	INTERNAL SCHEDULING FOR MULTIPLE ACCESS

    \item	FLEXIBLE SEARCHING (SOUNDEX)

    \item	ACCESS CONTROL (NOT STANDARDIZED)
    \end{nrtc}

\item	ALREADY INTEROPERABILITY TESTED AGAINST TWO (EMBRYONIC) IMPLEMENATIONS
\end{nrtc}
\end{bwslide}


\begin{bwslide}
\ctitle	{DIRECTORY SERVICES (cont.)}

\begin{nrtc}
\item	FOR NAME/ADDRESS RESOLUTION, ISODE USES A 
    \begin{nrtc}
    \item	``HIGHER-PERFORMANCE'' NAMESERVICE
    \end{nrtc}
    BUILT ON TOP OF QUIPU SINCE
    \begin{nrtc}
    \item	CONNECTION-ORIENTED OVERHEAD AND

    \item	PROTOCOL COMPLEXITY
    \end{nrtc}
    ARE TOO HIGH FOR THE ``SIMPLE'' FUNCTIONALITY NEEDED BY MOST APPLICATIONS

\item	AT WOLLONGONG, CHRIS MOORE IS HOSTING A PILOT PROJECT TO
	ACCELERATE DIRECTORY IMPLEMENTATION AND TESTING IN THE US
    \begin{nrtc}
    \item	ALSO, SITES IN THE UK AND AU ARE PARTICIPATING
    \end{nrtc}

\item	NEXT ROUND OF DEVELOPMENT FOCUSES ON USER INTERFACES, DISTRIBUTION
\end{nrtc}
\end{bwslide}


\begin{bwslide}
\ctitle	{VIRTUAL TERMINAL}

\begin{nrtc}
\item	MITRE HAS DEVELOPED A DIS VT IMPLEMENTATION

\item	ROUGHLY EQUIVALENT TO BSD TELNET IN TERMS OF FUNCTIONALITY
    \begin{nrtc}
    \item	(BASIC CLASS, TELNET PROFILE)
    \end{nrtc}

\item	INTEROPERABILITY TESTED AGAINST THE BRIDGE/3COM VT

\item	MITRE IS WORKING ON A FORMS CLASS IMPLEMENTATION

\item	ULTIMATELY, MUST BE UPGRADED TO IS IMPLEMENTATION
\end{nrtc}
\end{bwslide}


\begin{bwslide}
\ctitle	{OTHER APPLICATIONS\\ (NOT A PART OF ISODE)}

\begin{nrtc}
\item	NETWORK MANAGEMENT

\item	ODA/ODIF

\item	MOBILE X.400 PILOT
    \begin{nrtc}
    \item	MS-DOS CLIENT SIDE ONLY PORT DONE BY HP
    \end{nrtc}
\end{nrtc}
\end{bwslide}


\begin{bwslide}
\part	{WHAT'S PLANNED}\bf

\begin{nrtc}
\item	MESSAGE HANDLING SYSTEMS

\item	WHITE PAGES PILOT

\item	INTEROPERABILITY TESTING

\item	OSI-POSIX PROJECT
\end{nrtc}
\end{bwslide}


\begin{bwslide}
\part*	{MESSAGE HANDLING SYSTEMS}

\begin{nrtc}
\item	UCL AND UNott ARE DEVELOPING AN X.400 TRANSPORT SYSTEM (PP)

\item	USE EXPERIENCE GAINED FROM NUMEROUS SOPHISTICATED TEXT-BASED MESSAGE
	TRANSFER SYSTEMS

\item	OWES MANY OF ITS DESIGN IDEAS TO THE UNIVERSITY OF DELAWARE MESSAGE
	SYSTEM, MMDF

\item	WILL UTILIZE DIRECTORY SERVICES
\end{nrtc}
\end{bwslide}


\begin{bwslide}
\ctitle	{INTERESTING FEATURES}

\begin{nrtc}
\item	SUPPORT FOR A WIDE RANGE OF ENCODED INFORMATION TYPES 
    \begin{nrtc}
    \item	AND REFORMATTING BETWEEN THEM
    \end{nrtc}

\item	SUPPORT FOR DIFFERENT MESSAGE TRANSPORT PROTOCOLS
    \begin{nrtc}
    \item	AND CONVERSION BETWEEN THEM
    \end{nrtc}
    e.g., INCLUDES RFC987 (X.400 TO 821/822)

\item	ROBUSTNESS FOR USE IN LARGE SCALE SERVICE ENVIRONMENTS
\end{nrtc}
\end{bwslide}


\begin{bwslide}
\ctitle	{MAJOR GOALS}

\begin{nrtc}
\item	FULL X.400(84/88) SUPPORT, EXCEPT FOR X.400(88) SECURITY SERVICES

\item	PROVIDES A ``CLEAN'' INTERFACE FOR MESSAGE SUBMISSION AND DELIVERY
    \begin{nrtc}
    \item	TO SUPPORT A WIDE RANGE OF USER AGENTS,

    \item	AND APPLICATIONS OTHER THAN INTERPERSONAL MESSAGING
    \end{nrtc}

\item	QUEUE MANAGEMENT DONE VIA A ROS-BASED PROTOCOL
    \begin{nrtc}
    \item	SOPHISTICATED SCHEDULING OF MESSAGE DELIVERY

    \item	LOCAL AND REMOTE MONITORING FOR MANAGERS AND USERS

    \item	ROBUSTNESS REQUIRED TO SUPPORT HIGH LEVELS OF TRAFFIC

    \item	SUPPORT FOR ADMINISTRATIVE POLICIES ON SUBMISSION
    \end{nrtc}

\item	LIST EXPLODER AND LIST MANAGMENT    
\end{nrtc}
\end{bwslide}


\begin{bwslide}
\ctitle	{OTHER THINGS}

\begin{nrtc}
\item	TWO USER INTERFACES PLANNED
    \begin{nrtc}
    \item	MH INTERFACE

    \item	WINDOW-BASED INTERFACE
    \end{nrtc}

\item	INTEGRATION OF FAX PLANNED
\end{nrtc}
\end{bwslide}


\begin{bwslide}
\part*	{WHITE PAGES PILOT}\bf

\begin{nrtc}
\item	A GENERAL WHITE PAGES SERVICE IS NEEDED FOR THE INTERNET

\item	EXISTING WHOIS FACILITY, WHILE USEFUL, SUFFERS FROM TWO PROBLEMS
    \begin{nrtc}
    \item	CENTRALIZED DATABASE

    \item	LIMITED KINDS OF INFORMATION
    \end{nrtc}
\end{nrtc}
\end{bwslide}


\begin{bwslide}
\ctitle	{OSI DIRECTORY BASED PILOT PROJECT}

\begin{nrtc}
\item	A ``GRASS ROOTS'' EFFORT TO PROVIDE A WHITE PAGES SERVICE

\item	SEMI-PUBLIC INFORMATION (TELEPHONE/MAIL BOOKS)

\item	MOST SITES RESPONSIBLE FOR MAINTAINING INFORMATION

\item	ACCESS VIA NETWORK AND DIALUP LOGINS

\item	CRT- AND WINDOW-BASED INTERFACES PLANNED
\end{nrtc}
\end{bwslide}


\begin{bwslide}
\part*	{INTEROPERABILITY TESTING}\bf

\begin{nrtc}
\item	THERE IS NO SUBSTITUTE FOR PAIRWISE INTEROPERABILITY TESTING

\item	LET GROUPS SUCH AS OSInet, EUROSInet, etc., CONNECT TO HOSTS
	RUNNING X.25 AND ISODE

\item	TESTING IS UNATTENDED UNLESS PROBLEMS ARE WITH THE ISODE (gasp!)
\end{nrtc}
\end{bwslide}


\begin{bwslide}
\part*	{OSI-POSIX PROJECT}\bf

\begin{nrtc}
\item	GOAL: ACCELLERATE THE UBIQUITY OF OSI

\item	APPROACH: OPENLY AVAILABLE, COMPLETE OSI IMPLEMENTATION FOR NEXT MAJOR
	RELEASE OF BERKELEY \unix/

\item	FOR MORE DETAILS:
\begin{quote}
OSI PROTOCOLS WITHIN AN OPENLY AVAILABLE, POSIX-CONFORMANT, BERKELEY UNIX
ENVIRONMENT
\end{quote}
APPEARING IN ConneXions, OCTOBER, 1988
\end{nrtc}
\end{bwslide}


\begin{bwslide}
\diagram[p]{figure13}
\end{bwslide}


\begin{bwslide}
\diagram[p]{figure14}
\end{bwslide}


\end{document}
