% run this through SLiTeX with the appropriate wrapper

\dotopic	{UPPER LAYER INFRASTRUCTURE}

\begin{bwslide}
\part*	{OUTLINE}\bf

\begin{description}
\item[PART I:]	APPLICATION BINDING

\item[PART II:]	PRESENTATION ABSTRACTIONS

\item[PART III:] DEFINING A NEW SERVICE
\end{description}
\end{bwslide}


\begin{bwslide}
\part	{APPLICATION BINDING}\bf

\begin{nrtc}
\item	AN ASSOCIATION \emph{BINDS} AN INITIATOR TO A RESPONDER
    \begin{nrtc}
    \item	TYPICALLY FOR A CONSUMER/PROVIDER MODEL
    \end{nrtc}

\item	BINDING IS TWO-STEP:
    \begin{nrtc}
    \item	MAP SERVICE ONTO AVAILABLE ENTITIES
	\begin{nrtc}
	\item	 (VIA DIRECTORY SERVICE)
	\end{nrtc}

    \item	SELECT ENTITY BASED ON COMMUNICATIONS REQUIREMENTS
	\begin{nrtc}
	\item	 (VIA TRANSPORT SERVICE)
	\end{nrtc}
    \end{nrtc}
\end{nrtc}
\end{bwslide}


\begin{bwslide}
\ctitle	{MAPPING SERVICES ONTO ENTITIES}

\begin{nrtc}
\item	VIA OSI DIRECTORY USING USER-FRIENDLY NAMESERVICE
    \begin{nrtc}
    \item	(THERE ARE MANY OTHER WAYS)
    \end{nrtc}
\end{nrtc}
\begin{quote}\small\begin{verbatim}
AEI    aei;
struct PSAPaddr *pa;

char *string = user_typein;
char *context = "iso ftam";
int   interactive;
char *userdn = "";
char *passwd = "";

if ((aei = str2aeinfo (string, context, interactive,
                       userdn, passwd)) == NULLAEI)
    error ("unable to resolve service: %s", PY_pepy);

if ((pa = aei2addr (aei)) == NULLPA)
    error ("address translation failed");
\end{verbatim}\end{quote}
\end{bwslide}


\begin{bwslide}
\ctitle	{APPLICATION ENTITY INFORMATION}

\begin{quote}\small\begin{verbatim}
typedef struct AEInfo {
    PE      aei_ap_title;
    PE      aei_ae_qualifier;

    int     aei_ap_id;
    int     aei_ae_id;

    int     aei_flags;
#define AEI_NULL        0x00
#define AEI_AP_ID       0x01
#define AEI_AE_ID       0x02
}       AEInfo, *AEI;
#define NULLAEI         ((AEI) 0)

#define AEIFREE(aei)     ...

AEI     _str2aei ();
char   *sprintaei ();
\end{verbatim}\end{quote}
\end{bwslide}


\begin{bwslide}
\ctitle	{APPLICATION ENTITY INFORMATION (cont.)}

\begin{nrtc}
\item	SEE IF TAILORING VARIABLE \verb"ns_enable" IS ON
    \begin{nrtc}
    \item	IF SO, USE DIRECTORY TO RESOLVE NAME
    \end{nrtc}
    ON FAILURE, CONTINUE

\item	SEE IF COMPILE-TIME OPTION \verb"NOSTUB" IS OFF
    \begin{nrtc}
    \item	IF SO, USE STUB DIRECTORY TO RESOLVE NAME
    \end{nrtc}
\end{nrtc}
\end{bwslide}


\begin{bwslide}
\ctitle	{OSI DIRECTORY SUPPORT FOR NAMING}

\begin{nrtc}
\item	A DISTINGUISHED NAME (DN) IS AN ASN.1 OBJECT CONSISTING OF
    \begin{nrtc}
    \item	A SEQUENCE OF ATTRIBUTE/VALUE PAIRS
    \end{nrtc}
    IMPLYING A SUPERIOR/SUBORDINATE RELATIONSHIP, e.g.,
\begin{quote}\small\begin{verbatim}
countryName            = US
organizationName       = Performance Systems International
organizationalUnitName = Operations
commonName             = nisc
commonName             = filestore
\end{verbatim}\end{quote}
REFERS TO AN INFORMATION OBJECT IN THE OSI DIRECTORY

\item	SO FIRST STEP IS TO FIND DISTINGUISHED NAME
    \begin{nrtc}
    \item	USE THIS AS AP-TITLE
    \end{nrtc}
\end{nrtc}
\end{bwslide}


\begin{bwslide}
\ctitle	{USER-FRIENDLY NAMESERVICE}

\begin{nrtc}
\item	NAMES ARE ORDERED, UNTYPED, AND (POSSIBLY) INCOMPLETE, e.g.,
\begin{quote}\small\begin{verbatim}
% ftam "nisc, psi, us"

% ftam "nisc, operations, psi, us"
\end{verbatim}\end{quote}
	MIGHT RESOLVE TO THE SAME APPLICATION PROCESS

\item	ALGORITHM USES IMPRECISE MATCHING AND ASSIGNS ``GOODNESS'' LEVEL TO
	MATCHES

\item	USERS ARE QUERIED FOR ASSISTANCE ON QUESTIONABLE MATCHES
    \begin{nrtc}
    \item	(IF APPLICATION IS INTERACTIVE)
    \end{nrtc}
\end{nrtc}
\end{bwslide}


\begin{bwslide}
\ctitle	{USER-FRIENDLY NAMESERVICE (cont.)}

\begin{nrtc}
\item	ENTRIES IN \verb"isoaliases" FILE ARE USER-FRIENDLY STRINGS,
	e.g.,
\begin{quote}\small\begin{verbatim}
psi    "performance systems international, us"
nisc   "nisc, operations, psi, us"
\end{verbatim}\end{quote}

\item	ALGORITHM IS USED TO FIND ENTRIES WITH APPROPRIATE
	\verb"supportedApplicationContext", e.g.,
\begin{quote}\small\begin{verbatim}
1.0.8571.1. -- "iso ftam"
\end{verbatim}\end{quote}
	SO
\begin{quote}\small\begin{verbatim}
ftam "nisc, psi, us"
\end{verbatim}\end{quote}
	MIGHT YIELD AN APPLICATION ENTITY OF
\begin{quote}\smaller\begin{verbatim}
c=US
    @o=Performance Systems International@ou=Operations
    @cn=nisc@cn=filestore
\end{verbatim}\end{quote}

\item	NEXT STEP IS TO PERFORM DAP READ OF \verb"presentationAddress"

\item	IF READ FAILS, THEN CALL TO \verb"aei2addr" WILL FAIL
\end{nrtc}
\end{bwslide}


\begin{bwslide}
\ctitle {SPLIT-MODEL FOR NAMESERVICE}

\begin{nrtc}
\item	THE DEFAULT IS TO TALK TO A PROGRAM CONTAINING THE UFN

\item	SET TAILORING VARIABLE \verb"ns_address" TO ADDRESS OF PROGRAM,
	e.g.,
\begin{quote}\small\begin{verbatim}
localHost=17006
\end{verbatim}\end{quote}

\item	THIS IS STRICTLY A LOCAL ISSUE
\end{nrtc}
\vskip.5in
\diagram[p]{figureH-3}
\end{bwslide}


\begin{bwslide}
\part	{PRESENTATION ABSTRACTIONS}\bf

\begin{nrtc}
\item	PRESENTATION ELEMENTS

\item	PRESENTATION STREAMS
\end{nrtc}
\end{bwslide}


\begin{bwslide}
\part*	{PRESENTATION ELEMENTS}\bf

\begin{nrtc}
\item	AN INTERNAL FORM FOR AN INSTANCE OF A TYPE DESCRIBED BY ABSTRACT
	SYNTAX

\item	CAN REPRESENT ANY ASN.1 TYPE AS EITHER
    \begin{nrtc}
    \item	A STRING OF OCTETS OR BITS

    \item	A LINKED-LIST OF PRESENTATION ELEMENTS
    \end{nrtc}

\item	VIEWED AT TWO LEVELS:
    \begin{nrtc}
    \item	RAW (INTERNAL)

    \item	COOKED (OPAQUE)
    \end{nrtc}
\end{nrtc}
\end{bwslide}


\begin{bwslide}
\ctitle	{CONVERSION ROUTINES}

\begin{nrtc}
\item	DEFINE C STRUCTURE EQUIVALENT TO ASN.1 TYPE:
\begin{quote}\small\begin{verbatim}
typedef int integer;
\end{verbatim}\end{quote}

\item	STRUCTURE TO PE:
\begin{quote}\small\begin{verbatim}
PE      num2prim (i, class, id);
\end{verbatim}\end{quote}
\verb"NULLPE" RETURNED ON ERROR

\item	PE TO STRUCTURE:
\begin{quote}\small\begin{verbatim}
integer prim2num (pe);
\end{verbatim}\end{quote}
DISTINGUISHED VALUE RETURNED ON ERROR
    \begin{nrtc}
    \item	(CHECK \verb"pe_errno" FOR REASON)
    \end{nrtc}
\end{nrtc}
\end{bwslide}


\begin{bwslide}
\ctitle	{UTILITY ROUTINES}

\begin{quote}\small\begin{verbatim}
PE      pe_alloc (class, form, id);
int     pe_free (pe);

int     pe_cmp (p, q);
PE      pe_cpy (pe);

int     pe_pullup (pe);      /* for experts */

PE      pe_expunge (pe, r);  /*   .. */
int     pe_extract (pe, r);  /*   .. */
\end{verbatim}\end{quote}
\end{bwslide}


\begin{bwslide}
\ctitle	{QBUFs}

\begin{quote}\small\begin{verbatim}
struct qbuf {
    struct qbuf *qb_forw;       /* doubly-linked list */
    struct qbuf *qb_back;       /*   .. */

    int     qb_len;             /* length of data */
    char   *qb_data;            /* current pointer into data */
    char    qb_base[1];         /* extensible... */
};


PE      qb2prim_aux (qb, class, id, in_line);
struct qbuf *prim2qb ();

char   *qb2str (q);
struct qbuf *str2qb (s, len, head);

int     qb_pullup (qb);
\end{verbatim}\end{quote}
\end{bwslide}


\begin{bwslide}
\ctitle	{OBJECT IDENTIFIERS}

\begin{quote}\small\begin{verbatim}
typedef struct OIDentifier {
    int     oid_nelem;  /* number of sub-identifiers */

    unsigned int *oid_elements;
              /* the (ordered) list of sub-identifiers */
}                       OIDentifier, *OID;
#define NULLOID ((OID) 0)

PE      obj2prim (o, class, id);
OID     prim2oid (pe);
\end{verbatim}\end{quote}
\end{bwslide}


\begin{bwslide}
\ctitle	{OBJECT IDENTIFIERS (cont.)}

\begin{quote}\small\begin{verbatim}
int     oid_cmp (p, q);
OID     oid_cpy (p);
OID     oid_free (oid);

char   *sprintoid (p);
OID     str2oid (s);
\end{verbatim}\end{quote}
\end{bwslide}


\begin{bwslide}
\part*	{PRESENTATION STREAMS}\bf

\begin{nrtc}
\item	MAPS PEs TO/FROM DIFFERENT INPUT-OUTPUT ABSTRACTIONS, e.g.,
    \begin{nrtc}
    \item	in-core buffers;

    \item	\unix/ FILES;

    \item	OSI services (e.g., SESSION);

    \item	non-OSI services (e.g., UDP).
    \end{nrtc}

\item	TWO PARTS:
    \begin{nrtc}
    \item	A UNIFORM FRONT-END

    \item	ONE OF SEVERAL DOMAIN-SPECIFIC BACK-ENDs
    \end{nrtc}
\end{nrtc}
\end{bwslide}


\begin{bwslide}
\ctitle	{FRONT-END}

\begin{quote}\small\begin{verbatim}
PS      ps_alloc (io);
void    ps_free (ps);

int     ps_flush (ps);
int     ps_prime (ps);

PE      ps2pe (ps);
int     pe2ps (ps, pe);

PE      pl2pe (ps);
int     pe2pl (ps, pe);

int     ps_get_abs (pe);
\end{verbatim}\end{quote}
\end{bwslide}


\begin{bwslide}
\ctitle	{BACK-ENDs}

\begin{nrtc}
\item   STDIO:
\begin{quote}\small\begin{verbatim}
int     std_open (ps);
int     std_setup (ps, fp);
\end{verbatim}\end{quote}

\item   STRINGs:
\begin{quote}\small\begin{verbatim}
int     str_open (ps);
int     str_setup (ps, cp, cc, in_line);
\end{verbatim}\end{quote}

\item   DATAGRAMs:
\begin{quote}\small\begin{verbatim}
int     dg_open (ps);
int     dg_setup (ps, fd, size, rfx, wfx);
\end{verbatim}\end{quote}

\item   FULL-DUPLEX FDs:
\begin{quote}\small\begin{verbatim}
int     fdx_open (ps);
int     fdx_setup (ps, fd);
\end{verbatim}\end{quote}

\item   QBUFs (INPUT ONLY):
\begin{quote}\small\begin{verbatim}
int     qbuf_open (ps);
int     qbuf_setup (ps, qb);
\end{verbatim}\end{quote}

\item   UDVECs (OUTPUT ONLY):
\begin{quote}\small\begin{verbatim}
int     uvec_open (ps);
int     uvec_setup (ps, len);
\end{verbatim}\end{quote}
\end{nrtc}
\end{bwslide}


\begin{bwslide}
\part	{DEFINING A NEW SERVICE}\bf

\begin{nrtc}
\item	THREE THINGS TO BE DEFINED:
    \begin{nrtc}
    \item	APPLICATION CONTEXT

    \item	ABSTRACT SYNTAX

    \item	DIRECTORY ENTRY
    \end{nrtc}
\end{nrtc}
\vskip.15in
\diagram[p]{figureU-1}
\end{bwslide}


\begin{bwslide}
\ctitle	{APPLICATION CONTEXT/ABSTRACT SYNTAX}

\begin{nrtc}
\item	APPLICATION CONTEXT
    \begin{nrtc}
    \item	DESCRIBES THE RELATIONSHIP BETWEEN SERVICE ELEMENTS
    \end{nrtc}

\item	ABSTRACT SYNTAX
    \begin{nrtc}
    \item	DESCRIBES THE DATA STRUCTURES BEING EXCHANGED BY THE SERVICE
    \end{nrtc}

\item	DEFINED IN THE \verb"isobjects" FILE:
\begin{verbatim}
"iso ftam"                 1.0.8571.1
"ftam pci"                 1.0.8571.2.1
\end{verbatim}

\item	SYMBOLIC NAME USED BY PROGRAM, OID SENT VIA PROTOCOL
\end{nrtc}
\end{bwslide}


\begin{bwslide}
\ctitle	{DIRECTORY ENTRY}

\begin{quote}\small\begin{verbatim}
c=US
    @o=Performance Systems International@ou=Operations
    @cn=nisc@cn=filestore

    objectClass= top & applicationEntity & iSODEApplicationEntity &\
                        quipuObject
    presentationAddress= "ftam"/Internet=nisc.psi.net
    supportedApplicationContext= "iso ftam"
    execVector= "iso.ftam" -c

    description=
    locality=
    organizationName=
    organizationalUnitName=
    seeAlso=
\end{verbatim}\end{quote}
\end{bwslide}


\begin{bwslide}
\ctitle	{DIRECTORY ENTRY\\ (STATIC SERVERS)}

\begin{nrtc}
\item	DON'T USE \verb"iSODEApplicationEntity"

\item	MAKE SURE ADDRESS DOESN'T CONFLICT WITH \verb"iaed"
\end{nrtc}
\begin{quote}\small\begin{verbatim}
c=US
    @o=Performance Systems International@ou=Operations
    @cn=nisc@cn=pp qmgr

    objectClass= top & applicationEntity & quipuObject
    presentationAddress= "qmgr"/Internet=nisc.psi.net+2000
\end{verbatim}\end{quote}
\end{bwslide}


\begin{bwslide}
\ctitle	{FOR FURTHER READING}

\begin{nrtc}
\item	Volume 1: Application Services
\end{nrtc}
\end{bwslide}
