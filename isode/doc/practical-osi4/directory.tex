% run this through LaTeX with the appropriate wrapper

\dotopic{DIRECTORY SERVICES}


\begin{bwslide}
\part*	{OUTLINE}\bf

\begin{description}
\item[PART I:]		INFORMATION FRAMEWORK

\item[PART II:]		DISTRIBUTED OPERATIONS
\end{description}
\end{bwslide}


\begin{bwslide}
\ctitle	{ACKNOWLEDGEMENTS}

\begin{nrtc}
\item	ALL OF THESE OBSERVATIONS COME FROM RUNNING A WHITE PAGES PILOT

\item	STEPHEN E.~KILLE (UCL) AND COLIN J.~ROBBINS (X-TEL~LTD.)
	DESERVE A LOT OF CREDIT FOR THEIR WORK IN DIRECTORY

\item	ANTHONY E.~HODSON (ICL) AND OTHERS DESERVE A LOT OF CREDIT FOR
	THE WORK IN SHARED NAMESPACES
\end{nrtc}
\end{bwslide}


\begin{bwslide}
\ctitle	{INTRODUCTION TO THE DIRECTORY}

\begin{nrtc}
\item	A VERY BRIEF OVERVIEW TO ``LEVEL THE FIELD''

\item	THE DIRECTORY MANAGES \emph{INFORMATION OBJECTS} MODELED AS
	\emph{ENTRIES} IN THE DIRECTORY INFORMATION BASE (DIB)
\end{nrtc}
\end{bwslide}


\begin{bwslide}
\ctitle	{DIRECTORY CONCEPTS:\\ ENTRIES}

\vskip.5in
\diagram[p]{figureD-1}
\end{bwslide}


\begin{bwslide}
\ctitle	{DIRECTORY CONCEPT: SCHEMA}

\begin{nrtc}
\item	THE \verb"objectClass" ATTRIBUTE DETERMINES
    \begin{nrtc}
    \item	WHICH ATTRIBUTES MUST BE PRESENT

    \item	WHICH ATTRIBUTES MAY BE PRESENT
    \end{nrtc}

\item	ALL ENTRIES MUST BELONG TO THE \verb"top" OBJECT CLASS

\item	OBJECT CLASS PROPERTIES ARE INHERITED, e.g.,
    \begin{nrtc}
    \item	\verb"residentialPerson" IS SUBORDINATE TO \verb"person"
    \end{nrtc}
\end{nrtc}
\end{bwslide}


\begin{bwslide}
\ctitle	{DIRECTORY CONCEPTS: DIT}

\vskip.5in
\diagram[p]{figureD-2}
\end{bwslide}


\begin{bwslide}
\ctitle	{DIRECTORY CONCEPTS: DUAs, DSAs}

\vskip.5in
\diagram[p]{figureD-3}
\end{bwslide}


\begin{bwslide}
\ctitle	{DIRECTORY CONCEPTS: OPERATIONS}

\begin{nrtc}
\item	INTERROGATION:
    \begin{nrtc}
    \item	READ, COMPARE, LIST, SEARCH

    \item	ABANDON
    \end{nrtc}

\item	MODIFICATION:
    \begin{nrtc}
    \item	ADD ENTRY, REMOVE ENTRY, MODIFY ENTRY

    \item	MODIFY RDN
    \end{nrtc}
\end{nrtc}
\end{bwslide}


\begin{bwslide}
\part	{INFORMATION FRAMEWORK}\bf

\begin{nrtc}
\item	NAMING ARCHITECTURE

\item	INTERROGATION ALGORITHMS
\end{nrtc}
\end{bwslide}


\begin{bwslide}
\part*	{NAMING ARCHITECTURE}\bf

\begin{nrtc}
\item	NAMING IS HARD TO DO

\item	DIRECTORY STANDARDS ARE LESS THAN HELPFUL IN THIS REGARD
\end{nrtc}
\end{bwslide}


\begin{bwslide}
\ctitle	{PROBLEMS WITH NAMING}

\begin{nrtc}
\item	TOO MUCH FLEXIBILITY IN DIT STRUCTURE
    \begin{nrtc}
    \item	(DIFFICULT FOR INTERROGATION ALGORITHMS)
    \end{nrtc}

\item	NO NAMING AUTHORITIES
    \begin{nrtc}
    \item	(PART OF A MORE GENERAL PROBLEM)
    \end{nrtc}

\item	ADDMDs WANT TO MINIMIZE SHARING OF INFORMATION

\item	CAN NOT LEGISLATE HOW PRDMDs ARE ORGANIZED

\item	EACH COUNTRY MUST MAKE ``NATIONAL DECISION''
\end{nrtc}
\end{bwslide}


\begin{bwslide}
\ctitle	{ANNEX B OF ISO9594-7/X.521(88)}

\vskip.5in
\diagram[p]{figureD-33}
\end{bwslide}


\begin{bwslide}
\ctitle	{PROGRESS}

\begin{nrtc}
\item	SOME GROUPS HAVE BEEN (INDEPENDENTLY) WORKING ON NAMING ARCHITECTURES
    \begin{nrtc}
    \item	THORN (NOW COSINE/INTERNET)

    \item	EUROPEAN WORKSHOP FOR OPEN SYSTEMS (EWOS)

    \item	NORTH AMERICAN DIRECTORY FORUM (NADF)
    \end{nrtc}

\item	HOW MIGHT THE HYPOTHETICAL COUNTRY OF FREEDONIA (FF)
	DECIDE THIS:
    \begin{nrtc}
    \item	USE CIVIL AUTHORITY FOR NAMING

    \item	TREAT PRDMDs AS BLACK BOXES
    \end{nrtc}
\end{nrtc}
\end{bwslide}


\begin{bwslide}
\ctitle	{NAMING UNIVERSES}

\begin{nrtc}
\item	SEVERAL POSSIBLE NAMING UNIVERSES COULD BE REALIZED IN THE DIT:
    \begin{nrtc}
    \item	GEOGRAPHICAL

    \item	COMMUNITY

    \item	POLITICAL
    \end{nrtc}

\item	CHOICE OF UNIVERSE DETERMINES DIFFICULTY OF MAPPING QUERIES

\item	SUPPORT OF MULTIPLE UNIVERSES IS LIKELY TO BE UNNATURAL

\item	CHOOSE CIVIL AUTHORITY
    \begin{nrtc}
    \item	MINIMIZES AMOUNT OF NEW REGISTRATION INFRASTRUCTURE
    \end{nrtc}
\end{nrtc}
\end{bwslide}


\begin{bwslide}
\ctitle	{CHOICE OF RDNs}

\begin{nrtc}
\item	PRIMARY GOAL:
    \begin{nrtc}
    \item	AVOID NAMING COLLISIONS
    \end{nrtc}

\item	SECONDARY GOAL:
    \begin{nrtc}
    \item	VALUES SHOULD BE MEANINGFUL, i.e.,
	\begin{nrtc}
	\item	``USER-FRIENDLY''
	\end{nrtc}
    \end{nrtc}

\item	OBSERVATION:
    \begin{nrtc}
    \item	GUESSING CAN BE DONE VIA AVs RATHER THAN RDNs
	\begin{nrtc}
	\item	(AN ENTRY CAN HAVE MANY AVs, INDEPENDENT OF D-AVs)
	\end{nrtc}

    \item	SO ``USER-FRIENDLY'' RDNs AREN'T CRITICAL
	\begin{nrtc}
	\item	IN SPITE OF ANNEX~E OF ISO9594--2/X.501(88)
	\end{nrtc}
    \end{nrtc}
\end{nrtc}
\end{bwslide}


\begin{bwslide}
\ctitle	{AN EXAMPLE}

\begin{nrtc}
\item	CONSIDER AN ENTRY FOR
\begin{quote}\begin{verbatim}
Ministry of Finance
\end{verbatim}\end{quote}

\item	(SUPERFICIALLY) WOULD WANT RDN OF
\begin{quote}\begin{verbatim}
o=Ministry of Finance
\end{verbatim}\end{quote}

\item	BUT WOULD ALSO WANT AV OF
\begin{quote}\begin{verbatim}
o=MOF
\end{verbatim}\end{quote}
TO AID SEARCHING

\item	SO, INSTEAD COULD HAVE AVs OF
\begin{quote}\begin{verbatim}
o=Ministry of Finance
o=MOF
\end{verbatim}\end{quote}
TO AID SEARCHING, AND SOMETHING ELSE FOR AN RDN
\end{nrtc}
\end{bwslide}


\begin{bwslide}
\ctitle	{NAMING AT THE NATIONAL LEVEL}

\begin{nrtc}
\item	PROVINCES
\begin{quote}\begin{verbatim}
ffProvince OBJECT-CLASS
    SUBCLASS OF locality
    MUST CONTAIN { fipsProvNumericCode, -- FF FIPS PUB
                   fipsProvAlphaCode,   -- FF FIPS PUB
                   stateOrProviceName }
    ::= { ffObjectClass 1 }
\end{verbatim}\end{quote}
RDN IS \verb"fipsProvNumericCode"

\item	ORGANIZATIONS WITH NATIONAL STANDING
\begin{quote}\begin{verbatim}
ffOrganization OBJECT-CLASS
    SUBCLASS OF organization
    MUST CONTAIN { fnsiOrgNumericCode }  -- FNSI registry
    ::= { ffObjectClass 4 }
\end{verbatim}\end{quote}
RDN IS \verb"fnsiOrgNumericCode"
\end{nrtc}
\end{bwslide}


\begin{bwslide}
\ctitle	{NAMING AT THE NATIONAL LEVEL (cont.)}

\begin{nrtc}
\item	ADDMD OPERATORS
\begin{quote}\begin{verbatim}
ffADDMD OBJECT-CLASS
    SUBCLASS OF top
    MUST CONTAIN { addmdName }           -- FF registry
    ::= { ffObjectClass 6 }
\end{verbatim}\end{quote}
RDN IS \verb"addmdName"
\end{nrtc}
\end{bwslide}


\begin{bwslide}
\ctitle	{NAMING WITHIN A PROVINCE}

\begin{nrtc}
\item	POPULATED PLACES
\begin{quote}\begin{verbatim}
ffPlace OBJECT-CLASS
    SUBCLASS OF locality
    MUST CONTAIN { fipsPlaceNumericCode, -- FF FIPS PUB
                   localityName }
    ::= { ffObjectClass 2 }
\end{verbatim}\end{quote}
RDN IS \verb"fipsPlaceNumericCode"

\item	ORGANIZATIONS WITH REGIONAL STANDING:
    \begin{nrtc}
    \item	\verb"organization"
    \end{nrtc}
RDN IS \verb"organizationName" -- REGISTRAR IS PROVINCIAL SECRETARY

\item	REGIONAL GOVERNMENT:
    \begin{nrtc}
    \item	\verb"organization"
    \end{nrtc}
RDN IS \verb"o=Government"
\end{nrtc}
\end{bwslide}


\begin{bwslide}
\ctitle	{NAMING WITHIN A POPULATED PLACE}

\begin{nrtc}
\item	CITIZENS:
    \begin{nrtc}
    \item	\verb"residentialPerson"
    \end{nrtc}
RDN IS \verb"commonName" (USUALLY WITH SOMETHING ELSE)

\item	ORGANIZATIONS WITH LOCAL STANDING:
    \begin{nrtc}
    \item	\verb"organization"
    \end{nrtc}
RDN IS \verb"organizationName" -- REGISTRAR IS COUNTY CLERK (OR EQUIVALENT)

\item	LOCAL GOVERNMENT:
    \begin{nrtc}
    \item	\verb"organization"
    \end{nrtc}
RDN IS \verb"o=Government"
\end{nrtc}
\end{bwslide}


\begin{bwslide}
\ctitle	{NAMING OF OSI ENTITIES}

\begin{nrtc}
\item	APPLICATION PROCESSES:
    \begin{nrtc}
    \item	\verb"applicationProcess"
    \end{nrtc}
RDN IS \verb"commonName"

\item	APPLICATION ENTITIES:
\begin{quote}\begin{verbatim}
ffApplicationEntity OBJECT-CLASS
    SUBCLASS OF applicationEntity
    MUST CONTAIN { supportedApplicationContext }
    ::= { ffObjectClass 5 }
\end{verbatim}\end{quote}
RDN IS \verb"commonName"
\end{nrtc}
\end{bwslide}


\begin{bwslide}
\ctitle	{NAMING WITHIN ORGANIZATIONS/ADDMDs}

\begin{nrtc}
\item	UP TO EACH ORGANIZATION/ADDMD

\item	MAY BE UNPUBLISHED
    \begin{nrtc}
    \item	(ALMOST CERTAINLY IN LATTER CASE)
    \end{nrtc}
\end{nrtc}
\end{bwslide}


\begin{bwslide}
\ctitle	{FREEDONIAN DIT STRUCTURE}
\font\twlsf=cmss10 scaled \magstep 1
\twlsf

\[\begin{tabular}{|r|c|l|c|l|}
\hline
\multicolumn{1}{|c|}{\bf Level}&
\multicolumn{1}{c|}{\bf Element}&
	\multicolumn{1}{c|}{\bf objectClass}&
			\multicolumn{1}{c|}{\bf Superior}&
				\multicolumn{1}{c|}{\bf RDN}\\
\hline
root&
\ 0&	&		&	\\
\hline
international&
\ 1&	country&	0&	countryName\\
\hline
national&
 \ 2&	ffProvince&	1&	fipsProvNumericCode\\
&\ 3&	ffOrganization&	1&	fnsiOrgNumericCode\\
&\ 4&	ffADDMD&	1&	addmdName\\
\hline
regional&
 \ 5&	organization&	2&	organizationName\\
&\ 6&	ffPlace&	2&	fipsPlaceNumericCode\\
\hline
local&
\ 7&	organization&	6&	organizationName\\
& 8&	residentialPerson&
			6&	commonName, other\\
\hline
organizational&
\ 9&	organizationalUnit&
			3,5,7,9&
				organizationalUnitName\\
&10&	organizationalRole&
			3,5,7,9&
				commonName\\
&11&	organizationalPerson&
			3,5,7,9&
				commonName\\
\hline
application&
12&	applicationProcess&
			3,5,7,9&
				commonName\\
&13&	ffApplicationEntity&
			12&	commonName\\
\hline
\end{tabular}\]
\end{bwslide}


\begin{bwslide}
\ctitle	{LESSONS LEARNED}

\begin{nrtc}
\item	USE OF CIVIL INFRASTRUCTURE MINIMIZES NEW WORK

\item	CERTAINLY NOT PERFECT

\item	ALPHANUMERIC NAME FORMS PREFERRED FOR RDNs,
    \begin{nrtc}
    \item	BUT USE OF NUMERIC NAME FORMS
    \end{nrtc}
    ILLUSTRATES ``REAL'' RELATIONSHIP BETWEEN RDNs AND AVs

\item	HOW TO WRITE A FREEDONIAN BROWSER?
\end{nrtc}
\end{bwslide}


\begin{bwslide}
\part*	{INTERROGATION ALGORITHMS}\bf

\begin{nrtc}
\item	A CRITICAL OBSERVATION:
    \begin{nrtc}
    \item	AN INTERROGATION ALGORITHM MUST HAVE DETAILED
		KNOWLEDGE OF THE NAMING STRUCTURE
    \end{nrtc}

\item	GOAL:
    \begin{nrtc}
    \item	PROVIDE THE ``USER-FRIENDLY'' CHARACTERISTIC OF
		GUESSABILITY

    \item	USER SUPPLIES A (TEXTUAL) \emph{PURPORTED NAME}
		AND ALGORITHM IDENTIFIES ZERO OR MORE DNs
    \end{nrtc}
\end{nrtc}
\end{bwslide}


\begin{bwslide}
\ctitle	{INTERROGATION ALGORITHMS}

\begin{nrtc}
\item	START WITH (SOME) KNOWLEDGE OF THE DIT (SUB-)STRUCTURE

\item	TAKE INPUT FROM THE USER
    \begin{nrtc}
    \item	(ALONG WITH AN ENVIRONMENT OF USER-OPTIONS)
    \end{nrtc}

\item	INVOKE DIRECTORY OPERATIONS
    \begin{nrtc}
    \item	(MOSTLY SEARCH AND READ)
    \end{nrtc}

\item	IN-BETWEEN OPERATIONS
    \begin{nrtc}
    \item	INTERACT WITH USER (AS APPROPRIATE)
    \end{nrtc}
    TO FOCUS USE OF OPERATIONS
\end{nrtc}
\end{bwslide}


\begin{bwslide}
\ctitle	{THE CHALLENGE}

\begin{nrtc}
\item	DESIGN AN INTERROGATION ALGORITHM
    \begin{nrtc}
    \item	WHICH REQUIRES MINIMAL-KNOWLEDGE

    \item	WHILST EXECUTING RELATIVELY EFFICIENTLY
    \end{nrtc}

\item	NO STANDARDS IN THIS AREA

\item	BUT FIRST, WE NEED TO UNDERSTAND HOW SEARCHING WORKS
\end{nrtc}
\end{bwslide}


\begin{bwslide}
\ctitle	{DIRECTORY SEARCH OPERATION}

\begin{nrtc}
\item \verb"baseobject": WHERE SEARCH STARTS

\item \verb"subset": DEPTH OF SEARCH, ONE OF:
    \begin{nrtc}
    \item	\verb"baseobject", \verb"oneLevel", \verb"wholeSubtree"
    \end{nrtc}

\item \verb"filter": BOOLEAN EXPRESSION APPLIED TO ATTRIBUTES OF EACH ENTRY
\[\begin{tabular}{|r|l|}
\hline
\multicolumn{1}{|c|}{\bf operator}&
			\multicolumn{1}{c|}{\bf abbrev.}\\
\hline
logical-and&		$\land$\\
logical-or&		$\lor$\\
logical-not&		$\lnot$\\
equality&		$=$\\
substring-containment&	$\in$\\
greater-than or equal&	$\geq$\\
less-than or equal&	$\leq$\\
value is present&	$\exists$\\
approximately equal&	$\approx$\\
\hline
\end{tabular}\]

\item \verb"searchaliases": FILTER SHOULD BE APPLIED TO ALIASES

\item	\verb"entryInformationSelection": WHICH ATTRIBUTES SHOULD BE RETURNED
\end{nrtc}
\end{bwslide}


\begin{bwslide}
\ctitle	{SERVICE CONTROLS}

\begin{nrtc}
\item \verb"options", ANY OF:
\begin{quote}\begin{tabular}{l}
preferChaining\\
chainingProhibited\\
dontUseCopy\\
dontDereferenceAliases\\
localScope
\end{tabular}\end{quote}

\item \verb"priority", ONE OF:
    \begin{nrtc}
    \item	\verb"low", \verb"medium", \verb"high"
    \end{nrtc}

\item \verb"timelimit"/\verb"sizelimit": MAXIMUM NUMBER OF SECONDS/ENTRIES TO
BE USED/RETURNED

\item \verb"scopeOfReferral", ONE OF:
    \begin{nrtc}
    \item	\verb"dmd", \verb"country"
    \end{nrtc}
\end{nrtc}
\end{bwslide}


\begin{bwslide}
\ctitle	{ALGORITHM \#1:\\ KILLE'S USER-FRIENDLY NAMING SCHEME}

\begin{nrtc}
\item	USE LIKELY NAMING INFORMATION
\begin{quote}\small\begin{verbatim}
marshall rose, psi
\end{verbatim}\end{quote}
	(THESE ARE ORDERED)

\item	NAMES ARE (POSSIBLY) INCOMPLETE, e.g.,
\begin{quote}\small\begin{verbatim}
kille, ucl, gb
kille, cs, ucl, gb
\end{verbatim}\end{quote}
	MIGHT RESOLVE TO THE SAME ENTRY

\item	ALGORITHM USES IMPRECISE MATCHING AND ASSIGNS ``GOODNESS'' LEVEL TO
	MATCHES

\item	USERS ARE QUERIED FOR ASSISTANCE ON QUESTIONABLE MATCHES
\end{nrtc}
\end{bwslide}


\begin{bwslide}
\ctitle	{UFN: KNOWLEDGE}

\begin{nrtc}
\item	MULTI-LEVEL NAMING ARCHITECTURE

\item	KNOWLEDGE OF ISO3166 CODES FROM ROOT
\end{nrtc}
\end{bwslide}


\begin{bwslide}
\ctitle	{UFN: USER INPUT}

\begin{nrtc}
\item	ONE OR MORE ORDERED NAMING COMPONENTS, e.g.,
\begin{quote}\small\begin{verbatim}
marshall rose, psi
kille, cs, ucl, gb
L. Eagle, "Sue, Grabbit, and Runn", Oxford
\end{verbatim}\end{quote}

\item	USUALLY UNTYPED, BUT MAY BE
\begin{quote}\small\begin{verbatim}
James Hacker, l=Oxford, Widget Inc.
\end{verbatim}\end{quote}
\end{nrtc}
\end{bwslide}


\begin{bwslide}
\ctitle	{UFN: USER ENVIRONMENT\\ (PUBLIC DUA IN NORTH AMERICA)}

\[\begin{tabular}{|r|l|}
\hline
\multicolumn{1}{|c|}{\bf Range}&
		\multicolumn{1}{|c|}{\bf Bases}\\
\hline
1,2:&		\tt c=US\\
&		\tt c=CA\\
&		\tt -\\
\hline
3,+:&		\tt -\\
&		\tt c=US\\
&		\tt c=CA\\
\hline
\end{tabular}\]
\end{bwslide}


\begin{bwslide}
\ctitle	{UFN: USER ENVIRONMENT\\ (PRIVATE DUA)}

\[\begin{tabular}{|r|l|}
\hline
\multicolumn{1}{|c|}{\bf Range}&
		\multicolumn{1}{|c|}{\bf Bases}\\
\hline
1:&		\tt c=US@o=Performance Systems International@ou=Sales\\
&		\tt c=US@o=Performance Systems International\\
&		\tt c=US\\
&		\tt -\\
\hline
2:&		\tt c=US\\
&		\tt c=US@o=Performance Systems International\\
&		\tt -\\
\hline
3,+:&		\tt -\\
&		\tt c=US\\
&		\tt c=US@o=Performance Systems International\\
\hline
\end{tabular}\]
\end{bwslide}


\begin{bwslide}
\ctitle	{ALGORITHM: DATA STRUCTURES}

\hrule\vskip.15in
\begin{tgrind}\scriptsize
\let\linebox=\relax
\def\_{\ifstring{\char'137}\else\underline{\ }\fi}
\input figureD-5\relax
\end{tgrind}
\end{bwslide}


\begin{bwslide}
\ctitle	{ALGORITHM: DATA STRUCTURES (cont.)}

\hrule\vskip.15in
\begin{tgrind}\scriptsize
\let\linebox=\relax
\def\_{\ifstring{\char'137}\else\underline{\ }\fi}
\input figureD-6\relax
\end{tgrind}
\end{bwslide}


\begin{bwslide}
\ctitle	{ALGORITHM: TOP-LEVEL}

\hrule\vskip.15in
\begin{tgrind}\scriptsize
\let\linebox=\relax
\def\_{\ifstring{\char'137}\else\underline{\ }\fi}
\input figureD-7\relax
\end{tgrind}
\end{bwslide}


\begin{bwslide}
\ctitle	{ALGORITHM: FIND MATCH}

\hrule\vskip.15in
\begin{tgrind}\scriptsize
\let\linebox=\relax
\def\_{\ifstring{\char'137}\else\underline{\ }\fi}
\input figureD-8\relax
\end{tgrind}
\end{bwslide}


\begin{bwslide}
\ctitle	{ALGORITHM: FIND MATCH (cont.)}

\hrule\vskip.15in
\begin{tgrind}\scriptsize
\let\linebox=\relax
\def\_{\ifstring{\char'137}\else\underline{\ }\fi}
\input figureD-15\relax
\end{tgrind}
\end{bwslide}


\begin{bwslide}
\ctitle	{ALGORITHM: SEARCHING THE ROOT}

\hrule\vskip.15in
\begin{tgrind}\scriptsize
\let\linebox=\relax
\def\_{\ifstring{\char'137}\else\underline{\ }\fi}
\input figureD-9\relax
\end{tgrind}
\end{bwslide}


\begin{bwslide}
\ctitle	{ALGORITHM: SEARCHING BELOW THE ROOT}

\hrule\vskip.15in
\begin{tgrind}\scriptsize
\let\linebox=\relax
\def\_{\ifstring{\char'137}\else\underline{\ }\fi}
\input figureD-10\relax
\end{tgrind}
\end{bwslide}


\begin{bwslide}
\ctitle	{ALGORITHM: SEARCHING BELOW THE ROOT (cont.)}

\hrule\vskip.15in
\begin{tgrind}\scriptsize
\let\linebox=\relax
\def\_{\ifstring{\char'137}\else\underline{\ }\fi}
\input figureD-13\relax
\end{tgrind}
\end{bwslide}


\begin{bwslide}
\ctitle	{ALGORITHM: SEARCHING FOR LEAVES, etc.}

\hrule\vskip.15in
\begin{tgrind}\scriptsize
\let\linebox=\relax
\def\_{\ifstring{\char'137}\else\underline{\ }\fi}
\input figureD-11\relax
\end{tgrind}
\end{bwslide}


\begin{bwslide}
\ctitle	{ALGORITHM: SEARCH OPERATION}

\hrule\vskip.15in
\begin{tgrind}\scriptsize
\let\linebox=\relax
\def\_{\ifstring{\char'137}\else\underline{\ }\fi}
\input figureD-12\relax
\end{tgrind}
\end{bwslide}


\begin{bwslide}
\ctitle	{ALGORITHM: DETERMINE GOODNESS}

\hrule\vskip.15in
\begin{tgrind}\scriptsize
\let\linebox=\relax
\def\_{\ifstring{\char'137}\else\underline{\ }\fi}
\input figureD-14\relax
\end{tgrind}
\end{bwslide}


\begin{bwslide}
\ctitle	{AN EXAMPLE:\\ ``\verb"rose, psi"''}

\begin{nrtc}
\item	USE \verb"purportedMatch"
\begin{quote}\small\begin{verbatim}
base:    {
           { { countryName, "US" } }
         }
p:       { "psi", "rose" }
\end{verbatim}\end{quote}

\item	USE \verb"intermediateSearch"
\begin{quote}\small\begin{verbatim}
base:    {
           { { countryName, "US" } }
         }
s:       "psi"
\end{verbatim}\end{quote}
WHICH CALLS \verb"search"
\end{nrtc}
\end{bwslide}


\begin{bwslide}
\ctitle	{AN EXAMPLE (cont.)}

\begin{nrtc}
\item	SEARCH ARGUMENTS:
\begin{quote}\small\begin{verbatim}
baseObject:     {
                  { { countryName, "US" } }
                }
subsetType:     oneLevel
filter:
\end{verbatim}\end{quote}
\[\begin{array}{lllll}
\multicolumn{5}{l}{(\mbox{\tt organizationName}=\mbox{\tt "*PSI*"}}\\
&	&	\lor&	\multicolumn{2}{l}{\mbox{\tt organizationName}\approx\mbox{\tt "PSI"}}\\
&	&	\lor&	\multicolumn{2}{l}{\mbox{\tt localityName}=\mbox{\tt "*PSI*"}}\\
&	&	\lor&	\multicolumn{2}{l}{\mbox{\tt localityName}\approx\mbox{\tt "PSI"})}\\
&	\land&	\multicolumn{3}{l}{(\mbox{\tt objectClass}=\mbox{\tt organization}}\\
&	&	&	\lor&	    \mbox{\tt objectClass}=\mbox{\tt locality})\\
\end{array}\]

\item	RETURNS
\begin{quote}\small\begin{verbatim}
c=US@o=Performance Systems International
\end{verbatim}\end{quote}
A ``GOOD'' MATCH
\end{nrtc}
\end{bwslide}


\begin{bwslide}
\ctitle	{AN EXAMPLE (cont.)}

\begin{nrtc}
\item	\verb"purportedMatch" RECURSES WITH
\begin{quote}\small\begin{verbatim}
base:    {
           { { countryName, "US" } },
           { { organizationName,
               "Performance Systems International" } }
         }
p:       { "rose" }
\end{verbatim}\end{quote}

\item	USE \verb"leafSearch"
\begin{quote}\small\begin{verbatim}
base:          {
                 { { countryName, "US" } },
                 { { organizationName,
                     "Performance Systems International" }
                 }
               }
s:             "rose"
subsettype:    wholeSubtree
\end{verbatim}\end{quote}
WHICH CALLS \verb"search"
\end{nrtc}
\end{bwslide}


\begin{bwslide}
\ctitle	{AN EXAMPLE (cont.)}

\begin{nrtc}
\item	SEARCH ARGUMENTS:
\begin{quote}\small\begin{verbatim}
baseObject:    {
                 { { countryName, "US" } },
                 { { organizationName,
                     "Performance Systems International" }
                 }
               }
subsetType:    wholeSubtree
filter:
\end{verbatim}\end{quote}
\[\begin{array}{lll}
\multicolumn{3}{l}{\mbox{\tt commonName}=\mbox{\tt "*rose*"}}\\
&	\lor&	\mbox{\tt commonName}\approx\mbox{\tt "rose"}\\
&	\lor&	\mbox{\tt surName}=\mbox{\tt "*rose*"}\\
&	\lor&	\mbox{\tt surName}\approx\mbox{\tt "rose"}\\
&	\lor&	\mbox{\tt userID}=\mbox{\tt "*rose*"}\\
&	\lor&	\mbox{\tt userID}\approx\mbox{\tt "rose"}
\end{array}\]

\item	RETURNS
\begin{quote}\small\begin{verbatim}
c=US
    @o=Performance Systems International
    @ou=Research and Development
    @ou=Santa Clara
    @cn=Marshall Rose
\end{verbatim}\end{quote}
A ``GOOD'' MATCH
\end{nrtc}
\end{bwslide}


\begin{bwslide}
\ctitle	{ALGORITHM \#2:\\ AE LOOKUP}

\begin{nrtc}
\item	NEED TO DERIVE
    \begin{nrtc}
    \item	APPLICATION ENTITY TITLE
    \end{nrtc}

\item	FOR MAPPING INTO
    \begin{nrtc}
    \item	PRESENTATION ADDRESS
    \end{nrtc}

\item	USER SUPPLIES (PREFERABLY SHORT) TYPE-IN, e.g.,
\begin{quote}\small\begin{verbatim}
% ftam nisc,psi,us
\end{verbatim}\end{quote}
\end{nrtc}
\end{bwslide}


\begin{bwslide}
\ctitle	{APPROACH: AE LOOKUP WITH UFN}

\begin{nrtc}
\item	ONE POSSIBILITY
    \begin{nrtc}
    \item	LEVERAGE KILLE'S ALGORITHM
    \end{nrtc}

\item	USE UFN TO FIND \verb"applicationProcess"

\item	THEN SEARCH RESULTS TO FIND \verb"applicationEntity"

\item	IF MULTIPLE AE's
    \begin{nrtc}
    \item	ASK USER WHICH ONE (IF INTERACTIVE)

    \item	OTHERWISE, JUST TAKE FIRST
    \end{nrtc}
\end{nrtc}
\end{bwslide}


\begin{bwslide}
\ctitle	{AE/UFN: KNOWLEDGE}

\begin{nrtc}
\item	\verb"applicationEntity" RESIDES BELOW \verb"applicationProcess"

\item	IMMEDIATE SUBORDINATES FAVORED OVER DISTANT ONES
\end{nrtc}
\end{bwslide}


\begin{bwslide}
\ctitle	{AE/UFN: USER INPUT}

\begin{nrtc}
\item	ONE OR MORE ORDERED NAMING COMPONENTS
\begin{quote}\small\begin{verbatim}
nisc, psi, us
hubris, cs, ucl, gb
\end{verbatim}\end{quote}

\item	DESIRED \verb"supportedApplicationContext"
\end{nrtc}
\end{bwslide}


\begin{bwslide}
\ctitle	{ALGORITHM: TOP-LEVEL}

\hrule\vskip.15in
\begin{tgrind}\scriptsize
\let\linebox=\relax
\def\_{\ifstring{\char'137}\else\underline{\ }\fi}
\input figureD-21\relax
\end{tgrind}
\end{bwslide}


\begin{bwslide}
\ctitle	{ALGORITHM: SEARCH OPERATION}

\hrule\vskip.15in
\begin{tgrind}\scriptsize
\let\linebox=\relax
\def\_{\ifstring{\char'137}\else\underline{\ }\fi}
\input figureD-22\relax
\end{tgrind}
\end{bwslide}


\begin{bwslide}
\ctitle	{AN EXAMPLE:\\ ``\verb"ftam nisc,psi,us"''}

\begin{nrtc}
\item	USE \verb"aeLookup"
\begin{quote}\small\begin{verbatim}
p:       { "us", "psi", "nisc" }
sac:     1.0.8571.1.1
\end{verbatim}\end{quote}
WHICH CALLS \verb"friendlyMatch"

\item	RETURNS
\begin{quote}\small\begin{verbatim}
c=US
    @o=Performance Systems International
    @ou=Operations
    @cn=nisc
\end{verbatim}\end{quote}
\end{nrtc}
\end{bwslide}


\begin{bwslide}
\ctitle	{AN EXAMPLE (cont.)}

\begin{nrtc}
\item	USE \verb"aeSearch"
\begin{quote}\small\begin{verbatim}
base:          {
                 { { countryName, "US" } },
                 { { organizationName,
                     "Performance Systems International" } },
                 { { organizationalUnitName, "Operations" } },
                 { { commonName, "nisc" } }
               }
sac:            1.0.8571.1.1
subtreeType:    oneLevel
\end{verbatim}\end{quote}
WHICH INVOKES SEARCH WITH FILTER
$$\mbox{\tt supportedApplicationContext}=\mbox{\tt 1.0.8571.1.1}$$

\item	RETURNS
\begin{quote}\small\begin{verbatim}
c=US
    @o=Performance Systems International
    @ou=Operations
    @cn=nisc
    @cn=filestore
\end{verbatim}\end{quote}
\end{nrtc}
\end{bwslide}


\begin{bwslide}
\part	{DISTRIBUTED OPERATIONS}\bf

\begin{nrtc}
\item	KNOWLEDGE

\item	MODES OF OPERATION

\item	ENTRY GRANULARITY
\end{nrtc}
\end{bwslide}


\begin{bwslide}
\part*	{KNOWLEDGE}\bf

\begin{nrtc}
\item	NAMING CONTEXTS

\item	DSA INFORMATION MODEL
\end{nrtc}
\end{bwslide}


\begin{bwslide}
\ctitle	{NAMING CONTEXTS}

\begin{nrtc}
\item	NEED A MEANS FOR DETERMINING WHICH DSA HOLDS INFORMATION ON AN ENTRY

\item	DIVIDE THE ENTIRE DIT INTO NON-OVERLAPPING SUBTREES
	TERMED \emph{NAMING CONTEXTS},
	THE ``ROOT'' OF WHICH IS TERMED A \emph{CONTEXT PREFIX}

\item	EACH ENTRY RESIDES IN EXACTLY ONE NAMING CONTEXT
\end{nrtc}
\end{bwslide}


\begin{bwslide}
\ctitle	{A SKELETAL DIT}

\vskip.5in
\diagram[p]{figureD-35}
\end{bwslide}


\begin{bwslide}
\ctitle	{NAMING CONTEXTS AND KNOWLEDGE}

\begin{nrtc}
\item	EACH DSA MASTERS ONE OR MORE NAMING CONTEXTS

\item	HENCE, EACH ENTRY IS MASTERED IN ONE DSA

\item	SO, NOW NEED A WAY TO MODEL WHAT A DSA ``KNOWS'' AND ``NEEDS TO KNOW''
\end{nrtc}
\end{bwslide}


\begin{bwslide}
\ctitle	{DSA INFORMATION TREE}

\begin{nrtc}
\item	REPRESENTATION OF A NAMING CONTEXT WITHIN A DSA

\item	SAYS THREE THINGS:
    \begin{nrtc}
    \item	DSA KNOWS RDNs OF ALL SUPERIORS TO THE NAMING CONTEXT

    \item	DSA HAS COMPLETE INFORMATION ABOUT ENTRIES IN SUBTREE

    \item	DSA MUST KNOW ABOUT DSAs WHICH HOLD SUBORDINATES
    \end{nrtc}
\end{nrtc}
\end{bwslide}


\begin{bwslide}
\ctitle	{INFORMATION TREE IN ONE DSA}

\vskip.5in
\diagram[p]{figureD-36}
\end{bwslide}


\begin{bwslide}
\ctitle	{KNOWLEDGE INFORMATION}

\begin{nrtc}
\item	DSA HAS NO KNOWLEDGE OF ROOT-SUBORDINATES, THEN IT HAS:

\item	A SUPERIOR REFERENCE (DSA-AP=DSA NAME/ADDR)
    \begin{nrtc}
    \item	USED TO ``MOVE UP AND OVER''
    \end{nrtc}

\item	SUBORDINATE KNOWLEDGE FOR EACH NAMING CONTEXT IT MASTERS
    \begin{nrtc}
    \item	SUBORDINATE REFERENCE: DSA-AP AND CONTEXT PREFIX

    \item	NON-SPECIFIC SUBORDINATE REFERENCE: DSA-AP
    \end{nrtc}
    USED TO ``MOVE DOWN''

\item	CROSS-REFERENCES (DSA-AP AND CONTEXT PREFIX)
    \begin{nrtc}
    \item	USED TO ``MOVE OVER''
    \end{nrtc}
    (OPTIONAL)
\end{nrtc}
\end{bwslide}


\begin{bwslide}
\ctitle	{AN EXAMPLE:\\ DSA FOR CALIFORNIA}

\vskip.5in
\diagram[p]{figureD-37}
\end{bwslide}


\begin{bwslide}
\ctitle	{KNOWLEDGE INFORMATION (cont.)}

\begin{nrtc}
\item	DSA HAS KNOWLEDGE OF ROOT-SUBORDINATES
    \begin{nrtc}
    \item	IT'S CALLED A FIRST-LEVEL DSA
    \end{nrtc}
    THEN IT HAS:

\item	FOR EACH ROOT-SUBORDINATE,
    \begin{nrtc}
    \item	THE DSA-AP WHICH MASTERS THAT NAMING CONTEXT
    \end{nrtc}

\item	SUBORDINATE KNOWLEDGE FOR EACH NAMING CONTEXT IT MASTERS

\item	``REAL'' FIRST-LEVEL DSAs ARE LIKELY VIRTUAL ENTITIES
\end{nrtc}
\end{bwslide}


\begin{bwslide}
\part*	{MODES OF OPERATION}\bf

\begin{nrtc}
\item	REFERRAL v. CHAINING

\item	RELAYING
\end{nrtc}
\end{bwslide}


\begin{bwslide}
\ctitle	{REFERRAL v. CHAINING}

\begin{nrtc}
\item	ANOTHER HARD DECISION

\item	IN THE ABSENCE OF
    \begin{nrtc}
    \item	ACCOUNTING AND SETTLEMENT
    \end{nrtc}
	CHOICE IS MADE ON CONNECTIVITY
\end{nrtc}
\end{bwslide}


\begin{bwslide}
\ctitle	{USE OF PRESENTATION ADDRESS}

\begin{nrtc}
\item	MUST ACCOMMODATE DIVERSE COMMUNITIES

\item	CHAINING/REFERRAL
    \begin{nrtc}
    \item	SEE IF ORIGINATING NADDR IS IN SAME COMMUNITY

    \item	FOR DSA, LOOK AT PRESENTATION ADDRESS

    \item	FOR DUA, LOOK AT DAP ASSOCIATION
    \end{nrtc}
    AS ANY OF THE NADDRs FOR TARGET DSA

\item	DIRECT CONTACT
    \begin{nrtc}
    \item	SEE IF LOCAL END-SYSTEM IS IN SAME COMMUNITY
    \end{nrtc}
    AS ANY OF THE NADDRs FOR TARGET DSA

\item	REALLY A TREMENDOUS LAYERING VIOLATION
\end{nrtc}
\end{bwslide}


\begin{bwslide}
\ctitle	{RELAYING}

\begin{nrtc}
\item	MINIMALLY CONNECTED DSAs NEED AN
    \begin{nrtc}
    \item	AN APPLICATION RELAY
    \end{nrtc}

\item	SIMILAR IN CONCEPT TO SUPERIOR DSA
    \begin{nrtc}
    \item	BUT RELAY NEEDN'T BE MORE KNOWLEDGEABLE,
		SIMPLY BETTER CONNECTED
    \end{nrtc}
\end{nrtc}
\end{bwslide}


\begin{bwslide}
\part*	{ENTRY GRANULARITY}\bf

\begin{nrtc}
\item	IN A COMMERCIAL ENVIRONMENT
    \begin{nrtc}
    \item	NO SINGLE ENTITY CAN CLAIM SOLE RIGHT TO MANAGE ALL THE
		INFORMATION CORRESPONDING TO AN OBJECT
    \end{nrtc}

\item	WITH COMPETING DIRECTORY SERVICES
    \begin{nrtc}
    \item	THERE IS NO NATURAL ADMINISTRATOR FOR ENTRIES NAMED
		VIA ANNEX~B OR CIVIL AUTHORITY
    \end{nrtc}
\end{nrtc}
\end{bwslide}


\begin{bwslide}
\ctitle	{DISTRIBUTED ENTRIES}

\begin{nrtc}
\item	ONE SOLUTION
    \begin{nrtc}
    \item	WINDING ITS WAY THROUGH THE STANDARDS PROCESS
    \end{nrtc}

\item	A SERIES OF INVASIVE ARCHITECTURAL CHANGES BASED ON KNOWLEDGE
\end{nrtc}
\end{bwslide}


\begin{bwslide}
\ctitle	{SHARED NAMESPACES}

\begin{nrtc}
\item	ANOTHER SOLUTION
    \begin{nrtc}
    \item	DEVELOPED BY ANTHONY HODSON (ICL) AND OTHERS
    \end{nrtc}

\item	RELIES ON NAMING LINKS AND ADMINISTRATIVE COOPERATION

\item	NO CHANGES TO X.500 MODELS, SERVICE, PROCEDURES, OR PROTOCOLS
\end{nrtc}
\end{bwslide}


\begin{bwslide}
\ctitle	{A SHARED DOMAIN IN THE DIT}

\vskip.5in
\diagram[p]{figureD-34}
\end{bwslide}


\begin{bwslide}
\ctitle	{SHARING NAMESPACES}

\begin{nrtc}
\item	SERVICE PROVIDERS AGREE TO COOPERATIVELY MANAGE SHARED SPACE
    \begin{nrtc}
    \item	CONTAINING PUBLIC INFORMATION

    \item	PUBLISHED DIT STRUCTURE

    \item	AVAILABLE FOR INTERROGATION ONLY

    \item	HIGHLY-REPLICATED BETWEEN PROVIDERS
    \end{nrtc}

\item	EACH SERVICE PROVIDER MAINTAINS OWN PRIVATE SPACE
    \begin{nrtc}
    \item	(POSSIBLY) PRIVATE INFORMATION

    \item	(RESTRICTIVE) ADMINISTRATIVE POLICIES

    \item	AUTONOMOUSLY OPERATED
    \end{nrtc}
\end{nrtc}
\end{bwslide}


\begin{bwslide}
\ctitle	{CONTENTS OF ENTRIES IN THE SHARED SPACE}

\begin{nrtc}
\item	DAVs FOR RDNs

\item	AVs FOR SEARCHING

\item	NAMING LINKS

\item	(POSSIBLY) MANAGEMENT INFORMATION
\end{nrtc}
\end{bwslide}


\begin{bwslide}
\ctitle	{NAMING LINKS}

\begin{nrtc}
\item	USED AS POINTERS FROM SHARED SPACE TO PRIVATE SPACES
\begin{quote}\begin{verbatim}
namingLink ATTRIBUTE
    WITH ATTRIBUTE-SYNTAX distinguishedNameSyntax
    ::= { -- tbd -- }
\end{verbatim}\end{quote}

\item	AGENT ENTERS INTO PRIVATE AGREEMENT WITH PROVIDER
    \begin{nrtc}
    \item	TO ESTABLISH ENTRY IN SHARED SPACE

    \item	AND NAMING LINK TO PRIVATE SPACE
    \end{nrtc}

\item	NAMING LINKS ARE ADDED/REMOVED ON THE BASIS OF SUBSEQUENT PRIVATE
	AGREEMENTS BETWEEN AGENT AND VARIOUS PROVIDERS

\item	BACK LINKS MIGHT PROVIDE INVERSE MAPPING
\end{nrtc}
\end{bwslide}


\begin{bwslide}
\ctitle	{DUA OPERATION}

\begin{nrtc}
\item	WHEN AN INTERROGATION ALGORITHM YIELDS AN ENTRY WITH A NAMING LINK
    \begin{nrtc}
    \item	ASK USER IF NAMING LINK SHOULD BE FOLLOWED
    \end{nrtc}
    FOR FURTHER INFORMATION
\end{nrtc}
\end{bwslide}
