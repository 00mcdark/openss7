%% -*- tex -*- vim: ft=tex tw=78 nocin nosi
%% =========================================================================
%%
%% @(#) $Id: xcode.tex,v 1.1.2.1 2011-07-18 19:42:11 brian Exp $
%%
%% =========================================================================
%%
%% Copyright (c) 2001-2008  OpenSS7 Corporation <http://www.openss7.com/>
%%
%% All Rights Reserved.
%%
%% Permission is granted to make and distribute verbatim copies of this
%% manual provided the copyright notice and this permission notice are
%% preserved on all copies.
%%
%% Permission is granted to copy and distribute modified versions of this
%% manual under the conditions for verbatim copying, provided that the
%% entire resulting derived work is distributed under the terms of a
%% permission notice identical to this one.
%% 
%% Since the Linux kernel and libraries are constantly changing, this
%% manual page may be incorrect or out-of-date.  The author(s) assume no
%% responsibility for errors or omissions, or for damages resulting from
%% the use of the information contained herein.  The author(s) may not
%% have taken the same level of care in the production of this manual,
%% which is licensed free of charge, as they might when working
%% professionally.
%% 
%% Formatted or processed versions of this manual, if unaccompanied by
%% the source, must acknowledge the copyright and authors of this work.
%%
%% -------------------------------------------------------------------------
%%
%% U.S. GOVERNMENT RESTRICTED RIGHTS.  If you are licensing this Software
%% on behalf of the U.S. Government ("Government"), the following
%% provisions apply to you.  If the Software is supplied by the Department
%% of Defense ("DoD"), it is classified as "Commercial Computer Software"
%% under paragraph 252.227-7014 of the DoD Supplement to the Federal
%% Acquisition Regulations ("DFARS") (or any successor regulations) and the
%% Government is acquiring only the license rights granted herein (the
%% license rights customarily provided to non-Government users).  If the
%% Software is supplied to any unit or agency of the Government other than
%% DoD, it is classified as "Restricted Computer Software" and the
%% Government's rights in the Software are defined in paragraph 52.227-19
%% of the Federal Acquisition Regulations ("FAR") (or any successor
%% regulations) or, in the cases of NASA, in paragraph 18.52.227-86 of the
%% NASA Supplement to the FAR (or any successor regulations).
%%
%% =========================================================================
%% 
%% Commercial licensing and support of this software is available from
%% OpenSS7 Corporation at a fee.  See http://www.openss7.com/
%% 
%% =========================================================================
%%
%% Last Modified $Date: 2011-07-18 19:42:11 $ by $Author: brian $
%%
%% =========================================================================

\documentclass[letterpaper,final,notitlepage,twocolumn,10pt,twoside]{article}
\usepackage{ftnright}
\usepackage{makeidx}
\usepackage{pictex}
%\usepackage{psfig}
%\usepackage{graphics}
\usepackage{graphicx}
\usepackage{eepic}
%\usepackage{epsfig}
%\usepackage[dvips]{graphicx,epsfig}
%\usepackage[dvips]{epsfig}
%\usepackage{epsf}
\usepackage{natbib}
\usepackage{placeins}
%\usepackage{placeins}

\setlength{\voffset}{-1.2in}
\setlength{\topmargin}{0.2in}
\setlength{\headheight}{0.2in}
\setlength{\headsep}{0.3in}
\setlength{\topskip}{0.0in}
\setlength{\footskip}{0.3in}
\setlength{\textheight}{10.0in}

\setlength{\hoffset}{-1.0in}
\setlength{\oddsidemargin}{0.5in}
\setlength{\evensidemargin}{0.5in}
\setlength{\textwidth}{7.5in}

\setlength{\marginparwidth}{0.0in}
\setlength{\marginparsep}{0.0in}

\setlength{\columnsep}{0.3in}
\setlength{\columnwidth}{3.6in}
%\setlength{\columnseprule}{0.25pt}

\setlength{\paperheight}{11in}
\setlength{\paperwidth}{8.5in}

\let\Huge = \huge
\let\huge = \LARGE
\let\LARGE = \Large
\let\Large = \large
\let\large = \normalsize
\let\normalsize = \small
\let\small = \footnotesize
\let\footnotesize = \scriptsize
\let\scriptsize = \tiny

\makeatletter
\renewcommand\section{\@startsection {section}{1}{\z@}%
                                   {-2ex \@plus -1ex \@minus -.2ex}%
                                   {1ex \@plus .2ex}%
                                   {\normalfont\large\bfseries}}
\renewcommand\subsection{\@startsection{subsection}{2}{\z@}%
                                     {-1.5ex \@plus -.5ex \@minus -.2ex}%
                                     {1ex \@plus .2ex}%
                                     {\normalfont\normalsize\bfseries}}
\renewcommand\subsubsection{\@startsection{subsubsection}{3}{\z@}%
                                     {-1.25ex\@plus -.5ex \@minus -.2ex}%
                                     {1ex \@plus .2ex}%
                                     {\normalfont\normalsize\bfseries}}
\renewcommand\paragraph{\@startsection{paragraph}{4}{\z@}%
                                    {1.5ex \@plus .5ex \@minus .2ex}%
                                    {-1em}%
                                    {\normalfont\normalsize\bfseries\slshape}}
\renewcommand\subparagraph{\@startsection{subparagraph}{5}{\parindent}%
                                       {0ex \@plus 0ex \@minus 0ex}%
                                       {-1em}%
                                      {\normalfont\normalsize\bfseries\slshape}}
\makeatother

\setcounter{tocdepth}{3}
\setcounter{secnumdepth}{6}

\pagestyle{plain}
%\pagestyle{myheadings}
%\markboth{B. Bidulock}{B. Bidulock}

\makeglossary

\newcommand{\topfigrule}{\vspace{0.5ex}\rule{\columnwidth}{0.4pt}\vspace{0.5ex} }
\newcommand{\botfigrule}{\vspace{0.5ex}\rule{\columnwidth}{0.4pt}\vspace{0.5ex} }
\newcommand{\dblfigrule}{\vspace{0.5ex}\rule{\textwidth}{0.4pt}\vspace{0.5ex} }

%\bibliographystyle{unsrtnat}
%\bibliographystyle{plainnat}
%\bibliographystyle{ieeetr}
%\bibliographystyle{abbrvnat}
%\bibliographystyle{acm}
%\bibliographystyle{plainnat}
\bibliographystyle{alpha}

\begin{document}

%\begin{titlepage}
%\begin{center}
%    STREAMS vs. Sockets Performance Comparison\\
%    Experimental Test Results
%\end{center}
%\end{titlepage}

\title{RTP Forwarding and Transcoding\\[0.5ex]
	{\large \textsl{Design for Linux Fast-STREAMS}}}
\author{Brian F. G. Bidulock\thanks{bidulock@openss7.org}\\
	OpenSS7 Corporation}
\date{July 4, 2011}
\maketitle

\begin{abstract}
\addcontentsline{toc}{section}{Abstract}
\end{abstract}

\tableofcontents

\section{Background}

\subsection{Demand for Forwarding}

Supplying VoIP telephony has evolved into a ecosystem of providers.  Tier 1
providers (other than incumbent telephone companies) are dominated by resellers
that provide network access but rely on both local and long distance carriers
for providing PSTN network access.  When these Tier 1 VARs provide network
connectivity to larger and more sophisticated customers (IP-PBX owners), their
customers can use the IP addresses at the other end of a SIP connection or RTP
connection to determine which wholesaler is providing network connectivity to
the VAR, resulting in these customers negotiating directly with the wholesale
network provided for reduced rates.

This has cause a need for such VARs to obscure the IP addresses of their
wholesale network providers.  The SIP IP addresses can be obfuscated easily with
SIP proxies; however, the IP addresses of the RTP endpoints cannot so easily be
obscured.  This has created a need for simple forwarding of RTP packets with IP
address and port number translation so that the VAR's customers do not see the
IP addresses of the VAR's wholesaler.

\subsubsection{Gateway Forwarding}

Although there is a demand for forwarding to accommodate IP address
obfuscation, there is a basic requirement for forwarding that is required by
a regular Media Gateway (one that converts from TDM to IP).  The ability to
rapidly marshal G.711 encoded audio samples from the TDM interface to RTP
payload is a basic requirement of such a media gateway.  Therefore, any
development performed for the purpose of a forwarding IP-IP gateway, can be
reused in a regular TDM-IP gateway.

\subsection{Demand for Transcoding}

Wireless LTE (Long Term Evolution) and WiMAX networks use primarily VoIP to
process telephone calls.  This avoids the additional GSM infrastructure that
would otherwise be required for the processing of telephone calls.  This permits
a data-only service to be provided which is less onerous and capital intensive
than voice networks.

To minimize the bandwidth requirements for voice calls, primarily on the
on-the-air interface, LTE and WiMAX network operators use low-bit-rate codecs
such as G.729 Annex A with Annex B comfort noise.  This creates a demand for
media gateways that support G.729 and other low-bit-rate codecs.  However, PSTN
carriers primarily use G.711 codecs for transport of voice calls.  Many
installed media gateways lack the ability to provide G.729 and would required
significant investment in DSP add-on equipment.  Further compounding matters is
that existing MGs have primarily been purchased on grey markets and do not have
the warranty support necessary for upgrade to G.729.

This generates a market demand for transcoding equipment.  This demand will
continue to increase as additional WiMAX and LTE networks are rolled out.

\subsubsection{Gateway Transcoding}

Although there is a demand for transcoding to accommodate LTE/WiMAX network
evolution, there is a basic requirement for transcoding that is requried by a
regular Media Gateway (one that converts from TDM to IP).  The ability to
rapidly transcode G.711 encoded audio samples from the TDM interface to RTP
payloads other than G.711 is a basic requirement of such a media gateway.
Therefore, any development performed for the purpose of a transcoding IP-IP
gateway, can be reused in a regular TDM-IP gateway.

%Something \cite[]{Ritchie84} is illustrated in \textsl{Figure \ref{figure:xc01}}.
%\begin{figure}[htp]
%\center\includegraphics[width=\columnwidth]{xc01}
%\caption[MGIllustration1]{MG Illustration 1}
%\label{figure:xc01}
%\end{figure}
%
%Something is illustrated in \textsl{Figure \ref{figure:xc02}}.
%\begin{figure}[htp]
%\center\includegraphics[width=\columnwidth]{xc02}
%\caption[MGIllustration2]{MG Illustration 2}
%\label{figure:xc02}
%\end{figure}
%
%Something is illustrated in \textsl{Figure \ref{figure:xc03}}.
%\begin{figure}[htp]
%\center\includegraphics[width=\columnwidth]{xc03}
%\caption[MGIllustration3]{MG Illustration 3}
%\label{figure:xc03}
%\end{figure}

%Something is illustrated in \textsl{Figure \ref{figure:xc05}}.
%\begin{figure}[htp]
%\center\includegraphics[width=\columnwidth]{xc05}
%\caption[MGIllustration5]{MG Illustration 5}
%\label{figure:xc05}
%\end{figure}
%
%Something is illustrated in \textsl{Figure \ref{figure:xc06}}.
%\begin{figure}[htp]
%\center\includegraphics[width=\columnwidth]{xc06}
%\caption[MGIllustration6]{MG Illustration 6}
%\label{figure:xc06}
%\end{figure}

\section{Objective}

The objective of this RTP Forwarding and Transcoding implementation is to
provide an large-scale, integrated, carrier class, solution for performing
forwarding and transcoding.

\subsection{General Requirements}
\label{section:genreq}

In particular, for transcoding applications for LTE and WiMAX, it can be
expected that the PSTN carrier will have antiquated equipment for internal
MGC/MGs. (See the MGC and MG arrangement in relation to the S-SBG/D-SBG of {\sl
Figure \ref{figure:xc08}, page \pageref{figure:xc08}}.)  Some assumptions are as
follows:

\begin{itemize}
\item The MGs likely only support G.711, mu-law, with a 20 or 30 millisecond payload.
\item The MGCs and MGs likely do not support DTMF and Tone RTP payloads.
\item The MGCs likely do not support sophisticated early media (such as the
{\tt PRACK} or {\tt UPDATE} methods).
\end{itemize}

\begin{figure}[htp]
\center\includegraphics[width=\columnwidth]{xc08}
\caption[MGIllustration8]{Session Border Gateway}
\label{figure:xc08}
\end{figure}

\subsection{Forwarding Requirements}

Forwarding needs of VARs can be met with an decomposed or integrated S-SBG
(Signalling Session Border Gateway) and D-SBG (Data Session Border Gateway),
where the S-SBG acts as an outgoing and incoming SIP proxy and the D-SBG acts as
an IP-IP Media Gateway.  The positioning of such a gateway is illustrated in
\textsl{Figure \ref{figure:xc08}}.

It is fully expected that the integrated S-SBG/D-SBG appliance be capable of
completely replacing any SBC (Session Border Controller) equipment at the
boundary to the administrative domain.

\subsection{Transcoding Requirements}

The transcoding needs of providers to LTE and WiMAX networks can be met with a
decomposed or integrated S-SBG (Signalling Session Border Gateway) and D-SBG
(Data Session Border Gateway), where the S-SBG acts as an outgoing and incoming
SIP proxy and the D-SBG acts as an IP-IP Media Gateway.  The positioning of such
a gateway is illustrated in \textsl{Figure \ref{figure:xc08}}.

It is fully expected that the integrated S-SBG/D-SBG appliance be capable of
completely replacing any SBC (Session Border Controller) equipment at the
boundary to the administrative domain.

\section{Description}

The overall solution architecture has been designed and revised may times during
the evolution of {\sl The OpenSS7 Project}.  An MG (Media Gateway) stack has
been previously defined in detail.  The relationship between the {\sl STREAMS}
drivers and modules that make up the MG solution architecture and related
components are illustrated in {\sl Figure \ref{figure:xc04}}.  The figure
illustrates the software components necessary for implementation of a complete
H.248/MEGACO based Media Gateway (MG).  The upper level components (that are
outside the scope of this document) consist of:

\begin{figure}[htp]
\center\includegraphics[width=\columnwidth]{xc04}
\caption{Solution Architecture}
\label{figure:xc04}
\end{figure}

\begin{description}

\item[{\it H.248/MGCP Media Gateway Controller (H248) Driver:}] The H248
drivers is a multiplexing driver that is used to communicate with a remote
Media Gateway Controller (MGC) using H.248/MEGACO over UDP, TCP or SCTP.  This
driver parses messages and performs control of the MG Driver that provides the
actual MG function.

Implementing this as a {\sl STREAMS} driver is not necessary.  The XGCP
library H.248 stack implementation can be used to affect the same ends.  The
XGCP library interfaces with the underlying {\sl STREAMS} UDP, TCP and SCTP
pseudo-device drivers; however, the handling of H.248 messages is performed
within the library in user-space.

\item[{\it Media Gateway (MG) Driver:}]  The MG driver provides H.248-like
internal control of the software switching matrix and associated media stacks.
It uses the {\sl Media Gateway Interface (MGI)} \cite{MGI}.  It is also
responsible for the generation of tones, playing of audio files, etc; however,
its primary function is to control and manage the software switching
matrix.

Note that rather than directly communicating with the H248 driver; the control
of the MG driver can be performed using the XGCP user-space library with a
local (instead of remote) context.

\item[{\it Software Switching Matrix (MATRIX) Multiplexing Driver:}]  The
MATRIX implements a software switching matrix that switches media and control
messages between the upper and lower multiplex interfaces.

\item[{\it Media Stacks:}]  The media stacks make up the lower levels of the
architecture.  Media stacks exist for PDH, SDH and RTP.  (Media stacks also
exist for ATM and VToIP, but are not illustrated.)

\end{description}

The primary component of concern to the current work is the RTP media stack,
shown in insert in {\sl Figure \ref{figure:xc07}, page \pageref{figure:xc07}}.

\subsection{SIP Stack}

The SIP stack features the {\sl sofia} SIP libraries.  These libraries have
been (or will be) modified for performance as outlined in {\sl Section
\ref{section:sip}, page \pageref{section:sip}}.

Note also that to provide abstraction of the control interface to the {\sl
sofia} SIP stack, the XCC XOM-based API can be used instead.  The permits
swapping the underlying SIP implementation out for some other component at
some point.  Also, it can provide a homogeneous interface for SIP call
control, H.323 call control, ISUP call control, and ISDN call control.

\subsection{H.248 Stack}

There are no stand-along open source H.248 Stack implementations.  {\sl
Erlang} provides an open source licensed H.248 Stack, but it requires the
entire {\sl Erlang} environment.  Some other projects have integrated H.248
capability but it is meager and not worth propagating.  Unfortunately, this
means that the OpenSS7 project must develop its own H.248 stack.

The H.248 libraries utilize a XOM-based interface for the generation and
receipt of H.248/MEGACO messages called the XGCP.  The messages of this
interface can be used for the MGC side of H.248 communications, for the MG
side of H.248 communications, and for local MG control, depending on the
context selected when the API is initialized.  Because the identical service
interface and API is used both for MGC remote control as well as MG local
control, the same interface can be used for both.  That is, applications using
the XGCP can be made unaware of whether the MG services are local or remote.

This permits, for example, issuing an identical set of requests against a
remote mated MG as are requested of the local MG, permitting synchronization
of the mated MGs.

\subsection{MG Stack}

The general organization of {\sl STREAMS} drivers and modules is illustrated in
{\sl Figure \ref{figure:xc07}}.  The primary components of the media stack,
illustrated in {\sl Figure \ref{figure:xc07}}, are as follows:

\begin{figure}[htp]
\center\includegraphics[width=2.0in]{xc07}
\caption{STREAMS Drivers and Modules}
\label{figure:xc07}
\end{figure}

\begin{description}

\item[{\it Channel Multiplex Driver (CH-MUX):}] The channel multiplex driver
(CH-MUX) sits at the top of the forwarding/transcoding media stack.  The CH-MUX
is described in more detail in \textsl{Section \ref{section:chmux}, page
\pageref{section:chmux}}.

\item[{\it Transcode Modules (XC):}] The transcoding modules (XC) are specialized
modules that sit in an interior multiplexed Stream within the media stack.  The
XC modules are described in more detail in \textsl{Section \ref{section:xc},
page \pageref{section:xc}}.

\item[{\it RTP Multiplexing Driver (RTP-MUX):}] The RTP (Real-Time Transport
Protocol) multiplexing driver (RTP-MUX) sits and beneath the transcoding Streams
and provides a switching nexus for dispatching packets to and from the
appropriate transcoding Streams.  The RTP-MUX is described in more detail in
\textsl{Section \ref{section:rtpmux}, page \pageref{section:rtpmux}}.

\item[{\it UDP Driver:}] The UDP (User Datagram Protocol) driver sits at the
bottom of the media stack and is responsible for distributing packets to the
network and discriminating packets received from the network.  The UDP driver is
described in more detail in \textsl{Section \ref{section:udp}, page
\pageref{section:udp}}.

\end{description}

\subsubsection{Channel Mutliplexing (CH-MUX) Driver}
\label{section:chmux}

The channel multiplexing driver sits at the top of the RTP
forwarding/transcoding media stack.  It provides a \textsl{Channel Interface
(CHI)} \cite[]{CHI} service interface at the upper boundary.  It is responsible
for providing control of the RTP forwarding and transcoding performed by the
media stack beneath it.  This driver has the capability of being interfaced with
the Media Gateway Driver (MG), or an upper layer user application that controls
the RTP stack.

The CH-MUX driver is responsible for cross-connecting a transcoded media stream
between RTP termination points.  The media format at the upper interface is
\texttt{audio/pcmu/8000/1}, \texttt{audio/pcma/8000/1} or
\texttt{audio/l16/8000/1}.

The CH-MUX driver is also responsible for providing transparent adaptive jitter
buffering of downward data streams.  Transparent adaptive jitter buffering is
described in detail in {\sl Section \ref{section:tajb}, page
\pageref{section:tajb}}.

\subsubsection{Transcoding (XC) Module}
\label{section:xc}

The transcoding modules provide all of the heavy lifting (computationally
intense operations) necessary for transcoding from the PCMU, PCMA or L16
upstream format, to a specific codec's native downstream format, and {\it vise
versa}.

The use of {\sl STREAMS} modules for transcoding provides for several
performance gains: {\it Instruction Cache Acceleration} described in {\sl
Section \ref{section:ica}, page \pageref{section:ica}}; {\it Pipelining}, {\sl
Section \ref{section:pipelining}, page \pageref{section:pipelining}}; and, {\it
Super Scalar Execution}, {\sl Section \ref{section:sse}, page
\pageref{section:sse}}.

The RTP payload provided from above is not necessarily in sequence and may be
lossy.  The RTP payloads provided from below are also not necessarily in
sequence and may also be lossy.  The specific XC module is responsible for
sequencing and handling lost payloads as necessarily to the specific encoding or
decoding process.  This is performed using a technique called {\it Transparent
Adaptive Jitter Buffering} that is detailed in \textsl{Section
\ref{section:tajb}, page \pageref{section:tajb}}.

\subsubsection{RTP Multiplexing (RTP-MUX) Driver}
\label{section:rtpmux}

The RTP multiplexing driver sits toward the bottom of the RTP
forwarding/transcoding media stack.  It provides a \textsl{Channel Interface
(CHI)} \cite[]{CHI} service interface at the upper boundary, and a
\textsl{Network Provider Interface (NPI)} \cite[]{NPI} service interface at the
lower boundary.  It is responsible for interfacing the transcoding modules above
it to the UDP driver beneath it.  It is also responsible for performing RTP
forwarding.

The primary function of the RTP-MUX driver is to dispatch RTP packets arriving
from the UDP driver to the appropriate decoding stack (transcoding modules).

The RTP-MUX driver is also responsible for providing transparent adaptive jitter
buffering of upward data streams.  Transparent adaptive jitter buffering is
described in detail in {\sl Section \ref{section:tajb}, page
\pageref{section:tajb}}.

\subsubsection{UDP Driver}
\label{section:udp}

The UDP driver sits at the bottom of the RTP forwarding/transcoding media stack.
It provides a {\sl Network Provider Interface (NPI)} \cite[]{NPI} service
interface at the upper boundary and interfaces with {\sl Linux} networking at
the lower (internal) interface.  It is responsible for sending and receiving
RTP, RTCP and STUN packets to and from the network.

\section{Method}

This section describes some of the methods that are used to achieve large scale
operation and maximum performance.

\subsection{Multi-Stream Interfaces}
\label{section:msi}

The {\it Multi-Stream Interface} technique is used to provide large-scale RTP
forwarding and transcoding using {\sl STREAMS}.  Typical {\sl STREAMS}
interfaces such as the {\sl Network Provider Interface (NPI)} \cite[]{NPI}
provide a single stream for each network relation (connection between local
transport address and remote transport address).  While this is suitable for a
number of network connections on the order of tens or hundreds, it does not
scale under {\sl Linux} to the thousands, tens of thousands, or even hundreds of
thousands.  Because the intended scale of the appliance is 10,000 to 500,000
channels of forwarding and transcoding, and because the {\sl Linux} operating
system only supports several thousand open file descriptors system-wide, another
approach became necessary.

In an attempt to reuse existing service interfaces such as the {\sl Network
Provider Interface (NPI)}, a general approach to converting an existing {\it
per-Stream} interface to a {\it Multi-Stream Interface} was devised.  The
approach provides a separate device or clone minor that can be opened that
provides the multi-Stream interface from the primary device or clone minor that
provides the per-Stream interface.  Message primitives that are passed on the
multi-Stream interface have a 64-bit token prefixed to each message primitive.
This 64-bit token identifies the individual virtual Stream within the
multi-Stream interface.  There are a pair of 64-bit tokens defined: one defined
by the user of the service interface; the other by the provider.  For the most
part, message primitives that are passed from user to provider are prefixed with
the provider's tag; message primitives passed from provider to user, the user's
tag.  However, the first transaction (one that ``opens'' the virtual Stream) has
the user tag provided on the user primitive and provider tag on provider
primitive.  This mechanism provides for the initial exchange of tag values.  Tag
size was selected at 64-bits to permit either the user or provider to use a
memory address as the tag.

For the RTP forwarding/transcoding media stack, all interfaces use the
Multi-Stream Interface approach.\footnote{Note that this {\it Multi-Stream
Interface} approach is quite applicable to any situation where management of
multiplexing drivers become difficult due to the need to dynamically allocate
lower stream resources.}

\subsection{Performance Improvements}
\label{section:perf}

Software codecs and transcoding is not a new concept.  However, most software
codec implementations do not perform well, and the vast majority of
implementations are based on dedicated DSP (Digital Signal Processor) hardware.

Product literature available on the web indicates that software codecs and
transcoding can only accommodate a maximum of about 1000 channels, and even then
only for the less complex or computationally intense codecs.  Even dedicated DSP
approaches yields only about 1000 to 2000 channels per large form-factor module
populated with a DSP farm.  Some network processors (NP) appear to provides
larger scale implementation; however, costs are high and scalability is still
low.

In fitting with the objective of providing transcoding support on a single host
of between 10,000 and 500,000 simultaneous channels of G.729/G.711 transcoding,
performance improvements over the approaches taken by other software transcoding
implementations is necessary.

\subsubsection{Disadvantages of Current Approaches}
The disadvantages of other software transcoding approaches appear to be as
follows:

\begin{enumerate}

\item{\it Reference implementation.}  Many transcoding implementations
actually use copies of the ITU-T Reference implementation {\tt C}-language
code as their implementations.  The {\tt C}-language code reference
implementations were written with correctness and off-line verification in
mind and not any manner of performance.

\item{\it Poor scheduling.}  Because reference implementations of codecs are
designed for off-line verification, they are single-threaded.  This results
in poor processor core utilization when used to attempt real-time processing
of multiple channels.

\item{\it Context switching.} Most software transcoding implementations are
user-space implementations.  This causes additional context switches and
scheduling difficulties for each packet transcoded.

\item{\it Unpinned memory.}  As user space implementations base on the
off-line reference implementation, most software transcoding implementations
do not utilize pinned memory (locked memory) and a subject to the whims of
memory swapping.

\end{enumerate}

\subsubsection{Advantages of the Proposed Approach}
The advantages of the proposed approach are:

\begin{enumerate}

\item{\it Kernel implementation rather than user space.} Performing encoding and
transcoding using the {\sl STREAMS} framework within the kernel provides for
light-weight process scheduling with a significant reduction in context
switching.\footnote{Here ``context switching'' is not just the switch in
processor context, but the also the penalty incurred by cache misses and lock
contention.}  Also, the {\sl STREAMS} scheduler schedules light-weight processes
on a different basis that the {\sl Linux} process scheduler processes
heavy-weight processors or threads, particularly as regards processor
persistence and cache heat.  The {\sl Linux Fast-STREAMS} scheduler promotes
more efficient batch processing, whereas, the main scheduler favours low latency
single-shot execution over batch processing.  This is described in more detail
in {\sl Section \ref{section:ica}, page \pageref{section:ica}}.

\item{\it Utilization of the {\sl STREAMS} framework.}

The {\sl STREAMS} framework can be used for 

This is described in more detail in {\sl Section \ref{section:ica}, page
\pageref{section:ica}} and {\sl Section \ref{section:pipelining}, page
\pageref{section:pipelining}}.

\item{\it Utilization of specialized SIMD instruction sets on modern
processors.}

This is described in more detail in {\sl Section \ref{section:sse}, page
\pageref{section:sse}}.

\end{enumerate}

\paragraph{Instruction Cache Acceleration.} \label{section:ica}

{\it Instruction Cache Acceleration} is a byproduct of the {\sl STREAMS}
framework, provided that some design principles are followed when writing {\sl
STREAMS} drivers and modules.

Much ado has been made about data cache heat and efficiency in the {\sl Linux}
operating system; however, general processor design and implementation over the
last 20 or 30 years dictates that instruction cache heat and efficiency has a
much larger impact on system performance.  There are several characteristics of
the {\sl STREAMS} framework that provide for instruction cache efficiency:

\begin{description}

\item{\it Light Weight Processes:--} {\sl STREAMS} provides the ability to chain
modules and drivers along a Stream to break complex tasks into smaller more
manageable pieces.  These modules are independent of each other (execution is
asynchronous), yet without requiring an expensive context switch between
modules.

\item{\it Service Procedures:--} {\sl STREAMS} provides the ability to defer
processing of messages until a later point in time.  Event driven soft-real-time
systems have the difficulty that event showers can occur, overwhelming the
system.  Hard real-time synchronous systems (such as DSP) have the problem that
excess capacity (the processing capacity not needed to meet a deadline) remains
inaccessible.  Service procedures in {\sl STREAMS} cause event showers to be
handled through queuing and flow control.  Queuing causes multiple events of a
similar nature to be queued together, so that they can be processed within the
service procedure in a hard loop (without any context switch).  This causes
increased instruction (and even data) cache efficiency during event showers,
increasing the stability and efficiency of the system.

\item{\it Processor Persistence:--} Not really a characteristic of {\sl
STREAMS}, but a characteristic of {\sl Linux Fast-STREAMS}, processor
persistence is a situation where events scheduled for execution on adjacent
queues, callbacks or call-outs are scheduled on the same processor that invoked
them.  While one might think that better performance could be achieved by
scheduling them on any available processor, both data and instruction cache
ping-pong can result unless processor persistence is applied.  Also, processor
persistence reduces lock contention on multiprocessor systems.  Processor
persistence is also conducive to ``batching'' of messages: when a batch of
messages is being processed in a service procedure that passes messages up or
down stream, those messages will not be processed until the service procedure
returns (completes its batch).  This keeps other service procedures from being
scheduled too early, and allows the batch to flow through the {\sl STREAMS}
framework, increasing instruction and data cache efficiency further.

\end{description}

Considering these characteristics of the {\sl STREAMS} framework and {\sl Linux
Fast-STREAMS} implementation, several principles can be used in the design of
drivers and modules to accelerate instruction cache (and data cache) efficiency,
as follows:

\begin{enumerate}

\item {\it Always provide service procedures except for the most simple of
modules.}
Providing service procedures and using them wisely increases the performance
through batch processing, improved instruction cache efficiency, and pipelining.

\item {\it Always queue normal priority messages for processing by the service
procedure.}
Messages should always be queued quickly for processing by the service
procedure.  The only exception is some messages that have to be responded to
immediately (like {\tt M\_FLUSH}), and potentially some messages that terminate
on the module and do not result in messages being passed up or down stream.

\item {\it Always process fast-path messages out of the service procedure.}
The data fast-path should always be processed out the service procedure for
maximum performance and increased stability in the face of event showers.

\item {\it Keep fast-path service loops within the service procedure small and
tight.}
Processing of data in the fast-path out of the service procedures should be
small and tight as possible so that instruction cache efficiency can provide
maximum benefits.

\item {\it Break long compute-intensive tasks down into smaller units and
pipeline them in a chain of modules.}
Breaking larger tasks down into smaller and tighter units chained together in a
stream of modules (pipelining) provides better instruction cache and data cache
efficiency and provides more degrees of freedom and opportunities for multiple
processors to operate along different portions of the stream.

\end{enumerate}

\paragraph{Pipelining.}
\label{section:pipelining}

Pipelining can be achieved by proper organization and interconnection of
{\sl STREAMS} drivers and modules.  In general, where functions performed on
messages passed (upstream or downstream) differ, a separate {\it Stream}
should be used for each function.

For example, the IP driver intercepts packets that are bound for the D-SBG
function.  Packets can be processed as media-aware transcoding, or
media-agnostic forwarding.  Because these two functions utilize different
fast-paths, the IP driver discriminates between the two and passes packets for
transcoding to one upper {\it Stream} and packets for forwarding to another.

As another example, the RTP module processes RTP packets for transcoding.  The
transcoding is always performed between G.711 or 16-bit linear PCM.  Because
the functions performed depends on the coding of the RTP payload, the RTP
driver distinguishes between payloads and passed the packets to the
appropriate upper module chain responsible for transcoding that specific
payload.  This is illustrated in {\sl Figure \ref{figure:xc07}, page
\pageref{figure:xc07}}.

Pipelining in this fashion improves both instruction and data cache efficiency
by batching together packets upon which like functions are performed in
service procedures.

\paragraph{Super Scalar Execution.}
\label{section:sse}

Super-scalar execution (SSE) and related processing approaches (MMX, etc.)
provide a library accessing special purpose super-scalar execution instructions
in modern processors.  For the most part these are SIMD (Single Instruction
Multiple Data) instructions that provide a 2-wide, 4-wide, or 8-wide data path
with a single instruction pipeline.  For algorithms that are amenable to these
approaches, execution performance can be increased by a factor of the data path
width (2, 4 or 8) on a single processor.

Although these approaches are well known, most software codec implementations
simply utilize the off-line coded reference implementations of the codecs rather
than attempting to optimize them in any way.  This, of course, leads to much
poorer performance than is possible.

It has been stated that one of the advantages of using DSP over general
purpose RISC or CISC processors is that DSP can retrieve an entire block of
data at the same time and perform operations on that data in parallel.  Some
DSP are MIMD (Multiple Instruction Multiple Data) architectures.  By utilizing
the SIMD (Single Instruction Multiple Data) capabilities of modern general
purpose processors, and by utilizing pipelining (see {\sl Section
\ref{section:pipelining}, page \pageref{section:pipelining}}), multiple cores
can be invoked to achieve greater calculation performance than that achievable
using DSPs.  This is because digital signal processors often have
significantly lower clock and memory speeds and do not perform caching in the
normal sense: therefore, DSP cannot take advantages of cache acceleration
techniques such as those described under {\sl Instruction Cache Acceleration:
Section \ref{section:ica}, page \pageref{section:ica}}.

\subsubsection{SIP Implementation}
\label{section:sip}

The SIP implementation intended on being used is the {\sl sofia} SIP stack.
The {\sl sofia} SIP implementation is principally a shared-object library that
is used in a main event loop by an application program that implements the SIP
state machine.  Several performance improvements need to be made to the {\sl
sofia} libraries as follows:

\begin{enumerate}

\item The {\sl sofia} code base does not version its symbols.  Proper library
versioning needs to be added.

\item The {\sl sofia} code base does not address cancellation points for
threaded programs.  Proper cancellation treatment for asynchronous thread
cancellation needs to be added.

\item The {\sl sofia} code base does not organize functions by code heat.
Functions need to be organized by code heat.

\item The {\sl sofia} code base compiles without profiling.  The code base
needs to be profiled and branch prediction included in the code base.

\item The {\sl sofia} code base uses {\sl Linux} sockets for UDP, TCP and
SCTP.  The OpenSS7 {\sl STREAMS} implementations of these are more efficient
and provide a performance increase due to batching and reduced context
switching.  The sockets interface to UDP and SCTP need to be substituted for
{\sl XTI} interface to the OpenSS7 {\sl STREAMS} implementations of these.

\end{enumerate}

Unfortunately, the {\sl libsofia-sip} libraries have been added to recent
major distributions (such as Debian, Mageia, Fedora), largely in support of
the {\sl telepathy} package.  This means that there is little opportunity to
properly repackage the library.  On the other hand, it may be possible to
repackage the library as a different implementation of the same interface with
proper library versioning.

Note that the {\sl sofia} SIP implementation is not a full blown application,
but is, rather, a set of libraries and event handlers that can be used to
build a SIP application.

\subsection{Transparent Adaptive Jitter Buffering}
\label{section:tajb}

Adaptive and non-adaptive jitter buffering is a technique whereby the playback
of audio data is buffered and played back as a delayed program in an attempt to
alleviate any out of sequence arrival or jitter between inter-arrival times for
RTP packets.  The normal form of jitter buffering is only performed at the
termination of an audio stream and is not as necessary within the network.

For transcoding, most codecs provide some constraints on the ordering of data
chunks in the stream and require all adjacent packets before transcoding a given
segment can occur.  To accommodate transcoding, it is necessary to perform some
resequencing and buffering of segments to perform the transcoding function.
However, because playback is not being performed, and because most codecs permit
segments interior to the data stream to be encoded without consideration for the
signal outside the pertinent segment range, some actions performed by normal
terminating jitter buffering are not required.

The actions of normal jitter buffering that do not need to be performed to
accomplish transcoding are as follows:

\begin{description}

\item{\it Loss Detection:--} The transparent adaptive jitter buffer does not
have to determine when packets are lost by imposing a maximum delay as is
performed with normal jitter buffering.

\item{\it Avoiding Early Delivery:--}  Because playback is not being performed,
transparent adaptive jitter buffering does not have to worry about early
delivery.  As soon as a packet is available for transmission, it can be
transmitted regardless of the order in which the packets are processed.

\item{\it Equal Payload:--}  Normal jitter buffering typically requires playback
at a constant rate and requires that each playback segment be of the same size
and delivered synchronous with the playback clock.  This is not the case for
transparent adaptive jitter buffering that can coalesce multiple segments into a
larger payload (up to {\tt maxptime}) to more efficiently utilize the network.
This can help with jitter reduction as well as compensating for network
congestion upstream by reducing packet overheads downstream.

\end{description}

Transparent adaptive jitter buffering works as follows:

\begin{enumerate}

\item Packet payloads are added to the buffer using an insertion sort.

\item When the addition of a packet results in a contiguous segment sufficient
for encoding, a maximally sized segment is struck from the buffer and encoded.

\item When an excessive amount of segment fragments exist in the buffer, the
oldest segment fragments that constitute the excess are struck.  Depending on
the codec, loss reports might be generated for the struck segments.

\item Where one incoming packet maps to an integral number of segments, the
buffer is simply bypassed.

\end{enumerate}

Transparent jitter buffering, therefore, has the following characteristics that
are different from normal non-adaptive or adaptive jitter buffering:

\begin{itemize}

\item Segments are delivered as soon as they become available.

\item Jitter is maintained across the buffer to the greatest extent possible
considering the ordering needs of the encoding.

\end{itemize}

Note that most implementations perform normal RTP termination instead of
adaptive jitter-buffering.

\subsection{RTCP Packets}

When forwarding RTP and RTCP packets, RTCP packets are treated the same as any
other UDP packets and are simply IP address and port translated and forwarded.
Transcoding, on the other hand, is more complex and requires that RTCP packets
be transcoded to some degree as well, particularly as regards SR (Sender Report)
and XR (Extended Report) messages.

In the same spirit as adaptive jitter-buffering, the D-SBG is transcoding the
RTP stream and is not terminating it.  As a result, the adaptive jitter
buffering is considered part of the network.  RTCP reports are not generated
by the transcoding: transmitted RTCP packets are simply 1:1 translated
versions of received RTCP packets.

\subsection{DTMF Digits and Tones}

Many low bit-rate (LBR) codecs are incapable of successfully encoding DTMF
digits.  This is primarily because LBR codecs make assumptions that the media
content is voice.  Transport of DTMF digits can be performed in two ways: 1),
specification of DTMF digits as signalling events using SIP; and, 2),
specification of DTMF digits using a, possibly redundant, RTP payload.

From the perspective of forwarding, any RTP stream that is forwarded will also
contain DTMF RTP payloads when necessary.

From a transcoding perspective, however, some codecs (e.g. G.711) will carry the
DTMF digits in-band; whereas other (e.g. G.729) will carry them out of band
using one of the methods above.  When transcoding, there is a point at which the
out-of-band DTMF digits must be played and mixed in-band; and, a point at which
DTMF digits must be detected in-band and converted to out-of-band signals.

For the most part, for the LTE/WiMAX transcoding application, DTMF digits will
originate from mobile stations in the mobile network (out-of-band beside G.728
or G.729 codecs) rather than the PSTN.  So when transcoding from G.728 or G.729
to G.711 for PSTN  use, the out-of-band DTMF digits will have to be played and
juxtaposed with the media stream when transcoding from G.729 (for example) to
G.711.  The question is one of functional placement within the transcoding media
stack.  Two positions are opportune: at the top of the DTMF digit media
transcoding stream before delivering PCMU, PCMA or L16 media data to the top of
the CH-MUX multiplexing driver; or, at the top of the G.711 media transcoding
stream (on the way down).  The former cannot handle DTMF digits out-of-band
within the MG driver or application above the CH-MUX; the later can.

Because the MG application or module above the CH-MUX driver must be able to
handle DTMF digit events separate from the media stream, it is necessary to
choose the second positioning (DTMF digit media generation is handled at the top
of the G.711 codec media stack on way down).

In the other direction, the top of the G.711 codec media stack will also require
DTMF detection to be performed at the last stage and out-of-band DTMF digit
events reported to the top of the CH-MUX and possibly to the MG application or
module above.  The MG application or module, of course, needs to specify whether
DTMF digit detection is required or not, as part of the transcoding of the
stream.

Tones, on the other hand, are more likely to originate from the PSTN and require
delivery to the LTE/WiMAX network using out-of-band tones.  With SS7
connectivity of the carrier's internal MGC/MG complex, however, the MGC should
indicate most conditions using SIP signalling rather than attempting to indicate
those with in-band tones.

\subsection{Symmetric RTP}
\label{section:symrtp}

For general purpose RTP handling in the face of NAT and NAPT, the remote IP
address and port associated with the RTCP packet stream may be different from
the recommendations of RFC
3550.\footnote{RFC 3550 states that the RTCP port number should be odd and one
higher than the corresponding RTP port number, and with the same IP address.
This is not necessarily the case when NAT and NAPT middle-boxes exist in the
path.}

There are two ways to handle NAT/NAPT: ICE \cite{ICE} (using STUN \cite{STUN})
and latching.  With ICE, STUN messages are sent by the remote system to the
D-SBG's RTP and RTCP ports to determine the appropriate IP addresses and port
number for NAT/NAPT traversal.  The remote system will then report an
appropriate IP address and port combination for RTP and RTCP in the SDP Answer.
To accommodate this, the D-SBG should demultiplex and respond appropriately to
STUN messages received on the RTP and RTCP ports.  ICE/STUN has many problems
and impacts post-dial delay and early media and should not be used in gateway
scenarios.

With latching, RTP/RTCP streams are only half-formed (have only a local IP
address and port number) until a message is received on each port from the
remote system.  The D-SBG then latches to the source address contained in the
first messages received from the remote system.  These messages, of course, must
pass validation.  To accommodate this approach, the D-SBG should support such
latching.

Nevertheless, in the configuration most expected for LTE or WiMAX operation, or
in the case of VAR RTP forwarding, calls are expected to progress from the
S-SBG/D-SBG to an internal MGC/MG network, or pass out over the internet to a
remote S-SBG/D-SBG proxy.  In both cases, there should be not need for NAT/NAPT.
Any NAT/NAPT that needs to be performed is performed by the D-SBG itself.

\subsection{STUN Packets}

Interactive Connectivity Establishment (ICE) is described in {RFC 5245}
\cite{ICE}.  Simple traversal of UDP through NATs (STUN), is described in {RFC
3489} \cite{STUN}.  ICE sends STUN packets to RTP and RTCP ports on remote
systems expecting a response.  See the RFCs for details.  Because STUN
responses also need to be sent from UDP addresses and ports different from the
normal RTP and RTCP port numbers (to detect full or partial cone), STUN
packets need to be demultiplexed at the UDP driver and processed independent
of the media stack.  This is true for both the forwarding and transcoding
scenarios, because STUN packets need to terminate on the D-SBG.

\subsection{Early Media}
\label{section:early}

As discussed briefly in {\sl Section \ref{section:genreq}, page
\pageref{section:genreq}}, the carrier's LTE/WiMAX MGC/MG complex may be
outdated and without warranty.  This means that neither software nor hardware
upgrades are necessarily available.  Handling of early media by such systems
might be non-optimal.  Provisional reliable response and update methods for
early media might not be available.  Therefore, the media stack must be
prepared to receive candidate payload as early media.  Typically, and for
security considerations, this is likely only internal to the boundary formed
by the S-SBG/D-SBG: that is, facing the legacy MGC/MG complex.  External
interconnect can have stricter requirements.  Therefore, the media stack, and
in particular the UDP driver at the bottom, must be capable of being
instructed to permit early media on a termination point by termination point
basis.

Note that this issue is related to latching as described in {\sl Section
\ref{section:symrtp}, above}.

\subsection{Redundant Media}
\label{section:redmed}

\subsection{Carrier Grade Redundancy}
\label{section:grade}

Because the LTE and WiMAX application would typically be housed within a
incumbent carrier or CLEC's network, and due to the large scale planned for the
device, carrier grade redundancy will likely be necessary.  Link failures can
easily be accommodated with {\it Ethernet Bonding}, or by using APS (Automatic
Protection System) for 10Gig Ethernet.  Node failures, on the other hand, must
be handled by the system architecture and design.

\begin{figure}[htp]
\center\includegraphics[width=3.0in]{xc09}
\caption{Carrier Grade Redundancy}
\label{figure:xc09}
\end{figure}

Design for redundancy consists of decomposing the S-SBG and D-SBG functions
within the integrated platform, and providing multiple platforms that can handle
both functions for any failed platform within the cluster.  This can be
performed in a {\tt 1+1} or {\tt N+1} arrangement.  A {\tt 1+1} arrangement is
illustrated in {\sl Figure \ref{figure:xc09}, above}.

To maintain D-SBG synchronization between mated D-SBG, an H.248 interface can be
exposed between mated D-SBG.  When the S-SBG function establishes connectivity
through the D-SBG, the mated D-SBG can be sent the same connectivity
instructions using H.248 over an inter-D-SBG link.  SCTP (Stream Control
Transmission Protocol) with binary H.248 encoding should be used for this
purpose.  This synchronization results in a switching matrix with the mated
D-SBGs that is identical.  To handle D-SBG node failures, then, consists of
performing an IP takeover (gratuitous ARP) or APS fail-over between nodes.

Barring the availability of an H.248 interface to the D-SBG (primarily due to
the lack of a suitable stack), a custom protocol can be implemented between
mated D-SBG.  SCTP could also be used for transport of this protocol.  The
protocol could be a simple transaction update, where a checkpoint state is
transmitted between MG implementations.  It could also be a simple transport
of the MGI interface from one machine to another.  The MG implementation used,
illustrated in blue in {\sl Figure \ref{figure:xc09}}, is the {\sl OpenSS7} MG
Driver.  This is a kernel-space {\sl STREAMS} implementation.

To maintain S-SBG synchronization, a custom protocol can be implemented between
mated S-SBG.  SCTP could also be used for transport of this protocol.  The
protocol would be a simple transaction update, where a checkpoint state is
transmitted between SIP implementations.  The SIP implementation used,
illustrated in red in {\sl Figure \ref{figure:xc09}}, is the {\sl sofia} SIP
stack.  This is a user-space SIP implementation.

\subsubsection{Redundancy Architecture}

{\sl Figure \ref{figure:xc10}} illustrates the basic redundancy architecture.

\begin{figure}[htp]
\center\includegraphics[width=3.2in]{xc10}
\caption{Redundancy Architecture}
\label{figure:xc10}
\end{figure}

\paragraph{MG Features.}

The redundancy architecture has the following MG features:

Virtual MGs (VMG) have a working copy on one physical MG, and a redundant copy
on zero or more other physical MGs.
A Proxy MGC/MG (MG Cluster Controller, or MGCC) is used either collocated with
a physical MG or physical MGC.  The MGCC is responsible for acting like a
single, reliable, MG toward an MGC, and replicating state across the redundant
copies.
Virtual MG (VMG) copies normally communicate through the MGCC on the primary
(working) physical MG, for that VMG.

MGCs have a working copy on one physical MGC, and a redundant copy on zero or
more other physical MGCs.  This need not be a full virtual MGC, but can be a
partition within the working MGC for the physical MGC.
When a physical MG fails, its redundant copy on another physical MG will
eventually detect the loss of communications with the working physical MG.
Because it loses communication, it attempts to first contact the MGCC at the
working MGC.  When the communications attempt is successful, the working MGC
MGCC will have more information concerning the health of the working physical
MG because it either communicates with it or not.
When a protect VMG looses communication with the working VMG's MGCC and cannot
communicate with the working MGC, it fails over to attempting communications
with the protect MGC.  Again, the protect MGC will have more information
concerning th health of the physical MG containing the working VMG because it
either communicates with it or not.

\paragraph{MGC Features.}

The redundancy architecture has the following MGC features:

Each MGC cluster has a working copy on one physical MGC, and a redundant copy
on zero or more other physical MGC.  This need not be a full virtual MGC, but
can be a partition within the working MGC for the physical MGC.
Working and Protect MGC partitions do not necessarily communicate direct with
each other.  Some mechanism (such a HACMP IP takeover) is used for switchover
of call control sessions (e.g. SIP messaging); however, MEGACO/H.248 service
ports are different.

Communications linkages between MGC and MG use SCTP (Stream Control
Transmission Protocol) for best link failure detection, multi-homing, flow
control, and CMT.  Communications linkages between all necessary entities are
maintained at all times.  This provides an indication of the health of
physical MGC/MG entities at the other end of the SCTP association, and some
indication of the health of other entities.

MGC do not necessarily have to provide any means for direct state sharing
between the on-line (Working) and off-line (Protect) copies of the MGC.
Because the MGCs cannot function correctly (except to simply release calls)
without a functional MG anyway, MGC checkpoint state is saved on the VMG
images using the MGC information package.\cite[]{H24845}.
      
Unfortunately, the MGC information package will only store 128 octets per
termination, meaning that complete session state information (aside from what
information is already stored on the MG) must be contained within 256 octets.
Also, 128 octets of information can be saved on the root termination.  This
information could contain information about the last active VMG.

\subparagraph{MGC Failover.}

When the Working MGC failure is detected by the Protection MGC, the MGC
perform whatever IP takeover is necessary to acquire all of the SIP
communications that was destined for the old system.  In addition, the Protect
MGC begins to initiate actions necessary to assume control of the associated
VMG.

Because each MGC expects to have SCTP associations formed for communications
at all times (regardless of the {\tt ServiceChange} state of the MG), the
Protection MGC can easily determine whether the various VMG images are
accessible.  Further, when a VMG registers with the Protection MGC, additional
information can be determined from the {\tt ServiceChangeMethod}, which is one
of:

\begin{itemize}

\item{\it Graceful:} indicates that the specified terminations will be taken
out of service after the specified service change delay; established
connections are not yet affected, but the MGC should refrain from establishing
new connections and should attempt to gracefully tear down existing
connections on the termination(s) affected by the service change command.  The
MG should set the termination's service state at the expiry of the service
change delay or the removal of the termination from the active context
(whichever is first), to ``out of service.''

\item{\it Forced:} indicates that the specified terminations were taken
abruptly out of service and any established connections associated with them
may be lost.  For non-Root terminations, the MGC is responsible for cleaning
up the context (if any) with which the failed termination is associated.  At a
minimum, the terminations shall be subtracted from the context.  The
termination's service state should be ``out of service''.  For the root
termination, the MGC can assume that all connections are lost on the MG and
thus can consider that all terminations have been subtracted.

\item{\it Restart:} Indicates that service will be restored on the specified
terminations after expiration of the service change delay.  The service state
should be set to in-service upon expiry of the service change delay.

\item{\it Disconnected:} always applied with the root termination identifier,
indicates that the MG lost communication with the MGC, but it was subsequently
restored to the same MGC (possibly after trying other MGCs on a
pre-provisioned) list).  Since MG state may have changed, the MGC may wish to
use the Audit command to resynchronize its state with the MG's.

\item{\it Handoff:} sent from the MGC to the MG, this reason indicates that the
MGC is going out of service and a new MGC association must be established.
Sent from the MG to the MGC, this indicates that the MG is attempting to
establish a new association in accordance with a Handoff received from the MGC
with which it was previously associated.

\item{\it Failover:} sent from the MG to MGC to indicates the primary MG is
out of service and  secondary MG is taking over.  This service change method
is also sent from the MG to the MGC when the MG detects that MGC has failed.

\item{\it (Other mutually understood value):}

\end{itemize}

\paragraph{Change-over of VMG.}

H.248.1\cite[]{H2481} says that an MG will attempt to contact the primary MGC,
and then the secondary list in turn until it registers with an MGC.  This is
not a good way to provide carrier grade redundancy.  The VMGs will attempt to
register with {\b all} MGCs, and will maintain registration with each.  If
communications should be lost with an MGC, the VMG will attempt to reestablish
communications until it is successful.  If it should receive a hand-off from
an MGC, it will still maintain communications with that MGC.  The MG will
persist in forming SCTP associations with all MGC.  The inability to form an
SCTP association between an MG and any of the MGCs will be deemed a failure of
the MGC.

\subparagraph{Cold Start.}

The following steps are taken by an MGC upon registration with an MG:

\begin{enumerate}

\item Each MGC image will alway accept establishment of a control association
with each of the VMG images.

\item On an intial (cold start) of a VMG, the  VMG will register with service
change method `{\it Restart}.'  When the MGC is the active MGC image, and there
is an active VMG image registered, the MGC will begin synchronizing state with
the cold starting VMG image to bring it from cold-standby to hot-standby.  The
standby status of the VMG remains cold-standby until this process completes,
and then it moves to hot-standby.  When the MGC is a protecting image, no
further action is taken.

\item If a VMG should become disconnected from an MGC, it will reconnect and
register with service change method `{\it Disconnected}'.  When the MGC is the
active MGC image, the MGC will begin synchronizing state between the MGC and
the MG by applying all updates that occurred during the period of loss of
communication (typically only the release of sessions that could not be
released during the control association outage).  This is the case regardless
of whether the VMG image was active or standby.  During the resynchronization
process, the standby status of a protection VMG is cold-standby.  After
synchronization the protecting VMG moves to hot-standby status.  If the VMG
was active and is still the active image, it will be marked with an
availability status of degraded.  When the MGC is a protecting image, no
further action is taken.

\end{enumerate}

\subparagraph{Fail Over.}

When the MG detects a failure, the MG can send service changes on the root
terminations to indicate the failure as follows:

\begin{enumerate}

\item The MG, when it detects the failure of the active MG, will send a
service change on the root termination with method `{\it Failover}' to the
active MGC.  The service change address will identify the failed MG.  The MGC
may have already detected failure of the identified MG through other means
(e.g. the loss of an SCTP association or control association to the identified
MG).  If the VMG was in cold-standby (active synchronization), the VMG will
move on-line with a degraded availability status.

\item The MG, when it detects the failure of the active MGC, will send a
service change on the root termination with method `{\it Failover}' to the
standby MGC.  The service change address will identify the failed MGC.  The
MGC may have already detected failure of the identified MGC through other
means (e.g. HACMP hearbeat).  The MGC will begin state synchronization with
the MG to recover MGC state with the VMG.  When the VMG is a standby VMG, the
MGC will prefer to synchronize with the active VMG image instead of its
standby images.  The standby status of the MGC will be cold-standby prior to
this event, and will move to providing-service.

\end{enumerate}

\subparagraph{Switch Over.}
\label{section:switchover}

Manual or forced switch-over from an active MG to a protecting MG is performed
for a number of reasons.  The most common reason is for maintenance activity
on the working MG, such as hardware or software upgrades.  Manual switch-over
from an active MG is performed local to the element (at the EMS), or from a
management station (NMS).  The {\sl Optranex10G} supports both.  Switch over
invoked by NMS is performed using SNMP.

A given physical MG will typically be provisioned with working VMG images as
well as protection VMG images.  Switch-over is only applicable to the working
images, whereas, lock-out is applicable to the protection images, (see {\sl
Section \ref{section:lockout}}).

Although all control associations are established and maintained at all times,
the MG can send service changes on the root termination to indicate
switch-over conditions as follows:

\begin{enumerate}

\item When a manual or forced switch-over is invoked on an active VMG, the
active VMG will send a service change on the root termination with service
method `{\it Handoff}' and the service change address of the MG to which the
switchover should occur.  The MGC may accept or refuse the switch-over by its
response (error or success).  Some reasons for refusing the switch-over are:
the VMG to which to switch has a cold-standby standby status instead of
hot-standby (it is in the process of being synchronized); or, the VMG is not
currently accessible to the MGC.

\end{enumerate}

\subparagraph{Lock Out.}
\label{section:lockout}

Lock out is a condition that is used instead of switch-over for a protecting
VMG.  That is, to lock out an off-line, protection VMG image.  The locked out
condition keep other VMG and MGC from switching to the protection VMG image.
Again, the most common reason for lock out is for maintenance activity on the
protection MG, such as hardware or software upgrades.  Lock out of a standby
MG is performed local to the element (at the EMS), or from a management
station (NMS).  The {\sl Optranex10G} supports both; however, the local EMS
uses the same SNMP interface as the management station (NMS).

A given physical MG will typically be provisioned with working VMG images as
well as protection VMG images.  Lock out is only applicable to the protection
images, whereas, switch-over is applicable to the working images, (see {\sl
Section \ref{section:switchover}}).

To indicate that it is requesting lock-out, an MG may send a service change on
the root termination to the MGC to indicate lock-out conditions as follows:

\begin{enumerate}

\item When a manual or forced lock-out is invoked on an active VMG, the active
VMG will send a service change on the root termination with service method
`{\it Handoff}' and the service change address of the active MG (if it is
known, otherwise no service change address).  The MGC may accept or refuse the
lock-out by its response (error or success).  Some reasons for refusing the
switch-over are:  the VMG to be locked out is the only VMG that is
synchronized with the MGC; or, no other VMG is accessible.

\end{enumerate}

\subparagraph{Shut Down.}

At some times it may be necessary to shut down parts or all of a VMG.  This
can be invoked at the MG using local EMS or remotely via NMS.  A service
change command is sent to all MGC to indicate the shut down condition as
follows:

\begin{enumerate}

\item Shutdown of part of all of a VMG (all images) is performed using the
administrative state of the terminations invoked.  Shutting down the entire
VMG (all images) can be graceful or forced.\footnote{Graceful shutdown is
performed by setting the administrative state to {\it shutting-down}; forced
shutdown is performed by setting the administrative state to {\it locked}.}

\item For a abortive shutdown of all or part of the VMG (all images), a
service change on the appropriate terminations with service change method
`{\it Forced}' is performed.  For a graceful shutdown of all or part of the
VMG (all images), a service change on the appropriate terminations with
service change method `{\it Graceful}' is performed.

\end{enumerate}

\paragraph{Change-over of MGC.}

\paragraph{Partitioned systems.}

Difficulties can, of course, occur with a completely partitioned system.  That
is, if the system in {\sl Figure \ref{figure:xc10}} has its communications
network severed vertically down the middle.  Therefore, some reliable
mechanism must be provided to maintain reliable communication across the
vertical that will indicate the health of each partition to the other.

\subsection{MEGACO/H.248 Considerations}

A MEGAGO/H.248 stack\cite[]{H2481} is implemented in accordance with ITU-T
Recommendation H.248.1 (09/2005) and a number of packages.  The basic profile
is the D-SBG profile specified by ETSI\cite[]{BGF} and the MSF:\cite[]{MSF09}
ETSI ES 283 018 version 2.6.7.

\subsubsection{Ephemeral Termination Points}

The SBG is an IP-IP gateway.  Ephemeral termination points are normally chosen
by the D-SBG.  The S-SBG only specifies the IP realm or group (e.g.
`{\tt ip/<group>/\$}'); however, for redundancy purposes, it is necessary for
the MG cluster controller to be able to synchronize context state in redundant
Virtual MGs.  Therefore, ephemeral terminations will by given an orderly
naming that specifies the IP interface, address and port combination.  The SBG
profile requires that D-SBG label terminations
`{\tt ip/<group>/<interface>/<id>}', so for our purposes, termination points
will be labelled `{\tt ip/<realm>/<interface>/<addrindex>/<port>}'.  This way
the S-SBG can request `{\tt ip/<realm>/\$}', or simply `{\tt \$}' with a `{\tt
ipr=<realm>}' in the termination point descriptor.  Once the active D-SBG
image chooses an IP address and port number, it returns the termination point
identifier `{\tt ip/<realm>/<interface>/<addrindex>/<port>}' which can be used
to establish the identical context on the protecting D-SBG images.

\subsection{Invoking Forwarding vs. Transcoding}

As illustrated in {\sl Figure \ref{figure:xc04}} and {\ref{figure:xc07}}, RTP
forwarding is performed at a different level in the OpenSS7 media stack than
RTP transcoding.  This is because RTP forwarding is media-unaware,
media-transparent and media-neutral; does not require termination of RTP
media streams, and can most efficiently be handled directly at the UDP driver.

\subsection{Usage Record Collection}

For usage record collection for billing, it likely that the carrier's existing
MGC/MG solution provides sufficient usage record collection for the purposes of
billing.  Nevertheless, antiquated MGC/MG networks likely do not provide all of
the mechanisms necessary for performing interconnect reconciliation.  Therefore,
the S-SBG/D-SBG appliance should provide a mechanism for usage record collection
for the purposes of interconnect reconciliation and operational measurements and
OSS.

VARs, on the other hand, likely have usage record collection adequate for both
billing and reconciliation on their more modern MGC/MG configurations.
Therefore, for the forwarding application, it is not necessary to perform usage
record collection for the purpose of reconciliation.  Nevertheless, usage record
collection should be performed for the purpose of operational measurements and
OSS.


\section{Results}

\section{Analysis}

\section{Conclusions}

\section{Future Work}

\section{Related Work}

\clearpage
\section{Acronyms and Abbreviations} The following acronyms and
abbreviations are used throughout this document.

\setlength{\tabcolsep}{0.2em}
\setlength{\arraycolsep}{0.2em}
\begin{tabular}{ll}\\
ANSI   & {\it American National Standards Institute}\\
APS    & {\it Automatic Protection Switching}\\
ARP    & {\it Address Resolution Protocol}\\
CHI    & {\it Channel Interface} \cite[]{CHI}\\
CLEC   & {\it Competitive Local Exchange Carrier}\\
D-SBG  & {\it Data SBG}\\
DSP    & {\it Digital Signal Processor}\\
DTMF   & {\it Dual Tone Multi-Frequency}\\
EIA    & {\it Electronics Industry Association}\\
ETSI   & {\it European Telecommunications Standards Institute}\\
GSM    & {\it General Services Mobile}\\
ICE    & {\it Interactive Communications E...}\\
IP     & {\it Internet Protocol}\\
ITU    & {\it International Telecommunications Union}\\
ITU-T  & {\it ITU Telecommunications Sector}\\
LBR    & {\it Low Bit-Rate}\\
LTE    & {\it Long Term Evolution}\\
MEGACO & {\it Media Gateway Control}\\
MG     & {\it Media Gateway}\\
MGC    & {\it Media Gateway Controller}\\
MGCP   & {\it Media Gateway Control Protocol}\\
NAPT   & {\it Network Address and Port Translation}\\
NAT    & {\it Network Address Translation}\\
NPI    & {\it Network Provider Interface} \cite[]{NPI}\\
OSS    & {\it Operational Support System}\\
PCM    & {\it Pulse Code Modulation}\\
PCMA   & {\it PCM A-Law}\\
PCMU   & {\it PCM mu-Law}\\
PSTN   & {\it Public Switched Telephone Network}\\
RTCP   & {\it Real-Time Transport Control Protocol}\\
RTP    & {\it Real-Time Transport Protocol}\\
S-SBG  & {\it Signalling SBG}\\
SBG    & {\it Session Border Gateway}\\
SCTP   & {\it Stream Transmission Control Protocol}\\
SDP    & {\it Session Description Protocol}\\
SIMD   & {\it Single Instruction Multiple Data}\\
SIP    & {\it Session Initiation Protocol}\\
SR     & {\it Sender Report}\\
SS7    & {\it Signalling System No. 7}\\
SSE    & {\it Super Scalar Execution}\\
STUN   & {\it Simple Traversal of UDP through NATs}\\
TIA    & {\it Telecommunications Industry Association}\\
UDP    & {\it User Datagram Protocol}\\
VAR    & {\it Value Added Reseller}\\
XR     & {\it Extended Report}\\
\end{tabular}

\FloatBarrier
\addcontentsline{toc}{section}{References}
\bibliography{xcode}

\begin{appendix}

\section{Examples}

\end{appendix}

\end{document}
