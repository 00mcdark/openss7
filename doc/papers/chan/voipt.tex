% -*- tex -*- vim: ft=tex tw=100 nocin nosi
% =========================================================================
%
% @(#) doc/papers/chan/voipt.tex
%
% =========================================================================
%
% Copyright (c) 2008-2015  Monavacon Limited <http://www.monavacon.com/>
% Copyright (c) 2001-2008  OpenSS7 Corporation <http://www.openss7.com/>
% Copyright (c) 1997-2001  Brian F. G. Bidulock <bidulock@openss7.org>
%
% All Rights Reserved.
%
% Permission is granted to make and distribute verbatim copies of this
% manual provided the copyright notice and this permission notice are
% preserved on all copies.
%
% Permission is granted to copy and distribute modified versions of this
% manual under the conditions for verbatim copying, provided that the
% entire resulting derived work is distributed under the terms of a
% permission notice identical to this one.
% 
% Since the Linux kernel and libraries are constantly changing, this
% manual page may be incorrect or out-of-date.  The author(s) assume no
% responsibility for errors or omissions, or for damages resulting from
% the use of the information contained herein.  The author(s) may not
% have taken the same level of care in the production of this manual,
% which is licensed free of charge, as they might when working
% professionally.
% 
% Formatted or processed versions of this manual, if unaccompanied by
% the source, must acknowledge the copyright and authors of this work.
%
% -------------------------------------------------------------------------
%
% U.S. GOVERNMENT RESTRICTED RIGHTS.  If you are licensing this Software
% on behalf of the U.S. Government ("Government"), the following
% provisions apply to you.  If the Software is supplied by the Department
% of Defense ("DoD"), it is classified as "Commercial Computer Software"
% under paragraph 252.227-7014 of the DoD Supplement to the Federal
% Acquisition Regulations ("DFARS") (or any successor regulations) and the
% Government is acquiring only the license rights granted herein (the
% license rights customarily provided to non-Government users).  If the
% Software is supplied to any unit or agency of the Government other than
% DoD, it is classified as "Restricted Computer Software" and the
% Government's rights in the Software are defined in paragraph 52.227-19
% of the Federal Acquisition Regulations ("FAR") (or any successor
% regulations) or, in the cases of NASA, in paragraph 18.52.227-86 of the
% NASA Supplement to the FAR (or any successor regulations).
%
% =========================================================================
% 
% Commercial licensing and support of this software is available from
% OpenSS7 Corporation at a fee.  See http://www.openss7.com/
% 
% =========================================================================

\documentclass[letterpaper,final,notitlepage,twocolumn,10pt,twoside]{article}
\usepackage{ftnright}
\usepackage{makeidx}
\usepackage{pictex}
%\usepackage{psfig}
%\usepackage{graphics}
\usepackage{graphicx}
\usepackage{eepic}
%\usepackage{epsfig}
%\usepackage[dvips]{graphicx,epsfig}
%\usepackage[dvips]{epsfig}
%\usepackage{epsf}
\usepackage{natbib}
\usepackage{placeins}
%\usepackage{placeins}

\setlength{\voffset}{-1.2in}
\setlength{\topmargin}{0.2in}
\setlength{\headheight}{0.2in}
\setlength{\headsep}{0.3in}
\setlength{\topskip}{0.0in}
\setlength{\footskip}{0.3in}
\setlength{\textheight}{10.0in}

\setlength{\hoffset}{-1.0in}
\setlength{\oddsidemargin}{0.5in}
\setlength{\evensidemargin}{0.5in}
\setlength{\textwidth}{7.5in}

\setlength{\marginparwidth}{0.0in}
\setlength{\marginparsep}{0.0in}

\setlength{\columnsep}{0.3in}
\setlength{\columnwidth}{3.6in}
%\setlength{\columnseprule}{0.25pt}

\setlength{\paperheight}{11in}
\setlength{\paperwidth}{8.5in}

\let\Huge = \huge
\let\huge = \LARGE
\let\LARGE = \Large
\let\Large = \large
\let\large = \normalsize
\let\normalsize = \small
\let\small = \footnotesize
\let\footnotesize = \scriptsize
\let\scriptsize = \tiny

\makeatletter
\renewcommand\section{\@startsection {section}{1}{\z@}%
                                   {-2ex \@plus -1ex \@minus -.2ex}%
                                   {1ex \@plus .2ex}%
                                   {\normalfont\large\bfseries}}
\renewcommand\subsection{\@startsection{subsection}{2}{\z@}%
                                     {-1.5ex \@plus -.5ex \@minus -.2ex}%
                                     {1ex \@plus .2ex}%
                                     {\normalfont\normalsize\bfseries}}
\renewcommand\subsubsection{\@startsection{subsubsection}{3}{\z@}%
                                     {-1.25ex\@plus -.5ex \@minus -.2ex}%
                                     {1ex \@plus .2ex}%
                                     {\normalfont\normalsize\bfseries}}
\renewcommand\paragraph{\@startsection{paragraph}{4}{\z@}%
                                    {1.5ex \@plus .5ex \@minus .2ex}%
                                    {-1em}%
                                    {\normalfont\normalsize\bfseries\slshape}}
\renewcommand\subparagraph{\@startsection{subparagraph}{5}{\parindent}%
                                       {0ex \@plus 0ex \@minus 0ex}%
                                       {-1em}%
                                      {\normalfont\normalsize\bfseries\slshape}}
\makeatother

\setcounter{tocdepth}{3}
\setcounter{secnumdepth}{6}

\pagestyle{plain}
%\pagestyle{myheadings}
%\markboth{B. Bidulock}{B. Bidulock}

\makeglossary

\newcommand{\topfigrule}{\vspace{0.5ex}\rule{\columnwidth}{0.4pt}\vspace{0.5ex} }
\newcommand{\botfigrule}{\vspace{0.5ex}\rule{\columnwidth}{0.4pt}\vspace{0.5ex} }
\newcommand{\dblfigrule}{\vspace{0.5ex}\rule{\textwidth}{0.4pt}\vspace{0.5ex} }

%\bibliographystyle{unsrtnat}
%\bibliographystyle{plainnat}
%\bibliographystyle{ieeetr}
%\bibliographystyle{abbrvnat}
%\bibliographystyle{acm}
%\bibliographystyle{plainnat}
\bibliographystyle{alpha}

\begin{document}

%\begin{titlepage}
%\begin{center}
%    STREAMS vs. Sockets Performance Comparison\\
%    Experimental Test Results
%\end{center}
%\end{titlepage}

\title{High Density SONET/SDH MG Interface for VoIP Trunking\\[0.5ex]
	{\large \textsl{White Paper}}}
\author{Brian Bidulock\thanks{bidulock@openss7.org}\\
	The OpenSS7 Project}
\date{April 24, 2009}
\maketitle

\begin{abstract}
\addcontentsline{toc}{section}{Abstract}
This whitepaper discusses solutions to address the need for highly
cost-efficient (less than \$1000) large-scale (32,000 circuit) interface cards
for Media Gateways (MG) and integrated MG/MGC VoIP trunking using open-source
software.
\end{abstract}

% \tableofcontents

\section[Background]{Background}

% OpenSS7 OC-48 drivers and SIP Trunking set to disrupt VoIP Industry

Interworking of voice over IP to PSTN trunking has traditionally used a
decomposed architecture of Media Gateway (MG) and Media Gateway Controller (MGC
or SoftSwitch).  Media Gateways (MG) have been typically designed using Time
Division Multiplexing (TDM) components and principles suitable for the Public
Switched Telephone Network (PSTN).  This use of hard real-time TDM components in
access card and voice processing designs has led to reduced scalability through
the inappropriate application of PSTN TDM principles to VoIP trunking designs.

For example, the application of H.110 TDM busses and DSP resources in card and
system designs has increased the cost of MG subsystems as well as reducing
scalability, increasing vendor lock-in, and dedicating special purpose hardware
where low-cost scalable commodity host processing power would more than suffice.

Maximization of commodity host processor utilization and software
approaches\footnote{Such as the high-performance Linux Fast-STREAMS
\cite[]{LfS}.} leads to greatly reduced interface card per-channel costs as well
as providing for greater densities and scalability, and removing vendor lock-in
with a Commodity Off-The Shelf (COTS) solution.

\textsl{Table \ref{table:channel}} lists the various costs of legacy solutions
in comparison to the current design.
\begin{table}[htp]
\footnotesize
\begin{center}
\setlength{\tabcolsep}{0.3em}
\setlength{\arraycolsep}{0.3em}
\begin{tabular}{llcrr}\\
Approach & Card & Channels & Cost & Circuit\\
\hline
ES & Tor II & 96 & \$250.00 & \$2.50\\
ES & Tor III (8xE1) & 248 & \$1,400.00 & \$5.65\\
CG & Tor III (12xE1) & 372 & \$2,100.00 & \$5.65\\
CG & A300 (1xDS3) & 672 & \$3,000.00 & \$4.64\\
CG & PCI-532DE (2xDS3) & 1344 & \$4,000.00 & \$3.00\\
CG & A300 (3xDS3) & 2016 & \$9,000.00 & \$4.64\\
cPCI & OC-3 PTMC & 2016 & \$3,000.00 & \$1.50\\
cPCI & OC-12 PTMC & 8064 & \$10,000.000 & \$1.25\\
\hline
PCI & OC-48 (PCI) & 32356 & \$1000.00 & \$0.03\\
\hline
\end{tabular}\\
\caption{Solution Cost per Channel}
\label{table:channel}
\end{center}
\normalsize
\end{table}
The analysis provides the per-channel cost of various interface solutions
considering only the cost of interface equipment (and not the cost of the host
processors).  As can be seen from \textsl{Table \ref{table:channel}}, the cost
per channel of legacy approaches range from about \$1.25 per channel to \$5.65
per channel \textit{just for interface cards}, whereas the SONET/SDH design
presented in this paper provides for a cost of \$0.03 per channel: a dramatic
decrease.

\section[Tormenta Lessons]{Tormenta Lessons}

In general the low-density quad E1/T1/J1 Tormenta II and III cards were quite
successfully and are still widely manufactured and used.  When the Tormenta II
card was originally designed (2000, 2001 and 2002) the host equipment that could
house such a card were based on 32-bit 33 MHz PCI designs that could not sustain
full bus rate burst speeds.  This PCI design replaced the Tormenta I cards which
was an ISA card.  The ISA bus could not sustain a transfer rate that would
permit more than 2 T1s to be provided on the same card.  Competing proprietary
designs at the time (such as that provided by Pika) ran to \$5000 per card for a
single T1, or over \$50.00 per channel.

The Tormenta II and III card design is illustrated in \textsl{Figure
\ref{figure:voipt_fig05}}.
\begin{figure}[htp]
\center\includegraphics[width=3.5in]{voipt_fig05}
\caption[Tormenta]{Tormenta PCI}
\label{figure:voipt_fig05}
\end{figure}
The card consists of:

\begin{itemize}
	\item \textsl{Line interface transformers and connectors.}
		LIU transformers and RJ-45 PCI card faceplate connectors for
		interfacing 4 E1/T1/J1 lines. (100 Ohm balanced for T1 or J1 and
		120 Ohm balanced for E1.)
	\item \textsl{Framer and LIU.}
		Integrated quad E1/T1/J1 framers and line interface units
		(Maxim Dallas DS21Q352 (T1), DS21Q354 (E1) for Tormenta II; or
		DS2155 (E1/T1/J1) for Tormenta III).
	\item \textsl{Field Programmable Gate Array.}
		The Xilinx Spartan 3A XC2S50 5C Field Programmable Gate Array
		(FPGA) primarily provides buffering between the 8.192 MHz
		byte-interleaved TDM bus and the PCI Interface.
	\item \textsl{PCI Interface.}
		The PLX Technology PCI 9030 interface chip is a slave 32-bit 33
		MHz PCI target.  This chip provides host processor access to FPGA buffered E1, T1 or
		J1 frames.
% 	\item \textsl{Miscellaneous components.}
% 		Assorted crystals, decoupling capcitors, terminating resistors,
% 		clock dividers and synthesizers, transorbs, serial EPROM, etc.
\end{itemize}

The success (both engineering and economic) of the cards has taught several
lessons:

\begin{enumerate}
	\item \textsl{Reduce hardware functions.}
		This increases the longevity of the card.  Functions performed
		by specialized hardware should require hardware solutions due to
		real-time constraints (e.g. framers and line interface units).
	\item \textsl{Keep the design open and free.}
		This permits improvements in the design or implementation to be
		widely adopted.  It also permits a shared learning curve,
		promotes competitive pricing, and avoids vendor lock-in.
% 		This acknowledges the fact that hardware margins are razor thin
% 		and attempting to extract monopoly rents from a hardware design
% 		is self-illusory.
	\item \textsl{Move data to the host processor early.}
		Once section timing, framing and overheads have been processed,
		the data payloads should be moved to the host processor for
		processing.
	\item \textsl{Use host processor for scale.}
		Permits these functions to be performed where performance in the
		industry is scaling the fastest: commodity processors.
% 		\footnote{Strangely, Digium went the other way with their
% 		low-density PDH interface cards: placing echo suppression on the
% 		card rather than performing it in the host processor, leading to
% 		a significant increase in card pricing and an underutilization
% 		of server host resources.}
\end{enumerate}

Application of these principles has led to a SONET/SDH design where synchronous
payload extraction and tributary alignment is performing using a SONET/SDH
framer device, but payload processing has been moved to the commodity host
processor by passing VC payload directly to and from host memory using PCI bus
mastering.

% \subsubsection[CompactPCI]{CompactPCI}
% 
% \subsubsection[AdvancedTCA]{AdvancedTCA}

% \section[Technologies]{Technologies}
% 
% \subsection[Low-Density PDH]{Low-Density PDH}
% 
% \subsection[High-Density PDH]{High-Density PDH}
% 
% \subsection[CompactPCI]{CompactPCI}
% 
% % Most CompactPCI or PTMC designs incorporate the use of an H.110 bus.
% % Difficulties associated with H.110, H.MVIP or CTBus designs are that channel
% % content is placed on the TDM bus and is inaccessible to the host processor.
% % These favor low density (2048 channel) designs where the TDM bus is directly
% % connected to DSP or other real-time TDM processing.  These designs are built to
% % perform processing within 125 microsecond intervals.  This is overkill for a
% % G.711 RTP VoIP trunking application and the extra cost of performing this
% % real-time processing is prohibitive to VoIP applications.  G.711 RTP is
% % typically codified in 10ms or 30ms windows, with 30ms windows being far more
% % popular than 10ms.  A 30ms window of G.711 contains 240 samples and processing
% % each sample within 125 microseconds is simply not necessary.  Also, most echo
% % cancellation requries a 120ms tail (or buffering of 12 x 10 millisecond windows
% % or 4 x 30 millisecond windows) increasing the amount of on-chip or on-board
% % high-speed memory necessary for these designs.
% % 
% % By moving voice sample payload directly into host memory, there is a far lesser
% % need for on-chip or on-board high-speed memory and yet packatization delays for
% % 10 millisecond or 30 millisecond windows can easily be met.\footnote{Note that
% % codecs other than plain 8-bit PCM (G.711 A or Mu law companding) require
% % packetizeation windows containing sample blocks of even greater than 30
% % milliseconds.}
% 
% \subsection[AdvancedTCA]{AdvancedTCA}

\section[Application]{Application to MG and VoIP Trunking}

Normally, TDM systems are hard real-time systems where processing is performed
primarily in specialized hardware and supervisory and control functions are
performed using software and more general purpose processors.  This results in
latency of voice samples through a system being significantly less than 1
millisecond (typically 125 microseconds).

Media Gateways and VoIP Trunking, on the other hand, have significant latencies.
Therefore, hard real-time specialized hardware approaches are inefficient for
MG and VoIP trunking designs.  Hardware designed to perform operations within
125 microseconds (e.g. DSP, TDM bus and switching chips) are inappropriate
considering the packetization delay of an RTP G.711 encoded packet is 30
milliseconds (or 2 orders of magnitude greater) and echo cancellation algorithms
use a 120 millisecond tail (or 3 orders of magnitude greater).

By avoiding use of these high-cost components to perform TDM processing within
125 microseconds, the cost of interface equipment can be significantly reduced.
Criteria for functional placement are as follows:

\begin{enumerate}
\item \textsl{Is hard real-time response absolutely required by the function?}
	For VoIP, the answer to this question for all processing of the voice
	samples is \textit{no}.  DSP (Digital Signal Processors), and TDM (Time
	Division Multiplexing) bus hardware, are an emphatic \textit{no}.
\item \textsl{Can the function be performed using software and memory?}
	This includes all voice sample processing which can easily be performed
	in host memory with general purpose processors and software.
\item \textsl{Can the function be performed more efficiently by waiting and block
processing?}
	For VoIP, this answer is \textit{yes} for all voice sample processing.
	This includes codecs, echo cancellation, transcoding, tone detection and
	tone generation.
\item \textsl{Must the data be examined by, or otherwise transferred to, the host?}
	For VoIP this answer becomes \textit{yes}: all data must pass through
	a host processor anyway.
%	\footnote{Note that the term \textsl{embedded} and the use of network
%	processors for the most part can be seen simply as a ploy by vendors to
%	encapsulate and contain intellectual property for which they attempt to
%	extract monopoly rents.}
\end{enumerate}

\section[SONET/SDH Card Design]{SONET/SDH Card Design}

\textsl{Figure \ref{figure:voipt_fig06}} illustrates the overall SONET/SDH line
card design, with a x4-lane PCIe option shown in grey.
\begin{figure}[htp]
\center\includegraphics[width=3.5in]{voipt_fig06}
\caption[OC48card]{SONET/SDH Interface Card}
\label{figure:voipt_fig06}
\end{figure}
This design bears significant resemblance to the Tormenta II and III card
designs.  The guiding principles for the SONET/SDH design are: move data to the
host as early as possible; avoid external TDM bus interfaces; and, avoid
performing functions in specialized hardware.  The SONET/SDH OC-48 interface
card design consists of the same types of integrated components: Optical
Interface, SONET/SDH Framer, FPGA, PCI Interface, and optional PCIe bridge.

The estimated costs of the primary components on the card, and representative
pluggable optical transceiver costs are listed in \textsl{Table
\ref{table:components}}.
\begin{table}[htp]
\footnotesize
\begin{center}
\setlength{\tabcolsep}{0.3em}
\setlength{\arraycolsep}{0.3em}
\begin{tabular}{lrrr}\\
Component & Qty. & Unit Cost & Extended Cost\\
\hline
SFP-OC3-SX & 4 & \$45.00 & \$180.00\\
SFP-OC3/12/48-SX & 2 & \$278.00 & \$556.00\\
\hline
PM5334A & 1 & \$315.00 & \$315.00\\
XC3S1400A & 1 & \$73.00 & \$73.00\\
PCI9656 & 1 & \$45.00 & \$45.00\\
PEX8114 & 1 & \$45.00 & \$45.00\\
\hline
Total (w/o Transcievers) & & & \$478.00\\
Total (w/ 4 x OC-3) & & & \$659.00\\
Total (w/ 2 x OC-48) & & & \$1,034.00\\
\hline
\end{tabular}
\caption{Representative Component Costs}
\label{table:components}
\end{center}
\normalsize
\end{table}
The design satisfies the objective of having a full SONET/SDH interface card
with a cost less than \$1,000.00 per card.\footnote{\$478.00 without
transceivers, \$659.00 with 4 x SFP-OC3-SX short reach bidirectional
transceivers, \$1,034.00 with 2 x SFP-OC3/12/48 15km reach 1+1 APS
transceivers.}

\subsection[Optical Transceivers]{Optical Transceivers}

The SFP (Small Form-Factor Pluggable) transceivers are illustrated in
\textsl{Figure \ref{figure:voipt_fig04}}.
\begin{figure}[htp]
\center\includegraphics[width=2.0in]{voipt_fig04}
\caption[Tranceivers]{SFP Transceivers}
\label{figure:voipt_fig04}
\end{figure}
The optical interface consists of 4 Small Form-Factor (SFP) receptacles for SPF
optical (OC-3/12/48) or electrical (STS-3e/STM-1e 75 Ohm coaxial mini-BNC)
tranceivers.  These accept hot-pluggable transceivers.  For the card's
application (which is primarly short-reach interface to a SONET/SDH ADM or
DCCS), only short reach (SX) modules are necessary; however, the pluggable
tranceiver modules permit a range of reaches.  The optical or electrical
transceivers can be any combination of same speed tranceivers with a total
aggregate 1+1 protected or unprotected OC-48 line rate.
That is, the 4 SFP SX transceivers can be one of:
\begin{itemize}
\item \textsl{1 to 2 x OC-48.}
	One or two (1+1 APS) 2488 MHz (OC-48) SFP optical transceivers: duplex
	SMF optical transceivers with 0-10 km reach.
\item \textsl{1 to 4 x OC-12.}
	Up to 4 622 MHz SFP SX optical transceivers in unprotected or 1+1 APS
	configuration: duplex SMF optical transceivers.
\item \textsl{1 to 4 x EC-3/OC-3.}
	Up to 4 155 MHz SFP SX electrical or optical transceivers in unprotected
	or 1+1 APS configuration: bidirectional MMF or duplex SMF optical
	transceivers or duplex 75 Ohm electrical transceivers.
\end{itemize}
At the time of writing, the cost of these optical transceivers can be as little
as \$45.00 USD (OC-3 SFP SX bidirectional SMF) in small volumes, from various
suppliers, but can vary in price widely depending on the rated frequency and
range.\footnote{For example, SFP short reach selectable rate OC-3/12/48
transceivers can run to \$280 per transceiver.  Because of this variation, it
is necessary to have a flexible card solution that can support a wide range of
frequencies so that the optimal choice of SFP tranceiver can be selected.}

\subsection[SONET/SDH Framer]{SONET/SDH Framer}

The SONET/SDH framer is illustrated in \textsl{Figure \ref{figure:voipt_fig01}}.
\begin{figure}[htp]
\center\includegraphics[width=2.5in]{voipt_fig01}
\caption[SONET/SDH Framer]{SONET/SDH Framer}
\label{figure:voipt_fig01}
\end{figure}
The SONET/SDH Framer and LIU can support 1+1 OC-48 interfaces, 4 x OC-12
interfaces, 4 x OC-3/EC-3 interfaces.  The SONET/SDH framers can support a
maximum (fully 1+1 protected) capacity of 32,256 voice channels.  The minimum
capacity is one VT1.5/TU11 or 24 channels.  The interface scales from 1.536 kbps
to 2.064384 Gbps.  The SONET/SDH Framer supports an internal telecom bus that is a
77.76 MHz 32-bit or 4x8-bit parallel data interface bus that is used to clock
VC-3, VC-4, VC-4-4c frames between the SONET/SDH chip and the FPGA.

The SONET/SDH framer chip selected is the PMC Sierra PM5334A chip.  This is a
2x2488 (1+1 OC-48) SONET/SDH framer and TU3 mapper.  At the time of writing,
this chip had a cost of approximately \$315.00 USD in low
volumes.\footnote{
% Another alternative that was not as flexible was the PMC Sierra PM5332 1x2488
% OC-48 SONET/SDH framer and TU3 mapper, approx. \$268.00 per chip.  The detractor
% for the PM5332 is its inability to support OC-3 transceivers.  A more flexible
% chip that provides low-order tributary processing as well is the PM5336B, but
% its detractor is cost: approx. \$620.00 per chip.
SONET/SDH framers are also available from other suppliers such as Mindspeed and
TranSwitch.}

\subsection[Field Programmable Gate Array]{Field Programmable Gate Array}

The Field Programmable Gate Array (FPGA) is illustrated in \textsl{Figure
\ref{figure:voipt_fig02}}.
\begin{figure}[htp]
\center\includegraphics[width=2.5in]{voipt_fig02}
\caption[FPGA]{FPGA}
\label{figure:voipt_fig02}
\end{figure}
A FPGA is used to tranlsate between the internal 77.76 MHz parallel telecom bus
provided by the SONET/SDH framer chip and internal FIFOs used for dynamic DMA
bus mastered transfers between the card and the host.  One of the most
significant costs in FPGA chips is the amount of on-chip RAM banks that are
available.  Extensive use of on-chip RAM significantly increases the cost of the
chip.

Therefore, this design, rather than internally buffering 1ms of channel data (as
does the Spartan 3A XC2S50 in the Tormenta II design), relies on the ability of
the PCI interface chip to perform bus mastered DMA burst transfers to host
memory.  A far reduced amount of on-chip RAM is required and is only used in a
FIFO arrangement to buffer burst transfers to and from the PCI bus.  Payload
self-synchronizing scrambling/descrambling may also performed in the FPGA.

A valueable characteristic of modern FPGAs is the ability to use readily
available low-cost or free tools and open-source VDHL and synthesis to permit
design changes increasing the openness of the card design.

The FPGA chip selected is the Xilinx Spartan 3A Extended XC3S1400A-FG484-5C
chip.  At the time of writing, this chip had a cost of approximately \$73.00 USD
per device in low volumes.\footnote{VHDL design synthesis may require a
different part.  Suitable devices are also provided by other suppliers such as
Altera.}

\subsection[PCI Interface]{PCI Interface}

The PCI Interface is illustrated in \textsl{Figure \ref{figure:voipt_fig03}}.
\begin{figure}[htp]
\center\includegraphics[width=2.5in]{voipt_fig03}
\caption[PCI Interface]{PCI Interface}
\label{figure:voipt_fig03}
\end{figure}
A 64-bit 66MHz PCI interface is used to interface the card either directly to
the host, or optionally to a 4-lane PCI Express bridge to the host processor.
The PCI interface is capable of scatter/gather dynamic bus mastered DMA with
descriptor rings (in a similar fashion to many Ethernet or ATM interface card
designs).  This permits a maximum burst transfer rate of 4.224 Gbps which is
close to the full channel capacity of the OC-48 interface which is 4.129 Gbps.
An optional x4-lane PCI Express bridge (8 Gbps) provides PCI express interface
with an additional 10K buffering and read-ahead capability.

The interface can also operate at 32-bit 33MHz (or optionally into 1-lane PCI
Express) and in this mode is capable of transfering the full channel capacity of
an OC-12 interface or 8064 channels.  The PCI 2.2 interface is capable of 3.3V
or 5.0V operation.

This PCI interface design choice permits either a PCI card to be used in PCI-X
riser in popular 1U and 2U server chasses.  Use of a traditional PCI interface
permits reuse of existing server equipment.  Optional PCIe interface supports
use in newer generation server equipment only providing PCIe card expansion.

The PCI interface chips selected are the PLX Technology, Inc. PCI 9656-BA66BIG
chip, and optional PEX 8114 chip.  At the time of writing, each of these chips
had a cost of approximately \$45.00 USD in low volumes.\footnote{Suitable
bus-master PCI interfaces and PCIe/PCI-X bridge chips are available from other
suppliers such as NEC.}

% \subsection[Misellaneous Components]{Miscellaneous Components}
% 
% \paragraph*{Crystals.} The PM5334 framer requires a 77.76 MHz crystal for
% generated syncrhonous timing (network side) but is also capable of regenerating
% a 77.76 MHz derived clock recovered from the syncrhonous line.  The PCI 9656
% interface chip requires a 66.66 MHz crystal.

\section[Software]{Software}

Software for the SONET/SDH interface card can utilize OpenSS7 open-source
software. Software consists of a multiplex (MX) driver, RTP driver and media
gateway (MG) multiplexing driver as illustrated in \textsl{Figure
\ref{figure:voipt_fig07}}.
\begin{figure}[htp]
\center\includegraphics[width=3.0in]{voipt_fig07}
\caption[Software]{Associated Software}
\label{figure:voipt_fig07}
\end{figure}

The Multiplex (MX) driver is a device driver designed specifically for use with
the SONET/SDH card, but utilizes an interface (MXI) that is common to all PDH
drivers.  The driver performs low-order path processing, VC mapping, VT/TU
overhead processing and TU-3/VC-3 framing.  Tributaries are demultiplexed into
primary VT1.5/TU11 or VT2/TU12 streams that are linked beneath the Media Gateway
(MG) multiplexing pseudo-device driver.

RTP streams are managed by the RTP driver which is also linked beneath the MG
multiplexing driver.

The Media Gateway (MG) multiplexing driver performs media gateway functions
subject to either local control or external H.248 connection.  This permits
either stand-alone MG operation or integrated MG/MGC operation.  The MG driver
performs interconnection between the VT channels of the MX driver and the RTP
streams of the RTP driver and performs transcoding, echo cancellation and T.38
Fax processing.

A Zaptel\footnote{Zapata Telephony} (ZT) driver can be provided for transparent
support of IAX, Asterisk, FreeSwitch and other Zaptel/IAX supporting soft
real-time applications.

In general, because SPE and payload are processed in the host, it is possible to
develop drivers for ATM (AAL1, AAL2, AAL5), PoS, and other SONET/SDH layer 2
protocols without alterations to the interface card design.

\section[Host Considerations]{Host Considerations}

Typically the primary factors driving Host selection (for housing the interface
cards and providing Host compute power for performing voice processing
functions) are environmental in nature.

When the Host is to be installed in a Central Office environment, NEBS and ETSI
conformance, such as DC power and local alarm display, is necessary.  When the
host is to be installed in a data center, NEBS and ETSI compliance is too
honerous and a reduced cost host can be selected that meets the environmental
considerations of the data center.  Carrier grade hosts currently include the
NEBS/ETSI certified IBM x3650T 2U chassis (based on the Intel/Kontron TIGI2U
design) or the NEBS/ETSI compliant Kontron TIGW1U (based on the Intel TIGW1U
design).

In general, however, any Host that meets the needs of the target environment is
suitable for the OC-48 card design.

\subsection[Back-End Connectivity]{Back-End Connectivity}

The larger scale of the SONET/SDH OC-48 interface card design challenges the
back-end connectivity of the Host.  Channelized OC-48 provides 2.064 Gbps duplex
(4.129 Gpbs) G.711 voice samples.  With an overhead of 20\% for RTP headers (G.711
30ms coded voice), this requires Ethernet connectivity with 4.955 Gbps link
capacity.  This requires quad GigE interfaces.  Hosts (such as the x3650T) that
provide dual GigE interfaces must be equipped with additional GigE ports (or
upgraded to 10G or 20G links) for back-end network connectivity to be able to
process the full capacity of a single OC-48 (or quad OC-12) interface card.

\section[Example Applications]{Example Applications}

Following are two application examples for very high density VoIP trunking.

\subsection[1U Enterprise Application]{1U Enterprise Application}

A typical 1U Enterprise application utilizes commodity 1U servers (e.g. such as
the 1U HP Proliant DL360 G6 server \cite[]{DL360G6} or 2U HP Proliant DL380 G6
server \cite[]{DL380G6}).  The application may provide 1 x OC-48 (unprotected or
1+1 APS) or 4 x OC-12 (unprotected or 1+1 APS) connections processing VoIP
trunking into 4 GigE rails.  The maximum processing capacity is 32,256 channels.
System hardware cost is approximately \$6,000 providing for a CAPEX of \$0.15
per channel.  CAPEX can be further reduced by retargeting existing or
decommissioned high power consumption commodity 1U servers.

\subsection[1U ITSP Application]{1U ITSP Application}

A typical 1U Internet Telephony Service Provider application utilizes the
Kontron TIGW1U NEBS/ETSI compliant server or equivalent.  The application
provides 1 x OC-48 (1+1 APS) connections processing VoIP trunking into 4 GigE
rails.  The maximum processing capacity is 32,256 protection switched channels.
System hardware cost is approximately \$8,000 providing for a CAPEX of \$0.25
per channel.

\subsection[2U Carrier Application]{2U Carrier Application}

\textsl{Figure \ref{figure:voipt_fig08}} illustrates a 2U Carrier grade
application using the IBM x3650T NEBS/ETSI certified (or Intel/Kontron TIGI2U or
Kontron TIGH2U) server.
\begin{figure}[htp]
\center\includegraphics[width=3.0in]{voipt_fig08}
\caption[TIGI2U Carrier Chassis]{TIGI2U Carrier Chassis (Rear View)}
\label{figure:voipt_fig08}
\end{figure}
Other options include the NEBS/ETSI certified HP Proliant DL385 G5 CG
server.\cite[]{DL385G5CG}  The application provides 4 x 2 (1+1 APS) OC-12
connections and 4 x OC-3 connections processing VoIP trunking into 6 GigE rails.
The maximum processing capacity is 40,320 protection switched channels.  System
hardware cost is approximately \$15,000 for a CAPEX of \$0.37 per channel.

The application consists of:

\begin{enumerate}
	\item Three SONET/SDH Full-Height, Half-Length PCI cards running at
		3.3V, 64-bits, 66 MHz, BDM.
	\item Two sets of four SFP-OC12-SX duplex single-mode fiber
		small form-factor pluggable transceivers.  Each of two SONET/SDH
		cards has 2 working and 2 protect OC-12 interfaces.
	\item One set of four SFP-OC3-SX bidirectional single-mode fiber small
		form-factor pluggable transceivers.  One SONET/SDH card houses
		all four OC-3 interfaces.
	\item Dual GigE rails (onboard).
	\item Three dual GigE half-height, half-length PCIe four-lane cards.
\end{enumerate}


\section[Summary and Conclusions]{Summary and Conclusions}

With the application of criteria suited for a high-density VoIP trunking
application, it is possible to significantly reduce the cost per channel of
interace equipment.  Significant savings in interface card design can be
accomplished by removing hard real-time hardware DSP, switching and multiplexing
components and relying instead on low-cost commodity host processing power for
voice processing.

Removing low-density multiservice access interfaces in preference for
high-density SONET/SDH access components dramatically increases the scalability
of the solution.  Such a design, although significantly reduced in cost per
channel, can acheive significant densitites at reduced cost yet permitting
high-end scalability.  Use of commodity host processing power in fact ensures
scalability and longevity of the overall solution.

\FloatBarrier
\addcontentsline{toc}{section}{References}
\bibliography{voipt}

% \clearpage
% \begin{appendix}
% \end{appendix}

\end{document}

% =========================================================================
%
% $Log: voipt.tex,v $
% Revision 1.1.2.2  2009-09-01 09:09:46  brian
% - added text image files
%
% Revision 1.1.2.1  2009-06-21 10:41:31  brian
% - added files to new distro
%
% Revision 0.9.2.1  2009-04-29 11:31:11  brian
% - added VoIP trunking whitepaper
%
% =========================================================================
% -*- tex -*- vim: ft=tex tw=100 nocin nosi
