%% -*- tex -*- vim: ft=tex tw=78 nocin nosi
%% =========================================================================
%%
%% @(#) $Id$
%%
%% =========================================================================
%%
%% Copyright (c) 2001-2008  OpenSS7 Corporation <http://www.openss7.com/>
%%
%% All Rights Reserved.
%%
%% Permission is granted to make and distribute verbatim copies of this
%% manual provided the copyright notice and this permission notice are
%% preserved on all copies.
%%
%% Permission is granted to copy and distribute modified versions of this
%% manual under the conditions for verbatim copying, provided that the
%% entire resulting derived work is distributed under the terms of a
%% permission notice identical to this one.
%% 
%% Since the Linux kernel and libraries are constantly changing, this
%% manual page may be incorrect or out-of-date.  The author(s) assume no
%% responsibility for errors or omissions, or for damages resulting from
%% the use of the information contained herein.  The author(s) may not
%% have taken the same level of care in the production of this manual,
%% which is licensed free of charge, as they might when working
%% professionally.
%% 
%% Formatted or processed versions of this manual, if unaccompanied by
%% the source, must acknowledge the copyright and authors of this work.
%%
%% -------------------------------------------------------------------------
%%
%% U.S. GOVERNMENT RESTRICTED RIGHTS.  If you are licensing this Software
%% on behalf of the U.S. Government ("Government"), the following
%% provisions apply to you.  If the Software is supplied by the Department
%% of Defense ("DoD"), it is classified as "Commercial Computer Software"
%% under paragraph 252.227-7014 of the DoD Supplement to the Federal
%% Acquisition Regulations ("DFARS") (or any successor regulations) and the
%% Government is acquiring only the license rights granted herein (the
%% license rights customarily provided to non-Government users).  If the
%% Software is supplied to any unit or agency of the Government other than
%% DoD, it is classified as "Restricted Computer Software" and the
%% Government's rights in the Software are defined in paragraph 52.227-19
%% of the Federal Acquisition Regulations ("FAR") (or any successor
%% regulations) or, in the cases of NASA, in paragraph 18.52.227-86 of the
%% NASA Supplement to the FAR (or any successor regulations).
%%
%% =========================================================================
%% 
%% Commercial licensing and support of this software is available from
%% OpenSS7 Corporation at a fee.  See http://www.openss7.com/
%% 
%% =========================================================================
%%
%% Last Modified $Date$ by $Author$
%%
%% =========================================================================

\documentclass[letterpaper,final,notitlepage,twocolumn,10pt,twoside]{article}
\usepackage{ftnright}
\usepackage{makeidx}
\usepackage{pictex}
%\usepackage{psfig}
%\usepackage{graphics}
\usepackage{graphicx}
\usepackage{eepic}
%\usepackage{epsfig}
%\usepackage[dvips]{graphicx,epsfig}
%\usepackage[dvips]{epsfig}
%\usepackage{epsf}
\usepackage{natbib}
\usepackage{placeins}
%\usepackage{placeins}

\setlength{\voffset}{-1.2in}
\setlength{\topmargin}{0.2in}
\setlength{\headheight}{0.2in}
\setlength{\headsep}{0.3in}
\setlength{\topskip}{0.0in}
\setlength{\footskip}{0.3in}
\setlength{\textheight}{10.0in}

\setlength{\hoffset}{-1.0in}
\setlength{\oddsidemargin}{0.5in}
\setlength{\evensidemargin}{0.5in}
\setlength{\textwidth}{7.5in}

\setlength{\marginparwidth}{0.0in}
\setlength{\marginparsep}{0.0in}

\setlength{\columnsep}{0.3in}
\setlength{\columnwidth}{3.6in}
%\setlength{\columnseprule}{0.25pt}

\setlength{\paperheight}{11in}
\setlength{\paperwidth}{8.5in}

\let\Huge = \huge
\let\huge = \LARGE
\let\LARGE = \Large
\let\Large = \large
\let\large = \normalsize
\let\normalsize = \small
\let\small = \footnotesize
\let\footnotesize = \scriptsize
\let\scriptsize = \tiny

\makeatletter
\renewcommand\section{\@startsection {section}{1}{\z@}%
                                   {-2ex \@plus -1ex \@minus -.2ex}%
                                   {1ex \@plus .2ex}%
                                   {\normalfont\large\bfseries}}
\renewcommand\subsection{\@startsection{subsection}{2}{\z@}%
                                     {-1.5ex \@plus -.5ex \@minus -.2ex}%
                                     {1ex \@plus .2ex}%
                                     {\normalfont\normalsize\bfseries}}
\renewcommand\subsubsection{\@startsection{subsubsection}{3}{\z@}%
                                     {-1.25ex\@plus -.5ex \@minus -.2ex}%
                                     {1ex \@plus .2ex}%
                                     {\normalfont\normalsize\bfseries}}
\renewcommand\paragraph{\@startsection{paragraph}{4}{\z@}%
                                    {1.5ex \@plus .5ex \@minus .2ex}%
                                    {-1em}%
                                    {\normalfont\normalsize\bfseries\slshape}}
\renewcommand\subparagraph{\@startsection{subparagraph}{5}{\parindent}%
                                       {0ex \@plus 0ex \@minus 0ex}%
                                       {-1em}%
                                      {\normalfont\normalsize\bfseries\slshape}}
\makeatother

\setcounter{tocdepth}{3}
\setcounter{secnumdepth}{6}

\pagestyle{plain}
%\pagestyle{myheadings}
%\markboth{B. Bidulock}{B. Bidulock}

\makeglossary

\newcommand{\topfigrule}{\vspace{0.5ex}\rule{\columnwidth}{0.4pt}\vspace{0.5ex} }
\newcommand{\botfigrule}{\vspace{0.5ex}\rule{\columnwidth}{0.4pt}\vspace{0.5ex} }
\newcommand{\dblfigrule}{\vspace{0.5ex}\rule{\textwidth}{0.4pt}\vspace{0.5ex} }

%\bibliographystyle{unsrtnat}
%\bibliographystyle{plainnat}
%\bibliographystyle{ieeetr}
%\bibliographystyle{abbrvnat}
%\bibliographystyle{acm}
%\bibliographystyle{plainnat}
\bibliographystyle{alpha}

\begin{document}

%\begin{titlepage}
%\begin{center}
%    STREAMS vs. Sockets Performance Comparison\\
%    Experimental Test Results
%\end{center}
%\end{titlepage}

\title{Implementing SIGTRAN for Linux Fast-STREAMS\\[0.5ex]
	{\large \textsl{Design for Linux}}}
\author{Brian F. G. Bidulock\thanks{bidulock@openss7.org}\\
	OpenSS7 Corporation}
\date{June 16, 2007}
\maketitle

\begin{abstract}
\addcontentsline{toc}{section}{Abstract}
\end{abstract}

%\tableofcontents

\section[Background]{Background}

UNIX networking has a rich history.  The TCP/IP protocol suite was first
implemented by BBN using Sockets under a DARPA research project on 4.1aBSD and
then incorporated by the CSRG into 4.2BSD \cite[]{bsd}.  Lachmann and
Associates (Legent) subsequently implemented one of the first TCP/IP protocol
suite based on the Transport Layer Interface (TLI) \cite[]{tli} and STREAMS
\cite[]{magic}.  Two other predominant TCP/IP implementations on STREAMS
surfaced at about the same time: Wollongong and Mentat.

\subsection[STREAMS]{STREAMS}

STREAMS is a facility first presented in a paper by Dennis M. Ritchie in 1984
\cite[]{Ritchie84}, originally implemented on 4.1BSD and later part of the
\textsl{Bell Laboratories Eighth Edition UNIX}, incorporated into \textsl{UNIX
System V Release 3} and enhanced in \textsl{UNIX System V Release 4} and
further in \textsl{UNIX System V Release 4.2}.  STREAMS was used in SVR4 for
terminal input-output, pseudo-terminals, pipes, named pipes (FIFOs),
interprocess communication and networking.  STREAMS was used in SVR3 for
networking (in the NSU package).  Since its release in \textsl{System V
Release 3}, STREAMS has been implemented across a wide range of UNIX,
UNIX-like and UNIX-based systems, making its implementation and use an ipso
facto standard.

STREAMS is a facility that allows for a reconfigurable full duplex
communications path, \textit{Stream}, between a user process and a driver in
the kernel.  Kernel protocol modules can be pushed onto and popped from the
\textit{Stream} between the user process and driver.  The \textit{Stream} can
be reconfigured in this way by a user process.  The user process, neighbouring
protocol modules and the driver communicate with each other using a message
passing scheme.  This permits a loose coupling between protocol modules,
drivers and user processes, allowing a third-party and loadable kernel module
approach to be taken toward the provisioning of protocol modules on platforms
supporting STREAMS.

On \textsl{UNIX System V Release 4.2}, STREAMS was used for terminal
input-output, pipes, FIFOs (named pipes), and network communications.  Modern
UNIX, UNIX-like and UNIX-based systems providing STREAMS normally support some
degree of network communications using STREAMS; however, many do not support
STREAMS-based pipe and FIFOs\footnote{AIX, for example.} or terminal
input-output\footnote{HP-UX, for example.} without system reconfiguration.

\textsl{UNIX System V Release 4.2} supported four Application Programming
Interfaces (APIs) for accessing the network communications facilities of the
kernel:

\begin{description}

\item[{\it Transport Layer Interface (TLI).}]

\textsl{TLI} is an acronym for the \textit{Transport Layer Interface}
\cite[]{tli}.  The \textsl{TLI} was the non-standard interface provided by
SVR3 and SVR4, later standardized by \textit{X/Open} as the \textsl{XTI}
described below.  This interface operated differently than the XTI in subtle
ways, and is now deprecated.

\item[{\it X/Open Transport Interface (XTI).}]

\textsl{XTI} is an acronym for the \textsl{X/Open Transport Interface}
\cite[]{xti}.  The \textsl{X/Open Transport Interface} is a standardization of
the \textsl{UNIX System V Release 4}, \textsl{Transport Layer Interface}.  The
interface consists of an Application Programming Interface implemented as a
shared object library.  The shared object library communicates with a
transport provider \textit{Stream} using a service primitive interface called
the \textit{Transport Provider Interface}\cite[]{tpi}.

While \textsl{XTI} was implemented directly over STREAMS devices supporting
the \textit{Transport Provider Interface (TPI)} \cite[]{tpi} under SVR4,
several non-traditional approaches exist in implementation:

\item[{\it Berkeley Sockets.}]

Sockets uses the BSD interface that was developed by BBN for the TCP/IP
protocol suite under DARPA contract on 4.1aBSD and released in 4.2BSD.  BSD
Sockets provides a set of primary API functions that are typically implemented
as system calls.  The BSD Sockets interface is non-standard, operated
differently from the POSIX interface in subtle ways, and is now deprecated in
favour of the POSIX/SUS standard Sockets interface.

\item[{\it POSIX Sockets.}]

Sockets were standardized by X/Open, later the
OpenGroup,\footnote{\textit{http://www.opengroup.org/}} and IEEE in the POSIX
standardization process.  They appear in XNS 5.2 \cite[]{xns}, SUSv1
\cite[]{susv1}, SUSv2 \cite[]{susv2} and SUSv3 \cite[]{susv3}.   POSIX/SUS
Sockets is now the common application environment for accessing networking,
deprecating the XTI for TCP/IP networking applications.

\end{description}

On systems traditionally supporting Sockets and then retrofitted to support
STREAMS, there is one approach toward supporting \textsl{XTI} without
refitting the entire networking stack:\footnote{This approach is taken by
True64 (Digital) UNIX.}

\begin{description}

\item[{\it XTI over Sockets.}]

Several implementations of STREAMS on UNIX utilize the concept of \textsl{TPI}
over Sockets.  Following this approach, a STREAMS pseudo-device driver is
provided that hooks directly into internal socket system calls to implement
the driver, and yet the networking stack remains fundamentally BSD in style.

\end{description}

Typically there are two approaches to implementing XTI on systems not
supporting STREAMS:

\begin{description}

\item[{\it XTI Compatibility Library.}]

Several implementations of XTI on UNIX utilize the concept of an XTI
compatibility library.\footnote{One was even available for Linux at one
point.}  This is purely a shared object library approach to providing
\textsl{XTI}.  Under this approach it is possible to use the \textsl{XTI}
application programming interface, but it is not possible to utilize any of
the STREAMS capabilities of an underlying \textit{Transport Provider Interface
(TPI)} stream.

\item[{\it TPI over Sockets.}]

An alternate approach, taken by the \textsl{Linux iBCS} package was to provide
a pseudo-transport provider using a legacy character device to present the
appearance of a STREAMS transport provider.

\end{description}

Conversely, on systems supporting STREAMS, but not traditionally supporting
Sockets (such as SVR4), there are four approaches toward supporting BSD and
POSIX Sockets based on STREAMS:

\begin{description}

\item[{\it Compatibility Library}]

Under this approach, a compatibility library (\texttt{libsocket.o}) contains
the socket calls as library functions that internally invoke the TLI or TPI
interface to an underlying STREAMS transport provider.  This is the approach
originally taken by SVR4 \cite[]{magic}, but this approach has subsequently
been abandoned due to the difficulties regarding fork(2) and fundamental
incompatibilities deriving from a library only approach.

\item[{\it Library and cooperating \sl STREAMS module.}]

Under this approach, a cooperating module, normally called \texttt{sockmod},
is pushed on a Transport Provider Interface (TPI) Stream.  The library,
normally called \texttt{socklib} or simply \texttt{socket}, and cooperating
\texttt{sockmod} module provide the BBN or POSIX Socket API.  \cite[]{impbsd}
\cite[]{socklib}

\item[{\it Library and System Calls.}]

Under this approach, the BSD or POSIX Sockets API is implemented as system
calls with the sole exception of the \textbf{\texttt{socket}}(3) call.  The
underlying transport provider is still an \textsl{TPI}-based STREAMS transport
provider, it is just that system calls instead of library calls are used to
implement the interface.  \cite[]{socklib}

\item[{\it System Calls.}]

Under this approach, even the socket(3) call is moved into the kernel.
Conversion between POSIX/BSD Sockets calls and TPI service primitives is
performed completely within the kernel.  The sock2path(5) configuration file
is used to configure the mapping between STREAMS devices and socket types and
domains \cite[]{socklib}.

\end{description}

\subsubsection[Standardization]{Standardization.}

During the POSIX standardization process, networking and Sockets interfaces
were given special treatment to ensure that both the legacy Sockets approach
and the STREAMS approach to networking were compatible. POSIX has standardized
both the XTI and Sockets programmatic interface to networking.  STREAMS
networking has been POSIX compliant for many years, BSD Sockets, POSIX
Sockets, TLI and XTI interfaces, and were compliant in the \textsl{SVR4.2}
release.  The STREAMS networking provided by \textsl{Linux Fast-STREAMS}
package provides POSIX compliant networking.

Therefore, any application utilizing a Socket or Stream in a POSIX compliant
manner will also be compatible with STREAMS networking.\footnote{This
compatibility is exemplified by the \texttt{netperf} program which does not
distinguish between BSD or STREAMS based networking in their implementation or
use.}

\subsection[Linux Fast-STREAMS]{Linux Fast-STREAMS}

The first STREAMS package for Linux that provided SVR4 STREAMS capabilities
was the \textsl{Linux STREAMS (LiS)} package originally available from GCOM
\cite[]{LiS}.  This package exhibited incompatibilities with SVR 4.2 STREAMS
and other STREAMS implementations, was bugger and performed very poorly on
Linux.  These difficulties prompted the OpenSS7 Project \cite[]{openss7} to
implement an SVR 4.2 STREAMS package from scratch, with the objective of
production quality and high-performance, named \textsl{Linux Fast-STREAMS}
\cite[]{LfS}.

The OpenSS7 Project also maintains public and internal version of the
\textsl{LiS} package.  The last public release was \textit{LiS-2.18.3}; the
current internal release version is \textit{LiS-2.18.6}.  The current
production public release of \textsl{Linux Fast-STREAMS} is
\textit{streams-0.9.3}.

\section[Objective]{Objective}


\section[Description]{Description}


\section[Method]{Method}


\section[Results]{Results}

\section[Analysis]{Analysis}

\section[Conclusions]{Conclusions}

\section[Future Work]{Future Work}

\section[Related Work]{Related Work}

\FloatBarrier
\addcontentsline{toc}{section}{References}
\bibliography{sigtranimp}

\clearpage
\begin{appendix}
\end{appendix}

\end{document}
