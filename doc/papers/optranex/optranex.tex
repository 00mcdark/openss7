\documentclass[letterpaper,final,notitlepage,twocolumn,10pt,twoside]{article}
\usepackage{ftnright}
\usepackage{makeidx}
\usepackage{pictex}
%\usepackage{psfig}
%\usepackage{graphics}
\usepackage{graphicx}
\usepackage{eepic}
%\usepackage{epsfig}
%\usepackage[dvips]{graphicx,epsfig}
%\usepackage[dvips]{epsfig}
%\usepackage{epsf}
\usepackage{natbib}
\usepackage{placeins}
%\usepackage{placeins}

\setlength{\voffset}{-1.2in}
\setlength{\topmargin}{0.2in}
\setlength{\headheight}{0.2in}
\setlength{\headsep}{0.3in}
\setlength{\topskip}{0.0in}
\setlength{\footskip}{0.3in}
\setlength{\textheight}{10.0in}

\setlength{\hoffset}{-1.0in}
\setlength{\oddsidemargin}{0.5in}
\setlength{\evensidemargin}{0.5in}
\setlength{\textwidth}{7.5in}

\setlength{\marginparwidth}{0.0in}
\setlength{\marginparsep}{0.0in}

\setlength{\columnsep}{0.3in}
\setlength{\columnwidth}{3.6in}
%\setlength{\columnseprule}{0.25pt}

\setlength{\paperheight}{11in}
\setlength{\paperwidth}{8.5in}

\let\Huge = \huge
\let\huge = \LARGE
\let\LARGE = \Large
\let\Large = \large
\let\large = \normalsize
\let\normalsize = \small
\let\small = \footnotesize
\let\footnotesize = \scriptsize
\let\scriptsize = \tiny

\makeatletter
\renewcommand\section{\@startsection {section}{1}{\z@}%
                                   {-2ex \@plus -1ex \@minus -.2ex}%
                                   {1ex \@plus .2ex}%
                                   {\normalfont\large\bfseries}}
\renewcommand\subsection{\@startsection{subsection}{2}{\z@}%
                                     {-1.5ex \@plus -.5ex \@minus -.2ex}%
                                     {1ex \@plus .2ex}%
                                     {\normalfont\normalsize\bfseries}}
\renewcommand\subsubsection{\@startsection{subsubsection}{3}{\z@}%
                                     {-1.25ex\@plus -.5ex \@minus -.2ex}%
                                     {1ex \@plus .2ex}%
                                     {\normalfont\normalsize\bfseries}}
\renewcommand\paragraph{\@startsection{paragraph}{4}{\z@}%
                                    {1.5ex \@plus .5ex \@minus .2ex}%
                                    {-1em}%
                                    {\normalfont\normalsize\bfseries\slshape}}
\renewcommand\subparagraph{\@startsection{subparagraph}{5}{\parindent}%
                                       {0ex \@plus 0ex \@minus 0ex}%
                                       {-1em}%
                                      {\normalfont\normalsize\bfseries\slshape}}
\makeatother

\setcounter{tocdepth}{3}
\setcounter{secnumdepth}{6}

\pagestyle{plain}
%\pagestyle{myheadings}
%\markboth{B. Bidulock}{B. Bidulock}

\makeglossary

\newcommand{\topfigrule}{\vspace{0.5ex}\rule{\columnwidth}{0.4pt}\vspace{0.5ex} }
\newcommand{\botfigrule}{\vspace{0.5ex}\rule{\columnwidth}{0.4pt}\vspace{0.5ex} }
\newcommand{\dblfigrule}{\vspace{0.5ex}\rule{\textwidth}{0.4pt}\vspace{0.5ex} }

%\bibliographystyle{unsrtnat}
%\bibliographystyle{plainnat}
%\bibliographystyle{ieeetr}
%\bibliographystyle{abbrvnat}
%\bibliographystyle{acm}
\bibliographystyle{plainnat}
%\bibliographystyle{alpha}

\begin{document}

%\begin{titlepage}
%\begin{center}
%    STREAMS vs. Sockets Performance Comparison\\
%    Experimental Test Results
%\end{center}
%\end{titlepage}

\title{High Density SONET/SDH MG Interface for VoIP Trunking\\[0.5ex]
	{\large \textsl{Draft Patent Application}}}
\author{Brian Bidulock\thanks{bidulock@openss7.org}\\
	OpenSS7 Corporation}
\date{February 7, 2010}
\maketitle

\begin{abstract}
\addcontentsline{toc}{section}{Abstract}
An embodiment of a synchronous network to packet mapping and conversion device
and methods for operating same are described.  This device may support the
interconnection of synchronous optical networks using the SONET and SDH
standards to Internet Protocol packet networks, and may be capable of
converting plesiochronous digital hierarchy (PDH) time-domain multiplexed
(TDM) signals contained within synchronous optical network frames to and from
representations of the TDM samples, such representations contained in packets
transferred over an interface to and from a packet network.
\end{abstract}

% \tableofcontents

\FloatBarrier
\addcontentsline{toc}{section}{References}
\bibliography{optranex}

\begin{figure}[htp]
\center\includegraphics[width=3.5in]{optranex_fig01}
\label{figure:optranex_fig01}
\end{figure}

\section[Field]{Field of the Invention}

The present invention relates to the transformation of time-domain multiplexed
(TDM) samples contained in plesiochronous digital hierarchy (PDH) signals
transported by SONET/SDH synchronous optical networks, this transformation
being from the time domain multiplexed (TDM) samples to representations of the
TDM samples contained within packets exchanged over a packet network.  One
possible application of this transformation being the interconnection of a
public switched telephone network (PSTN) and an Internet Protocol based
internetwork, commonly referred to by one skilled in the art as a media
gateway (MG) function.

\section[Background]{Background to the Invention}

Aspects of the present invention relate to the efficient conversion of signals
formatted for transport over traditional telecommunications synchronous
optical networks to signals formatted for transport over the well known
Internet Protocol (IP). \nocite{rfc791}

In a traditional telecommunications synchronous optical network, data is
transmitted from one location to another as a continuous stream of digital
information moving from the source to the destination at a constant rate.  The
stream is organized as a sequence of frames, each frame containing a fixed
number of fields in a defined order, each field of the same length.  The
Synchronous Optical Network (SONET) and Synchronous Digital Hierarchy (SDH)
are the principle synchronous optical network standards currently in use.

Synchronous optical networking (SONET) and synchronous digital hierarchy (SDH)
are mechanisms commonly used for transporting time division multiplexing (TDM)
samples embedded within a plesiochronous digital hierarchy (PDH) signal on
a communications link.  Transmission networks for transporting TDM signals
over a network of SONET/SDH communications links are commonly used within the
field of the present invention for connecting telephone network switching
equipment (e.g. telephone switches) into a telecommunications network.
Consider for example, a single optical fibre connecting multiple voice or data
channels between call switches of a telephone network.

Another common mechanism for transporting TDM signals over a communications
network makes use of the well known Internet Protocol (IP) for transport.  For
example, the audio-visual profile of the Real Time Protocol (RTP) standardized
by the Internet Engineering Task Force (IETF) is commonly used to transport
pulse code modulation (PCM) encoded voice signals across a packet network
using RTP. \nocite{rfc1889}

Telephone network operators are keen to make use of both SONET/SDH and packet
networks to carry user traffic.  To enable packet networks to interwork with
conventional SONET/SDH networks, it is necessary to provide a mechanism for
efficiently converting to and from the SONET/SDH format and the packet format.

Current state of the art mechanisms for converting between the SONET/SDH
format and the TDM formats used by telephone switches use a mechanism to
convert between the SONET/SDH format used by the transport network and the TDM
format used to connect to network switches.  The traditional mechanisms used
to convert TDM formatted signals into the SONET/SDH signals used by the
transport network adhere to stringent delay and jitter requirements.

Synchronous optical networks are the primary transport mechanism for long
distance transmission of information.  Some legacy equipment (e.g. a
traditional telephone switch) requires lower speed time-division-multiplexed
(TDM) network connections following the plesiochronous digital hierarchy
(PDH) as standardized by the International Telecommunications Union in
recommendations \nocite{g703} G.703 and \nocite{g704} G.704.

The functionality that may be needed to connect a SONET or SDH synchronous
optical network and a packet network includes that of an Add-Drop Multiplexer
(ADM), terminal multiplexer (TM), a Digital Cross-Connect (DCC), and a
Multi-Services Provisioning Platform (MSPP).  An ADM may be used to transport
SONET or SDH traffic on network ring topologies.  The ADM are nodes on such
rings that are used to arbitrate (add or drop) traffic to or from the ring.  A
SONET/SDH terminal multiplexer performs a function similar to that of an ADM
except that the network connection is not in a ring configuration.  A terminal
multiplexer terminates a high-speed point-to-point SONET path, and hands off a
number of lower rate lines and paths on the client side.  A DCC is a circuit
switch and digital multiplexer used to switch and groom traffic between
different lines and paths.  A network may include several ADM and terminal
multiplexers to arbitrate or terminate traffic along rings or point-to-point
connections, and a DCC may be used to multiplex and switch the traffic between
all the paths.  A Multi-Service Provisioning Platform combines the
functionality of the DCC, the ADM, and the terminal along with the ability to
support data protocols such as Ethernet to the client users.

Recent advances in integrated circuits has permitted the functions of the ADM,
TM, DCC and MSPP to be combined or individually performed using application
specific integrated circuits (ASIC) or field programmable gate arrays (FPGA).

A current approach, well known to one skilled in the art, for converting
isochronous TDM signals to and from packets containing representations of
these TDM signals uses a mechanism that converts between the TDM signals
carried on low-speed plesiochronous digital hierarchy (PDH) interfaces
provided by a Digital Cross-Connect and used to connect telephone signals
(e.g. T1 carriers) and packets containing representations of those TDM
signals.  Examples of mechanisms in current use and well known to one skilled
in the art include digital signal processors (DSP).  Digital signal processors
are specialized devices that are capable of transforming the isochronous TDM
signals to representations of those signals without degrading the low delay
and jitter exhibited by the TDM signals.  For examples, mechanisms built from
digital signal processors are in existence within the telephone network that
add as little as 125 microseconds of delay between between a voice sample in
the TDM signal at the input of the DSP and the representation of the voice
sample suitable for placement in a packet at the output of the DSP.

In contrast, packet networks used to carry TDM signals using the well known
Internet Protocol (IP), have delay and jitter requirements that are greatly
relaxed in comparison to those traditionally required for conversions between
SONET/SDH signals and TDM signals, and further for conversions to and from TDM
signals and packets containing the representations of those signals.  Methods
well known to one of skill in the art for compensating for increased jitter is
by transforming increased jitter to, instead, an increased delay by buffering
the signal for a period and then playing out the signal at a later time, and
on more regular intervals, reducing jitter.  Methods well known to one of
skill in the art for compensating for the reduced perceived voice quality
cause by increased delay include algorithms for performing suppression or
elimination of the effects of echo and singing.

Also, the very nature of the placing in packets of representations of the TDM
signals requires that a plurality of consecutive voice samples within the TDM
signal be collected, formatted, and placed into each packet.  Collection of
the necessary plurality of TDM samples requires that the packet into which
their representations will be contained must await the last TDM sample before
being transmitted.  For commonly used voice over packet formats familiar to
one skilled in the art, such as that of the Internet Engineering Task Force
audio-visual profile for the Real-Time Transport protocol, the delay from the
first TDM sample in the sequence to the last may typically be from as little
as 10 milliseconds to as much as 30 milliseconds in duration.  This duration
of time translates to a delay from the moment in time that a given TDM sample
enters the telecommunications system at the sender to the the moment of time
that the given TDM sample exits the telecommunications network at the
receiver.

A limitation of approaches using ADM, terminal multiplexers, DCC and DSP to
convert TDM signals transported by SONET/SDH networks and representations of
those TDM signals in packets transported over a packet network lies in the
conversion of SONET/SDH serial data streams to the representations of TDM
signals suitable for transport in a packet over the packet network.
Currently, these conversions use traditional digital multiplex equipment (DME)
and digital signal processors (DSP) to maintain the low delay and jitter
characteristics of the TDM signals as transported by the synchronous optical
network in the Add-Drop Multiplexers that add and drop the signals from
optical networks in ring topologies; in the terminal multiplexers and Digital
Cross-Connects that extract and switch intermediate and low-speed isochronous
TDM signals from the SONET/SDH signals; and in Digital Signal Processors used
to transform TDM signals to representations of the TDM signals suitable to be
included in packets transferred to and from a packet network.  Maintaining the
low delay and jitter required of traditional TDM digital multiplex equipment,
and also the use of digital signal processors to achieve same, contributes
significantly to the cost of systems using these traditional methods.

Consequently, there is a need for a compact, low power and cost-effective
means of interconnecting synchronous optical networks and packet networks for
the transport of TDM signals and representations of the TDM signals in packets.
Such means should avoid the sizable, high-power and inefficient mechanisms
that are predicated on the preservation of the low delay and jitter
characteristics of TDM signals during the conversion process.

Further limitations and disadvantages of conventional and traditional
approaches will become apparent to one of skill in the art, through comparison
of such systems with some aspects of the present invention as set forth in the
remainder of the present application with reference to the drawings.

\section[Summary]{Summary of the Invention}

According to a first aspect of the present invention there is provided a
method of transforming the isochronous TDM signals contained in pleisochronous
digital hierarchy (PDH) frames transported by a SONET/SDH network to and from
packets containing representations of the transported TDM signal information
for transmission and reception over a packet network.

Another aspect of the current invention relates to a method that, unlike the
current and traditional methods, relaxes timing constraints placed on the
process of converting isochronous TDM signals carried in PDH frames by the
SONET/SDH network to and from the representation of the TDM signals contained
in packets transmitted and received over a packet network.  This aspect of the
present invention relaxes the timing constraints placed on the conversion
method, at one or more stages of the conversion, to the lesser timing
constraints required by the transport of the representations of the TDM
signals in packets over a packet network, from the more stringent timing
constraints required for the transport of the TDM signals over a traditional
telephone network.

An aspect of the current invention relates to the relaxation of time
constraints that allows the TDM signals to be extracted from the PDH frames
carried in synchronous payload envelopes by the SONET/SDH network to be saved
in a storage mechanism or delayed in a delay mechanism until the data therein
can be processed in blocks more efficiently by general purpose processors.

An embodiment of the present invention may comprise at least one optical
network interface for terminating a receive serial data stream, and for
generating a transmit data stream; and, at least one system interface for
transferring received data to a storage or delay mechanism, and for retrieving
data to be transmitted from the storage or delay mechanism; at least one
packet interface for receiving packets from a packet network, and transmitting
packets to a packet network; and a mechanism for converting the data
transferred to the storage or delay mechanism into a sequence of packets for
transmission, and for converting received packets into data saved to the
storage or delay mechanism.

The at least one optical network interface of an embodiment in accordance with
the present invention may be compatible with the Telcordia GR-253-CORE
\nocite{gr253core} or corresponding later SONET standard, and may be
compatible with the International Telecommunication Union Synchronous Digital
Hierarchy (SDH) standard.  It may be capable of supporting at least SONET
STS-3 capacity on each of the receive serial data stream and the transmit data
stream, and may support automatic protection switching compliant with the
American National Standards Institute T1.105.01-5000 or corresponding later
standard.  It may use optical or electrical interface modules compatible with
the Small Form Factor Committee INF-8074i Specification for SFP (Small
Formfactor Pluggable) Transceiver Revision 1.0 or corresponding later
standard. \nocite{inf8074}

The at least one packet interface of an embodiment in accordance with the
present invention may be compatible with the IEEE 802.3ab, IEEE 802.3z or
later equivalent Gigabit Ethernet standard.

The at least one system interface of an embodiment in accordance with the
present invention may be compatible with the Peripheral Component Interconnect
Special Interest Group (PCI-SIG) PCI Express Card Electromechanical
Specification Revision 1.1 or later equivalent standard.  \nocite{PCIeCEM} The
at least one system interface so compatibile may allow an embodiment of the
current invention to transfer SONET/SDH data for conversion between the
SONET/SDH interface and a storage or delay mechanism to defer processing of
the data until the stored data can be more efficiently processed in block
operations.

These and other advantages, aspects, and novel features of the present
invention, as well as details of illustrated embodiments, thereof, will be
more fully understood from the following description and drawings.

\onecolumn
\begin{figure}[htp]
\center\includegraphics[width=9.0in,angle=90]{optranex_fig01}
\caption[Figure1]{}
\label{figure:optranex_fig01}
\end{figure}
\twocolumn

\section[Drawings]{Brief Description of the Drawings}

\textbf{Figure \ref{figure:optranex_fig01}} comprises a block diagram
illustrating one preferred embodiment of the current invention.

\section[Description]{Detailed Description of Certain Embodiments}

An embodiment of the present invention may provide the functionality needed to
allow TDM signals carried over a synchronous optical network to be converted
to and from representations of the TDM signals within packets transported over
a packet network in a more efficient manner that may be lower in cost, consume
less power, or occupy a smaller physical space, than existing alternatives,
and may be incorporated into an add-in circuit card for a general purpose
computer.  It may also be used to convert TDM signals to their packet
representations in a cost effective manner incorporated into an add-in circuit
card for an add/drop multiplexer (ADM), a terminal multiplexer (TM), a Digital
Cross-Connect (DCC) or a multi-services provisioning platform (MSPP).  It may
also be used to convert TDM signals to their packet representations in
application specific integrated circuits (ASIC) or field programmable gate
arrays (FPGA) integrated with other components on an add-in card with a
system.

An embodiment in accordance with the present invention may comprise at least
one synchronous optical network interface, at least one system interface, and
at least one packet network interface.

Another embodiment in accordance with the current invention may comprise at
least one synchronous optical network interface and at least one system
interface.  In such an embodiment, the at least one synchronous optical
network interface may be capable of providing for the transport of both TDM
signals and representations of TDM signals in packets, both using the
synchronous optical network for transport.



\textbf{Figure \ref{figure:optranex_fig01}} shows a block diagram illustrating
the functional elements in one embodiment of the present invention.  An
embodiment in accordance with the present invention may support both SONET TDM
and packet services and may function as a SONET/SDH/PDH/IP media gateway.  The
embodiment in accordance with the current invention comprises at least one
SONET/SDH network interface \textbf{101} and \textbf{102}; at least one packet
network interface \textbf{111}; and at least one system interface
\textbf{107}.  This embodiment comprises a PCI Express add-in card
\textbf{112} that contains a SONET/SDH transceiver \textbf{103} that may
convert optical signals into electrical signals; SONET/SDH framer \textbf{104}
extracts synchronous payload envelopes (SPE) from the SONET/SDH stream; FPGA
or ASIC \textbf{105} passes the SPE to and from the PCI Express interface chip
\textbf{106} that may transfer SPE to and from system memory \textbf{108} in
an asynchronous transaction-oriented manner over a system bus \textbf{107}.  A
computer processor \textbf{109} reads and writes the contents of the SPE so
transferred to and from memory and converts between the TDM samples contained
in the PDH signals embedded in the SPE to the representation of the TDM
signals contained in packets; and transfers the packets to the packet network
\textbf{111} using the network interface card \textbf{110}.  In accordance
with the specification of the current invention, the computer processor
\textbf{109} has a relaxed time constraint for each of: the extraction of PDH
signals embedded within the SPE, the demultiplex of the TDM samples
contained within the PDH signals, the conversion of the TDM samples to
representations of the TDM samples suitable to be included in packets, and the
inclusion of the representation of the TDM samples in packets transfered to
the packet network \textbf{111}.


In an alternative embodiment in accordance with the current invention, the at
least one network interface \textbf{111} may be instead combined with the
synchronous optical network interface \textbf{101} and \textbf{102}.  In this
embodiment in accordance with the current invention, the packets containing
representations of TDM samples are instead embedded into SPE and passed by the
computer processor \textbf{109} over the system bus \textbf{107} into system
memory \textbf{108} where the SPE are subsequently read by the PCI Express
interface chip \textbf{106} via the FPGA or ASIC device \textbf{105} to the
SONET/SDH framer \textbf{104} and SONET/SDH transceiver \textbf{103} for
transmission to the SONET/SDH network \textbf{101} or \textbf{102}.  This
embodiment of the current invention may not require the capabilities of a
network interface card \textbf{110} nor packet network interface \textbf{111};
but, rather, the packet network is attached to the SONET/SDH interface
\textbf{101} or \textbf{102}.

In a further embodiment in accordance with the current invention, the
SONET/SDH transceiver \textbf{103}, SONET/SDH framer \textbf{104}, FPGA or
ASIC device \textbf{105}, PCI Express interface chip \textbf{106}, and network
interface card \textbf{110}, may be severally or individually combined into at
least one FPGA or ASIC.

In an additional embodiment in accordance with the current invention, the FPGA
or ASIC device \textbf{105} may be capable of extracting the PDH frames
embedded within the SONET/SDH SPE and transferring those PDH frames to the
system memory \textbf{108} via the PCI Express interface chip \textbf{106}
and system bus \textbf{107}.  In this embodiment, the computer processor
\textbf{109} acts to extract the TDM samples from the PDH frames and converts
them to representations of the TDM samples in packets transferred to the
packet network \textbf{111}.  In this embodiment in accordance with the
current invention, time constraints on the processing of PDH frames is relaxed
to those of the packet network \textbf{111} to allow for efficient processing.



Accordingly, the present invention may be realized in hardware, or software,
or a combination of hardware and software.  The present invention may be
realized in a centralized fashion in one computer system, or in a distributed
fashion where different elements are spread across several interconnected
computer systems.  Any kind of computer system or other apparatus adapted for
carrying out the methods described herein is suited.  A typical combination of
hardware and software may be a general-purpose computer system with a computer
program that, when being loaded and executed, controls the computer system
such that it carries out the methods described herein.

The present invention also be embedded in a computer program product, which
comprises all the features enabling the implementation of the methods
described herein, and which when loaded in a computer system is able to carry
out these methods.  Computer program in the present context means any
expression, in any language, code or notation, of a set of instructions
intended to cause a system having an information processing capability to
perform in a particular function either directly or after either or both of
the following: a) conversion to another language, code or notation; b)
reproduction in a different material form.

Notwithstanding, the invention and its inventive arrangements disclosed herein
may be embodied in other forms without departing from the spirit or essential
attributes thereof.  Accordingly, reference should be made to the following
claims, rather than to the foregoing specification, as indicating the scope of
the invention.  In this regard, the description above is intended by way of
example only and is not intended to limit the present invention in any way,
except as set forth in the following claims.

While the present invention has been described with reference to certain
embodiments, it will be understood by one skilled in the art that various
changes may be made and equivalents may be substituted without departing from
the scope of the present invention.  Also, many modifications may be made to
adapt a particular situation or material to the teachings of the present
invention without departing from its scope.  Therefore, it is intended that
the present invention not be limited to the particular embodiment disclosed,
but that the present invention will include all embodiments falling within the
scope of the appended claims.

It will be appreciated by the person of skill in the are that various
modifications may be made to the embodiments described above without departing
from the scope of the present invention.  For example, the relaxation of TDM
timing constraints may occur at different stages between the SONET/SDH signal
and the packet containing representations of the TDM signal, depending upon
the availability of commonly used components.  Interfaces between components
may be optical or electrical in nature.  The invention is applicable to
systems with any number of channels and streams including single
stream/multiple channel systems and multiple stream/single channel systems.
Conversion may be performed by one or more application specific integrated
circuits (ASIC), field programmable gate arrays (FPGA), microprocessors or
computer processors.  The SONET/SDH to packet conversion may be implemented in
software rather than hardware, or by a combination of hardware and software.

\section[Claims]{Claims}

What is claimed is:

\begin{enumerate}
\item A media gateway devices that comprises at least one synchronous optical
network interface; at least one system interface; and, at least one packet
network interface; where
\begin{description}
\item one or more synchronous payload envelopes (SPE) are transferred to and
from the at least one optical network interface to and from a storage or delay
mechanism permitting asynchronous extraction and insertion of plesiochronous
digital hierarch (PDH) frames from and to the contents of the one or more SPE
in an asychronous sequence not subject to regular isochronous time constraints;
\item one ore more plesiochronous digital hierarchy (PDH) frames are
transferred to and from a storage or delay mechanism permitting asynchronous
demultiplexing and multiplexing of time-domain-multiplexing (TDM) samples from
and to the one or more PDH frames in an asychronous sequence not subject to
regular isochronous time constraints;
\item one or more TDM samples are converted to and from representations of the
TDM samples suitable for insertion into or extraction from one or more
packets in an asynchronous sequence not subject to regular isochronous time
constraints;
\item one or more representations of TDM samples are transferred to or from a
storage or delay mechanism to be inserted into or extracted from the sequence
of one or more packets in an asychronous sequence not subject to regular
isochronous time constraints; and,
\item one or more packets are transferred to or from the at least one packet
network interface in an asynchronous sequence not subject to regular
isochronous time constraints.
\end{description}
\item The device of claim 1, wherein the at least one packet network interface
is not necessary for transfering packets to the packet network because packets
are instead transfered over the at least one synchronous optical network
interface.
\item The device of claim 1, wherein the extraction or insertion of PDH frames
from the SPE of the SONET/SDH signal is performed instead in the traditional
isochronous manner.
\item The device of claim 1, wherein the demultiplexing or multiplexing of TDM
samples from and to the PDH frames is performed instead in the traditional
isochronous manner.
\item The device of claim 3, wherein the demultiplexing or multiplexing of TDM
samples from and to the PDH frames is performed instead in the traditional
isochronous manner.
\item The device of claim 4, wherein the conversion of TDM samples to and from
representations of the TDM samples for placement into or removal from packets
is performed instead in the traditional isochronous manner.
\item The device of claim 2, wherein the extraction or insertion of PDH frames
from the SPE of the SONET/SDH signal is performed instead in the traditional
isochronous manner.
\item The device of claim 2, wherein the demultiplexing or multiplexing of TDM
samples from and to the PDH frames is performed instead in the traditional
isochronous manner.
\item The device of claim 7, wherein the demultiplexing or multiplexing of TDM
samples from and to the PDH frames is performed instead in the traditional
isochronous manner.
\item The device of claim 8, wherein the conversion of TDM samples to and from
representations of the TDM samples for placement into or removal from packets
is performed instead in the traditional isochronous manner.
\end{enumerate}

% \clearpage
% \begin{appendix}
% \end{appendix}

\end{document}
