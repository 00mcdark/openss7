\documentclass[letterpaper,final,notitlepage,twocolumn,10pt,twoside]{article}
\usepackage{ftnright}
\usepackage{makeidx}
\usepackage{pictex}
%\usepackage{psfig}
%\usepackage{graphics}
\usepackage{graphicx}
\usepackage{eepic}
%\usepackage{epsfig}
%\usepackage[dvips]{graphicx,epsfig}
%\usepackage[dvips]{epsfig}
%\usepackage{epsf}
\usepackage{natbib}
\usepackage{placeins}
%\usepackage{placeins}

\setlength{\voffset}{-1.2in}
\setlength{\topmargin}{0.2in}
\setlength{\headheight}{0.2in}
\setlength{\headsep}{0.3in}
\setlength{\topskip}{0.0in}
\setlength{\footskip}{0.3in}
\setlength{\textheight}{10.0in}

\setlength{\hoffset}{-1.0in}
\setlength{\oddsidemargin}{0.5in}
\setlength{\evensidemargin}{0.5in}
\setlength{\textwidth}{7.5in}

\setlength{\marginparwidth}{0.0in}
\setlength{\marginparsep}{0.0in}

\setlength{\columnsep}{0.3in}
\setlength{\columnwidth}{3.6in}
%\setlength{\columnseprule}{0.25pt}

\setlength{\paperheight}{11in}
\setlength{\paperwidth}{8.5in}

%\let\Huge = \huge
%\let\huge = \LARGE
%\let\LARGE = \Large
%\let\Large = \large
%\let\large = \normalsize
%\let\normalsize = \small
%\let\small = \footnotesize
%\let\footnotesize = \scriptsize
%\let\scriptsize = \tiny

\makeatletter
\renewcommand\section{\@startsection {section}{1}{\z@}%
                                   {-2ex \@plus -1ex \@minus -.2ex}%
                                   {1ex \@plus .2ex}%
                                   {\normalfont\large\bfseries}}
\renewcommand\subsection{\@startsection{subsection}{2}{\z@}%
                                     {-1.5ex \@plus -.5ex \@minus -.2ex}%
                                     {1ex \@plus .2ex}%
                                     {\normalfont\normalsize\bfseries}}
\renewcommand\subsubsection{\@startsection{subsubsection}{3}{\z@}%
                                     {-1.25ex\@plus -.5ex \@minus -.2ex}%
                                     {1ex \@plus .2ex}%
                                     {\normalfont\normalsize\bfseries}}
\renewcommand\paragraph{\@startsection{paragraph}{4}{\z@}%
                                    {1.5ex \@plus .5ex \@minus .2ex}%
                                    {-1em}%
                                    {\normalfont\normalsize\bfseries\slshape}}
\renewcommand\subparagraph{\@startsection{subparagraph}{5}{\parindent}%
                                       {0ex \@plus 0ex \@minus 0ex}%
                                       {-1em}%
                                      {\normalfont\normalsize\bfseries\slshape}}
\makeatother

\setcounter{tocdepth}{3}
\setcounter{secnumdepth}{6}

\pagestyle{plain}
%\pagestyle{myheadings}
%\markboth{B. Bidulock}{B. Bidulock}

\makeglossary

\newcommand{\topfigrule}{\vspace{0.5ex}\rule{\columnwidth}{0.4pt}\vspace{0.5ex} }
\newcommand{\botfigrule}{\vspace{0.5ex}\rule{\columnwidth}{0.4pt}\vspace{0.5ex} }
\newcommand{\dblfigrule}{\vspace{0.5ex}\rule{\textwidth}{0.4pt}\vspace{0.5ex} }

%\bibliographystyle{unsrtnat}
%\bibliographystyle{plainnat}
%\bibliographystyle{ieeetr}
%\bibliographystyle{abbrvnat}
%\bibliographystyle{acm}
%\bibliographystyle{plainnat}
\bibliographystyle{alpha}

\begin{document}

%\begin{titlepage}
%\begin{center}
%    STREAMS vs. Sockets Performance Comparison\\
%    Experimental Test Results
%\end{center}
%\end{titlepage}

\title{STREAMS-based vs. Legacy Pipe Performance Comparison\\[0.5ex]
	{\large \textsl{Experiment Test Results for Linux}}}
\author{Brian F. G. Bidulock\thanks{bidulock@openss7.org}\\
	OpenSS7 Corporation}
\date{May 1, 2007}
\maketitle

\begin{abstract}
With the objective of contrasting performance between STREAMS and legacy
approaches to system facilities, comparison is made between the tested
performance of the Linux legacy pipe implementation and the STREAMS-based
pipes using \textsl{Linux Fast-STREAMS} \cite[]{LfS}.
\end{abstract}

%\tableofcontents

\section[Background]{Background}

Pipes have a rich history in the UNIX operating system.  Present on early Bell
Laboratories UNIX Versions, pipes found their way into both BSD and System V
releases.  Finally, in 4.4BSD pipes are implemented with Sockets and in System
V Release 4 pipes are implemented with STREAMS.

\subsection[STREAMS]{STREAMS}

STREAMS is a facility first presented in a paper by Dennis M. Ritchie in 1984
\cite[]{Ritchie84}, originally implemented on 4.1BSD and later part of Bell
Laboratories Eighth and Ninth Edition UNIX, incorporated into UNIX System V
Release 3.0, and enhanced in UNIX System V Release 4 and 4.2.  STREAMS was
used in SVR4 for terminal input/output, pseudo-terminals, pipes, named pipes
(FIFOs), interprocess communication and networking.  Since its release in UNIX
System V Release 3, STREAMS has been implemented across a wide range of UNIX,
UNIX-like and UNIX-based systems, making its implementation and use an ipso
facto standard.

STREAMS is a facility that allows for a reconfigurable full duplex
communications path, \textit{Stream}, between a user process and a driver in
the kernel.   Kernel protocol modules can be pushed onto and popped from the
\textit{Stream} between the user process and driver.  The \textit{Stream} can
be reconfigured in this way by a user process.  The user process, neighbouring
protocol modules and the driver communicate with each other using a message
passing scheme.  This permits a loose coupling between protocol modules,
drivers and user processes, allowing a third-party and loadable kernel module
approach to be taken toward the provisioning of protocol modules on platforms
supporting STREAMS.

On UNIX System V Release 4.2, STREAMS was used for terminal input-output,
pipes, FIFOs (named pipes), and network communications.  Modern UNIX,
UNIX-like and UNIX-based systems providing STREAMS normally support some
degree of network communications using STREAMS; however, many do not support
STREAMS-based pipe and FIFOs\footnote{For example, AIX.} or terminal
input-output\footnote{For example, HP-UX.} directly or without
reconfiguration.

\subsection[Pipe Implementation]{Pipe Implementation}

Traditionally there have been two approaches to implementation of pipes and
named pipes (FIFOs):

\subsubsection*{Legacy Approach to Pipes.}

Under the 4.1BSD or SVR3 approach, pipes were implemented using anonymous
FIFOs.  That is, when a pipe was opened, a new instance of a FIFO was
obtained, but which was not attached to a node in the file system and which
had two file descriptors: one open for writing and the other opened for
reading.  As FIFOs are a fundamentally unidirectional concept, legacy pipes
can only pass data in one direction.  Also, legacy pipes do not support the
concept of record boundaries and only support a byte stream.  Each end of the
pipe uses a the legacy interface and they do not provide any of the advanced
capabilities provided by STREAMS.

\subsubsection*{SVR4 Approach to Pipes.}

Under the SVR4 approach, both pipes and FIFOs are implemented using STREAMS
\cite[]{magic}.  When a pipe is opened, a new instance of a STREAMS-based pipe
is obtained, but which is attached to a non-accessible node in the
\texttt{fifofs} file system instead of the normal STREAMS \texttt{specfs} file
system.  Although one file descriptor was opened for read and the other for
write, with a STREAMS-based pipe it is possible to reopen both for reading and
writing.

\begin{figure}[htbp]
\center\includegraphics[width=3.0in]{streamspipe}
\caption[STREAMS-Based Pipes]{STREAMS-Based Pipes}
\label{figure:streamspipe}
\end{figure}

The STREAMS-based pipes provide the same rich set of facilities that are also
available for other STREAMS devices such as pseudo-terminals and network
interfaces.  As a result, STREAMS-based pipes provide a number of capabilities
that are not provided by legacy pipes:

\begin{description}

\item[\textit{Full Duplex.}]

STREAMS-based pipes are full duplex pipes.  That is, each end of the pipe can
be used for reading and writing.  To accomplish the same effect with legacy
pipes requires that two legacy pipes be opened.

\item[\textit{Pushable Modules.}]

STREAMS-based pipes can have STREAMS modules pushed an popped from either end
of the pipe, just as any other STREAMS device.

\item[\textit{File Attachment.}]

STREAMS-based pipes can have either end (or both ends) attached to a node in
the file system using fattach(3) \cite[]{advprog}.

\item[\textit{File Descriptor Passing.}]

STREAMS-based pipes can pass file descriptors across the pipe using the
\texttt{I\_SENDFD} and \texttt{I\_RECVFD} input-output controls
\cite[]{advprog}.

\item[\textit{Record Boundary Preservation.}]

STREAMS-based pipes can preserve record boundaries and can pass messages
atomically using the getmsg(2) and putmsg(2) system calls.

\item[\textit{Prioritization of Messages.}]

STREAMS-based pipes can pass messages in priority bands using the getpmsg(2)
and putpmsg(2) system calls.

\end{description}

\subsubsection*{BSD Approach to Pipes.}

As of 4.2BSD, with the introduction of Sockets, pipes were implemented using
the networking subsystem (UNIX domain sockets) for what was cited as
\textit{"performance reasons"} \cite[]{bsd}.  The pipe(2) library call
effectively calls sockpair(3) and obtains a pair of connected sockets in the
UNIX domain as illustrated in \textit{Figure \ref{figure:socketpipe}}.

\begin{figure}[htbp]
\center\includegraphics[width=2.5in]{socketpipe}
\caption[4.2BSD Pipes]{4.2BSD Pipes}
\label{figure:socketpipe}
\end{figure}

Knowing the result of this testing, I can only imagine that the
\textit{"performance reasons"} had to do with the lack of a full flow
control mechanism in the legacy file system based pipes.

\subsubsection*{Linux Approach to Pipes.}

Linux adopts the legacy (4.1BSD or SVR3 pre-STREAMS) approach to pipes.  Pipes
are file system based, and obtain an \texttt{inode} from the \texttt{pipefs}
file system as illustrated in \textit{Figure \ref{figure:legacypipe}}.  Pipes
are unnamed FIFOs, unidirectional byte streams, and do not provide any of the
capabilities of STREAMS-based pipes or socket pairs in the UNIX
domain.\footnote{It has been said of Linux that, without STREAMS, it is just
another BSD... ...and not a very good one.}

\begin{figure}[htbp]
\center\includegraphics[width=3.0in]{legacypipe}
\caption[Linux Legacy Pipes]{Linux Legacy Pipes}
\label{figure:legacypipe}
\end{figure}

\subsubsection*{Standardization.}

During the POSIX standardization process, pipes and FIFOs were given special
treatment to ensure that both the legacy approach to pipes, 4.2BSD approach
and the STREAMS-based approach to pipes were compatible.  POSIX has
standardized the programmatic interface to pipes.  STREAMS-based pipes have
been POSIX compliant for many years and were POSIX compliant in the
\textsl{SVR4.2} release.  The STREAMS-based pipes provided by the
\textsl{Linux Fast-STREAMS} package provides POSIX compliant STREAMS-based
pipes.

As a result, any application utilizing a legacy Linux pipe in a POSIX
compliant manner will also be compatible with STREAMS-based
pipes.\footnote{This compatibility is exemplified by the perftest(8) program
which does not distinguish between legacy and STREAMS-based pipes in their
implementation or use.}

\subsection[Linux Fast-STREAMS]{Linux Fast-STREAMS}

The first STREAMS package for Linux that provided SVR4 STREAMS capabilities
was the \textsl{Linux STREAMS (LiS)} package originally available from GCOM.
This package exhibited incompatibilities with SVR 4.2 STREAMS and other
STREAMS implementations, was buggy and performed very poorly on Linux.  These
difficulties prompted the OpenSS7 Project \cite[]{openss7} to implement an SVR
4.2 STREAMS package from scratch, with the objective of production quality and
high-performance, named \textsl{Linux Fast-STREAMS}.

The OpenSS7 Project \cite[]{openss7} also maintains public and internal
versions of the \textsl{LiS} package.  The last public release was
\textit{LiS-2.18.3}; the current internal release version is
\textit{LiS-2.18.6}.  The current production public release of \textsl{Linux
Fast-STREAMS} is \textit{streams-0.9.3}.

\section[Objective]{Objective}

The objective of the current study is to determine whether, for the Linux
operating system, the newer STREAMS-based pipe approach is (from the
perspective of performance) a viable replacement for the legacy
4.1BSD/SVR3-style pipes provided by Linux.  As a side objective, a comparison
is also made to STREAMS-based pipes implemented on the deprecated LiS (Linux
STREAMS) package.  This comparison will demonstrate one reason why
\textsl{Linux Fast-STREAMS} was written in the first place.

\paragraph*{Misconceptions}

When developing STREAMS, the authors oft times found that there were a number
of preconceptions from Linux advocates about both STREAMS and STREAMS-based
pipes, as follows:

\begin{itemize}

\item STREAMS is slow.

\item STREAMS is more flexible, but less efficient \cite[]{lkmlfaq}.

\item STREAMS performs poorly on uniprocessor and even poorer on SMP.

\item STREAMS-based pipes are slow.

\item STREAMS-based pipes are unnecessarily complex and cumbersome.

\end{itemize}

It is possible that the proponents of these statements have worked in the past
with an improper or under-performing STREAMS implementation (such as
\textsl{LiS}); however, the current study aims to prove that none of these
statements are correct for the STREAMS-based pipes provided by the
high-performance \textsl{Linux Fast-STREAMS}.

%The Linux kernel mailing list has this to say about STREAMS:
%
%\begin{quote}
%\begin{itemize}
%\item (REG) STREAMS allow you to "push" filters onto a network stack.  The
%idea is that you can have a very primitive network stream of data, and then
%"push" a filter ("module") that implements TCP/IP or whatever on top of that.
%Conceptually, this is very nice, as it allows clean separation of your
%protocol layers.  Unfortunately, implementing STREAMS poses many peformance
%pproblems.  Some Unix STREAMS based server telnet implementations even ran the
%data up to user space and back down again to a pseudo-tty driver, which is
%very inefficient.
%
%STREAMS will \textbf{never} be available in the standard Linux kernel, it will
%remain a separate implementation with some add-on kernel support (that come
%with the STREAMS package).  Linux and his networking gurus are unanimous in
%their decision to keep STREAMS out of the kernel.  They have stated several
%times on the kernel list when this topic comes up that even optional support
%will not be included.
%
%\item
%(REW, quoting Larry McVoy) "It's too bad, I can see why some people think they
%are cool, but the performance cost - both on uniprocessors and even more so on
%SMP boxes - is way too high for STREAMS to ever get added to the Linux
%kernel."
%
%Please stop asking for them, we have agreement amounst the head guy, the
%networking guys, and the fringe folks like myself that they aren't going in.
%
%\item
%(REG, quoting Dave Grothe, the STREAMS guy) STREAMS is a good framework for
%implementing complex and/or deep protocol stacks having nothing to do with
%TCP/IP, such as SNA.  It trades some efficiency for flexibility.  You may find
%the Linux STREAMS package (LiS) to be quite useful if you need to port
%protocol drivers from Solaris or UnixWare, as Caldera did.
%\end{itemize}
%The Linux STREAMS (LiS) package is available for download if you want to use
%STREAMS for Linux.  The following site also contains a dissenting view, which
%supports STREAMS.
%\end{quote}

\section[Description]{Description}

The three implementations tested vary in their implementation details.  These
implementation details are described below.

\subsection[Linux Pipes]{Linux Pipes}

Linux pipes are implemented using a file-system approach similar to that of
4.1BSD, or that of SVR3 and SVR2 releases, or their common Bell Laboratories
predecessors, as illustrated in \textit{Figure \ref{figure:legacypipe}}.  It
should be noted that 4.4BSD (and releases after 4.2BSD) implements pipes using
Sockets and the networking subsystem \cite[]{bsd}.  Also, note that SVR4
implemented pipes using STREAMS.  As such, the Linux pipe implementation is
both archaic and deprecated.

\paragraph*{Write side processing.}

In response to a write(2) system call, message bytes are copied from user space
to kernel directly into a preallocated buffer.  The tail pointer is pushed on
the buffer.  If the buffer is full at the time of the system call, the calling
process blocks, or the system call fails and returns an error number
(\texttt{EAGAIN} or \texttt{EWOULDBLOCK}).

\paragraph*{Read side processing.}

In response to a read(2) system call, message bytes are copied from the
preallocated buffer to user space.  The head pointer is pushed on the buffer.
If the buffer is empty at the time of the system call, the calling process
blocks, or the system call fails and returns an error number (\texttt{EAGAIN}
or \texttt{EWOULDBLOCK}).

\paragraph*{Buffering.}

If a writer goes to write and there is no more room left in the buffer for the
requested write, the writer blocks or the system call is failed
(\texttt{EAGAIN}).  If a reader goes to read and there are no bytes in the
buffer, the reader blocks or the system call is failed (\texttt{EAGAIN}).  If
there are fewer bytes in the buffer than requested by the read operation, the
available bytes are returned.  No queueing or flow control is performed.

\paragraph*{Scheduling.}

When a writer is blocked or polling for write, the writer is awoken once there
is room to write at least 1 byte into the buffer.  When a reader is blocked or
polling for read, the reader is awoken once there is at least 1 byte in the
buffer.

\subsection[STREAMS-based Pipes]{STREAMS-based Pipes}

STREAMS-based pipes are implemented using a specialized STREAMS driver that
connects the read and write queues of two \textit{Stream heads} in a twisted
pair configuration as illustrated in \textit{Figure \ref{figure:streamspipe}}.
Aside from a few specialized settings particular to pipes, each \textit{Stream
head} acts in the same fashion as the \textit{Stream head} for any other
STREAMS device or pseudo-device.

\paragraph*{Write side processing.}

In response to a write(2) system call, message bytes are copied from user space
into allocated message blocks.  Message blocks are passed downstream to the
next module (read \textit{Stream head}) in the \textit{Stream}.  If flow
control is in effect on the write queue at the time of the system call, the
calling process blocks, or the system call fails and returns an error number
(\texttt{EAGAIN}).
Also, STREAMS has a write message coalescing feature that allows message
blocks to be held temporarily on the write queue awaiting execution of the
write queue service procedure (invoked by the STREAMS scheduler) or the
occurrence of another write operation.

\paragraph*{Read side processing.}

In response to a read(2) system call, message blocks are removed from the read
queue and message bytes copied from kernel to user space.  If there are no
message blocks in the read queue at the time of the system call, the calling
process blocks, or the system call fails and returns and error number
(\texttt{EAGAIN}).
Also, STREAMS has a read notification feature that causes a read notification
message (\texttt{M\_READ}) containing the requested number of bytes to be
issued  and passed downstream before blocking.
STREAMS has an additional read-fill mode feature which causes the read side to
attempt to satisfy the entire read request before returning to the user.

\paragraph*{Buffering.}

If a writer goes to write and the write queue is flow controlled, the writer
blocks or the system call is failed (\texttt{EAGAIN}).  If a reader goes to
read and there are no message blocks available, the reader blocks or the
system call is failed (\texttt{EAGAIN}).  If there are fewer bytes available in
message blocks on the read queue than requested by the read operation, the
available bytes are returned.  Normal STREAMS queueing and flow control is
performed as message blocks are passed along the write side or removed from
the read queue.

\paragraph*{Scheduling.}

When a write is blocked or polling for write, the writer is awoken once flow
control subsides on the write side.  Flow control subsides when the downstream
module's queue on the write side falls below its low water mark, the
\textit{Stream} is back-enabled, and the write queue service procedure runs.
When a reader is blocked or polling for read, the reader is awoken once the
read queue service procedure runs.  The read queue service procedure is
scheduled when the first message block is placed on the read queue after an
attempt to remove a message block from the queue failed.  The service
procedure runs when the STREAMS scheduler runs.

\section[Method]{Method}

To test the performance of STREAMS-based pipes, the \textsl{Linux
Fast-STREAMS} package was used \cite[]{LfS}.  The \textsl{Linux Fast-STREAMS}
package builds and install Linux loadable kernel modules and includes the
\texttt{perftest} program used for testing.  For comparison, the \textsl{LiS}
package \cite[]{LiS} was used for comparison.

\subsection[Test Program]{Test Program}

To test the maximum throughput performance of both legacy pipes and
STREAMS-based pipes, a test program was written, called \texttt{perftest}.
The \texttt{perftest} program is part of the \textsl{Linux Fast-STREAMS}
distribution \cite[]{LfS}.  The test program performs the following actions:

\begin{enumerate}

\item Opens either a legacy pipe or a STREAMS-based pipe.

\item Forks two child processes: a writer child process and a reader child
process.

\item The writer child process closes the reading end of the pipe.

\item The writer child process starts an interval timer.

\item The writer child process begins writing data to the pipe end with the
write(2) system call.

\item As the writer child process writes to the pipe end, it tallies the
amount of data written.  When the interval timer expires, the tally is output
and the interval timer restarted.

\item The reader child process closes the writing end of the pipe.

\item The reader child process starts an interval timer.

\item The reader child process begins reading data from the pipe end with the
read(2) system call.

\item As the reader child process reads from the pipe end, it tallies the
amount of data written.  When the interval timer expires, the tally is output
and the interval timer restarted.

\end{enumerate}

The test program thus simulates a typical use of pipes in a Linux system.  The
\texttt{perftest\_script} performance testing script was used to obtain
repeatable results (see \textit{Appendix \ref{section:script}}).

\subsection[Disributions Tested]{Distrbutions Tested}

To remove the dependence of test results on a particular Linux kernel or
machine, various Linux distributions were used for testing.  The distributions
tested are as follows:

\begin{tabular}{ll}\\
Distribution & Kernel\\
\hline
RedHat 7.2 & 2.4.20-28.7\\
CentOS 4 & 2.6.9-5.0.3.EL\\
CentOS 5 & 2.6.18-8-el5\\
SuSE 10.0 OSS & 2.6.13-15-default\\
Ubuntu 6.10 & 2.6.17-11-generic\\
Ubuntu 7.04 & 2.6.20-15-generic\\
Fedora Core 6 & 2.6.20-1.2933.fc6
\end{tabular}\\

\subsection[Test Machines]{Test Machines}

To remove the dependence of test results on a particular machine, various
machines were used for testing as follows:

\footnotesize
\begin{tabular}{llll}\\
Hostname & Processor & Memory & Architecture\\
\hline
porky & 2.57GHz PIV & 1Gb (333MHz) & i686 UP\\
pumbah & 2.57GHz PIV & 1Gb (333MHz) & i686 UP\\
daisy & 3.0GHz i630 HT & 1Gb (400MHz) & x86\_64 SMP\\
mspiggy & 1.7GHz PIV & 1Gb (333MHz) & i686 UP\\
\end{tabular}\\
\normalsize

\section[Results]{Results}

The results for the various distributions and machines is tabulated in
\textit{Appendix \ref{section:rawdata}}.  The data is tabulated as follows:

\begin{description}

\item[{\it Performance.}]

Performance is charted by graphing the number of writes per second against the
logarithm of the write size.

\item[{\it Delay.}]

Delay is charted by graphing the number of second per write against the write
size.  The delay can be modelled as a fixed write overhead per write operation
and a fixed overhead per byte written.  This model results in a linear graph
with the intercept at 1 byte per write representing the fixed per-write
overhead, and the slope of the line representing the per-byte cost.  As all
implementations use the same primary mechanisms for copying bytes to and from
user space, it is expected that the slope of each graph will be similar and
that the intercept will reflect most implementation differences.

\item[{\it Throughput.}]

Throughput is charted by graphing the logarithm of the product of the number
of writes per second and the message size against the logarithm of the message
size.  It is expected that these graphs will exhibit strong log-log-linear
(power function) characteristics.  Any curvature in these graphs represent
throughput saturation.

\item[{\it Improvement.}]

Improvement is charted by graphing the quotient of the writes per second of
the implementation and the writes per second of the Linux legacy pipe
implementation as a percentage against the write size.  Values over 0\%
represent an improvement over Linux legacy pipes, whereas values under 0\%
represent the lack of an improvement.

\end{description}

The results are organized in the section that follow in order of the machine
tested.

\subsection[Porky]{Porky}

Porky is a 2.57GHz Pentium IV (i686) uniprocessor machine with 1Gb of memory.
Linux distributions tested on this machine are as follows:

\begin{tabular}{ll}\\
Distribution & Kernel\\
\hline
Fedora Core 6 & 2.6.20-1.2933.fc6\\
CentOS 4 & 2.6.9-5.0.3.EL\\
SuSE 10.0 OSS & 2.6.13-15-default\\
Ubuntu 6.10 & 2.6.17-11-generic\\
Ubuntu 7.04 & 2.6.20-15-generic\\
\end{tabular}

\subsubsection[Fedora Core 6]{Fedora Core 6}

Fedora Core 6 is the most recent full release Fedora distribution.  This
distribution sports a 2.6.20-1.2933.fc6 kernel with the latest patches.  This
is the \texttt{x86} distribution with recent updates.

\begin{figure}[p]
\includegraphics[width=\columnwidth]{perftest_fc6_perf}
\caption[FC6 on Porky Performance]{FC6 on Porky Performance}
\label{figure:fc6perf}
\end{figure}

\begin{figure}[p]
\includegraphics[width=\columnwidth]{perftest_fc6_delay}
\caption[FC6 on Porky Delay]{FC6 on Porky Delay}
\label{figure:fc6delay}
\end{figure}

\begin{figure}[p]
\includegraphics[width=\columnwidth]{perftest_fc6_thrput}
\caption[FC6 on Porky Throughput]{FC6 on Porky Throughput}
\label{figure:fc6thrput}
\end{figure}

\begin{figure}[p]
\includegraphics[width=\columnwidth]{perftest_fc6_comp}
\caption[FC6 on Porky Comparison]{FC6 on Porky Comparison}
\label{figure:fc6comp}
\end{figure}

\begin{description}

\item[Performance.]

\textit{Figure \ref{figure:fc6perf}} illustrates
the performance of \textsl{LiS}, \textsl{Linux Fast-STREAMS} and Linux legacy
pipes across a range of write sizes.  As can be see from \textit{Figure
\ref{figure:fc6perf}}, the performance of \textsl{LiS} is dismal across the
entire range of write sizes.  The performance of \textsl{Linux Fast-STREAMS}
STREAMS-based pipes, on the other hand, is superior across the entire range of
write sizes.
Performance of \textsl{Linux Fast-STREAMS} is a full order of magnitude better
than \textsl{LiS}.

\item[Delay.]

\textit{Figure \ref{figure:fc6delay}}
illustrates the average write delay for \textsl{LiS}, \textsl{Linux
Fast-STREAMS} and Linux legacy pipes across a range of write sizes.  The slope
of all three curves is comparable and about the same.  This indicates that
each implementation is only slightly dependent upon the size of the message
and each implementation has a low per-byte processing overhead.  This is as
expected as pipes primarily copy data from user space to the kernel just to
copy it back to user space on the other end.  Note that the intercepts, on the
other hand, differ to a significant extent.  \textsl{Linux Fast-STREAMS}
STREAMS-based pipes have by far the lowest per-write overhead (about half that
of the Linux legacy pipes, and a sixth of \textsl{LiS} pipes).

\item[Throughput.]

\textit{Figure \ref{figure:fc6thrput}}
illustrates the throughput experienced by \textsl{LiS}, \textsl{Linux
Fast-STREAMS} and Linux legacy pipes across a range of write sizes.  As can be
seen from \textit{Figure \ref{figure:fc6thrput}}, all implementations exhibit
strong power function characteristics, indicating structure and robustness for
each implementation (regardless of performance).

\item[Improvement.]

\textit{Figure \ref{figure:fc6comp}} illustrates
the improvement over Linux legacy pipes of \textsl{Linux Fast-STREAMS}
STREAMS-based pipes.  The improvement of \textsl{Linux Fast-STREAMS} over
Linux legacy pipes is marked: improvements range from a significant 75\%
increase in performance at large write sizes, to a staggering 200\% increase
in performance at lower write sizes.

\end{description}

\subsubsection[CentOS 4.0]{CentOS 4.0}

CentOS 4.0 is a clone of the RedHat Enterprise 4 distribution.  This is the
\texttt{x86} version of the distribution.  The distribution sports a
2.6.9-5.0.3.EL kernel.

\begin{figure}[p]
\includegraphics[width=\columnwidth]{perftest_centos_perf}
\caption[CentOS 4.0 on Porky Performance]{CentOS 4.0 on Porky Performance}
\label{figure:centosperf}
\end{figure}

\begin{figure}[p]
\includegraphics[width=\columnwidth]{perftest_centos_delay}
\caption[CentOS 4.0 on Porky Delay]{CentOS 4.0 on Porky Delay}
\label{figure:centosdelay}
\end{figure}

\begin{figure}[p]
\includegraphics[width=\columnwidth]{perftest_centos_thrput}
\caption[CentOS 4.0 on Porky Throughput]{CentOS 4.0 on Porky Throughput}
\label{figure:centosthrput}
\end{figure}

\begin{figure}[p]
\includegraphics[width=\columnwidth]{perftest_centos_comp}
\caption[CentOS 4.0 on Porky Comparison]{CentOS 4.0 on Porky Comparison}
\label{figure:centoscomp}
\end{figure}

\begin{description}

\item[Performance.]

\textit{Figure \ref{figure:centosperf}} illustrates the performance of
\textsl{LiS}, \textsl{Linux Fast-STREAMS} and Linux legacy pipes across a
range of write sizes.  As can be see from \textit{Figure
\ref{figure:centosperf}}, the performance of \textsl{LiS} is dismal across the
entire range of write sizes.  The performance of \textsl{Linux Fast-STREAMS}
STREAMS-based pipes, on the other hand, is superior across the entire range of
write sizes.
Performance of \textsl{Linux Fast-STREAMS} is a full order of magnitude better
than \textsl{LiS}.

\item[Delay.]

\textit{Figure \ref{figure:centosdelay}} illustrates the average write delay
for \textsl{LiS}, \textsl{Linux Fast-STREAMS} and Linux legacy pipes across a
range of write sizes.  The slope of all three curves is comparable and about
the same.  This indicates that each implementation is only slightly dependent
upon the size of the message and each implementation has a low per-byte
processing overhead.  This is as expected as pipes primarily copy data from
user space to the kernel just to copy it back to user space on the other end.
Note that the intercepts, on the other hand, differ to a significant extent.
\textsl{Linux Fast-STREAMS} STREAMS-based pipes have by far the lowest
per-write overhead (about half that of the Linux legacy pipes, and a sixth of
\textsl{LiS} pipes).

\item[Throughput.]

\textit{Figure \ref{figure:centosthrput}} illustrates the throughput
experienced by \textsl{LiS}, \textsl{Linux Fast-STREAMS} and Linux legacy
pipes across a range of write sizes.  As can be seen from \textit{Figure
\ref{figure:centosthrput}}, all implementations exhibit strong power function
characteristics, indicating structure and robustness for each implementation
(regardless of performance).

\item[Improvement.]

\textit{Figure \ref{figure:centoscomp}} illustrates the improvement over Linux
legacy pipes of \textsl{Linux Fast-STREAMS} STREAMS-based pipes.  The
improvement of \textsl{Linux Fast-STREAMS} over Linux legacy pipes is marked:
improvements range from a significant 100\% increase in performance at large
write sizes, to a staggering 275\% increase in performance at lower write
sizes.

\end{description}


\subsubsection[SuSE 10.0 OSS]{SuSE 10.0 OSS}

SuSE 10.0 OSS is the public release version of the SuSE/Novell distribution.
There have been two releases subsequent to this one: the 10.1 and recent 10.2
releases.  The SuSE 10 release sports a 2.6.13 kernel and the
2.6.13-15-default kernel was the tested kernel.

\begin{figure}[p]
\includegraphics[width=\columnwidth]{perftest_suse_perf}
\caption[SuSE 10.0 OSS on Porky Performance]{SuSE 10.0 OSS on Porky Performance}
\label{figure:suseperf}
\end{figure}

\begin{figure}[p]
\includegraphics[width=\columnwidth]{perftest_suse_delay}
\caption[SuSE 10.0 OSS on Porky Delay]{SuSE 10.0 OSS on Porky Delay}
\label{figure:susedelay}
\end{figure}

\begin{figure}[p]
\includegraphics[width=\columnwidth]{perftest_suse_thrput}
\caption[SuSE 10.0 OSS on Porky Throughput]{SuSE 10.0 OSS on Porky Throughput}
\label{figure:susethrput}
\end{figure}

\begin{figure}[p]
\includegraphics[width=\columnwidth]{perftest_suse_comp}
\caption[SuSE 10.0 OSS on Porky Comparison]{SuSE 10.0 OSS on Porky Comparison}
\label{figure:susecomp}
\end{figure}

\begin{description}

\item[Performance.]

\textit{Figure \ref{figure:suseperf}} illustrates the performance of
\textsl{LiS}, \textsl{Linux Fast-STREAMS} and Linux legacy pipes across a
range of write sizes.  As can be see from \textit{Figure
\ref{figure:suseperf}}, the performance of \textsl{LiS} is dismal across the
entire range of write sizes.  The performance of \textsl{Linux Fast-STREAMS}
STREAMS-based pipes, on the other hand, is superior across the entire range of
write sizes.
Performance of \textsl{Linux Fast-STREAMS} is a full order of magnitude better
than \textsl{LiS}.

\item[Delay.]

\textit{Figure \ref{figure:susedelay}} illustrates the average write delay for
\textsl{LiS}, \textsl{Linux Fast-STREAMS} and Linux legacy pipes across a
range of write sizes.  The slope of the delay curves are similar for all
implementations, as expected.  The zero intercept of \textsl{Linux
Fast-STREAMS} is, however, far superior to that of legacy Linux and a full
order of magnitude better than the under-performing \textsl{LiS}.

\item[Throughput.]

\textit{Figure \ref{figure:susethrput}} illustrates the throughput experienced
by \textsl{LiS}, \textsl{Linux Fast-STREAMS} and Linux legacy pipes across a
range of write sizes.  As can be seen from \textit{Figure
\ref{figure:susethrput}}, all implementations exhibit strong power function
characteristics, indicating structure and robustness for each implementation.
The \textsl{Linux Fast-STREAMS} curve exhibits a downward concave
characteristic at large message sizes indicating that the memory bus saturates
at about 10Gbps.

\item[Improvement.]

\textit{Figure \ref{figure:susecomp}} illustrates the improvement over Linux
legacy pipes of \textsl{Linux Fast-STREAMS} STREAMS-based pipes.  The
improvement of \textsl{Linux Fast-STREAMS} over Linux legacy pipes is
significant: improvements range from a 100\% increase in performance at large
write sizes, to a 475\% increase in performance at lower write sizes.

\end{description}

\subsubsection[Ubuntu 6.10]{Ubuntu 6.10}

Ubuntu 6.10 is the current release of the Ubuntu distribution.  The Ubuntu
6.10 release sports a 2.6.15 kernel.  The tested distribution had current
updates applied.

\begin{figure}[p]
\includegraphics[width=\columnwidth]{perftest_ubuntu_perf}
\caption[Ubuntu 6.10 on Porky Performance]{Ubuntu 6.10 on Porky Performance}
\label{figure:ubuntuperf}
\end{figure}

\begin{figure}[p]
\includegraphics[width=\columnwidth]{perftest_ubuntu_delay}
\caption[Ubuntu 6.10 on Porky Delay]{Ubuntu 6.10 on Porky Delay}
\label{figure:ubuntudelay}
\end{figure}

\begin{figure}[p]
\includegraphics[width=\columnwidth]{perftest_ubuntu_thrput}
\caption[Ubuntu 6.10 on Porky Throughput]{Ubuntu 6.10 on Porky Throughput}
\label{figure:ubuntuthrput}
\end{figure}

\begin{figure}[p]
\includegraphics[width=\columnwidth]{perftest_ubuntu_comp}
\caption[Ubuntu 6.10 on Porky Comparison]{Ubuntu 6.10 on Porky Comparison}
\label{figure:ubuntucomp}
\end{figure}

\begin{description}

\item[Performance.]

\textit{Figure \ref{figure:ubuntuperf}} illustrates the performance of
\textsl{LiS}, \textsl{Linux Fast-STREAMS} and Linux legacy pipes across a
range of write sizes.  As can be see from \textit{Figure
\ref{figure:ubuntuperf}}, the performance of \textsl{LiS} is dismal across the
entire range of write sizes.  The performance of \textsl{Linux Fast-STREAMS}
STREAMS-based pipes, on the other hand, is superior across the entire range of
write sizes.  Performance of \textsl{Linux Fast-STREAMS} is a full order of
magnitude better than \textsl{LiS}.

\item[Delay.]

\textit{Figure \ref{figure:ubuntudelay}} illustrates the average write delay
for \textsl{LiS}, \textsl{Linux Fast-STREAMS} and Linux legacy pipes across a
range of write sizes.  Again, the slope of the delay curves is similar, but
\textsl{Linux Fast-STREAMS} exhibits a greatly reduced intercept indicating
superior per-message overheads.

\item[Throughput.]

\textit{Figure \ref{figure:ubuntuthrput}} illustrates the throughput
experienced by \textsl{LiS}, \textsl{Linux Fast-STREAMS} and Linux legacy
pipes across a range of write sizes.  As can be seen from \textit{Figure
\ref{figure:ubuntuthrput}}, all implementations exhibit strong power function
characteristics, indicating structure and robustness for each implementation.
Again \textsl{Linux Fast-STREAMS} appears to saturate the memory bus
approaching 10Gbps.

\item[Improvement.]

\textit{Figure \ref{figure:ubuntucomp}} illustrates the improvement over Linux
legacy pipes of \textsl{Linux Fast-STREAMS} STREAMS-based pipes.  The
improvement of \textsl{Linux Fast-STREAMS} over Linux legacy pipes is
significant: improvements range from a 75\% increase in performance at large
write sizes, to a 200\% increase in performance at lower write sizes.

\end{description}

\subsubsection[Ubuntu 7.04]{Ubuntu 7.04}

Ubuntu 7.04 is the current release of the Ubuntu distribution.  The Ubuntu
7.04 release sports a 2.6.20 kernel.  The tested distribution had current
updates applied.

\begin{figure}[p]
\includegraphics[width=\columnwidth]{perftest_u704_perf}
\caption[Ubuntu 7.04 on Porky Performance]{Ubuntu 7.04 on Porky Performance}
\label{figure:ubuntuperf}
\end{figure}

\begin{figure}[p]
\includegraphics[width=\columnwidth]{perftest_u704_delay}
\caption[Ubuntu 7.04 on Porky Delay]{Ubuntu 7.04 on Porky Delay}
\label{figure:ubuntudelay}
\end{figure}

\begin{figure}[p]
\includegraphics[width=\columnwidth]{perftest_u704_thrput}
\caption[Ubuntu 7.04 on Porky Throughput]{Ubuntu 7.04 on Porky Throughput}
\label{figure:ubuntuthrput}
\end{figure}

\begin{figure}[p]
\includegraphics[width=\columnwidth]{perftest_u704_comp}
\caption[Ubuntu 7.04 on Porky Comparison]{Ubuntu 7.04 on Porky Comparison}
\label{figure:ubuntucomp}
\end{figure}

\begin{description}

\item[Performance.]

\textit{Figure \ref{figure:ubuntuperf}} illustrates the performance of
\textsl{LiS}, \textsl{Linux Fast-STREAMS} and Linux legacy pipes across a
range of write sizes.  As can be see from \textit{Figure
\ref{figure:ubuntuperf}}, the performance of \textsl{LiS} is dismal across the
entire range of write sizes.  The performance of \textsl{Linux Fast-STREAMS}
STREAMS-based pipes, on the other hand, is superior across the entire range of
write sizes.  Performance of \textsl{Linux Fast-STREAMS} is a full order of
magnitude better than \textsl{LiS}.

\item[Delay.]

\textit{Figure \ref{figure:ubuntudelay}} illustrates the average write delay
for \textsl{LiS}, \textsl{Linux Fast-STREAMS} and Linux legacy pipes across a
range of write sizes.  Again, the slope of the delay curves is similar, but
\textsl{Linux Fast-STREAMS} exhibits a greatly reduced intercept indicating
superior per-message overheads.

\item[Throughput.]

\textit{Figure \ref{figure:ubuntuthrput}} illustrates the throughput
experienced by \textsl{LiS}, \textsl{Linux Fast-STREAMS} and Linux legacy
pipes across a range of write sizes.  As can be seen from \textit{Figure
\ref{figure:ubuntuthrput}}, all implementations exhibit strong power function
characteristics, indicating structure and robustness for each implementation.
Again \textsl{Linux Fast-STREAMS} appears to saturate the memory bus
approaching 10Gbps.

\item[Improvement.]

\textit{Figure \ref{figure:ubuntucomp}} illustrates the improvement over Linux
legacy pipes of \textsl{Linux Fast-STREAMS} STREAMS-based pipes.  The
improvement of \textsl{Linux Fast-STREAMS} over Linux legacy pipes is
significant: improvements range from a 75\% increase in performance at large
write sizes, to a 200\% increase in performance at lower write sizes.

\end{description}

\subsection[Pumbah]{Pumbah}

Pumbah is a 2.57GHz Pentium IV (i686) uniprocessor machine with 1Gb of memory.
This machine differs from Porky in memory type only (Pumbah has somewhat
faster memory than Porky.) Linux distributions tested on this machine are as
follows:

\begin{tabular}{ll}\\
Distribution & Kernel\\
\hline
RedHat 7.2 & 2.4.20-28.7\\
\end{tabular}\\[1.0ex]

Pumbah is a control machine and is used to rule out differences between recent
2.6 kernels and one of the oldest and most stable 2.4 kernels.

\subsubsection[RedHat 7.2]{RedHat 7.2}

RedHat 7.2 is one of the oldest (and arguably the most stable) glibc2 based
releases of the RedHat distribution.  This distribution sports a 2.4.20-28.7
kernel.  The distribution has all available updates applied.

\begin{figure}[p]
\includegraphics[width=\columnwidth]{perftest_rh7_perf}
\caption[RH7.2 on Pumbah Performance]{RH7.2 on Pumbah Performance}
\label{figure:rh7perf}
\end{figure}

\begin{figure}[p]
\includegraphics[width=\columnwidth]{perftest_rh7_delay}
\caption[RH7.2 on Pumbah Delay]{RH7.2 on Pumbah Delay}
\label{figure:rh7delay}
\end{figure}

\begin{figure}[p]
\includegraphics[width=\columnwidth]{perftest_rh7_thrput}
\caption[RH7.2 on Pumbah Throughput]{RH7.2 on Pumbah Throughput}
\label{figure:rh7thrput}
\end{figure}

\begin{figure}[p]
\includegraphics[width=\columnwidth]{perftest_rh7_comp}
\caption[RH7.2 on Pumbah Comparison]{RH7.2 on Pumbah Comparison}
\label{figure:rh7comp}
\end{figure}

\begin{description}

\item[Performance.]

\textit{Figure \ref{figure:rh7perf}} illustrates the performance of
\textsl{LiS}, \textsl{Linux Fast-STREAMS} and Linux legacy pipes across a
range of write sizes.  As can be see from \textit{Figure
\ref{figure:rh7perf}}, the performance of \textsl{LiS} is dismal across the
entire range of write sizes.  The performance of \textsl{Linux Fast-STREAMS}
STREAMS-based pipes, on the other hand, is superior across the entire range of
write sizes.  At a write size of one byte, the performance of \textsl{Linux
Fast-STREAMS} is an order of magnitude greater than \textsl{LiS}.

\item[Delay.]

\textit{Figure \ref{figure:rh7delay}} illustrates the average write delay for
\textsl{LiS}, \textsl{Linux Fast-STREAMS} and Linux legacy pipes across a
range of write sizes.  The slope of all three graphs is similar, indicating
that memory caching and copy to and from user performance on a byte-by-byte
basis is similar.  The intercepts, on the other hand, are drastically
different.  \textsl{LiS} per-message overheads are massive.  \textsl{Linux
Fast-STREAMS} and Linux legacy pipes are far better.  STREAMS-based pipes have
about one third of the per-message overhead of legacy pipes.

\item[Throughput.]

\textit{Figure \ref{figure:rh7thrput}} illustrates the throughput experienced
by \textsl{LiS}, \textsl{Linux Fast-STREAMS} and Linux legacy pipes across a
range of write sizes.  As can be seen from \textit{Figure
\ref{figure:rh7thrput}}, all implementations exhibit strong power function
characteristics, indicating structure and robustness for each implementation
(despite performance differences).  On Pumbah, as was experienced on Porky,
\textsl{Linux Fast-STREAMS} is beginning to saturate the memory bus at 10Gbps.

\item[Improvement.]

\textit{Figure \ref{figure:rh7comp}} illustrates the improvement over Linux
legacy pipes of \textsl{Linux Fast-STREAMS} STREAMS-based pipes.  The
improvement of \textsl{Linux Fast-STREAMS} over Linux legacy pipes is
significant: improvements range from a 75\% increase in performance at large
write sizes, to a 175\% increase in performance at lower write sizes.
\textsl{LiS} pipes waddle in at a 75\% {\em decrease} in performance.

\end{description}

\subsection[Daisy]{Daisy}
Daisy is a 3.0GHz i630 (x86\_64) hyper-threaded machine with 1Gb of memory.
Linux distributions tested on this machine are as follows:

\begin{tabular}{ll}\\
Distribution & Kernel\\
\hline
Fedora Core 6 & 2.6.20-1.2933.fc6\\
CentOS 5 & 2.6.18-8-el5\\
\end{tabular}\\[1.0ex]

This machine is used as an SMP control machine.  Most of the test were
performed on uniprocessor non-hyper-threaded machines.  This machine is
hyper-threaded and runs full SMP kernels.  This machine also supports EMT64 and
runs \texttt{x86\_64} kernels.  It is used to rule out both SMP differences as
well as 64-bit architecture differences.

\subsubsection[Fedora Core 6 (x86\_64)]{Fedora Core 6 (x86\_64)}

Fedora Core 6 is the most recent full release Fedora distribution.  This
distribution sports a 2.6.20-1.2933.fc6 kernel with the latest patches.  This
is the \texttt{x86\_64} distribution with recent updates.

\begin{figure}[p]
\includegraphics[width=\columnwidth]{perftest_smp_perf}
\caption[FC6 on Daisy Performance]{FC6 on Daisy Performance}
\label{figure:smpperf}
\end{figure}

\begin{figure}[p]
\includegraphics[width=\columnwidth]{perftest_smp_delay}
\caption[FC6 on Daisy Delay]{FC6 on Daisy Delay}
\label{figure:smpdelay}
\end{figure}

\begin{figure}[p]
\includegraphics[width=\columnwidth]{perftest_smp_thrput}
\caption[FC6 on Daisy Throughput]{FC6 on Daisy Throughput}
\label{figure:smpthrput}
\end{figure}

\begin{figure}[p]
\includegraphics[width=\columnwidth]{perftest_smp_comp}
\caption[FC6 on Daisy Comparison]{FC6 on Daisy Comparison}
\label{figure:smpcomp}
\end{figure}

\begin{description}

\item[Performance.]

\textit{Figure \ref{figure:smpperf}} illustrates the performance of
\textsl{LiS}, \textsl{Linux Fast-STREAMS} and Linux legacy pipes across a
range of write sizes.  As can be see from \textit{Figure
\ref{figure:smpperf}}, the performance of \textsl{LiS} is dismal across the
entire range of write sizes.  The performance of \textsl{Linux Fast-STREAMS}
STREAMS-based pipes, on the other hand, is superior across the entire range of
write sizes.  The performance of \textsl{Linux Fast-STREAMS} is almost an
order of magnitude greater than that of \textsl{LiS}.

\item[Delay.]

\textit{Figure \ref{figure:smpdelay}} illustrates the average write delay for
\textsl{LiS}, \textsl{Linux Fast-STREAMS} and Linux legacy pipes across a
range of write sizes.  Again the slope appears to be the same for all
implementations, except Linux legacy pipes which exhibit some anomalies
below 1024 byte write sizes.  The intercept for \textsl{Linux Fast-STREAMS} is
again much superior to the other two implementations.

\item[Throughput.]

\textit{Figure \ref{figure:smpthrput}} illustrates the throughput experienced
by \textsl{LiS}, \textsl{Linux Fast-STREAMS} and Linux legacy pipes across a
range of write sizes.  As can be seen from \textit{Figure
\ref{figure:smpthrput}}, all implementations exhibit strong power function
characteristics, indicating structure and robustness for each implementation.

\item[Improvement.]

\textit{Figure \ref{figure:smpcomp}} illustrates the improvement over Linux
legacy pipes of \textsl{Linux Fast-STREAMS} STREAMS-based pipes.  The
improvement of \textsl{Linux Fast-STREAMS} over Linux legacy pipes is
significant: improvements range from a 100\% increase in performance at large
write sizes, to a 175\% increase in performance at lower write sizes.
\textsl{LiS} again drags in at -75\%.

\end{description}

\subsubsection[CentOS 5 (x86\_64)]{CentOS 5 (x86\_64)}

CentOS 5 is the most recent full release CentOS distribution.  This
distribution sports a 2.6.18-8-el5 kernel with the latest patches.  This
is the \texttt{x86\_64} distribution with recent updates.

\begin{figure}[p]
\includegraphics[width=\columnwidth]{perftest_cos5_perf}
\caption[CentOS 5 on Daisy Performance]{CentOS 5 on Daisy Performance}
\label{figure:cos5perf}
\end{figure}

\begin{figure}[p]
\includegraphics[width=\columnwidth]{perftest_cos5_delay}
\caption[CentOS 5 on Daisy Delay]{CentOS 5 on Daisy Delay}
\label{figure:cos5delay}
\end{figure}

\begin{figure}[p]
\includegraphics[width=\columnwidth]{perftest_cos5_thrput}
\caption[CentOS 5 on Daisy Throughput]{CentOS 5 on Daisy Throughput}
\label{figure:cos5thrput}
\end{figure}

\begin{figure}[p]
\includegraphics[width=\columnwidth]{perftest_cos5_comp}
\caption[CentOS 5 on Daisy Comparison]{CentOS 5 on Daisy Comparison}
\label{figure:cos5comp}
\end{figure}

\begin{description}

\item[Performance.]

\textit{Figure \ref{figure:cos5perf}} illustrates the performance of
\textsl{LiS}, \textsl{Linux Fast-STREAMS} and Linux legacy pipes across a
range of write sizes.  As can be see from \textit{Figure
\ref{figure:cos5perf}}, the performance of \textsl{LiS} is dismal across the
entire range of write sizes.  The performance of \textsl{Linux Fast-STREAMS}
STREAMS-based pipes, on the other hand, is superior across the entire range of
write sizes.  The performance of \textsl{Linux Fast-STREAMS} is almost an
order of magnitude greater than that of \textsl{LiS}.

\item[Delay.]

\textit{Figure \ref{figure:cos5delay}} illustrates the average write delay for
\textsl{LiS}, \textsl{Linux Fast-STREAMS} and Linux legacy pipes across a
range of write sizes.  Again the slope appears to be the same for all
implementations, except Linux legacy pipes which exhibit some anomalies
below 1024 byte write sizes.  The intercept for \textsl{Linux Fast-STREAMS} is
again much superior to the other two implementations.

\item[Throughput.]

\textit{Figure \ref{figure:cos5thrput}} illustrates the throughput experienced
by \textsl{LiS}, \textsl{Linux Fast-STREAMS} and Linux legacy pipes across a
range of write sizes.  As can be seen from \textit{Figure
\ref{figure:cos5thrput}}, all implementations exhibit strong power function
characteristics, indicating structure and robustness for each implementation.

\item[Improvement.]

\textit{Figure \ref{figure:cos5comp}} illustrates the improvement over Linux
legacy pipes of \textsl{Linux Fast-STREAMS} STREAMS-based pipes.  The
improvement of \textsl{Linux Fast-STREAMS} over Linux legacy pipes is
significant: improvements range from a 100\% increase in performance at large
write sizes, to a 175\% increase in performance at lower write sizes.
\textsl{LiS} again drags in at -75\%.

\end{description}

\subsection[Mspiggy]{Mspiggy}
Mspiggy is a 1.7Ghz Pentium IV (M-processor) uniprocessor notebook (Toshiba
Satellite 5100) with 1Gb of memory.  Linux distributions tested on this
machine are as follows:

\begin{tabular}{ll}\\
Distribution & Kernel\\
\hline
SuSE 10.0 OSS & 2.6.13-15-default\\
\end{tabular}\\[1.0ex]

Note that this is the same distribution that was also tested on Porky.  The
purpose of tesing on this notebook is to rule out the differences between
machine architectures on the test results.  Tests performed on this machine
are control tests.

\subsubsection[SuSE 10.0 OSS]{SuSE 10.0 OSS}

SuSE 10.0 OSS is the public release version of the SuSE/Novell distribution.
There have been two releases subsequent to this one: the 10.1 and recent 10.2
releases.  The SuSE 10 release sports a 2.6.13 kernel and the
2.6.13-15-default kernel was the tested kernel.

\begin{figure}[p]
\includegraphics[width=\columnwidth]{perftest_nb_perf}
\caption[SuSE 10.0 OSS on Mspiggy Performance]{SuSE 10.0 OSS on Mspiggy Performance}
\label{figure:nbperf}
\end{figure}

\begin{figure}[p]
\includegraphics[width=\columnwidth]{perftest_nb_delay}
\caption[SuSE 10.0 OSS on Mspiggy Delay]{SuSE 10.0 OSS on Mspiggy Delay}
\label{figure:nbdelay}
\end{figure}

\begin{figure}[p]
\includegraphics[width=\columnwidth]{perftest_nb_thrput}
\caption[SuSE 10.0 OSS on Mspiggy Throughput]{SuSE 10.0 OSS on Mspiggy Throughput}
\label{figure:nbthrput}
\end{figure}

\begin{figure}[p]
\includegraphics[width=\columnwidth]{perftest_nb_comp}
\caption[SuSE 10.0 OSS on Mspiggy Comparison]{SuSE 10.0 OSS on Mspiggy Comparison}
\label{figure:nbcomp}
\end{figure}

\begin{description}

\item[Performance.]

\textit{Figure \ref{figure:nbperf}} illustrates the performance of
\textsl{LiS}, \textsl{Linux Fast-STREAMS} and Linux legacy pipes across a
range of write sizes.  As can be see from \textit{Figure \ref{figure:nbperf}},
the performance of \textsl{LiS} is dismal across the entire range of write
sizes.  The performance of \textsl{Linux Fast-STREAMS} STREAMS-based pipes, on
the other hand, is superior across the entire range of write sizes.
\textsl{Linux Fast-STREAMS} again performs a full order of magnitude better
than \textsl{LiS}.

\item[Delay.]

\textit{Figure \ref{figure:nbdelay}} illustrates the average write delay for
\textsl{LiS}, \textsl{Linux Fast-STREAMS} and Linux legacy pipes across a
range of write sizes.  The slope of the delay curves is, again, similar, but
the intercept for \textsl{Linux Fast-STREAMS} is far superior.

\item[Throughput.]

\textit{Figure \ref{figure:nbthrput}} illustrates the throughput experienced
by \textsl{LiS}, \textsl{Linux Fast-STREAMS} and Linux legacy pipes across a
range of write sizes.  As can be seen from \textit{Figure
\ref{figure:nbthrput}}, all implementations exhibit strong power function
characteristics, indicating structure and robustness for each implementation.
\textsl{Linux Fast-STREAMS} again begins to saturate the memory bus at 10Gbps.

\item[Improvement.]

\textit{Figure \ref{figure:nbcomp}} illustrates the improvement over Linux
legacy pipes of \textsl{Linux Fast-STREAMS} STREAMS-based pipes.  The
improvement of \textsl{Linux Fast-STREAMS} over Linux legacy pipes is
significant: improvements range from a 100\% increase in performance at large
write sizes, to a staggering 400\% increase in performance at lower write
sizes.

\end{description}

\section[Analysis]{Analysis}

The results across the various distributions and machines tested are
consistent enough to draw some conclusions from the test results.

\subsection[Discussion]{Discussion}

The test results reveal that the maximum throughput performance, as tested by
the \texttt{perftest} program, of STREAMS-based pipes (as implemented by
\textsl{Linux Fast-STREAMS}) is remarkably superior to that of legacy Linux
pipes, regardless of write or read sizes.  In fact, STREAMS-based pipe
performance at smaller write/read sizes is significantly greater (as much as
200-400\%) than that of legacy pipes.  The performance of \textsl{LiS} is
dismal (approx. 75\% {\em decrease}) compared to legacy Linux pipes.

Looking at only the legacy Linux and \textsl{Linux Fast-STREAMS}
implementations, the difference can be described by analyzing the
implementations.

\paragraph*{Write side processing.}

Linux legacy pipes use a simple method on the write side of the pipe.  The
pipe copies bytes from the user into a preallocated page, by pushing a tail
pointer.  If there is a sleeping reader process, the process is awoken.  If
there is no more room in the buffer, the write process sleeps or fails.

STREAMS, on the other hand, uses full flow control.  On the write side of the
STREAMS-based pipe, the \textit{Stream head} allocates a message block and
copies the bytes from the user to the message block and places the message
block onto the \textit{Stream}.  This results in placing the message on the
opposite \textit{Stream head}.  If a reader is sleeping on the opposite
\textit{Stream head}, the \textit{Stream head}'s read queue service procedure
is scheduled.  If the \textit{Stream} is flow controlled, the writing process
sleeps or fails.

STREAMS has the feature that when a reader finds insufficient bytes available
to satisfy the read, it issues an \texttt{M\_READ} message downstream
requesting a specific number of bytes.  When the writing \textit{Stream head}
receives this message, it attempts to satisfy the full read request before
sending data downstream.

\textsl{Linux Fast-STREAMS} also has the feature that when flow control is
exerted, it saves the message buffer and a subsequent write of the same size
is added to the same buffer.

\paragraph*{Read side processing.}

On the read side of the legacy pipe, bytes are copied from the preallocated
page buffer to the user, pulling a head pointer.  If there are no bytes
available to be read in the buffer, the reading process sleeps or fails.  When
bytes have been read from the buffer and a process is sleeping waiting to
write, the sleeping process is awoken.

STREAMS again uses full flow control.  On the read side of the STREAMS-based
pipe, messages are removed from the \textit{Stream head} read queue, copied to
the user, and then the message is either freed (when all the bytes contained
are consumed) or placed back on the \textit{Stream head} read queue.  If the
read queue was previously full and falls beneath the low water mark for the
read queue, the \textit{Stream} is back-enabled.  Back-enabling results in the
service procedure of the write side queue of the other \textit{Stream head} to
be scheduled for service.  If there are no bytes available to be read, the
reading process sleeps or fails.

STREAMS has the additional feature that if there are no bytes to be read, it
can issue an \texttt{M\_READ} message downstream requesting the number of
bytes that were issued to the read(2) system call.

\paragraph*{Buffering.}

There are two primary differences in the buffering approaches used by legacy
and STREAMS-based pipes:

\begin{enumerate}

\item Legacy pipes use preallocated pinned kernel pages to store data using a
simply head and tail pointer approach.

\item STREAMS-based pipes use full flow control with STREAMS message blocks
and message queues.

\end{enumerate}

One would expect that the STREAMS-based approach would present significant
overheads in comparison to the legacy approach; however, the lack of flow
control in the Linux approach is problematic.

\paragraph*{Scheduling.}

Legacy pipes schedule by waking a reading process whenever data is available
in the buffer to be read, and waking a writing process whenever there is room
available in the buffer to write.  While accomplishing buffering, this does
not provide flow control or scheduling.  By not providing even the hysteresis
afforded by Sockets, the write and read side thrash the scheduler as bytes are
written to and removed from the pipe.

STREAMS-based pipes, on the other hand, use the scheduling mechanisms of
STREAMS.  When messages are written to the reading \textit{Stream head} and a
reader is sleeping, the service procedure for the reading \textit{Stream
head}'s read queue is scheduled for later execution.  When the STREAMS
scheduler later runs, the reading process is awoken.  When message are read
from the reading \textit{Stream head} read queue and the queue was previously
flow controlled, and the byte count falls below the low water mark defined for
the queue, the writing \textit{Stream head} write queue service procedure is
scheduled.  Once the STREAMS scheduler later runs, the writing process is
awoken.

\textsl{Linux Fast-STREAMS} is designed to run tasks queued to the STREAMS
scheduler on the same processor as the queueing process or task.  This avoids
unnecessary context switches.

The STREAMS-based pipe approach results in fewer wakeup events being
generated.  Because there are fewer wakeup events, there are fewer context
switches.  The reading process is permitted to consume more messages before
the writing process is awoken; and the writing process is permitted to write
more messages before the reading process is awoken.

\paragraph*{Result.}

The result of the differences between the legacy and the STREAMS based
approach is that fewer context switches result: writing processes are allowed
to write more messages before a blocked reader is awoken and the reading
process is allowed to read more messages before a blocked writer is awoken.
This results in greater code path and data cache efficiency and significantly
less scheduler thrashing between the reading and writing process.

The increased performance of the STREAMS-based pipes can be explained as
follows:

\begin{itemize}

\item

The STREAMS message coalescing features allows the complexity of the write
side process to approach that of the legacy approach.  This feature provides a
boost to performance at message sizes smaller than a \texttt{FASTBUF}.  The
size of a \texttt{FASTBUF} on 32-bit systems is 64 bytes; on 64-bit systems,
128 bytes.  (However, this STREAMS feature is not sufficient to explain the
dramatic performance gains, as close to the same performance is exhibited with
the feature disabled.)

\item

The STREAMS read notification feature allows the write side to exploit
efficiencies from the knowledge of the amount of data that was requested by
the read side.  (However, this STREAMS feature is also not sufficient to
explain the performance gains, as close to the same performance is exhibited
with the feature disabled.)

\item

The STREAMS read fill mode feature permits the read side to block until the
full read request is satisfied, regardless of the \texttt{O\_NONBLOCK} flags
setting associated with the read side of the pipe.  (Again, this STREAMS
feature is not sufficient to explain the performance gains, as close to the
same performance is exhibited with the feature disabled.)

\item

The STREAMS flow control and scheduling mechanisms permits the read side to
read more messages between wakeup events; and also permits the write side to
write more messages between wakeup events.  This results in superior code and
data caching efficiencies and a greatly reduced number of context switches.
This is the only difference that explains the full performance increase in
STREAMS-based pipes over legacy pipes.

\end{itemize}

\section[Conclusions]{Conclusions}

These experiments have shown that the \textsl{Linux Fast-STREAMS}
implementation of STREAMS-based pipes outperforms the legacy Linux pipe
implementation by a significant amount (up to a factor of 5) and outperform
the \textsl{LiS} implementation by a staggering amount (up to a factor of 25).

\begin{quote}
\textit{The \textsl{Linux Fast-STREAMS} implementation of STREAMS-based pipes
is superior by a significant factor across all systems and kernels tested.}
\end{quote}

While it can be said that all of the preconceptions regarding STREAMS and
STREAMS-based pipes are applicable to the under-performing \textsl{LiS}, and
may very well be applicable to historical implementations of STREAMS, these
preconceptions with regard to STREAMS and STREAMS-based pipes are dispelled
for the high-performance \textsl{Linux Fast-STREAMS} by these test results.

\begin{itemize}

\item \textit{STREAMS is fast.}

Contrary to the preconception that STREAMS must be slower because it is more
complex, in fact the reverse has been shown to be true for \textsl{Linux
Fast-STREAMS} in these experiments.  The STREAMS flow control and scheduling
mechanisms serve to adapt well and increase both code and data cache as well
as scheduler efficiency.

\item \textit{STREAMS is more flexible {\em and} more efficient.}

Contrary to the preconception that STREAMS trades flexibility for efficiency
(that is, that STREAMS is somehow less efficient because it is more flexible),
in fact has shown to be untrue for \textsl{Linux Fast-STREAMS}, which is {\em
both} more flexible {\em and} more efficient.  Indeed, the performance gains
achieved by STREAMS appear to derive from its more sophisticated queueing,
scheduling and flow control model. (Note that this is in fitting with the
statements made about 4.2BSD pipes being implemented with UNIX domain sockets
for \textit{"performance reasons"} \cite[]{bsd}.)

\item \textit{Linux Fast-STREAMS adequately exploits parallelisms on SMP.}

Contrary to the preconception that STREAMS must be slower due to complex
locking and synchronization mechanisms, \textsl{Linux Fast-STREAMS} performed
as well on SMP (hyperthreaded) machines as on UP machines and still
outperformed legacy Linux pipes with over 100\% improvements at all write
sizes.

\item \textit{STREAMS-based pipes are fast.}

Contrary to the preconception that STREAMS-based pipes must be slower because
STREAMS-based pipes provide such a rich set of features as well as providing
full duplex operation where legacy pipes only unidirectional operation, the
reverse has been shown in these experiments for \textsl{Linux Fast-STREAMS}.
By utilizing STREAMS flow control and scheduling, STREAMS-based pipes indeed
perform better than legacy pipes.

\item \textit{STREAMS-based pipes are neither unnecessarily complex nor cumbersome.}

Contrary to the preconception that STREAMS-based pipes must be poorer due to
their increased implementation complexity, the reverse has shown to be true in
these experiments for \textsl{Linux Fast-STREAMS}.  Also, the fact that
legacy, STREAMS and 4.2BSD pipes conform to the same standard (POSIX), means
that they are no more cumbersome from a programming perspective.  Indeed a
POSIX conforming application will not know the difference between the
implementation (with the exception that superior performance will be
experienced on STREAMS-based pipes).

\item \textit{\textsl{LiS} performs poorly.}

Despite claiming to be an adequate implementation of SVR4 STREAMS,
\textsl{LiS} performance is dismal enough to make it unusable.  Due to
conformance and implementation errors, \textsl{LiS} was already deprecated by
\textsl{Linux Fast-STREAMS}, and these tests exemplify why a replacement for
\textsl{LiS} was necessary and why support for \textsl{LiS} was abandoned by
the OpenSS7 Project \cite[]{openss7}.  \textsl{LiS} pipe performance tested
about half that of legacy Linux pipes and a full order of magnitude slower
than \textsl{Linux Fast-STREAMS}.

\end{itemize}

\section[Future Work]{Future Work}

There are two future work items that immediately come to mind:

\begin{enumerate}

\item

It is fairly straightforward to replace the pipe implementation of an
application that uses shared libraries from underneath it using preloaded
libraries.  The \textsl{Linux Fast-STREAMS} \texttt{libstreams.so} library can
be preloaded, replacing the pipe(2) library call with the STREAMS-based pipe
equivalent.  A suitable application that uses pipes extensively could be
benchmarked both on legacy Linux pipes and STREAMS-based pipes to determine
the efficiencies achieved over a less narrowly defined workload.

\item

Because STREAMS-based pipes exhibit superior performance in these respects, it
can be expected that STREAMS pseudo-terminals will also exhibit superior
performance over the legacy Linux pseudo-terminal implementation.  STREAMS
pseudo-terminals utilize the STREAMS mechanisms for flow control and
scheduling, whereas the Linux pseudo-terminal implementation uses the
over-simplified approach taken by legacy pipes.

\end{enumerate}

\section[Related Work]{Related Work}

A separate paper comparing a TPI STREAMS implementation of \textsl{UDP} with
the Linux BSD Sockets implementation has also been prepared.  That paper also
shows significant performance improvements for STREAMS attributable to the
similar causes.

\FloatBarrier
\addcontentsline{toc}{section}{References}
\bibliography{piperesults}

\clearpage
\begin{appendix}

\section{Performance Testing Script}
\label{section:script}

A performance testing script (\texttt{perftest\_sctipt}) was used to obtain
repeatable results.  The script was executed as:

\begin{quote}
\scriptsize
\begin{verbatim}
$#> ./perftest_script -a -S10 --hiwat=$((1<<16)) --lowat=$((1<<13))
\end{verbatim}
\normalsize
\end{quote}

The script is as follows:

\scriptsize
\begin{verbatim}
#!/bin/bash
set -x
interval=5
testtime=2
command=`echo $0 | sed -e 's,.*/,,'`
perftestn=
perftest=
if [ -x `pwd`/perftest ] ; then
  perftest=`pwd`/perftest
elif [ -x /usr/lib/streams/perftest ] ; then
  perftest=/usr/lib/streams/perftest
elif [ -x /usr/libexec/streams/perftest ] ; then
  perftest=/usr/libexec/streams/perftest
elif [ -x /usr/lib/LiS/perftest ] ; then
  perftest=/usr/lib/LiS/perftest
elif [ -x /usr/libexec/LiS/perftest ] ; then
  perftest=/usr/libexec/LiS/perftest
fi
if [ -x `pwd`/perftestn ] ; then
  perftestn=`pwd`/perftestn
elif [ -x /usr/lib/streams/perftestn ] ; then
  perftestn=/usr/lib/streams/perftestn
elif [ -x /usr/libexec/streams/perftestn ] ; then
  perftestn=/usr/libexec/streams/perftestn
elif [ -x /usr/lib/LiS/perftestn ] ; then
  perftestn=/usr/lib/LiS/perftestn
elif [ -x /usr/libexec/LiS/perftestn ] ; then
  perftestn=/usr/libexec/LiS/perftestn
fi
[ -n "$perftestn" ] || [ -n "$perftest" ] || exit 1
scls=
if [ -x `pwd`/scls ] ; then
        scls=`pwd`/scls
elif [ -x /usr/sbin/scls ] ; then
        scls=/usr/sbin/scls
fi
(
  set -x
  [ -n "$scls" ] && $scls -a -c -r pipe pipemod
  for size in 4096 2048 1024 512 256 128 64 32 16 8 4 2 1
  do
    [ -n "$perftest"  ] && $perftest  -q \
      -r -t $testtime -i $interval -m nullmod -p 0 -s $size ${1+$@}
    [ -n "$perftestn" ] && $perftestn -q \
      -r -t $testtime -i $interval -m nullmod -p 0 -s $size ${1+$@}
    [ -n "$scls"      ] && $scls -a -c -r pipe pipemod bufmod nullmod
  done
) 2>&1 | tee `hostname`.$command.`date -uIseconds`.log
\end{verbatim}
\normalsize


\section{Raw Data}
\label{section:rawdata}

Following are the raw data points captured using the \textsl{perftest\_script}
benchmarking script:

\textit{Table \ref{table:fc6data}} lists the raw data from the
\texttt{perftest} program that was used in preparing graphs for Fedora Core 6
(i386) on Porky.

\textit{Table \ref{table:centosdata}} lists the raw data from the
\texttt{perftest} program that was used in preparing graphs for CentOS 4 on
Porky.

\textit{Table \ref{table:susedata}} lists the raw data from the
\texttt{perftest} program that was used in preparing graphs for SuSE OSS 10 on
Porky.

\textit{Table \ref{table:ubuntudata}} lists the raw data from the
\texttt{perftest} program that was used in preparing graphs for Ubuntu 6.10 on
Porky.

\textit{Table \ref{table:rh7data}} lists the raw data from the
\texttt{perftest} program that was used in preparing graphs for RedHat 7.2 on
Pumbah.

\textit{Table \ref{table:smpdata}} lists the raw data from the
\texttt{perftest} program that was used in preparing graphs for Fedora Core 6
(x86\_64) HT on Daisy.

\textit{Table \ref{table:nbdata}} lists the raw data from the
\texttt{perftest} program that was used in preparing graphs for SuSE 10.0 OSS
on Mspiggy.

\begin{table}[hp]
\footnotesize
\center\begin{tabular}{rrrr}\\
\hline
Size & LiS & STREAMS & Linux\\
\hline
\hline
1 & 37188 & 344307 & 116966\\
2 & 37284 & 351804 & 117820\\
4 & 37179 & 347164 & 116381\\
8 & 37030 & 338055 & 117887\\
16 & 37225 & 329919 & 117822\\
32 & 36999 & 317133 & 116595\\
64 & 36809 & 302554 & 116686\\
128 & 35127 & 283041 & 117284\\
256 & 34828 & 271630 & 114657\\
512 & 34807 & 263021 & 114821\\
1024 & 34607 & 247080 & 111825\\
2048 & 34204 & 214279 & 106369\\
4096 & 33139 & 176842 & 100510\\
\hline
\end{tabular}
\caption{Raw data for Fedora Core 6 on Porky}
\label{table:fc6data}
\normalsize
\end{table}

\begin{table}[hp]
\footnotesize
\center\begin{tabular}{rrrr}\\
\hline
Size & LiS & STREAMS & Linux\\
\hline
\hline
1 & 53119 & 479434 & 132195\\
2 & 53066 & 505597 & 132293\\
4 & 53289 & 501230 & 131201\\
8 & 53216 & 475951 & 132182\\
16 & 53254 & 464013 & 131688\\
32 & 52952 & 438519 & 131697\\
64 & 52499 & 407751 & 129409\\
128 & 50065 & 379356 & 130188\\
256 & 49348 & 372393 & 126861\\
512 & 49297 & 360773 & 125318\\
1024 & 48598 & 336727 & 123318\\
2048 & 48274 & 290614 & 117809\\
4096 & 47004 & 227778 & 110875\\
\hline
\end{tabular}
\caption{Raw data for CentOS 4.0 on Porky}
\label{table:centosdata}
\normalsize
\end{table}

\begin{table}[hp]
\footnotesize
\center\begin{tabular}{rrrr}\\
\hline
Size & LiS & STREAMS & Linux\\
\hline
\hline
1 & 53119 & 961820 & 168049\\
2 & 53066 & 933673 & 176267\\
4 & 53289 & 942865 & 172912\\
8 & 53216 & 837034 & 168898\\
16 & 53254 & 827399 & 166427\\
32 & 52952 & 740263 & 172185\\
64 & 52499 & 659878 & 169231\\
128 & 50065 & 582512 & 174005\\
256 & 49348 & 580011 & 166646\\
512 & 49297 & 547149 & 167829\\
1024 & 48598 & 512452 & 152447\\
2048 & 48274 & 413858 & 154813\\
4096 & 47004 & 307174 & 138756\\
\hline
\end{tabular}
\caption{Raw data for SuSE 10.0 OSS on Porky}
\label{table:susedata}
\normalsize
\end{table}

\begin{table}[hp]
\footnotesize
\center\begin{tabular}{rrrr}\\
\hline
Size & LiS & STREAMS & Linux\\
\hline
\hline
1 & 53119 & 430184 & 144855\\
2 & 53066 & 433274 & 143835\\
4 & 53289 & 425094 & 145879\\
8 & 53216 & 407647 & 143399\\
16 & 53254 & 394244 & 141268\\
32 & 52952 & 372063 & 144056\\
64 & 52499 & 354598 & 139854\\
128 & 50065 & 339602 & 141793\\
256 & 49348 & 324405 & 140269\\
512 & 49297 & 311610 & 134445\\
1024 & 48598 & 292892 & 136385\\
2048 & 48274 & 255374 & 127651\\
4096 & 47004 & 202755 & 116218\\
\hline
\end{tabular}
\caption{Raw data for Ubuntu 6.10 on Porky}
\label{table:ubuntudata}
\normalsize
\end{table}

\begin{table}[hp]
\footnotesize
\center\begin{tabular}{rrrr}\\
\hline
Size & LiS & STREAMS & Linux\\
\hline
\hline
1 & 53160 & 497439 & 209223\\
2 & 53440 & 499519 & 199566\\
4 & 53252 & 496272 & 187440\\
8 & 53097 & 489615 & 188829\\
16 & 53179 & 485036 & 182148\\
32 & 52926 & 469102 & 185174\\
64 & 53535 & 457550 & 182383\\
128 & 49452 & 416632 & 178087\\
256 & 49584 & 396356 & 177204\\
512 & 49169 & 381209 & 165517\\
1024 & 48992 & 355111 & 173222\\
2048 & 47970 & 303334 & 163572\\
4096 & 46598 & 240386 & 136522\\
\hline
\end{tabular}
\caption{Raw data for RedHat 7.2 on Pumbah}
\label{table:rh7data}
\normalsize
\end{table}

\begin{table}[hp]
\footnotesize
\center\begin{tabular}{rrrr}\\
\hline
Size & LiS & STREAMS & Linux\\
\hline
\hline
1 & 37188 & 334896 & 146553\\
2 & 37284 & 334796 & 122048\\
4 & 37179 & 329476 & 140025\\
8 & 37030 & 341612 & 160396\\
16 & 37225 & 333520 & 125678\\
32 & 36999 & 325169 & 125124\\
64 & 36809 & 302603 & 109340\\
128 & 35127 & 278490 & 128133\\
256 & 34828 & 247379 & 122689\\
512 & 34807 & 235190 & 104739\\
1024 & 34607 & 215718 & 83447\\
2048 & 34204 & 187982 & 81301\\
4096 & 33139 & 150392 & 77118\\
\hline
\end{tabular}
\caption{Raw data for Fedora Core 6 on Daisy}
\label{table:smpdata}
\normalsize
\end{table}

\begin{table}[hp]
\footnotesize
\center\begin{tabular}{rrrr}\\
\hline
Size & LiS & STREAMS & Linux\\
\hline
\hline
1 & 53119 & 614186 & 120757\\
2 & 53066 & 611735 & 119166\\
4 & 53289 & 603726 & 121366\\
8 & 53216 & 547847 & 120595\\
16 & 53254 & 528422 & 118542\\
32 & 52952 & 468888 & 117535\\
64 & 52499 & 441368 & 116674\\
128 & 50065 & 400486 & 115725\\
256 & 49348 & 392344 & 114541\\
512 & 49297 & 380014 & 113128\\
1024 & 48598 & 345346 & 104937\\
2048 & 48274 & 275962 & 104450\\
4096 & 47004 & 203886 & 94279\\
\hline
\end{tabular}
\caption{Raw data for SuSE 10.0 OSS on Mspiggy}
\label{table:nbdata}
\normalsize
\end{table}

\end{appendix}

\end{document}
